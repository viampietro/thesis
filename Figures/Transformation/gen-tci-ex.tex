\documentclass{standalone}

\usepackage[english]{babel}

% to define font size

\usepackage{ulem}
\usepackage{moresize}
\usepackage{anyfontsize}

% to use colors

\usepackage[dvipsnames]{xcolor}
\usepackage{MnSymbol}

% to use tikz and its libraries

\usepackage{tikz-timing}
\usepackage{tikz}

\usetikzlibrary{backgrounds}
\usetikzlibrary{positioning, calc, arrows, shapes, automata, petri, patterns,decorations.markings}

% to use tikzmark, to place and refer to marks outside the current figure

\tikzset{every picture/.style={remember picture}}

% styles for transitions

\tikzset{transition/.append style={fill=black!20, thick}}
\tikzset{transition/.append style={fill=black!20, thick}}

% styles for test and inhib arcs.

\tikzstyle{test}=[pre, *-]
\tikzstyle{inhib}=[pre, o-]

\usepackage{circuitikzgit}
\ctikzset{
  logic ports=ieee,
}

% Arrow positioning in a path

\tikzset{->-/.style={decoration={
  markings,
  mark=at position #1 with {\arrow{>}}},postaction={decorate}}}

\tikzset{-<-/.style={decoration={
  markings,
  mark=at position #1 with {\arrow{<}}},postaction={decorate}}}

% shift values

\newcommand{\outportshift}{0mm}
\newcommand{\outportidpshift}{0mm}

%%%%%%%%%%%%%%%%%%%%%%%%%%%%%%%%%%%%%%%%%%%%%%%%%%
%                  BEGIN DOCUMENT                %
%%%%%%%%%%%%%%%%%%%%%%%%%%%%%%%%%%%%%%%%%%%%%%%%%%

\begin{document}

\begin{circuitikz}

  \ctikzset{multipoles/dipchip/width=2}
  \ctikzset{multipoles/dipchip/pin spacing=.18}

  % \node[draw,rectangle,inner sep=7mm] (tl) {
  %   \begin{circuitikz}

  %     % PCI idp

      % interface
      % \draw       
      % node [dipchip, num pins=16, hide numbers,
      % no topmark, external pins width=0]
      % (idp) {$\mathtt{id}_{p_0}$};

      % \draw ($(idp.bpin 1)$) ++(0,0.1) -- ++(0.1,-0.1) node[right, font=\ssmall] {\tt clk} -- ++(-0.1,-0.1) ;
      
      % \draw ($(idp.bpin 2)$) --++ (-.3,0) coordinate (idprst);
      % \node at (idp.bpin 2) [anchor=west, font=\ssmall]  {\tt rst};

      % \draw ($(idp.bpin 3)$) -- ++(-.3,0) coordinate (idpim);
      % \node at (idp.bpin 3) [anchor=west, font=\ssmall]  {\tt im};
      % \node at (idpim) [anchor=east, font=\ssmall,xshift=\outportshift] {\tt \textcolor{blue}{1}};
      
      % \draw ($(idp.bpin 4)$) -- ++(-.3,0) coordinate (idpiaw0);
      % \node at (idp.bpin 4) [anchor=west, font=\ssmall]  {\tt iaw(\textcolor{blue}{0})};
      % \node at (idpiaw0) [anchor=east, font=\ssmall,xshift=\outportshift] {\tt \textcolor{blue}{1}};

      % \draw ($(idp.bpin 5)$) -- ++(-.3,0) coordinate (idpoat0);
      % \node at (idp.bpin 5) [anchor=west, font=\ssmall]  {\tt oat(\textcolor{blue}{0})};
      % \node at (idpoat0) [anchor=east, font=\ssmall,xshift=\outportshift] {\tt \textcolor{blue}{0}};

      % \draw ($(idp.bpin 6)$) -- ++(-.3,0) coordinate (idpoaw0);
      % \node at (idp.bpin 6) [anchor=west, font=\ssmall]  {\tt oaw(\textcolor{blue}{0})};
      % \node at (idpoaw0) [anchor=east, font=\ssmall,xshift=\outportshift] {\tt \textcolor{blue}{1}};
      
      % \draw ($(idp.bpin 7)$) --++ (-.3,0) coordinate (idpitf0);
      % \node at (idp.bpin 7) [anchor=west, font=\ssmall]  {\tt itf(\textcolor{blue}{0})};

      % \draw ($(idp.bpin 8)$) --++ (-.3,0) coordinate (idpotf0);
      % \node at (idp.bpin 8) [anchor=west, font=\ssmall]  {\tt otf(\textcolor{blue}{0})};

      % \draw ($(idp.bpin 15)$) --++(.3,0) coordinate (idpoav0);
      % \node at (idp.bpin 15) [anchor=east, font=\ssmall, xshift=\outportidpshift]  {\tt oav(\textcolor{blue}{0})};
      
      % \draw ($(idp.bpin 13)$) --++(.3,0) coordinate (idprtt0) ;
      % \node at (idp.bpin 13) [anchor=east, font=\ssmall, xshift=\outportidpshift]  {\tt rtt(\textcolor{blue}{0})};

      % \draw ($(idp.bpin 11)$) -- ++(.3,0) coordinate (idppauths0);
      % \node at (idp.bpin 11) [anchor=east, font=\ssmall, xshift=\outportidpshift]  {\tt pauths(\textcolor{blue}{0})};

      % \draw[red,->-=.4] ($(idp.bpin 9)$) --++(.6,0) coordinate (idpmarked);
      % \node at (idp.bpin 9) [anchor=east, font=\ssmall, xshift=\outportidpshift]  {\tt m};

      % % generic map
      % \node[anchor=north] at (idp.south) {\ssmall\tt
      %   \begin{tabular}{@{}c@{}}
      %     \textcolor{blue}{(ian, 1), (oan, 1)} \\
      %   \end{tabular}
      % };
      
      % TCI idt
      \ctikzset{multipoles/dipchip/width=2.2}
      \draw       
      node [dipchip, anchor=east, num pins=16, hide numbers,
      no topmark, external pins width=0]
      (idt) {$\mathtt{id}_{t_0}$};

      \draw ($(idt.bpin 1)$) ++(0,0.1) -- ++(0.1,-0.1) node[right, font=\ssmall] {\tt clk} -- ++(-0.1,-0.1) ;
      
      \draw ($(idt.bpin 2)$) --++ (-.3,0) coordinate (idtrst);
      \node at (idt.bpin 2) [anchor=west, font=\ssmall]  {\tt rst};

      \draw ($(idt.bpin 4)$) -- ($(idt.bpin 4)+(-.3,0)$) coordinate (idtA);
      \node at (idt.bpin 4) [anchor=west, font=\ssmall]  {\tt A};
      \node at (idtA) [anchor=east, font=\ssmall,xshift=\outportshift] {\tt \textcolor{blue}{1}};
      
      \draw ($(idt.bpin 5)$) -- ($(idt.bpin 5)+(-.3,0)$) coordinate (idtB);
      \node at (idt.bpin 5) [anchor=west, font=\ssmall]  {\tt B};
      \node at (idtB) [anchor=east, font=\ssmall,xshift=\outportshift] {\tt \textcolor{blue}{3}};

      \draw ($(idt.bpin 6)$) -- ++(-.3,0) coordinate (idtpauths0);
      \node at (idt.bpin 6) [anchor=west, font=\ssmall]  {\tt pauths(\textcolor{blue}{0})};
      
      \draw ($(idt.bpin 7)$) --++ (-.3,0) coordinate (idtiav0);
      \node at (idt.bpin 7) [anchor=west, font=\ssmall]  {\tt iav(\textcolor{blue}{0})};

      \draw ($(idt.bpin 8)$) --++ (-.3,0) coordinate (idtrt0) ;
      \node at (idt.bpin 8) [anchor=west, font=\ssmall]  {\tt rt(\textcolor{blue}{0})};

      \draw ($(idt.bpin 3)$) --++ (-.3,0) coordinate (idtic0);
      \node at (idt.bpin 3) [anchor=west, font=\ssmall]  {\tt ic(\textcolor{blue}{0})};

      \draw[red,->-=.4] ($(idt.east)$) --++ (.6,0) coordinate (idtfired);
      \node at (idt.east) [anchor=east, font=\ssmall, xshift=\outportshift]  {\tt f};

      \node[anchor=north] at (idt.south) {\ssmall\tt
        \begin{tabular}{@{}c@{}}
          \textcolor{blue}{(tt, temp\_a\_b), (ian, 1),} \\
          \textcolor{blue}{(cn, 1), (mtc, 3)} \\
        \end{tabular}
      };
      % Process functions
  %     \ctikzset{multipoles/dipchip/width=2.2}
  %     \draw       
  %     node [dipchip, anchor=east, num pins=6, hide numbers,
  %     no topmark, external pins width=0]
  %     (funps) at ($(idt.east)+(5,0)$) {$\mathtt{functions}$};

  %     \draw ($(funps.bpin 1)$) ++(0,0.1) -- ++(0.1,-0.1) node[right, font=\ssmall,xshift=-6mm] {\tt clk} -- ++(-0.1,-0.1) ;
      
  %     \coordinate (funpsrst) at ($(funps.bpin 2)+(-.3,0)$);
  %     \node at (funps.bpin 2) [anchor=west, font=\ssmall, xshift=-6mm]  {\tt rst};
      
  %     \coordinate (funpsf) at ($(funps.bpin 3)+(-.3,0)$);
  %     \node at (funps.bpin 3) [anchor=west, font=\ssmall, xshift=-6mm]  {};
      
  %     % Process actions
  %     \draw       
  %     node [dipchip, anchor=north, num pins=6, hide numbers,
  %     no topmark, external pins width=0]
  %     (actps) at ($(funps.south)-(0,2)$) {$\mathtt{actions}$};

  %     \draw ($(actps.bpin 1)$) ++(0,0.1) -- ++(0.1,-0.1) node[right, font=\ssmall,xshift=-6mm] {\tt clk} -- ++(-0.1,-0.1) ;
      
  %     \coordinate (actpsrst) at ($(actps.bpin 2)+(-.3,0)$);
  %     \node at (actps.bpin 2) [anchor=west, font=\ssmall, xshift=-6mm]  {\tt rst};
      
  %     \coordinate (actpsm) at ($(actps.bpin 3)+(-.3,0)$);
  %     \node at (actps.bpin 3) [anchor=west, font=\ssmall, xshift=-6mm] {};
  %   \end{circuitikz}
  % };

  % % TOP-LEVEL PORTS
  
  % \draw ($(tl.west)+(0,2)$) ++(0,0.1) -- ++(0.1,-0.1) coordinate (tlclk) node[left, xshift=-4mm]{\tt clk} -- ++(-0.1,-0.1);
  % \draw ($(tl.west)+(0,1)$) -- ++(-.3,0) coordinate (tlrst) node [left] {\tt rst};
  % \draw ($(tl.west)$) -- ++(-.3,0) coordinate (idc0) node [left] {\tt id$_{c_0}$};

  % \draw ($(tl.east)+(0,1)$) -- ++(.3,0) coordinate (idf0) node [right] {\tt id$_{f_0}$};

  % \draw ($(tl.east)+(0,-1)$) -- ++(.3,0) coordinate (ida0) node [right] {\tt id$_{a_0}$};

  % % INTERCONNECTIONS between TL and INTERNAL BEHAVIOR

  % % clk to clks
  % \draw [red,->-=.4] (tlclk) --++(1,0) |- (idt.bpin 1);
  % \draw [red,->-=.4] (tlclk) --++(1,0) -|++ (0,-2.6) -- (idp.bpin 1);
  % \draw [red,->-=.4] (tlclk) --++(1,0) -|++ (0,1.5) -| ($(funps.bpin 1)-(.3,0)$) -- ($(funps.bpin 1)$);
  % \draw [red](tlclk) --++(1,0) -|++ (0,1.5) -| ($(actps.bpin 1)-(.8,0)$) -- ($(actps.bpin 1)$);

  % % rst to rsts
  % \draw [blue,->-=.4] (tlrst) --++(3,0) |- (idt.bpin 2);
  % \draw [blue,->-=.4] (tlrst) --++(.7,0) |- (idp.bpin 2);
  % \draw [blue,->-=.4] (tlrst) --++(.7,0) --++ (0,2.7) -| ($(funps.bpin 2)-(.5,0)$) -- ($(funps.bpin 2)$);
  % \draw [blue] (tlrst) --++(.7,0) --++ (0,2.7) -| ($(funps.bpin 2)-(.5,0)$) |- ($(actps.bpin 2)$);

  % % idc0 to ic(0)
  % \draw [Green,->-=.4] (idc0) -|  (idtic0) -- ($(idtic0)+(.3,0)$);

  % % functions to idf0
  % \draw [Purple,->-=.2] (funps.east) -| ($(idf0)-(.6,0)$) -- (idf0);

  % % actions to ida0
  % \draw [Purple,->-=.2] (actps.east) -| ($(ida0)-(.6,0)$) -- (ida0);

  % % INTERCONNECTIONS in INTERNAL BEHAVIOR
  
  % % idt.fired to idp.itf/idp.otf/functions
  % \draw [Orange,->-=.4] (idt.east) -- (idtfired) |- ($(idpitf0)-(0,.7)$) |- (idp.bpin 8);
  % \draw [Orange] (idt.east) -- (idtfired) |- ($(idpitf0)-(0,.7)$) |- (idp.bpin 7);
  % \draw [Orange,->-=.4] (idt.east) -- (idtfired) |- (funpsf) --++(.3,0);
  
  % % idp.oav(0) to idt.iav(0)
  % \draw [Orange,->-=.4] (idp.bpin 9) -- (idpoav0) -| ($(idtiav0)-(.2,0)$) -- (idt.bpin 6);

  % % idp.rtt(0) to idt.rt(0)
  % \draw [Orange,->-=.4] (idp.bpin 13) -- (idprtt0) -| ($(idtrt0)-(.4,0)$) -- (idt.bpin 7);

  % % idp.marked to actions
  % \draw [Orange,->-=.4] (idp.bpin 15) -- (idpmarked) -| ($(actpsm)-(.6,0)$) -- (actps.bpin 3);
  
  % \draw [->-=.4] (tlclk) --++(1,0) -|++ (0,-2.6) -- (idp.bpin 1);
  % \draw [->-=.4] (tlclk) --++(1,0) -|++ (0,2) -| ($(funps.bpin 1)-(.3,0)$) -- ($(funps.bpin 1)$);
  % \draw (tlclk) --++(1,0) -|++ (0,2) -| ($(actps.bpin 1)-(.8,0)$) -- ($(actps.bpin 1)$);
  
\end{circuitikz}


\end{document}
%%% Local Variables:
%%% mode: latex
%%% TeX-master: t
%%% End:
