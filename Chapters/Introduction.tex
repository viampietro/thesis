% Chapter Template

\chapter{Introduction}

\label{chap:intro}

%----------------------------------------------------------------------------------------
%	MODEL-BASED SYSTEM ENGINEERING AND THE HILECOP METHODOLOGY
%----------------------------------------------------------------------------------------

\section{Model-based System Engineering and the \hilecop{} methodology}
\label{sec:mse}

\begin{itemize}
\item present the principles of Model-based System Engineering
\item present the context of neuroprotheses and the need for a safe
  design of digital circuits
\item present the \hilecop{} methodology
\end{itemize}

\section{Verifying the \hilecop{} methodology}
\label{sec:hilecop-verif}

\begin{itemize}

\item \textsc{Research question (a.k.a the problem):}\\
  \tikz[text width=.8\textwidth]{ \node[draw, rectangle]{\small \sc
      Can we prove the \hilecop{} model-to-text transformation is
      semantic-preserving?};}
\item \textsc{Additional questions:}
  \begin{itemize}[label=+]
  \item How do we prove that?
  \item What are the similarities with compiler verification?
  \item What are the specificities?
  \end{itemize}
\item present the task motivation: the creation of safe digital
  circuits; first step, the behavior of the input model must be
  preserved in the output VHDL program
\item present the model-to-text transformation with more details (but
  not too much since a whole chapter will be dedicated to it)
\end{itemize}

\begin{itemize}[label=\ding{212}]
\item ``inspired by the work on compiler verification, we had an idea
  about how to prove that the transformation is semantic-preserving,
  and which steps to follow''.
\end{itemize}

Then, present the structure of the memoir and the content of the
different chapters. The structure of the memoir follows the different
steps necessary to establish the proof of semantic preservation. Each
chapter corresponds to a step.

%%% Local Variables:
%%% mode: latex
%%% TeX-master: "../main"
%%% End:
