% Chapter Template

\chapter{Introduction}
\label{chap:intro}

With the use of every human-bred technology is associated a
risk. Regarding the nature of the technology, and the broader system
in which it is involved, the consequences of a failure can be
dramatic. Thus arises the notion of the safety of systems;
\cite{Bowen1993} gives the following definition of safety:

\begin{center}
  ``Safety can [\dots] be defined as the freedom from exposure to
  danger, or the exemption from hurt, injury or loss.''
\end{center}

A \textit{safety-critical} system can be understood as a system for
which the safety aspect is the main concern, being that important
consequences, such as direct human losses, natural catastrophes, or
economic disasters, could result of the failure of the system. In this
thesis, conducted in the field of computer sciences, we are of course
particularly interested in safety-critical \textit{computer}
systems. The concept of computer system encompasses both the low-level
hardware-related and the more abstract software-related aspects
involved in computer technologies. Nowadays, computers pervade a
considerable number of objects and technologies that pave our
every-day life, including safety-critical systems. Thus, the risk
associated with the use of computers in certain critical applications
is real. Failures of safety-critical computer systems have happened
and continue happen; the list of critical incidents maintained by the
ACM Committee on Compters and Public Policy and Peter G. Neumann ever
since the mid 80s \cite{Neumann1994} is always growing\footnote{The
  \emph{risks digest} website continues to register the
  computer-related incidents that resulted or could result in
  important damages: \url{https://catless.ncl.ac.uk/Risks/} }.

To ensure the safety of computer systems involved in critical domains
such as avionics, railway, power plants or medicine, there exists a
number of standards and norms developed by international
organizations. These standards set of a number of rules and techniques
to be followed for the design, the production and the validation of
safety-critical computer systems. To cite some well-known standards,
the \href{https://www.eurocae.net/}{EUROCAE} and
\href{https://www.rtca.org/}{RTCA} organisms has devised the
ED-12C/DO-178C and ED-80/DO-254 industry standards for the development
cycle of software and hardware computer systems involved in avionics;
the \href{https://www.cencenelec.eu/}{CENELEC} has defined the
EN-50128 standard for the development of softwares for railway control
and protection systems; the \href{https://www.iec.ch}{IEC} is at the
origin of the IEC-60880 standard for the development of softwares
involved in the control of power plants. In this thesis, we are
interested in verifying a \textit{computer-related} methodology
involved in the production of safety-critical medical devices; thus,
we need to cite the EU 2017/145 regulation
text\footnote{\url{http://data.europa.eu/eli/reg/2017/745/2020-04-24}}
that sets the standard for the development of medical devices,
including how to validate the technologies involved in the production
line.

The rules imposed by the standards vary with respect to the criticity
of the considered systems; for instance, in the medical field, the
regulation text 2017/145 of the EU, pertaining to the marketing of
medical devices, sets a different requirement level whether we are
considering the production of dressings (level 0), or of
neuroprotheses (highest-level, level 4). The IEC does the same by
defining of SIL (Safety Integrity Level) measure that qualifies the
criticity level of a system.

Among the mandatory procedures, prescribed by the standards, that must
be followed to validate a computer system involved in a
safety-critical system, tests (unit, functional or integration tests)
or simulation (especially applied to hardwares) are to be noted.
However, in the case of the development safety-critical computer
systems, a particular kind of methods, called \textit{formal methods}
(FM), are also applied. In formal methods applied to computer systems,
a computer system is considered as a mathematical object
\cite{Bjorner2014}. As pointed out in \cite{Bowen1993}, ``formal
methods address \textit{correctness} issues'', that is whether or not
a system delivers the required service. The perks of formal methods
are to set a formal mathematical framework around a computer system.
This framework will permit to reason about the system and to prove
that the system meets some required properties. Thus, a ``formal
methods'' framework for computer systems needs at least a formal
requirement language, i.e. with a formal semantics, to express the
properties that a given system must verify. The expression of these
required properties is called the \textit{specification} of the
system.  On the hardware side, we can cite some specification
formalisms such as CCS \cite{Milner1980}, CSP \cite{Hoare1978}, Petri
nets \cite{Petri1962} or TLA+ \cite{Lamport1994} to describe reactive
systems (i.e. systems that continuously interact with an environment);
hardware description languages such as \vhdl{} \cite{Lipsett1986} and
\textsf{Verilog} \cite{IEEE1996} can also be considered as formal
specification languages for hardware designs if provided with a formal
semantics and embedded in a formal proof system (see for instance
\cite{Borrione1992}). On the software side, we can cite specification
languages such as VDM \cite{Bjorner1987a}, the Z notation
\cite{Abrial2013} on which is based the B language \cite{Abrial1991}
(included in the broader B-method); but also, all the theorem provers
and proof assistants that come with their own specification languages
such as \isahol{} \cite{Nipkow2002}, \coq{} \cite{Coq2021},
\textsf{PVS} \cite{Crow1995}, etc.  A FM framework must also provide a
formal proof system to reason about the formal specification of the
system. Some FM frameworks come with means to implement the computer
system or simplified version of the system (i.e. a model) in a formal
setting. This latter kind of framework enables to check if the
implementation of a system always complies with its specification,
i.e. the \textit{correctness/soundness} of the system, and if all the
aspects of the specification are met by the implementation, i.e. the
\textit{completeness} of the system.

Even though the purpose is always to check the correctness of systems,
there exist families of formal verification techniques. Through
relation that exists between hardware and software regarding their
common theoretical foundations, the techniques applied to the formal
verification of hardware computer systems or software computer systems
are quite similar. Thus, most of the techniques presented here apply
to both hardware and software developments.




%%% Local Variables:
%%% mode: latex
%%% TeX-master: "../main"
%%% End:
