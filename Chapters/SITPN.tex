\chapter{Implementation of the \hilecop{} high-level models}
\label{chap:hilecop-models}

\textsc{Research question of the chapter:}\\
\tikz[text width=.8\textwidth, align=center]{
  \node[draw, rectangle]{%
    \normalsize\sc
    \begin{tabular}{c}
      Which formal semantics should we consider\\
      in the case of \hilecop{} high-level models?
    \end{tabular}};
}

%----------------------------------------------------------------------------------------
%	GENERAL FACTS ABOUT PETRI NETS & SEMANTICS
%----------------------------------------------------------------------------------------

\section{General facts about Petri nets}
\label{sec:pn}

\begin{itemize}
\item Make a general presentation of Petri nets.\\
  \pnote{maybe put the general presentation of PNs in appendix, and
    focus here on the PNs semantics.}
\item Literature review on PNs semantics.\\
  \pnote{don't forget to
    mention linear logic when talking about PNs semantics}
\end{itemize}

\ding{212} CONCLUSION: ``we choose to adopt the formal semantics set
in previous work''.

\rnote{insist on the fact that most of the formalization is not my
  contribution, but comes from previous works}

\section{Synchronously executed Petri Nets}
\label{sec:sitpn-informal}

\begin{itemize}
\item Informal presentation of SITPNs\\
  \pnote{check out what we did for SEFM and PNSE} 
\end{itemize}

\section{Formal definition of SITPNs \& semantics}
\label{sec:sitpn-formal}

\begin{itemize}
\item Present the formal structure of SITPNs, main definitions, and semantics\\
  \pnote{check out SEFM and PNSE, and the last formalization of SITPN
    semantics (with the new definition of the set of \textit{Fired} transitions)}\\
  \bnote{it's better to present the changes made in the semantics in
    the chapter about the proof}
\item Present the contribution to formalizing the notion of
  well-defined SITPN, and motivate the reason of the contribution
  
  \pnote{the motivation is as follows: if sitpn is not well-defined,
    then the transformation returns an error + well-definition
    otherwise cannot prove behavior similarity}
\end{itemize}

\section{\coq{} implementation of SITPNs}
\label{sec:sitpn-coq-impl}

\begin{itemize}
\item Talk about the dependtly-typed version
\item Give the SITPN structure + SITPN state structure
\item Give the state transition relation
\item Give the full execution relation
\item Talk about the implementation of the set of fired transitions
\end{itemize}

\textsc{CONCLUSION OF THE CHAPTER:}

\fbox{\parbox{\textwidth}{
    \begin{itemize}
  \item We will consider the formal semantics defined in
    previous theses to perform the proof of semantic preservation
  \item Reminder of the contributions: an operational formalization of
    the set of fired transitions + formalization of conflict
    resolution and well-defined SITPNs + \coq{} implementation
  \end{itemize}}}

%%% Local Variables:
%%% mode: latex
%%% TeX-master: "../main"
%%% End:
