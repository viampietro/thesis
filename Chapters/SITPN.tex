\chapter{Implementation of the \hilecop{} high-level models}
\label{chap:hilecop-models}

In this chapter, we present the input formalism of our transformation
function: Synchronously executed Interpreted Time Petri Nets with
priorities (SITPNs). For the main part, the formalization of the SITPN
structure and semantics is the result of a former thesis
\cite{Leroux2014}. However, we contributed to the simplification of
the definition of the SITPN structure and its semantics, and added
complementary definitions for the purpose of the proof of behavior
preservation. Our main contribution to this part lies in the
implementation of the SITPN structure and semantics with the \coq{}
proof assistant. This chapter is structured as follows:
Section~\ref{sec:sitpn-informal} is a reminder on the principles
underlying the PN formalism and also gives an informal presentation of
SITPNs; Section~\ref{sec:sitpn-defs} lays out the formal definitions
of the SITPN structure and semantics; Section~\ref{sec:sitpn-impl}
deals with the implementation of SITPNs with the \coq{} proof
assistant.

\section{Informal presentation of Synchronously executed Petri nets}
\label{sec:sitpn-informal}
Here, fundamentals on the Petri net formalism are outlined, and
certain classes of Petri nets are described more precisely. Then, the
specificities of the Petri nets used to design the behavior of
electronic components in the \hilecop{} methodology are presented. For
more information on the topic of Petri nets, the reader can refer to
\cite{David1994}, \cite{Murata1989}, or \cite{Diaz2001}.

\subsection{Preliminary notions on Petri nets}
\label{subsec:pn-formalism}

Petri nets, invented by C. A. Petri \cite{Petri1962}, are used to
model a broad range of dynamic systems: resource sharing between
concurrent processes \cite{David1994}, behavior of agents in
multi-agent systems \cite{Celaya2007}, behavior of digital components
\cite{Yakovlev2006}. A Petri net is a directed graph, composed of two
types of node: place nodes (\emph{circles}) and transition nodes
(\emph{squares} or \emph{lines}). As shown in
Figure~\ref{fig:pn-example}, place nodes usually represent a part of
the state of the modelled system, here the states of two computer
processes and a semaphor; transition nodes usually refer to events
triggering the system evolution (or state changing).

\begin{figure}[H]
  \centering
  \includegraphics[keepaspectratio=true, width=\textwidth]{Figures/SITPN/pn-example}
  \caption[An example of Petri net]{An example of Petri net - A
    semaphor to prevent the parallel execution of \textit{Treatment 1}
    and \textit{Treatment 2}.}
  \label{fig:pn-example}
\end{figure}

\paragraph{Edges}
In a Petri net, directed edges link together places and
transitions. Places cannot be linked to other places, and the same
stands for transitions.  There are two kinds of edges, \textit{pre} or
\textit{incoming} edges, going from a place to a transition, and
\textit{post} or \textit{outcoming} edges, going from a transition to
a place. Places linked to a transition $t$ by incoming
(resp. outcoming) edges will be referred to as the \textit{input}
(resp. \textit{output}) of $t$. The same stands for a place $p$. For
instance, in Figure~\ref{fig:pn-example}, $p_0$ and $sem$ are the
input places of $t_0$, and $p_1$ is the output place of $t_0$; $t_1$
and $t_3$ are the input transitions of place $sem$, and $t_0$ and
$t_2$ are the output transitions of $sem$. Some weight --a natural
number-- is associated to the edges of a Petri net. If no label
appears on the edge then one is the default weight. Petri nets are
said to be \textit{generalized} when edge weights are possibly greater
than one.

\paragraph{Marking}
In Figure~\ref{fig:pn-example}, places $p_0$, $p_3$ and $sem$ are
marked with tokens, represented by little black circles.  This means
that places $p_0$, $p_3$ and $Sem$ are currently active.  The
distribution of tokens over places is called the \textit{marking} of
the net. The marking of a Petri net reflects the overall state of the
modelled system at a certain moment in its activity cycle.

\paragraph{Transition firing}
In a Petri net, the marking evolves based on a token
consump\-tion-production system. Transitions consume tokens from their
input places, and produce tokens to their output places. This whole
system is called \textit{transition firing}. In order to be
\textit{firable}, a transition must be \textit{sensitized} (or
\textit{enabled}), meaning that the number of tokens in each of its
input places must be equal or greater than the weight of its incoming
edges. For instance, in Figure~\ref{fig:pn-example}, the transition
$t_0$ is sensitized because the weight of the arc ($p_0$, $t_0$) is of
one (default value), and place $p_0$ is marked with one token, and the
same stands for the number of tokens in place $sem$ and the weight of
the arc ($sem$, $t_0$). As a counter example, transition $t_3$ is not
sensitized because there is no tokens in its input place $p_2$.
Depending on the class of PNs that is considered, other parameters
affect the \textit{firability} of transitions (see interpreted Petri
nets, time Petri nets and
Section~\ref{subsec:hpn-particularities}). When a sensitized
transition is fired, tokens are retrieved from its input places (as
much tokens as the weight of the arcs) and produced in its output
places (as much tokens as the weight of the arcs).  This process
represents the occurence of an event --denoted by the transition--
triggering the evolution of the system from one state to
another. Figure~\ref{fig:firing-example} shows the state of the PN of
Figure~\ref{fig:pn-example} after the firing of the transition $t_0$.

\begin{figure}[H]
  \centering
  \includegraphics[keepaspectratio=true, width=\textwidth]{Figures/SITPN/firing-example}
  \caption[An example of transition firing]{The PN of
    Figure~\ref{fig:pn-example} after the firing of transition $t_0$.}
  \label{fig:firing-example}
\end{figure}

In Figure~\ref{fig:firing-example}, the tokens in the input places of
$t_0$, i.e. places $p_0$ and $sem$ have been consumed, and one token
has been produced in the output place $p_1$. The current marking
indicates that the task ``Treatment 1'' is being performed (place
$p_1$ is active).

In Figure~\ref{fig:pn-example}, transition $t_0$ and $t_2$ are enabled
at the same time. However, the \emph{standard} semantics of Petri nets
is such that only one transition can be fired in that case. Either
$t_0$ consumes the token in place $sem$ or $t_2$ does, but never
both. Thus, the transition firing process in the standard PN semantics
is an undeterministic process. From the marking of
Figure~\ref{fig:pn-example}, two marking are reachable: the marking
resulting of the firing of transition $t_0$ and the one resulting of
the firing of transition $t_2$. Also, the transition firing process is
asynchronous. As soon as a transition is enabled, the transition
firing process can be triggered. 

\paragraph{Extended Petri nets}
The class of \textit{extended} Petri nets introduces the inhibitor and
test edges. As shown in Figure~\ref{fig:inhib-test-arcs}, test arc
tips are black circles and inhibitor arc tips are white
circles. Inhibitor and test edges are incoming edges, always coming
from a place toward a transition.
%
\begin{figure}[H]
  \centering
  \includegraphics[keepaspectratio=true, width=.2\textwidth]{Figures/SITPN/inhib-arc}  
  \hspace{50pt}
  \includegraphics[keepaspectratio=true, width=.2\textwidth]{Figures/SITPN/test-arc}
  \caption[Two examples of extended Petri nets.]{Two examples of
    extended Petri nets; on the left side, a PN with inhibitor arcs;
    on the right side, a PN with test arcs.}
  \label{fig:inhib-test-arcs}
\end{figure}
The particularity of the inhibitor and test edges is that they are not
consuming tokens in input places after the firing of a transition.
Indeed, they are just testing the number of tokens in incoming places
to determine if the transition is enabled. Inhibitor arcs ensure that
the number of tokens in input places is strictly lower than their
weights; test arcs ensure that the number of tokens in pre-places is
equal or greater than their weights. Therefore, on the left side of
Figure~\ref{fig:inhib-test-arcs}, transition $t_0$ is sensitized
because there is strictly less than one token in place $p_0$ and
strictly less than two tokens in place $p_1$. On the right side of
Figure~ \ref{fig:inhib-test-arcs}, transition $t_0$ is sensitized
because there is at least one token in place $p_0$ and three tokens in
place $p_1$.

\paragraph{Interpreted Petri nets}
Interpreted Petri nets (IPN) \cite{David1994} are intended to describe
the interaction between a system and its outside
environment. Interpretation introduces three new concepts:

\begin{itemize}

\item Continuous actions, associated to the places of a Petri net.
  Actions associated to a place $p$ are activated as long as $p$ is
  marked. For instance, when modelling a controller with a IPN,
  actions can correspond to the setting of a electric signal
  controlling some actuator (e.g, maintaining a LED on).
  
\item Functions (or discrete actions), associated to the transitions
  of a Petri net. When a transition $t$ is fired, all functions
  associated to $t$ are executed.  Functions can be any kind of
  discrete operations --variable incrementation, for instance--
  manipulating both internal variable and external signal values.
  
\item Conditions, associated to the transitions of a Petri net.
  Conditions are boole\-an expressions receiving their values from the
  environment of the PN. In an IPN, a transition is firable only if
  all its associated conditions are \texttt{true} (or \texttt{false}
  in the case where an inverse condition is associated).
  
\end{itemize}

Figure~\ref{fig:ipn} illustrates the use of actions, functions and
conditions in an interpreted Petri net.

\begin{figure}[H]
  \centering
  \includegraphics[keepaspectratio=true, width=.7\textwidth]{Figures/SITPN/interpreted-pn}
  \caption[An example of Interpreted Petri net.]{An example of
    Interpreted Petri net; on the left side, the interpreted Petri
    net; on the right side, examples of tests associated to conditions
    and operations associated to actions and functions.}
  \label{fig:ipn}
\end{figure}%
%
In Figure~\ref{fig:ipn}, the action $a_0$ is activated as place $p_0$
is marked by one token. Also, function $f_0$ will be executed at the
firing of $t_0$, that is if condition $c_0$ is \texttt{true} and $t_0$
is sensitized. On the right side of Figure~\ref{fig:ipn}, we associate
a semantics to conditions, actions and functions in terms of concrete
tests or operations. However, when considering the semantics of IPNs,
what is of interest to us is the value of conditions and the execution
state of actions and functions. We are not interested in interpreting
the Boolean expressions associated to conditions but only to retrieve
their value; likewise, we are only interested in the fact that a given
action/function is activated/executed but not in what is its effect on
the environment.

\paragraph{Time Petri nets}

In a time Petri net (TPN), time intervals are associated to
transitions. The goal of associating a time interval to a transition
is to constrain the firing of this transition to a certain time
window. As shown in Figure~\ref{fig:tpn}, time intervals are of the
form $[a, b]$, where $a\in\mathbb{N}^{*}$ and
$b\in\mathbb{N}^{*}\sqcup\{\infty\}$. Time intervals can also be
defined with real numbers but in this thesis we not interested in this
kind of time intervals. In Figure~\ref{fig:tpn}, time counters are
represented in red between diamond brackets. The current value of time
counters is part of the state of the TPN, along with its current
marking, whereas time intervals are part of the static structure of
the TPN.

\begin{figure}[H]
  \centering
  \includegraphics[keepaspectratio=true, width=0.2\textwidth]{Figures/SITPN/time-pn}
  \caption{An example of time Petri net.}
  \label{fig:tpn}
\end{figure}

For each sensitized transition associated with a time interval, time
counters are incremented at a certain time step, previously defined by
the modeller. For instance, in the case of SITPNs, i.e. Petri nets
used in the \hilecop{} methodology, the reference time step for the
incrementation of time counters is the clock cycle.

When a transition associated with a time interval is fired or
disabled, a reset order is sent to the transition to set its time
counter to zero. The value of reset orders (Boolean values) is also a
part of the TPN state.  In time Petri nets, a transition is firable
only if its time counter value is within its time interval. For
instance, in Figure~\ref{fig:tpn}, only transition $t_0$ is firable.

There are multiple possible firing policy for TPNs. Here, we will only
consider the \textit{imperative} firing policy: as soon as a time
counter reaches the lower bound of a time interval, the associated
transition must be fired.

\paragraph{Petri nets with priorities}

Two transitions are in structural conflict if they have a common input
place connected through a \textit{basic} arc (i.e. neither inhibitor
nor test). When two transitions in structural conflict are firable at
the same time, then, the conflict becomes \textit{effective}. A Petri
net with priorities, it is possible to specify a firing priority in
the case where the conflict between two transitions becomes
effective. In that case, the transition with the highest firing
priority will always be fired
first. Figure~\ref{fig:structural-conflict} illustrates the
application of a priority relation to solve the effective conflict
between two transitions.

\begin{figure}[H]
  \centering
  \includegraphics[keepaspectratio=true,width=.6\textwidth]{Figures/SITPN/structural-conflict}
  \caption[An example of transitions in structural and effective
  conflict.]{An example of transitions in structural and effective
    conflict. In subfigure (b), the dotted arrow represents the
    priority relation between $t_0$ and $t1$. The transition with the
    highest firing priority is at the source of the arrow; here,
    transition $t_0$.}
  \label{fig:structural-conflict}
\end{figure}

\subsection{Particularities of SITPNs}
\label{subsec:hpn-particularities}

Here, we will informally present the specificities of the Petri nets
describing the internal behavior of the \hilecop{} high-level model
components. These Petri nets are called: Synchronously executed,
extended, generalized, Interpreted, Time Petri Nets with priorities or
SITPNs. SITPNs are a combination of multiple classes of PNs, namely:
extended PNs, generalized PNs, interpreted PNs, time PNs and PNs with
priorities. These classes were presented in the above section. We will
now talk about another aspect of SITPNs that constitutes the
originality of the formalism compared to the standard PN semantics:
its synchronous execution.

The class of interpreted Petri nets increases the expressiveness of
the \hilecop{} high-level models. However, to ensure the safe
execution of functions after the synthesis of the designed circuit on
a FPGA card, the whole system must be synchronized with a clock signal
\cite{Leroux2014}. As a consequence, a clock signal also regulates the
evolution of SITPNS (i.e. it is a part of their semantics). The
evolution of an SITPN is \textit{synchronized} with two clock events:
the rising edge and the falling edge of the
signal. Figure~\ref{fig:sync-exec} depicts the process of state
evolution, following the clock signal.

\begin{figure}[H]
  \centering
  \includegraphics[keepaspectratio=true, width=.9\textwidth]{Figures/SITPN/sync-exec}
  \caption{Evolution of an SITPN synchronized with a clock signal.}
  \label{fig:sync-exec}
\end{figure}

Considering the different classes of PNs that define SITPNs, the state
of an SITPN is characterized by its marking, the value of time
counters, the reset orders assigned to time counters, the
execution/activation status of actions/functions (Boolean values), and
the value of conditions (also Boolean). As shown in figure
\ref{fig:sync-exec}, the state evolution process of an SITPN is
divided in two parts is divided in two steps. At the rising of the
clock signal, the marking is updated, i.e. transitions are fired,
reset orders are sent to all transitions that have been fired or
disabled by the firing process, and all functions associated to fired
transitions are executed. Then, on the falling edge of the clock
signal, fresh condition values are provided by the environment, the
time counter values are either incremented, reset or values are
stalling (see the following remark on locked time counters), and all
action associated with marked places are
activated. Figure~\ref{fig:sitpn-state-exec} gives an example of the
evolution of the state of a given SITPN through one clock cycle. The
aim of this figure and the explanation that follows is to give some
hints to the reader about the semantics of SITPNs before giving its
formal definition in Section~\ref{sec:sitpn-sem}.

\begin{figure}[H]
  \centering
  \includegraphics[keepaspectratio=true, width=.9\textwidth]{Figures/SITPN/sitpn-state-evol}
  \caption[Evolution of an SITPN over one clock cycle.]{Evolution of
    an SITPN over one clock cycle. Conditions appear in green when
    their value is \texttt{true} and in red otherwise; actions and
    functions appear in green when they are activated/executed and in
    red otherwise; time counters appear in red and between diamond
    brackets; time counters appear in blue when they receive reset
    orders.}
  \label{fig:sitpn-state-exec}
\end{figure}

From Step~1 to Step~2, the rising edge of the clock signal triggers
the SITPN state evolution. Here, transition $t_0$ is fired. At Step~1,
transition $t_0$ gathers all the necessary conditions to trigger the
firing process, namely: $t_0$ is enabled by the current marking,
condition $c_0$ is \texttt{true} (appears in green), $t_0$'s time
counter value is within the time interval. As a consequence, one token
is consumed in place $p_0$ and one token is produced in place $p_1$,
function $f_0$ is executed at Step~2 (appears in green) and a reset
order is sent to the time counter of $t_0$ (appears in blue). From
Step~2 to Step~3, the activation activation status are updated; $a_0$
stays activated place $p_0$ is marked; $a_1$ becomes newly activated
as $p_1$ is marked. The time counter values are updated; $t_0$'s time
counter is set to zero as the transition previously received a reset
order. However, as $t_0$ is still sensitized by the new marking, its
time counter is incremented. Thus, the resulting time counter value at
Step~3 is one (i.e. result of reset plus incrementation). Also, the
condition values are retrieved from the environment. As a consequence,
condition $c_0$ takes the value \texttt{false}.

\paragraph{A remark on priorities}

The semantics of synchronous execution is that all transitions are
fired at the same time. In Figure~\ref{fig:double-consum}, transitions
$t_0$ and $t_1$ are both sensitized by place $p_0$, and consequently
are both fired at the same time. The system acts as if two tokens were
available in place $p_0$, one for the firing of $t_0$ and another for
the firing of $t_1$.

\begin{figure}[H]
  \centering
  \includegraphics[keepaspectratio=true, width=.8\textwidth]{Figures/SITPN/double-consum}
  \caption[Double consumption of token in a SITPN.]{Double consumption
    of token in a SITPN. On the left side, the current marking before
    the firing of $t_0$ and $t_1$; on the right side, the marking
    resulting of the firing of $t_0$ and $t_1$. The arrow indicates
    the occurence of a rising edge that triggers the firing process.}
  \label{fig:double-consum}
\end{figure}

In the context of an SITPN, a branching like the one of
Figure~\ref{fig:sitpn-state-exec}, normally interpreted as an
disjunctive branching, takes the semantics of a conjunctive branching
when no priority are prescribed between the conflicting
transitions. To avoid the phenomenon of ``double consumption'' of
tokens, we enforce the resolution of the structural conflict by means
of mutual exclusion or through the application of priorities. This
policy about the resolution of structural conflict is part of the
definition of a well-defined SITPN presented in
Section~\ref{sec:sitpn-wd}, and is mandatory to produce safe models of
digital systems.

When a structural conflict between transitions is solved with
priorities, the firing process follows a slightly different
mechanism. As illustrated in Figure~\ref{fig:resid-marking}, to
determine which transitions of $t_0$, $t_1$ and $t_2$ must be fired, a
\emph{residual marking} is computed by following the priority
order. For each transition of the group $t_0$, $t_1$ and $t_2$, the
residual marking represents the remnant of tokens in $p_0$ after the
firing of transitions with a higher firing priority. Thus, in the
semantics of SITPNs, we add an extra condition to the firing of a
transition: to be fired, a transition must be enabled by the current
marking, must have all its conditions valuated to \texttt{true}, must
have its time counter within its time interval and must be enabled by
the residual marking. The computation of the residual marking only
applies the consumption phase of the firing process, i.e. no tokens
are generated.

\begin{figure}[H]
  \centering
  \includegraphics[keepaspectratio=true, width=.9\textwidth]{Figures/SITPN/resid-marking}
  \caption[Computation of the residual marking of a group of
  conflicting transitions.]{Computation of the residual marking for a
    group of conflicting transitions. At \circled{1}
    (resp. \circled{2} and \circled{3}), the residual marking for
    transition $t_0$ (resp. $t_1$ and $t_2$). Condition $c_0$ is in
    red to indicate that it is currently valuated to \texttt{false}.}
  \label{fig:resid-marking}
\end{figure}

In Figure~\ref{fig:resid-marking}, the residual marking for $t_0$
corresponds to the marking obtained after the firing of all
transitions with a higher priority than $t_0$. As $t_0$ is the
transition with the highest firing priority, the residual marking for
$t_0$ is equal to the current marking. Transition $t_0$ gathers all
the conditions to be firable and is enabled by the residual marking;
thus, $t_0$ is fired. The residual marking for $t_1$ is the marking
obtained after the firing of $t_0$, i.e. the only transition with a
higher priority.  As illustrated at \circled{2}, $t_1$ is enabled by
the residual marking but it does not gather all the conditions to be
firable; indeed, condition $c_0$ associated to $t_0$ is
\texttt{false}. Thus, $t_0$ is not fired. The residual marking for
$t_2$ is obtained after the firing of $t_0$ only. Indeed, transition
$t_1$ has a higher firing priority than $t_2$, however, it is not part
of the set of fired transitions and thus it is not taken into account
in the computation of the residual marking for $t_2$. Thus, the
residual marking at \circled{3} enables transition $t_2$, and as $t_2$
gathers all the conditions to be firable, then $t_2$ is fired.

\paragraph{Locked time counters}

SITPNs inherit from both the properties of time PNs and interpreted
PNs. As a consequence, the phenomenon of \emph{locked} time counters
is a consequence of this combination. As illustrated in
Figure~\ref{fig:locked-tc}, the value of a time counter can overreach
the upper bound of the associated time interval.

\begin{figure}[H]
  \centering
  \includegraphics[keepaspectratio=true, width=.5\textwidth]{Figures/SITPN/locked-tc}
  \caption[An example of locked time counter.]{An example of locked
    time counter. Condition $c$ is equal \texttt{false} and thus
    appears in red.}
  \label{fig:locked-tc}
\end{figure}

This situation can only arise if a condition hinders the firing of a
given transition while the considered transition is still enabled by
the marking. As a consequence, the time counter will be incremented at
every clock cycle until the upper bound of the time interval is
overreached. Then, at this point, the time counter is said to be
\emph{locked} and its value will no more evolve.  In
Figure~\ref{fig:locked-tc}, condition $c$ is valuated to
\texttt{false}, thus, transition $t$ can not be fired but is still
sensitized by the marking. As a consequence, on the next falling edge,
the time counter of transition $t$ is incremented and overreaches the
upper bound of interval $[2,4]$ and thus becomes locked.



%%% Local Variables:
%%% mode: latex
%%% TeX-master: "../../main"
%%% End:


\section{Formal definition of the SITPN structure and semantics}
\label{sec:sitpn-defs}
We hope that the reader has now a fair understanding of the concepts
underlying the SITPNs and of the dynamics governing the SITPN state
evolution process. In this section, we give the formal definition of
the SITPN structure and of its execution semantics. We also introduce
the concept of a \emph{well-defined} SITPN at the end of the section.

\subsection{SITPN structure}
\label{sec:struct-and-wd}

\noindent{}The structure of SITPNs is formally defined as follows:

\begin{definition}[SITPN]
  \label{def:sitpn}
  A synchronously executed, extended, generalized, interpreted, and
  time Petri net with priorities is a tuple
  ${<}P,T,pre,post,M_0,{\succ},\mathcal{A},\mathcal{C},\mathcal{F},
  \mathbb{A},\mathbb{C},\mathbb{F},{I_s}{>}$, where we have:
  % 
  \begin{enumerate}
  \item $P=\{p_0,\ldots,p_n\}$, a finite set of places.
  \item $T=\{t_0,\ldots,t_m\}$, a finite set of transitions.
  \item
    $pre\in{}P\rightarrow{}T\nrightarrow(\mathbb{N}^{*}\times\{\mathtt{basic},\mathtt{inhib},\mathtt{test}\})$,
    the function associating a weight to place-transition edges.
  \item $post\in{}T\rightarrow{}P\nrightarrow\mathbb{N}^{*}$, the
    function associating a weight and a type to transition-place
    edges.
  \item $M_0\in{}P\rightarrow\mathbb{N}$, the initial marking of the SITPN.
  \item $\succ\subseteq{}(T\times{}T)$, the priority relation which is
    a partial order over the set of transitions.
  \item $\mathcal{A}=\{a_0,\ldots,a_i\}$, a finite set of continuous actions.
  \item $\mathcal{F}=\{f_0,\ldots,f_k\}$, a finite set of functions
    (instantaneous actions).
  \item $\mathcal{C}=\{c_0,\ldots,c_j\}$, a finite set of conditions.
  \item $\mathbb{A}$ $\in$ ${}P$ $\rightarrow$ $\mathcal{A}$
    $\rightarrow$ $\mathbb{B}$, the function associating actions to
    places.  $\forall{}p\in{}P$, $\forall{}a\in\mathcal{A}$,
    $\mathbb{A}(p,a)=\mathtt{true}$, if $a$ is associated to $p$,
    $\mathbb{A}(p,a)=\mathtt{false}$ otherwise.
  \item $\mathbb{F}\in{}T\rightarrow\mathcal{F}\rightarrow\mathbb{B}$,
    the function associating functions to transitions.
    $\forall{}t\in{}T,~\forall{}f\in\mathcal{F},$
    $\mathbb{F}(t,f)=\mathtt{true}$, if $f$ is associated to $t$,
    $\mathbb{F}(t,f)=\mathtt{false}$ otherwise.
    
  \item $\mathbb{C} \in T \rightarrow \mathcal{C} \rightarrow\{-1,0,1\}$, the
    function associating conditions to transitions.
    $\forall t \in T$, $\forall c \in \mathcal{C}$,
    $\mathbb{C}(t,c)=1$, if $c$ is associated to $t$,
    $\mathbb{C}(t,c)=-1$, if $\bar{c}$ is associated to $t$,
    $\mathbb{C}(t,c)=0$ otherwise.
  \item $I_s\in{}T\nrightarrow\mathbb{I}^{+}$, the partial function
    associating static time intervals to transitions, where
    $\mathbb{I}^{+}\subseteq(\mathbb{N}^{*}\times(\mathbb{N^{*}}\sqcup\{\infty\}))$.
    $T_i$ denotes the definition domain of $I_s$, i.e. the set of time
    transitions.
  \end{enumerate}
\end{definition}

\subsection{SITPN State}
\label{sec:sitpn-state}

The SITPN semantics describes the evolution of the state of an SITPN
through a given number of clock cycles; thus, we must first define the
SITPN state structure:

\begin{definition}[SITPN State]
  \label{def:sitpnstate}
  For a given $sitpn\in{}SITPN$, let $S(sitpn)$ be the set of possible
  states of $sitpn$. An SITPN state $s\in{}S(sitpn)$ is a tuple
  ${<}M,I,reset_t,ex,cond{>}$, where:
  \begin{enumerate}
  \item $M\in{}P\rightarrow\mathbb{N}$ is the current marking of sitpn.
  \item\label{item:sitpn-state-tc} $I\in{}T_i{}\rightarrow\mathbb{N}$
    is the function mapping time transitions to their current time
    counter value.
  \item\label{item:sitpn-state-rst}
    $reset_t\in{}T_i\rightarrow\mathbb{B}$ is the function mapping
    time transitions to time counter reset orders (defined as
    Booleans).
  \item $ex\in{}\mathcal{A}\sqcup\mathcal{F}\rightarrow\mathbb{B}$ is
    the function representing the current activation (resp. execution)
    state of actions (resp. functions).
  \item $cond\in\mathcal{C}\rightarrow\mathbb{B}$ is the function representing the
    current value of conditions (defined as Booleans).
  \end{enumerate}
\end{definition}

In Items~\ref{item:sitpn-state-tc} and \ref{item:sitpn-state-rst} of
Definition~\ref{def:sitpnstate}, \emph{time} transitions refer to
transitions with a time interval, i.e. the transitions belonging to
the domain of $I_s$.

\subsection{Preliminary definitions and fired transitions}
\label{sec:fired-trans}

Before formalizing the full SITPN semantics, we must introduce some
definitions and notations, especially the definition of a
\emph{firable} and a \emph{fired} transition. We use the two following
notations to simplify the formalization of the SITPN semantics.

\begin{notation}[Relations between markings]
  \label{not:markings}
  For all relation $\mathcal{R}$ existing between two marking
  functions $M$ and $M'$, the expression $\mathcal{R}(M,M')$ is a
  notation for $\forall{}p\in{}P,~\mathcal{R}(M(p),M'(p))$. For
  instance, $M'=M-\sum\limits_{t_i\in{}Pr(t)}pre(t_i)$ is a notation
  for
  $\forall{}p\in{}P,~M'(p)=M(p)-\sum\limits_{t_i\in{}Pr(t)}pre(p,t_i)$.
\end{notation}

\begin{notation}[Sum expressions and arc types]
  \label{not:sum-exprs}
  Many times in this document, we need to express the number of tokens
  coming in or out of places, after the firing of a certain subset of
  transitions. To do so, we use two kinds of sum expression:
  \begin{enumerate}
  \item The first kind of expression computes a number of output
    tokens. For instance, for a given place $p$,
    $\sum\limits_{t\in{}T'}pre(p,t)$ where $T'\subseteq{}T$.

    The expression $\sum\limits_{t\in{}T'}pre(p,t)$ is a notation for
    $\sum\limits_{t\in{}T'}\begin{cases}\omega~if~pre(p,t)=(\omega,\mathtt{basic})\\
      0~otherwise \end{cases}$

    When computing a sum of output tokens (i.e. resulting of a firing
    process), we want to add to the sum the weight of the arc between
    place $p$ and a transition $t\in{}T'$ only if there exists an arc
    of type $\mathtt{basic}$ from $p$ to $t$ (remember that the test
    and inhibitor never lead to the withdrawal of tokens during the
    firing process). Otherwise, we add 0 to the sum as it is a neutral
    element of the addition operator over natural numbers.
    
  \item The second kind of expression computes a number of input
    tokens.  For instance, for a given place $p$,
    $\sum\limits_{t\in{}T'}post(p,t)$ where $T'\subseteq{}T$.

    The expression $\sum\limits_{t\in{}T'}post(p,t)$ is a notation for
    $\sum\limits_{t\in{}T'}\begin{cases}\omega~if~post(t,p)=\omega\\
      0~otherwise \\ \end{cases}$

    Here, we add the weight of the arc from $t$ to $p$ only if there
    exists such an arc; we add 0 to the sum otherwise.
  \end{enumerate}
  Therefore, in the remainder of the document, we will use the
  conciser notation $\sum\limits_{t\in{}T'}pre(p,t)$ to denote an
  output token sum, and $\sum\limits_{t\in{}T'}post(t,p)$ to denote an
  input token sum.
\end{notation}

\noindent{}We give the formal definition of the sensitization (see
Section~\ref{subsec:pn-formalism} for an informal definition) of a
transition by a given marking as follows:

\begin{definition}[Sensitization]
  \label{def:sens}
  A transition $t\in{}T$ is said to be sensitized, or enabled, by a
  marking $M$, which is noted $t\in{}Sens(M)$, if
  $\forall{}p\in{}P,\forall\omega\in\mathbb{N}^{*},~\big(pre(p,t)=(\omega,\mathtt{basic})\vee{}pre(p,t)=(\omega,\mathtt{test})\big)\Rightarrow{}M(p)\ge{}\omega$,
  and $pre(p,t)=(\omega,\mathtt{inhib})\Rightarrow{}M(p)<{}\omega$.
\end{definition}

\noindent{}We give the formal definition of a \emph{firable}
transition at a given SITPN state as follows:

\begin{definition}[Firability]
  \label{def:firable}
  A transition $t\in{}T$ is said to be firable at a state
  $s={<}M,I,reset_t,ex,cond{>}$, which is noted $t\in{}Firable(s)$, if
  $t\in{}Sens(M)$, and $t\notin{}T_i$ or $I(t)\in{}I_s(t)$, and
  $\forall c \in \mathcal{C}, \mathbb{C}(t, c) = 1 \Rightarrow cond(c)
  = 1$ and $\mathbb{C}(t, c) = -1 \Rightarrow cond(c) = 0$.
\end{definition}

As explained in Section~\ref{subsec:hpn-particularities}, the
firability conditions are not sufficient for a transition to be
fired. A transition must also be enabled by the residual marking to go
through the firing process. Definition~\ref{def:fired} gives the
formal definition of a fired transition at a given SITPN state:

\begin{definition}[Fired]
  \label{def:fired}
  A transition $t\in{}T$ is said to be fired at the SITPN state
  $s={<}M,I,reset_t,ex,$ $cond{>}$, which is noted $t\in{}Fired(s)$,
  if $t\in{}Firable(s)$ and
  $t\in{}Sens\big(M-\sum\limits_{t_i\in{}Pr(t)}pre(t_i)\big)$, where
  $Pr(t)=\{t_i~|~t_i\succ{}t\wedge{}t_i\in{}Fired(s)\}$.
\end{definition}

One can notice that the definition of the set of fired transitions is
recursive. Indeed, to compute the residual marking necessary to the
definition of a fired transition, the $Pr$ set must be defined. For a
given transition $t$, the $Pr$ set represents all the transitions with
a higher firing priority than $t$ that are also fired transitions;
hence the recursive definition.

In Definition~\ref{def:fired}, the marking
$M-\sum\limits_{t_i\in{}Pr(t)}pre(t_i)$ formally qualifies the
residual marking for a given transition $t$ and at a given SITPN state
$s$.

\subsection{SITPN Semantics}
\label{sec:sitpn-sem}

We formalize the semantics of a given SITPN as a transition
system. The SITPN state transition relation defined in the SITPN
semantics as two cases of definition, one for each clock event.  The
SITPN state transition relation describes the evolution of the state
of a SITPN.

\begin{definition}[SITPN Semantics]
  \label{def:semantics}
  The semantics of a given $sitpn\in{}SITPN$ is the transition system
  ${<}L,E_c,\rightarrow{>}$ where:
  \begin{itemize}[label=-]
  \item $s_0\in{}S(sitpn)$ is the initial state of the SITPN, such
    that
    $s_0=<M_0,O_\mathbb{N},O_\mathbb{B},O_\mathbb{B},O_\mathbb{B}>$,
    where $M_0$ is the initial marking of the SITPN, $O_\mathbb{N}$ is
    a function that always returns 0, $O_\mathbb{B}$ is a function
    that always returns \texttt{false}.
  \item $L\subseteq{}\{\uparrow,\downarrow\}\times{}\mathbb{N}$ is the
    set of transition labels. A label is a couple $(clk,\tau)$
    composed of a clock event $clk\in\{\uparrow,\downarrow\}$, and a
    time value $\tau\in\mathbb{N}$ expressing the current count of
    clock cycles.
  \item
    $E_c\in{}\mathbb{N}\rightarrow\mathcal{C}\rightarrow\mathbb{B}$ is
    the environment function, which gives (Boolean) values to
    conditions ($\mathcal{C}$) depending on the count of clock cycles
    ($\mathbb{N}$).
  \item $\rightarrow\subseteq{}S(sitpn)\times{}L\times{}S(sitpn)$ is the SITPN state
    transition relation, which is noted
    $E_c,\tau\vdash{}s\xrightarrow{clk}s'$ where
    $s,s'\in{}S(sitpn)$ and $(clk,\tau)\in{}L$, and which is defined
    as follows:
    \begin{itemize}[label=$\square$]
    \item $\forall\tau\in\mathbb{N}$, $\forall{}s,s'\in{}S(sitpn)$, we
      have $E_c,\tau\vdash{}s\xrightarrow{\downarrow}s'$, where
      $s=<M,I,reset_t,ex,cond>$ and $s'=<M,I',reset_t,ex',cond'>$, if:
      \begin{enumerate}[label=(\arabic*)]
      \item\label{it:cond-env} $cond'$ is the function giving the
        (Boolean) values of conditions that are extracted from the
        environment $E_c$ at the clock count
        $\tau$, i.e.:\\
        $\forall{}c\in{}\mathcal{C},~cond'(c)=E_c(\tau,c)$.
      \item\label{it:activate-actions} All the actions associated
        with at least one
        marked place in the marking $M$ are activated, i.e.:\\
        $\forall{}a\in{}\mathcal{A},~ex'(a)=\sum\limits_{p\in{}marked(M)}\mathbb{A}(p,a)$
        where $marked(M)=\{p'\in{}P~\vert~M(p')>0\}$.
      \item\label{it:reset-counters} All the time transitions that are
        sensitized by the marking $M$ and received the order to reset
        their time intervals, have their time counter reset and
        incremented, i.e.:\\
        $\forall{}t\in{}T_i,~t\in{}Sens(M)\land{}reset_t(t)=\mathtt{true}
        \Rightarrow{}I'(t)=1$.
      \item\label{it:inc-counters} All the time transitions that are
        sensitized by the marking $M$, and
        did not receive a reset order, increment their time counters if time counters are still active, i.e.:\\
        $\forall{}t\in{}T_i,~t\in{}Sens(M)\land{}reset_t(t)=\mathtt{false}
        \land{}(I(t)\le{}upper(I_s(t))\lor{}upper(I_s(t))=\infty)\Rightarrow{}$
        $I'(t)=I(t)+1$.
      \item\label{it:locked-counters} All the time transitions
        verifying the same
        conditions as above, but with locked counters, keep having locked counters (values are stalling), i.e.:\\
        $\forall{}t\in{}T_i,~t\in{}Sens(M)\land{}reset_t(t)=\mathtt{false}
        \land{}I(t)>{}upper(I_s(t))\land{}upper(I_s(t))\neq\infty\Rightarrow{}$
        $I'(t)=I(t)$.
      \item\label{it:reset-not-sens} All the time transitions disabled by the marking $M$ have their time counters set to zero, i.e.:\\
        $\forall{}t\in{}T_i,~t\notin{}Sens(M)\Rightarrow{}I'(t)=0$.
      \end{enumerate}
    \item $\forall\tau\in\mathbb{N}$, $\forall{}s,s'\in{}S(sitpn)$, we
      have $E_c,\tau\vdash{}s\xrightarrow{\uparrow}s'$, where
      $s=<M,I,reset_t,ex,cond>$ and $s'=<M',I,reset_t',ex',cond>$, if:
      \begin{enumerate}[label=(\arabic*),resume]
      \item\label{it:new-marking} $M'$ is the new marking resulting
        from
        the firing of all the transitions contained in $Fired(s)$, i.e.:\\
        $\forall{}p\in{}P,~M'(p)=M(p)-\sum\limits_{t\in{}Fired(s)}pre(p,t)+\sum\limits_{t\in{}Fired(s)}post(t,p)$.
        
      \item\label{it:reset-order} A time transition receives a reset
        order if it is fired at state $s$, or, if there exists a place
        $p$ connected to $t$ by a \texttt{basic} or \texttt{test arc}
        and at least one output transition of $p$ is fired and the
        transient marking of $p$ disables $t$; no reset order is sent
        otherwise:
        \begin{equation*}
          \begin{split}
            \forall{}t\in{}T_i,~& t\in{}Fired(s) \\
            & \lor\big(\exists{}p\in{}P,\omega\in\mathbb{N}^{*},~pre(p,t)=(\omega,\mathtt{basic})\lor{}pre(p,t)=(\omega,\mathtt{test}) \\
            & \quad\quad\land\sum\limits_{t_i\in{}Fired(s)}pre(p,t_i)>0 \\
            & \quad\quad\land{}M(p)-\sum\limits_{t_i\in{}Fired(s)}pre(p,t_i)<\omega\big)\Rightarrow{}reset'_t(t)=\mathtt{true}, \\
            & and~reset'_t(t)=\mathtt{false}~otherwise. \\
          \end{split}
        \end{equation*}
      \item\label{it:exec-fun} All functions associated with at least one fired transition are executed, i.e:\\
        $\forall{}f\in{}\mathcal{F},~ex'(f)=\sum\limits_{t\in{}Fired(s)}\mathbb{F}(t,f)$.
      \end{enumerate}
    \end{itemize}
  \end{itemize}
\end{definition}

Rules~\ref{it:cond-env} to \ref{it:reset-not-sens} describe the SITPN
state evolution at the falling edge of the clock
signal. Rules~\ref{it:cond-env} and \ref{it:activate-actions} pertain
to the update of condition values and to the update of the activation
status of actions. Note that in Rule~\ref{it:activate-actions} (and
also in Rule~\ref{it:exec-fun}), the sum expression corresponds to the
Boolean sum expression, i.e. the application of the \texttt{or}
operator over the elements of the iterated
set.Rules~\ref{it:reset-counters}, \ref{it:inc-counters},
\ref{it:locked-counters} and \ref{it:reset-not-sens} focus on the
update of time counter values.  In Rule~\ref{it:inc-counters} of the
SITPN semantics, the \emph{active} time counters refer to the time
counters that have not yet overreached the upper bound of their
associated time interval. Of course, a time counter is always active
when the upper bound is infinite. In Rule~\ref{it:locked-counters},
the \emph{locked} time counters refer to the time counters that have
overreached the upper bound of their associated time interval. Of
course, time counters can never be locked in the presence of an
infinite upper bound. In Rules~\ref{it:inc-counters} and
\ref{it:locked-counters}, for a given time interval $i$, $upper(i)$
denotes the upper bound of the time interval, and $lower(i)$ denotes
the lower bound of the time interval.

Rules~\ref{it:new-marking} to \ref{it:exec-fun} describe the SITPN
state evolution at the rising edge of the clock signal.
Rule~\ref{it:new-marking} corresponds to the marking update. The
computation of the new marking uses the set of fired transitions at
state $s$, i.e. $Fired(s)$. Rule~\ref{it:exec-fun} pertains to the
update of the function execution status. Rule~\ref{it:reset-order}
computes the reset orders for time transitions. There are two cases
where a time transition receives the order to reset its time
counter. First, if the transition is one of the fired transitions at
state $s$, then its time counter must be reset on the next falling
edge. Second, if the transition is disabled in a \emph{transient}
manner, then its time counter must also be
reset. Figure~\ref{fig:trans-marking} illustrates the case of a
transition disabled by the \emph{transient} marking, i.e. the marking
obtained after the token consumption phase of the firing process.

\begin{figure}[H]
  \centering
  \includegraphics[keepaspectratio=true, width=.8\textwidth]{Figures/SITPN/trans-marking}
  \caption[Transient marking and reset orders.]{An example of
    transition that receives a reset order after being disabled by the
    transient marking. At \circled{1}, the marking before the firing
    of transitions $t_0$ and $t_2$; at \circled{2}, the transient
    marking; at \circled{3}, the marking at the end of the firing
    process.}
  \label{fig:trans-marking}
\end{figure}

In Figure~\ref{fig:trans-marking}, the situation at \circled{1}
describes the state of the SITPN before a rising edge. Based on the
current SITPN state at \circled{1}, transition $t_0$ and $t_2$ will be
fired on the next rising edge event. % Situation~\circled{2} precedes
% Situation~\circled{3}, but both happen at the occurrence of the rising
% edge of the clock signal.
Situation~\circled{2} depicts the marking obtained after the
consumption phase of the firing process (once the rising edge
occurred), i.e. the so-called \emph{transient}
marking. Situation~\circled{3} corresponds to the marking at the end
of the firing process, where $t_0$ and $t_2$ have been fired. At
\circled{3}, transition $t_1$ is enabled by the marking. However, at
\circled{2}, the transient marking disables $t_1$ and thus $t_1$ must
receive a reset order (represented by a \textcolor{blue}{blue} time
counter).  This reset order will be taken into account at the next
falling edge event, and the time counter associated with transition
$t_1$ will then be reset.

\subsection{SITPN Execution}
\label{sec:sitpn-exec}

As a part of the SITPN semantics, we define here the SITPN execution
and SITPN full execution relations. These relations bind a given SITPN
to the execution trace, i.e. a time-ordered list of states, that it
produces when executed over a given number of clock cycles. These
definitions are additional elements corresponding to our own
contribution to the formalization of the SITPN semantics.

% \begin{definition}[SITPN Execution Cycle]
%   For a given $sitpn\in{}SITPN$, two states $s,s''\in{}S(sitpn)$, a
%   clock cycle count $\tau\in\mathbb{N}$, and an environment
%   $E_c\in\mathbb{N}\rightarrow{}\mathcal{C}\rightarrow{}\mathbb{B}$,
%   $sitpn$ passes from state $s$ to state $s''$ in one clock cycle,
%   written $E,\tau\vdash{}sitpn,s\xrightarrow{\uparrow,\downarrow}s''$
%   iff $\exists{}s'$
%   s.t. $E_c,\tau\vdash{}sitpn,s\xrightarrow{\uparrow}s'$ and
%   $E_c,\tau\vdash{}sitpn,s'\xrightarrow{\downarrow}s''$.
% \end{definition}

\begin{definition}[SITPN execution]
  \label{def:sitpn-exec}
  For a given $sitpn\in{}SITPN$, a starting state $s\in{}S(sitpn)$, a
  clock cycle count $\tau\in\mathbb{N}$, and an environment
  $E_c\in\mathbb{N}\rightarrow{}\mathcal{C}\rightarrow{}\mathbb{B}$,
  $sitpn$ yields the execution trace $\theta$ from starting state $s$,
  written $E_c,\tau\vdash{}sitpn,s\rightarrow{}\theta$, by following
  the two rules below:
  
  \begin{table}[H]
    \begin{tabular}{@{}l}
      {\fontsize{10}{12}\selectfont
      \textsc{ExecutionEnd}} \\
      
      {\begin{prooftree}
          \infer0 {E_c,0\vdash{}sitpn,s\rightarrow{}[~]}
        \end{prooftree}} 
    \end{tabular}
  \end{table}  
  \begin{table}[H]
  \begin{tabular}{@{}l}
    {\fontsize{10}{12}\selectfont
    \textsc{ExecutionLoop}} \\
    
    {\begin{prooftree}[template={\inserttext}]

        \hypo{$E_c,\tau\vdash{}sitpn,s\xrightarrow{\uparrow}s'$}
        \hypo{$E_c,\tau\vdash{}sitpn,s'\xrightarrow{\downarrow}s''$}
        \hypo{$E_c,\tau-1\vdash{}sitpn,s''\rightarrow{}\theta$}
        
        \infer3[$\tau>0$]{$E_c,\tau\vdash{}sitpn,s\rightarrow{}(s' :: s'' :: \theta)$}
      \end{prooftree}} 
  \end{tabular}
\end{table}
\end{definition}

The \textsc{ExecutionEnd} rule states that the execution of a
$sitpn\in{}SITPN$, starting from a state $s\in{}S(sitpn)$ in the
environment
$E_c\in{}\mathbb{N}\rightarrow\mathcal{C}\rightarrow\mathbb{B}$,
yields an empty execution trace if the clock count comes down to $0$.

The \textsc{ExecuteLoop} rule describes how the execution trace
related to the execution of a $sitpn\in{}SITPN$ is built in the case
where the clock count $\tau$ is greater than zero. The final execution
trace is composed of a head state $s'$, followed by state $s''$ and
the tail trace $\theta$. The $::$ operator builds a new trace by
adding a new element at the head of an existing trace. Starting from
state $s$, $sitpn$ reaches state $s'$ after a rising edge event; then
from state $s'$, it reaches state $s''$ after a falling edge event.
Finally, the execution trace $\theta$ is obtained through the
recursive call to the SITPN execution relation where $sitpn$ is
executed during $\tau-1$ cycles starting from state $s''$.


\begin{definition}[SITPN full execution]
  \label{def:sitpn-full-exec}
  For a given $sitpn\in{}SITPN$, a clock cycle count
  $\tau\in\mathbb{N}$, and an environment
  $E_c\in\mathbb{N}\rightarrow{}\mathcal{C}\rightarrow{}\mathbb{B}$,
  $sitpn$ yields the execution trace $\theta$ starting from its
  initial state $s_0\in{}S(sitpn)$ (as defined in
  Definition~\ref{def:semantics}), written
  $E_c,\tau\vdash{}sitpn\rightarrow{}\theta$, by following the two
  rules below:
  
  \begin{table}[H]
    \begin{tabular}{@{}l}
      {\fontsize{10}{12}\selectfont\textsc{FullExec0}} \\
      
      {\begin{prooftree}[template={\inserttext}]
          
          \infer0{$E_c,0\vdash{}sitpn\xrightarrow{full}[s_0]$}
        \end{prooftree}} 
    \end{tabular}
  \end{table}

  \begin{table}[H]
    \begin{tabular}{@{}l}
      {\fontsize{10}{12}\selectfont\textsc{FullExecCons}} \\
      
      {\begin{prooftree}[template={\inserttext}]
          \hypo{$E_c,\tau\vdash{}s_0\srarrow{\uparrow_0}{\fontsize{6}{8}\selectfont}s_0$}
          \hypo{$E_c,\tau\vdash{}s_0\srarrow{\downarrow}{\fontsize{6}{8}\selectfont}s$}
          \hypo{$E_c,\tau-1\vdash{}sitpn,s\rightarrow\theta_s$}
          \infer3[$\tau>0$]{$E_c,\tau\vdash{}sitpn\xrightarrow{full}(s_0 :: s_0 :: s :: \theta_s)$}
        \end{prooftree}} 
    \end{tabular}
  \end{table}
\end{definition}

The \textsc{FullExecCons} rule of the SITPN full execution relation
(Definition~\ref{def:sitpn-full-exec}) appeals to the SITPN execution
relation (Definition~\ref{def:sitpn-exec}). However, the definition of
the SITPN full execution relation is necessary because the first cycle
of execution, starting from the initial state $s_0$, is particular. As
shown in the premises of Rule~\textsc{FullExecCons}, the first rising
edge is idle. We consider that no transitions are fired during the
first rising edge. Thus, the first rising edge does not change the
initial state $s_0$, and we denote the particular first rising edge
with the sign $\uparrow_0$ over the SITPN transition relation.

\subsection{Well-definition of a SITPN}
\label{sec:sitpn-wd}

To be able to transform a given SITPN into a \vhdl{} design and also
to perform the proof of semantic preservation, a SITPN must verify
some properties ensuring its \emph{well-definition}. Here, we
formalize the predicate stating that a given SITPN is well-defined.

% \paragraph{Conflict Definition} In the definition of an SITPN, the
% priority relation is a mean to solve a situation of conflict in a pair
% of transitions. We will keep the definition of a conflict as simple as
% possible. Informally, the transitions of a pair are in conflict if
% they have an common input place, and if both are linked to this input
% place by a \texttt{basic} arc. Figure~\ref{fig:basic-conflict} depicts
% a situation of conflict between two transitions.

% At some point of the execution of the SITPN, the marking possibly
% enables the two transitions of a conflicting pair in such a manner
% that the firing of one transition disables the other; then, the
% conflict is said to be \emph{effective}. The behavior of PNs is
% fundamentally asynchronous, and a token can only be consumed by one
% transition. However, in a synchronous setting as the one of the SITPN,
% all transitions are first elected to be fired, and then all fired at
% the same time.  Therefore, the situation can arise where a same token
% is consumed by two transitions, on behalf of them being transitions in
% effective conflict that are both elected to be fired (e.g,
% Figure~\ref{fig:basic-conflict}). 

% \begin{figure}[H]
%   \centering
%   \includegraphics[keepaspectratio,width=.3\linewidth]{Figures/SITPN/struct-conflict-with-basic}
%   \caption{Example of conflict between two transitions}
%   \label{fig:basic-conflict}
% \end{figure}


The main interest of the well-definition predicate is to prevent the
phenomenon of the ``double consumption'' of tokens at the execution of
a SITPN. In a well-defined SITPN, a conflict resolution strategy must
be applied to every group of transitions in structural conflict.  We
must be able to decide which transition in a conflicting pair will be
fired when the conflict becomes effective. Thus, we give the formal
definition of a conflicting pair of transitions and of a conflict
group.

\begin{definition}[Conflict]
  \label{def:conflict}
  For a given $sitpn\in{}SITPN$, two transitions $t,t'\in{}T$ are in
  conflict if there exist a place $p\in{}P$ and two weights
  $\omega,\omega'\in\mathbb{N}^{*}$ such that
  $pre(p,t)=(\omega,\mathtt{basic})$ and
  $pre(p,t')=(\omega',\mathtt{basic})$.
\end{definition}

A conflict group qualifies a finite set of transitions that are all in
conflict with each other through at least a common input place. In
Figure~\ref{fig:conflict-not-trans}, the set $\{t_0,t_1\}$ is a
conflict group.  The formal definition of a conflict group is as
follows:

\begin{definition}[Conflict Group]
  \label{def:cgroup}
  For a given $sitpn\in{}SITPN$, $T_c\subseteq{}T$ is a conflict group
  if there exists a place $p$ such that
  $\forall{}t\in{}T,\big(\exists{}\omega\in\mathbb{N}^{*},~pre(p,t)=(\omega,\mathtt{basic})\big)\Leftrightarrow{}t\in{}T_c$.
\end{definition}

Contrary to the statement made in \cite[p. 67]{Leroux2014}, we no more
consider the notion of conflict as being transitive. To illustrate
this, Figure~\ref{fig:conflict-not-trans} shows two conflict groups:
$\{t_0,t_1\}$ and $\{t_1,t_2\}$. In a well-defined $SITPN$ (see
Section~\ref{sec:sitpn-wd}), all conflicts in a conflict group must be
dealt with, i.e. for all pair of transitions in the group the conflict
must be solved. However, we no more consider transitions $t_0$ and
$t_2$ as in conflict. We argue that even when no conflict resolution
technique is applied between transitions in the same situation as
$t_0$ and $t_2$, the execution of the $SITPN$ can neither result in
the double-consumption of a token, nor in the case where a transition
is not elected to be fired even though it ought to be. Therefore, we
no more consider the construction of merged conflict group (i.e,
conflict groups must be merged into one if their intersection is not
empty; e.g, $\{t_0,t_1,t_2\}$ in Figure~\ref{fig:conflict-not-trans})
as being necessary.

\begin{figure}[H]
  \centering
  \includegraphics[keepaspectratio,width=.4\linewidth]{Figures/SITPN/conflict-not-trans}
  \caption[An example of two separate conflict groups.]{An example of
    two separate conflict groups, namely: $\{t_0,t_1\}$ and
    $\{t_1,t_2\}$.}
  \label{fig:conflict-not-trans}
\end{figure}

When the conflict between a pair of transitions becomes effective,
there are two ways to be sure that only one transition will be
fired. The first way is to define a firing order through a priority
relation. The second way is to use a mean of mutual exclusion. A mean
of mutual exclusion ensures that the two transitions of a conflicting
pair will never be firable at the same time. For now, we only consider
two ways of mutual exclusion, namely: mutual exclusion with
complementary conditions and mutual exclusion with inhibitor
arcs. Here, we give the formal definition of these two means of mutual
exclusion.

% \begin{definition}[Mutual exclusion with disjoint time intervals]
%   \label{def:mutex-ti}
%   Given two conflicting transitions $t_0$ and $t_1$, $t_0$ and $t_1$
%   are in mutual exclusion with disjoint time intervals if there exists
%   $a,b\in\mathbb{N}^{*}$ and $c,d\in\mathbb{N}^{*}\sqcup\{\infty\}$
%   such that $I_s(t_0)=[a,b]$ and $I_s(t_1)=[c,d]$ and there is no
%   overlapping between $[a,b]$ and $[c,d]$.
% \end{definition}

\begin{definition}[Mutual exclusion with complementary conditions]
  \label{def:mutex-conds}
  Given two conflicting transitions $t_0$ and $t_1$, $t_0$ and $t_1$
  are in mutual exclusion with complementary conditions if there
  exists $c\in\mathcal{C}$ such that
  $(\mathbb{C}(t_0,c)=1\land{}\mathbb{C}(t_1,c)=-1)$ or
  $(\mathbb{C}(t_0,c)=-1\land{}\mathbb{C}(t_1,c)=1)$.
\end{definition}

\begin{definition}[Mutual exclusion with an inhibitor arc]
  \label{def:mutex-inhib} Given two conflicting transitions $t_0$ and
  $t_1$, $t_0$ and $t_1$ are in mutual exclusion with an inhibitor arc
  if there exists $p\in{}P$ and $\omega\in{}\mathbb{N}^{*}$ such that
  $(pre(p,t_0)=(\omega,\mathtt{basic})\lor{}pre(p,t_0)=(\omega,\mathtt{test}))\land{}pre(p,t_1)=(\omega,\mathtt{inhib})$
  or
  $(pre(p,t_1)=(\omega,\mathtt{basic})\lor{}pre(p,t_1)=(\omega,\mathtt{test}))\land{}pre(p,t_0)=(\omega,\mathtt{inhib})$.
\end{definition}

Figure~\ref{fig:mutex} illustrates the two means of mutual exclusion
that can be applied to solve a conflict between two transitions.

\begin{figure}[H]
  \centering
  \includegraphics[keepaspectratio,width=.5\linewidth]{Figures/SITPN/mutex}
  \caption[Examples of conflicting transitions in mutual exclusion.]{
    Examples of conflicting transitions in mutual exclusion. At
    \circled{1}, an example of mutual exclusion with complementary
    conditions; at \circled{2}, an example of mutual exclusion with an
    inhibitor arc.}
  \label{fig:mutex}
\end{figure}

In Figure~\ref{fig:mutex}, in situation \circled{1}, condition $c_1$
is associated to $t_1$ and the complementary condition is associated
to $t_0$ thus creating the mutual exclusion. In situation \circled{2},
the arcs $(p_0,t_0)$ and $(p_0,t_1)$ ensure the mutual exclusion
between transitions $t_0$ and $t_1$. Note that in the structure of
mutual exlcusion with an inhibitor arc, the weight of the inhibitor
arc and of the basic or test arc must be the same; otherwise, the
mutual exclusion is not effective.

A given $sitpn\in{}SITPN$ is well-defined if it enforces some
properties needed on the \hilecop{} source models before the
transformation into \vhdl{}. If the properties, layed out in
Definition~\ref{def:wd-sitpn}, are not ensured, they will lead to
compile-time errors during the transformation of the SITPN into a
\vhdl{} design.

\begin{definition}[Well-defined SITPN]\label{def:wd-sitpn}
  A given $sitpn\in{}SITPN$ is well-defined if:
  \begin{itemize}
  \item $T\neq\emptyset$, the set of transitions must not be empty.
  \item $P\neq\emptyset$, the set of places must not be empty.
  \item There is no isolated place, i.e, a place that has neither
    input nor output transitions:\\
    $\nexists{}p\in{}P,~input(p)=\emptyset\wedge{}output(p)=\emptyset$,
    where $input(p)$ (resp. $output(p)$) denotes the set of input
    (resp. output) transitions of $p$.
  \item There is no isolated transition, i.e, a transition that has
    neither
    input nor output places:\\
    $\nexists{}t\in{}T,~input(t)=\emptyset\wedge{}output(t)=\emptyset$,
    where $input(t)$ (resp. $output(t)$) denotes the set of input
    (resp. output) places of $t$.
  \item For all conflict group as defined in
    Definition~\ref{def:cgroup}, either all conflicts (i.e. for all
    pair of transitions in the conflict group) are solved by one of
    the mean of mutual exclusion, or, the priority relation is a
    strict total order over the transitions of the conflict group.
  \end{itemize}
\end{definition}

\subsection{Boundedness of a SITPN}
\label{sec:sitpn-bounded}

We conclude the formalization of the SITPN structure and semantics by
the expression of the boundedness of a SITPN model with respect to its
execution trace. In the manner of the well-definition property, the
boundedness of a SITPN model is a mandatory condition to apply the
semantic preservation theorem (cf. Remark~\ref{rem:bounded-sitpn} in
Chapter~\ref{chap:proof}). A SITPN model is bounded if there exists a
\textit{bound} for the number of tokens that the places can hold in
the course of the execution of the model; formally:

\begin{definition}[Bounded SITPN]
  \label{def:bounded-sitpn}
  A given $sitpn\in{}SITPN$ is said to be bounded if for all execution
  environment
  $E_c\in\mathbb{N}\rightarrow\mathcal{C}\rightarrow\mathbb{B}$, clock
  cycle count $\tau\in\mathbb{N}$, execution trace
  $\theta\in\mathtt{list}(S(sitpn))$ such that
  $E_c,\tau\vdash{}sitpn\xrightarrow{full}\theta$, then there exists a
  bound $k\in\mathbb{N}$ such that for all $p\in{}P$ and
  $s\in{}\theta$, $s.M(p)\le{}k$.
\end{definition}

We extend the definition of a bounded SITPN model to a version where
the bound denoting the maximal marking of each place of the model is
passed through a function $b\in{}P\rightarrow\mathbb{N}$.

\begin{definition}[Bounded SITPN through a maximal marking function]
  A given $sitpn\in{}SITPN$ is said to be bounded through the maximal
  marking function $b\in{}P\rightarrow\mathbb{N}$, written
  $\lceil{}sitpn\rceil^b$, if for all execution environment
  $E_c\in\mathbb{N}\rightarrow\mathcal{C}\rightarrow\mathbb{B}$, clock
  cycle count $\tau\in\mathbb{N}$, execution trace
  $\theta\in\mathtt{list}(S(sitpn))$ such that
  $E_c,\tau\vdash{}sitpn\xrightarrow{full}\theta$, then for all
  $p\in{}P$ and $s\in{}\theta$, $s.M(p)\le{}b(p)$.
\end{definition}

%%% Local Variables:
%%% mode: latex
%%% TeX-master: "../../main"
%%% End:


\section{\coq{} implementation of SITPNs}
\label{sec:sitpn-impl}
In this section, we present our mechanization of the SITPN structure
and semantics with the \coq{} proof assistant. The source code is
available to the reader at the address
\url{https://github.com/viampietro/ver-hilecop}. More precisey, the
implementation of the SITPN structure and semantics is to be found
under the \texttt{sitpn/dp} directory. We have made a first
implementation of SITPNs without the use of dependent types. For this
first version, we have also implemented a SITPN interpret (a so-called
\emph{token player}) and proved that the interpret was sound and
complete w.r.t the SITPN semantics. This first implementation of the
SITPNs and the formal proof of soundness and completeness are
available at \url{https://github.com/viampietro/sitpns}.

\subsection{Implementation of the SITPN and the SITPN state structure}
\label{sec:sitpn-struct-impl}

Listing~\ref{lst:sitpn-struct-impl} presents the implementation of the
SITPN structure as a \coq{} record type. The implementation is almost
similar to the formal definition of the SITPN structure given in
Definition~\ref{def:sitpn}.

\begin{lstlisting}[language=coq,caption={Implementation of the SITPN structure in \coq{}.},label={lst:sitpn-struct-impl},framexleftmargin=1.5em,xleftmargin=2em,]
Record Sitpn := BuildSitpn {
 
  places : list nat; #\label{line:pls}#
  transitions : list nat; #\label{line:trs}#
  P := { p | (fun p0 => In p0 places) p }; #\label{line:P}#
  T := { t | (fun t0 => In t0 transitions) t };
 
  pre : P -> T -> option (ArcT * $\mathbb{N}^{*}$); #\label{line:pre}#
  post : T -> P -> option $\mathbb{N}^{*}$;
  $M_0$ : P -> nat;
  $I_s$ : T -> option TimeInterval; #\label{line:Is}#
      
  conditions : list nat; #\label{line:conds}#
  actions : list nat; #\label{line:acts}#
  functions : list nat; #\label{line:funs}#
  $\mathcal{C}$ := { c | (fun c0 => In c0 conditions) c };
  $\mathcal{A}$ := { a | (fun a0 => In a0 actions) a };
  $\mathcal{F}$ := { f | (fun f0 => In f0 functions) f };
      
  $\mathbb{C}$ : T -> $\mathcal{C}$ -> MOneZeroOne; #\label{line:mone}# 
  $\mathbb{A}$ : P -> $\mathcal{A}$ -> bool;
  $\mathbb{F}$ : T -> $\mathcal{F}$ -> bool;

  pr : T -> T -> Prop; #\label{line:pr}#
      
}.
\end{lstlisting}

We use lists of natural numbers, i.e. \texttt{list nat} in \coq{}, to
define the finite sets of places (Line~\ref{line:pls}), transitions
(Line~\ref{line:trs}), actions (Line~\ref{line:acts}), conditions
(Line~\ref{line:conds}) and functions (Line~\ref{line:funs}) in the
\texttt{Sitpn} record. We want to use these finite sets in the
signature of functions appearing in the structure (e.g use the finite
set of places $P$ in the signature of the initial marking
$M_0\in{}P\rightarrow\mathbb{N}$). To do so, we leverage the \coq{}
\texttt{sig} type to define subsets of elements verifying a certain
property. Thus, we define the finite set $P$ as the subset of natural
numbers that are members of the \texttt{places} list
(Line~\ref{line:P}). We use the \texttt{In} relation defined in the
\coq{} standard library to express the membership of a natural number
regarding the elements of the \texttt{places} list. Also, the
\texttt{ArcT} type (Line~\ref{line:pre}) implements the set
$\{\mathtt{inhib},\mathtt{test},\mathtt{basic}\}$; the
\texttt{TimeInterval} type (Line~\ref{line:Is}) implements the set
$\mathbb{I}^{+}$ of time intervals, and the \texttt{MOneZeroOne} type
(Line~\ref{line:mone}) implements the set $\{0,1,-1\}$. The priority
relation is implemented by the \texttt{pr} function
(Line~\ref{line:pr}) taking two transitions in parameter and
projecting to the type of logical propositions, i.e. the \texttt{Prop}
type.

Listing~\ref{lst:sitpn-state-impl} presents the implementation of the
SITPN state structure as a \coq{} record type.

\begin{lstlisting}[language=coq,caption={Implementation of the SITPN state structure in \coq{}.},label={lst:sitpn-state-impl},framexleftmargin=1.5em,xleftmargin=2em,]
Record SitpnState (sitpn : Sitpn) := BuildSitpnState {

  M : P sitpn -> nat;
  I : $T_i$ sitpn -> nat;
  reset : $T_i$ sitpn -> bool;
  cond : C sitpn -> bool;
  ex : A sitpn + F sitpn -> bool;

}.
\end{lstlisting}

The \texttt{SitpnState} type definition depends on a given SITPN
passed as a parameter; it is an example of dependent type.  Projection
functions are automatically generated to access the attributes of a
record at the declaration of a type with the \texttt{Record} keyword.
Thus, in Listing~\ref{lst:sitpn-state-impl}, we can refer to the set
of places of \texttt{sitpn} with the term \texttt{P sitpn}. The term
\texttt{$T_i$ sitpn} denotes the set of time transitions of
\texttt{sitpn}. The set of time transitions for a given SITPN is
declared as a \texttt{sig} type qualifying to the subset of transitions
with an associated time interval.

\subsection{Implementation of the SITPN semantics}
\label{sec:sitpn-sem-impl}

Here, we present our implementation of the SITPN semantics. In
Listing~\ref{lst:sitpn-state-trans-rel-impl}, we give an excerpt of
the implementation of the SITPN state transition relation, i.e. the
core of the SITPN semantics.

\begin{lstlisting}[language=coq,caption={Excerpt of the implementation of the SITPN state transition relation in \coq{}.},label={lst:sitpn-state-trans-rel-impl},framexleftmargin=1.5em,xleftmargin=2em,]
Inductive SitpnStateTransition 
  (sitpn : Sitpn) ($E_c$ : nat -> C sitpn -> bool) ($\tau$ : nat) (s s' : SitpnState sitpn) : 
  Clk -> Prop :=
| SitpnStateTransition_falling :

    (* Rule #\ref{it:activate-actions}# *)
    (forall a marked sum, 
       Sig_in_List (P sitpn) (fun p => M s p > 0) marked -> #\label{line:sig-marked}#
       BSum (fun p => $\mathbb{A}$ p a) marked sum -> #\label{line:bsum-marked}#
       ex s' (inl a) = sum) ->

    (* Rules #\ref{it:reset-counters}, \ref{it:inc-counters}, \ref{it:locked-counters} and \ref{it:reset-not-sens}# *)
    (forall (t : Ti sitpn), ~Sens (M s) t -> I s' t = 0) ->
    (forall (t : Ti sitpn), Sens (M s) t -> reset s t = true -> I s' t = 1) ->
    (forall (t : Ti sitpn),
        Sens (M s) t ->
        reset s t = false ->
        (TcLeUpper s t \/ upper t = i+) -> I s' t = S (I s t)) ->
    (forall (t : Ti sitpn),
        Sens (M s) t ->
        reset s t = false ->
        (upper t <> i+ /\ TcGtUpper s t) -> I s' t = S (I s t)) ->
    
    (** Conclusion *)
    SitpnStateTransition $E_c$ $\tau$ s s' $\downarrow$

| SitpnStateTransition_rising:

    (** Rule #\ref{it:new-marking}# *)  
    (forall fired, IsNewMarking s fired (M s')) ->

    (* Rule #\ref{it:exec-fun}# *)
    (forall f fired sum, #\label{line:exec-fun}#
       IsFiredList s fired -> 
       BSum (fun t => $\mathbb{F}$ t f) fired sum -> 
       ex s' (inr f) = sum) ->

    (* Conclusion *)
    SitpnStateTransition $E_c$ $\tau$ s s' $\uparrow$.
\end{lstlisting}

The SITPN state transition relation is implemented in \coq{} as an
inductive type with two constructors, i.e. one for each clock
event. The relation has 6 parameters: an SITPN, an environment $E_c$,
a clock count $\tau$, two SITPN states \coqe|s| and \coqe|s'| and a
clock event. Note that the two states \coqe|s| and \coqe|s'| are bound
to the SITPN parameter through their type, i.e. \texttt{SitpnState
  sitpn}.

In the construction case \texttt{SitpnStateTransition_falling}, we
give the implementation of Rules~\ref{it:activate-actions},
\ref{it:reset-counters}, \ref{it:inc-counters},
\ref{it:locked-counters} and \ref{it:reset-not-sens} defined in the
SITPN semantics.  The sum term of Rule~\ref{it:activate-actions},
i.e. $\sum\limits_{p\in{}marked(M)}\mathbb{A}(p,a)$, is implemented by
Lines~\ref{line:sig-marked} and \ref{line:bsum-marked}. At
Line~\ref{line:sig-marked}, the \texttt{Sig_in_List} predicate states
that all the inhabitant of the \texttt{P sitpn} type (i.e. the places
of \texttt{sitpn}) that verifies the property \texttt{(fun p => M s p
  > 0)} (i.e. the marking of a place is greater than zero at state
\texttt{s}) are members of the \texttt{marked} list. Because we can
not iterate over the elements of a given \texttt{sig} type, we use the
\texttt{Sig_in_List} relation to convert a \texttt{sig} type into a
list. Lists are iterable by definition. At
Line~\ref{line:bsum-marked}, the \texttt{BSum} relation states that
\texttt{sum} is the Boolean sum obtained by applying the function
\texttt{(fun p => $\mathbb{A}$ p a)} to the elements of the
\texttt{marked} list. Rules~\ref{it:reset-counters},
\ref{it:inc-counters}, \ref{it:locked-counters} and
\ref{it:reset-not-sens} are almost similar in their implementation to
the description of Definition~\ref{def:semantics}. The \coq{} term
\texttt{Sens (M s) t} implements the term $t\in{}Sens(M)$. Due to the
particular nature of the upper bound of a time interval, i.e. defined
over the set $\mathbb{N}^{*}\sqcup{}\{\infty\}$, the test that the
current time counter of a given transition $t$ is less than or equal
to the upper bound is implemented by a separate predicate
\texttt{TcLeUpper}. Similarly, the \texttt{TcGtUpper} predicate
implements the inverse test.

In the construction case \texttt{SitpnStateTransition_rising}, we give
the implementation of Rules~\ref{it:new-marking} and \ref{it:exec-fun}
defined in the SITPN semantics. In the implementation of
Rule~\ref{it:new-marking}, the \texttt{IsNewMarking} predicate hides
away the expression:

$\forall{}p\in{}P,~M'(p)=M(p)-\sum\limits_{t\in{}Fired(s)}$ $pre(p,t)$
$+\sum\limits_{t\in{}Fired(s)}post(t,p)$.

In its definition, the \texttt{IsNewMarking} predicate first checks
that the \texttt{fired} list implements the set of fired transitions
at state \texttt{s}. Then, it builds the marking at state \texttt{s'}
for each place $p$, i.e. \texttt{(M s')}, by consuming and producing a
number of tokens starting from the marking of $p$ at state
\texttt{s}. The \texttt{fired} list is helpful to qualify the input
token sum and the output token sum for a given place. Similarly to the
implementation of Rule~\ref{it:activate-actions}, the implementation
of Rule~\ref{it:exec-fun} at Line~\ref{line:exec-fun} leverages the
\texttt{BSum} predicate to compute the Boolean sum
$\sum\limits_{t\in{}Fired(s)}\mathbb{F}(t,f)$. The term
\texttt{IsFiredList s fired} states that the \texttt{fired} list
implements the set of fired transitions at state \texttt{s} so we can
use the \texttt{fired} list to compute the above sum.

%%% Local Variables:
%%% mode: latex
%%% TeX-master: "../../main"
%%% End:


%%% Local Variables:
%%% mode: latex
%%% TeX-master: "../main"
%%% End:
