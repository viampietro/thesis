%%%%%%%%%%%%%%%%%%%%%%%%%%%%%%%%%%%%%%%%%%%%%%%
%%%%%% SIMILAR INITIAL STATES HYPOTHESES %%%%%%
%%%%%%%%%%%%%%%%%%%%%%%%%%%%%%%%%%%%%%%%%%%%%%%

\begin{definition}[Initial State Hypotheses]
  \label{def:init-states-hyps}
  Given an $sitpn\in{}SITPN$, $d\in{}design$,
  $\gamma\in{}WM(sitpn,d)$,
  $\Delta\in{}ElDesign(d,\mathcal{D}_\mathcal{H}),\sigma_{e},\sigma_0\in{}\Sigma(\Delta)$,
  assume that:
  \begin{itemize}
  \item SITPN $sitpn$ translates into design $d$:
    $\lfloor{}sitpn\rfloor_\mathcal{H}=(d,\gamma)$
  \item $\Delta$ is the elaborated version of $d$, $\sigma_e$ is the
    default state of $\Delta$, i.e, state of $\Delta$ where all signals have their default value:\\
    $\mathcal{D}_\mathcal{H},\emptyset\vdash{}d\srarrow{elab}{\fontsize{6}{8}\selectfont}(\Delta,\sigma_{e})$
    
  \item $\sigma_0$ is the initial state of $\Delta$: 
    $\Delta,\sigma_{e}\vdash{}d.cs\srarrow{init}{\fontsize{6}{8}\selectfont}\sigma_0$
  \end{itemize}
\end{definition}

\def\inithyps{For all $sitpn\in{}SITPN$, $d\in{}design$,
  $\gamma\in{}WM(sitpn,d)$,
  $\Delta\in{}ElDesign(d,\mathcal{D}_\mathcal{H}),\sigma_{e},\sigma_0\in{}\Sigma(\Delta)$
  that verify the hypotheses of Def.~\ref{def:init-states-hyps},}

%%%%%%%%%%%%%%%%%%%%%%%%%%%%%%%%%%%%%%%%%%
%%%%%% SIMILAR INITIAL STATES LEMMA %%%%%%
%%%%%%%%%%%%%%%%%%%%%%%%%%%%%%%%%%%%%%%%%%

\begin{lemma}[Similar Initial States]
  \label{lem:sim-init-states}
  \inithyps{} then $\gamma\vdash{}s_0\sim\sigma_0$.
\end{lemma}

\begin{proof}
  By definition of \nameref{sec:state-sim}, 6 subgoals.
  \begin{frameb}
    \begin{enumerate}
    \item
      $\forall{}p\in{}P,id_p\in{}Comps(\Delta),\sigma_p^0\in\Sigma(\Delta(id_p))~s.t.~\gamma(p)=id_p$
      and $\sigma_0(id_p)=\sigma_p^0,$
      $~s_0.M(p)=\sigma_p^0("s\_marking")$.\label{it:marking-eq}
    \item
      $\forall{}t\in{}T_i,id_t\in{}Comps(\Delta),\sigma_t^0\in\Sigma(\Delta(id_t))~s.t.~\gamma(t)=id_t$ and $\sigma_0(id_t)=\sigma_t^0,$\\
      $upper(I_s(t))=\infty\land{}s_0.I(t)\le{}lower(I_s(t))\Rightarrow{}s_0.I(t)=\sigma_t^0("s\_tc")\land{}$\\
      $upper(I_s(t))=\infty\land{}s_0.I(t)>{}lower(I_s(t))\Rightarrow{}\sigma_t^0("s\_tc")=lower(I_s(t))\land{}$\\
      $upper(I_s(t))\neq\infty\land{}s_0.I(t)>{}upper(I_s(t))\Rightarrow{}\sigma_t^0("s\_tc")=upper(I_s(t))\land{}$\\
      $upper(I_s(t))\neq\infty\land{}s_0.I(t)\le{}upper(I_s(t))\Rightarrow{}s_0.I(t)=\sigma_t^0("s\_tc")$.\label{it:tc-eq}
    \item
      $\forall{}t\in{}T_i,id_t\in{}Comps(\Delta),\sigma_t^0\in\Sigma(\Delta(id_t))~s.t.~\gamma(t)=id_t$
      and
      $\sigma_0(id_t)=\sigma_t^0,~$\\
      $s_0.reset_t(t)=\sigma_t^0("s\_reinit\_time\_counter")$.\label{it:reset-eq}
    \item
      $\forall{}c\in\mathcal{C},id_c\in{}Ins(\Delta)~s.t.~\gamma(c)=id_c,~s_0.cond(c)=\sigma_0(id_c)$.\label{it:cond-eq}
    \item
      $\forall{}a\in\mathcal{A},id_a\in{}Outs(\Delta)~s.t.~\gamma(a)=id_a,~s_0.ex(a)=\sigma_0(id_a)$.\label{it:action-eq}
    \item
      $\forall{}f\in\mathcal{F},id_f\in{}Outs(\Delta)~s.t.~\gamma(f)=id_f,~s_0.ex(f)=\sigma_0(id_f)$.\label{it:fun-eq}
    \end{enumerate}
  \end{frameb}

  \begin{itemize}
  \item Apply Lemma~\nameref{lem:init-states-eq-marking} to solve \ref{it:marking-eq}.
  \item Apply Lemma~\nameref{lem:init-states-eq-tc} to solve \ref{it:tc-eq}.
  \item Apply Lemma~\nameref{lem:init-states-eq-rorders} to solve \ref{it:reset-eq}.
  \item Apply Lemma~\nameref{lem:init-states-cond-vals} to solve \ref{it:cond-eq}.
  \item Apply Lemma~\nameref{lem:init-states-act-exec} to solve \ref{it:action-eq}.
  \item Apply Lemma~\nameref{lem:init-states-fun-exec} to solve \ref{it:fun-eq}.
  \end{itemize}
\end{proof}

\subsection{Initial states and marking}
\label{sec:init-states-marking}

\begin{lemma}[Initial States Equal Marking]
  \label{lem:init-states-eq-marking}
  \inithyps{} then
  $\forall{}p\in{}P,id_p\in{}Comps(\Delta),\sigma_p^0\in\Sigma(\Delta(id_p))~s.t.~\gamma(p)=id_p$
  and $\sigma_0(id_p)=\sigma_p^0,$
  $~s_0.M(p)=\sigma_p^0("s\_marking")$.
\end{lemma}

\begin{proof}
  Given a $p\in{}P$, an $id_p\in{}Comps(\Delta)$ and a
  $\sigma_p^0\in\Sigma(\Delta(id_p))~s.t.~\gamma(p)=id_p$ and
  $\sigma_0(id_p)=\sigma_p^0,$ let's show that\\
  \framebox{$s_0.M(p)=\sigma_p^0("s\_marking")$.}\\

  \noindent{}By definition of $id_p$, there exist
  $gm_p,ipm_p,opm_p~s.t.~\mathtt{comp}(id_p,"place",gm_p,ipm_p,opm_p)\in{}d.cs$.\\

  \noindent By property of the \hvhdl{} initialization relation, the P
  design behavior (process ``\texttt{marking}''), and\\
  $\mathtt{comp}(id_p,"place",gm_p,ipm_p,opm_p)\in{}d.cs$, then
  $\sigma_p^0("s\_marking")=\sigma_p^0("initial\_marking")$.\\

  \noindent{}Rewriting $\sigma_p^0("s\_marking")$ as $\sigma_p^0("initial\_marking")$,
  \framebox{$\sigma_p^0("initial\_marking")=s_0.M(p)$.}\\
  
  \noindent By construction,
  ${<}\mathtt{id_p.initial\_marking\Rightarrow}M_0(p){>}\in{}ipm_p$. By
  property of the \hvhdl{} initialization relation, and
  $\mathtt{comp}(id_p,"place",gm_p,ipm_p,opm_p)\in{}d.cs$,
  then $\sigma_p^0("initial\_marking")=M_0(p)$.\\

  \noindent{}By definition of $s_0$, rewriting $s_0.M(p)$ as $M_0(p)$,
  \colorbox{red!20}{$\sigma_p^0("initial\_marking")=s_0.M(p)$.}
  
\end{proof}

\subsection{Initial states and time counters}
\label{sec:init-states-tc}

\begin{lemma}[Initial States Equal Time Counters]
  \label{lem:init-states-eq-tc}
  \inithyps{} then
  $\forall{}t\in{}T_i,id_t\in{}Comps(\Delta),\sigma_t^0\in\Sigma(\Delta(id_t))~s.t.~\gamma(t)=id_t$ and $\sigma_0(id_t)=\sigma_t^0,$\\
  $upper(I_s(t))=\infty\land{}s_0.I(t)\le{}lower(I_s(t))\Rightarrow{}s_0.I(t)=\sigma_t^0("s\_tc")\land{}$\\
  $upper(I_s(t))=\infty\land{}s_0.I(t)>{}lower(I_s(t))\Rightarrow{}\sigma_t^0("s\_tc")=lower(I_s(t))\land{}$\\
  $upper(I_s(t))\neq\infty\land{}s_0.I(t)>{}upper(I_s(t))\Rightarrow{}\sigma_t^0("s\_tc")=upper(I_s(t))\land{}$\\
  $upper(I_s(t))\neq\infty\land{}s_0.I(t)\le{}upper(I_s(t))\Rightarrow{}s_0.I(t)=\sigma_t^0("s\_tc")$.
\end{lemma}

\begin{proof}
  \noindent{}Given a $t\in{}T_i$, an $id_t\in{}Comps(\Delta)$ and a
  $\sigma_t^0\in\Sigma(\Delta(id_t))~s.t.~\gamma(t)=id_t$ and
  $\sigma_0(id_t)=\sigma_t^0$, let's show that:
  \begin{enumerate}
  \item \framebox{$upper(I_s(t))=\infty\land{}s_0.I(t)\le{}lower(I_s(t))\Rightarrow{}s_0.I(t)=\sigma_t^0("s\_tc")$}
  \item \framebox{$upper(I_s(t))=\infty\land{}s_0.I(t)>{}lower(I_s(t))\Rightarrow{}\sigma_t^0("s\_tc")=lower(I_s(t))$}
  \item \framebox{$upper(I_s(t))\neq\infty\land{}s_0.I(t)>{}upper(I_s(t))\Rightarrow{}\sigma_t^0("s\_tc")=upper(I_s(t))$}
  \item \framebox{$upper(I_s(t))\neq\infty\land{}s_0.I(t)\le{}upper(I_s(t))\Rightarrow{}s_0.I(t)=\sigma_t^0("s\_tc")$}
  \end{enumerate}

  \noindent{}By definition of $id_t$, there exist $gm_t,ipm_t,opm_t$
  s.t.
  $\mathtt{comp}(id_t,"transition",gm_t,ipm_t,opm_t)\in{}d.cs$.\\

  \noindent{}Then, let's show the 4 previous subgoals.
  
  \begin{enumerate}
  \item Assume $upper(I_s(t))=\infty\land{}s_0.I(t)\le{}lower(I_s(t))$, then show \framebox{${}s_0.I(t)=\sigma_t^0("s\_tc")$.}\\
    Rewriting $s_0.I(t)$ as $0$, by definition of $s_0$, \framebox{$\sigma_t^0("s\_tc")=0$.}

    \noindent By property of the \hvhdl{} initialization relation, the
    T design behavior (process ``\texttt{time\_counter}''), and
    $\mathtt{comp}(id_t,"transition",gm_t,ipm_t,opm_t)\in{}d.cs$, then
    \colorbox{red!20}{$\sigma_t^0("s\_tc")=0$.}
  \item Assume $upper(I_s(t))=\infty\land{}s_0.I(t)>{}lower(I_s(t))$,
    then show \framebox{$\sigma_t^0("s\_tc")=lower(I_s(t))$}.  By
    definition, $lower(I_s(t))\in\mathbb{N}^{*}$ and
    $s_0.I(t)=0$. Then, \colorbox{red!20}{$lower(I_s(t)){}<0$ is a
      contradiction.}
  \item Assume
    $upper(I_s(t))\neq\infty\land{}s_0.I(t)>{}upper(I_s(t))$, then
    show \framebox{$\sigma_t^0("s\_tc")=upper(I_s(t))$}.  By definition,
    $upper(I_s(t))\in\mathbb{N}^{*}$ and $s_0.I(t)=0$. Then,
    \colorbox{red!20}{$upper(I_s(t)){}<0$ is a contradiction.}
  \item Assume
    $upper(I_s(t))\neq\infty\land{}s_0.I(t)\le{}upper(I_s(t))$, then
    show \framebox{$s_0.I(t)=\sigma_t^0("s\_tc")$}.\\
 
    Rewriting $s_0.I(t)$ as $0$, by definition of $s_0$,
    \framebox{$\sigma_t^0("s\_tc")=0$.}

    \noindent By property of the \hvhdl{} initialization relation, the
    T design behavior (process ``\texttt{time\_counter}''), and
    $\mathtt{comp}(id_t,"transition",gm_t,ipm_t,opm_t)\in{}d.cs$, then
    \colorbox{red!20}{$\sigma_t^0("s\_tc")=0$.}
  \end{enumerate}
\end{proof}

\subsection{Initial states and reset orders}
\label{sec:init-states-rorders}

\begin{lemma}[Initial States Equal Reset Orders]
  \label{lem:init-states-eq-rorders}
  \inithyps{} then
  $\forall{}t\in{}T_i,id_t\in{}Comps(\Delta),\sigma_t^0\in\Sigma(\Delta(id_t))~s.t.~\gamma(t)=id_t$
  and
  $\sigma_0(id_t)=\sigma_t^0,~s_0.reset_t(t)=\sigma_t^0("s\_reinit\_time\_counter")$.
\end{lemma}

\begin{proof}
  Given a $t\in{}T_i$, an $id_t\in{}Comps(\Delta)$ and a
  $\sigma_t^0\in\Sigma(\Delta(id_t))~s.t.~\gamma(t)=id_t$, let's show
  that\\
  \framebox{$s_0.reset_t(t)=\sigma_t^0("s\_reinit\_time\_counter")$}.\\
  
  \noindent{}Rewriting $s_0.reset_t(t)$ as $false$, by definition of
  $s_0$,
  \framebox{$\sigma_t^0("s\_reinit\_time\_counter")=false$.}\\
  
  \noindent{}By definition of $id_t$, there exist
  $gm_t,ipm_t,opm_t~s.t.$
  $~\mathtt{comp}(id_t,"transition",gm_t,ipm_t,opm_t)\in{}d.cs$.\\
  
  \noindent By property of the \hvhdl{} initialization relation, the T
  design behavior (process \texttt{reinit\_time\_counter}
  \texttt{\_evaluation}), and
  $\mathtt{comp}(id_t,"transition",gm_t,ipm_t,opm_t)\in{}d.cs$,\\
  we know
  $\sigma_t^0("s\_reinit\_time\_counter")=\prod\limits_{i=0}^{\Delta(id_t)("in\_arcs\_nb")-1}\sigma_t^0("rt")(i)$,
  where $\Delta(id_t)("in\_arcs\_nb")$ is the value of the generic
  constant $"in\_arcs\_nb"$ stored in the elaborated design
  $\Delta(id_t)$ (which, by property of the \hvhdl{} elaboration
  relation, is an elaborated version of the T design).\\

  \noindent{}Rewriting
  $\sigma_t^0("s\_reinit\_time\_counter")$ as $\prod\limits_{i=0}^{\Delta(id_t)("in\_arcs\_nb")-1}\sigma_t^0("rt")(i)$,\\
  \framebox{$\prod\limits_{i=0}^{\Delta(id_t)("in\_arcs\_nb")-1}\sigma_t^0("rt")(i)=false$.}\\
  
  \noindent{}For all $t\in{}T$ (resp. $p\in{}P$), let $input(t)$
  (resp. $input(p)$) be the set of input places of $t$ (resp. input
  transitions of $p$), and let $output(t)$ (resp. $output(p)$) be the
  set of output places of $t$ (resp. output transitions of $p$).\\

  \noindent{}Case analysis on $input(t)$ (2 CASES).

  \begin{itemize}
  \item \textbf{CASE} $input(t)=\emptyset$.

    By construction,
    ${<}\mathtt{id_t.in\_arcs\_nb\Rightarrow}1{>}\in{}gm_t$, and by
    property of the elaboration relation,\\
    $\Delta(id_t)("in\_arcs\_nb")=1$.  By construction,
    $<\mathtt{id_t.rt(0)\Rightarrow}false>\in{}ipm_t$,
    and by property of the initialization relation, $\sigma_t^0("rt")(0)=false$.\\

    \noindent{}Rewriting $\Delta(id_t)("in\_arcs\_nb")$ as $1$ and
    $\sigma_t^0("rt")(0)$ as $false$,\\
    \colorbox{red!20}{$\prod\limits_{i=0}^{\Delta("in\_arcs\_nb")-1}\sigma_t^0("rt")(i)=\sigma_t^0("rt")(0)=false$.}
    
  \item \textbf{CASE} $input(t)\neq\emptyset$.

    We know
    $\prod\limits_{i=0}^{\Delta(id_t)("in\_arcs\_nb")-1}\sigma_t^0("rt")(i)=false\equiv$
    $\exists{}i\in[0,\Delta(id_t)("in\_arcs\_nb")-1]~s.t.~\sigma_t^0("rt")(i)=false$.\\
    \framebox{$\exists{}i\in[0,\Delta(id_t)("in\_arcs\_nb")-1]~s.t.~\sigma_t^0("rt")(i)=false$.}\\
    \noindent{}Since $input(t)\neq\emptyset,~\exists{}p~s.t.~p\in{}input(t)$. Let's take such a $p\in{}input(t)$.\\
    
    \noindent{}By construction, for all $p\in{}P$, there exist $id_p~s.t.~\gamma(p)=id_p$.

    \noindent{}By definition of $id_p$, there exist
    $gm_p,ipm_p,opm_p~s.t.~\mathtt{comp}(id_p,"place",gm_p,ipm_p,opm_p)\in{}d.cs$.\\

    \noindent{}By construction, for all $p\in{}P$, $t\in{}T$
    s.t. $p\in{}input(t)$ and $t\in{}output(p)$, for all $id_p,id_t$
    s.t. $\gamma(p)=id_p$ and $\gamma(t)=id_t$, for all
    $gm_p,ipm_p,opm_p$
    s.t. $\mathtt{comp}(id_p,"place",gm_p,ipm_p,opm_p)\in{}d.cs$ and
    $gm_t,ipm_t,opm_t$
    s.t. $\mathtt{comp}(id_t,"transition",gm_t,ipm_t,opm_t)\in{}d.cs$,
    there exist $i\in[0,\vert{}input(t)\vert{}-1]$,
    $j\in[0,\vert{}output(p)\vert{}-1]$, $id_{ji}$
    s.t. ${<}\mathtt{id_p.rtt(j)\Rightarrow}id_{ji}{>}\in{}opm_p$ and
    ${<}\mathtt{id_t.rt(i)\Rightarrow}id_{ji}{>}\in{}ipm_t$. Let's take such a $i$, $j$ and $id_{ji}$.\\

    \noindent{}By construction, for all $t\in{}T$
    s.t. $input(t)\neq{}\emptyset$, $id_t,gm_t,ipm_t,opm_t$
    s.t. $\gamma(t)=id_t$ and\\
    $\mathtt{comp}(id_t,"transition",gm_t,ipm_t,opm_t)\in{}d.cs$, then
    ${<}\mathtt{id_t.in\_arcs\_nb\Rightarrow}\vert{}input(t)\vert{}{>}\in{}gm_t$.\\

    \noindent{}By property of the \hvhdl{} elaboration relation and
    ${<}\mathtt{id_t.in\_arcs\_nb\Rightarrow}\vert{}input(t)\vert{}{>}\in{}gm_t$,
    we know $\Delta(id_t)("in\_arcs\_nb)=\vert{}input(t)\vert$.\\

    \noindent{}Rewriting $\Delta(id_t)("in\_arcs\_nb)$ as
    $\vert{}input(t)\vert$, we have
    $i\in[0, \Delta(id_t)("in\_arcs\_nb)-1]$. Let's take that i to
    prove the goal.\\

    \framebox{$\sigma_t^0("rt")(i)=false$.}\\

    \noindent{}By property of the \hvhdl{} initialization relation and
    ${<}\mathtt{id_t.rt(i)\Rightarrow}id_{ji}{>}\in{}ipm_t$, we know
    $\sigma_t^0("rt")(i)=\sigma_0("id_{ji}")$.\\

    \noindent{}Rewriting $\sigma_t^0("rt")(i)$ as $\sigma_0("id_{ji}")$, \framebox{$\sigma_0("id_{ji}")=false$.}\\

    \noindent{}By property of the \hvhdl{} elaboration and initialization
    relations, and
    $\mathtt{comp}(id_p,"place",gm_p,ipm_p,opm_p)\in{}d.cs$, there
    exists a $\sigma_p^0\in{}\Sigma(\Delta(id_p))$
    s.t. $\sigma_0(id_p)=\sigma_p^0$.\\

    \noindent{}By property of the \hvhdl{} initialization relation and
    $<\mathtt{id_p.rtt(j)\Rightarrow}id_{ji}>\in{}opm_p$, we know
    $\sigma_0("id_{ji}")=\sigma_p^0("rtt")(j)$.\\

    \noindent{}Rewriting $\sigma_0("id_{ji}")$ as
    $\sigma_p^0("rtt")(j)$, \framebox{$\sigma_p^0("rtt")(j)=false$.}\\

    \noindent{}By property of the \hvhdl{} initialization relation,
    the P design behavior (process
    $\mathtt{reinit\_transitions\_ti}$-\\$\mathtt{me\_evaluation}$), and
    $\mathtt{comp}(id_p,"place",gm_p,ipm_p,opm_p)\in{}d.cs$, we know
    that\\ for all $j\in{}[0,\Delta(id_p)("out\_arcs\_nb")-1]$,
    $\sigma_p^0("rtt")(j)=false$.\\

    \noindent{}By construction, for all $p\in{}P$
    s.t. $output(p)\neq{}\emptyset$,
    $id_p\in{}Comps(\Delta),gm_p,ipm_p,opm_p$ s.t. $\gamma(p)=id_p$
    and $\mathtt{comp}(id_p,"transition",gm_p,ipm_p,opm_p)\in{}d.cs$,
    then
    ${<}\mathtt{id_p.out\_arcs\_nb\Rightarrow}\vert{}output(p)\vert{}{>}\in{}gm_p$.\\

    \noindent{}By property of the \hvhdl{} elaboration relation and
    ${<}\mathtt{id_p.out\_arcs\_nb\Rightarrow}\vert{}output(p)\vert{}{>}\in{}gm_p$,
    we know $\Delta(id_p)("out\_arcs\_nb")=\vert{}output(p)\vert$.\\

    \noindent{}Rewriting $\vert{}output(p)\vert$ as
    $\Delta(id_p)("out\_arcs\_nb)$, we have
    $j\in[0, \Delta(id_p)("out\_arcs\_nb)-1]$.
    Then, we can deduce
    \colorbox{red!20}{$\sigma_p^0("rtt")(j)=false$}.
  \end{itemize}
  
\end{proof}

\subsection{Initial states and condition values}
\label{sec:init-states-cond-vals}

\begin{lemma}[Initial States Equal Condition Values]
  \label{lem:init-states-cond-vals}
  \inithyps{} 
  then
  $\forall{}c\in\mathcal{C},id_c\in{}Ins(\Delta)~s.t.~\gamma(c)=id_c,~s_0.cond(c)=\sigma_0(id_c)$.
\end{lemma}

\begin{proof}
  Given a $c\in\mathcal{C}$ and an
  $id_c\in{}Ins(\Delta)~s.t.~\gamma(c)=id_c$, let's show that
  \fbox{$s_0.cond(c)=\sigma_0(id_c)$.}\\

  \noindent{}Rewriting $s_0.cond(c)$ as $false$, by definition of
  $s_0$, \fbox{$\sigma_0(id_c)=false$.}

  \noindent{}By construction, $id_c$ is an input port identifier of
  boolean type in the \hvhdl{} design $d$.

  \noindent{}By property, of the \hvhdl{} elaboration relation,
  $\sigma_e(id_c)=false$, where $false$ is the default value
  associated to signals of the boolean type during the elaboration
  (see definition of default value in chapter \hvhdl{} semantics).

  \noindent{}By property of the \hvhdl{} initialization relation, we
  have $\sigma_e(id_c)=\sigma_0(id_c)$ (i.e, input ports are not
  assigned during the initialization phase).

  \noindent{}Rewriting $\sigma_e(id_c)$ as $false$,
  \colorbox{red!20}{$\sigma_0(id_c)=false$.}
  
\end{proof}

\subsection{Initial states and action executions}
\label{sec:init-states-act-exec}

\begin{todobox}
  Correction: $id_f$ is assigned by the reset block of the function process
\end{todobox}

\begin{lemma}[Initial States Equal Action Executions]
  \label{lem:init-states-act-exec}
  \inithyps{} 
  then
  $\forall{}a\in\mathcal{A},id_a\in{}Outs(\Delta)~s.t.~\gamma(a)=id_a,~s_0.ex(a)=\sigma_0(id_a)$.
\end{lemma}

\begin{proof}
  Given a $a\in\mathcal{A}$ and an
  $id_a\in{}Outs(\Delta)~s.t.~\gamma(a)=id_a$, let's show that
  \fbox{$s_0.ex(a)=\sigma_0(id_a)$.}\\

  \noindent{}Rewriting $s_0.ex(a)$ as $false$, by definition of
  $s_0$, \fbox{$\sigma_0(id_a)=false$.}

  \noindent{}By construction, $id_a$ is an output port identifier of
  boolean type in the \hvhdl{} design $d$.

  \noindent{}By property, of the \hvhdl{} elaboration relation,
  $\sigma_e(id_a)=false$, where $false$ is the default value
  associated to signals of the boolean type during the elaboration
  (see definition of default value in chapter \hvhdl{} semantics).

  \noindent{}By construction, we know that the output port identifier
  $id_a$ is assigned in the generated \texttt{action} process, only at
  the falling edge phase of the simulation cycle (i.e, the assignment
  takes place in a \texttt{falling} statement block).
  
  \noindent{}By property of the \hvhdl{} initialization relation, and
  we have $\sigma_e(id_a)=\sigma_0(id_a)$ (i.e, process
  \texttt{action} is idle during the initialization phase).

  \noindent{}Rewriting $\sigma_e(id_a)$ as $false$,
  \colorbox{red!20}{$\sigma_0(id_a)=false$.}
  
\end{proof}

\subsection{Initial states and function executions}
\label{sec:init-states-fun-exec}

\begin{todobox}
  Correction: $id_f$ is assigned by the reset block of the function process
\end{todobox}

\begin{lemma}[Initial States Equal Function Executions]
  \label{lem:init-states-fun-exec}
  \inithyps{} 
  then
  $\forall{}f\in\mathcal{F},id_f\in{}Outs(\Delta)~s.t.~\gamma(f)=id_f,~s_0.ex(f)=\sigma_0(id_f)$.
\end{lemma}

\begin{proof}
  Given a $f\in\mathcal{F}$ and an
  $id_f\in{}Outs(\Delta)~s.t.~\gamma(f)=id_f$, let's show that
  \fbox{$s_0.ex(f)=\sigma_0(id_f)$.}\\

  \noindent{}Rewriting $s_0.ex(f)$ as $false$, by definition of $s_0$,
  \fbox{$\sigma_0(id_f)=false$.}

  \noindent{}By construction, $id_f$ is an output port identifier of
  boolean type in the \hvhdl{} design $d$.

  \noindent{}By property, of the \hvhdl{} elaboration relation,
  $\sigma_e(id_f)=false$, where $false$ is the default value
  associated to signals of the boolean type during the elaboration
  (see definition of default value in chapter \hvhdl{} semantics).

  \noindent{}By construction, we know that the output port identifier
  $id_f$ is assigned in the generated \texttt{function} process (i.e,
  \texttt{function} is the process identifier), only at the rising
  edge phase of the simulation cycle (i.e, the assignment takes place
  in a \texttt{rising} statement block).
  
  \noindent{}By property of the \hvhdl{} initialization relation, and
  we have $\sigma_e(id_f)=\sigma_0(id_f)$ (i.e, process
  \texttt{function} is idle during the initialization phase).

  \noindent{}Rewriting $\sigma_e(id_f)$ as $false$,
  \colorbox{red!20}{$\sigma_0(id_f)=false$.}
  
\end{proof}

%%% Local Variables:
%%% mode: latex
%%% TeX-master: "../../main"
%%% End:
