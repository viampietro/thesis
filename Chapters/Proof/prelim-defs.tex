\begin{definition}[SITPN-to-\hvhdl{} Design Binder]
  Given a $sitpn\in{}SITPN$ and a \hvhdl{} design $d\in{}design$, a
  SITPN-to-\hvhdl{} design binder $\gamma\in{}WM(sitpn,d)$ is a tuple
  ${<}PMap,TMap,\mathcal{C}_{id},\mathcal{A}_{id},\mathcal{F}_{id},CMap,AMap,FMap{>}$
  where:
  \begin{itemize}
    \fontsize{9}{10}\selectfont
  \item
    $sitpn={<}P,T,pre,test,inhib,post,M_0,{\succ},\mathcal{A},\mathcal{C},\mathcal{F},
    \mathbb{A},\mathbb{C},\mathbb{F},{I_s}{>}$
  \item $d=$ \vhdle|design| \textit{$id_{ent}$ $id_{arch}$ gens ports sigs behavior}
  \item $PMap\in{}P\rightarrow{}P_{id}$ where $P_{id}=\{id~|~\mathtt{comp}(id,"place",gm,ipm,opm)\in{}behavior\}$
  \item $TMap\in{}T\rightarrow{}T_{id}$ where $T_{id}=\{id~|~\mathtt{comp}(id,"transition",gm,ipm,opm)\in{}behavior\}$
  \item $\mathcal{C}_{id}\subseteq\{id~|~(\mathtt{in}, id, t)\in{}ports\wedge{}id\notin\{"clk","rst"\}\}$
  \item $\mathcal{A}_{id}\subseteq\{id~|~(\mathtt{out}, id, t)\in{}ports\}$
  \item $\mathcal{F}_{id}\subseteq\{id~|~(\mathtt{out}, id, t)\in{}ports\}$
  \item $CMap\in\mathcal{C}\rightarrow\mathcal{C}_{id}$
  \item $AMap\in\mathcal{A}\rightarrow\mathcal{A}_{id}$
  \item $FMap\in\mathcal{F}\rightarrow\mathcal{F}_{id}$        
  \end{itemize}
\end{definition}

\begin{definition}[Similar Environments]
  For a given $sitpn\in{}SITPN$, a \hvhdl{} design $d\in{}design$, a
  design store
  $\mathcal{D}\in{}entity\mhyphen{}id\nrightarrow{}design$, an
  elaborated version $\Delta\in{}ElDesign(d,\mathcal{D})$ of design
  $d$, and a binder $\gamma\in{}WM(sitpn,d)$, the environment
  $E_p\in{}(\mathbb{N}\times{}\{\uparrow,\downarrow\})\rightarrow{}Ins(\Delta)\rightarrow{}value$,
  that yields the value of the primary input ports of $\Delta$ at a
  given simulation cycle and a given clock event, and the environment
  $E_c$, that yields the value of conditions of $sitpn$ at a given
  execution cycle, are similar, noted
  $\gamma\vdash{}E_p\stackrel{env}{=}E_c$, iff for all
  $\tau\in{}\mathbb{N}$, $clk\in\{\uparrow,\downarrow\}$,
  $c\in\mathcal{C}$, $id_c\in{}Ins(\Delta)$ s.t.  $\gamma(c)=id_c$,
  $E_p(\tau,clk)(id_c)=E_c(\tau)(c)$.
\end{definition}

\subsection{State Similarity}
\label{sec:state-sim}

\begin{definition}[General State Similarity]
  \label{def:state-sim}
  For a given $sitpn\in{}SITPN$, a \hvhdl{} design $d\in{}design$, an
  elaborated design $\Delta\in{}ElDesign(d,\mathcal{D}_\mathcal{H})$,
  and a binder $\gamma\in{}WM(sitpn,d)$, an SITPN state
  $s\in{}S(sitpn)$ and a design state $\sigma\in\Sigma(\Delta)$ are
  similar, written $\gamma\vdash{}s\sim\sigma$ iff
  \begin{enumerate}
  \item $\forall{}p\in{}P,id_p\in{}Comps(\Delta)~s.t.~\gamma(p)=id_p,$
    $~s.M(p)=\sigma(id_p)("s\_marking")$.
  \item
    $\forall{}t\in{}T_i,id_t\in{}Comps(\Delta)~s.t.~\gamma(t)=id_t,$\\
    $\big(upper(I_s(t))=\infty\land{}s.I(t)\le{}lower(I_s(t))\Rightarrow{}s.I(t)=\sigma(id_t)("s\_time\_counter")\big)$\\
    $\land\big(upper(I_s(t))=\infty\land{}s.I(t)>{}lower(I_s(t))\Rightarrow{}\sigma(id_t)("s\_time\_counter")=lower(I_s(t))\big)$\\
    $\land\big(upper(I_s(t))\neq\infty\land{}s.I(t)>{}upper(I_s(t))\Rightarrow{}\sigma(id_t)("s\_time\_counter")=upper(I_s(t))\big)$\\
    $\land\big(upper(I_s(t))\neq\infty\land{}s.I(t)\le{}upper(I_s(t))\Rightarrow{}s.I(t)=\sigma(id_t)("s\_time\_counter")\big)$.
  \item
    $\forall{}t\in{}T_i,id_t\in{}Comps(\Delta)~s.t.~\gamma(t)=id_t,$
    $s.reset_t(t)=\sigma(id_t)("s\_reinit\_time\_counter")$.
  \item
    $\forall{}c\in\mathcal{C},id_c\in{}Ins(\Delta)~s.t.~\gamma(c)=id_c,~s.cond(c)=\sigma(id_c)$.
  \item
    $\forall{}a\in\mathcal{A},id_a\in{}Outs(\Delta)~s.t.~\gamma(a)=id_a,~s.ex(a)=\sigma(id_a)$.
  \item
    $\forall{}f\in\mathcal{F},id_f\in{}Outs(\Delta)~s.t.~\gamma(f)=id_f,~s.ex(f)=\sigma(id_f)$.
  \end{enumerate}
\end{definition}

\begin{definition}[Post Rising Edge State Similarity]
  \label{def:post-re-state-sim}
  For a given $sitpn\in{}SITPN$, a \hvhdl{} design $d\in{}design$, an
  elaborated design $\Delta\in{}ElDesign(d,\mathcal{D}_\mathcal{H})$,
  and a binder $\gamma\in{}WM(sitpn,d)$, a clock cycle count
  $\tau\in\mathbb{N}$, and an SITPN execution environment
  $E_c\in\mathbb{N}\rightarrow\mathcal{C}\rightarrow\mathbb{B}$, an
  SITPN state $s\in{}S(sitpn)$ and a design state
  $\sigma\in\Sigma(\Delta)$ are similar after a rising edge happening
  at clock cycle count $\tau$, written
  $\gamma,E_c,\tau\vdash{}s\stackrel{\uparrow}{\sim}\sigma$ iff
  \begin{enumerate}
  \item
    $\forall{}p\in{}P,id_p\in{}Comps(\Delta)~s.t.~\gamma(p)=id_p,~s.M(p)=\sigma(id_p)("s\_marking")$.
  \item
    $\forall{}t\in{}T_i,id_t\in{}Comps(\Delta)~s.t.~\gamma(t)=id_t,$\\
    $\big(upper(I_s(t))=\infty\land{}s.I(t)\le{}lower(I_s(t))\Rightarrow{}s.I(t)=\sigma(id_t)("s\_time\_counter")\big)$\\
    $\land\big(upper(I_s(t))=\infty\land{}s.I(t)>{}lower(I_s(t))\Rightarrow{}\sigma(id_t)("s\_time\_counter")=lower(I_s(t))\big)$\\
    $\land\big(upper(I_s(t))\neq\infty\land{}s.I(t)>{}upper(I_s(t))\Rightarrow{}\sigma(id_t)("s\_time\_counter")=upper(I_s(t))\big)$\\
    $\land\big(upper(I_s(t))\neq\infty\land{}s.I(t)\le{}upper(I_s(t))\Rightarrow{}s.I(t)=\sigma(id_t)("s\_time\_counter")\big)$.
  \item
    $\forall{}t\in{}T_i,id_t\in{}Comps(\Delta)~s.t.~\gamma(t)=id_t,$
    $s.reset_t(t)=\sigma(id_t)("s\_reinit\_time\_counter")$.
  \item
    $\forall{}a\in\mathcal{A},id_a\in{}Outs(\Delta)~s.t.~\gamma(a)=id_a,~s.ex(a)=\sigma(id_a)$.
  \item
    $\forall{}f\in\mathcal{F},id_f\in{}Outs(\Delta)~s.t.~\gamma(f)=id_f,~s.ex(f)=\sigma(id_f)$.
  \item $\forall{}t\in{}T,id_t\in{}Comps(\Delta)~s.t.~\gamma(t)=id_t,$
    $t\in{}Sens(s.M)\Leftrightarrow\sigma(id_t)("s\_enabled")=\mathtt{true}$.
  \item $\forall{}t\in{}T,id_t\in{}Comps(\Delta)~s.t.~\gamma(t)=id_t,$
    $t\notin{}Sens(s.M)\Leftrightarrow\sigma(id_t)("s\_enabled")=\mathtt{false}$.
  \item
    $\forall{}t\in{}T,id_t\in{}Comps(\Delta)~s.t.~\gamma(t)=id_t,$\\
    $\sigma(id_t)("s\_condition\_combination")=
    \prod\limits_{c\in{}conds(t)}
    \begin{cases}
      E_c(\tau,c) & if~\mathbb{C}(t,c)=1 \\
      \mathtt{not}(E_c(\tau,c)) & if~\mathbb{C}(t,c)=-1 \\
    \end{cases}$\\
    where
    $conds(t)=\{c\in\mathcal{C}~\vert~\mathbb{C}(t,c)=1\lor\mathbb{C}(t,c)=-1\}$.
  \end{enumerate}
\end{definition}

\begin{definition}[Post Falling Edge State Similarity]
  \label{def:post-fe-state-sim}
  For a given $sitpn\in{}SITPN$, a \hvhdl{} design $d\in{}design$, an
  elaborated design $\Delta\in{}ElDesign(d,\mathcal{D}_\mathcal{H})$,
  and a binder $\gamma\in{}WM(sitpn,d)$, an SITPN state
  $s\in{}S(sitpn)$ and a design state $\sigma\in\Sigma(\Delta)$ are
  similar after a falling edge, written
  $\gamma\vdash{}s\stackrel{\downarrow}{\sim}\sigma$ iff
  $\gamma\vdash{}s\sim\sigma$ (Def.~\ref{def:state-sim}, general state
  similarity) and
  \begin{enumerate}
  \item $\forall{}p\in{}P,id_p\in{}Comps(\Delta)~s.t.~\gamma(p)=id_p,$
    $\sum\limits_{t\in{}Fired(s)}pre(p,t)=\sigma(id_p)("s\_output\_token\_sum")$.
  \item $\forall{}p\in{}P,id_p\in{}Comps(\Delta)~s.t.~\gamma(p)=id_p,$
    $\sum\limits_{t\in{}Fired(s)}post(t,p)=\sigma(id_p)("s\_input\_token\_sum")$.
  \item $\forall{}t\in{}T,id_t\in{}Comps(\Delta)~s.t.~\gamma(t)=id_t,$
    $t\in{}Firable(s)\Leftrightarrow\sigma(id_t)("s\_firable")=\mathtt{true}$.
  \item $\forall{}t\in{}T,id_t\in{}Comps(\Delta)~s.t.~\gamma(t)=id_t,$
    $t\notin{}Firable(s)\Leftrightarrow\sigma(id_t)("s\_firable")=\mathtt{false}$.
  \item $\forall{}t\in{}T,id_t\in{}Comps(\Delta)~s.t.~\gamma(t)=id_t,$
    $t\in{}Fired(s)\Leftrightarrow\sigma(id_t)("fired")=\mathtt{true}$.
  \item $\forall{}t\in{}T,id_t\in{}Comps(\Delta)~s.t.~\gamma(t)=id_t,$
    $t\notin{}Fired(s)\Leftrightarrow\sigma(id_t)("fired")=\mathtt{false}$.
  \end{enumerate}
\end{definition}

\begin{definition}[Execution Trace Similarity]
  \label{def:exec-trace-sim}
  For a given $sitpn\in{}SITPN$, a \hvhdl{} design $d\in{}design$, an
  elaborated design $\Delta\in{}ElDesign(d,\mathcal{D}_\mathcal{H})$,
  and a binder $\gamma\in{}WM(sitpn,d)$, the execution trace
  $\theta_s\in{}\mathtt{list}(S(sitpn))$ and the simulation trace
  $\theta_\sigma\in\mathtt{list}(\Sigma(\Delta))$ are similar, written
  $\gamma\vdash{}\theta_s\sim\theta_\sigma$, according to the
  following rules:

  \begin{tabular}{@{}l}
    {\fontsize{9}{11}\selectfont\textsc{SimTraceNil}} \\
    
    {\begin{prooftree}[template={\fontsize{9}{11}\selectfont\inserttext}]        
        \infer0{$\gamma\vdash{}[~]\sim{}[~]$}
      \end{prooftree}} 
  \end{tabular}
  \begin{tabular}{@{}l}
    {\fontsize{9}{11}\selectfont\textsc{SimTraceCons}} \\
    
    {\begin{prooftree}[template={\fontsize{9}{11}\selectfont\inserttext}]

        \hypo{$\gamma\vdash{}s\sim\sigma$}
        \hypo{$\gamma\vdash{}\theta_s\sim{}\theta_\sigma$}
        \infer2{$\gamma\vdash{}(s :: \theta_s)\sim{}(\sigma :: \theta_\sigma)$}
      \end{prooftree}} 
  \end{tabular}  
\end{definition}

\subsection{Equality between big operator expressions}
\label{sec:eq-bi-op-expr}

Many times in the proceeding of the following proof, the equality
between two sum or product expressions must be estbalished; for
instance:

$\sum\limits_{a\in{}A}f(a)=\sum\limits_{b\in{}B}g(b)$ where $A$ and
$B$ are finite sets, $f\in\mathbb{A}\rightarrow\mathbb{N}$ and
$g\in{}B\rightarrow\mathbb{N}$

To prove such an equality, Theorem~\ref{thm:big-op-eq} is used,
considering that the sum operator used in the above equation is a big
operator over the triplet ${<}\mathbb{N},0,+{>}$. A big operator is
defined as follows:

\begin{definition}[Big Operator]
  Given a triplet ${<}A,*,e{>}$ such that $A$ is a set,
  $*\in{}A\rightarrow{}A\rightarrow{}A$ is a commutative and
  associative operator over $A$, and $e\in{}A$ is a neutral element of
  $*$, then for all finite set $B$, and application
  $f\in{}B\rightarrow{}A$, a big operator $\Omega$ is recursively
  defined as follows: $\mathop{\Omega}\limits_{b\in{}B}f(b)=
  \begin{cases}
    e & if~B=\emptyset \\
    f(b)*\mathop{\Omega}\limits_{b'\in{}B\setminus\{b\}}f(b') & otherwise
  \end{cases}
  $
\end{definition}

Then, we can prove the following theorem concerning the equality
between two big operator expressions.

\begin{thm}[Big Operator Equality]
  \label{thm:big-op-eq}
  For all a triplet ${<}A,*,e{>}$ such that $A$ is a set,
  $*\in{}A\rightarrow{}A\rightarrow{}A$ is a commutative and
  associative operator over $A$, and $e\in{}A$ is a neutral element of
  $*$, and for all finite sets $B$ and $C$, and applications
  $f\in{}B\rightarrow{}A$ and $g\in{}C\rightarrow{}A$, assume that:
  \begin{itemize}
  \item there exists an injection $\iota\in{}B\rightarrow{}C$
    s.t. $\forall{}b\in{}B,~f(b)=g(\iota(b))$
  \item $\vert{}B\vert=\vert{}C\vert$
  \end{itemize}
  then
  $\mathop{\Omega}\limits_{b\in{}B}f(b)=\mathop{\Omega}\limits_{c\in{}C}g(c)$.
\end{thm}

\begin{proof}
  Let us reason by induction over $\mathop{\Omega}\limits_{b\in{}B}f(b)$:

  \begin{itemize}
  \item \textbf{BASE CASE} $B=\emptyset$:\\
    Then $\vert{}C\vert=\vert{}B\vert=0$, and $C=\emptyset$.
    By definition of $\Omega$:
    \begin{eqnarray}
      \mathop{\Omega}\limits_{b\in{}B}f(b)=e\label{eq:bemp} \\
      \mathop{\Omega}\limits_{c\in{}C}g(c)=e\label{eq:cemp}
    \end{eqnarray}
    Rewriting the goal with \eqref{eq:bemp} and \eqref{eq:cemp},
    \qedbox{tautology}.
  \item \textbf{INDUCTION CASE} $B\neq\emptyset$:
    \begin{ih}
      For all finite set $C'$ verifying:
      \begin{itemize}
      \item $\exists{}$ an injection
        $\iota'\in{}B\setminus\{b\}\rightarrow{}C'~s.t.~$
        $\forall{}b'\in{}B\setminus\{b\},~f(b')=g(\iota(b'))$
      \item $\vert{}B\setminus\{b\}\vert=\vert{}C'\vert$
      \end{itemize}
      then $f(b)*\mathop{\Omega}\limits_{b'\in{}B\setminus\{b\}}f(b')=f(b)*\mathop{\Omega}\limits_{c'\in{}C'}g(c)$
    \end{ih}
    
    The goal is \fbox{$f(b)*\mathop{\Omega}\limits_{b'\in{}B\setminus\{b\}}f(b')=\mathop{\Omega}\limits_{c\in{}C}g(c)$}

    \noindent{}Let us take $\iota\in{}B\rightarrow{}C$
    s.t. $\forall{}b\in{}B,~f(b)=g(\iota(b))$, then:
    \begin{equation}
      f(b)=g(\iota(b))\label{eq:b-iota-b}
    \end{equation}
    \noindent{}Also, by definition of $\Omega$:
    \begin{equation}
      \mathop{\Omega}\limits_{c\in{}C}g(c)=g(\iota(b))*\mathop{\Omega}\limits_{c'\in{}C\setminus\{\iota(b)\}}\label{eq:omega-C}
    \end{equation}
    \noindent{}Rewriting the goal with \eqref{eq:omega-C} and
    \eqref{eq:b-iota-b},\\
    \fbox{$f(b)*\mathop{\Omega}\limits_{b'\in{}B\setminus\{b\}}f(b')=f(b)*\mathop{\Omega}\limits_{c'\in{}C\setminus\{\iota(b)\}}g(c')$}
    
    \noindent{}Let us apply the induction hypothesis with $C'=C\setminus\{\iota(b)\}$; then there are two points to prove:
    \begin{enumerate}
    \item \fbox{$\vert{}B\setminus\{b\}\vert=\vert{}C\setminus\{\iota(b)\}\vert$.} Trivial as $\vert{}B\vert=\vert{}C\vert$.
    \item \fbox{\parbox{\linewidth}{$\exists{}$ an injection
          $\iota'\in{}B\setminus\{b\}\rightarrow{}C\setminus\{\iota(b)\}~s.t.~$
          $\forall{}b'\in{}B\setminus\{b\},f(b')=g(\iota'(b'))$}}
    \end{enumerate}
    Let us define a
    $\iota'\in{}B\setminus\{b\}\rightarrow{}C\setminus\{\iota(b)\}$ as
    follows:
    $\forall{}b'\in{}B\setminus\{b\},~\iota'(b)=\iota(b)$. Let us show
    that this definition is correct by proving that\\
    \fbox{$\forall{}b'\in{}B\setminus\{b\},~\iota(b')\in{}C\setminus\{\iota(b)\}$.}

    \noindent{}Given a $b'\in{}B\setminus\{b\}$, let us show
    \fbox{$\iota(b')\in{}C\setminus\{\iota(b)\}$.}

    \noindent{}By definition of $\iota$, $\iota(b')\in{}C$; then,
    there are 2 cases:
    \begin{itemize}
    \item \textbf{CASE} $\iota(b')=\iota(b)$, then by definition of
      $\iota$ as an injective function: $b'=b$. Then,
      \qedbox{$b\in{}B\setminus\{b\}$ is a contradiction.}
    \item \textbf{CASE} \qedbox{$\iota(b')\in{}C\setminus\{\iota(b)\}$.}
    \end{itemize}

    \noindent{}Now let us get back to the previous goal. Using
    $\iota'$ to prove it, there are 2 points to prove:
    \begin{itemize}
    \item \fbox{$\iota'$ is injective.} Trivial, by definition of $\iota'$.
    \item
      \fbox{$\forall{}b'\in{}B\setminus\{b\},~f(b')=g(\iota'(b'))$.}
      Trivial, by definition of $\iota'$.
    \end{itemize}

  \end{itemize}
\end{proof}

\begin{todobox}
  Add a remark on how to convert a sequence of indexes into a finite
  set, and what is the cardinality of the finite set:\\
  $\mathop{\Omega}\limits_{i=n}^m{}f(i)$ then $\vert[n,m]\vert=(m-n)+1$ when $m\ge{}n$
\end{todobox}

%%% Local Variables:
%%% mode: latex
%%% TeX-master: "../../main"
%%% End:
