Before presenting our behavior preservation theorem, we must clarify
the meaning of semantic preservation between an SITPN and a \hvhdl{}
design. To do so, we must define:

\begin{enumerate}
\item What does semantic similarity mean between an SITPN state and a
  \hvhdl{} state?
\item When, in the course of the execution of an SITPN and a \hvhdl{}
  design, does this semantic similarity must hold?
\end{enumerate}

We must relate the elements that constitute the execution state of an
SITPN to the elements that constitute the execution state of a
\hvhdl{} design. An SITPN state is an abstract structure relating the
places, transitions, actions, functions and conditions of a given
SITPN to the values of certain domains (see Section~\todo{Add
  ref.}). A \hvhdl{} design state is composed a signal store mapping
signals to values, and of a component store mapping component
instances to their own internal states (which are themselves design
states). Thanks to the binder function $\gamma$ generated alongside
the transformation from an SITPN to a \hvhdl{} design, we are able to
relate the elements of the SITPN structure to the component instance
states and signal values of the \hvhdl{} design state. Thus, the state
similarity relation, depending on a $\gamma$ binder and expressing a
semantic match between an SITPN state and a \hvhdl{} design, is
defined as follows:

\begin{definition}[General state similarity]
  \label{def:state-sim}
  For a given $sitpn\in{}SITPN$, a \hvhdl{} design $d\in{}design$, an
  elaborated design $\Delta\in{}ElDesign(d,\mathcal{D}_\mathcal{H})$,
  and a binder $\gamma\in{}WM(sitpn,d)$, an SITPN state
  $s\in{}S(sitpn)$ and a design state $\sigma\in\Sigma(\Delta)$ are
  similar, written $\gamma\vdash{}s\sim\sigma$ iff
  \begin{enumerate}
  \item\label{item:sim-mark} $\forall{}p\in{}P,id_p\in{}Comps(\Delta)~s.t.~\gamma(p)=id_p,$
    $~s.M(p)=\sigma(id_p)("s\_marking")$.
  \item\label{item:sim-tc}
    $\forall{}t\in{}T_i,id_t\in{}Comps(\Delta)~s.t.~\gamma(t)=id_t,$\\
    $\big(upper(I_s(t))=\infty\land{}s.I(t)\le{}lower(I_s(t))\Rightarrow{}s.I(t)=\sigma(id_t)("s\_time\_counter")\big)$\\
    $\land\big(upper(I_s(t))=\infty\land{}s.I(t)>{}lower(I_s(t))\Rightarrow{}\sigma(id_t)("s\_time\_counter")=lower(I_s(t))\big)$\\
    $\land\big(upper(I_s(t))\neq\infty\land{}s.I(t)>{}upper(I_s(t))\Rightarrow{}\sigma(id_t)("s\_time\_counter")=upper(I_s(t))\big)$\\
    $\land\big(upper(I_s(t))\neq\infty\land{}s.I(t)\le{}upper(I_s(t))\Rightarrow{}s.I(t)=\sigma(id_t)("s\_time\_counter")\big)$.
  \item\label{item:sim-reset}
    $\forall{}t\in{}T_i,id_t\in{}Comps(\Delta)~s.t.~\gamma(t)=id_t,$
    $s.reset_t(t)=\sigma(id_t)("s\_reinit\_time\_counter")$.
  \item\label{item:sim-cond}
    $\forall{}c\in\mathcal{C},id_c\in{}Ins(\Delta)~s.t.~\gamma(c)=id_c,~s.cond(c)=\sigma(id_c)$.
  \item\label{item:sim-act}
    $\forall{}a\in\mathcal{A},id_a\in{}Outs(\Delta)~s.t.~\gamma(a)=id_a,~s.ex(a)=\sigma(id_a)$.
  \item\label{item:sim-fun}
    $\forall{}f\in\mathcal{F},id_f\in{}Outs(\Delta)~s.t.~\gamma(f)=id_f,~s.ex(f)=\sigma(id_f)$.
  \end{enumerate}
\end{definition}

In Item~\ref{item:sim-mark}, based on the $\gamma$ binder, we relate
the marking value of a place $p$ at state $s$ to the value of the
$\mathtt{s\_marking}$ signal inside the internal state of the place
component instance (PCI) $id_p$. The expression $\sigma(id_p)$ returns
the internal state of PCI $id_p$ by looking up the component store of
state $\sigma$. Items~\ref{item:sim-tc} and \ref{item:sim-reset}
similarly relate the value of time counters (resp. reset orders) of
transitions to the value of the signals $\mathtt{s\_time\_counter}$
(resp. $\mathtt{s\_reinit\_time\_counter}$) in the internal state of
the corresponding transition component instances (TCIs). In item
\ref{item:sim-cond} (resp. \ref{item:sim-act} and \ref{item:sim-fun}),
the boolean value of conditions (resp. actions and functions) are
compared to the value of input (resp. output) ports of the \hvhdl{}
design, also based on the $\gamma$ binder.

As one can observe in Item~\ref{item:sim-tc}, the relation between the
value of a time counter and the value of the
$\mathtt{s\_time\_counter}$ signal is a particular. It is due to the
definition domain of time intervals. In the definition of the SITPN
structure, a time interval $i$ is defined as follows: $i=[a,b]$ where
$a\in\mathbb{N}^{*}$ and $b\in\mathbb{N}^{*}\sqcup\{\infty\}$. In the
SITPN semantics, depending on certain conditions, a time counter
possibly increments its value until it reaches the upper bound of the
associated time interval. Therefore, a time counter associated to a
time interval with an infinite upper bound will possibly increment its
value indefinitely. While acceptable in the theoretical world, this is
not acceptable is the world of hardware circuits where all dimensions
and values are finite. On the \hvhdl{} side, the signal
$\mathtt{s\_time\_counter}$, which value represents the value of a
time counter, will stop its incrementation to the lower bound of the
time interval in the case where the upper bound is infinite. As long
as the value of the time counter is less than or equal to the lower
bound of the time interval, we look for a perfect equality between the
value of the time counter and the value of the
$\mathtt{s\_time\_counter}$ signal. When the time counter reaches the
lower bound, the values possibly diverge (i.e, the time counter value
continues to be incremented while the value of the
$\mathtt{s\_time\_counter}$ signal stalls). In that case, we are only
interested in knowing that the value of the
$\mathtt{s\_time\_counter}$ signal is equal to the value of the lower
bound of the time interval. The two last points of
Item~\ref{item:sim-tc} are necessary to cover the case where a time
counter has overreached the upper bound of its time interval. In that
case, the time counter becomes \textit{locked}. The
$\mathtt{s\_time\_counter}$ signal can not overreached the upper bound
of the time interval without causing an overflow. Thus, the value of
the $\mathtt{s\_time\_counter}$ signal diverges from the value of its
corresponding time counter when the time counter overreaches the upper
bound of its time interval.  While the time counter is less than or
equal to the upper bound of its time interval, we look for a perfect
equality between the value of the time counter and the value of the
$s\_time\_counter$ signal.  When the time counter overreaches the
upper bound, the value of the time counter stalls to upper bound plus
one, and the value of $s\_time\_counter$ stalls to upper bound. In
that case, we are only interested in knowing that the value of the
$\mathtt{s\_time\_counter}$ signal is equal to the value of the upper
bound of the time interval.

The second question that we asked above was: when does the state
similarity relation must hold in the course of the execution? The
source and target representations are both synchronously
executed. Thus, we find it natural to check that the state similarity
relation holds at the end of a clock cycle. However, due to
modifications resulting after a bug detection (see
Section~\ref{sec:detailled-proof}), the state similarity relation of
Definition~\ref{sec:state-sim-relation} does not hold at the end of a
clock cycle. The equality between the value of reset orders and the
value of the $\mathtt{s\_reinit\_time\_counter}$ signals
(Item~\ref{item:sim-reset}) is not verified. However, this semantic
divergence is without effect. New reset orders are computed at the
beginning of a clock cycle such that the relation of
Item~\ref{item:sim-reset} holds in the middle of the clock cycle (i.e,
just before the falling edge of the clock). This is the only moment
during the clock cycle where the $\mathtt{s\_reinit\_time\_counter}$
signal is actually involved in the computation of other signals
value. Thus, it is sufficient that Item~\ref{item:sim-reset} holds
only in the middle of the clock cycle. However, we must now define two
state similarity relation; one that checks the semantic similarity
after the rising edge of the clock signal (i.e, in the middle of the
clock cycle), and one that checks the semantic similarity after the
falling edge of the clock signal (i.e, at the end of the clock
cycle). The state similarity relation after a rising edge is defined
as follows:

\begin{definition}[Post rising edge state similarity]
  \label{def:post-re-state-sim}
  For a given $sitpn\in{}SITPN$, a \hvhdl{} design $d\in{}design$, an
  elaborated design $\Delta\in{}ElDesign(d,\mathcal{D}_\mathcal{H})$,
  and a binder $\gamma\in{}WM(sitpn,d)$, an SITPN state
  $s\in{}S(sitpn)$ and a design state $\sigma\in\Sigma(\Delta)$ are
  similar after a rising edge happening, written
  $\gamma\vdash{}s\stackrel{\uparrow}{\sim}\sigma$ iff
  \begin{enumerate}
  \item
    $\forall{}p\in{}P,id_p\in{}Comps(\Delta)~s.t.~\gamma(p)=id_p,~s.M(p)=\sigma(id_p)("s\_marking")$.
  \item
    $\forall{}t\in{}T_i,id_t\in{}Comps(\Delta)~s.t.~\gamma(t)=id_t,$\\
    $\big(upper(I_s(t))=\infty\land{}s.I(t)\le{}lower(I_s(t))\Rightarrow{}s.I(t)=\sigma(id_t)("s\_time\_counter")\big)$\\
    $\land\big(upper(I_s(t))=\infty\land{}s.I(t)>{}lower(I_s(t))\Rightarrow{}\sigma(id_t)("s\_time\_counter")=lower(I_s(t))\big)$\\
    $\land\big(upper(I_s(t))\neq\infty\land{}s.I(t)>{}upper(I_s(t))\Rightarrow{}\sigma(id_t)("s\_time\_counter")=upper(I_s(t))\big)$\\
    $\land\big(upper(I_s(t))\neq\infty\land{}s.I(t)\le{}upper(I_s(t))\Rightarrow{}s.I(t)=\sigma(id_t)("s\_time\_counter")\big)$.
  \item
    $\forall{}t\in{}T_i,id_t\in{}Comps(\Delta)~s.t.~\gamma(t)=id_t,$
    $s.reset_t(t)=\sigma(id_t)("s\_reinit\_time\_counter")$.
  \item
    $\forall{}a\in\mathcal{A},id_a\in{}Outs(\Delta)~s.t.~\gamma(a)=id_a,~s.ex(a)=\sigma(id_a)$.
  \item
    $\forall{}f\in\mathcal{F},id_f\in{}Outs(\Delta)~s.t.~\gamma(f)=id_f,~s.ex(f)=\sigma(id_f)$.
  \end{enumerate}
\end{definition}

Definition~\ref{def:post-re-state-sim} is similar to
Definition~\ref{def:state-sim} in all points, except for the value of
conditions. A condition of an SITPN is implemented by an input port in
the resulting \hvhdl{} top-level design. In the \hvhdl{} semantics,
the value of primary input ports (i.e, the input ports of the
top-level design) are updated at each clock edge. In the SITPN
semantics, the value of conditions are updated only at the falling
edge of the clock. Consider that a given SITPN is executed at clock
cycle $\tau$; after the rising edge of the clock, the value of
conditions are equal to their value at clock cycle $\tau-1$, whereas
the value primary input ports have been updated to fresh values. Thus,
we will have to wait for the next falling edge to reach the equality
between condition values and input port values. Therefore, there is a
semantic divergence between the value of conditions and the value of
input ports in the middle of the clock cycle, i.e. just before the
next falling edge of the clock signal. However, similarly to the case
of reset orders and $\mathtt{s\_reinit\_time\_counter}$ signals,
conditions and their corresponding input ports are only involved in
computations at the falling edge of the clock cycle.  Thus, it is
sufficient that Item~\ref{item:sim-cond} holds only right after the
falling of the clock signal.

The state similarity relation draws out a correspondence between the
values hold by an SITPN state and the values of the signals declared
in a \hvhdl{} design state. However, to complete the proof of semantic
preservation, we sometimes have to relate the value of signals to the
value of expressions or predicates involved in the SITPN
semantics. For instance, consider a given SITPN state $s$ and a given
\hvhdl{} design state $\sigma$, and consider a transition $t$ and its
corresponding TCI $id_t$. It is useful to show that, after a rising
edge, the value of signal $s\_enabled$ at state $\sigma(id_t)$, where
$\sigma(id_t)$ denotes the internal state of component instance $id_t$
at state $\sigma$, is equal to the predicate $t\in{}Sens(s.M)$ stating
that the transition $t$ is sensitized (or \textit{enabled}) by the
marking at state $s$ (i.e, $s.M$). Thus, for the convenience of the
proof, we enrich our definitions of the state similarity relations
with formulas relating \hvhdl{} signals to SITPN semantics predicates
and expressions. Consequently, the \textit{full} post rising edge
state similarity relation is defined as follows:

\begin{definition}[Full post rising edge state similarity]
  \label{def:full-post-re-state-sim}
  For a given $sitpn\in{}SITPN$, a \hvhdl{} design $d\in{}design$, an
  elaborated design $\Delta\in{}ElDesign(d,\mathcal{D}_\mathcal{H})$,
  and a binder $\gamma\in{}WM(sitpn,d)$, a clock cycle count
  $\tau\in\mathbb{N}$, and an SITPN execution environment
  $E_c\in\mathbb{N}\rightarrow\mathcal{C}\rightarrow\mathbb{B}$, an
  SITPN state $s\in{}S(sitpn)$ and a design state
  $\sigma\in\Sigma(\Delta)$ are fully similar after a rising edge
  happening at clock cycle count $\tau$, written
  $\gamma,E_c,\tau\vdash{}s\stackrel{\uparrow}{\approx}\sigma$ iff
  $\gamma\vdash{}s\stackrel{\uparrow}{\sim}\sigma$
  (Definition~\ref{def:post-re-state-sim}) and
  \begin{enumerate}
  \item $\forall{}t\in{}T,id_t\in{}Comps(\Delta)~s.t.~\gamma(t)=id_t,$
    $t\in{}Sens(s.M)\Leftrightarrow\sigma(id_t)("s\_enabled")=\mathtt{true}$.
  \item $\forall{}t\in{}T,id_t\in{}Comps(\Delta)~s.t.~\gamma(t)=id_t,$
    $t\notin{}Sens(s.M)\Leftrightarrow\sigma(id_t)("s\_enabled")=\mathtt{false}$.
  \item
    $\forall{}t\in{}T,id_t\in{}Comps(\Delta)~s.t.~\gamma(t)=id_t,$\\
    $\sigma(id_t)("s\_condition\_combination")=
    \prod\limits_{c\in{}conds(t)}
    \begin{cases}
      E_c(\tau,c) & if~\mathbb{C}(t,c)=1 \\
      \mathtt{not}(E_c(\tau,c)) & if~\mathbb{C}(t,c)=-1 \\
    \end{cases}$\\
    where
    $conds(t)=\{c\in\mathcal{C}~\vert~\mathbb{C}(t,c)=1\lor\mathbb{C}(t,c)=-1\}$.
  \end{enumerate}
\end{definition}

Definition~\ref{def:full-post-re-state-sim} extends
Definition~\ref{def:post-re-state-sim} with the correspondence of the
sensitization of transitions and the value of signal $s\_enabled$, and
the computation of the boolean product of condition values and the
value of signal $s\_condition\_combination$.

Now, let us define the state similarity relation describing how things
must be compared between an SITPN state and a \hvhdl{} design state
after the falling edge of a clock signal:

\begin{definition}[Post falling edge state similarity]
  \label{def:post-fe-state-sim}
  For a given $sitpn\in{}SITPN$, a \hvhdl{} design $d\in{}design$, an
  elaborated design $\Delta\in{}ElDesign(d,\mathcal{D}_\mathcal{H})$,
  and a binder $\gamma\in{}WM(sitpn,d)$, an SITPN state
  $s\in{}S(sitpn)$ and a design state $\sigma\in\Sigma(\Delta)$ are
  similar after a falling edge, written
  $\gamma\vdash{}s\stackrel{\downarrow}{\sim}\sigma$ iff
  \begin{enumerate}
  \item $\forall{}p\in{}P,id_p\in{}Comps(\Delta)~s.t.~\gamma(p)=id_p,$
    $~s.M(p)=\sigma(id_p)("s\_marking")$.
  \item
    $\forall{}t\in{}T_i,id_t\in{}Comps(\Delta)~s.t.~\gamma(t)=id_t,$\\
    $\big(upper(I_s(t))=\infty\land{}s.I(t)\le{}lower(I_s(t))\Rightarrow{}s.I(t)=\sigma(id_t)("s\_time\_counter")\big)$\\
    $\land\big(upper(I_s(t))=\infty\land{}s.I(t)>{}lower(I_s(t))\Rightarrow{}\sigma(id_t)("s\_time\_counter")=lower(I_s(t))\big)$\\
    $\land\big(upper(I_s(t))\neq\infty\land{}s.I(t)>{}upper(I_s(t))\Rightarrow{}\sigma(id_t)("s\_time\_counter")=upper(I_s(t))\big)$\\
    $\land\big(upper(I_s(t))\neq\infty\land{}s.I(t)\le{}upper(I_s(t))\Rightarrow{}s.I(t)=\sigma(id_t)("s\_time\_counter")\big)$.
  \item
    $\forall{}c\in\mathcal{C},id_c\in{}Ins(\Delta)~s.t.~\gamma(c)=id_c,~s.cond(c)=\sigma(id_c)$.
  \item
    $\forall{}a\in\mathcal{A},id_a\in{}Outs(\Delta)~s.t.~\gamma(a)=id_a,~s.ex(a)=\sigma(id_a)$.
  \item
    $\forall{}f\in\mathcal{F},id_f\in{}Outs(\Delta)~s.t.~\gamma(f)=id_f,~s.ex(f)=\sigma(id_f)$.
  \end{enumerate}
\end{definition}

As explained above, Definition~\ref{def:post-fe-state-sim} is similar
to Definition~\ref{def:state-sim} except for the equality between
reset orders and the value of the $s\_reinit\_time\_counter$
signals. The extended version of the post falling edge state
similarity relation is defined as follows:

\begin{definition}[Full post falling edge state similarity]
  \label{def:full-post-fe-state-sim}
  For a given $sitpn\in{}SITPN$, a \hvhdl{} design $d\in{}design$, an
  elaborated design $\Delta\in{}ElDesign(d,\mathcal{D}_\mathcal{H})$,
  and a binder $\gamma\in{}WM(sitpn,d)$, an SITPN state
  $s\in{}S(sitpn)$ and a design state $\sigma\in\Sigma(\Delta)$ are
  fully similar after a falling edge, written
  $\gamma\vdash{}s\stackrel{\downarrow}{\approx}\sigma$ iff
  $\gamma\vdash{}s\stackrel{\downarrow}{\sim}\sigma$
  (Definition~\ref{def:post-fe-state-sim}) and
  \begin{enumerate}
  \item $\forall{}t\in{}T,id_t\in{}Comps(\Delta)~s.t.~\gamma(t)=id_t,$
    $t\in{}Firable(s)\Leftrightarrow\sigma(id_t)("s\_firable")=\mathtt{true}$.
  \item $\forall{}t\in{}T,id_t\in{}Comps(\Delta)~s.t.~\gamma(t)=id_t,$
    $t\notin{}Firable(s)\Leftrightarrow\sigma(id_t)("s\_firable")=\mathtt{false}$.
  \item $\forall{}t\in{}T,id_t\in{}Comps(\Delta)~s.t.~\gamma(t)=id_t,$
    $t\in{}Fired(s)\Leftrightarrow\sigma(id_t)("fired")=\mathtt{true}$.
  \item $\forall{}t\in{}T,id_t\in{}Comps(\Delta)~s.t.~\gamma(t)=id_t,$
    $t\notin{}Fired(s)\Leftrightarrow\sigma(id_t)("fired")=\mathtt{false}$.
  \item $\forall{}p\in{}P,id_p\in{}Comps(\Delta)~s.t.~\gamma(p)=id_p,$
    $\sum\limits_{t\in{}Fired(s)}pre(p,t)=\sigma(id_p)("s\_output\_token\_sum")$.
  \item $\forall{}p\in{}P,id_p\in{}Comps(\Delta)~s.t.~\gamma(p)=id_p,$
    $\sum\limits_{t\in{}Fired(s)}post(t,p)=\sigma(id_p)("s\_input\_token\_sum")$.
  \end{enumerate}
\end{definition}

\noindent{}Definition~\ref{def:full-post-fe-state-sim} extends
Definition~\ref{def:post-fe-state-sim} by drawing out a correspondence
between:
\begin{itemize}
\item the firability of transitions and the value of the signal
  $s\_firable$
\item the firing status of transitions (i.e, transitions are fired or
  not) and the value of the output port $fired$
\item the sum of tokens consumed by the firing process and the value
  of the signal $s\_output\_token\_sum$
\item the sum of tokens produced by the firing process and the value
  of the signal $s\_input\_token\_sum$
\end{itemize}

%%% Local Variables:
%%% mode: latex
%%% TeX-master: "../../main"
%%% End:
