Before stating the behavior preservation theorem, we must clarify the
meaning of semantic preservation between an SITPN and a \hvhdl{}
design. To do so, we must define:

\begin{enumerate}
\item what does semantical matching means between an SITPN state and an \hvhdl{} state?
\item when, in the course of the execution of an SITPN and an \hvhdl{}
  design, does this semantical matching must hold?
\end{enumerate}

We must relate the elements that constitute the execution state of an
SITPN to the elements that constitute the execution state of an
\hvhdl{} design. An SITPN state is an abstract structure relating the
places, transitions, actions, functions and conditions of a given
SITPN to the values of certain domains (see Section~\todo{Add
  ref.}). A \hvhdl{} design state is composed a signal store mapping
signals to values, and of a component store mapping component
instances to their own internal states. Thanks to the binder function
$\gamma$ generated alongside the transformation from an SITPN to a
\hvhdl{} design, we are able to relate the elements of the SITPN
structure to the component instances and signals on the \hvhdl{}
side. Thus, the state similarity relation expressing a semantical
match between an SITPN state and an \hvhdl{} design is defined as
follows:

\begin{definition}[General state similarity]
  \label{def:state-sim}
  For a given $sitpn\in{}SITPN$, a \hvhdl{} design $d\in{}design$, an
  elaborated design $\Delta\in{}ElDesign(d,\mathcal{D}_\mathcal{H})$,
  and a binder $\gamma\in{}WM(sitpn,d)$, an SITPN state
  $s\in{}S(sitpn)$ and a design state $\sigma\in\Sigma(\Delta)$ are
  similar, written $\gamma\vdash{}s\sim\sigma$ iff
  \begin{enumerate}
  \item\label{item:sim-mark} $\forall{}p\in{}P,id_p\in{}Comps(\Delta)~s.t.~\gamma(p)=id_p,$
    $~s.M(p)=\sigma(id_p)("s\_marking")$.
  \item\label{item:sim-tc}
    $\forall{}t\in{}T_i,id_t\in{}Comps(\Delta)~s.t.~\gamma(t)=id_t,$\\
    $\big(upper(I_s(t))=\infty\land{}s.I(t)\le{}lower(I_s(t))\Rightarrow{}s.I(t)=\sigma(id_t)("s\_time\_counter")\big)$\\
    $\land\big(upper(I_s(t))=\infty\land{}s.I(t)>{}lower(I_s(t))\Rightarrow{}\sigma(id_t)("s\_time\_counter")=lower(I_s(t))\big)$\\
    $\land\big(upper(I_s(t))\neq\infty\land{}s.I(t)>{}upper(I_s(t))\Rightarrow{}\sigma(id_t)("s\_time\_counter")=upper(I_s(t))\big)$\\
    $\land\big(upper(I_s(t))\neq\infty\land{}s.I(t)\le{}upper(I_s(t))\Rightarrow{}s.I(t)=\sigma(id_t)("s\_time\_counter")\big)$.
  \item\label{item:sim-reset}
    $\forall{}t\in{}T_i,id_t\in{}Comps(\Delta)~s.t.~\gamma(t)=id_t,$
    $s.reset_t(t)=\sigma(id_t)("s\_reinit\_time\_counter")$.
  \item\label{item:sim-cond}
    $\forall{}c\in\mathcal{C},id_c\in{}Ins(\Delta)~s.t.~\gamma(c)=id_c,~s.cond(c)=\sigma(id_c)$.
  \item\label{item:sim-act}
    $\forall{}a\in\mathcal{A},id_a\in{}Outs(\Delta)~s.t.~\gamma(a)=id_a,~s.ex(a)=\sigma(id_a)$.
  \item\label{item:sim-fun}
    $\forall{}f\in\mathcal{F},id_f\in{}Outs(\Delta)~s.t.~\gamma(f)=id_f,~s.ex(f)=\sigma(id_f)$.
  \end{enumerate}
\end{definition}

In Item~\ref{item:sim-mark}, based on the $\gamma$ binder, we relate
the marking value of a place $p$ to the value of the $s\_marking$
signal inside the internal state of the place component instance
$id_p$. Items~\ref{item:sim-tc} and \ref{item:sim-reset} similarly
relate the value of time counters (resp. reset orders) of transitions
to the value of the signals $s\_time\_counter$
(resp. $s\_reinit\_time\_counter$) in the internal state of the
corresponding transition component instances. In item
\ref{item:sim-cond} (resp. \ref{item:sim-act} and \ref{item:sim-fun}),
the boolean value of conditions (resp. actions and functions) are
compared to the value of input (resp. output) ports of the \hvhdl{}
design, also based on the $\gamma$ binder.

\begin{todobox}
  Explain the time counter particular relation.
\end{todobox}

The second question that we asked above was: when does this state
similarity relation must hold in the course of the execution? The
source and target representations are both synchronously
executed. Thus, we find it natural to check that the state similarity
relation holds at the end of a clock cycle. However, due to
modifications resulting after a bug detection and correction (see
Section~\ref{sec:detailled-proof}), the state similarity relation of
Definition~\ref{sec:state-sim-relation} does not hold at the end of a
clock cycle. The equality between reset orders
(Item~\ref{item:sim-reset}) is not verified. However, this semantic
divergence is without effect. New reset orders are computed at the
beginning of a clock cycle such that the relation of
Item~\ref{item:sim-reset} holds in the middle of the clock cycle (i.e,
just before the falling edge of the clock). This is the only moment
during the clock cycle where the $s\_reinit\_time\_counter$ signal is
actually involved in the computation of other signals value. Thus, it
is sufficient that Item~\ref{item:sim-reset} holds only in the middle
of the clock cycle. However, we must now defined two state similarity
relation; one that checks the semantic matching after the rising edge
of the clock signal (i.e, in the middle of the clock cycle), and one
that checks the semantic matching after the falling edge of the clock
signal (i.e, at the end of the clock cycle). The state similarity
relation after a rising edge is defined as follows:

\begin{definition}[Post rising edge state similarity]
  \label{def:post-re-state-sim}
  For a given $sitpn\in{}SITPN$, a \hvhdl{} design $d\in{}design$, an
  elaborated design $\Delta\in{}ElDesign(d,\mathcal{D}_\mathcal{H})$,
  and a binder $\gamma\in{}WM(sitpn,d)$, a clock cycle count
  $\tau\in\mathbb{N}$, and an SITPN execution environment
  $E_c\in\mathbb{N}\rightarrow\mathcal{C}\rightarrow\mathbb{B}$, an
  SITPN state $s\in{}S(sitpn)$ and a design state
  $\sigma\in\Sigma(\Delta)$ are similar after a rising edge happening
  at clock cycle count $\tau$, written
  $\gamma,E_c,\tau\vdash{}s\stackrel{\uparrow}{\sim}\sigma$ iff
  \begin{enumerate}
  \item
    $\forall{}p\in{}P,id_p\in{}Comps(\Delta)~s.t.~\gamma(p)=id_p,~s.M(p)=\sigma(id_p)("s\_marking")$.
  \item
    $\forall{}t\in{}T_i,id_t\in{}Comps(\Delta)~s.t.~\gamma(t)=id_t,$\\
    $\big(upper(I_s(t))=\infty\land{}s.I(t)\le{}lower(I_s(t))\Rightarrow{}s.I(t)=\sigma(id_t)("s\_time\_counter")\big)$\\
    $\land\big(upper(I_s(t))=\infty\land{}s.I(t)>{}lower(I_s(t))\Rightarrow{}\sigma(id_t)("s\_time\_counter")=lower(I_s(t))\big)$\\
    $\land\big(upper(I_s(t))\neq\infty\land{}s.I(t)>{}upper(I_s(t))\Rightarrow{}\sigma(id_t)("s\_time\_counter")=upper(I_s(t))\big)$\\
    $\land\big(upper(I_s(t))\neq\infty\land{}s.I(t)\le{}upper(I_s(t))\Rightarrow{}s.I(t)=\sigma(id_t)("s\_time\_counter")\big)$.
  \item
    $\forall{}t\in{}T_i,id_t\in{}Comps(\Delta)~s.t.~\gamma(t)=id_t,$
    $s.reset_t(t)=\sigma(id_t)("s\_reinit\_time\_counter")$.
  \item
    $\forall{}a\in\mathcal{A},id_a\in{}Outs(\Delta)~s.t.~\gamma(a)=id_a,~s.ex(a)=\sigma(id_a)$.
  \item
    $\forall{}f\in\mathcal{F},id_f\in{}Outs(\Delta)~s.t.~\gamma(f)=id_f,~s.ex(f)=\sigma(id_f)$.
  \end{enumerate}
\end{definition}

Definition~\ref{def:post-re-state-sim} is similar to
Definition~\ref{def:state-sim} in all points, except for the value of
conditions. A condition of an SITPN is implemented by an primary input
port in the resulting \hvhdl{} design. In \hvhdl{} semantics, the
value of primary input ports (i.e, the input ports of the top-level
design) are updated at each clock edge. In the SITPN semantics, the
value of conditions are updated only at the falling edge of the
clock. Consider that a given SITPN is executed at clock cycle $\tau$;
after the rising edge of the clock, the value of conditions are equal
to their value at clock cycle $\tau-1$, whereas the value primary
input ports have been updated to fresh values. Thus, we will have to
wait for the next falling edge to reach the equality between condition
values and input port values.

The state similarity relation draws out a correspondence between the
values hold by an SITPN state and the values of the signals declared
in an \hvhdl{} design state. However, to complete the proof of
semantic preservation, we sometimes have to relate the value of
signals to the value of expressions or predicates involved in the
SITPN semantics. For instance, consider a given SITPN state $s$ and a
given \hvhdl{} design state $\sigma$, and consider a transition $t$
and its corresponding transition component instance $id_t$. It is
useful to show that, after a rising edge, the value of signal
$s\_enabled$ at state $\sigma(id_t)$, where $\sigma(id_t)$ denotes the
internal state of component instance $id_t$ at state $\sigma$, is
equal to the predicate $t\in{}Sens(s.M)$ stating that the transition
$t$ is sensitized (or \textit{enabled}) by the marking at state $s$
(i.e, $s.M$). Thus, for the convenience of the proof, we enrich our
definitions of the state similiraty relations with formulas relating
\hvhdl{} signals to SITPN semantics predicates and
expressions. Consequently, the \textit{full} post rising edge state
similarity relation is defined as follows:

\begin{definition}[Full post rising edge state similarity]
  \label{def:full-post-re-state-sim}
  For a given $sitpn\in{}SITPN$, a \hvhdl{} design $d\in{}design$, an
  elaborated design $\Delta\in{}ElDesign(d,\mathcal{D}_\mathcal{H})$,
  and a binder $\gamma\in{}WM(sitpn,d)$, a clock cycle count
  $\tau\in\mathbb{N}$, and an SITPN execution environment
  $E_c\in\mathbb{N}\rightarrow\mathcal{C}\rightarrow\mathbb{B}$, an
  SITPN state $s\in{}S(sitpn)$ and a design state
  $\sigma\in\Sigma(\Delta)$ are fully similar after a rising edge
  happening at clock cycle count $\tau$, written
  $\gamma,E_c,\tau\vdash{}s\stackrel{\uparrow}{\approx}\sigma$ iff
  $\gamma\vdash{}s\stackrel{\uparrow}{\sim}\sigma$
  (Definition~\ref{def:post-re-state-sim}) and
  \begin{enumerate}
  \item $\forall{}t\in{}T,id_t\in{}Comps(\Delta)~s.t.~\gamma(t)=id_t,$
    $t\in{}Sens(s.M)\Leftrightarrow\sigma(id_t)("s\_enabled")=\mathtt{true}$.
  \item $\forall{}t\in{}T,id_t\in{}Comps(\Delta)~s.t.~\gamma(t)=id_t,$
    $t\notin{}Sens(s.M)\Leftrightarrow\sigma(id_t)("s\_enabled")=\mathtt{false}$.
  \item
    $\forall{}t\in{}T,id_t\in{}Comps(\Delta)~s.t.~\gamma(t)=id_t,$\\
    $\sigma(id_t)("s\_condition\_combination")=
    \prod\limits_{c\in{}conds(t)}
    \begin{cases}
      E_c(\tau,c) & if~\mathbb{C}(t,c)=1 \\
      \mathtt{not}(E_c(\tau,c)) & if~\mathbb{C}(t,c)=-1 \\
    \end{cases}$\\
    where
    $conds(t)=\{c\in\mathcal{C}~\vert~\mathbb{C}(t,c)=1\lor\mathbb{C}(t,c)=-1\}$.
  \end{enumerate}
\end{definition}

Definition~\ref{def:full-post-re-state-sim} extends
Definition~\ref{def:post-re-state-sim} with the correspondence of the
sensitization of transitions and the value of signal $s\_enabled$, and
the computation of the boolean product of condition values and the
value of signal $s\_condition\_combination$.

The state similarity relation after a falling edge is defined as
follows:

\begin{definition}[Post falling edge state similarity]
  \label{def:post-fe-state-sim}
  For a given $sitpn\in{}SITPN$, a \hvhdl{} design $d\in{}design$, an
  elaborated design $\Delta\in{}ElDesign(d,\mathcal{D}_\mathcal{H})$,
  and a binder $\gamma\in{}WM(sitpn,d)$, an SITPN state
  $s\in{}S(sitpn)$ and a design state $\sigma\in\Sigma(\Delta)$ are
  similar after a falling edge, written
  $\gamma\vdash{}s\stackrel{\downarrow}{\sim}\sigma$ iff
  \begin{enumerate}
  \item $\forall{}p\in{}P,id_p\in{}Comps(\Delta)~s.t.~\gamma(p)=id_p,$
    $~s.M(p)=\sigma(id_p)("s\_marking")$.
  \item
    $\forall{}t\in{}T_i,id_t\in{}Comps(\Delta)~s.t.~\gamma(t)=id_t,$\\
    $\big(upper(I_s(t))=\infty\land{}s.I(t)\le{}lower(I_s(t))\Rightarrow{}s.I(t)=\sigma(id_t)("s\_time\_counter")\big)$\\
    $\land\big(upper(I_s(t))=\infty\land{}s.I(t)>{}lower(I_s(t))\Rightarrow{}\sigma(id_t)("s\_time\_counter")=lower(I_s(t))\big)$\\
    $\land\big(upper(I_s(t))\neq\infty\land{}s.I(t)>{}upper(I_s(t))\Rightarrow{}\sigma(id_t)("s\_time\_counter")=upper(I_s(t))\big)$\\
    $\land\big(upper(I_s(t))\neq\infty\land{}s.I(t)\le{}upper(I_s(t))\Rightarrow{}s.I(t)=\sigma(id_t)("s\_time\_counter")\big)$.
  \item
    $\forall{}c\in\mathcal{C},id_c\in{}Ins(\Delta)~s.t.~\gamma(c)=id_c,~s.cond(c)=\sigma(id_c)$.
  \item
    $\forall{}a\in\mathcal{A},id_a\in{}Outs(\Delta)~s.t.~\gamma(a)=id_a,~s.ex(a)=\sigma(id_a)$.
  \item
    $\forall{}f\in\mathcal{F},id_f\in{}Outs(\Delta)~s.t.~\gamma(f)=id_f,~s.ex(f)=\sigma(id_f)$.
  \end{enumerate}
\end{definition}

As explained above, Definition~\ref{def:post-fe-state-sim} is similar
to Definition~\ref{def:state-sim} except for the equality between
reset orders and the value of signal $s\_reinit\_time\_counter$.

The extended version of the post falling edge state similarity
relation is as follows:

\begin{definition}[Full Post falling edge state similarity]
  \label{def:full-post-fe-state-sim}
  For a given $sitpn\in{}SITPN$, a \hvhdl{} design $d\in{}design$, an
  elaborated design $\Delta\in{}ElDesign(d,\mathcal{D}_\mathcal{H})$,
  and a binder $\gamma\in{}WM(sitpn,d)$, an SITPN state
  $s\in{}S(sitpn)$ and a design state $\sigma\in\Sigma(\Delta)$ are
  fully similar after a falling edge, written
  $\gamma\vdash{}s\stackrel{\downarrow}{\approx}\sigma$ iff
  $\gamma\vdash{}s\stackrel{\downarrow}{\sim}\sigma$
  (Definition~\ref{def:post-fe-state-sim}) and
  \begin{enumerate}
  \item $\forall{}t\in{}T,id_t\in{}Comps(\Delta)~s.t.~\gamma(t)=id_t,$
    $t\in{}Firable(s)\Leftrightarrow\sigma(id_t)("s\_firable")=\mathtt{true}$.
  \item $\forall{}t\in{}T,id_t\in{}Comps(\Delta)~s.t.~\gamma(t)=id_t,$
    $t\notin{}Firable(s)\Leftrightarrow\sigma(id_t)("s\_firable")=\mathtt{false}$.
  \item $\forall{}t\in{}T,id_t\in{}Comps(\Delta)~s.t.~\gamma(t)=id_t,$
    $t\in{}Fired(s)\Leftrightarrow\sigma(id_t)("fired")=\mathtt{true}$.
  \item $\forall{}t\in{}T,id_t\in{}Comps(\Delta)~s.t.~\gamma(t)=id_t,$
    $t\notin{}Fired(s)\Leftrightarrow\sigma(id_t)("fired")=\mathtt{false}$.
  \item $\forall{}p\in{}P,id_p\in{}Comps(\Delta)~s.t.~\gamma(p)=id_p,$
    $\sum\limits_{t\in{}Fired(s)}pre(p,t)=\sigma(id_p)("s\_output\_token\_sum")$.
  \item $\forall{}p\in{}P,id_p\in{}Comps(\Delta)~s.t.~\gamma(p)=id_p,$
    $\sum\limits_{t\in{}Fired(s)}post(t,p)=\sigma(id_p)("s\_input\_token\_sum")$.
  \end{enumerate}
\end{definition}

\noindent{}Definition~\ref{def:post-fe-state-sim} extends
Definition~\ref{def:post-fe-state-sim} by drawing out a correspondence
between:
\begin{itemize}
\item the firability of transitions and the value of the signal $s\_firable$
\item the firing status of transitions (i.e, transitions are fired or
  not) and the value of the output port $fired$
\item the sum of tokens consumed by the firing process and the value
  of the signal $s\_output\_token\_sum$
\item the sum of tokens produced by the firing process and the value
  of the signal $s\_input\_token\_sum$
\end{itemize}



%%% Local Variables:
%%% mode: latex
%%% TeX-master: "../../main"
%%% End:
