In this section, we will lay out the major theorems and lemmas stating
that the \hilecop{} transformation function is semantic preserving. We
will also present the written proofs for these theorems and lemmas. 

\subsection{Proof notations}
\label{sec:pf-notations}

To add some readability to our proofs, we use the following notations:

\begin{itemize}
\item Following the sense of reading, the most recent framed box
  denotes the current pending goal (what we are currently trying to
  prove): \framebox{$\forall{}n\in\mathbb{N},~n>0\lor{}n=0$}
\item A red framed box denotes a completed goal (i.e, equivalent to
  qed): \colorbox{red!20}{$\mathtt{true}=\mathtt{true}$}
\item A green framed box denotes the current induction
  hypothesis: \begin{ih}$\forall{}n\in\mathbb{N},~n+1>0$\end{ih}
\item The mention \textbf{CASE} directly follows an item bullet to
  denote a case during a proof by case analysis.
\end{itemize}

During a proof, we constantly refer to the names of the constants and
signals declared in the \hvhdl{} place and transition designs. Some
constants and signals have very long names, and therefore we use
aliases to refer to them in the following
proofs. Table~\ref{tab:consts-and-sigs-ref} gives the full
correspondence between constants and signals, and their aliases.

\subsection{Preliminary definitions}
\label{sec:beh-pres-prelim-defs}

We define here some relations that are necessary to formalize the
theorem of behavior preservation.

In an SITPN, the conditions associated to transitions receive fresh
boolean values from an execution environment at each falling edge of
the clock.  During the simulation of a top-level design, the input
ports of the design receive fresh values from a simulation environment
at each clock event. The transformation function generates an input
port in the top-level design that will mimick the behavior of a given
SITPN condition. The binder $\gamma$, generated alongside the
top-level design, relates a given condition $c$ to its corresponding
input port identifier $id_c$. To compare the execution/simulation
traces of an SITPN and a \hvhdl{} design, we must assume that the
execution/simulation environments assign similar values to conditions
and to their corresponding input ports at a given clock
cycle. Definition~\ref{def:sim-env} states that the execution
environment for a given SITPN and the simulation environment for a
given \hvhdl{} design are similar.

\begin{definition}[Similar environments]
  \label{def:sim-env}
  For a given $sitpn\in{}SITPN$, a \hvhdl{} design $d\in{}design$, a
  design store
  $\mathcal{D}\in{}entity\mhyphen{}id\nrightarrow{}design$, an
  elaborated version $\Delta\in{}ElDesign(d,\mathcal{D})$ of design
  $d$, and a binder $\gamma\in{}WM(sitpn,d)$, the environment
  $E_p\in{}(\mathbb{N}\times{}\{\uparrow,\downarrow\})\rightarrow{}Ins(\Delta)\rightarrow{}value$,
  that yields the value of the primary input ports of $\Delta$ at a
  given simulation cycle and a given clock event, and the environment
  $E_c$, that yields the value of conditions of $sitpn$ at a given
  execution cycle, are similar, noted
  $\gamma\vdash{}E_p\stackrel{env}{=}E_c$, iff for all
  $\tau\in{}\mathbb{N}$, $clk\in\{\uparrow,\downarrow\}$,
  $c\in\mathcal{C}$, $id_c\in{}Ins(\Delta)$ s.t.  $\gamma(c)=id_c$,
  $E_p(\tau,clk)(id_c)=E_c(\tau)(c)$.
\end{definition}

Definition~\ref{def:sim-env} also states that every input port of the
top-level design related to a SITPN condition by the $\gamma$ binder
has a stable boolean value during a whole clock cycle. That is to say,
in the context of Definition~\ref{def:sim-env}, there exists no $id_c$
such that $E_p(\tau,\uparrow)(id_c)\neq{}E_p(\tau,\downarrow)(id_c)$.

To prove that the behavior of an SITPN and a \hvhdl{} design are
similar, we want to compare the states composing their
execution/simulation traces.  The relation presented in
Definition~\ref{def:exec-trace-sim} permits to compare such traces.

\begin{definition}[Execution trace similarity]
  \label{def:exec-trace-sim}
  For a given $sitpn\in{}SITPN$, a \hvhdl{} design $d\in{}design$, an
  elaborated design $\Delta\in{}ElDesign(d,\mathcal{D}_\mathcal{H})$,
  and a binder $\gamma\in{}WM(sitpn,d)$, the execution trace
  $\theta_s\in{}\mathtt{list}(S(sitpn))$ and the simulation trace
  $\theta_\sigma\in\mathtt{list}(\Sigma(\Delta))$ are similar, written
  $\gamma\vdash{}\theta_s\stackrel{clk}{\sim}\theta_\sigma$, where
  $clk\in\{\uparrow,\downarrow\}$, according to the following rules:

  \begin{tabular}{@{}l}
    {\fontsize{9}{11}\selectfont\textsc{SimTraceNil}} \\
    
    {\begin{prooftree}[template={\fontsize{11}{13}\selectfont\inserttext}]        
        \infer0[$clk\in{}\{\uparrow,\downarrow\}$]{$\gamma\vdash{}[~]\stackrel{clk}{\sim}{}[~]$}
      \end{prooftree}} 
  \end{tabular}
  \begin{tabular}{@{}l}
    {\fontsize{9}{11}\selectfont\textsc{SimTrace$\uparrow$}} \\
    
    {\begin{prooftree}[template={\fontsize{11}{13}\selectfont\inserttext}]

        \hypo{$\gamma\vdash{}s\stackrel{\uparrow}{\sim}\sigma$}
        \hypo{$\gamma\vdash{}\theta_s\stackrel{\downarrow}{\sim}{}\theta_\sigma$}
        \infer2{$\gamma\vdash{}(s :: \theta_s)\stackrel{\uparrow}{\sim}{}(\sigma :: \theta_\sigma)$}
      \end{prooftree}} 
  \end{tabular}
  \begin{tabular}{@{}l}
    {\fontsize{9}{11}\selectfont\textsc{SimTrace$\downarrow$}} \\
    
    {\begin{prooftree}[template={\fontsize{11}{13}\selectfont\inserttext}]

        \hypo{$\gamma\vdash{}s\stackrel{\downarrow}{\sim}\sigma$}
        \hypo{$\gamma\vdash{}\theta_s\stackrel{\uparrow}{\sim}{}\theta_\sigma$}
        \infer2{$\gamma\vdash{}(s :: \theta_s)\stackrel{\downarrow}{\sim}{}(\sigma :: \theta_\sigma)$}
      \end{prooftree}} 
  \end{tabular}
\end{definition}

In Definition~\ref{def:exec-trace-sim}, the clock event symbol on top
of the $\sim$ sign indicates the kind of clock event that led to the
production of the states at the head of the traces. The execution
trace similarity relation expects that the states composing the traces
have been alternatively produced by a rising edge and then by a
falling edge. By construction, the traces must have the same length to
respect the execution trace similarity relation.

To handle the case of an execution/simulation trace beginning by a
initial state, that is, a state neither reached after a rising nor
after falling edge, we give a slightly different definition of the
execution trace similarity relation in
Definition~\ref{def:full-exec-trace-sim}.

\begin{definition}[Full execution trace similarity]
  \label{def:full-exec-trace-sim} For a given $sitpn\in{}SITPN$, a
  \hvhdl{} design $d\in{}design$, an elaborated design
  $\Delta\in{}ElDesign(d,\mathcal{D}_\mathcal{H})$, and a binder
  $\gamma\in{}WM(sitpn,d)$, the execution trace
  $\theta_s\in{}\mathtt{list}(S(sitpn))$ and the simulation trace
  $\theta_\sigma\in\mathtt{list}(\Sigma(\Delta))$ are fully similar,
  written $\gamma\vdash{}\theta_s\sim\theta_\sigma$, according to the
  following rules:

  \begin{tabular}{@{}l}
    {\fontsize{9}{11}\selectfont\textsc{FullSimTraceNil}} \\
    
    {\begin{prooftree}[template={\fontsize{11}{13}\selectfont\inserttext}]
        \infer0{$\gamma\vdash{}[~]\sim{}[~]$}
      \end{prooftree}}
  \end{tabular}
  \begin{tabular}{@{}l}
    {\fontsize{9}{11}\selectfont\textsc{FullSimTraceCons}} \\
    
    {\begin{prooftree}[template={\fontsize{11}{13}\selectfont\inserttext}]

        \hypo{$\gamma\vdash{}s\sim\sigma$}
        \hypo{$\gamma\vdash{}\theta_s\stackrel{\uparrow}{\sim}{}\theta_\sigma$}
        \infer2{$\gamma\vdash{}(s :: \theta_s)\sim{}(\sigma ::
          \theta_\sigma)$}
      \end{prooftree}}
  \end{tabular}
\end{definition}

The full execution trace similarity relation indicates that the head
states of traces must verify the general state similarity relation,
and that the tail of the traces must respect the execution state
similarity relation starting a rising edge. 

\subsection{The behavior preservation theorem}
\label{sec:beh-pres-thm-pf}

%%%%%%%%%%%%%%%%%%%%%%%%%%%%%%%%%%%%%%%%%%%
%%%%%% BEHAVIOR PRESERVATION THEOREM %%%%%%
%%%%%%%%%%%%%%%%%%%%%%%%%%%%%%%%%%%%%%%%%%%

Theorem~\ref{thm:beh-pres} states that the \hilecop{} transformation
is semantic preserving when the input model is a well-defined
SITPN. As a complementary task, we could show that if the
transformation function returns a couple design and binder, and not an
error, then the input SITPN is well-defined. 

\begin{thm}[Behavior Preservation]
  \label{thm:beh-pres}
  For all well-defined $sitpn\in{}SITPN$, an \hvhdl{} design
  $d\in{}design$, a binder $\gamma\in{}WM(sitpn,d)$, a clock cycle
  count $\tau\in\mathbb{N}$, a execution environment
  $E_c\in{}\mathbb{N}\rightarrow{}\mathcal{C}\rightarrow{}\mathbb{B}$
  and an execution trace $\theta_s\in\mathtt{list}(S(sitpn))$ s.t.
  \begin{itemize}
  \item SITPN $sitpn$ translates into design $d$ and yields a binder
    $\gamma$: $\lfloor{}sitpn\rfloor_\mathcal{H}=(d,\gamma)$
  \item SITPN $sitpn$ yields the execution trace $\theta_s$ after
    $\tau$ execution cycles in environment $E_c$:\\
    $E_c,\tau\vdash{}sitpn\xrightarrow{full}\theta_s$
  \end{itemize}
  
  \noindent{}then there exists an elaborated design
  $\Delta\in{}ElDesign(d,\mathcal{D}_\mathcal{H})$ s.t. for all
  simulation environment
  $E_p\in{}(\mathbb{N}\times{}\{\uparrow,\downarrow\})\rightarrow{}Ins(\Delta)\rightarrow{}value$,
  verifying
  \begin{itemize}
  \item Simulation/Execution environments are similar:
    $\gamma\vdash{}E_p\stackrel{env}{=}E_c$
  \end{itemize}
  then there exists a simulation trace
  $\theta_\sigma\in\mathtt{list}(\Sigma(\Delta))$ s.t.
  \begin{itemize}
  \item Under the \hilecop{} design store $\mathcal{D}_\mathcal{H}$
    and with an empty generic constant dimensioning function
    ($\emptyset$), design d yields the simulation trace
    $\theta_\sigma$ after $\tau$
    simulation cycles:\\
    $\mathcal{D}_\mathcal{H},\Delta,\emptyset,E_p,\tau\vdash{}\mathrm{d}\xrightarrow{full}\theta_\sigma$
  \item Traces $\theta_s$ and $\theta_\sigma$ are fully similar:
    $\theta_s\sim\theta_\sigma$
  \end{itemize}
\end{thm}

\begin{niproof}
  Given a $sitpn\in{}SITPN$, a $d\in{}design$, a
  $\gamma\in{}WM(sitpn,d)$, a $\tau\in\mathbb{N}$, an
  $E_c\in{}\mathbb{N}\rightarrow{}\mathcal{C}\rightarrow{}\mathbb{B}$
  and a $\theta_s\in\mathtt{list}(S(sitpn))$, let us show that\\
  \framebox{$\exists\Delta,~\forall{}E_p,~\gamma\vdash{}E_p\stackrel{env}{=}E_c,~\exists\theta_\sigma$
    s.t.
    $\mathcal{D}_\mathcal{H},\Delta,\emptyset,E_p,\tau\vdash{}\mathrm{d}\xrightarrow{full}\theta_\sigma\land\theta_s\sim\theta_\sigma$}\\

  Appealing to Theorems~\nameref{thm:elab-ex}, \nameref{thm:init-ex}
  and \nameref{thm:sim-ex}, let us take an elaborated design $\Delta$,
  two design states $\sigma_e,\sigma_0\in\Sigma(\Delta)$, and a
  simulation trace $\theta_\sigma\in{}$ such that:
  \begin{itemize}
  \item $\mathcal{D}_\mathcal{H},\emptyset\vdash{}d\srarrow{elab}{\fontsize{6}{8}\selectfont}(\Delta,\sigma_{e})$
  \item $\mathcal{D}_\mathcal{H},\Delta,\sigma_{e}\vdash{}d.cs\srarrow{init}{\fontsize{6}{8}\selectfont}\sigma_0$
  \item $\mathcal{D}_\mathcal{H},E_p,\Delta,\tau,\sigma_0\vdash{}\mathrm{d.cs}\rightarrow\theta_\sigma$
  \end{itemize}
  
  
  \noindent{}By definition of the \hvhdl{} full simulation relation,
  we have:
  \begin{equation}
    \begin{split}
      \mathcal{D}_\mathcal{H},\Delta,\emptyset,E_p,\tau\vdash{}\mathrm{d}\xrightarrow{full}\theta_\sigma\equiv
      \exists\sigma_e,\sigma_0\in\Sigma(\Delta),~\mathcal{D}_\mathcal{H},\emptyset\vdash{}d\srarrow{elab}{\fontsize{6}{8}\selectfont}(\Delta,\sigma_{e})\\
      \land\mathcal{D}_\mathcal{H},\Delta,\sigma_{e}\vdash{}d.cs\srarrow{init}{\fontsize{6}{8}\selectfont}\sigma_0\\
      \land\mathcal{D}_\mathcal{H},E_p,\Delta,\tau,\sigma_0\vdash{}\mathrm{d.cs}\rightarrow\theta_\sigma
    \end{split}
  \label{eq:full-sim}
  \end{equation}
  
  Rewriting the goal with \eqref{eq:full-sim}:
  \begin{frameb}
    $\exists\Delta,~\forall{}E_p,~\gamma\vdash{}E_p\stackrel{env}{=}E_c,~\exists\theta_\sigma,\sigma_e,\sigma_0$
    s.t.
    $\mathcal{D}_\mathcal{H},\emptyset\vdash{}d\srarrow{elab}{\fontsize{6}{8}\selectfont}(\Delta,\sigma_{e})
    \land\mathcal{D}_\mathcal{H},\Delta,\sigma_{e}\vdash{}d.cs\srarrow{init}{\fontsize{6}{8}\selectfont}\sigma_0
    \land\mathcal{D}_\mathcal{H},E_p,\Delta,\tau,\sigma_0\vdash{}\mathrm{d.cs}\rightarrow\theta_\sigma\land\theta_s\sim\theta_\sigma$
  \end{frameb}

  Let us use $\Delta$, $\sigma_e$, $\sigma_0\in\Sigma(\Delta)$ and
  $\theta_\sigma$ to prove the goal:
  \begin{frameb}
    $\mathcal{D}_\mathcal{H},\emptyset\vdash{}d\srarrow{elab}{\fontsize{6}{8}\selectfont}(\Delta,\sigma_{e})
    \land\mathcal{D}_\mathcal{H},\Delta,\sigma_{e}\vdash{}d.cs\srarrow{init}{\fontsize{6}{8}\selectfont}\sigma_0
    \land\mathcal{D}_\mathcal{H},E_p,\Delta,\tau,\sigma_0\vdash{}\mathrm{d.cs}\rightarrow\theta_\sigma\land\theta_s\sim\theta_\sigma$
  \end{frameb}
  
  We assumed the three first points of the goal, and the last point,
  i.e $\theta_s\sim\theta_\sigma$, is proved by appealing to
  Theorem~\nameref{thm:full-bisim}.

\end{niproof}

To prove Theorem~\ref{thm:beh-pres}, we must first prove that for all
\hvhdl{} design returned by the transformation function, there exists
an elaborated version of it (Theorem~\nameref{thm:elab-ex}); then, we
must prove that we can always build a simulation trace respecting the
\hvhdl{} simulation relation over $\tau$ simulation cycles
(Theorem~\nameref{thm:init-ex} and \nameref{thm:sim-ex}). Finally, we
can establish that the behaviors are similar by comparing the
respective SITPN execution and \hvhdl{} simulation traces. In this
thesis, we are focusing on the proof that the execution/simulation
traces are similar. For now, we choose to disregard the proof of
theorems \nameref{thm:elab-ex}, \nameref{thm:init-ex} and
\nameref{thm:sim-ex} stating the existence of an elaborated design and
of a simulation trace for all \hvhdl{} design returned by the
\hilecop{} transformation function.

\begin{thm}[Elaboration]
  \label{thm:elab-ex}
  For all $sitpn\in{}SITPN$, $d\in{}design$, $\gamma\in{}WM(sitpn,d)$
  s.t.
  \begin{itemize}
  \item $\lfloor{}sitpn\rfloor_\mathcal{H}=(d,\gamma)$
  \end{itemize}
  \noindent{}then there exists an elaborated design
  $\Delta\in{}ElDesign(d,\mathcal{D}_\mathcal{H})$ and a design state
  $\sigma_e\in\Sigma(\Delta)$ s.t.
  \begin{itemize}
  \item $\Delta$ is the elaborated version of design $d$, and
    $\sigma_e$ is the default design state of $\Delta$:
    $\mathcal{D}_\mathcal{H},\emptyset\vdash{}d\srarrow{elab}{\fontsize{6}{8}\selectfont}(\Delta,\sigma_{e})$
  \end{itemize}
\end{thm}

\begin{thm}[Initialization]
  \label{thm:init-ex}
  For all $sitpn\in{}SITPN$, $d\in{}design$, $\gamma\in{}WM(sitpn,d)$,
  $\Delta\in{}ElDesign(d,\mathcal{D}_\mathcal{H})$,
  $\sigma_e\in\Sigma(\Delta)$ s.t.
  \begin{itemize}
  \item $\lfloor{}sitpn\rfloor_\mathcal{H}=(d,\gamma)$ and
    $\mathcal{D}_\mathcal{H},\emptyset\vdash{}d\srarrow{elab}{\fontsize{6}{8}\selectfont}(\Delta,\sigma_{e})$
  \end{itemize}
  \noindent{}then there exists a design state
  $\sigma_0\in\Sigma(\Delta)$ s.t.
  \begin{itemize}
  \item $\sigma_0$ is the initial simulation state:
    $\mathcal{D}_\mathcal{H},\Delta,\sigma_{e}\vdash{}d.cs\srarrow{init}{\fontsize{6}{8}\selectfont}\sigma_0$
  \end{itemize}
\end{thm}

\begin{thm}[Simulation]
  \label{thm:sim-ex}
  For all $sitpn\in{}SITPN$, $d\in{}design$, $\gamma\in{}WM(sitpn,d)$,
  $\Delta\in{}ElDesign(d,\mathcal{D}_\mathcal{H})$,
  $\sigma_e,\sigma_0\in\Sigma(\Delta)$ s.t.
  \begin{itemize}
  \item $\lfloor{}sitpn\rfloor_\mathcal{H}=(d,\gamma)$ and
    $\mathcal{D}_\mathcal{H},\emptyset\vdash{}d\srarrow{elab}{\fontsize{6}{8}\selectfont}(\Delta,\sigma_{e})$
    and
    $\mathcal{D}_\mathcal{H},\Delta,\sigma_{e}\vdash{}d.cs\srarrow{init}{\fontsize{6}{8}\selectfont}\sigma_0$
  \end{itemize}
  
  \noindent{}then for all simulation environment
  $E_p\in{}(\mathbb{N}\times{}\{\uparrow,\downarrow\})\rightarrow{}Ins(\Delta)\rightarrow{}value$,
  and simulation cycle count $\tau\in\mathbb{N}$, there exists s
  simulation trace $\theta_\sigma\in\mathtt{list}(\Sigma(\Delta))$
  s.t.
  \begin{itemize}
  \item Design $d$ yields the simulation trace $\theta_\sigma$ after
    $\tau$ simulation cycles, starting from initial state $\sigma_0$:\\
    $\mathcal{D}_\mathcal{H},E_p,\Delta,\tau,\sigma_0\vdash{}\mathrm{d.cs}\rightarrow\theta_\sigma$
  \end{itemize}
\end{thm}

\subsection{The bisimulation theorem}
\label{sec:bisim-thm-and-proof}

%%%%%%%%%%%%%%%%%%%%%%%%%%%%%%%%%%%%%%%%%%%
%%%%%%%%%% FULL BISIMULATION THM %%%%%%%%%%
%%%%%%%%%%%%%%%%%%%%%%%%%%%%%%%%%%%%%%%%%%%

\begin{thm}[Full Bisimulation]
  \label{thm:full-bisim}
  For all $sitpn\in{}SITPN$, $d\in{}design$, $\gamma\in{}WM(sitpn,d)$,
  $\tau\in\mathbb{N}$,
  $E_c\in{}\mathbb{N}\rightarrow{}\mathcal{C}\rightarrow{}\mathbb{B}$,
  $\theta_s\in\mathtt{list}(S(sitpn))$,
  $\Delta\in{}ElDesign(d,\mathcal{D}_\mathcal{H})$,
  $E_p\in{}(\mathbb{N}\times{}\{\uparrow,\downarrow\})\rightarrow{}Ins(\Delta)\rightarrow{}value$,
  $\theta_\sigma\in\mathtt{list}(\Sigma(\Delta))$ s.t.
  \begin{itemize}
  \item $\lfloor{}sitpn\rfloor_\mathcal{H}=(d,\gamma)$
  \item $\gamma\vdash{}E_p\stackrel{env}{=}E_c$ 
  \item $E_c,\tau\vdash{}sitpn\xrightarrow{full}\theta_s$
  \item $\mathcal{D}_\mathcal{H},\Delta,\emptyset,E_p,\tau\vdash{}\mathrm{d}\xrightarrow{full}\theta_\sigma$
  \end{itemize}
  then
 $\theta_s\sim\theta_\sigma$

\end{thm}

\begin{proof}
  Case analysis on $\tau$ (2 CASES).

  \begin{itemize}
  \item \textbf{CASE} $\tau=0$. By definition of the SITPN full execution and
    the \hvhdl{} full simulation relations:
    \begin{itemize}
    \item $\mathcal{D}_\mathcal{H},\emptyset\vdash{}d\srarrow{elab}{\fontsize{6}{8}\selectfont}(\Delta,\sigma_{e})$
    \item
      $\Delta,\sigma_{e}\vdash{}d.cs\srarrow{init}{\fontsize{6}{8}\selectfont}\sigma_0$
    \item $\theta_s=[s_0]$ and $\theta_\sigma=[\sigma_0]$
    \end{itemize}
    \framebox{$\gamma\vdash{}s_0\sim{}\sigma_0$} (by def. of similar execution trace relation).
    Solved by applying Lemma~\nameref{lem:sim-init-states}.
  \item \textbf{CASE} $\tau>0$. By definition of the SITPN full execution and
    the \hvhdl{} full execution relations:
    \begin{itemize}
    \item $E_c,\tau\vdash{}s_0\srarrow{\uparrow_0}{\fontsize{6}{8}\selectfont}s_0$
    \item $E_c,\tau\vdash{}s_0\srarrow{\downarrow}{\fontsize{6}{8}\selectfont}s$
    \item $E_c,\tau-1\vdash{}sitpn,s\rightarrow\theta_s$
    \item $\mathcal{D}_\mathcal{H},\emptyset\vdash{}d\srarrow{elab}{\fontsize{6}{8}\selectfont}(\Delta,\sigma_{e})$
    \item $\Delta,\sigma_{e}\vdash{}d.cs\srarrow{init}{\fontsize{6}{8}\selectfont}\sigma_0$
    \item $E_p,\Delta,\tau,\sigma_0\vdash{}\mathrm{d.cs}\rightarrow\theta$
    \end{itemize}
    \framebox{$\gamma\vdash{}(s_0 :: s :: \theta_s)\sim{}(\sigma_0 :: \theta)$}\\\\
    By definition of the \hvhdl{} full simulation relation, we know:
    \begin{itemize}
    \item $E_p,\Delta,\tau,\sigma_0\vdash{}d.cs\xrightarrow{\uparrow,\downarrow}\sigma$
    \item $E_p,\Delta,\tau-1,\sigma\vdash{}\mathrm{d.cs}\rightarrow\theta_\sigma$
    \end{itemize}
    where $\theta=\sigma :: \theta_\sigma$.\\
    
    Rewriting $\theta$ as $\sigma :: \theta_\sigma$, \fbox{$\gamma\vdash{}(s_0 :: s :: \theta_s)\sim{}(\sigma_0 :: \sigma :: \theta_\sigma)$}\\\\
    3 subgoals (by def. of \nameref{def:exec-trace-sim}).
    \begin{enumerate}
    \item $\gamma\vdash{}s_0\sim\sigma_0$ (solved by applying Lemma~\nameref{lem:sim-init-states}).
    \item $\gamma\vdash{}s\sim\sigma$ (solved by applying Lemma~\nameref{lem:first-cycle}).
    \item $\gamma\vdash{}\theta_s\sim\theta_\sigma$ (solved by applying Lemma~\nameref{lem:simulation}).
    \end{enumerate}
  \end{itemize}
\end{proof}

%%%%%%%%%%%%%%%%%%%%%%%%%%%%%%%
%%%%%% FIRST CYCLE LEMMA %%%%%%
%%%%%%%%%%%%%%%%%%%%%%%%%%%%%%%

\begin{lemma}[First Cycle]
  \label{lem:first-cycle}
  For all
  $sitpn\in{}SITPN,d\in{}design,\gamma\in{}WM(sitpn,d),s\in{}S(sitpn),
  \Delta\in{}ElDesign(d,\mathcal{D}_\mathcal{H}),$
  $\sigma_{e},\sigma_0,\sigma\in{}\Sigma(\Delta)$,
  $E_c\in{}\mathbb{N}\rightarrow{}\mathcal{C}\rightarrow{}\mathbb{B}$,
  $E_p\in{}(\mathbb{N}\times{}\{\uparrow,\downarrow\})\rightarrow{}Ins(\Delta)\rightarrow{}value$,
  assume that:
  \begin{itemize}
  \item $\lfloor{}sitpn\rfloor_\mathcal{H}=(d,\gamma)$ and
    $\mathcal{D}_\mathcal{H},\emptyset\vdash{}d\srarrow{elab}{\fontsize{6}{8}\selectfont}(\Delta,\sigma_{e})$
    and $\gamma\vdash{}E_p\stackrel{env}{=}E_c$
  \item $\sigma_0$ is the initial state of $\Delta$: 
    $\Delta,\sigma_{e}\vdash{}d.cs\srarrow{init}{\fontsize{6}{8}\selectfont}\sigma_0$
  \item First execution cycle for $d$: $E_p,\Delta,\tau,\sigma_0\vdash{}d.cs\xrightarrow{\uparrow,\downarrow}\sigma$
  \item
    Particular first execution cycle for $sitpn$ (first rising edge is idle):\\
    $E_c,\tau\vdash{}s_0\srarrow{\uparrow_0}{\fontsize{6}{8}\selectfont}s_0$
    and
    $E_c,\tau\vdash{}s_0\srarrow{\downarrow}{\fontsize{6}{8}\selectfont}s$
  \end{itemize}
  then $\gamma\vdash{}s\stackrel{\downarrow}{\sim}{}\sigma$.
\end{lemma}

\begin{proof}
  Let's show that the first execution cycle leads to two states
  verifying the \nameref{def:post-fe-state-sim} relation:
  \fbox{$\gamma\vdash{}s\stackrel{\downarrow}{\sim}{}\sigma$.}\\

  \noindent{}By definition of the \hvhdl{} cycle relation, we have:
  \begin{itemize}
  \item $\mathtt{Inject}_\uparrow(\sigma_0, E_p, \tau, \sigma_{injr})$
    and
    $\Delta,\sigma_{injr}\vdash\mathrm{d.cs}\xrightarrow{\uparrow}\sigma_r$
    and
    $\Delta,\sigma_r\vdash\mathrm{d.cs}\xrightarrow{\theta}\sigma'$
  \item
    $\mathtt{Inject}_\downarrow(\sigma', E_p, \tau, \sigma_{injf})$
    and
    $\Delta,\sigma_{injf}\vdash\mathrm{d.cs}\xrightarrow{\downarrow}\sigma_f$
    and
    $\Delta,\sigma_f\vdash\mathrm{d.cs}\xrightarrow{\theta'}\sigma$
  \end{itemize}
  
  \noindent{}Then, we can apply the \nameref{lem:fe} lemma to solve \fbox{$\gamma\vdash{}s\stackrel{\downarrow}{\sim}{}\sigma$.}\\

  \noindent{}One premise of the \nameref{lem:fe} lemma remains to be
  proved:
  \fbox{$\gamma,E_c,\tau\vdash{}s_0\stackrel{\uparrow}{\sim}{}\sigma'$.}\\

  \noindent{}Then, we can apply the \nameref{lem:fst-re} lemma to
  solve \fbox{$\gamma,E_c,\tau\vdash{}s_0\stackrel{\uparrow}{\sim}{}\sigma'$.}
  
\end{proof}

%%%%%%%%%%%%%%%%%%%%%%%%%%%%%%%%%%%%%%%%
%%%%%%%%%% BISIMULATION LEMMA %%%%%%%%%%
%%%%%%%%%%%%%%%%%%%%%%%%%%%%%%%%%%%%%%%%

\begin{lemma}[Bisimulation]
  \label{lem:simulation}
  For all $sitpn$, $d$, $\gamma$, $E_p$, $E_c$, $\tau$, $s$, $\theta_s$,
  $\sigma$, $\theta_\sigma$, $\Delta$, $\sigma_e$, assume that:
  \begin{itemize}
  \item $\lfloor{}sitpn\rfloor_\mathcal{H}=(d,\gamma)$ and
    $\gamma\vdash{}E_p\stackrel{env}{=}E_c$ and
    $\mathcal{D}_\mathcal{H},\emptyset\vdash{}d\srarrow{elab}{\fontsize{7}{9}\selectfont}\Delta,\sigma_e$
  \item Starting states are similar as intended after a falling edge:
    $\gamma\vdash{}s\stackrel{\downarrow}{\sim}\sigma$
  \item $E_c,\tau\vdash{}sitpn,s\rightarrow\theta_s$
  \item $E_p,\Delta,\tau,\sigma\vdash{}d.cs\rightarrow\theta_\sigma$
  \end{itemize}
  then $\gamma\vdash{}\theta_s\sim{}\theta_\sigma$.
\end{lemma}

\begin{proof}
  Induction on $\tau$.
  \begin{itemize}
  \item Base case, $\tau=0$: traces are empty, trivial.
  \item Induction case, $\tau > 0$:
    \begin{ih}
      $\forall{}s,\sigma,\theta_s,\theta_\sigma$ s.t.
      $\gamma\vdash{}s\stackrel{\downarrow}{\sim}\sigma$ and
      $E_c,\tau-1\vdash{}sitpn,s\rightarrow\theta_s$ and
      $E_p,\Delta,\tau-1,\sigma\vdash{}d.cs\rightarrow\theta_\sigma$
      then
      $\gamma\vdash{}\theta_s\sim{}\theta_\sigma$.
    \end{ih}
    By definition of the SITPN execution and the \hvhdl{} simulation
    relations for $\tau>0$: 
    \begin{itemize}
    \item $E,\tau\vdash{}sitpn,s\xrightarrow{\uparrow,\downarrow}{}s'$ and $E_c,\tau-1\vdash{}sitpn,s\rightarrow\theta_s$.
    \item $E_p,\Delta,\tau,\sigma\vdash{}\mathrm{d.cs}\xrightarrow{\uparrow,\downarrow}{\fontsize{7}{9}\selectfont}\sigma'$ and
      $E_p,\Delta,\tau-1,\sigma\vdash{}d.cs\rightarrow\theta_\sigma$.
    \end{itemize}
    \framebox{$\gamma\vdash{}(s' :: \theta_s)\sim{}(\sigma' :: \theta_\sigma)$}.\\\\
    2 subgoals (by def. of \nameref{def:exec-trace-sim}).
    \begin{enumerate}
    \item \framebox{$\gamma\vdash{}s'\sim{}\sigma'$} (solved with \nameref{lem:step}).
    \item \framebox{$\gamma\vdash{}\theta_s\sim{}\theta_\sigma$} (solved with
      \nameref{lem:step} and IH).
    \end{enumerate}
    
  \end{itemize}
\end{proof}

%%%%%%%%%%%%%%%%%%%%%%%%%%%%%%%%
%%%%%%%%%% STEP LEMMA %%%%%%%%%%
%%%%%%%%%%%%%%%%%%%%%%%%%%%%%%%%

\begin{lemma}[Step]
  \label{lem:step}
  For all $sitpn$, $d$, $\gamma$, $E_p$, $E_c$, $\tau$, $s$,
  $s''$, $\sigma$, $\sigma''$, $\Delta$, $\sigma_e$, assume that:
  \begin{itemize}
  \item $\lfloor{}sitpn\rfloor_\mathcal{H}=(d,\gamma)$ and
    $E_p\stackrel{env}{=}E_c$ and
    $\mathcal{D}_\mathcal{H},\emptyset\vdash{}d\srarrow{elab}{\fontsize{7}{9}\selectfont}\Delta,\sigma_e$
  \item $\gamma\vdash{}s\stackrel{\downarrow}{\sim}\sigma$ 
  \item From state $s$ to $s''$ in one execution cycle:
    $E_c,\tau\vdash{}sitpn,s\srarrow{\uparrow,\downarrow}{\fontsize{7}{9}\selectfont}s''$
  \item From state $\sigma$ to $\sigma''$ in one simulation cycle:
    $E_p,\Delta,\tau,\sigma\vdash{}d.cs\srarrow{\uparrow,\downarrow}{\fontsize{7}{9}\selectfont}\sigma''$
  \end{itemize}
  then $\gamma\vdash{}s''\stackrel{\downarrow}{\sim}{}\sigma''$.
\end{lemma}

\begin{proof}
  By def. of the SITPN and \hvhdl{} cycle relations:
  \begin{itemize}
  \item $E_c,\tau\vdash{}sitpn,s\xrightarrow{\uparrow}s'$ and
    $E_c,\tau\vdash{}sitpn,s'\xrightarrow{\downarrow}s''$
  \item
    $\mathtt{Inject}_\uparrow(\sigma, E_p, \tau, \sigma_{injr})$
    and
    $\Delta,\sigma_{injr}\vdash\mathrm{d.cs}\xrightarrow{\uparrow}\sigma_r$
    and
    $\Delta,\sigma_r\vdash\mathrm{d.cs}\xrightarrow{\theta}\sigma'$
  \item
    $\mathtt{Inject}_\downarrow(\sigma', E_p, \tau, \sigma_{injf})$
    and
    $\Delta,\sigma_{injf}\vdash\mathrm{d.cs}\xrightarrow{\downarrow}\sigma_f$
    and
    $\Delta,\sigma_f\vdash\mathrm{d.cs}\xrightarrow{\theta'}\sigma''$
  \end{itemize}
  Solved by applying \nameref{lem:re} and then ``Falling Edge'' lemmas.
\end{proof}

%%% Local Variables:
%%% mode: latex
%%% TeX-master: "../../main"
%%% End:
