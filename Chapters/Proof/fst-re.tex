\begin{definition}[First Rising Edge Hypotheses]
  \label{def:fst-re-hyps}
  Given an
  $sitpn\in{}SITPN,d\in{}design,\gamma\in{}WM(sitpn,d),
  \Delta\in{}ElDesign(d,\mathcal{D}_\mathcal{H}),$
  $\sigma_{e},\sigma_0,\sigma_i,\sigma_{\uparrow},\sigma\in{}\Sigma(\Delta)$,
  $E_c\in{}\mathbb{N}\rightarrow{}\mathcal{C}\rightarrow{}\mathbb{B}$,
  $E_p\in{}(\mathbb{N}\times{}\{\uparrow,\downarrow\})\rightarrow{}Ins(\Delta)\rightarrow{}value$,
  $\tau\in\mathbb{N}$, assume that:
  \begin{itemize}
  \item $\lfloor{}sitpn\rfloor_\mathcal{H}=(d,\gamma)$ and
    $\mathcal{D}_\mathcal{H},\emptyset\vdash{}d\srarrow{elab}{\fontsize{6}{8}\selectfont}(\Delta,\sigma_{e})$
    and $\gamma\vdash{}E_p\stackrel{env}{=}E_c$
  \item $\sigma_0$ is the initial state of $\Delta$: 
    $\Delta,\sigma_{e}\vdash{}d.cs\srarrow{init}{\fontsize{6}{8}\selectfont}\sigma_0$
  \item $E_c,\tau\vdash{}s_0\srarrow{\uparrow_0}{\fontsize{6}{8}\selectfont}s_0$
  \item $\mathtt{Inject}_\uparrow(\sigma_0, E_p, \tau, \sigma_i)$
    and
    $\Delta,\sigma_i\vdash\mathrm{d.cs}\xrightarrow{\uparrow}\sigma_{\uparrow}$
    and
    $\Delta,\sigma_{\uparrow}\vdash\mathrm{d.cs}\xrightarrow{\theta}\sigma$
  \end{itemize}
  
\end{definition}

\def\fstrehyps{For all $sitpn,d,\gamma,\Delta,$
  $\sigma_{e},\sigma_0,\sigma_i,\sigma_{\uparrow},\sigma$, $E_c$, $E_p$,
  $\tau$ that verify the hypotheses of Def.~\ref{def:fst-re-hyps},}

%%%%%%%%%%%%%%%%%%%%%%%%%%%%%%%%%%%%%%%%%%%%%
%%%%%%%%%% FIRST RISING EDGE LEMMA %%%%%%%%%%
%%%%%%%%%%%%%%%%%%%%%%%%%%%%%%%%%%%%%%%%%%%%%

\begin{lemma}[First Rising Edge]
  \label{lem:fst-re}
  \fstrehyps{} then
  $\gamma,E_c,\tau\vdash{}s_0\stackrel{\uparrow}{\sim}{}\sigma$.
\end{lemma}

\begin{proof}
  By definition of \nameref{def:post-re-state-sim}, 6 subgoals.
  \begin{frameb}
    \begin{enumerate}
    \item
      $\forall{}p\in{}P,id_p\in{}Comps(\Delta),\sigma_p\in\Sigma(\Delta(id_p))~s.t.~\gamma(p)=id_p$
      and $\sigma(id_p)=\sigma_p,$
      $~s_0.M(p)=\sigma_p("s\_marking")$.\label{it:fst-re-marking-eq}
    \item
      $\forall{}t\in{}T_i,id_t\in{}Comps(\Delta),\sigma_t\in\Sigma(\Delta(id_t))~s.t.~\gamma(t)=id_t$ and $\sigma(id_t)=\sigma_t,$\\
      $upper(I_s(t))=\infty\land{}s_0.I(t)\le{}lower(I_s(t))\Rightarrow{}s_0.I(t)=\sigma_t("s\_tc")\land{}$\\
      $upper(I_s(t))=\infty\land{}s_0.I(t)>{}lower(I_s(t))\Rightarrow{}\sigma_t("s\_tc")=lower(I_s(t))\land{}$\\
      $upper(I_s(t))\neq\infty\land{}s_0.I(t)>{}upper(I_s(t))\Rightarrow{}\sigma_t("s\_tc")=upper(I_s(t))\land{}$\\
      $upper(I_s(t))\neq\infty\land{}s_0.I(t)\le{}upper(I_s(t))\Rightarrow{}s_0.I(t)=\sigma_t("s\_tc")$.\label{it:fst-re-tc-eq}
    \item
      $\forall{}t\in{}T_i,id_t\in{}Comps(\Delta),\sigma_t\in\Sigma(\Delta(id_t))~s.t.~\gamma(t)=id_t$
      and
      $\sigma(id_t)=\sigma_t,~$\\
      $s_0.reset_t(t)=\sigma_t("s\_reinit\_time\_counter")$.\label{it:fst-re-reset-eq}
    \item
      $\forall{}a\in\mathcal{A},id_a\in{}Outs(\Delta)~s.t.~\gamma(a)=id_a,~s_0.ex(a)=\sigma(id_a)$.\label{it:fst-re-action-eq}
    \item
      $\forall{}f\in\mathcal{F},id_f\in{}Outs(\Delta)~s.t.~\gamma(f)=id_f,~s_0.ex(f)=\sigma(id_f)$.\label{it:fst-re-fun-eq}
    \item
      $\forall{}t\in{}T_i,id_t\in{}Comps(\Delta)~s.t.~\gamma(t)=id_t$,\\
      $t\in{}Sens(s.M)\Leftrightarrow\sigma(id_t)("s\_enabled")=\mathtt{true}$.\label{it:fst-re-sens-eq}
    \item
      $\forall{}t\in{}T,id_t\in{}Comps(\Delta)~s.t.~\gamma(t)=id_t,$\\
      $\sigma(id_t)("s\_condition\_combination")=
      \prod\limits_{c\in{}conds(t)}
      \begin{cases}
        E_c(\tau,c) & if~\mathbb{C}(t,c)=1 \\
        \mathtt{not}(E_c(\tau,c)) & if~\mathbb{C}(t,c)=-1 \\
      \end{cases}$\\
      where
      $conds(t)=\{c\in\mathcal{C}~\vert~\mathbb{C}(t,c)=1\lor\mathbb{C}(t,c)=-1\}$.\label{it:fst-re-cond-comb-eq}
    \end{enumerate}
  \end{frameb}

  \begin{itemize}[label=--]
  \item Apply Lemma~\nameref{lem:fst-re-equal-marking} to solve \ref{it:fst-re-marking-eq}.
  \item Apply Lemma~\nameref{lem:fst-re-equal-tc} to solve \ref{it:fst-re-tc-eq}.
  \item Apply Lemma~\nameref{lem:fst-re-equal-reset-orders} to solve \ref{it:fst-re-reset-eq}.
  \item Apply Lemma ``First Rising Edge Equal Action Executions'' to solve \ref{it:fst-re-action-eq}.
  \item Apply Lemma ``First Rising Edge Equal Function Executions '' to solve \ref{it:fst-re-fun-eq}.
  \item Apply Lemma ``Rising Edge Equal Sensitized'' to solve \ref{it:fst-re-sens-eq}.
  \item Apply Lemma ``Rising Edge Equal Condition Combination'' to solve \ref{it:fst-re-cond-comb-eq}.
  \end{itemize}
  
\end{proof}

\subsection{First rising edge and marking}
\label{sec:fst-re-marking}

\begin{lemma}[First Rising Edge Equal Marking]
  \label{lem:fst-re-equal-marking}
  \fstrehyps{} then
  $\forall{}p\in{}P,id_p\in{}Comps(\Delta),\sigma_p\in\Sigma(\Delta(id_p))~s.t.~\gamma(p)=id_p$
  and $\sigma(id_p)=\sigma_p,$ $~s_0.M(p)=\sigma_p("s\_marking")$.
\end{lemma}

\begin{proof}
  Given a $p$, $id_p$, $\sigma_p$ s.t. $\gamma(p)=id_p$ and
  $\sigma(id_p)=\sigma_p$, let us show that
  \fbox{$~s_0.M(p)=\sigma_p("s\_marking")$.}
  
  \noindent{}By definition of $id_p$, there exist
  $gm_p,ipm_p,opm_p~s.t.~\mathtt{comp}(id_p,"place",gm_p,ipm_p,opm_p)\in{}d.cs$.\\
  
  \noindent By property of the \hvhdl{} elaboration relation, the
  \hvhdl{} initialization relation, the $\mathtt{Inject}_\uparrow$
  relation, the \hvhdl{} rising edge relation and
  $\mathtt{comp}(id_p,"place",gm_p,ipm_p,opm_p)\in{}d.cs$, there exist
  a
  $\sigma_p^{e},\sigma_p^{0},\sigma_p^{injr},\sigma_p^{r}\in{}\Sigma(\Delta)$
  s.t.  $\sigma_{e}(id_p)=\sigma_p^{e}$ and
  $\sigma_{0}(id_p)=\sigma_p^{0}$ and
  $\sigma_i(id_p)=\sigma_p^{injr}$ and
  $\sigma_{r}(id_p)=\sigma_p^{r}$ .

  \begin{pcomm}
    From the elaboration to the end of the first rising edge phase, an
    internal state is associated with the P component instance
    $id_p$ in the component store of the top-level design $d$.
  \end{pcomm}
  
  \noindent{} By property of the \hvhdl{} rising edge relation, the P design behavior (process ``\texttt{marking}''), and\\
  $\mathtt{comp}(id_p,"place",gm_p,ipm_p,opm_p)\in{}d.cs$, then\\
  $\sigma_p^{r}("s\_marking")=\sigma_p^{injr}("s\_marking")+\sigma_p^{injr}("s\_input\_token\_sum")-\sigma_p^{injr}("s\_output\_token\_sum")$.

  \begin{pcomm}
    Result of the execution of the process ``\texttt{marking}'' that performs the signal assignment\\
    $\mathtt{s\_marking\Leftarrow{}s\_marking+s\_input\_token\_sum-s\_output\_token\_sum}$.
  \end{pcomm}

  \noindent{} By property of the \hvhdl{} stabilize relation, the P design behavior (process ``\texttt{marking}''), and\\
  $\mathtt{comp}(id_p,"place",gm_p,ipm_p,opm_p)\in{}d.cs$, then
  $\sigma_p^{r}("s\_marking")=\sigma_p("s\_marking")$.

  \begin{pcomm}
    As it is only assigned by the process ``\texttt{marking}'', and as
    the process ``\texttt{marking}'' is never executed during the
    stabilization phase, the ``\texttt{s\_marking}'' signal has an
    invariant value during the stabilization phase.
  \end{pcomm}

  \noindent{} Rewriting $\sigma_p("s\_marking")$ as
  $\sigma_p^{r}("s\_marking")$,
  and $\sigma_p^{r}("s\_marking")$ as\\
  $\sigma_p^{injr}("s\_marking")+\sigma_p^{injr}("s\_input\_token\_sum")-\sigma_p^{injr}("s\_output\_token\_sum")$,\\
  \fbox{$s_0.M(p)=\sigma_p^{injr}("s\_marking")+\sigma_p^{injr}("s\_input\_token\_sum")-\sigma_p^{injr}("s\_output\_token\_sum")$.}\\

  \noindent{} By property of the $\mathtt{Inject}_\uparrow$ relation,
  $\sigma_p^{injr}("s\_marking")=\sigma_p^{0}("s\_marking")$ and\\
  $\sigma_p^{injr}("s\_input\_token\_sum")=\sigma_p^{0}("s\_input\_token\_sum")$
  and\\
  $\sigma_p^{injr}("s\_output\_token\_sum")=\sigma_p^{0}("s\_output\_token\_sum")$.
  Rewriting the above,\\
  \fbox{$s_0.M(p)=\sigma_p^{0}("s\_marking")+\sigma_p^{0}("s\_input\_token\_sum")-\sigma_p^{0}("s\_output\_token\_sum")$.}\\

  \begin{todobox}
    Detail the two lemmas giving this property.
  \end{todobox}
  \noindent{}By property of the \hvhdl{} initialization relation,
  $\sigma_p^{0}("s\_input\_token\_sum")=0$ and\\
  $\sigma_p^{0}("s\_output\_token\_sum")=0$. Rewriting the above,
  \fbox{$s_0.M(p)=\sigma_p^{0}("s\_marking")$.}\\

  \noindent{} Applying the \nameref{lem:init-states-eq-marking} lemma,
  \qedbox{$s_0.M(p)=\sigma_p^{0}("s\_marking")$.}
\end{proof}

\subsection{First rising edge and time counters}
\label{sec:fst-re-tc}

\begin{lemma}[First Rising Edge Equal Time Counters]
  \label{lem:fst-re-equal-tc}
  \fstrehyps{} then\\
  $\forall{}t\in{}T_i,id_t\in{}Comps(\Delta),\sigma_t\in\Sigma(\Delta(id_t))~s.t.~\gamma(t)=id_t$ and $\sigma(id_t)=\sigma_t,$\\
  $upper(I_s(t))=\infty\land{}s_0.I(t)\le{}lower(I_s(t))\Rightarrow{}s_0.I(t)=\sigma_t("s\_tc")\land{}$\\
  $upper(I_s(t))=\infty\land{}s_0.I(t)>{}lower(I_s(t))\Rightarrow{}\sigma_t("s\_tc")=lower(I_s(t))\land{}$\\
  $upper(I_s(t))\neq\infty\land{}s_0.I(t)>{}upper(I_s(t))\Rightarrow{}\sigma_t("s\_tc")=upper(I_s(t))\land{}$\\
  $upper(I_s(t))\neq\infty\land{}s_0.I(t)\le{}upper(I_s(t))\Rightarrow{}s_0.I(t)=\sigma_t("s\_tc")$.
\end{lemma}

\begin{proof}
  \noindent{}Given a $t\in{}T_i$, an $id_t\in{}Comps(\Delta)$ and a
  $\sigma_t\in\Sigma(\Delta(id_t))~s.t.~\gamma(t)=id_t$ and
  $\sigma(id_t)=\sigma_t$, let's show that:
  \begin{enumerate}
  \item \framebox{$upper(I_s(t))=\infty\land{}s_0.I(t)\le{}lower(I_s(t))\Rightarrow{}s_0.I(t)=\sigma_t("s\_tc")$}
  \item \framebox{$upper(I_s(t))=\infty\land{}s_0.I(t)>{}lower(I_s(t))\Rightarrow{}\sigma_t("s\_tc")=lower(I_s(t))$}
  \item \framebox{$upper(I_s(t))\neq\infty\land{}s_0.I(t)>{}upper(I_s(t))\Rightarrow{}\sigma_t("s\_tc")=upper(I_s(t))$}
  \item \framebox{$upper(I_s(t))\neq\infty\land{}s_0.I(t)\le{}upper(I_s(t))\Rightarrow{}s_0.I(t)=\sigma_t("s\_tc")$}
  \end{enumerate}

  \noindent{}By definition of $id_t$, there exist $gm_t,ipm_t,opm_t$
  s.t.
  $\mathtt{comp}(id_t,"transition",gm_t,ipm_t,opm_t)\in{}d.cs$.\\

  \noindent By property of the \hvhdl{} elaboration relation, the
  \hvhdl{} initialization relation, the $\mathtt{Inject}_\uparrow$
  relation, the \hvhdl{} rising edge relation and
  $\mathtt{comp}(id_t,"transition",gm_t,ipm_t,opm_t)\in{}d.cs$, there
  exist a
  $\sigma_t^{e},\sigma_t^{0},\sigma_t^{injr},\sigma_t^{r}\in{}\Sigma(\Delta)$
  s.t.  $\sigma_{e}(id_t)=\sigma_t^{e}$ and
  $\sigma_{0}(id_t)=\sigma_t^{0}$ and
  $\sigma_i(id_t)=\sigma_t^{injr}$ and
  $\sigma_{r}(id_t)=\sigma_t^{r}$ .

  \begin{pcomm}
    From the elaboration to the end of the first rising edge phase, an
    internal state is associated with the T component instance $id_t$
    in the component store of the top-level design $d$.
  \end{pcomm}
  
  \noindent{}Then, let's show the 4 previous subgoals.
  
  \begin{enumerate}
  \item Assume $upper(I_s(t))=\infty\land{}s_0.I(t)\le{}lower(I_s(t))$, then show \framebox{${}s_0.I(t)=\sigma_t("s\_tc")$.}\\
    \noindent{} By property of the $\mathtt{Inject_\uparrow}$
    relation, the \hvhdl{} rising edge and stabilize relations, and\\
    $\mathtt{comp}(id_t,"transition",gm_t,ipm_t,opm_t)\in{}d.cs$,
    $\sigma_t("s\_tc")=\sigma_t^0("s\_tc")$.
    \begin{pcomm}
      The above equality is deduced from the two following facts:
      
      \begin{itemize}
      \item The process ``\texttt{time\_counter}'' is the only process
        that assigns signal \texttt{s\_tc} in the T component
        behavior, and it is never executed during the rising edge and
        stabilization phases.
        
      \item The values of component instances' internal signals are
        invariant through the $\mathtt{Inject_\uparrow}$ relation.
      \end{itemize}
    \end{pcomm}

  \noindent{} Rewriting $\sigma_t("s\_tc")$ as $\sigma_t^0("s\_tc")$,
  \fbox{${}s_0.I(t)=\sigma_t^0("s\_tc")$.}\\

  Applying the \nameref{lem:init-states-eq-tc} lemma,
  \qedbox{${}s_0.I(t)=\sigma_t^0("s\_tc")$.}
  
  \item Assume $upper(I_s(t))=\infty\land{}s_0.I(t)>{}lower(I_s(t))$,
    then show \framebox{$\sigma_t("s\_tc")=lower(I_s(t))$}.  By
    definition, $lower(I_s(t))\in\mathbb{N}^{*}$ and
    $s_0.I(t)=0$. Then, \colorbox{red!20}{$lower(I_s(t)){}<0$ is a
      contradiction.}
  \item Assume
    $upper(I_s(t))\neq\infty\land{}s_0.I(t)>{}upper(I_s(t))$, then
    show \framebox{$\sigma_t("s\_tc")=upper(I_s(t))$}.  By definition,
    $upper(I_s(t))\in\mathbb{N}^{*}$ and $s_0.I(t)=0$. Then,
    \colorbox{red!20}{$upper(I_s(t)){}<0$ is a contradiction.}
  \item Assume
    $upper(I_s(t))\neq\infty\land{}s_0.I(t)\le{}upper(I_s(t))$, then
    show \framebox{$s_0.I(t)=\sigma_t("s\_tc")$}.\\

    \noindent{} By property of the $\mathtt{Inject_\uparrow}$
    relation, the \hvhdl{} rising edge and stabilize relations, and\\
    $\mathtt{comp}(id_t,"transition",gm_t,ipm_t,opm_t)\in{}d.cs$,
    $\sigma_t("s\_tc")=\sigma_t^0("s\_tc")$.\\

    \noindent{} Rewriting $\sigma_t("s\_tc")$ as $\sigma_t^0("s\_tc")$,
    \fbox{${}s_0.I(t)=\sigma_t^0("s\_tc")$.}\\

    Applying the \nameref{lem:init-states-eq-tc} lemma,
    \qedbox{${}s_0.I(t)=\sigma_t^0("s\_tc")$.}
  \end{enumerate}
\end{proof}

\subsection{First rising edge and reset orders}
\label{sec:fst-re-reset-orders}

\begin{lemma}[First Rising Edge Equal Reset Orders]
  \label{lem:fst-re-equal-reset-orders}
  \fstrehyps{} then\\
  $\forall{}t\in{}T,id_t\in{}Comps(\Delta)~s.t.~\gamma(t)=id_t,$\\
  $s_0.reset_t(t)=\sigma(id_t)("s\_reinit\_time\_counter")$.
\end{lemma}

\begin{table}[h]
  \begin{tabular}{|c|c|c|c|}
    \hline
    \multicolumn{4}{|c|}{\textbf{Constants and signals reference}} \\
    \hline
    \textit{Full name} & \textit{Alias} & \textit{Category} & \textit{Type} \\
    \hline
    $"s\_reinit\_time\_counter"$ & $"srtc"$ & internal signal (T) & $\mathbb{B}$ \\
    \hline
    $"input\_arcs\_number"$ & $"ian"$ & generic constant (T) & $\mathbb{N}$ \\
    \hline
    $"reinit\_time"$ & $"rt"$ & input port (T) & $\mathbb{B}$ \\
    \hline
    $"reinit\_transition\_time"$ & $"rtt"$ & output port (P) & $\mathbb{B}$ \\
    \hline
    $"output\_arcs\_types"$ & $"oat"$ & input port (P) & $\{\mathtt{BASIC},\mathtt{TEST},\mathtt{INHIB}\}$ \\
    \hline
    $"s\_marking"$ & $"sm"$ & internal signal (P) & $\mathbb{N}$ \\
    \hline
    $"s\_output\_token\_sum"$ & $"sots"$ & internal signal (P) & $\mathbb{N}$ \\
    \hline
    $"output\_arcs\_weights"$ & $"oaw"$ & input port (P) & $\mathbb{N}$ \\
    \hline
    $"output\_transition\_fired"$ & $"otf"$ & input port (P) & $\mathbb{B}$ \\
    \hline
  \end{tabular}
\end{table}

\begin{proof}
  Given a $t\in{}T$ and an $id_t\in{}Comps(\Delta)$
  s.t. $\gamma(t)=id_t$, let us show that
  \fbox{$s_0.reset_t(t)=\sigma(id_t)("srtc")$.}

  \exT

  \noindent{}By property of the \hvhdl{} stabilize relation and
  \InCsCompT{}, then
  $\sigma(id_t)("srtc")=\sum\limits_{i=0}^{\Delta(id_t)("input\_arcs\_number")-1}\sigma(id_t)("reinit\_time")[i]$.

  \noindent{}\fbox{$s_0.reset_t(t)=\sum\limits_{i=0}^{\Delta(id_t)("ian")-1}\sigma(id_t)("rt")[i]$.}\\
  
  \noindent{}Case analysis on $input(t)$ (2 CASES):

  \begin{itemize}
  \item \textbf{CASE} $input(t)=\emptyset$:\\

    By construction,
    ${<}\mathtt{input\_arcs\_number\Rightarrow{}1}{>}\in{}gm_t$, and
    by property of the \hvhdl{} elaboration relation, then
    $\Delta(id_t)("ian")=1$.  By construction,
    $<\mathtt{reinit\_time(0)\Rightarrow{}false}>\in{}ipm_t$,
    and by property of the \hvhdl{} stabilize relation, $\sigma(id_t)("rt")[0]=false$.\\

    \noindent{}Rewriting $\Delta(id_t)("ian")$ as $1$ and
    $\sigma(id_t)("rt")[0]$ as $false$, and by definition of $s_0$,
    \qedbox{$s_0.reset_t(t)=\sum\limits_{i=0}^{\Delta("ian")-1}\sigma(id_t)("rt")[i]=\sigma(id_t)("rt")[0]=false$.}
    
  \item \textbf{CASE} $input(t)\neq{}\emptyset$:\\

    By construction,
    ${<}\mathtt{input\_arcs\_number\Rightarrow{}}\vert{}input(t)\vert{>}\in{}gm_t$, and
    by property of the \hvhdl{} elaboration relation, then
    $\Delta(id_t)("ian")=\vert{}input(t)\vert$.

    Rewriting $\Delta(id_t)("ian")$ as $\vert{}input(t)\vert$,
    \fbox{$s_0.reset_t(t)=\sum\limits_{i=0}^{\vert{}input(t)\vert-1}\sigma(id_t)("rt")[i]$.}

    By definition of $s_0$, $s_0.reset_t(t)=false$. Rewriting
    $s_0.reset_t(t)$ as $false$,\\
    \fbox{$\sum\limits_{i=0}^{\vert{}input(t)\vert-1}\sigma(id_t)("rt")[i]=false$.}

    Given a $i\in[0,\vert{}input(t)\vert-1]$, let us show
    \fbox{$\sigma(id_t)("rt")[i]=false$.}
    
    By construction, and $input(t)\neq{}\emptyset$, there exist
    ${}p\in{}input(t)$ and $id_p\in{}Comps(\Delta)$
    s.t. $\gamma(p)=id_p$.

    \exP By construction for all $i\in{}[0,\vert{}input(t)\vert-1],$
    there exist $j\in[0,\vert{}output(p)\vert-1]$ and
    $id_{ji}\in{}Sigs(\Delta)$ s.t.
    ${<}\mathtt{reinit\_transition\_time(j)\Rightarrow{}id_{ji}}{>}\in{}opm_p$
    and ${<}\mathtt{reinit\_time(i)\Rightarrow{}id_{ji}}{>}\in{}ipm_t$.

    \noindent{}By property of the \hvhdl{} stabilize relation,
    ${<}\mathtt{reinit\_transition\_time(j)\Rightarrow{}id_{ji}}{>}\in{}opm_p$
    and
    ${<}\mathtt{reinit\_time(i)\Rightarrow{}id_{ji}}{>}\in{}ipm_t$,
    then $\sigma(id_t)("rt")[i]=\sigma(id_{ji})=\sigma(id_p)("rtt")[j]$.

    Rewriting $\sigma(id_t)("rt")[i]$ as $\sigma(id_{ji})$ and
    $\sigma(id_{ji})$ as $\sigma(id_p)("rtt")[j]$,
    \fbox{$\sigma(id_p)("rtt")[j]=false$.}

    \noindent{}By property of the \hvhdl{} rising edge and stabilize
    relations,\\
    \begin{equation*}
      \begin{split}
        \sigma(id_p)("rtt")[j]=& ((\sigma_0(id_p)("oat")[j]=\mathtt{BASIC}+\sigma_0(id_p)("oat")[j]=\mathtt{TEST}) \\
        & .(\sigma_0(id_p)("sm")-\sigma_0(id_p)("sots")<\sigma_0(id_p)("oaw")[j])\\
        & .(\sigma_0(id_p)("sots")>0))\\
        & +(\sigma_0(id_p)("otf")[j]) \\
      \end{split}
    \end{equation*}

    Rewriting the goal with the above equation,
    \begin{equation*}
      \fbox{$\begin{split}
          false=& ((\sigma_0(id_p)("oat")[j]=\mathtt{BASIC}+\sigma_0(id_p)("oat")[j]=\mathtt{TEST}) \\
          & .(\sigma_0(id_p)("sm")-\sigma_0(id_p)("sots")<\sigma_0(id_p)("oaw")[j])\\
          & .(\sigma_0(id_p)("sots")>0))\\
          & +(\sigma_0(id_p)("otf")[j]) \\
        \end{split}$}
    \end{equation*}
    
    \begin{todobox}
      Add a lemma + proof in section initial states for fired = false
      after initialization.
    \end{todobox}

    \noindent{}By property of the \hvhdl{} initialization and the
    $\mathtt{Inject}_{\uparrow}$ relations, then
    $\sigma_0(id_p)("otf")[j]=false$. Rewriting
    $\sigma_0(id_p)("otf")[j]$ as $false$ and simplifying the goal,
    \begin{equation*}
      \fbox{$\begin{split}
          false=& ((\sigma_0(id_p)("oat")[j]=\mathtt{BASIC}+\sigma_0(id_p)("oat")[j]=\mathtt{TEST}) \\
          & .(\sigma_0(id_p)("sm")-\sigma_0(id_p)("sots")<\sigma_0(id_p)("oaw")[j])\\
          & .(\sigma_0(id_p)("sots")>0))\\
        \end{split}$}
    \end{equation*}

    \begin{todobox}
      Add a lemma + proof in section initial states for output token
      sum = 0 after initialization.
    \end{todobox}
    
    \noindent{}By property of the \hvhdl{} initialization and the
    $\mathtt{Inject}_{\uparrow}$ relations, then
    $\sigma_0(id_p)("sots")=0$. Rewriting
    $\sigma_0(id_p)("sots")$ as $0$ and simplifying the goal,
    \qedbox{$false=false$}
    
  \end{itemize}
\end{proof}

\subsection{First rising edge and action executions}
\label{sec:fst-re-actions-ex}

\begin{lemma}[First Rising Edge Equal Action Executions]
  \label{lem:fst-re-equal-action-ex}
  \fstrehyps{} then\\
  $\forall{}a\in\mathcal{A},id_a\in{}Outs(\Delta)~s.t.~\gamma(a)=id_a,~s_0.ex(a)=\sigma(id_a)$.\\
\end{lemma}

\begin{proof}
  Given an $a\in\mathcal{A}$ and an
  $id_a\in{}Outs(\Delta)~s.t.~\gamma(a)=id_a$, let us show that
  \fbox{$s_0.ex(a)=\sigma(id_a)$.}\\

  \noindent{}Rewriting $s_0.ex(a)$ as $false$, by definition of $s_0$,
  \fbox{$\sigma(id_a)=false$.}

  \noindent{}By construction, $id_a$ is an output port identifier of
  boolean type in the \hvhdl{} design $d$ assigned only during a
  falling edge phase in the \texttt{``action''} process.

  \noindent{}By property of the \hvhdl{} $\mathtt{Inject_{\uparrow}}$,
  rising edge and stabilize relations, then
  $\sigma(id_a)=\sigma_0(id_a)$.
  
  \noindent{}Thanks to the Lemma \nameref{lem:init-states-act-exec},
  $\sigma_0(id_a)=false$.

  \noindent{}Rewriting $\sigma(id_a)$ as $\sigma_0(id_a)$, and
  $\sigma_0(id_a)$ as $false$, \qedbox{$false=false$.}
  
\end{proof}

\subsection{First rising edge and function executions}
\label{sec:fst-re-fun-ex}

\begin{lemma}[First Rising Edge Equal Function Executions]
  \label{lem:fst-re-equal-fun-ex}
  \fstrehyps{} then\\
  $\forall{}f\in\mathcal{F},id_f\in{}Outs(\Delta)~s.t.~\gamma(f)=id_f,~s_0.ex(f)=\sigma(id_f)$.
\end{lemma}

\begin{proof}
  Given an $f\in\mathcal{F}$ and an $id_f\in{}Outs(\Delta)$
  s.t. $\gamma(f)=id_f$, let us show that
  \fbox{$s_0.ex(f)=\sigma(id_f)$.}

  \noindent{}Rewriting $s_0.ex(f)$ as $false$, by definition of $s_0$,
  \fbox{$\sigma(id_f)=false$.}

  \noindent{}By construction, the \texttt{``function''} process is a part of
  design $d$'s behavior, i.e\\
  $\mathtt{ps}("function", \emptyset, sl, ss)\in{}d.cs$.
  
  \noindent{}By construction $id_f$ is an output port of design $d$,
  and it is only assigned in the body of the \texttt{``function''}
  process. Let $trs(f)$ be the set of transitions associated to
  function $f$, i.e
  $trs(f)=\{t\in{}T~\vert~\mathbb{F}(t,f)=true\}$. Then, depending on
  $trs(f)$, there are two cases of assignment of output port $id_f$:
  
  \begin{itemize}
  \item \textbf{CASE} $trs(f)=\emptyset$:\\
    \noindent{}By construction,
    $\mathtt{id_f\Leftarrow{}false}\in{}ss_{\uparrow}$ where
    $ss_\uparrow$ is the part of the \texttt{``function''} process
    body executed during the rising edge phase.

    \noindent{}By property of the \hvhdl{} rising edge and the
    stabilize relation, then \\ \qedbox{$\sigma(id_f)=false$.}
  \item \textbf{CASE} $trs(f)\neq\emptyset$:\\
    \noindent{}By construction,
    $\mathtt{id_f\Leftarrow{}id_{ft_0}+\dots+id_{ft_n}}\in{}ss_\uparrow$
    where $ss_\uparrow$ is the part of the \texttt{``function''}
    process body executed during the rising edge phase, and
    $n=\vert{}trs(f)\vert-1$, and for all $i\in[0,n-1]$, $id_{ft_i}$
    is a internal signal of design $d$.

    \noindent{}By property of the $\mathtt{Inject}_\uparrow$, the
    \hvhdl{} rising edge and stabilize relation, then
    $\sigma(id_f)=\sigma_0(id_{ft_0})+\dots+\sigma_0(id_{ft_n})$.

    \noindent{}Rewriting $\sigma(id_f)$ as
    $\sigma_0(id_{ft_0})+\dots+\sigma_0(id_{ft_n})$, then\\
    \fbox{$\sigma_0(id_{ft_0})+\dots+\sigma_0(id_{ft_n})=false$.}

    \noindent{}By construction, for all $id_{ft_i}$, there exist a
    $t_i\in{}trs(f)$ and an $id_{t_i}$ s.t. $\gamma(t_i)=id_{t_i}$.

    By definition of $id_{t_i}$, there exist $gm_{t_i}$, $ipm_{t_i}$
    and $opm_{t_i}$ s.t.\\
    $\mathtt{comp}(id_{t_i}, "transition", gm_{t_i}, ipm_{t_i},
    opm_{t_i})\in{}d.cs$.

    By construction,
    ${<}\mathtt{fired\Rightarrow{id_{ft_i}}}{>}\in{}opm_{t_i}$, and by
    property of the initialization relation
    $\sigma_0(id_{ft_i})=\sigma_0(id_{t_i})("fired")$.

    \noindent{}Rewriting $\sigma_0(id_{ft_i})$ as
    $\sigma_0(id_{t_i})("fired")$, then \\
    \fbox{$\sigma_0(id_{t_0})("fired")+\dots+\sigma_0(id_{t_n})("fired")=false$.}

    \noindent{}By property of the initialization relation, we know
    that for all $t\in{}T$ and $id_t\in{}Comps(\Delta)$
    s.t. $\gamma(t)=id_t$, then $\sigma_0(id_t)("fired")=false$.

    Rewriting all $\sigma_0(id_{t_i})("fired")$ as $false$ and
    simplifying the goal, then\\ \qedbox{$false=false$.}
    
  \end{itemize}

  
\end{proof}

%%% Local Variables:
%%% mode: latex
%%% TeX-master: "../../main"
%%% End:
