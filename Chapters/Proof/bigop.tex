\documentclass{article}

\usepackage{amsfonts,amssymb,amsmath,amsthm}
\usepackage{stackengine}

\newtheorem{definition}{Definition}
\newtheorem{notation}{Notation}
\newtheorem{thm}{Theorem}
\newtheorem{remark}{Remark}
\newtheorem{hypothesis}{Hypothesis}
\newtheorem{lemma}{Lemma}

\usepackage{xcolor}

\newcommand\qedbox[1]{\colorbox{red!20}{#1}}

\begin{document}

\begin{definition}[Big Operator]
  Given a triplet ${<}A,*,e{>}$ such that $A$ is a set,
  $*\in{}A\rightarrow{}A\rightarrow{}A$ is a commutative and
  associative operator over $A$, and $e\in{}A$ is a neutral element of
  $*$, then for all finite set $B$, and application
  $f\in{}B\rightarrow{}A$, a big operator $\Omega$ is recursively
  defined as follows:\\
  $\mathop{\Omega}\limits_{b\in{}B}f(b)=
  \begin{cases}
    e & if~B=\emptyset \\
    f(b)*\mathop{\Omega}\limits_{b'\in{}B\setminus\{b\}}f(b') & otherwise
  \end{cases}
  $
\end{definition}

Then, we can prove the following theorem concerning the equality
between two big operator expressions.

\begin{thm}[Big Operator Equality]
  \label{thm:big-op-eq}
  For all a triplet ${<}A,*,e{>}$ such that $A$ is a set,
  $*\in{}A\rightarrow{}A\rightarrow{}A$ is a commutative and
  associative operator over $A$, and $e\in{}A$ is a neutral element of
  $*$, and for all finite sets $B$ and $C$, and applications
  $f\in{}B\rightarrow{}A$ and $g\in{}C\rightarrow{}A$, assume that:
  \begin{itemize}
  \item there exists an injection $\iota\in{}B\rightarrow{}C$
    s.t. $\forall{}b\in{}B,~f(b)=g(\iota(b))$
  \item $\vert{}B\vert=\vert{}C\vert$
  \end{itemize}
  then
  $\mathop{\Omega}\limits_{b\in{}B}f(b)=\mathop{\Omega}\limits_{c\in{}C}g(c)$.
\end{thm}

\begin{proof}
  Assuming the above premises, let us show
  \fbox{$\mathop{\Omega}\limits_{b\in{}B}f(b)=\mathop{\Omega}\limits_{c\in{}C}g(c)$.}
  
  Let us reason by induction over $\mathop{\Omega}\limits_{b\in{}B}f(b)$:

  \begin{itemize}
  \item \textbf{BASE CASE} $B=\emptyset$:\\
    Then $\vert{}C\vert=\vert{}B\vert=0$, and $C=\emptyset$.
    By definition of $\Omega$:
    \begin{eqnarray}
      \mathop{\Omega}\limits_{b\in{}B}f(b)=e\label{eq:bemp} \\
      \mathop{\Omega}\limits_{c\in{}C}g(c)=e\label{eq:cemp}
    \end{eqnarray}
    Rewriting the goal with \eqref{eq:bemp} and \eqref{eq:cemp},
    \qedbox{tautology}.
  \item \textbf{INDUCTION CASE} $B\neq\emptyset$:

    
    \begin{itemize}
    \item Induction hypothesis:\\
      For all finite set $C'$ verifying:
      \begin{itemize}
      \item $\exists{}$ an injection
        $\iota'\in{}B\setminus\{b\}\rightarrow{}C'~s.t.~$
        $\forall{}b'\in{}B\setminus\{b\},~f(b')=g(\iota(b'))$
      \item $\vert{}B\setminus\{b\}\vert=\vert{}C'\vert$
      \end{itemize}
      then
      $f(b)*\mathop{\Omega}\limits_{b'\in{}B\setminus\{b\}}f(b')=f(b)*\mathop{\Omega}\limits_{c'\in{}C'}g(c)$
    \end{itemize}
    
    The goal is \fbox{$f(b)*\mathop{\Omega}\limits_{b'\in{}B\setminus\{b\}}f(b')=\mathop{\Omega}\limits_{c\in{}C}g(c)$}

    \noindent{}Let us take $\iota\in{}B\rightarrow{}C$
    s.t. $\forall{}b\in{}B,~f(b)=g(\iota(b))$, then:
    \begin{equation}
      f(b)=g(\iota(b))\label{eq:b-iota-b}
    \end{equation}
    \noindent{}Also, by definition of $\Omega$:
    \begin{equation}
      \mathop{\Omega}\limits_{c\in{}C}g(c)=g(\iota(b))*\mathop{\Omega}\limits_{c'\in{}C\setminus\{\iota(b)\}}\label{eq:omega-C}
    \end{equation}
    \noindent{}Rewriting the goal with \eqref{eq:omega-C} and
    \eqref{eq:b-iota-b},\\
    \fbox{$f(b)*\mathop{\Omega}\limits_{b'\in{}B\setminus\{b\}}f(b')=f(b)*\mathop{\Omega}\limits_{c'\in{}C\setminus\{\iota(b)\}}g(c')$}
    
    \noindent{}Let us apply the induction hypothesis with $C'=C\setminus\{\iota(b)\}$; then there are two points to prove:
    \begin{enumerate}
    \item \fbox{$\vert{}B\setminus\{b\}\vert=\vert{}C\setminus\{\iota(b)\}\vert$.} Trivial as $\vert{}B\vert=\vert{}C\vert$.
    \item \fbox{\parbox{\linewidth}{$\exists{}$ an injection
          $\iota'\in{}B\setminus\{b\}\rightarrow{}C\setminus\{\iota(b)\}~s.t.~$
          $\forall{}b'\in{}B\setminus\{b\},f(b')=g(\iota'(b'))$}}
    \end{enumerate}
    Let us define a
    $\iota'\in{}B\setminus\{b\}\rightarrow{}C\setminus\{\iota(b)\}$ as
    follows:
    $\forall{}b'\in{}B\setminus\{b\},~\iota'(b)=\iota(b)$. Let us show
    that this definition is correct by proving that\\
    \fbox{$\forall{}b'\in{}B\setminus\{b\},~\iota(b')\in{}C\setminus\{\iota(b)\}$.}

    \noindent{}Given a $b'\in{}B\setminus\{b\}$, let us show
    \fbox{$\iota(b')\in{}C\setminus\{\iota(b)\}$.}

    \noindent{}By definition of $\iota$, $\iota(b')\in{}C$; then,
    there are 2 cases:
    \begin{itemize}
    \item \textbf{CASE} $\iota(b')=\iota(b)$, then by definition of
      $\iota$ as an injective function: $b'=b$. Then,
      \qedbox{$b\in{}B\setminus\{b\}$ is a contradiction.}
    \item \textbf{CASE} \qedbox{$\iota(b')\in{}C\setminus\{\iota(b)\}$.}
    \end{itemize}

    \noindent{}Now let us get back to the previous goal. Using
    $\iota'$ to prove it, there are 2 points to prove:
    \begin{itemize}
    \item \fbox{$\iota'$ is injective.} Trivial, by definition of $\iota'$.
    \item
      \fbox{$\forall{}b'\in{}B\setminus\{b\},~f(b')=g(\iota'(b'))$.}
      Trivial, by definition of $\iota'$.
    \end{itemize}

  \end{itemize}
\end{proof}

\subsection{Equality between big operator expressions}
\label{sec:eq-bi-op-expr}

Many times in the proceeding of the following proof, the equality
between two sum or product expressions must be estbalished; for
instance:

$\sum\limits_{a\in{}A}f(a)=\sum\limits_{b\in{}B}g(b)$ where $A$ and
$B$ are finite sets, $f\in\mathbb{A}\rightarrow\mathbb{N}$ and
$g\in{}B\rightarrow\mathbb{N}$

To prove such an equality, Theorem~\ref{thm:big-op-eq} is used,
considering that the sum operator used in the above equation is a big
operator over the triplet ${<}\mathbb{N},0,+{>}$. A big operator is
defined as follows:

\begin{definition}[Big Operator]
  Given a triplet ${<}A,*,e{>}$ such that $A$ is a set,
  $*\in{}A\rightarrow{}A\rightarrow{}A$ is a commutative and
  associative operator over $A$, and $e\in{}A$ is a neutral element of
  $*$, then for all finite set $B$, and application
  $f\in{}B\rightarrow{}A$, a big operator $\Omega$ is recursively
  defined as follows: $\mathop{\Omega}\limits_{b\in{}B}f(b)=
  \begin{cases}
    e & if~B=\emptyset \\
    f(b)*\mathop{\Omega}\limits_{b'\in{}B\setminus\{b\}}f(b') & otherwise
  \end{cases}
  $
\end{definition}

Then, we can prove the following theorem concerning the equality
between two big operator expressions.

\begin{thm}[Big Operator Equality]
  \label{thm:big-op-eq}
  For all a triplet ${<}A,*,e{>}$ such that $A$ is a set,
  $*\in{}A\rightarrow{}A\rightarrow{}A$ is a commutative and
  associative operator over $A$, and $e\in{}A$ is a neutral element of
  $*$, and for all finite sets $B$ and $C$, and applications
  $f\in{}B\rightarrow{}A$ and $g\in{}C\rightarrow{}A$, assume that:
  \begin{itemize}
  \item there exists an injection $\iota\in{}B\rightarrow{}C$
    s.t. $\forall{}b\in{}B,~f(b)=g(\iota(b))$
  \item $\vert{}B\vert=\vert{}C\vert$
  \end{itemize}
  then
  $\mathop{\Omega}\limits_{b\in{}B}f(b)=\mathop{\Omega}\limits_{c\in{}C}g(c)$.
\end{thm}

\begin{proof}
  Let us reason by induction over $\mathop{\Omega}\limits_{b\in{}B}f(b)$:

  \begin{itemize}
  \item \textbf{BASE CASE} $B=\emptyset$:\\
    Then $\vert{}C\vert=\vert{}B\vert=0$, and $C=\emptyset$.
    By definition of $\Omega$:
    \begin{eqnarray}
      \mathop{\Omega}\limits_{b\in{}B}f(b)=e\label{eq:bemp} \\
      \mathop{\Omega}\limits_{c\in{}C}g(c)=e\label{eq:cemp}
    \end{eqnarray}
    Rewriting the goal with \eqref{eq:bemp} and \eqref{eq:cemp},
    \qedbox{tautology}.
  \item \textbf{INDUCTION CASE} $B\neq\emptyset$:
    \begin{ih}
      For all finite set $C'$ verifying:
      \begin{itemize}
      \item $\exists{}$ an injection
        $\iota'\in{}B\setminus\{b\}\rightarrow{}C'~s.t.~$
        $\forall{}b'\in{}B\setminus\{b\},~f(b')=g(\iota(b'))$
      \item $\vert{}B\setminus\{b\}\vert=\vert{}C'\vert$
      \end{itemize}
      then $f(b)*\mathop{\Omega}\limits_{b'\in{}B\setminus\{b\}}f(b')=f(b)*\mathop{\Omega}\limits_{c'\in{}C'}g(c)$
    \end{ih}
    
    The goal is \fbox{$f(b)*\mathop{\Omega}\limits_{b'\in{}B\setminus\{b\}}f(b')=\mathop{\Omega}\limits_{c\in{}C}g(c)$}

    \noindent{}Let us take $\iota\in{}B\rightarrow{}C$
    s.t. $\forall{}b\in{}B,~f(b)=g(\iota(b))$, then:
    \begin{equation}
      f(b)=g(\iota(b))\label{eq:b-iota-b}
    \end{equation}
    \noindent{}Also, by definition of $\Omega$:
    \begin{equation}
      \mathop{\Omega}\limits_{c\in{}C}g(c)=g(\iota(b))*\mathop{\Omega}\limits_{c'\in{}C\setminus\{\iota(b)\}}\label{eq:omega-C}
    \end{equation}
    \noindent{}Rewriting the goal with \eqref{eq:omega-C} and
    \eqref{eq:b-iota-b},\\
    \fbox{$f(b)*\mathop{\Omega}\limits_{b'\in{}B\setminus\{b\}}f(b')=f(b)*\mathop{\Omega}\limits_{c'\in{}C\setminus\{\iota(b)\}}g(c')$}
    
    \noindent{}Let us apply the induction hypothesis with $C'=C\setminus\{\iota(b)\}$; then there are two points to prove:
    \begin{enumerate}
    \item \fbox{$\vert{}B\setminus\{b\}\vert=\vert{}C\setminus\{\iota(b)\}\vert$.} Trivial as $\vert{}B\vert=\vert{}C\vert$.
    \item \fbox{\parbox{\linewidth}{$\exists{}$ an injection
          $\iota'\in{}B\setminus\{b\}\rightarrow{}C\setminus\{\iota(b)\}~s.t.~$
          $\forall{}b'\in{}B\setminus\{b\},f(b')=g(\iota'(b'))$}}
    \end{enumerate}
    Let us define a
    $\iota'\in{}B\setminus\{b\}\rightarrow{}C\setminus\{\iota(b)\}$ as
    follows:
    $\forall{}b'\in{}B\setminus\{b\},~\iota'(b)=\iota(b)$. Let us show
    that this definition is correct by proving that\\
    \fbox{$\forall{}b'\in{}B\setminus\{b\},~\iota(b')\in{}C\setminus\{\iota(b)\}$.}

    \noindent{}Given a $b'\in{}B\setminus\{b\}$, let us show
    \fbox{$\iota(b')\in{}C\setminus\{\iota(b)\}$.}

    \noindent{}By definition of $\iota$, $\iota(b')\in{}C$; then,
    there are 2 cases:
    \begin{itemize}
    \item \textbf{CASE} $\iota(b')=\iota(b)$, then by definition of
      $\iota$ as an injective function: $b'=b$. Then,
      \qedbox{$b\in{}B\setminus\{b\}$ is a contradiction.}
    \item \textbf{CASE} \qedbox{$\iota(b')\in{}C\setminus\{\iota(b)\}$.}
    \end{itemize}

    \noindent{}Now let us get back to the previous goal. Using
    $\iota'$ to prove it, there are 2 points to prove:
    \begin{itemize}
    \item \fbox{$\iota'$ is injective.} Trivial, by definition of $\iota'$.
    \item
      \fbox{$\forall{}b'\in{}B\setminus\{b\},~f(b')=g(\iota'(b'))$.}
      Trivial, by definition of $\iota'$.
    \end{itemize}

  \end{itemize}
\end{proof}

\begin{todobox}
  Add a remark on how to convert a sequence of indexes into a finite
  set, and what is the cardinality of the finite set:\\
  $\mathop{\Omega}\limits_{i=n}^m{}f(i)$ then $\vert[n,m]\vert=(m-n)+1$ when $m\ge{}n$
\end{todobox}

\end{document}