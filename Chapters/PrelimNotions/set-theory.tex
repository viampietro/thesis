\subsection{Intuitionistic first order logic}
\label{sec:fol-intuitionistic}

The intuitionistic first order logic constitutes our framework for the
expression and the interpretation of logical formulas.  The language
to express logical formulas is the same between classical and
intuitionistic first order logic. A logical formula is either:

\begin{itemize}
\item a predicate (i.e. an atomic formula). A predicate $P$ possibly
  takes $n$ parameters as inputs and is interpreted to either true,
  represented by the $\top$ symbol, or false, represented by the
  $\bot$ symbol. We write $P(x_0,\dots,x_n)$ to denote an $n$-ary
  predicate.

\item the composition of subformulas with one of the following
  connectors: the conjunction $\land$, the disjunction $\lor$, the
  implication $\Rightarrow$, the double implication $\Leftrightarrow$
  
\item a subformula prefixed by the universal quantifier $\forall$ or
  the existantial quantifier $\exists$. For instance, the formula
  $\forall{}x.P(x)$ denotes the atomic formula $P(x)$ where $x$ is a
  unversally quantified variable of the formula. As a shorthand
  notation, we write $\forall{}x,y,z.\dots$ to denote
  $\forall{}x,\forall{}y,\forall{}z.\dots$. The same stands for the
  existantial quantifier $\exists$.
\end{itemize}





The difference between the classical first order logic and the
intuitionistic one relies in the absence of the \emph{law of the
  excluded middle} in the latter logic. The \emph{law of the excluded
  middle} considers that for all n-ary predicate $P\in{}X_0$ where
$x_0\in{}X_0,\dots,x_n\in{}X_n$, either $P(x_0,\dots,x_n)$ is valuated
to true, or


\subsection{Set theory}
\label{sec:set-theory}

In this thesis, we use set theory as the base formalism for all our
mathematical definitions and proofs. In the set theory, a set
represents a group of elements called the members of the set. For
every set $X$, we write $x\in{}X$ to denote that the element $x$ is a
member of set $X$. From here, there are multiple ways to define a set.




%%% Local Variables:
%%% mode: latex
%%% TeX-master: "../../main"
%%% End:
