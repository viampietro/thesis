In the proofs presented in this thesis, we often rely on
\textit{induction}. Here are some reminders on induction principles to
help the reader understand the proofs of Chapter~\ref{chap:proof} and
Appendix~\ref{app:sem-preserv-proof}.

\subsection{Well-founded induction}
\label{sec:well-founded-ind}

The most general principle of induction is called
\textit{well-founded} induction. From well-founded induction derives
all induction principles presented afterwards.

To introduce well-founded induction, let us define a well-founded
relation.

\begin{definition}[Well-founded relation]
  A binary relation $\prec$ over a set $A$ is well-founded if there
  exist no infinite descending chain,
  i.e. $\dots\prec{}a_i\prec\dots\prec{}a_1\prec{}a_0$.
\end{definition}

For instance, the \textit{strictly less than} relation $<$ over the
set of natural numbers is a well-founded relation.

Let $\prec$ be a well-founded binary relation on a set $A$.  The
principle of well-founded induction on the relation $\prec$ says that
in order to prove that a property $P$ holds for all elements of $A$,
it suffices to prove that $P$ holds of any $a\in{}A$ whenever $P$
holds for all $b\in{}A$ such that $b\prec{}a$, formally:

\begin{center}
  $\big(\forall{}a\in{}A,([\forall{}b\in{}A,b\prec{}a\Rightarrow{}P(b)]\Rightarrow{}P(a))\big)\Rightarrow\forall{}a\in{}A,P(a)$
\end{center}

\subsection{Structural induction}
\label{sec:struct-ind}

Sometimes, reasoning by induction requires to follow the structure of
a given set, i.e. the formation rules of a given set. This kind of
reasoning is called structural induction.

Let us consider the formation rules of the set of natural numbers:

\begin{table}[H]
  \centering
  \begin{tabular}{l}
    {\fontsize{10}{12}\selectfont\textsc{Zero}} \\
    {\begin{prooftree}[template=\fontsize{11}{13}\selectfont\inserttext]
        \infer0[]{
          $0\in\mathbb{N}$
        }
      \end{prooftree}} \\
  \end{tabular}
  \begin{tabular}{l}
    {\fontsize{10}{12}\selectfont\textsc{Succ}} \\
    {\begin{prooftree}[template=\fontsize{11}{13}\selectfont\inserttext]
        \hypo{$n\in\mathbb{N}$}
        \infer1[]{
          $n+1\in\mathbb{N}$
        }
      \end{prooftree}} \\
  \end{tabular}
\end{table}

These rules state that zero is a natural number and that for every
natural number, its direct successor is also a natural number.
Structural induction describes a way to deduce that a property holds
for the set of natural numbers, first by stating that the property
holds for zero, i.e. the minimal element of the set, then by stating
that if the property holds for a given number then it holds for its
successor. Thus, knowing that $P(0)$ holds, we can deduce that $P(1)$
holds, $P(2)$ holds, $P(3)$ holds, etc. Following the structural
induction scheme, given a property $P$, to prove that $P$ holds for
all natural numbers, it is sufficient to prove that:

\begin{itemize}
\item $P$ holds for $0$
\item if $P$ holds for a given $n$ then it holds at $n+1$
\end{itemize}

To take another example, if we want to prove that a given property $P$
holds for the set of arithmetic expressions described in
Section~\ref{sec:rule-based-def}, we must prove that:

\begin{itemize}
\item $P$ holds for all natural number $n$
\item $P$ holds for all identifiers $id$
\item if $P$ holds for all sub-expressions $e_0$ and $e_1$, then $P$
  holds for $e_0+e_1$
\end{itemize}

A proof that leverages structural induction follows the structure of
the elements we are reasoning upon.  In this thesis, we are using
structural induction to prove that a sum expression verifies a certain
property. Thus, the structural induction follows the recursive
definition of the sum term, which is, for any set $A$, function
$f\in{}A\rightarrow\mathbb{N}$ and $X\subseteq{}A$ and :

\begin{center}
  $\sum\limits_{x\in{}X}f(x)=
  \begin{cases}
    0~\mathtt{if}~X=\emptyset \\
    f(x)+\sum\limits_{x'\in{}X'}f(x')~\mathtt{if}~X=\{x\}\cup{}X' \\
  \end{cases}$
\end{center}

In the second computation branch, it is left implicit that set $X'$ is
strict subset of $X$ such that $x\notin{}X'$ or
$X'=X\setminus\{x\}$. Given a set $A$ and a function
$f\in{}A\rightarrow{}\mathbb{N}$, to prove that for all
$X\subseteq{}A$, the property $P(X,\sum\limits_{x\in{}X}f(x))$ holds,
we must show that:

\begin{itemize}
\item $\forall{}X\subseteq{}A,~X=\emptyset\Rightarrow{}P(\emptyset,0)$
\item $\forall{}X\subseteq{}A,x\in{}X,X'\subset{}X,$\\
  $X=\{x\}\cup{}X'$
  $\Rightarrow{}P(X',\sum\limits_{x'\in{}X'}f(x'))$
  $\Rightarrow{}P(\{x\}\cup{}X',f(x)+\sum\limits_{x'\in{}X'}f(x'))$
\end{itemize}

The induction follows the structure of the function. In this specific
case, structural induction is often refered to as \textit{functional}
induction. Let us prove Proposition~\ref{prop:sum-expr-pf} to
illustrate the use of structural induction over a sum term:

\begin{proposition}
  \label{prop:sum-expr-pf}
  For all $X\subset\mathbb{N}$ a finite set of natural numbers,
  $\sum\limits_{x\in{}X}2x$ is even, i.e.
  \begin{center}
    $\exists{}k\in\mathbb{N}$ s.t. $\sum\limits_{x\in{}X}2x=2k$
  \end{center}
\end{proposition}

\begin{niproof}
  Let us define the property $P$ as follows:
  \begin{center}
    $P(X,\sum\limits_{x\in{}X}2x)\equiv\exists{}k\in\mathbb{N}$
    s.t. $\sum\limits_{x\in{}X}2x=2k$
  \end{center}

  Then, let us use structural induction to prove
  $P(X,\sum\limits_{x\in{}X}2x)$.\\

  First, let us show $P(\emptyset,0)$, i.e.  $\exists{}k\in\mathbb{N}$
  s.t. $0=2k$. Let us take $k=0$ to build a tautology.\\

  Then, given a $X'\subset{}X$ and a $x\in{}X$ s.t. $X=\{x\}\cup{}X'$,
  and assuming that $P(X',\sum\limits_{x'\in{}X'}2x')$ holds (i.e. the
  induction \textit{hypothesis}), let us show
  $P(\{x\}\cup{}X',2x+\sum\limits_{x'\in{}X'}2x')$. Appealing to the
  induction hypothesis, let us take a $j$ such that
  $\sum\limits_{x'\in{}X'}2x'=2j$. Rewriting
  $\sum\limits_{x'\in{}X'}2x'$ as $2j$:
  \begin{itemize}[label=$\Rightarrow$]
  \item $\exists{}k\in\mathbb{N}$ s.t. $2x+2j=2k$
  \item $\exists{}k\in\mathbb{N}$ s.t. $2(x+j)=2k$
  \item Then, let us take $k=x+j$ to obtain a tautology.
  \end{itemize}
\end{niproof}

\subsection{Rule induction}
\label{sec:rule-ind}

A specific kind of structural induction, called \textit{rule}
induction, is applied to prove properties over sets that are defined
by rule instances. Let us take the evaluation relation for arithmetic
expressions used in Section~\ref{sec:rule-based-def} to illustrate the
principle of rule induction. To prove that a property $P$ holds for
the evaluation relation of arithmetic expressions, which is a subset
of triplets
$(\mathtt{string}\rightarrow\mathbb{N})\times{}e\times\mathbb{N}$, we
must prove that:

\begin{itemize}
\item For all
  $s\in\mathtt{string}\nrightarrow\mathbb{N},n\in\mathbb{N}$,
  $P(s,n,n)$
\item For all
  $s\in\mathtt{string}\nrightarrow{}\mathbb{N},id\in\mathtt{string}$,
  if $id\in\mathtt{dom}(s)$ then $P(s,id,s(id))$
\item For all
  $s\in\mathtt{string}\nrightarrow\mathbb{N},e_0,e_1\in{}e,n,m\in\mathbb{N}$,\\
  if $s\vdash{}e_0\rightarrow{}n$ and $P(s,e_0,n)$, and
  $s\vdash{}e_1\rightarrow{}m$ and $P(s,e_1,m)$\\
  then $P(s,e_0+e_1,n+m)$
\end{itemize}

Rule induction states that in order to prove a property over a set
defined by rule instances, the property must hold in any construction
case of the considered set. The idea is that if the property is
preserved from the premises of rules to the conclusions then the
property holds for all the elements of the set. 

Let us give an application of rule induction to prove a property over
the evaluation relation of arithmetic expressions.  First, we define,
through the three following rules, the relation $\in_r$ stating that a
given identifier $id$ is referenced in an arithmetic expression $e$,
written $id\in_r{}e$:

\begin{table}[H]
  \centering
  \begin{tabular}{l}
    {\fontsize{10}{12}\selectfont\textsc{InRId}} \\
    {\begin{prooftree}[template=\fontsize{11}{13}\selectfont\inserttext]
        \infer0[]{
          $id\in_r{}id$
        }
      \end{prooftree}} \\
  \end{tabular}
  \begin{tabular}{l}
    {\fontsize{10}{12}\selectfont\textsc{InRAddL}} \\
    {\begin{prooftree}[template=\fontsize{11}{13}\selectfont\inserttext]
        \hypo{$id\in_r{}e_0$}
        \infer1[]{
          $id\in_r{}e_0+e_1$
        }
      \end{prooftree}} \\
  \end{tabular}
  \begin{tabular}{l}
    {\fontsize{10}{12}\selectfont\textsc{InRAddR}} \\
    {\begin{prooftree}[template=\fontsize{11}{13}\selectfont\inserttext]
        \hypo{$id\in_r{}e_1$}
        \infer1[]{
          $id\in_r{}e_0+e_1$
        }
      \end{prooftree}} \\
  \end{tabular}
\end{table}

Then, the property of Proposition~\ref{prop:arith-expr-pf} states that
an arithmetic expression that contains references to identifiers that
are not part of the current state's domain can not be evaluated.

\begin{proposition}
  \label{prop:arith-expr-pf}
  Let $id\in\mathtt{string}$. For all state $s$, arithmetic expression
  $e$, and natural number $n$,
  \begin{center}
    $id\notin{}\mathtt{dom}(s)\land{}id\in_r{}e\Rightarrow{}\lnot{}s\vdash{}e\rightarrow{}n$
  \end{center}
\end{proposition}

\begin{niproof}
  Let us define the property $P$ as follows:
  \begin{center}
    $P(s,e,n)\equiv{}id\notin{}\mathtt{dom}(s)\land{}id\in_r{}e\Rightarrow{}\lnot{}s\vdash{}e\rightarrow{}n$
  \end{center}

  Then, let us use rule induction to prove $P(s,e,n)$.\\

  First, we must prove $P(s,n,n)$. Assuming $id\in_r{}n$, there is a
  contradiction as no rule instance defining the relation $\in_r$
  includes the case where the considered expression is a natural
  number.\\

  Then, we must prove $P(s,id',s(id'))$, assuming that
  $id'\in\mathtt{dom}(s)$.  We know that $id\in_r{}id'$, and thus
  $id=id'$. Then, there is a contradiction between
  $id\in\mathtt{dom}(s)$ and $id\notin{}\mathtt{dom}(s)$.\\

  Finally, we must prove $P(s,e_0+e_1,n+m)$, assuming that
  $s\vdash{}e_0\rightarrow{}n$ and $P(s,e_0,n)$, and
  $s\vdash{}e_1\rightarrow{}m$ and $P(s,e_1,m)$.  We know that
  $id\in_r{}e_0+e_1$; this hypothesis has either be constructed by
  applying Rule~\textsc{InRAddL} or Rule~\textsc{InRAddR}.  If
  Rule~\textsc{InRAddL} has been applied, then we know $id\in_r{}e_0$;
  thus, from $P(s,e_0,n)$, we can deduce
  $\lnot{}s\vdash{}e_0\rightarrow{}n$, which contradicts
  $s\vdash{}e_0\rightarrow{}n$. We can perform the proof similarly if
  Rule~\textsc{InRAddR} has been applied.
\end{niproof}

%%% Local Variables:
%%% mode: latex
%%% TeX-master: "../../main"
%%% End:
