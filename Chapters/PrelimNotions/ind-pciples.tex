In the proceedings of proofs described in this thesis, we often
perform proofs by induction. Here are some reminders on the induction
principles to help the reader understand the proofs of Chapter and
Appendices.

\subsection{Mathematical induction}
\label{sec:math-ind}

The principle of mathematical induction is tied to the set of natural
numbers. It states that, in order to prove that a property $P$ holds
for all natural numbers, it is sufficient to prove that:

\begin{itemize}
\item $P$ holds for $0$
\item if $P$ holds for a given $n$ then it holds at $n+1$
\end{itemize}

When, assuming that $P(n)$ to try and then try to prove $P(n+1)$ for a
given $n$, $P(n)$ is called the induction hypothesis. Thus,
mathematical induction describes a way to deduce that a property holds
for the set of natural numbers; first by stating that the property
holds for zero; then by giving a strategy to prove that the property
$P$ holds for any number $n$ by applying $P(m)$ implies $P(m+1)$
starting from $0$.

\subsection{Structural induction}
\label{sec:struct-ind}

Sometimes, reasoning by induction necesitates to follow the structure
of a given set, i.e. the formation rules of a given set. For instance,
if we want to prove that a given property $P$ holds for the set of
arithmetic expressions given as an example in
Section~\ref{sec:rule-based-def}, we must prove that:

\begin{itemize}
\item $P$ holds for all natural number $n$
\item $P$ holds for all identifiers $id$
\item if $P$ holds for all sub-expressions $e_0$ and $e_1$, then $P$
  holds for $e_0+e_1$
\end{itemize}

A proof that leverages structural induction follows the structure of
the elements we are reasoning upon. In this thesis, we are using
structural induction especially to prove the equality between two sum
expressions; for instance, given a two sets $A$ and $B$, an
application $f\in{}A\rightarrow{}\mathbb{N}$, and an application
$g\in{}B\rightarrow{}\mathbb{N}$, say we want to prove :

\begin{center}
  $\sum\limits_{a\in{}A}f(a)=\sum\limits_{b\in{}B}g(b)$ 
\end{center}

What we often do is to use structural induction on one of the two sum
terms, i.e. the left or the right operand of the equality. Here, the
structural induction follows the recursive definition of the sum term,
i.e.:

\begin{center}
  $\sum\limits_{a\in{}A}f(a)\equiv
  \begin{cases}
    0~\mathtt{if}~A=\emptyset \\
    f(a)+\sum\limits_{a'\in{}A\setminus\{a\}}f(a')~otherwise \\
  \end{cases}$
\end{center}

Thus, to prove the above equality, we must prove that:

\begin{itemize}
\item $0=\sum\limits_{b\in{}B}g(b)$ when $A=\emptyset$
\item if
  $\sum\limits_{a'\in{}A\setminus\{a\}}f(a')=\sum\limits_{b\in{}B'}g(b)$
  for all $B'\subseteq{}B$, then
  $f(a)+\sum\limits_{a'\in{}A\setminus\{a\}}f(a')=\sum\limits_{b\in{}B}g(b)$
  holds
\end{itemize}

The induction follows the structure of the function. In this specific
case, structural induction is often refered to as \textit{functional}
induction.

\subsection{Rule induction}
\label{sec:rule-ind}

A specific kind of induction, called \textit{rule} induction, applies
when wanting to prove properties over sets that are defined by
inference rules (or rule instances). For instance, let us take the
evaluation relation for arithmetic expressions used in
Section~\ref{sec:rule-based-def} to illustrate the principle of rule
induction. To prove that a property $P$ holds for the evaluation
relation of arithmetic expressions, which is a subset of triplets
$(\mathtt{string}\rightarrow\mathbb{N})\times{}e\times\mathbb{N}$, we
must prove that:

\begin{itemize}
\item $P(s,n,n)$ for all $s,n$
\item if $id\in\mathtt{dom}(s)$ then $P(s,id,s(id))$ for all $s,id$
\item if $s\vdash{}e_0\rightarrow{}n$ and $P(s,e_0,n)$, and
  $s\vdash{}e_1\rightarrow{}m$ and $P(s,e_1,m)$ then
  $P(s,e_0+e_1,n+m)$ for all $s,e_0,e_1,n,m$
\end{itemize}

Rule induction states that in order to prove a property over a set
defined by rule instances, the property must hold in any construction
case of the considered set. The idea is that if the property is
preserved from the premises of rules to the conclusions then the
property holds for all the elements of the set. 

Let us give an application of rule induction to prove a property over
the evaluation relation of arithmetic expressions.  First, we define,
through the three following rules, the relation $\in_r$ stating that a
given identiifer $id$ is referenced in an arithmetic expression $e$,
written $id\in_r{}e$:

\begin{table}[H]
  \centering
  \begin{tabular}{l}
    {\fontsize{10}{12}\selectfont\textsc{InRId}} \\
    {\begin{prooftree}[template=\fontsize{11}{13}\selectfont\inserttext]
        \infer0[]{
          $id\in_r{}id$
        }
      \end{prooftree}} \\
  \end{tabular}
  \begin{tabular}{l}
    {\fontsize{10}{12}\selectfont\textsc{InRAddL}} \\
    {\begin{prooftree}[template=\fontsize{11}{13}\selectfont\inserttext]
        \hypo{$id\in_r{}e_0$}
        \infer1[]{
          $id\in_r{}e_0+e_1$
        }
      \end{prooftree}} \\
  \end{tabular}
  \begin{tabular}{l}
    {\fontsize{10}{12}\selectfont\textsc{InRAddR}} \\
    {\begin{prooftree}[template=\fontsize{11}{13}\selectfont\inserttext]
        \hypo{$id\in_r{}e_1$}
        \infer1[]{
          $id\in_r{}e_0+e_1$
        }
      \end{prooftree}} \\
  \end{tabular}
\end{table}

Then, the property of Proposition states that an arithmetic expression
that contains references to identifiers that are not part of the
current state's domain can not be evaluated.

\begin{proposition}
  Let $id\in\mathtt{string}$. For all state $s$, arithmetic expression
  $e$, and natural number $n$,
  \begin{center}
    $id\notin{}\mathtt{dom}(s)\land{}id\in_r{}e\Rightarrow{}\lnot{}s\vdash{}e\rightarrow{}n$
  \end{center}
\end{proposition}

\begin{proof}
  Let us define the property $P$ as follows:
  \begin{center}
    $P(s,e,n)\equiv{}id\notin{}\mathtt{dom}(s)\land{}id\in_r{}e\Rightarrow{}\lnot{}s\vdash{}e\rightarrow{}n$
  \end{center}

  Then, let us use rule induction to prove $P(s,e,n)$.

  First, we must prove $P(s,n,n)$. Assuming $id\in_r{}n$, there is a
  contradiction as no rule instance defining the relation $\in_r$
  includes the case where the considered expression is a natural
  number.

  Then, we must prove $P(s,id',s(id'))$, assuming that
  $id'\in\mathtt{dom}(s)$.  We know that $id\in_r{}id'$, and thus
  $id=id'$. Then, there is a contradiction between
  $id\in\mathtt{dom}(s)$ and $id\notin{}\mathtt{dom}(s)$.

  Finally, we must prove $P(s,e_0+e_1,n+m)$, assuming that
  $s\vdash{}e_0\rightarrow{}n$ and $P(s,e_0,n)$, and
  $s\vdash{}e_1\rightarrow{}m$ and $P(s,e_1,m)$.  We know that
  $id\in_r{}e_0+e_1$; this hypothesis has either be constructed by
  applying Rule~\textsc{InRAddL} or Rule~\textsc{InRAddR}.  If
  Rule~\textsc{InRAddL} has been applied, then we know $id\in_r{}e_0$;
  thus, from $P(s,e_0,n)$, we can deduce
  $\lnot{}s\vdash{}e_0\rightarrow{}n$, which contradicts
  $s\vdash{}e_0\rightarrow{}n$. We can perform the proof in a similar
  way if Rule~\textsc{InRAddR} has been applied.
\end{proof}

%%% Local Variables:
%%% mode: latex
%%% TeX-master: "../../main"
%%% End:
