In this section, we present the \coq{} proof assistant
\cite{Coq2021}. The \coq{} proof assistant is the framework we use to
encode the different semantics and programs involved in the \hilecop{}
model-to-text transformation, and also to formally verify the proof of
semantic preservation. Here, we give an overview of the different
concepts underlying the \coq{} proof assistant. The aim is to give to
the reader the tools to understand the different listings presenting
\coq{} code in the following chapters. For a thorough presentation of
the \coq{} proof assistant, the reader can refer to the reference
manual\footnote{\url{https://coq.inria.fr/distrib/current/refman/}},
or to \cite{Chlipala2013,Paulin-Mohring2012,Bertot2004}.


In the context of software engineering, the \coq{} proof assistant
provides a way to implement program specifications using a rich type
system, to implement programs leveraging functional programming, and
to mechanize proofs of soundness and completeness between
specifications and programs.

%%% Local Variables:
%%% mode: latex
%%% TeX-master: "../../main"
%%% End:
