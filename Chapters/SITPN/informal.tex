Here, fundamentals on the Petri net formalism are outlined, and
certain classes of Petri nets are described more precisely. Then, the
specificities of the Petri nets used to design the behavior of
electronic components in the \hilecop{} methodology are presented. For
more information on the topic of Petri nets, the reader can refer to
\cite{David1994}, \cite{Murata1989}, or \cite{Diaz2001}.

\subsection{Preliminary notions on Petri nets}
\label{subsec:pn-formalism}

Petri nets, invented by C. A. Petri \cite{Petri1962}, are used to
model a broad range of dynamic systems: resource sharing between
concurrent processes \cite{David1994}, behavior of agents in
multi-agent systems \cite{Celaya2007}, behavior of digital components
\cite{Yakovlev2006}. A Petri net is a directed graph, composed of two
types of node: place nodes (\emph{circles}) and transition nodes
(\emph{squares} or \emph{lines}). As shown in
Figure~\ref{fig:pn-example}, place nodes usually represent a part of
the state of the modelled system, here the states of two computer
processes and a semaphor; transition nodes usually refer to events
triggering the system evolution (or state changing).

\begin{figure}[H]
  \centering
  \includegraphics[keepaspectratio=true, width=\textwidth]{Figures/SITPN/pn-example}
  \caption[An example of Petri net]{An example of Petri net - A
    semaphor to prevent the parallel execution of \textit{Treatment 1}
    and \textit{Treatment 2}.}
  \label{fig:pn-example}
\end{figure}

\paragraph{Edges}
In a Petri net, directed edges link together places and
transitions. Places cannot be linked to other places, and the same
stands for transitions.  There are two kinds of edges, \textit{pre} or
\textit{incoming} edges, going from a place to a transition, and
\textit{post} or \textit{outcoming} edges, going from a transition to
a place. Places linked to a transition $t$ by incoming
(resp. outcoming) edges will be referred to as the \textit{input}
(resp. \textit{output}) of $t$. The same stands for a place $p$. For
instance, in Figure~\ref{fig:pn-example}, $p_0$ and $sem$ are the
input places of $t_0$, and $p_1$ is the output place of $t_0$; $t_1$
and $t_3$ are the input transitions of place $sem$, and $t_0$ and
$t_2$ are the output transitions of $sem$. Some weight --a natural
number-- is associated to the edges of a Petri net. If no label
appears on the edge then one is the default weight. Petri nets are
said to be \textit{generalized} when edge weights are possibly greater
than one.

\paragraph{Marking}
In Figure~\ref{fig:pn-example}, places $p_0$, $p_3$ and $sem$ are
marked with tokens, represented by little black circles.  This means
that places $p_0$, $p_3$ and $Sem$ are currently active.  The
distribution of tokens over places is called the \textit{marking} of
the net. The marking of a Petri net reflects the overall state of the
modelled system at a certain moment in its activity cycle.

\paragraph{Transition firing}
In a Petri net, the marking evolves based on a token
consump\-tion-production system. Transitions consume tokens from their
input places, and produce tokens to their output places. This whole
system is called \textit{transition firing}. In order to be
\textit{firable}, a transition must be \textit{sensitized} (or
\textit{enabled}), meaning that the number of tokens in each of its
input places must be equal or greater than the weight of its incoming
edges. For instance, in Figure~\ref{fig:pn-example}, the transition
$t_0$ is sensitized because the weight of the arc ($p_0$, $t_0$) is of
one (default value), and place $p_0$ is marked with one token, and the
same stands for the number of tokens in place $sem$ and the weight of
the arc ($sem$, $t_0$). As a counter example, transition $t_3$ is not
sensitized because there is no tokens in its input place $p_2$.
Depending on the class of PNs that is considered, other parameters
affect the \textit{firability} of transitions (see interpreted Petri
nets, time Petri nets and
Section~\ref{subsec:hpn-particularities}). When a sensitized
transition is fired, tokens are retrieved from its input places (as
much tokens as the weight of the arcs) and produced in its output
places (as much tokens as the weight of the arcs).  This process
represents the occurence of an event --denoted by the transition--
triggering the evolution of the system from one state to
another. Figure~\ref{fig:firing-example} shows the state of the PN of
Figure~\ref{fig:pn-example} after the firing of the transition $t_0$.

\begin{figure}[H]
  \centering
  \includegraphics[keepaspectratio=true, width=\textwidth]{Figures/SITPN/firing-example}
  \caption[An example of transition firing]{The PN of
    Figure~\ref{fig:pn-example} after the firing of transition $t_0$.}
  \label{fig:firing-example}
\end{figure}

In Figure~\ref{fig:firing-example}, the tokens in the input places of
$t_0$, i.e. places $p_0$ and $sem$ have been consumed, and one token
has been produced in the output place $p_1$. The current marking
indicates that the task ``Treatment 1'' is being performed (place
$p_1$ is active).

In Figure~\ref{fig:pn-example}, transition $t_0$ and $t_2$ are enabled
at the same time. However, the \emph{standard} semantics of Petri nets
is such that only one transition can be fired in that case. Either
$t_0$ consumes the token in place $sem$ or $t_2$ does, but never
both. Thus, the transition firing process in the standard PN semantics
is an undeterministic process. From the marking of
Figure~\ref{fig:pn-example}, two marking are reachable: the marking
resulting of the firing of transition $t_0$ and the one resulting of
the firing of transition $t_2$. Also, the transition firing process is
asynchronous. As soon as a transition is enabled, the transition
firing process can be triggered. 

\paragraph{Extended Petri nets}
The class of \textit{extended} Petri nets introduces the inhibitor and
test edges. As shown in Figure~\ref{fig:inhib-test-arcs}, test arc
tips are black circles and inhibitor arc tips are white
circles. Inhibitor and test edges are incoming edges, always coming
from a place toward a transition.
%
\begin{figure}[H]
  \centering
  \includegraphics[keepaspectratio=true, width=.2\textwidth]{Figures/SITPN/inhib-arc}  
  \hspace{50pt}
  \includegraphics[keepaspectratio=true, width=.2\textwidth]{Figures/SITPN/test-arc}
  \caption[Two examples of extended Petri nets.]{Two examples of
    extended Petri nets; on the left side, a PN with inhibitor arcs;
    on the right side, a PN with test arcs.}
  \label{fig:inhib-test-arcs}
\end{figure}
The particularity of the inhibitor and test edges is that they are not
consuming tokens in input places after the firing of a transition.
Indeed, they are just testing the number of tokens in incoming places
to determine if the transition is enabled. Inhibitor arcs ensure that
the number of tokens in input places is strictly lower than their
weights; test arcs ensure that the number of tokens in pre-places is
equal or greater than their weights. Therefore, on the left side of
Figure~\ref{fig:inhib-test-arcs}, transition $t_0$ is sensitized
because there is strictly less than one token in place $p_0$ and
strictly less than two tokens in place $p_1$. On the right side of
Figure~ \ref{fig:inhib-test-arcs}, transition $t_0$ is sensitized
because there is at least one token in place $p_0$ and three tokens in
place $p_1$.

\paragraph{Interpreted Petri nets}
Interpreted Petri nets (IPN) \cite{David1994} are intended to describe
the interaction between a system and its outside
environment. Interpretation introduces three new concepts:

\begin{itemize}

\item Continuous actions, associated to the places of a Petri net.
  Actions associated to a place $p$ are activated as long as $p$ is
  marked. For instance, when modelling a controller with a IPN,
  actions can correspond to the setting of a electric signal
  controlling some actuator (e.g, maintaining a LED on).
  
\item Functions (or discrete actions), associated to the transitions
  of a Petri net. When a transition $t$ is fired, all functions
  associated to $t$ are executed.  Functions can be any kind of
  discrete operations --variable incrementation, for instance--
  manipulating both internal variable and external signal values.
  
\item Conditions, associated to the transitions of a Petri net.
  Conditions are boole\-an expressions receiving their values from the
  environment of the PN. In an IPN, a transition is firable only if
  all its associated conditions are \texttt{true} (or \texttt{false}
  in the case where an inverse condition is associated).
  
\end{itemize}

Figure~\ref{fig:ipn} illustrates the use of actions, functions and
conditions in an interpreted Petri net.

\begin{figure}[H]
  \centering
  \includegraphics[keepaspectratio=true, width=.7\textwidth]{Figures/SITPN/interpreted-pn}
  \caption[An example of Interpreted Petri net.]{An example of
    Interpreted Petri net; on the left side, the interpreted Petri
    net; on the right side, examples of tests associated to conditions
    and operations associated to actions and functions.}
  \label{fig:ipn}
\end{figure}%
%
In Figure~\ref{fig:ipn}, the action $a_0$ is activated as place $p_0$
is marked by one token. Also, function $f_0$ will be executed at the
firing of $t_0$, that is if condition $c_0$ is \texttt{true} and $t_0$
is sensitized. On the right side of Figure~\ref{fig:ipn}, we associate
a semantics to conditions, actions and functions in terms of concrete
tests or operations. However, when considering the semantics of IPNs,
what is of interest to us is the value of conditions and the execution
state of actions and functions. We are not interested in interpreting
the Boolean expressions associated to conditions but only to retrieve
their value; likewise, we are only interested in the fact that a given
action/function is activated/executed but not in what is its effect on
the environment.

\paragraph{Time Petri nets}

In a time Petri net (TPN), time intervals are associated to
transitions. The goal of associating a time interval to a transition
is to constrain the firing of this transition to a certain time
window. As shown in Figure~\ref{fig:tpn}, time intervals are of the
form $[a, b]$, where $a\in\mathbb{N}^{*}$ and
$b\in\mathbb{N}^{*}\sqcup\{\infty\}$. Time intervals can also be
defined with real numbers but in this thesis we not interested in this
kind of time intervals. In Figure~\ref{fig:tpn}, time counters are
represented in red between diamond brackets. The current value of time
counters is part of the state of the TPN, along with its current
marking, whereas time intervals are part of the static structure of
the TPN.

\begin{figure}[H]
  \centering
  \includegraphics[keepaspectratio=true, width=0.2\textwidth]{Figures/SITPN/time-pn}
  \caption{An example of time Petri net.}
  \label{fig:tpn}
\end{figure}

For each sensitized transition associated with a time interval, time
counters are incremented at a certain time step, previously defined by
the modeller. For instance, in the case of SITPNs, i.e. Petri nets
used in the \hilecop{} methodology, the reference time step for the
incrementation of time counters is the clock cycle.

When a transition associated with a time interval is fired or
disabled, a reset order is sent to the transition to set its time
counter to zero. The value of reset orders (Boolean values) is also a
part of the TPN state.  In time Petri nets, a transition is firable
only if its time counter value is within its time interval. For
instance, in Figure~\ref{fig:tpn}, only transition $t_0$ is firable.

There are multiple possible firing policy for TPNs. Here, we will only
consider the \textit{imperative} firing policy: as soon as a time
counter reaches the lower bound of a time interval, the associated
transition must be fired.

\paragraph{Petri nets with priorities}

Two transitions are in structural conflict if they have a common input
place connected through a \textit{basic} arc (i.e. neither inhibitor
nor test). When two transitions in structural conflict are firable at
the same time, then, the conflict becomes \textit{effective}. A Petri
net with priorities, it is possible to specify a firing priority in
the case where the conflict between two transitions becomes
effective. In that case, the transition with the highest firing
priority will always be fired
first. Figure~\ref{fig:structural-conflict} illustrates the
application of a priority relation to solve the effective conflict
between two transitions.

\begin{figure}[H]
  \centering
  \includegraphics[keepaspectratio=true,width=.6\textwidth]{Figures/SITPN/structural-conflict}
  \caption[An example of transitions in structural and effective
  conflict.]{An example of transitions in structural and effective
    conflict. In subfigure (b), the dotted arrow represents the
    priority relation between $t_0$ and $t1$. The transition with the
    highest firing priority is at the source of the arrow; here,
    transition $t_0$.}
  \label{fig:structural-conflict}
\end{figure}

\subsection{Particularities of SITPNs}
\label{subsec:hpn-particularities}

Here, we will informally present the specificities of the Petri nets
describing the internal behavior of the \hilecop{} high-level model
components. These Petri nets are called: Synchronously executed,
extended, generalized, Interpreted, Time Petri Nets with priorities or
SITPNs. SITPNs are a combination of multiple classes of PNs, namely:
extended PNs, generalized PNs, interpreted PNs, time PNs and PNs with
priorities. These classes were presented in the above section. We will
now talk about another aspect of SITPNs that constitutes the
originality of the formalism compared to the standard PN semantics:
its synchronous execution.

The class of interpreted Petri nets increases the expressiveness of
the \hilecop{} high-level models. However, to ensure the safe
execution of functions after the synthesis of the designed circuit on
a FPGA card, the whole system must be synchronized with a clock signal
\cite{Leroux2014}. As a consequence, a clock signal also regulates the
evolution of SITPNS (i.e. it is a part of their semantics). The
evolution of an SITPN is \textit{synchronized} with two clock events:
the rising edge and the falling edge of the
signal. Figure~\ref{fig:sync-exec} depicts the process of state
evolution, following the clock signal.

\begin{figure}[H]
  \centering
  \includegraphics[keepaspectratio=true, width=.9\textwidth]{Figures/SITPN/sync-exec}
  \caption{Evolution of an SITPN synchronized with a clock signal.}
  \label{fig:sync-exec}
\end{figure}

Considering the different classes of PNs that define SITPNs, the state
of an SITPN is characterized by its marking, the value of time
counters, the reset orders assigned to time counters, the
execution/activation status of actions/functions (Boolean values), and
the value of conditions (also Boolean). As shown in figure
\ref{fig:sync-exec}, the state evolution process of an SITPN is
divided in two parts is divided in two steps. At the rising of the
clock signal, the marking is updated, i.e. transitions are fired,
reset orders are sent to all transitions that have been fired or
disabled by the firing process, and all functions associated to fired
transitions are executed. Then, on the falling edge of the clock
signal, the fresh condition values are provided by the environment,
the time counter values are either incremented, reset or values are
stalling (see the following remark on locked time counters), and all
action associated with marked places are activated. Figure gives an
example of the evolution of the state of a given SITPN through one
clock cycle. The aim of this figure and the explanation that follows
is to give some hints to the reader about the semantics of SITPNs
before giving its formal definition in Section~\ref{sec:sitpn-sem}.

From Step~1 to Step~2, transition $t_0$ is fired. Indeed, transition
$t_0$ gathers all the necessary conditions to trigger the firing
process, namely: $t_0$ is enabled by the current marking, condition
$c_0$ is \texttt{true} (appears in green), $t_0$'s time counter value
is within the time interval. As a consequence, one token is consumed
in place $p_0$ and one token is produced in place $p_1$, function
$f_0$ is executed at Step~2 (appears in green) and a reset order is
sent to the time counter of $t_0$ (appears in blue). From Step~2 to
Step~3, the activation activation status are updated; $a_0$ stays
activated place $p_0$ is marked; $a_1$ becomes newly activated as
$p_1$ is marked. The time counter values are updated; $t_0$'s time
counter is set to zero as the transition previously received a reset
order. However, as $t_0$ is still sensitized by the new marking, its
time counter is incremented. Thus, the resulting time counter value at
Step~3 is one (i.e. result of reset plus incrementation). Also, the
condition values are retrieved from the environment. As a consequence,
condition $c_0$ takes the value \texttt{false}.

\paragraph{A remark on priorities}

Some issues arise when trying to express an \textit{or} branching with
Hilecop Petri nets, as shown by figure
% \ref{fig:structural-conflict}.(a).
% \begin{figure}[H]
%   \centering
%   \includegraphics[keepaspectratio=true,width=.4\textwidth]{Figures/SITPN/structural-conflict}
%   \caption{Transitions in structural conflict.}
%   \label{fig:structural-conflict}
% \end{figure}
The semantics of synchronous execution is that all transitions are
fired at the same time. Then, in figure
\ref{fig:structural-conflict}.a, transitions $t_0$ and $t_1$ are both
sensitized by place $p_0$, and consequently are both fired at the same
time. The system acts as if two tokens were available in place $p_0$,
one for the firing of $t_0$ and another for the firing of $t_1$. In
the situation depicted by figure \ref{fig:structural-conflict}.a,
$t_0$ and $t_1$ are said to be in \textit{structural conflict} with
each other, meaning they have one of their pre-places in common. To
resolve structural conflicts, priorities are drawn between transitions
which will determine a firing order in case of conflict. In figure
\ref{fig:structural-conflict}.b, the dotted arrow represents the
priority relation between transition $t_0$ and $t_1$.  Here, $t_0$ has
a higher firing priority than $t_1$.

%%% Local Variables:
%%% mode: latex
%%% TeX-master: "../../main"
%%% End:
