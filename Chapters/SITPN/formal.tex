We hope that the reader has now a fair understanding of the concepts
underlying the SITPNs and of the dynamics governing the SITPN state
evolution process. In this section, we give the formal definition of
the SITPN structure and of its execution semantics. We also introduce
the concept of a \emph{well-defined} SITPN at the end of the section.

\subsection{SITPN structure}
\label{sec:struct-and-wd}

\noindent{}The structure of SITPNs is formally defined as follows:

\begin{definition}[SITPN]
  \label{def:sitpn}
  A synchronously executed, extended, generalized, interpreted, and
  time Petri net with priorities is a tuple
  ${<}P,T,pre,post,M_0,{\succ},\mathcal{A},\mathcal{C},\mathcal{F},
  \mathbb{A},\mathbb{C},\mathbb{F},{I_s}{>}$, where we have:
  % 
  \begin{enumerate}
  \item $P=\{p_0,\ldots,p_n\}$, a finite set of places.
  \item $T=\{t_0,\ldots,t_m\}$, a finite set of transitions.
  \item
    $pre\in{}P\rightarrow{}T\nrightarrow(\mathbb{N}^{*}\times\{\mathtt{basic},\mathtt{inhib},\mathtt{test}\})$,
    the function associating a weight to place-transition edges.
  \item $post\in{}T\rightarrow{}P\nrightarrow\mathbb{N}^{*}$, the
    function associating a weight and a type to transition-place
    edges.
  \item $M_0\in{}P\rightarrow\mathbb{N}$, the initial marking of the SITPN.
  \item $\succ\subseteq{}(T\times{}T)$, the priority relation which is
    a partial order over the set of transitions.
  \item $\mathcal{A}=\{a_0,\ldots,a_i\}$, a finite set of continuous actions.
  \item $\mathcal{F}=\{f_0,\ldots,f_k\}$, a finite set of functions
    (instantaneous actions).
  \item $\mathcal{C}=\{c_0,\ldots,c_j\}$, a finite set of conditions.
  \item $\mathbb{A}$ $\in$ ${}P$ $\rightarrow$ $\mathcal{A}$
    $\rightarrow$ $\mathbb{B}$, the function associating actions to
    places.  $\forall{}p\in{}P$, $\forall{}a\in\mathcal{A}$,
    $\mathbb{A}(p,a)=\mathtt{true}$, if $a$ is associated to $p$,
    $\mathbb{A}(p,a)=\mathtt{false}$ otherwise.
  \item $\mathbb{F}\in{}T\rightarrow\mathcal{F}\rightarrow\mathbb{B}$,
    the function associating functions to transitions.
    $\forall{}t\in{}T,~\forall{}f\in\mathcal{F},$
    $\mathbb{F}(t,f)=\mathtt{true}$, if $f$ is associated to $t$,
    $\mathbb{F}(t,f)=\mathtt{false}$ otherwise.
    
  \item $\mathbb{C} \in T \rightarrow \mathcal{C} \rightarrow\{-1,0,1\}$, the
    function associating conditions to transitions.
    $\forall t \in T$, $\forall c \in \mathcal{C}$,
    $\mathbb{C}(t,c)=1$, if $c$ is associated to $t$,
    $\mathbb{C}(t,c)=-1$, if $\bar{c}$ is associated to $t$,
    $\mathbb{C}(t,c)=0$ otherwise.
  \item $I_s\in{}T\nrightarrow\mathbb{I}^{+}$, the partial function
    associating static time intervals to transitions, where
    $\mathbb{I}^{+}\subseteq(\mathbb{N}^{*}\times(\mathbb{N^{*}}\sqcup\{\infty\}))$.
    $T_i$ denotes the definition domain of $I_s$, i.e. the set of time
    transitions.
  \end{enumerate}
\end{definition}

\subsection{SITPN State}
\label{sec:sitpn-state}

The SITPN semantics describes the evolution of the state of an SITPN
through a given number of clock cycles; thus, we must first define the
SITPN state structure:

\begin{definition}[SITPN State]
  \label{def:sitpnstate}
  For a given $sitpn\in{}SITPN$, let $S(sitpn)$ be the set of possible
  states of $sitpn$. An SITPN state $s\in{}S(sitpn)$ is a tuple
  ${<}M,I,reset_t,ex,cond{>}$, where:
  \begin{enumerate}
  \item $M\in{}P\rightarrow\mathbb{N}$ is the current marking of sitpn.
  \item\label{item:sitpn-state-tc} $I\in{}T_i{}\rightarrow\mathbb{N}$
    is the function mapping time transitions to their current time
    counter value.
  \item\label{item:sitpn-state-rst}
    $reset_t\in{}T_i\rightarrow\mathbb{B}$ is the function mapping
    time transitions to time counter reset orders (defined as
    Booleans).
  \item $ex\in{}\mathcal{A}\sqcup\mathcal{F}\rightarrow\mathbb{B}$ is
    the function representing the current activation (resp. execution)
    state of actions (resp. functions).
  \item $cond\in\mathcal{C}\rightarrow\mathbb{B}$ is the function representing the
    current value of conditions (defined as Booleans).
  \end{enumerate}
\end{definition}

In Items~\ref{item:sitpn-state-tc} and \ref{item:sitpn-state-rst} of
Definition~\ref{def:sitpnstate}, \emph{time} transitions refer to
transitions with a time interval, i.e. the transitions belonging to
the domain of $I_s$.

\subsection{Preliminary definitions and fired transitions}
\label{sec:fired-trans}

Before formalizing the full SITPN semantics, we must introduce some
definitions and notations, especially the definition of a
\emph{firable} and a \emph{fired} transition. We use the two following
notations to simplify the formalization of the SITPN semantics.

\begin{notation}[Relations between markings]
  \label{not:markings}
  For all relation $\mathcal{R}$ existing between two marking
  functions $M$ and $M'$, the expression $\mathcal{R}(M,M')$ is a
  notation for $\forall{}p\in{}P,~\mathcal{R}(M(p),M'(p))$. For
  instance, $M'=M-\sum\limits_{t_i\in{}Pr(t)}pre(t_i)$ is a notation
  for
  $\forall{}p\in{}P,~M'(p)=M(p)-\sum\limits_{t_i\in{}Pr(t)}pre(p,t_i)$.
\end{notation}

\begin{notation}[Sum expressions and arc types]
  \label{not:sum-exprs}
  Many times in this document, we need to express the number of tokens
  coming in or out of places, after the firing of a certain subset of
  transitions. To do so, we use two kinds of sum expression:
  \begin{enumerate}
  \item The first kind of expression computes a number of output
    tokens. For instance, for a given place $p$,
    $\sum\limits_{t\in{}T'}pre(p,t)$ where $T'\subseteq{}T$.

    The expression $\sum\limits_{t\in{}T'}pre(p,t)$ is a notation for
    $\sum\limits_{t\in{}T'}\begin{cases}\omega~if~pre(p,t)=(\omega,\mathtt{basic})\\
      0~otherwise \end{cases}$

    When computing a sum of output tokens (i.e. resulting of a firing
    process), we want to add to the sum the weight of the arc between
    place $p$ and a transition $t\in{}T'$ only if there exists an arc
    of type $\mathtt{basic}$ from $p$ to $t$ (remember that the test
    and inhibitor never lead to the withdrawal of tokens during the
    firing process). Otherwise, we add 0 to the sum as it is a neutral
    element of the addition operator over natural numbers.
    
  \item The second kind of expression computes a number of input
    tokens.  For instance, for a given place $p$,
    $\sum\limits_{t\in{}T'}post(p,t)$ where $T'\subseteq{}T$.

    The expression $\sum\limits_{t\in{}T'}post(p,t)$ is a notation for
    $\sum\limits_{t\in{}T'}\begin{cases}\omega~if~post(t,p)=\omega\\
      0~otherwise \\ \end{cases}$

    Here, we add the weight of the arc from $t$ to $p$ only if there
    exists such an arc; we add 0 to the sum otherwise.
  \end{enumerate}
  Therefore, in the remainder of the document, we will use the
  conciser notation $\sum\limits_{t\in{}T'}pre(p,t)$ to denote an
  output token sum, and $\sum\limits_{t\in{}T'}post(t,p)$ to denote an
  input token sum.
\end{notation}

\noindent{}We give the formal definition of the sensitization (see
Section~\ref{subsec:pn-formalism} for an informal definition) of a
transition by a given marking as follows:

\begin{definition}[Sensitization]
  \label{def:sens}
  A transition $t\in{}T$ is said to be sensitized, or enabled, by a
  marking $M$, which is noted $t\in{}Sens(M)$, if
  $\forall{}p\in{}P,\forall\omega\in\mathbb{N}^{*},~\big(pre(p,t)=(\omega,\mathtt{basic})\vee{}pre(p,t)=(\omega,\mathtt{test})\big)\Rightarrow{}M(p)\ge{}\omega$,
  and $pre(p,t)=(\omega,\mathtt{inhib})\Rightarrow{}M(p)<{}\omega$.
\end{definition}

\noindent{}We give the formal definition of a \emph{firable}
transition at a given SITPN state as follows:

\begin{definition}[Firability]
  \label{def:firable}
  A transition $t\in{}T$ is said to be firable at a state
  $s={<}M,I,reset_t,ex,cond{>}$, which is noted $t\in{}Firable(s)$, if
  $t\in{}Sens(M)$, and $t\notin{}T_i$ or $I(t)\in{}I_s(t)$, and
  $\forall c \in \mathcal{C}, \mathbb{C}(t, c) = 1 \Rightarrow cond(c)
  = 1$ and $\mathbb{C}(t, c) = -1 \Rightarrow cond(c) = 0$.
\end{definition}

As explained in Section~\ref{subsec:hpn-particularities}, the
firability conditions are not sufficient for a transition to be
fired. A transition must also be enabled by the residual marking to go
through the firing process. Definition~\ref{def:fired} gives the
formal definition of a fired transition at a given SITPN state:

\begin{definition}[Fired]
  \label{def:fired}
  A transition $t\in{}T$ is said to be fired at the SITPN state
  $s={<}M,I,reset_t,ex,$ $cond{>}$, which is noted $t\in{}Fired(s)$,
  if $t\in{}Firable(s)$ and
  $t\in{}Sens\big(M-\sum\limits_{t_i\in{}Pr(t)}pre(t_i)\big)$, where
  $Pr(t)=\{t_i~|~t_i\succ{}t\wedge{}t_i\in{}Fired(s)\}$.
\end{definition}

One can notice that the definition of the set of fired transitions is
recursive. Indeed, to compute the residual marking necessary to the
definition of a fired transition, the $Pr$ set must be defined. For a
given transition $t$, the $Pr$ set represents all the transitions with
a higher firing priority than $t$ that are also fired transitions;
hence the recursive definition.

In Definition~\ref{def:fired}, the marking
$M-\sum\limits_{t_i\in{}Pr(t)}pre(t_i)$ formally qualifies the
residual marking for a given transition $t$ and at a given SITPN state
$s$.

\subsection{SITPN Semantics}
\label{sec:sitpn-sem}

We formalize the semantics of a given SITPN as a transition
system. The SITPN state transition relation defined in the SITPN
semantics as two cases of definition, one for each clock event.  The
SITPN state transition relation describes the evolution of the state
of a SITPN.

\begin{definition}[SITPN Semantics]
  \label{def:semantics}
  The semantics of a given $sitpn\in{}SITPN$ is the transition system
  ${<}L,E_c,\rightarrow{>}$ where:
  \begin{itemize}[label=-]
  \item $s_0\in{}S(sitpn)$ is the initial state of the SITPN, such
    that
    $s_0=<M_0,O_\mathbb{N},O_\mathbb{B},O_\mathbb{B},O_\mathbb{B}>$,
    where $M_0$ is the initial marking of the SITPN, $O_\mathbb{N}$ is
    a function that always returns 0, $O_\mathbb{B}$ is a function
    that always returns \texttt{false}.
  \item $L\subseteq{}\{\uparrow,\downarrow\}\times{}\mathbb{N}$ is the
    set of transition labels. A label is a couple $(clk,\tau)$
    composed of a clock event $clk\in\{\uparrow,\downarrow\}$, and a
    time value $\tau\in\mathbb{N}$ expressing the current count of
    clock cycles.
  \item
    $E_c\in{}\mathbb{N}\rightarrow\mathcal{C}\rightarrow\mathbb{B}$ is
    the environment function, which gives (Boolean) values to
    conditions ($\mathcal{C}$) depending on the count of clock cycles
    ($\mathbb{N}$).
  \item $\rightarrow\subseteq{}S(sitpn)\times{}L\times{}S(sitpn)$ is the SITPN state
    transition relation, which is noted
    $E_c,\tau\vdash{}s\xrightarrow{clk}s'$ where
    $s,s'\in{}S(sitpn)$ and $(clk,\tau)\in{}L$, and which is defined
    as follows:
    \begin{itemize}[label=$\square$]
    \item $\forall\tau\in\mathbb{N}$, $\forall{}s,s'\in{}S(sitpn)$, we
      have $E_c,\tau\vdash{}s\xrightarrow{\downarrow}s'$, where
      $s=<M,I,reset_t,ex,cond>$ and $s'=<M,I',reset_t,ex',cond'>$, if:
      \begin{enumerate}[label=(\arabic*)]
      \item\label{it:cond-env} $cond'$ is the function giving the
        (Boolean) values of conditions that are extracted from the
        environment $E_c$ at the clock count
        $\tau$, i.e.:\\
        $\forall{}c\in{}\mathcal{C},~cond'(c)=E_c(\tau,c)$.
      \item\label{it:activate-actions} All the actions associated
        with at least one
        marked place in the marking $M$ are activated, i.e.:\\
        $\forall{}a\in{}\mathcal{A},~ex'(a)=\sum\limits_{p\in{}marked(M)}\mathbb{A}(p,a)$
        where $marked(M)=\{p'\in{}P~\vert~M(p')>0\}$.
      \item\label{it:reset-counters} All the time transitions that are
        sensitized by the marking $M$ and received the order to reset
        their time intervals, have their time counter reset and
        incremented, i.e.:\\
        $\forall{}t\in{}T_i,~t\in{}Sens(M)\land{}reset_t(t)=\mathtt{true}
        \Rightarrow{}I'(t)=1$.
      \item\label{it:inc-counters} All the time transitions that are
        sensitized by the marking $M$, and
        did not receive a reset order, increment their time counters if time counters are still active, i.e.:\\
        $\forall{}t\in{}T_i,~t\in{}Sens(M)\land{}reset_t(t)=\mathtt{false}
        \land{}(I(t)\le{}upper(I_s(t))\lor{}upper(I_s(t))=\infty)\Rightarrow{}$
        $I'(t)=I(t)+1$.
      \item\label{it:locked-counters} All the time transitions
        verifying the same
        conditions as above, but with locked counters, keep having locked counters (values are stalling), i.e.:\\
        $\forall{}t\in{}T_i,~t\in{}Sens(M)\land{}reset_t(t)=\mathtt{false}
        \land{}I(t)>{}upper(I_s(t))\land{}upper(I_s(t))\neq\infty\Rightarrow{}$
        $I'(t)=I(t)$.
      \item\label{it:reset-not-sens} All the time transitions disabled by the marking $M$ have their time counters set to zero, i.e.:\\
        $\forall{}t\in{}T_i,~t\notin{}Sens(M)\Rightarrow{}I'(t)=0$.
      \end{enumerate}
    \item $\forall\tau\in\mathbb{N}$, $\forall{}s,s'\in{}S(sitpn)$, we
      have $E_c,\tau\vdash{}s\xrightarrow{\uparrow}s'$, where
      $s=<M,I,reset_t,ex,cond>$ and $s'=<M',I,reset_t',ex',cond>$, if:
      \begin{enumerate}[label=(\arabic*),resume]
      \item\label{it:new-marking} $M'$ is the new marking resulting
        from
        the firing of all the transitions contained in $Fired(s)$, i.e.:\\
        $\forall{}p\in{}P,~M'(p)=M(p)-\sum\limits_{t\in{}Fired(s)}pre(p,t)+\sum\limits_{t\in{}Fired(s)}post(t,p)$.
        
      \item\label{it:reset-order} A time transition receives a reset
        order if it is fired at state $s$, or, if there exists a place
        $p$ connected to $t$ by a \texttt{basic} or \texttt{test arc}
        and at least one output transition of $p$ is fired and the
        transient marking of $p$ disables $t$; no reset order is sent
        otherwise:
        \begin{equation*}
          \begin{split}
            \forall{}t\in{}T_i,~& t\in{}Fired(s) \\
            & \lor\big(\exists{}p\in{}P,\omega\in\mathbb{N}^{*},~pre(p,t)=(\omega,\mathtt{basic})\lor{}pre(p,t)=(\omega,\mathtt{test}) \\
            & \quad\quad\land\sum\limits_{t_i\in{}Fired(s)}pre(p,t_i)>0 \\
            & \quad\quad\land{}M(p)-\sum\limits_{t_i\in{}Fired(s)}pre(p,t_i)<\omega\big)\Rightarrow{}reset'_t(t)=\mathtt{true}, \\
            & and~reset'_t(t)=\mathtt{false}~otherwise. \\
          \end{split}
        \end{equation*}
      \item\label{it:exec-fun} All functions associated with at least one fired transition are executed, i.e:\\
        $\forall{}f\in{}\mathcal{F},~ex'(f)=\sum\limits_{t\in{}Fired(s)}\mathbb{F}(t,f)$.
      \end{enumerate}
    \end{itemize}
  \end{itemize}
\end{definition}

Rules~\ref{it:cond-env} to \ref{it:reset-not-sens} describe the SITPN
state evolution at the falling edge of the clock
signal. Rules~\ref{it:cond-env} and \ref{it:activate-actions} pertain
to the update of condition values and to the update of the activation
status of actions. Note that in Rule~\ref{it:activate-actions} (and
also in Rule~\ref{it:exec-fun}), the sum expression corresponds to the
Boolean sum expression, i.e. the application of the \texttt{or}
operator over the elements of the iterated
set.Rules~\ref{it:reset-counters}, \ref{it:inc-counters},
\ref{it:locked-counters} and \ref{it:reset-not-sens} focus on the
update of time counter values.  In Rule~\ref{it:inc-counters} of the
SITPN semantics, the \emph{active} time counters refer to the time
counters that have not yet overreached the upper bound of their
associated time interval. Of course, a time counter is always active
when the upper bound is infinite. In Rule~\ref{it:locked-counters},
the \emph{locked} time counters refer to the time counters that have
overreached the upper bound of their associated time interval. Of
course, time counters can never be locked in the presence of an
infinite upper bound. In Rules~\ref{it:inc-counters} and
\ref{it:locked-counters}, for a given time interval $i$, $upper(i)$
denotes the upper bound of the time interval, and $lower(i)$ denotes
the lower bound of the time interval.

Rules~\ref{it:new-marking} to \ref{it:exec-fun} describe the SITPN
state evolution at the rising edge of the clock signal.
Rule~\ref{it:new-marking} corresponds to the marking update. The
computation of the new marking uses the set of fired transitions at
state $s$, i.e. $Fired(s)$. Rule~\ref{it:exec-fun} pertains to the
update of the function execution status. Rule~\ref{it:reset-order}
computes the reset orders for time transitions. There are two cases
where a time transition receives the order to reset its time
counter. First, if the transition is one of the fired transitions at
state $s$, then its time counter must be reset on the next falling
edge. Second, if the transition is disabled in a \emph{transient}
manner, then its time counter must also be
reset. Figure~\ref{fig:trans-marking} illustrates the case of a
transition disabled by the \emph{transient} marking, i.e. the marking
obtained after the token consumption phase of the firing process.

\begin{figure}[H]
  \centering
  \includegraphics[keepaspectratio=true, width=.8\textwidth]{Figures/SITPN/trans-marking}
  \caption[Transient marking and reset orders.]{An example of
    transition that receives a reset order after being disabled by the
    transient marking. At \circled{1}, the marking before the firing
    of transitions $t_0$ and $t_2$; at \circled{2}, the transient
    marking; at \circled{3}, the marking at the end of the firing
    process.}
  \label{fig:trans-marking}
\end{figure}

In Figure~\ref{fig:trans-marking}, the situation at \circled{1}
describes the state of the SITPN before a rising edge. Based on the
current SITPN state at \circled{1}, transition $t_0$ and $t_2$ will be
fired on the next rising edge event. % Situation~\circled{2} precedes
% Situation~\circled{3}, but both happen at the occurrence of the rising
% edge of the clock signal.
Situation~\circled{2} depicts the marking obtained after the
consumption phase of the firing process (once the rising edge
occurred), i.e. the so-called \emph{transient}
marking. Situation~\circled{3} corresponds to the marking at the end
of the firing process, where $t_0$ and $t_2$ have been fired. At
\circled{3}, transition $t_1$ is enabled by the marking. However, at
\circled{2}, the transient marking disables $t_1$ and thus $t_1$ must
receive a reset order (represented by a \textcolor{blue}{blue} time
counter).  This reset order will be taken into account at the next
falling edge event, and the time counter associated with transition
$t_1$ will then be reset.

\subsection{SITPN Execution}
\label{sec:sitpn-exec}

As a part of the SITPN semantics, we define here the SITPN execution
and SITPN full execution relations. These relations bind a given SITPN
to the execution trace, i.e. a time-ordered list of states, that it
produces when executed over a given number of clock cycles. These
definitions are additional elements corresponding to our own
contribution to the formalization of the SITPN semantics.

% \begin{definition}[SITPN Execution Cycle]
%   For a given $sitpn\in{}SITPN$, two states $s,s''\in{}S(sitpn)$, a
%   clock cycle count $\tau\in\mathbb{N}$, and an environment
%   $E_c\in\mathbb{N}\rightarrow{}\mathcal{C}\rightarrow{}\mathbb{B}$,
%   $sitpn$ passes from state $s$ to state $s''$ in one clock cycle,
%   written $E,\tau\vdash{}sitpn,s\xrightarrow{\uparrow,\downarrow}s''$
%   iff $\exists{}s'$
%   s.t. $E_c,\tau\vdash{}sitpn,s\xrightarrow{\uparrow}s'$ and
%   $E_c,\tau\vdash{}sitpn,s'\xrightarrow{\downarrow}s''$.
% \end{definition}

\begin{definition}[SITPN execution]
  \label{def:sitpn-exec}
  For a given $sitpn\in{}SITPN$, a starting state $s\in{}S(sitpn)$, a
  clock cycle count $\tau\in\mathbb{N}$, and an environment
  $E_c\in\mathbb{N}\rightarrow{}\mathcal{C}\rightarrow{}\mathbb{B}$,
  $sitpn$ yields the execution trace $\theta$ from starting state $s$,
  written $E_c,\tau\vdash{}sitpn,s\rightarrow{}\theta$, by following
  the two rules below:
  
  \begin{table}[H]
    \begin{tabular}{@{}l}
      {\fontsize{10}{12}\selectfont
      \textsc{ExecutionEnd}} \\
      
      {\begin{prooftree}
          \infer0 {E_c,0\vdash{}sitpn,s\rightarrow{}[~]}
        \end{prooftree}} 
    \end{tabular}
  \end{table}  
  \begin{table}[H]
  \begin{tabular}{@{}l}
    {\fontsize{10}{12}\selectfont
    \textsc{ExecutionLoop}} \\
    
    {\begin{prooftree}[template={\inserttext}]

        \hypo{$E_c,\tau\vdash{}sitpn,s\xrightarrow{\uparrow}s'$}
        \hypo{$E_c,\tau\vdash{}sitpn,s'\xrightarrow{\downarrow}s''$}
        \hypo{$E_c,\tau-1\vdash{}sitpn,s''\rightarrow{}\theta$}
        
        \infer3[$\tau>0$]{$E_c,\tau\vdash{}sitpn,s\rightarrow{}(s' :: s'' :: \theta)$}
      \end{prooftree}} 
  \end{tabular}
\end{table}
\end{definition}

The \textsc{ExecutionEnd} rule states that the execution of a
$sitpn\in{}SITPN$, starting from a state $s\in{}S(sitpn)$ in the
environment
$E_c\in{}\mathbb{N}\rightarrow\mathcal{C}\rightarrow\mathbb{B}$,
yields an empty execution trace if the clock count comes down to $0$.

The \textsc{ExecuteLoop} rule describes how the execution trace
related to the execution of a $sitpn\in{}SITPN$ is built in the case
where the clock count $\tau$ is greater than zero. The final execution
trace is composed of a head state $s'$, followed by state $s''$ and
the tail trace $\theta$. The $::$ operator builds a new trace by
adding a new element at the head of an existing trace. Starting from
state $s$, $sitpn$ reaches state $s'$ after a rising edge event; then
from state $s'$, it reaches state $s''$ after a falling edge event.
Finally, the execution trace $\theta$ is obtained through the
recursive call to the SITPN execution relation where $sitpn$ is
executed during $\tau-1$ cycles starting from state $s''$.


\begin{definition}[SITPN full execution]
  \label{def:sitpn-full-exec}
  For a given $sitpn\in{}SITPN$, a clock cycle count
  $\tau\in\mathbb{N}$, and an environment
  $E_c\in\mathbb{N}\rightarrow{}\mathcal{C}\rightarrow{}\mathbb{B}$,
  $sitpn$ yields the execution trace $\theta$ starting from its
  initial state $s_0\in{}S(sitpn)$ (as defined in
  Definition~\ref{def:semantics}), written
  $E_c,\tau\vdash{}sitpn\rightarrow{}\theta$, by following the two
  rules below:
  
  \begin{table}[H]
    \begin{tabular}{@{}l}
      {\fontsize{10}{12}\selectfont\textsc{FullExec0}} \\
      
      {\begin{prooftree}[template={\inserttext}]
          
          \infer0{$E_c,0\vdash{}sitpn\xrightarrow{full}[s_0]$}
        \end{prooftree}} 
    \end{tabular}
  \end{table}

  \begin{table}[H]
    \begin{tabular}{@{}l}
      {\fontsize{10}{12}\selectfont\textsc{FullExecCons}} \\
      
      {\begin{prooftree}[template={\inserttext}]
          \hypo{$E_c,\tau\vdash{}s_0\srarrow{\uparrow_0}{\fontsize{6}{8}\selectfont}s_0$}
          \hypo{$E_c,\tau\vdash{}s_0\srarrow{\downarrow}{\fontsize{6}{8}\selectfont}s$}
          \hypo{$E_c,\tau-1\vdash{}sitpn,s\rightarrow\theta_s$}
          \infer3[$\tau>0$]{$E_c,\tau\vdash{}sitpn\xrightarrow{full}(s_0 :: s_0 :: s :: \theta_s)$}
        \end{prooftree}} 
    \end{tabular}
  \end{table}
\end{definition}

The \textsc{FullExecCons} rule of the SITPN full execution relation
(Definition~\ref{def:sitpn-full-exec}) appeals to the SITPN execution
relation (Definition~\ref{def:sitpn-exec}). However, the definition of
the SITPN full execution relation is necessary because the first cycle
of execution, starting from the initial state $s_0$, is particular. As
shown in the premises of Rule~\textsc{FullExecCons}, the first rising
edge is idle. We consider that no transitions are fired during the
first rising edge. Thus, the first rising edge does not change the
initial state $s_0$, and we denote the particular first rising edge
with the sign $\uparrow_0$ over the SITPN transition relation.

\subsection{Well-definition of a SITPN}
\label{sec:sitpn-wd}

To be able to transform a given SITPN into a \vhdl{} design and also
to perform the proof of semantic preservation, a SITPN must verify
some properties ensuring its \emph{well-definition}. Here, we
formalize the predicate stating that a given SITPN is well-defined.

% \paragraph{Conflict Definition} In the definition of an SITPN, the
% priority relation is a mean to solve a situation of conflict in a pair
% of transitions. We will keep the definition of a conflict as simple as
% possible. Informally, the transitions of a pair are in conflict if
% they have an common input place, and if both are linked to this input
% place by a \texttt{basic} arc. Figure~\ref{fig:basic-conflict} depicts
% a situation of conflict between two transitions.

% At some point of the execution of the SITPN, the marking possibly
% enables the two transitions of a conflicting pair in such a manner
% that the firing of one transition disables the other; then, the
% conflict is said to be \emph{effective}. The behavior of PNs is
% fundamentally asynchronous, and a token can only be consumed by one
% transition. However, in a synchronous setting as the one of the SITPN,
% all transitions are first elected to be fired, and then all fired at
% the same time.  Therefore, the situation can arise where a same token
% is consumed by two transitions, on behalf of them being transitions in
% effective conflict that are both elected to be fired (e.g,
% Figure~\ref{fig:basic-conflict}). 

% \begin{figure}[H]
%   \centering
%   \includegraphics[keepaspectratio,width=.3\linewidth]{Figures/SITPN/struct-conflict-with-basic}
%   \caption{Example of conflict between two transitions}
%   \label{fig:basic-conflict}
% \end{figure}


The main interest of the well-definition predicate is to prevent the
phenomenon of the ``double consumption'' of tokens at the execution of
a SITPN. In a well-defined SITPN, a conflict resolution strategy must
be applied to every group of transitions in structural conflict.  We
must be able to decide which transition in a conflicting pair will be
fired when the conflict becomes effective. Thus, we give the formal
definition of a conflicting pair of transitions and of a conflict
group.

\begin{definition}[Conflict]
  \label{def:conflict}
  For a given $sitpn\in{}SITPN$, two transitions $t,t'\in{}T$ are in
  conflict if there exist a place $p\in{}P$ and two weights
  $\omega,\omega'\in\mathbb{N}^{*}$ such that
  $pre(p,t)=(\omega,\mathtt{basic})$ and
  $pre(p,t')=(\omega',\mathtt{basic})$.
\end{definition}

A conflict group qualifies a finite set of transitions that are all in
conflict with each other through at least a common input place. In
Figure~\ref{fig:conflict-not-trans}, the set $\{t_0,t_1\}$ is a
conflict group.  The formal definition of a conflict group is as
follows:

\begin{definition}[Conflict Group]
  \label{def:cgroup}
  For a given $sitpn\in{}SITPN$, $T_c\subseteq{}T$ is a conflict group
  if there exists a place $p$ such that
  $\forall{}t\in{}T,\big(\exists{}\omega\in\mathbb{N}^{*},~pre(p,t)=(\omega,\mathtt{basic})\big)\Leftrightarrow{}t\in{}T_c$.
\end{definition}

Contrary to the statement made in \cite[p. 67]{Leroux2014}, we no more
consider the notion of conflict as being transitive. To illustrate
this, Figure~\ref{fig:conflict-not-trans} shows two conflict groups:
$\{t_0,t_1\}$ and $\{t_1,t_2\}$. In a well-defined $SITPN$ (see
Section~\ref{sec:sitpn-wd}), all conflicts in a conflict group must be
dealt with, i.e. for all pair of transitions in the group the conflict
must be solved. However, we no more consider transitions $t_0$ and
$t_2$ as in conflict. We argue that even when no conflict resolution
technique is applied between transitions in the same situation as
$t_0$ and $t_2$, the execution of the $SITPN$ can neither result in
the double-consumption of a token, nor in the case where a transition
is not elected to be fired even though it ought to be. Therefore, we
no more consider the construction of merged conflict group (i.e,
conflict groups must be merged into one if their intersection is not
empty; e.g, $\{t_0,t_1,t_2\}$ in Figure~\ref{fig:conflict-not-trans})
as being necessary.

\begin{figure}[H]
  \centering
  \includegraphics[keepaspectratio,width=.4\linewidth]{Figures/SITPN/conflict-not-trans}
  \caption[An example of two separate conflict groups.]{An example of
    two separate conflict groups, namely: $\{t_0,t_1\}$ and
    $\{t_1,t_2\}$.}
  \label{fig:conflict-not-trans}
\end{figure}

When the conflict between a pair of transitions becomes effective,
there are two ways to be sure that only one transition will be
fired. The first way is to define a firing order through a priority
relation. The second way is to use a mean of mutual exclusion. A mean
of mutual exclusion ensures that the two transitions of a conflicting
pair will never be firable at the same time. For now, we only consider
two ways of mutual exclusion, namely: mutual exclusion with
complementary conditions and mutual exclusion with inhibitor
arcs. Here, we give the formal definition of these two means of mutual
exclusion.

% \begin{definition}[Mutual exclusion with disjoint time intervals]
%   \label{def:mutex-ti}
%   Given two conflicting transitions $t_0$ and $t_1$, $t_0$ and $t_1$
%   are in mutual exclusion with disjoint time intervals if there exists
%   $a,b\in\mathbb{N}^{*}$ and $c,d\in\mathbb{N}^{*}\sqcup\{\infty\}$
%   such that $I_s(t_0)=[a,b]$ and $I_s(t_1)=[c,d]$ and there is no
%   overlapping between $[a,b]$ and $[c,d]$.
% \end{definition}

\begin{definition}[Mutual exclusion with complementary conditions]
  \label{def:mutex-conds}
  Given two conflicting transitions $t_0$ and $t_1$, $t_0$ and $t_1$
  are in mutual exclusion with complementary conditions if there
  exists $c\in\mathcal{C}$ such that
  $(\mathbb{C}(t_0,c)=1\land{}\mathbb{C}(t_1,c)=-1)$ or
  $(\mathbb{C}(t_0,c)=-1\land{}\mathbb{C}(t_1,c)=1)$.
\end{definition}

\begin{definition}[Mutual exclusion with an inhibitor arc]
  \label{def:mutex-inhib} Given two conflicting transitions $t_0$ and
  $t_1$, $t_0$ and $t_1$ are in mutual exclusion with an inhibitor arc
  if there exists $p\in{}P$ and $\omega\in{}\mathbb{N}^{*}$ such that
  $(pre(p,t_0)=(\omega,\mathtt{basic})\lor{}pre(p,t_0)=(\omega,\mathtt{test}))\land{}pre(p,t_1)=(\omega,\mathtt{inhib})$
  or
  $(pre(p,t_1)=(\omega,\mathtt{basic})\lor{}pre(p,t_1)=(\omega,\mathtt{test}))\land{}pre(p,t_0)=(\omega,\mathtt{inhib})$.
\end{definition}

Figure~\ref{fig:mutex} illustrates the two means of mutual exclusion
that can be applied to solve a conflict between two transitions.

\begin{figure}[H]
  \centering
  \includegraphics[keepaspectratio,width=.5\linewidth]{Figures/SITPN/mutex}
  \caption[Examples of conflicting transitions in mutual exclusion.]{
    Examples of conflicting transitions in mutual exclusion. At
    \circled{1}, an example of mutual exclusion with complementary
    conditions; at \circled{2}, an example of mutual exclusion with an
    inhibitor arc.}
  \label{fig:mutex}
\end{figure}

In Figure~\ref{fig:mutex}, in situation \circled{1}, condition $c_1$
is associated to $t_1$ and the complementary condition is associated
to $t_0$ thus creating the mutual exclusion. In situation \circled{2},
the arcs $(p_0,t_0)$ and $(p_0,t_1)$ ensure the mutual exclusion
between transitions $t_0$ and $t_1$. Note that in the structure of
mutual exlcusion with an inhibitor arc, the weight of the inhibitor
arc and of the basic or test arc must be the same; otherwise, the
mutual exclusion is not effective.

A given $sitpn\in{}SITPN$ is well-defined if it enforces some
properties needed on the \hilecop{} source models before the
transformation into \vhdl{}. If the properties, layed out in
Definition~\ref{def:wd-sitpn}, are not ensured, they will lead to
compile-time errors during the transformation of the SITPN into a
\vhdl{} design.

\begin{definition}[Well-defined SITPN]\label{def:wd-sitpn}
  A given $sitpn\in{}SITPN$ is well-defined if:
  \begin{itemize}
  \item $T\neq\emptyset$, the set of transitions must not be empty.
  \item $P\neq\emptyset$, the set of places must not be empty.
  \item There is no isolated place, i.e, a place that has neither
    input nor output transitions:\\
    $\nexists{}p\in{}P,~input(p)=\emptyset\wedge{}output(p)=\emptyset$,
    where $input(p)$ (resp. $output(p)$) denotes the set of input
    (resp. output) transitions of $p$.
  \item There is no isolated transition, i.e, a transition that has
    neither
    input nor output places:\\
    $\nexists{}t\in{}T,~input(t)=\emptyset\wedge{}output(t)=\emptyset$,
    where $input(t)$ (resp. $output(t)$) denotes the set of input
    (resp. output) places of $t$.
  \item For all conflict group as defined in
    Definition~\ref{def:cgroup}, either all conflicts (i.e. for all
    pair of transitions in the conflict group) are solved by one of
    the mean of mutual exclusion, or, the priority relation is a
    strict total order over the transitions of the conflict group.
  \end{itemize}
\end{definition}

\subsection{Boundedness of a SITPN}
\label{sec:sitpn-bounded}

We conclude the formalization of the SITPN structure and semantics by
the expression of the boundedness of a SITPN model with respect to its
execution trace. In the manner of the well-definition property, the
boundedness of a SITPN model is a mandatory condition to apply the
semantic preservation theorem (cf. Remark~\ref{rem:bounded-sitpn} in
Chapter~\ref{chap:proof}). A SITPN model is bounded if there exists a
\textit{bound} for the number of tokens that the places can hold in
the course of the execution of the model; formally:

\begin{definition}[Bounded SITPN]
  \label{def:bounded-sitpn}
  A given $sitpn\in{}SITPN$ is said to be bounded if for all execution
  environment
  $E_c\in\mathbb{N}\rightarrow\mathcal{C}\rightarrow\mathbb{B}$, clock
  cycle count $\tau\in\mathbb{N}$, execution trace
  $\theta\in\mathtt{list}(S(sitpn))$ such that
  $E_c,\tau\vdash{}sitpn\xrightarrow{full}\theta$, then there exists a
  bound $k\in\mathbb{N}$ such that for all $p\in{}P$ and
  $s\in{}\theta$, $s.M(p)\le{}k$.
\end{definition}

We extend the definition of a bounded SITPN model to a version where
the bound denoting the maximal marking of each place of the model is
passed through a function $b\in{}P\rightarrow\mathbb{N}$.

\begin{definition}[Bounded SITPN through a maximal marking function]
  A given $sitpn\in{}SITPN$ is said to be bounded through the maximal
  marking function $b\in{}P\rightarrow\mathbb{N}$, written
  $\lceil{}sitpn\rceil^b$, if for all execution environment
  $E_c\in\mathbb{N}\rightarrow\mathcal{C}\rightarrow\mathbb{B}$, clock
  cycle count $\tau\in\mathbb{N}$, execution trace
  $\theta\in\mathtt{list}(S(sitpn))$ such that
  $E_c,\tau\vdash{}sitpn\xrightarrow{full}\theta$, then for all
  $p\in{}P$ and $s\in{}\theta$, $s.M(p)\le{}b(p)$.
\end{definition}

%%% Local Variables:
%%% mode: latex
%%% TeX-master: "../../main"
%%% End:
