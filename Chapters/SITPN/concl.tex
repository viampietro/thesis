The class of SITPNs is a particular class of PNs used to model the
behavior of components in the \hilecop{} high-level models. The
synchronous evolution of SITPNs constitutes the originality of the
model compared to the standard PNs semantics.  In this chapter, we
gave an informal and formal presentation of SITPNs and their execution
semantics. Two previous Ph.D. theses contributed, for the most part,
to the formalization of the SITPN structure and semantics. However, we
helped simplify the semantics of SITPNs. We passed from 14 rules in
the definition of the SITPN semantics given in \cite{Leroux2014} to 9
rules in our current definition of semantics. Also, as presented in at
the end of Section~\ref{sec:sitpn-sem}, we completed some rules when
they happened to be insufficient to prove the theorem of behavior
preservation.  Finally, we defined the execution relations for the
SITPN semantics and formalized the well-definition property for the
SITPN structure.

Our other contribution was to implement the SITPN structure and
semantics with the \coq{} proof assistant. There are two
implementations: one with and one without dependent types. For the
version without dependent types, we implemented a SITPN interpreter or
token player. We also proved a soundness and completeness theorem
between the interpreter and the formalized SITPN semantics. The first
implementation of the SITPNs in \coq{} represents 5000 lines of
specification and 7000 lines of proof. The second implementation,
which was presented in Section~\ref{sec:sitpn-impl}, leverages
dependent types. This implementation is closer to the formal
definition given in Definition~\ref{def:sitpn}. We chose this
implementation to mechanize the proof of the behavior preservation
theorem (see Chapter~\ref{chap:proof}).

%%% Local Variables:
%%% mode: latex
%%% TeX-master: "../../main"
%%% End:
