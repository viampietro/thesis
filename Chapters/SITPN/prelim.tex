In this section, the reader is assumed to have of previous knowledge
of the Petri net formalism. For more information on the topic of Petri
nets, see \cite{david:alla}, \cite{murata}, or \cite{diaz}. Here,
fundamentals on the Petri net formalism are briefly outlined, and
certain classes of Petri nets are described more precisely. Then, the
specificities of Petri nets used to design the behavior of electronic
component architecture in the Hilecop software are presented.

\subsection{Classes of Petri nets}
\label{subsec:pn-formalism}

Petri nets, invented by C. A. Petri \cite{petri}, are used to model a
broad range of dynamic systems: resource sharing between concurrent
processes \cite{david:alla}, behavior of agents in multi-agent systems
\cite{celaya:desrochers}, behavior of digital components
\cite{yakovlev:gomes}. A Petri net is a directed graph, composed of
two types of node: place nodes (\emph{circles}) and transition nodes
(\emph{squares} or \emph{lines}). As shown in figure
\ref{fig:pn-example}, place nodes usually represent a part of the
state of the modelled system, here the states of two computer
processes and a semaphor; transition nodes usually refer to events
triggering the system evolution (or state changing).

\begin{figure}[H]
  \centering
  \includegraphics[keepaspectratio=true, width=\textwidth]{Figures/SITPN/pn-example}
  \caption[An example of Petri net]{An example of Petri net - A
    semaphor to prevent the parallel execution of \textit{treatment 1}
    and \textit{treatment 2}.}
  \label{fig:pn-example}
\end{figure}

In a Petri net, the \textit{marking}, which corresponds to the
distribution of tokens over places --represented by little black
circles-- denotes the current state of the system. The system evolves
thanks to transition firing, responsible for the
consumption/production of tokens in a Petri net.  The purpose of
Hilecop Petri nets is to model the behavior of electronic components
involved in an electronic system architecture.  Therefore, to meet the
requirements of electronic system modelling, Hilecop Petri nets
combine the properties of multiple classes of Petri nets, which are
presented here.

Some formalization is necessary to explain the process of transition
firing. Let $P$, the set of places, $T$, the set of transitions,
$pre: P \times T \rightarrow N$, a weight function over transition
incoming arcs, $post: T \times P \rightarrow N$, a weight function
over transition outcoming arcs, and $M : P \rightarrow N$, the marking
function. Thus, a Petri net $PN$ is an uplet
$PN=<P, T, pre, post, M_0>$, where $M_0$ is the initial marking of the
net. Let $S = <PN, M>$ denotes the state of Petri net $PN$ where $M$
is the current marking of the net. First, let us define the concept of
sensitized transition.

\begin{definition}
  Let $PN = <P, T, pre, post, M_0>$, a Petri net and $M$, a marking function.
  A transition $t \in T$ is sensitized at state $S= <PN, M>$ if
  $\forall p \in P$, $ pre(p, t) \le M(p)$.
\end{definition}

Then, let us define the process of transition firing.

\begin{definition}
  Let $PN = <P, T, pre, post, M_0>$, a Petri net, and $M$, $M'$ two
  marking functions.  The firing of transition $t \in T$ leads from
  state $S = <PN,$\\$ M>$ to state $S' = <PN, M'>$ if $\forall p \in
  P, M'(p) = M(p) - pre(p, t) + post(t, p)$.
\end{definition}


\paragraph{Interpreted Petri nets.}
Interpreted Petri nets (IPN) are useful to design the interaction
between a system and its outside environment. As they describe
electronic systems which are not completely cut from the outside
world, Hilecop Petri nets are in need of interpretation.
Thus, interpretation introduces three new concepts:

\begin{itemize}
\item Continuous actions, associated to the places of a Petri net.
  Actions associated to a place $p$ are activated as long as $p$ is
  marked. Actions correspond to the setting of a electric signal
  controlling some actuator (e.g, maintaining a LED on).
\item Functions (or discrete actions), associated to the transitions
  of a Petri net. When a transition $t$ is fired, all functions
  associated to $t$ are executed.  Functions can be any kind of
  discrete operations --variable incrementation, for instance--
  manipulating both internal variable and external signal values.
\item Conditions, associated to the transitions of a Petri net.
  Conditions are boole\-an expressions, calculated using both internal
  variable and external signal values.  If the value of condition $c$
  associated to transition $t$ is \textit{false} then transition $t$
  is not firable.
\end{itemize}

The figure \ref{fig:pn-classes}.a illustrates the use of actions,
functions and conditions in an interpreted Petri net.

\begin{figure}[H]
  \centering
  \subfloat[Interpreted Petri net.]{
    \includegraphics[keepaspectratio=true, width=0.5\textwidth]{Figures/SITPN/interpreted-pn}
  }\hspace{8pt}
  \subfloat[Time Petri net]{
    \includegraphics[keepaspectratio=true, width=0.2\textwidth]{Figures/SITPN/time-pn}
  }
  \caption{Different classes of Petri nets.}
  \label{fig:pn-classes}
\end{figure}%
%
In figure \ref{fig:pn-classes}, the action $a_0$ is activated as
place $P_0$ is marked by one token. Also, function $f_0$ will be
executed at the firing of $T_0$, that is if condition $c_0$ is
\textit{true} and $T_0$ is sensitized.

\paragraph{Time Petri nets.}

In a time Petri net (TPN), time intervals, and time counters are
associated to transitions. The goal of associating a time interval to
a transition is to constraint the firing of this transition to a
certain time window. As shown in figure \ref{fig:pn-classes}.b, time
intervals are of the form $[a, b]$, where $a \in \mathbb{N}^*$ and
$b \in \{\mathbb{N}^* \cup \infty\}$. Time counters are represented
between diamond brackets.  For each sensitized transition associated
with a time interval, time counters are incremented at a certain time
step, previously defined by the modeller. Then, time counters are
reset when transitions are fired or somehow disabled. In time Petri
nets, a transition is firable only if its time counter value is within
its time interval. For instance, in figure, only transition $T_0$ is
firable.

\paragraph{Edges.}

In a Petri net, directed edges link together places and
transitions. Places cannot be linked to other places, and the same
stands for transitions.  There are two kinds of edges, \textit{pre} or
\textit{incoming} edges, going from a place to a transition, and
\textit{post} or \textit{outcoming} edges, going from a transition to
a place. Places linked to a transition $t$ by incoming
(resp. outcoming) edges will be referred to as \textit{pre-places}
(resp. \textit{post-places}) of $t$. Some weight --a natural number--
is associated to the edges of a Petri net. If no label appears on the
edge then one is the default weight. Petri nets are said to be
\textit{generalized} when edge weights are possibly greater than one.

\paragraph{Marking.}
In figure \ref{fig:pn-example}, places $P_0$, $P_3$ and $Sem$ are
marked with tokens, represented by little black circles.  This means
that places $P_0$, $P_3$ and $Sem$ are currently active.  The
distribution of tokens over places is called the \textit{marking} of
the net. The marking of a Petri net reflects the overall state of the
modelled system at a certain moment in its activity cycle.

\paragraph{Transition firing.}
In a Petri net, the marking evolves based on a token
consump\-tion-production system. Transitions will consume tokens in
incoming places, and produce tokens in outcoming places. This whole
system is called \textit{transition firing}. In order to be firable, a
transition must be \textit{sensitized} (or \textit{enabled}), meaning
that the number of tokens in each of its incoming places must be equal
or greater than the weight of its incoming edges. For instance, in
figure 1, the transition $T_0$ is sensitized because the weight of the
arc ($P_0$, $T_0$) is of one (default value), and place $P_0$ is
marked with one token. As a counter example, transition $T_4$ is not
sensitized. Indeed, there is only one token in the place $Sem$, where
at least two were required plus one token in place $P_4$.  Other
parameters affect the firability of transitions, some of them will be
presented in section \ref{subsec:hpn-particularities}. When a sensitized
transition is fired, tokens are retrieved from its incoming places (as
much tokens as the weight of the arcs) and produced in its outcoming
places (as much tokens as the weight of the arcs).  This process
represent the raise of an event --denoted by the transition--
triggering the passage from one state of the system to another.

\paragraph{Inhibitor and test edges.}
The class of \textit{extended} Petri nets introduces
the inhibitor and test edges. As shown in figure
\ref{fig:inhib-test-arcs}, test arc tips are black circles and
inhibitor arc tips are white circles. Inhibitor and test edges are
incoming edges, always coming from a place toward a transition.
%
\begin{figure}[H]
  \centering
  \subfloat[]{
    \includegraphics[keepaspectratio=true, width=0.2\textwidth]{pictures/inhib-arc}
  }
  \hspace{50pt}
  \subfloat[]{
    \includegraphics[keepaspectratio=true, width=0.2\textwidth]{pictures/test-arc}
  }
  
  \caption{An example of inhibitor and test arcs.}
  \label{fig:inhib-test-arcs}
\end{figure}
The particularity of the inhibitor and test edges is that they are not
consuming tokens in pre-places after the firing of a transition.
Indeed, they are just testing the number of tokens in incoming places
to determine if the transition is enabled. Inhibitor arcs ensure that
the number of tokens in pre-places is strictly lower than their
weights; test arcs ensure that the number of tokens in pre-places is
equal or greater than their weights. Therefore, in figure
\ref{fig:inhib-test-arcs}.a, transition $T_0$ is sensitized because
there is strictly less than one token in place $P_0$ and strictly less
than two tokens in place $P_1$. In the same way, in figure
\ref{fig:inhib-test-arcs}.b, transition $T_0$ is sensitized because
there is at least one token in place $P_0$ and three tokens in place
$P_1$.

\subsection{Particularities of Hilecop Petri nets}
\label{subsec:hpn-particularities}

This section goal is to show how the choices regarding the properties
of Hilecop Petri nets are strongly related to their final FPGA-based
implementation.
 
% To understand the meaning of internal and external
% variable values --$heat\_sensor\_value$, $x$ and $y$, in figure
% \ref{fig:pn-classes}-- used by actions, functions and conditions,
% one as to remember that in its final state, an electronic component,
% described at a high-level with Hilecop Petri nets, will be handling
% \textit{signals} on a FPGA card. Some signals will be provided by the
% outside world --sensors, or by any component involved in a
% distributed architecture--; some signals represent internal variables
% handled by the component.

\paragraph{Synchronous execution.}

A clock signal regulates the evolution of Hilecop Petri nets, meaning
that the evolution of an Hilecop Petri net is \textit{synchronized}
with two clock events: the rising edge and the falling edge of the
signal. Figure \ref{fig:sync-exec} depicts the process of transition
firing and global state evolution, following the clock signal.

\begin{figure}[H]
  \centering
  \includegraphics[keepaspectratio=true, width=1.0\textwidth]{Figures/SITPN/sync-exec}
  \caption{Evolution of the Hilecop Petri nets synchronized with a clock signal.}
  \label{fig:sync-exec}
\end{figure}

As shown in figure \ref{fig:sync-exec}, the marking evolution process
is divided in two steps. First, on \textcircled{1}, firable
transitions are collected and actions are either started or stopped
depending on the marking of the associated places. Then, on
\textcircled{3}, all previously collected firable transitions are
fired, and the associated functions are executed. The implementation
of places and transitions on FPGA is responsible for the separation of
transition firing in two steps \cite{leroux}.

Why do we need a synchronous execution for Hilecop Petri nets?  While
functions execute themselves instantaneously at an abstract level, it
is not the case when implemented on FPGA. Indeed, on FPGA, the
execution of functions takes a certain amount of time related to the
propagation of electric signals inside the electronic circuit. Figure
illustrates the effect of the time taken by the execution of function
$f_0$.
\begin{figure}[H]
  \centering
  \includegraphics[keepaspectratio=true,width=0.9\textwidth]{Figures/SITPN/async-exec-issue}
  \caption{Effect of the duration of functions in firing decisions \cite{leroux}. }
  \label{fig:async-exec-issue}
\end{figure}

As it is standard in Petri net theory, the marking evolves in an
\textit{asynchronous} way, that is, transitions are fired one at a
time. In figure, $T_0$ is fired before $T_1$, which triggers the execution
of $f_0$. Therefore, two cases arise:
\begin{itemize}
\item $f_0$ has not completed yet when the decision to fire $T_1$ is
  taken, although the execution of $f_0$ affects the firing $T_1$.
\item $f_0$ has completed when the decision to fire $T_1$ is taken.
\end{itemize}
One can see how the amount of time taken by the execution of functions
influences the evolution of a Petri net implemented on FPGA.  If the
evolution of Hilecop Petri nets is asynchronous, there are no
automatic means to ensure that $f_0$ completes every time before the
firing of $T_1$ \cite{leroux}.  At the end of the Hilecop production
line, when Petri nets are implemented on FPGA, it is possible
leveraging an electronic circuit analyzer to determine the longest
signal propagation path of the system. Then, it is easy to deduce an
upper time bound for all treatments --which are nothing more than
signal propagations-- taking place in the physical system. Eventually,
the clock signal frequency will be set accordingly, to permit that all
functions have enough time to be completely executed within half a
clock period (\textcircled{4} in figure \ref{fig:sync-exec}). Thus the
use of a clock signal and the orientation of Hilecop Petri nets toward
a synchronous execution are mandatory.

\paragraph{Conflict resolution.}

The choice of a synchronous evolution is not without
consequences. Indeed, some issues arise when trying to express an
\textit{or} branching with Hilecop Petri nets, as shown by figure
\ref{fig:structural-conflict}.a.
\begin{figure}[H]
  \centering
  \includegraphics[keepaspectratio=true,width=.4\textwidth]{Figures/SITPN/structural-conflict}
  \caption{Transitions in structural conflict.}
  \label{fig:structural-conflict}
\end{figure}
The semantics of synchronous execution is that all transitions are
fired at the same time. Then, in figure \ref{fig:structural-conflict}.a,
transitions $T_0$ and $T_1$ are both sensitized by place $P_0$, and
consequently are both fired at the same time. The system acts as if
two tokens were available in place $P_0$, one for the firing of $T_0$
and another for the firing of $T_1$. Such a behavior is indeed
erroneous. Moreover, the expression of an \textit{or} branching leads
to undeterminism if one is not able to tell in any case which
transition will consume the token. Such an assumption is dangerous,
all the more considering the design of electronic components involve
in critical systems. In the situation depicted by figure
\ref{fig:structural-conflict}.a, $T_0$ and $T_1$ are said to be in
\textit{structural conflict} with each other, meaning they have one of
their pre-places in common. To resolve structural conflicts,
priorities are drawn between transitions which will determine a firing
order in case of conflict. In figure \ref{fig:structural-conflict}.b,
the dotted arrow represents the priority relation between transition
$T_0$ and $T_1$.  Here, $T_0$ has a higher firing priority than $T_1$.

%%% Local Variables:
%%% mode: latex
%%% TeX-master: "../../main"
%%% End:
