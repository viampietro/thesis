\chapter{The \hilecop{} methodology}
\label{chap:hilecop}

The \hilecop{} methodology consists of a process for the design and
implementation of critical digital systems. This methodology relies on
several transformations going from abstract models to concrete FPGA
implementations through the production of VHDL
code. Fig.~\ref{fig:hilecop-wf} details the global workflow of
\hilecop{}. In this figure, Step~1 corresponds to the model of a
digital system. This model is built with component diagrams and the
behavior of each component is described by means of
PNs. Fig.~\ref{fig:components-and-pn} provides an example of such
model. As shown in this figure, an internal behavior and an interface
outline the structure of components. The component interface exposes
places, transitions, and signals, which are references to nodes of the
internal behavior, from which the components can be assembled to get
the global behavior of the digital system.
%
\begin{figure}[t]
\centering
\includegraphics[keepaspectratio=true,width=\textwidth]{Hilecop/hilecop-wf.eps}
\caption{Workflow of the \hilecop{} Methodology}
\label{fig:hilecop-wf}
\end{figure}
%
\begin{figure}[t]
\centering
\includegraphics[keepaspectratio=true,width=0.8\textwidth]
{Hilecop/components-and-pn.eps}
\caption{An Example of \hilecop{} Model}
\label{fig:components-and-pn}
\end{figure}

Next, in Fig.~\ref{fig:hilecop-wf}, the transformation from Step~1 to
Step~2 flattens the model. The internal behaviors are connected
according to the interface compositions, and embedding component
structures are removed. The analysis phase, going from Step~2 to
Step~1, aims to produce a model that is conflict-free (see
Sec.~\ref{sec:sitpn-struct} for more details about the definition of a
conflict), bounded, and deadlock-free, using model-checking
techniques.  After several iterations, the model should reach
soundness and is then said to be implementation-ready.

From Step~2 to Step~3, VHDL source code is then generated by means of
a model-to-text transformation. This generated code describes the
hardware system that will be implemented in a FPGA circuit. From
Step~3 to Step~4, the VHDL compilation/synthesis and the FPGA
programming are finally performed using industrial tools.

In this work, we focus on the part of the workflow in Fig~\ref{fig:hilecop-wf}
that is framed with dotted lines, i.e. the model-to-text transformation between
Step~2 and Step~3. In particular, we aim to prove that through this
model-to-text transformation, the behavior described by the initial model is
preserved in the generated VHDL code. To do so, we have to implement the
structure of PNs used in the \hilecop{} models, together with the semantics
providing the evolution rules of these PNs.

%%% Local Variables:
%%% mode: latex
%%% TeX-master: "../main"
%%% End:
