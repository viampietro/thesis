\chapter{The \hilecop{} methodology}
\label{chap:hilecop}

In this chapter, we present the context of our work, and more
specifically, the subject of our verification task, i.e. \hilecop{}, a
methodology for the design and implementation of critical digital
systems.  In Section~\ref{sec:design-crit-digit-systms}, we motivate
the use of Model-Based Systems Engineering (MBSE) and formal methods
in the design and production of safety-critical digital systems; in
Section~\ref{sec:intro-hilecop}, we give an overall presentation of
the \hilecop{} methodology which applies both the principles of MBSE
and formal methods; in Section~\ref{sec:verif-hilecop}, we point out
the specific transformation phase, intervening in the \hilecop{}
methodology, that we propose to verify and discuss on how we propose
to verify it and which research questions does it give rise to.

\section{Designing critical digital systems}
\label{sec:design-crit-digit-systms}

According to Moore's law \cite{Moore2006}, the complexity of digital
integrated circuits is always increasing. To give an example, the
cut-of-the-edge \emph{AMD Epyc Rome} microprocessor (2019) is made out
of 50 billions of transistors. Composing billions of transistors on a
wired circuit is no more a task for humans but is very suited to
computers. However, engineers need to think about the design of
digital circuits in a way that is understandable for
humans. Therefore, they need high-level views of the circuits they are
designing in order to work together and to communicate about the
designs. The domain of Model-Based Systems Engineering (MBSE)
\cite{Long2011} proposes a framework to help engineers to design and
produce digital circuits, in a well-documented, safe and reliable
way. Comparable to what Model Driven Engineering (MDE) does in the
world of software engineering, models are first order concepts in
MBSE. A model represents a simplified view of real object. As
illustrated in Figure~\ref{fig:MBSE-ps}, a MBSE process describes a
way to design a digital circuit starting from a high-level view of the
system. This high-level view can follow a graphical formalism such as
SysML \cite{Friedenthal2014} or Petri nets \cite{Petri1962}, or a
textual one such as SystemC \cite{Black2009} or VHDL
\cite{Ashenden2010}. Then, the MBSE process describe many refinements
phases (the green arrows in Figure~\ref{fig:MBSE-ps}) during which the
input model will be transformed; at each refinement phase, the model
goes down in abstraction towards its final implementation as a
hardware circuit. A refinement phase, which is also a transformation
phase, can be performed automatically or manually.

\begin{figure}[H]
  \centering
  \includegraphics[keepaspectratio,width=.6\textwidth]{Figures/Hilecop/MBSE-ps}
  \caption[A Model-Based Systems Engineering process.]{A Model-Based
    Systems Engineering process; REQ stands for requirements, BEH for
    behavior, ARCH for architecture, Dgn V\&V for design verification
    and validation. This figure is an excerpt from \cite{Long2011}.}
  \label{fig:MBSE-ps}
\end{figure}

In the case where the digital circuit being designed is a
safety-critical system, i.e. on its behavior depends the life of
people, an MBSE process will often employ formal models, i.e. models
with a formal mathematical definition, as the design formalism.  Thus,
these models enable a certain extent of mathematical reasoning to
prove that safety properties are met during the design V\&V phase
(cf. Figure~\ref{fig:MBSE-ps}).

\section{Introducing the \hilecop{} methodology}
\label{sec:intro-hilecop}

The INRIA CAMIN team (former DEMAR team) has developed a new
technology of neuroprotheses \cite{Guiraud2006}. Neuroprotheses are
medical devices which purpose is to electro-stimulate the nerves of
patients suffering from moving disabilities. The nerves are responding
to the stimulation, i.e an electric influx, in order to activate the
muscles and so that the patient can recover some movements. Thus,
controlling the intensity and the form of the electric signal sent to
the patient's nerve is critical point of the device overall
functioning. These two parameters are controlled by a digital hardware
circuit (i.e. a microcontroller) that is a part of the
neuroprosthesis. Therefore, the design of such digital systems becomes
utterly critical as a faulty circuit could result in the injury of
patients. To assist the engineers in the design and the implementation
of these critical digital systems, the CAMIN team came up with a
process called the ``\hilecop{} methodology'' \cite{Andreu2009}.  This
methodology follows the principles of a MBSE process and relies on
several transformations going from abstract models to concrete FPGA
implementations through the production of VHDL
code. Figure~\ref{fig:hilecop-wf} details the global workflow of
\hilecop{}. 

\begin{figure}[H]
\centering
\includegraphics[keepaspectratio=true,width=\textwidth]{Figures/Hilecop/hilecop-wf}
\caption[Workflow of the \hilecop{} methodology.]{Workflow of the
  \hilecop{} methodology; horizontal double arrows indicate the
  transformation phases, i.e. the refinement phases in MBSE terms;
  simple arrows indicate different kinds of operations performed at a
  given step.}
\label{fig:hilecop-wf}
\end{figure}

In Figure~\ref{fig:hilecop-wf}, Step~1 corresponds to the design phase
of a critical digital system. At this step, the engineers produce a
model of the wanted system; the leveraged model formalism is a
graphical formalism specially designed for the methodology and based
on component diagrams. Figure~\ref{fig:components-and-pn} provides an
example of such a model.
%
\begin{figure}[H]
\centering
\includegraphics[keepaspectratio=true,width=\textwidth] {Figures/Hilecop/abs-model}
\caption[An example of \hilecop{} high-level model.]{An Example of
  \hilecop{} high-level model. Black diamonds represent \vhdl{}
  signals.}
\label{fig:components-and-pn}
\end{figure}
%
As shown in Figure~\ref{fig:components-and-pn}, a component of the
\hilecop{} high-level model formalism is represented by a box having
an internal behavior and an interface that permits the connection to
other components. The internal behavior of a component is defined with
a specific kind of Petri Net (PN) model. These PNs and their
distiguishing features will be thoroughly presented in
Chapter~\ref{chap:hilecop-models}. The component interface exposes
places, transitions, and signals, which are references to the elements
of its internal behavior, to the outside so that multiple components
can be assembled. Each component has a clock and a reset input port
(\texttt{clk} and \texttt{rst}) in its interface. This is the mark
that the \hilecop{} has been built for the design of synchronous
digital systems. To a certain extent, \vhdl{} signals can be
integrated to the high-level components to represent a direct wiring
between components. A component behavior can also be defined through
the composition of other components. In that case, we talk about a
composite structure.

Next, in Figure~\ref{fig:hilecop-wf}, the transformation from Step~1
to Step~2 flattens the model. The internal behaviors are connected
according to the interface compositions, and embedding component
structures are removed. Figure~\ref{fig:impl-model} gives the result
of the flattening phase for the model of
Figure~\ref{fig:components-and-pn}.
%
\begin{figure}[H]
\centering
\includegraphics[keepaspectratio=true,width=\textwidth] {Figures/Hilecop/impl-model}
\caption[Global Petri net model.]{A global Petri net model obtained
  after the flattening of a \hilecop{} high level model.}
\label{fig:impl-model}
\end{figure}
%
The PN formalism is a formal model and therefore permits to apply
mathematical reasoning on its instances. Particularly, a PN model can
be analyzed, and a proof that a given model meet some properties can
be automatically produced through the direct analysis of the structure
or through the use of model checking techniques. This feature of PNs
has been one of the reason of the adoption of this formalism as the
basis of the design of critical digital systems. A whole thesis has
been dedicated to developed new methods to analyze the \hilecop{} PN
models \cite{Merzoug2018}. In fact, the transformation of the abstract
model is a bit different in preparation of the model analysis. The
transformation adds new information to the flattened model for the
purpose of the analysis. Figure~\ref{fig:impl-model} only gives the
flattened version of the model that is not produced for the analysis
but in preparation of the next transformation into VHDL design.  The
analysis phase is here to convince the engineers that they are
designing a safe system. The analysis process is a round trip between
Step~1 and Step~2.  It aims at producing a model that is conflict-free
(see Section~\ref{sec:sitpn-struct} for more details about the
definition of a conflict), bounded, and deadlock-free, using
model-checking techniques.  After several iterations, the model should
reach soundness and is then said to be implementation-ready.

From Step~2 to Step~3, \vhdl{} source code is then generated by means
of a model-to-text transformation. The generated code describes a
\vhdl{} design, i.e. a textual description of a hardware system, which
has an interface defining input and output ports and an internal
behavior called an architecture. Details about the syntax and the
semantics of the \vhdl{} language will be given in
Chapter~\ref{chap:hvhdl}. Figure succinctly illustrates the
transformation happening between Step~2 and Step~3.
%
\begin{figure}[H]
\centering
\includegraphics[keepaspectratio=true,width=.8\textwidth] {Figures/Hilecop/vhdl-generation}
\caption[Generation of a \vhdl{} design from a Petri net.]{Generation
  of a top-level \vhdl{} design from a Petri net.}
\label{fig:vhdl-gen}
\end{figure}
%
For the purpose of the \hilecop{} methodology, two VHDL designs have
been defined: the place design that is a hardware description of a PN
place (circle nodes in a PN) and the transition design that is a
hardware description of a PN transition (square nodes in a PN). Like
all \vhdl{} designs, the place and the transition design have an input
and output port interface, and their own internal behavior. A \vhdl{}
design is a mould to describe a hardware component. Thus, a design can
be instantiated in the behavior of other designs in order to obtain
more complex behaviors. As illustrated in Figure, the transformation
from Step~2 to Step~3 creates a place design (or component) instance
(PCI) and a transition design (or component) instance (TCI) for each
place and transition of the input Petri net. Then, the PCIs and TCIs
are connected together through their input and output port
interfaces. These connections mimic the arc connections between the
places and the transitions of the input PN. 



From Step~3 to Step~4, the VHDL compilation/synthesis and the FPGA
programming are finally performed using industrial tools.

In this work, we focus on the part of the workflow in Fig~\ref{fig:hilecop-wf}
that is framed with dotted lines, i.e. the model-to-text transformation between
Step~2 and Step~3. In particular, we aim to prove that through this
model-to-text transformation, the behavior described by the initial model is
preserved in the generated VHDL code. To do so, we have to implement the
structure of PNs used in the \hilecop{} models, together with the semantics
providing the evolution rules of these PNs.

\section{Verifying the \hilecop{} methodology}
\label{sec:verif-hilecop}

The development of these medical
devices is at the base of the creation of the Neurrinov company. The
Neurrinov company is now looking for the industrial development of
such neuroprotheses.


%%% Local Variables:
%%% mode: latex
%%% TeX-master: "../main"
%%% End:
