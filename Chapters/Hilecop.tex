\chapter{The \hilecop{} methodology}
\label{chap:hilecop}

%----------------------------------------------------------------------------------------
%	MODEL-BASED SYSTEM ENGINEERING AND THE HILECOP METHODOLOGY
%----------------------------------------------------------------------------------------

\section{Model-based System Engineering and the \hilecop{} methodology}
\label{sec:mse}

\begin{itemize}
\item present the principles of Model-based System Engineering
\item present the domain of formal methods
\item present the \hilecop{} methodology
\end{itemize}

\section{Verifying the \hilecop{} methodology}
\label{sec:hilecop-verif}

\begin{itemize}

\item \textsc{Research question (a.k.a the problem):}\\
  \fbox{\parbox{\linewidth}{\sc Can we prove that the \hilecop{}
      model-to-text transformation is semantic-preserving?}}
  
\item \textsc{Additional questions:}
  \begin{itemize}[label=+]
  \item How do we prove that?
  \item What are the similarities with compiler verification?
  \item What are the specificities?
  \end{itemize}
\item present the task motivation: the creation of safe digital
  circuits; first step, the behavior of the input model must be
  preserved in the output VHDL program
\item present the model-to-text transformation, i.e zoom in on the
  step in the \hilecop{} methodology figure (but not too much since a
  whole chapter will be dedicated to the transformation)
\end{itemize}

\pnote{talk about the existing \hilecop{} software, and give views
  showing what an input model looks like}

\pnote{transition idea: ``inspired by the work on compiler
  verification, we had an idea about how to prove that the
  transformation is semantic-preserving, and which steps to follow''}

%%% Local Variables:
%%% mode: latex
%%% TeX-master: "../main"
%%% End:
