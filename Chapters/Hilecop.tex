\chapter{The \hilecop{} methodology}
\label{chap:hilecop}

In this chapter, we present the context of our work, and more
specifically, the subject of our verification task, i.e. \hilecop{}, a
methodology for the design and production of critical digital systems.
In Section~\ref{sec:design-crit-digit-systms}, we motivate the use of
Model-Based Systems Engineering (MBSE) and formal methods in the
design and production of safety-critical digital systems; in
Section~\ref{sec:intro-hilecop}, we give an overall presentation of
the \hilecop{} methodology which applies both the principles of MBSE
and formal methods; in Section~\ref{sec:verif-hilecop}, we point out
the specific transformation phase, intervening in the \hilecop{}
methodology, that we propose to verify and discuss on how we propose
to verify it and which research questions does it give rise to.

\section{Designing critical digital systems}
\label{sec:design-crit-digit-systms}

According to Moore's law \cite{Moore2006}, the complexity of digital
integrated circuits is always increasing. To give an example, the
cut-of-the-edge \emph{AMD Epyc Rome} microprocessor (2019) is made out
of 50 billions of transistors. Composing billions of transistors on a
wired circuit is no more a task for humans but is very suited to
computers. However, engineers need to think about the design of
digital circuits in a way that is understandable for
humans. Therefore, they need high-level views of the circuits they are
designing in order to work together and to communicate about the
designs. The domain of Model-Based Systems Engineering (MBSE)
\cite{Long2011} proposes a framework to help engineers to design and
produce digital circuits, in a well-documented, safe and reliable
way. Comparable to what Model Driven Engineering (MDE) does in the
world of software engineering, models are first order concepts in
MBSE. A model represents a simplified view of real object. As
illustrated in Figure~\ref{fig:MBSE-ps}, a MBSE process describes a
way to design a digital circuit starting from a high-level view of the
system. This high-level view can follow a graphical formalism such as
SysML \cite{Friedenthal2014} or Petri nets \cite{Petri1962}, or a
textual one such as SystemC \cite{Black2009} or VHDL
\cite{Ashenden2010}. Then, the MBSE process describe many refinements
phases (the green arrows in Figure~\ref{fig:MBSE-ps}) during which the
input model will be transformed; at each refinement phase, the model
goes down in abstraction towards its final implementation as a
hardware circuit. A refinement phase, which is also a transformation
phase, can be performed automatically or manually.

\begin{figure}[H]
  \centering
  \includegraphics[keepaspectratio,width=.6\textwidth]{Figures/Hilecop/MBSE-ps}
  \caption[A Model-Based Systems Engineering process.]{A Model-Based
    Systems Engineering process; REQ stands for requirements, BEH for
    behavior, ARCH for architecture, Dgn V\&V for design verification
    and validation. This figure is an excerpt from \cite{Long2011}.}
  \label{fig:MBSE-ps}
\end{figure}

In the case where the digital circuit being designed is a
safety-critical system, i.e. on its behavior depends the life of
people, an MBSE process will often employ formal models, i.e. models
with a formal mathematical definition, as the design formalism.  Thus,
these models enable a certain extent of mathematical reasoning to
prove that safety properties are met during the design V\&V phase
(cf. Figure~\ref{fig:MBSE-ps}).

\section{Introducing the \hilecop{} methodology}
\label{sec:intro-hilecop}

The \hilecop{} methodology consists of a process for the design and
implementation of critical digital systems. This methodology relies on
several transformations going from abstract models to concrete FPGA
implementations through the production of VHDL
code. Fig.~\ref{fig:hilecop-wf} details the global workflow of
\hilecop{}. In this figure, Step~1 corresponds to the model of a
digital system. This model is built with component diagrams and the
behavior of each component is described by means of
PNs. Fig.~\ref{fig:components-and-pn} provides an example of such
model. As shown in this figure, an internal behavior and an interface
outline the structure of components. The component interface exposes
places, transitions, and signals, which are references to nodes of the
internal behavior, from which the components can be assembled to get
the global behavior of the digital system.
%
\begin{figure}[t]
\centering
\includegraphics[keepaspectratio=true,width=\textwidth]{Figures/Hilecop/hilecop-wf.eps}
\caption{Workflow of the \hilecop{} Methodology}
\label{fig:hilecop-wf}
\end{figure}
%
\begin{figure}[t]
\centering
\includegraphics[keepaspectratio=true,width=0.8\textwidth]
{Hilecop/components-and-pn.eps}
\caption{An Example of \hilecop{} Model}
\label{fig:components-and-pn}
\end{figure}

Next, in Fig.~\ref{fig:hilecop-wf}, the transformation from Step~1 to
Step~2 flattens the model. The internal behaviors are connected
according to the interface compositions, and embedding component
structures are removed. The analysis phase, going from Step~2 to
Step~1, aims to produce a model that is conflict-free (see
Sec.~\ref{sec:sitpn-struct} for more details about the definition of a
conflict), bounded, and deadlock-free, using model-checking
techniques.  After several iterations, the model should reach
soundness and is then said to be implementation-ready.

From Step~2 to Step~3, VHDL source code is then generated by means of
a model-to-text transformation. This generated code describes the
hardware system that will be implemented in a FPGA circuit. From
Step~3 to Step~4, the VHDL compilation/synthesis and the FPGA
programming are finally performed using industrial tools.

In this work, we focus on the part of the workflow in Fig~\ref{fig:hilecop-wf}
that is framed with dotted lines, i.e. the model-to-text transformation between
Step~2 and Step~3. In particular, we aim to prove that through this
model-to-text transformation, the behavior described by the initial model is
preserved in the generated VHDL code. To do so, we have to implement the
structure of PNs used in the \hilecop{} models, together with the semantics
providing the evolution rules of these PNs.

\section{Verifying the \hilecop{} methodology}
\label{sec:verif-hilecop}



%%% Local Variables:
%%% mode: latex
%%% TeX-master: "../main"
%%% End:
