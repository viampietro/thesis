\chapter{\hvhdl{}: a target hardware description language}
\label{chap:hvhdl}

The main purpose of this chapter is to present the target language of
the \hilecop{} transformation, i.e. the \vhdl{} language. The
formalization and the implementation of the \vhdl{} language syntax and
semantics is mandatory to reason about the programs generated by the
\hilecop{} model-to-text transformation. Thus, we want the reader to
clearly understand the structure and the semantics of the language to
be able to fully grab the proof of semantic preservation presented in
Chapter~\ref{chap:proof}.  Specifically, we present here the \hvhdl{}
language, our own synthesizable subset of the \vhdl{} language. This
subset permits to encode the programs generated by the \hilecop{}
transformation. We devise a formal semantics for \hvhdl{} which is a
simplification of the simulation semantics of the \vhdl{} language. The
formalization of the \hvhdl{} semantics and its implementation is one
contribution of this thesis to the many formalization of the \vhdl{}
language found in the literature. The chapter is structured as
follows.  In Section~\ref{sec:vhdl-lang-pres}, we give an informal
presentation of the \vhdl{} language syntax and semantics. In
Section~\ref{sec:choosing-vhdl}, we present the state of the art
pertaining to the formalization of the \vhdl{} language semantics. In
Section~\ref{sec:abstractSyntax}, \ref{sec:sem-rules},
\ref{sec:elab-rules} and \ref{sec:sim-rules}, we give the full
formalization of the \hvhdl{} language, a subset of the \vhdl{}
language. Section~\ref{sec:ex-full-sim} illustrates the formal
\hvhdl{} semantics with an example. Finally,
Section~\ref{sec:hvhdl-impl} outlines the implementation of the
\hvhdl{} syntax and semantics with the \coq{} proof assistant.

The \hilecop{} transformation generates a \vhdl{} design implementing
an input SITPN model. The transformation generates and connects the
component instances of two previously defined \vhdl{} designs: the
\texttt{place} design, i.e. a \vhdl{} implementation of a SITPN place,
and the \texttt{transition} design, i.e. a \vhdl{} implementation of a
SITPN transition. These designs were defined by the INRIA CAMIN team
at the creation of the \hilecop{} methodology. In the following
sections, we will be using fragments of the definition of the
\texttt{place} and \texttt{transition} designs to illustrate the
content of \vhdl{} programs and the rules of the \vhdl{} language
semantics. The reader will find the source code of the \texttt{place}
and \texttt{transition} designs in concrete and abstract syntax in
Appendices~\ref{app:place-design} and \ref{app:trans-design}.

\section{Presentation of the \vhdl{} language}
\label{sec:vhdl-lang-pres}
The intent here is to give an overview of the VHDL language, its
purpose, its main syntactal constructs, and an informal description of
its semantics as presented in the Language Reference Manual (LRM)
\cite{VHDL2000}. The VHDL language offers a lot of possibility in
terms of hardware (and even software) description. Here, we are not
trying to be exhaustive in our presentation of the language. We will
only maintain our description of the VHDL concepts in the scope that
is of interest to us. The readers that are interested in learning more
about the VHDL language can refer to \cite{VHDL2000},
\cite{Ashenden2010} and \cite{Pedroni2020}.

\subsection{Main concepts}
\label{sec:vhdl-main-concepts}

The VHDL acronym stands for Very high speed integrated circuit
Hardware Description Language. As its name indicates, the main purpose
of the VHDL language is to describe hardware circuits. There are two
approaches to the description of circuits. The first aims at the
simulation of the described circuits, and the second aims at the
synthesis of described circuits on physical supports. Thus, the
constructs of the VHDL language must be interpreted depending on the
purpose of the designer.  For instance, the language gives the
possibility to describe the connection of wires inside a circuit. A
wire is represented by the concept of \textit{signal}.  In the context
of circuit simulation, a \textit{signal} can be compared to a
variable; it has a given type and holds a value that fluctuates in the
course of the simulation. In the context of synthesis, a signal really
describes a physical wire and must be considered as so. From these two
approaches to circuit description arise two ways of considering the
semantics of the language (see Section~\ref{sec:choosing-vhdl}).

In VHDL, a top-level program is called a \emph{design}. A design
describes a hardware circuit. As explained in
Chapter~\ref{chap:hilecop}, the hilecop{} transformation generates a
VHDL design implementing the input SITPN model. To do so, the
transformation generates and connects the component instances of two
previously defined VHDL designs: the \emph{place} design that
implements the concept of a SITPN place, and the \emph{transition}
design that implements a SITPN transition. These designs were defined
by the INRIA CAMIN team at the creation of the \hilecop{}
methodology. In the following sections, we will be using excerpts of
the definition of the place and transition designs to illustrate the
content of VHDL programs and the rules of the VHDL language
semantics. The reader will find the source code of the place and
transition designs in concrete and abstract syntax in
Appendices~\ref{app:place-design} and \ref{app:trans-design}.

A VHDL design is composed of two descriptive parts. The first part is
called the entity and describes the interfaces of a circuit, namely:
the input and output ports, and the generic
constants. Listing~\ref{lst:trans-design-entity} is an excerpt of the
transition design's entity that defines the generic constants, the
input and output port interfaces of the
design. Figure~\ref{fig:trans-design-entity} is a visual
representation of the interfaces of the transition design.

\begin{lstlisting}[language=VHDL,label={lst:trans-design-entity},
caption={The entity part of the transition design in concrete VHDL
  syntax.},framexleftmargin=1.5em,xleftmargin=2em,numbers=left,
numberstyle=\tiny\ttfamily]
entity transition is
  generic(
    transition_type      : transition_t := NOT_TEMPORAL;
    input_arcs_number    : natural := 1; 
    conditions_number    : natural := 1;
    maximal_time_counter : natural := 1
    );        
  port(
    clock                   : in std_logic;
    reset_n                 : in std_logic;
    input_conditions        : in std_logic_vector(conditions_number-1 downto 0);
    time_A_value            : in natural range 0 to maximal_time_counter;
    time_B_value            : in natural range 0 to maximal_time_counter;
    input_arcs_valid        : in std_logic_vector(input_arcs_number-1 downto 0);
    reinit_time             : in std_logic_vector(input_arcs_number-1 downto 0);
    priority_authorizations : in std_logic_vector(input_arcs_number-1 downto 0);
    fired                   : out std_logic 
    );
end transition;
\end{lstlisting}

The generic clause of the entity holds the declaration of the generic
constants.  The purpose of generic constants is either to represent
some dimensions of the design (e.g. the size of ports, internal
signals\dots) or to represent constant values used throughout the
design. In Listing~\ref{lst:trans-design-entity}, one can see that the
\texttt{conditions_number} generic constant gives a dimension to the
type of the \texttt{input_conditions} input port, which is an array of
Boolean values with indexes ranging from 0 to
\texttt{conditions_number-1} (that is the meaning of
\texttt{std_logic_vector (conditions_number-1 downto 0)}.  The port
clause holds the declaration of input and output ports of the
design. The \texttt{in} keyword indicates the declaration of an input
port and the \texttt{out} indicates the declaration of an output port.

\begin{figure}[H]
  \centering
  \includegraphics[keepaspectratio=true, width=.6\textwidth]{Figures/H-VHDL/trans-design-entity}
  \caption[A representation of the transition design entity.]{A
    representation of the transition design entity. On the left side,
    the input port interface of the transition design; \textit{cn}
    stands for \textit{conditions\_number} and \textit{ian} stands for
    \textit{input\_arcs\_number}, i.e. two of the generic constants
    declared in the generic clause of the transition design entity;
    the numbers at the right of the input pins represent the pin
    indexes. On the right side, the output port interface of the
    transition design.}
  \label{fig:trans-design-entity}
\end{figure}

The second part of a VHDL design is called the architecture. The
architecture describes the internal behavior of the design. It
declares all the internal signals involved in the description of the
design behavior. Then, there are three ways to describe the behavior
itself: by using processes, by instantiating other designs (also
called, component instantiations), or by combining both technics (the
latter option is chosen in the VHDL designs generated by the
\hilecop{} transformation).

\paragraph{Behavior specification with processes}

The first way is to specify one or multiple processes. Processes are
concurrent statements that describes the wiring or the operations
performed on the signals of a given design.  A process declares a
sensitivy list that corresponds to the signals read in the process
statement body; also, it possibly declares internal variables. The
sensitivity list is only useful for the purpose of simulation. It
permits to resume the execution of a process when the value of one of
the signals of its sensitivity list
changes. Listing~\ref{lst:trans-design-arch} gives an excerpt of the
transition design architecture containing the declarative part of the
architecture (i.e. internal signals) and three of the processes
describing the transition design behavior, namely: the
\texttt{condition_evaluation} process, the \texttt{firable} process
and the \texttt{fired_evaluation} process.

\begin{lstlisting}[language=VHDL,label={lst:trans-design-arch},
caption={An excerpt of the architecture part of the transition design in concrete VHDL
  syntax.},framexleftmargin=1.5em,xleftmargin=2em,numbers=left,
numberstyle=\tiny\ttfamily]
architecture transition_architecture of transition is
  signal s_condition_combination : std_logic; #\label{line:scc}#
  signal s_enabled               : std_logic;
  signal s_firable               : std_logic;   
  signal s_firing_condition      : std_logic;
  signal s_priority_combination  : std_logic;
  signal s_reinit_time_counter   : std_logic; 
  signal s_time_counter          : natural range 0 to maximal_time_counter; #\label{line:stc}#
begin
  
  condition_evaluation : process(input_conditions) #\label{line:condev-ps}#
    variable v_internal_condition : std_logic;
  begin
    v_internal_condition := '1';
    
    for i in 0 to conditions_number - 1 loop #\label{line:for-begin}#
      v_internal_condition := v_internal_condition and input_conditions(i);
    end loop; #\label{line:for-end}#
    
    s_condition_combination <= v_internal_condition; #\label{line:assign-scc}#
  end process condition_evaluation;
    
  $\dots$

  firable : process(reset_n, clock)
  begin
    if (reset_n = '0') then
      s_firable <= '0';
    elsif falling_edge(clock) then
      s_firable <= s_firing_condition;
    end if;
  end process firable;

  fired_evaluation : process (s_firable, s_priority_combination)
  begin
    fired <= s_firable and s_priority_combination;
  end process fired_evaluation;      
  
end transition_architecture;
\end{lstlisting}

In Listing~\ref{lst:trans-design-arch}, from Line~\ref{line:scc} to
Line~\ref{line:stc}, the architecture declares the internal signals of
the transition design. Then, Line~\ref{line:condev-ps} begins the
declaration of the \texttt{condition_evaluation} process. The
sensitivity list of the \texttt{condition_evaluation} process holds
one signal, the \texttt{input_conditions} input port, and declares a
local variable \texttt{v_internal_condition}.

In the statement body of a process, the designer can use control flow
statements common to most of the generic programming languages (if
statement, for loops\dots), and also statements that are specific to
the VHDL language. The most representative statement, and the one of
interest to us, is the \emph{signal assignment} statement. The signal
assignment statement relate a given signal identifier to a source
expression. For instance, at Line~\ref{line:assign-scc} of
Listing~\ref{lst:trans-design-arch}, the signal assignment statement,
represented with the $\Leftarrow$ operator, assigns the value of the
internal variable \texttt{v_internal_condition} to signal
\texttt{s_condition_evaluation}; the \texttt{v_internal_variable} that
itself holds the Boolean product between the members of the
\texttt{input_conditions} input port performed in the \texttt{for}
loop of Lines~\ref{line:for-begin} to \ref{line:for-end}.

When considering a VHDL design in the point of view of hardware
synthesis, a signal assignment statement specifies a wiring between a
target signal identifier and other source
signals. Figure~\ref{fig:trans-design-arch-excerpt-1} gives a
synthesis-oriented view of the processes described in
Listing~\ref{lst:trans-design-arch}.

\begin{figure}[H]
  \centering
  \includegraphics[keepaspectratio=true, width=.8\textwidth]{Figures/H-VHDL/trans-design-arch-excerpt-1}
  \caption[Representation of a part of the transition design
  architecture.]{A representation of a part of the transition design
    architecture comprising three processes. On the left side, the
    \texttt{condition_evaluation} process connecting the
    \texttt{input_conditions} input port to the
    \texttt{s_condition_combination} internal signal; the
    \texttt{firable} process in the middle; on the right side, the
    \texttt{fired_evaluation} process connecting the
    \texttt{s_firable} and the \texttt{s_priority_combination} signals
    to the \texttt{fired} output port.}
  \label{fig:trans-design-arch-excerpt-1}
\end{figure}

In Figure~\ref{fig:trans-design-arch-excerpt-1}, the
\texttt{condition_evaluation} process is represented as an
\texttt{and} port performing the product over the elements of the
\texttt{input_conditions} input port. The \texttt{fired_evaluation}
process is a simple \texttt{and} gate connecting the \texttt{fired}
output port to the \texttt{s_firable} and
\texttt{s_priority_combination} internal signals. The \texttt{firable}
process is a \textit{synchronous} process. It responds to the event of
the \texttt{clock} signal. In its statement body, the \texttt{firable}
process assigns the value of the internal signal
\texttt{s_firing_condition} to the signal \texttt{s_firable} only at
the occurrence of the falling edge of the \texttt{clock} signal
(captured by the expression \texttt{falling_edge(clock)} where
\texttt{falling_edge} is a primitive function of the VHDL
language). In the point of view of simulation, there are no
distinction between synchronous processes and \emph{combinational}
processes, i.e. processes that are executed independently of the
occurrence of a clock signal. However, in the point of view of
synthesis, processes responding to a \texttt{clock} signal follow the
rules of the synchronous logic, whereas, combinational processes
follow the rules of combinational logic.

To complete the presentation of the statements to be found in the body
of processes, the VHDL language is also equipped with \texttt{timing}
constructs, i.e. statements that explicitly specify an amount of time
in a given time unit. The signal assignement statement possibly
specifies a time clause indicating when the assignment must be
performed. For instance, the signal assignment statement specifying
that the value of signal \texttt{b} must be assigned to signal
\texttt{a} in 3 milliseconds takes the form: \texttt{$a\Leftarrow{}b$
  in 3 ms}. When no time clause is specified for a signal assignment
statement, we talk about a $\delta$-delay signal assignment, i.e. the
application of the signal assignment is related to some $\delta$
interval corresponding the propagation time through a wire. When a
time clause is specified, we talk about an unit-delay signal
assignment. $\delta$-delay signal assignments are synthetizable,
meaning they have an equivalent implementation on a physical device,
whereas, unit-delay signal assignments can not be
synthetized. Unit-delay signal assignments do not appear neither in
the VHDL designs generated by \hilecop{} transformation nor in the
declaration of the place and transition designs. We are only
mentioning their existence here because they are the witnesses of the
two time paradigms that inhabit the simulation algorithm describe the
semantics of the VHDL language: $\delta$ time and real time.

\paragraph{Behavior specification with design instances}

The second way to specify the behavior of a design is to use other
designs, or rather instances of other designs, as components of the
circuits. In that case, the design is said to be composite as it
embeds instances of other designs in its own behavior. Also, a design
at the highest level of embedding, i.e. that is not instantiated as a
part of another design's behavior, is called a \emph{top-level}
design. The design instantiation, or component instantiation,
statement permits to instantiate a design in an embedding
architecture. When instantiating a design with a design instantiation
statement, the designer provides the component instance with an
identifier. Then, the design instance must be dimensioned; this is
performed through a generic map that associates the generic constants
of the design being instantiated to a static value. Finally, the
designer specifies how the component instance is connected to the
other elements of the architecture. A port map associates the input
ports and output ports of the component instance to expressions or to
the signals of the embedding
architecture. Listing~\ref{lst:trans-design-instantiation} shows an
example of instantiation of the \hilecop{}'s transition design. This
instance is involved in the definition of the behavior of an embedding
design called \texttt{toplevel}.

\begin{lstlisting}[language=VHDL,label={lst:trans-design-instantiation},
caption={An example of design instantiation statement in the architecture of the \texttt{toplevel} design. Here, the design being instantiated is the transition design.},framexleftmargin=1.5em,xleftmargin=2em,numbers=left,
numberstyle=\tiny\ttfamily]
architecture toplevel_architecture of toplevel is
begin
  #\dots#
  $id_t$ : entity transition 
  generic map (
    transition_type => NOT_TEMPORAL,
    input_arcs_number => 1,
    conditions_number => 1,
    maximal_time_counter => 1
  )
  port map (
    clock => clock,
    reset_n => reset_n,
    time_A_value => 0,
    time_B_value => 0,
    input_conditions(0) => $id_0$,
    input_arcs_valid(0) => $id_1$,
    priority_authorizations(0) => '1',
    reinit_time(0) => $id_2$,
    fired => $id_3$
  );
  #\dots#
end toplevel_architecture;
\end{lstlisting}

In Listing~\ref{lst:trans-design-instantiation}, the transition
component instance has the identifier $id_t$. Following the
\texttt{entity} keyword is the name of the design being instantiated;
here, the transition design. Then, the generic map associates the
generic constants of the transition design (i.e. the left side of the
arrow, also called the \emph{formal} part) to static values (i.e. the
right side of the arrow called the \emph{actual} part). This permits
the dimensiniong the component instance. For example, remember that
the \texttt{input_arcs_number} generic constant value determines the
number of elements in the composite input ports
\texttt{input_arcs_valid}, \texttt{priority_authorizations} and
\texttt{reinit_time}. The port map associates the input ports of the
transition design to expressions. For instance, the
\texttt{time_A_value} input port is connected to the constant value
$0$, and the \texttt{input_conditions} input port is connected to the
internal signal $id_0$ at index $0$. The port map also associates the
output ports with signal identifiers. Contrary to the association of
input ports, output ports can not be associated to expressions as
output port association describes a direct wiring. In the port map
described in Listing~\ref{lst:trans-design-instantiation}, the
association $\mathtt{fired}\Rightarrow{}id_3$ expresses that the
\texttt{fired} output port is connected to the signal $id_3$, where
signal $id_3$ is defined in the embedding
design. Figure~\ref{fig:trans-design-instance} illustrates the
transition design instance $id_t$ and the wiring of its input and
output port interfaces inside the toplevel design.

\begin{figure}[H]
  \centering
  \includegraphics[keepaspectratio=true, width=.8\textwidth]{Figures/H-VHDL/trans-design-instance}
  \caption[Visual representation of a design instantiation
  statement.]{Visual representation of a design instantiation
    statement. Here, the figure represents the transition design
    instance described in
    Listing~\ref{lst:trans-design-instantiation}.}
  \label{fig:trans-design-instance}
\end{figure}

% The wait statement stalls the execution of a process
% during a certain amount of time, or until some event arises on a given
% signal. An event arises on a signal when its value changes. Every
% process declared in an architecture possess a implicit wait statement
% at the end of its statement body. This statement waits on an event on
% one of the signals of the sensitivity list. This implicit wait
% statement is only useful for the purpose of simulation. Indeed, the
% wait statement is no synthetizable. 

\subsection{Informal semantics of the VHDL language}
\label{sec:vhdl-informal-sem}

Even though, in practice, there are two ways to consider a VHDL
design, i.e. a synthesis-oriented way and a simulation-oriented way,
the LRM does not define a synthesis-oriented semantics for the VHDL
language.  A synthesis-oriented semantics gives an interpretation to a
design by describing an equivalent in a lower level formalism, closer
to the physical circuit. For instance, the Verilog language gives a
synthesis-oriented semantics to its programs by defining an equivalent
RTL level description \cite{Verilog2005}.  The LRM gives an informal
semantics to VHDL designs through the definition of a simulation
algorithm. The purpose of simulation is to compute the evolution of
the values of signals during a certain time interval. Through the
simulation process, the designer is able to control the behavior of
the modeled circuits and to detect flaws in the evolution of the
signal values.

Former to the simulation, the LRM defines an elaboration phase that
permits to transform a design into a simulation-ready execution
model. The elaboration phase has several goals. First, it builds the
simulation environment and a starting design simulation state. The
simulation environment is built based on the declarative parts of the
top-level design; it maps the signals to their types. In the design
simulation state, each signal is associated with a current value and
with a driver. A driver maps time points to values and the association
between a given time point and a signal value is called a
transaction. The necessity of drivers is explained by the presence of
unit-delay signal assignments. A unit-delay signal assignment specify
a time clause indicating when a giving assignment must be performed,
e.g. \texttt{$a\Leftarrow{}b$ in 3ms} (signal \texttt{a} takes the
value of signal \texttt{b} in 3 milliseconds). Thus, when a unit-delay
signal assignment is executed in the course of a simulation, its
effect is to change the driver of the target signal by posting a new
transaction. For instance, let $T_c$ by the current simulation time,
the execution of statement \texttt{$a\Leftarrow$ true in 2ns} sets a
new transaction in the driver of signal $a$. The new transaction
associates the value \texttt{true} to the time point
$T_c+\mathtt{2ns}$. Note that without unit-delay signal assignments,
drivers are not needed as all assignments take their effects at the
current simulation time.
% Also, each signal is associated with a
% graph representing the relation to its source, i.e. the signals
% involved in the computation of the target signal value. For instance,
% the source signals of the \texttt{fired} output port represented in
% Figure~\ref{fig:trans-design-arch-excerpt-1} are the
% \texttt{s_firable} and the \texttt{s_priority_combination}
% signals. 
Second, the elaboration checks the well-formedness of the design by
performing static type-checking on the behavioral part of the
design. It also checks that the connection between signals respect
certain rules, for instance, that there are no multiply-driven
signals, i.e. signals that are written to by multiple
processes. Finally, the elaboration operates some transformations over
the VHDL code, and thus builds the \emph{execution} model. To
summarize, all concurrent statements of the behavioral part are
transformed until the top-level design behavior is only composed of
processes.

After the elaboration, the top-level design, or rather its
corresponding execution model, is ready to be simulated. Two entities
are involved in the simulation: the \emph{sea} of processes obtained
after the elaboration of the top-level design, and a \emph{kernel}
process. The kernel process orchestrates the simulation; it handles
the time of the simulation, i.e. it holds a variable describing the
current time of the simulation, and resumes the execution of
processes. Figure~\ref{fig:vhdl-sim-alg}, which is an excerpt from
\cite{Borger1995}, describes the structure of the VHDL simulation
algorithm.

\begin{figure}[H]
  \centering
  \includegraphics[keepaspectratio=true, width=.7\textwidth]{Figures/H-VHDL/vhdl-sim-alg}
  \caption[The VHDL simulation loop.]{The VHDL simulation loop. Excerpt from \cite{Borger1995}.}
  \label{fig:vhdl-sim-alg}
\end{figure}


The simulation starts with the initialization phase. During the
initialization phase all processes are run exactly once. Then, the
simulation cycles are structured as follows.  All processes execute
their statement body concurrently. New transactions are posted in the
drivers of signals for every interpreted signal assignment
statement. The execution goes on until all processes have executed
their statement body and then have reached a suspension state. When,
all processes are suspended, the kernel process takes
over. Figure~\ref{fig:kernel-ps} shows the activity diagram associated
with the kernel process.

\begin{figure}[H]
  \centering
  \includegraphics[keepaspectratio=true, width=.35\textwidth]{Figures/H-VHDL/kernel-ps}
  \caption[The activity diagram of the kernel process.]{The activity
    diagram of the kernel process. Square boxes represent activities,
    diamond nodes are decision nodes. The black circle at the top
    represents the starting point of the activities; the other black
    circle in the middle of the diagram represents the end of all
    activities.}
  \label{fig:kernel-ps}
\end{figure}

As shown in Figure~\ref{fig:kernel-ps}, the kernel process will then
determine the kind of simulation cycle that will be performed next.
There are two kinds of cycles: delta cycle or time cycle. If the value
of a signal changes at the current time point, i.e. its driver holds a
transaction at the current time point with a new value, then a delta
cycle must be performed.  Then, the simulation time does not change.
The kernel process updates the current value of signals and their
drivers, and wakes up the processes sensitive to the signals that
obtained new values. The repetition of multiple delta cycles
corresponds to the stabilization of signal values, i.e. the
propagation of values through the wires, that takes effect in a
negligeable $\delta$ time. If all signal values are stable at the
current time point, then a time cycle must be performed.  The kernel
process looks up the drivers for the next time point where the value
of a given signal will change. Then, the kernel process advances the
simulation time to this next time point before updating the signal
values and resuming the execution of processes.  The simulation goes
on like this, alternating between delta and time cycles, until the
current time value reaches the time specified for the end of the
simulation.

%%% Local Variables:
%%% mode: latex
%%% TeX-master: "../../main"
%%% End:


\section{Choosing a formal semantics for \vhdl{}}
\label{sec:choosing-vhdl}
In the previous section, we presented the main concepts underlying the
\vhdl{} language and its informal semantics. We want to prove that the
\hilecop{} transformation that generates \vhdl{} code from SITPNs
preserves the behavior of the initial model (i.e, the SITPN model)
into the generated \vhdl{} program.  A formal semantics for the \vhdl{}
language is therefore a necessary element to be able to reason about
the generated \vhdl{} programs, and moreover to be able to compare their
behaviors with the behaviors of the source SITPN models. Keeping that
in mind, which formal semantics should we consider for \vhdl{}?

The same holds for any task: there is a trade off between finding a
tool designed by others that will fit our needs, and creating our own
tool that will mitigate the gaps between our needs and what is
available in the literature. In the present case, the tool is a formal
semantics for \vhdl{}. Adopting a fully-set semantics found in the
literature as a base ground for the implementation of a formal
semantics for \vhdl{} has multiple perks. First, it reduces the
formalization effort, which is not a lesser point considering that the
proof ahead might be long and must still be completed within the time
span of the thesis. Still, the semantics would need to be implemented
in \coq{}, if no implementation exists (or not written in
\coq{}). Second, the formal semantics of programming languages found
in the literature are often general in their approach, this to provide
a generic framework to reason about programs. However, we must not
lose sight of our goal which is to prove behavior preservation; a
generic formal semantics could turn out to be too complex, or
necessitate too much tweaking and thus hinder the fulfillment of our
task. On the other side, creating our own formal semantics for
\vhdl{}, based on the work of others, is the best way to fit our needs
in compliance with our final aim. However, the pitfalls are that the
resulting semantics might prove to be very specific, therefore
preventing others from using it. Also, a work of formalization would
be necessary which, as we already stated, would be time-consuming. In
order to determine whether we ought to use an existing semantics or
design a new one, we must first clearly specify our needs regarding
the \vhdl{} language.

\subsection{Specifying our needs: \hilecop{} and \vhdl{}}
\label{subsec:specifying-needs}

Two elements are of major influence to the specification of our needs
for a formal semantics: first, the context of \hilecop{} and the
specificities of the \vhdl{} programs that are generated; second, the
context of theorem proving. These two aspects entail the following
considerations.

\paragraph{The need for coverage} The \hilecop{} methodology generates
particular \vhdl{} programs.  Even if some transformations can be
operated on the generated programs to simplify them, the looked-for
formal semantics must be able to deal with a certain subset of the
\vhdl{} language. Especially, this subset must include:

\begin{itemize}
\item 0-delay (or $\delta$-delay) signal assignments (equivalent to
  unit-delay signal assignment with a ``0 ns'' after clause)
\item component instantiation statements with generic constant and
  port mapping
\item entity's generic constant clauses (declaration of generic
  constants in a design entity)
\end{itemize}

\hilecop{}'s \vhdl{} programs only deal with 0-delay signal
assignments because they are the only kind of signal assignments that
can be synthesized. As a matter of fact, the industrial
compiler/synthesizer used in the \hilecop{} methodology only accepts
\vhdl{} programs with no timing constructs (i.e, no delayed signal
assignments) as inputs. % However, 0-delay signal assignments do not
% specify a delay after which the assignment of a value to a signal must
% be performed. The time taken by a signal assignment to be performed
% (the time of propagation) is only relevant in relation to the clock
% period that regulates the modeled circuit. It is sufficient if the
% semantics handles only this dimension of time, even if the \vhdl{}
% language gathers both the ``dimension of delta time and the dimension
% of real time'' \cite{Reetz95}.

Regarding component instantiation statements, the \vhdl{} LRM
describes a way to transform these statements into equivalent process
statements and block constructs \cite[p.~141]{VHDL2000} during the
elaboration of the design. However, we want to preserve the
hierarchical structure provided by the component instantiation
statements arguing that it will be easier to compare the state of a
given SITPN model with a \vhdl{} design state with an explicit
hierarchical structure. Indeed, there exists a mapping between places
and transitions of an SITPN and their mirror (generated by the
transformation) place and transition component instances (PCIs and
TCIs). This one-to-one correspondence might turn out to be handy to
perform the proof of behavior preservation. Obviously, the semantics
must cover the evaluation of process statements which are the core
concurrent statements of \vhdl{} programs.

The types of signals and variables used in \hilecop{} \vhdl{} designs
must have finite ranges of values. For instance, a \vhdl{} signal that
ranges over $\mathbb{N}$ cannot be synthesized on a physical circuit.
Indeed, $\mathbb{N}$ has an infinite number of values, and would
therefore require an infinite number of latches to be physically
implemented. Moreover, as the number of latches used to implement a
digital circuit greatly impacts the power consumption of the circuit,
the types of signals and variables must be as constrained as possible
to optimize the dimensioning of the circuit. The generic constants,
declared in the entity part of a design, are involved in the
dimensioning of the circuit. The generic constants define the bound of
the array and natural range types for the different signals and
variables declared in the \texttt{place} and \texttt{transition}
designs' architecture. When a place or a transition component is
instantiated, that is during the transformation of the SITPN model
into \vhdl{} code, its generic constants receive values via a generic
map; we call it the dimensioning of the component instance. Therefore,
generic constant clauses must belong to the subset of the \vhdl{}
language covered by the semantics.

\paragraph{The need for a synchronous execution} The second property
of \hilecop{}'s generated \vhdl{} programs is their synchronous
execution. The digital circuits designed with the \hilecop{}
methodology are all synchronously executed on physical target. The
generated \vhdl{} designs declare a clock signal as an input port of
their entity port interface. Thus, the behavioral part of the designs
contains two kinds of processes: \emph{synchronous} processes, i.e.
processes that are sensitive to the clock signal, and
\emph{combinational} processes, i.e. processes that are not sensitive
to the clock signal, and that are permanently running until the
stabilization of the signal values. Synchronous processes react to the
events of the clock signal, i.e. the rising and the falling edge, and
possess blocks of sequential statements that are only executed at the
precise moment of the clock event\footnote{These blocks are guarded by
  the expressions \texttt{rising\_edge(clk)} and
  \texttt{falling\_edge(clk)}.}. Therefore, we need a semantics that
is able to deal with synchronism, and that explicitly integrates the
synchronization with a clock signal into the expression of the
simulation
cycle. % Thus, the two dimensions of time, respectively delta
% time and ``real'' time \cite{Reetz1995}, that are part of the \vhdl{}
% language, must be handled by our semantics.
% Delta time pertains to
% delta-cycles, which are triggered by the execution of delta-delay
% signal assignments, that is, the only kind of signal assignments found
% in \hilecop{} \vhdl{} programs. Real time pertains to time-steps. When
% all signal values are stable (when there are no more delta-cycles),
% the simulation advances the current time value to the next time point
% where a relevant event happens.
% A relevant event could be the
% execution of the next pending signal assignment that was associated
% with a delay, or the end of a \texttt{wait for x} statement (where
% \texttt{x} specifies a time delay). In our case, neither delayed
% signal assignments nor wait statements are part the covered \vhdl{}
% language subset.
% In our case, the only two relevant events to which the simulation
% advances during a time-step are the rising edge and the falling edge
% of the clock signal.

% \paragraph{Other considerations} Considering the kind of proof that
% needs to be established, we would rather consider an operational
% semantics for \vhdl{}. The reason is that, in the \ccert{} project
% \cite{Leroy2009}, which is one of our major inpsiration source, the
% whole C compiler toolchain is verified by reasoning over the
% operational semantics of the source and target languages.
A last
consideration pertains to whether or not the \vhdl{} semantics must
explicitly handle errors. As the SITPN semantics does not include the
production of error values, the handling of errors by the \vhdl{}
semantics is not a mandatory aspect.

\paragraph{Qualifying criterions}
\label{sec:qualifying-criterions}

We here give the list of the qualifying criterions that will help to
analyze the different \vhdl{} semantics encountered in the literature
and that are presented in the next section. The three most relevant
criterions are:

\begin{itemize}
\item \emph{Synchronism}. Regarding this criterion, there are three
  possibilities:
  \begin{itemize}
  \item Synchronism is not expressible in the considered \vhdl{}
    semantics; this completely disqualifies the adoption of the
    semantics.
  \item Synchronism is expressible in the considered \vhdl{}
    semantics. Synchronism is expressible if time-steps are handle in
    the semantics, at least to be able to represent clock events.
  \item Synchronism is explicit, i.e. the simulation loop is built
    around the occurrences of clock events.
  \end{itemize}
  We will foster the semantics that explicitly formalize a
  synchronized execution of a \vhdl{} design.
\item \emph{Component instantiation}. Either the semantics handle the
  component instantiation statement in its simulation rules, or
  component instantiation statements must be transformed in order to
  be executed. We will foster the semantics that handle component
  instantiation statements without transformation.
\item \emph{Elaboration}.  This criterion pertains to the
  formalization of the elaboration phase as integrated to the \vhdl{}
  semantics. This criterion also expresses the ability of the
  semantics to handle constrained types, i.e. arrays and natural
  ranges, and generic constant clauses that are both dealt with during
  the \emph{elaboration} phase. Either the semantics handle these
  constructs or it does not. Of course, we will foster the first kind
  of semantics.
\end{itemize}

\subsection{Looking for an existing formal semantics}
\label{subsec:looking-for-sem}

Here, we give a summary of the work found in the literature pertaining
to the formalization of the \vhdl{} language semantics. Articles are
gathered and presented depending on the type of semantics used in the
formalization (operational, denotational, axiomatic\dots). Each
semantics is analyzed regarding the needs that were previously
expressed.

\paragraph{Denotational semantics}
Some authors have been interested in giving a formal denotational
semantics to \vhdl{}. In a general manner, these authors want to reason
about \vhdl{} programs: prove properties over a \vhdl{} program, prove that
two programs are equivalent\dots

In \cite{Fuchs1995}, the authors give a denotational semantics to the
\vhdl{} language within the \textsf{Focus} \cite{Dederichs1993}
framework, a method for the development of distributed systems. Signal
values and their evolution through time are represented as streams of
values. Statements are denoted as stream-processing
functions. Processes are stream-processing functions that takes input
signal streams (signals of the sensitivity list) and yields
transaction traces (i.e, waveforms) over output signals (i.e, signal
that are written by the process). Transaction traces are merged
together as the result of the concurrent execution of
processes. % Resolution functions are used in case of multiply-driven
% signals (i.e, signals that receive a value from multiple processes).
The authors only consider 0-delay signal assignments in their
semantics, stating that it is sufficient to ``consider time at a
logical level to model both synchronous and asynchronous designs''.
However, some transformations must be applied to a design that has a
synchronous execution to express its equivalent only with 0-delay
signal assignments. Therefore, this semantics does not express
synchronism of execution in an explicit manner. Moreover, the
component instantiation statements are not dealt with, and no mention
is made of the elaboration phase.

In \cite{Breuer1995a}, the authors give a denotational, yet
relational, semantics to the \vhdl{} language. A state of a \vhdl{}
design is represented by a function binding signals to values; a
worldline is a time-ordered list of states. Statements (including
processes) are denoted in the semantics by a relation that binds an
input couple, composed of a time point and a worldline, to an output
couple of the same type. Multiple input and output couples possibly
satisfy the relation denoting a particular statement; thus, the
semantics is nondeterministic.  The semantics tries to abstract from
the formalization of the simulation cycle as it is done in the
LRM. The authors want to establish a semantics that is abstract enough
to be able to compare all other works of formalization with the
authors' semantics. The authors also give an axiomatic semantics (i.e,
in the Hoare logic style) which is proved to be sound and complete
with the first denotational semantics. A \textsf{Prolog}
\cite{Colmerauer1990} implementation of the axiomatic semantics is
given. Regarding our needs, the semantics only deals with unit-delay
signal assignments. However, this semantics enables the representation
of a $\delta$-delay signal assignment with a unit-delay signal
assignment adorned with a ``\texttt{after 0 ns}'' time clause. The
hierarchical structure of designs is not preserved, and, although
expressible, the semantics does not explicitly express a synchronous
simulation cycle.

The denotational semantics expressed in \cite{Pandey1999} uses
interval temporal logic as an underlying model. Leveraging this
underlying model, the authors are interested in proving some
properties over \vhdl{} designs to help compilers to optimize the code,
for instance, by using rewrite rules proved to be valid against the
model. Some of the proofs laid out by the authors are embedded in PVS
\cite{Owre1994}. The expression of the dynamic model uses many
concepts described in the LRM, like drivers, port association, driving
and effective values for signals. The semantics deals with both
unit-delay and $\delta$-delay signal assignments. The semantics works
on fully-elaborated designs, therefore, it does not deal with
component instantiation statements. Moreover, interval temporal logic
is useful to reason on the \vhdl{} designs in the presence of delays,
however, it looses its interest for designs presenting only 0-delay
assignments.

In \cite{Borrione1995}, the author states that ``denotational
semantics is more adequate for mathematical reasoning''. The author
formalizes the \vhdl{} semantics to prove the equivalence between
\vhdl{} programs (for instance, a specification and an
implementation). What is of major interest regarding our needs is that
the author has expressed a simulation cycle for synchronous
designs. Therefore, a distinction is made between combinational and
synchronous processes in the abstract syntax. Moreover, this work
formalizes the elaboration part of a \vhdl{} design former to the
simulation; also, the elaboration keeps the hierarchical setting of
the \vhdl{} design, that is component instantiation statements are not
replaced by processes.  Due to the time abstraction, the semantics
only deals with 0-delay signal assignments. It is explained by the
fact that the reference time-unit is the clock period (i.e, the only
known time-step), and the advancing of time, happening during the
simulation cycle as described in the LRM, is captured within the
setting of the simulation cycle. % Also, the semantics takes primary
% inputs into account (i.e, input ports of the top-level design); to
% preserve a synchronous behavior for the simulated design, the
% hypothesis is made that the values of the primary inputs are stable
% between two clock events.
% The only critic that can be made to
% this semantics regarding our needs is that it is expressed in
% denotational style.

\paragraph{Operational semantics}

Multiple works formalize an operational semantics for \vhdl{}.  These
works are interested in the formal description of the \vhdl{} simulator.
The aim is to devise a formal semantics that acts as a formal
specification for a simulator.

In \cite{Breuer1995}, a formal description of a \emph{functional}
semantics for \vhdl{} is laid out based on stream-processing
functions. The semantics is expressed with the functional programming
language \textsf{Gofer} \cite{Jones1994}, thus enabling the computation
of execution traces, that is, the computation of the streams
representing the values taken by signals over time.  As in the former
work of the same author \cite{Breuer1995a}, only unit-delay
signal assignments are dealt with, however, this time the author
describes a deterministic operational semantics. Regarding our needs,
this work is neither interested in preserving the hierarchical
structure of \vhdl{} designs, and no mention is made regarding how a
design is elaborated, nor in expressing an explicit synchronous
simulation cycle.

In \cite{Borger1995}, the authors formalize the simulation loop of the
LRM using Evolving Algebra machines (EA-machines). All important
constructs of the \vhdl{} language are represented as records; processes
are represented as concurrent agents running pseudo-codes, and the
simulation control flow is passed to and fro between the kernel
process (i.e, the simulation orchestrator) and the rest of the
processes that execute the design behavior. This semantics implements
closely the simulation loop as described in the LRM. Therefore, it is
very rich and deals with most of the \vhdl{} constructs, including the
two time paradigms of the language (i.e. $\delta$ time and unit
time). Moreover, the semantics works on fully-elaborated designs,
therefore, component instantiation statements are omitted. However, a
synchronous execution is fully expressible even if not explicitly
embedded in the expression of the simulation loop.

In \cite{VanTassel1995}, the author presents a natural semantics for
\vhdl{}. The simulation loop is expressed by inference rules, and the
execution of processes is based on the events over signals of their
corresponding sensitivity lists. The execution of statements computes
transaction traces, that is, the drivers of the signals. The semantics
deals both with unit and delta delay signal assignments. Regarding our
needs, this semantics does not entirely cover the subset of \vhdl{} we
are interested in. Component instantiation statements are not dealt
with.  A synchronous execution is expressible within the semantics,
although it would be hidden in the inference rule formalizing the
generic simulation loop.  Also, the semantics does not provide its
simulation loop with a simulation horizon (a maximum number of
simulation cycles). The simulation ends when signal values evolve no
more.  % The question of the influence of the
% environment, measured through the values of the primary inputs of a
% design, is not discussed.

In \cite{Goossens1995}, the author presents an operational semantics
for \vhdl{} in the small-step style. The semantics follows closely the
simulation cycle described in the LRM; however it is very concise and
clear. The covered \vhdl{} subset comprises both unit and delta-delay
signal assignments. There is an interesting discussion about the
non-determinism of \vhdl{}, since it is a concurrent programming
language: it entails that non-determinism is only existent at the
processes level, that is, internal sequential statement of processes
can be executed in a nondeterministic manner (referred to as A
actions, that is, \emph{internal} actions).  But at every delta or
time step (referred to as $\delta$ and T actions) of the execution,
the design state can be computed in a deterministic manner, since all
processes have reached a suspension point at the end of their inner
body. The author is interested in comparing the behaviors of two
\vhdl{} designs by proving that some relation of equivalence holds
between the two. He describes two strategies to compare \vhdl{}
programs. The first one is bisimulation; it is based on the comparison
of the sequence of actions (either A, $\delta$ or T actions) performed
by the two programs. The second one is observational equivalence; it
is based on the observation of the value of the output signals of two
\vhdl{} programs (the observees), that receive values in their input
signals from another \vhdl{} program (the observer). The observer
stimulates the entries of the observees and reaches a success state
based on its observations of the value of the outputs. Regarding our
needs, this semantics permits the description of our synchronous
simulation cycle. However, like most of the semantics presented here,
the component instantiation statement is not supported as it stands,
but it is rather transformed into the equivalent processes
statements. Small-step semantics is not needed in our case because we
are only interested in the values of signals at the delta and time
steps (for us, time steps correspond to clock events). We are not
interested in capturing the design states in the middle of the
execution of a process body. We are more interested in "weak
bisimulation", therefore forsaking the internal actions % (i.e, A
% actions, execution of a process body that does not end in a wait
% statement)
performed by a \vhdl{} design.
% A natural operational semantics in the style of Van Tassel's
% \cite{VanTassel1995} is sufficient in our case.
In \cite{Thirunarayan2001}, the authors extend the work of
\cite{Goossens1995}, especially by handling shared variables, in the
presence of which a \vhdl{} program can have a concrete
nonderterministic behavior. The authors are also interested in the
equivalence between two \vhdl{} programs, and they are interested in
determining a unique meaning property for \vhdl{} programs. The unique
meaning property states that the execution of a \vhdl{} design in the
presence of shared variables is unique. This work is interesting as it
points out the fact that the \vhdl{} language is not only subject to
``\textit{benign} nondeterminism''. By benign nonderterminism, the
authors of \cite{Thirunarayan2001} mean that the only moment where the
state of a \vhdl{} design can not be decided in a deterministic way is
when the processes are in the middle of the execution of their
statement body. However, the state of a \vhdl{} design at that moment
is of no interest; it corresponds to nothing regarding the concrete
functioning of a hardware circuit. Also, two different processes can
never be writing to the same signal at the same time. If such a design
happens, this is a case of \textit{multiply-driven} signal, which is
utterly forbidden. So, there can be no nondeterminism, regarding the
value of a signal, coming from the concurrent execution of two
processes (at least when shared variables are not involved).

% However, we are not interested in
% dealing with constructs so advanced as shared variables, therefore,
% this work is not really relevant to us.

\paragraph{Translational semantics}
Another kind of semantics, called ``translational'', formalizes the
\vhdl{} language semantics by translating a \vhdl{} design into
another formal model. Thus, the semantics of \vhdl{} is modeled by the
translation and the formal semantics of the target model. The target
model has the ability to model concurrency, which is one of the
specificity of \vhdl{}. Moreover, target models are chosen regarding
the tools they provide for analysis, and thus, a translational
semantics for \vhdl{} is often related to model checking
considerations.

In \cite{Reetz1995}, the author expresses the formal semantics of
\vhdl{} by translating a \vhdl{} design into a corresponding
\emph{flowgraph}. All \vhdl{} constructs, ranging from sequential
statements to concurrent processes, are expressed with individual
flowgraphs that are then composed together through their
interfaces. The simulation cycle of \vhdl{} is also encoded by means
of connected flow graphs: one for the ``execution part'' of the
semantics, that is, all processes run until suspension, and one for
the update part (i.e, the kernel process). Flowgraphs come with a
large amount of tools for analysis, and this translational semantics
is involved in the setting of a framework to reason about \vhdl{}
programs using multiple technics (automatic theorem proving, model
checking\dots). All these technics rely on the flowgraph formalism.

In \cite{Dohmen1995}, the author introduces a translational semantics
for \vhdl{} based on deterministic finite-state automatons. Again, the
reason for using such automatons lies in the existence of many
analysis tools. Moreover, forcing the generation of deterministic
automatons improves the time execution of model-checking technics.
The translation is performed on an elaborated \vhdl{} design; a data
space stores the values of signals and variables, and automatons
represent the control-flow of \vhdl{} statements. Each \vhdl{} statement is
associated to a specific automaton; sequence of statements are
achieved by automaton composition. The simulation kernel is also
represented by a specific automaton. Processes are composed together
with respect to synchronization states, i.e. states that permit to
pass the control from one process to another, therefore achieving
determinism in the control flow of the overall automaton.

In \cite{Olcoz1995}, the author presents a translation from \vhdl{} to
Coloured Petri Nets (CPNs) thus giving a formal semantics to the
\vhdl{} constructs. The author approach to the \vhdl{} semantics is a
strict translation of the ``event-based'' \vhdl{} simulator by means
of Petri nets.  The author translates \vhdl{} execution models (sea of
processes) into CPNs, and also translates the kernel process into a
CPN. The kernel process has previously been expressed as a \vhdl{}
process so that the translation into CPN is similar to the translation
of other processes.  Signals are not represented in the subnets,
instead, three shared variables depict the signal states: one variable
for the driving, one for the effective and one for the current value
of a given signal (see \cite[p.167]{VHDL2000} for the details on the
values associated with signals during the simulation).  Color domains
of places in the subnets represent the different types of \vhdl{}
domains.  Variables are represented by tokens.  Values in drivers are
represented by sequences of transactions (equivalent to waveforms);
the author defines a set of functions that are convenient to handle
sequences of transactions.  Sequential statements are partitioned into
two kinds: control flow (if, loop, case\dots) and notation (operations
on signals and variables) nets.  Processes subnets are made by the
fusing of each sequential statements in the process body. There is a
special \emph{Resume} place that can be set by the kernel process to
resume the activity of a process.  Concurrency is not discussed here,
as the Petri net models are inherently concurrent models.  The kernel
process is a broad CPN having some of its places interfaced with the
process subnets.  The decoloration of the Petri net enables the
analysis of the model and the detection of dead-locks.

In \cite{Deharbe1995}, the author gives a formal semantics to \vhdl{}
by transforming a \vhdl{} design into an abstract machine,
i.e. defined by a set of inputs, outputs, states and transition
function over states and outputs. The author is interested in the
verification of properties over \vhdl{} designs (temporal properties)
or to prove equivalence between designs (bisimulation).  To operate
this transformation, only a subset of \vhdl{} is considered, otherwise
a finite-state representation is not reachable.  The covered \vhdl{}
subset consists of objects with finite types, and no quantitative
timing constructs (no after clause in signal
assignments).  % The author claims that a \vhdl{}
% design is implemented by an abstract machine if they have the same
% observational behavior, i.e. given the same inputs they yield the
% same outputs.
The transformation generates a decision diagram (i.e. a control flow
graph) and a state space for each process defined in the design's
behavior. The decision diagram encodes the transition function over
states and outputs.  Process statements are composed with a special
composition operator to obtain a global abstract machine. Moreover,
the article lays out a method to transform a block statement into an
abstract machine. The initiative is to be noticed as there are only a
few papers, dealing with the formalization of the \vhdl{} semantics, that
are interested in such hierarchical constructs as block or component
instantiation statements. The article concludes with an expression of
the space of complexity entailed by the transformation of a \vhdl{}
design into an abstract machine.

Although the translational semantics described above meet most of the
qualifying criterions in relation to our needs, we are not especially
interested in implementing one of these. The main reason being that it
would require to implement the transformation from the abstract
\vhdl{} syntax to the target model, in addition to the implementation
of the semantics of the target model.

Table~\ref{tab:sum-vhdl-sem} summarizes the analysis of the \vhdl{}
semantics encountered during our literature
review. Table~\ref{tab:sum-vhdl-sem} compares the different \vhdl{}
semantics in relation to our qualifying criterions (see
Section~\ref{sec:qualifying-criterions}).

\newpage

\newcommand{\YO}{\cellcolor{YellowOrange}}
\newcommand{\Gre}{\cellcolor{Green}}
\newcommand{\LGr}{\cellcolor{lightgray}}
\newcommand{\R}{\cellcolor{Red}}

\thispagestyle{empty}

\begin{landscape}

\begin{table}[H]
  \resizebox{1.4\textwidth}{!}{%
    \renewcommand{\arraystretch}{1.5}
    \begin{tabular}{ll|c|c|c|c|c|c|c|c|c|c|c|c|c|}

      % TABLE HEADER
      \cline{3-14}
      & &  &  &  &  &  &  &  &  &  &  &  &  \\
      &
      & \begin{tabular}[t]{@{}c@{}}\rotatebox[origin=r]{270}{Fuchs and Mendler} \\ \cite{Fuchs1995} \end{tabular}  
      & \begin{tabular}[t]{@{}c@{}}\rotatebox[origin=r]{270}{Breuer et al.} \\ \cite{Breuer1995a} \end{tabular}
      & \begin{tabular}[t]{@{}c@{}}\rotatebox[origin=r]{270}{Pandey et al.} \\ \cite{Pandey1999} \\ \end{tabular}
      & \begin{tabular}[t]{@{}c@{}}\rotatebox[origin=r]{270}{Borrione and Salem } \\ \cite{Borrione1995} \\\end{tabular}
      & \begin{tabular}[t]{@{}c@{}}\rotatebox[origin=r]{270}{Breuer et al. } \\ \cite{Breuer1995} \\ \end{tabular}
      & \begin{tabular}[t]{@{}c@{}}\rotatebox[origin=r]{270}{Börger et al. } \\ \cite{Borger1995} \\ \end{tabular}
      & \begin{tabular}[t]{@{}c@{}}\rotatebox[origin=r]{270}{Van Tassel} \\ \cite{VanTassel1995} \\ \end{tabular}
      & \begin{tabular}[t]{@{}c@{}}\rotatebox[origin=r]{270}{Goossens} \\ \cite{Goossens1995} \\ \end{tabular}
      & \begin{tabular}[t]{@{}c@{}}\rotatebox[origin=r]{270}{Reetz and Kropf} \\ \cite{Reetz1995} \\ \end{tabular}
      & \begin{tabular}[t]{@{}c@{}}\rotatebox[origin=r]{270}{Döhmen and Herrmann} \\ \cite{Dohmen1995} \\ \end{tabular}
      & \begin{tabular}[t]{@{}c@{}}\rotatebox[origin=r]{270}{Olcoz} \\ \cite{Olcoz1995} \\ \end{tabular}
      & \begin{tabular}[t]{@{}c@{}}\rotatebox[origin=r]{270}{Déharbe and Borrione} \\ \cite{Deharbe1995} \\ \end{tabular}
      \\

      % SEMANTICS DESCRIPTION

      
      % kind of semantics (D, O, T).
      \hline
      \multicolumn{1}{|c|}{}
      & Kind & \YO D & \YO D, A & \YO D & \YO D & \Gre O & \Gre O & \Gre O & \Gre O & \YO T & \YO T & \YO T & \YO T \\
      
      % purpose of the formalization.
      \cline{2-14}
      \multicolumn{1}{|c|}{\multirow{-2}{*}{\begin{tabular}[c]{@{}c@{}}Semantics\\ Description\end{tabular}}}
      & Purpose & \LGr AR, ATP & \LGr AR & \LGr AR & \LGr AR & \Gre SS & \Gre SS & \Gre SS, ITP & \Gre SS, MC & \Gre ATP, MC, ITP
                                   & \Gre MC, ITP & \LGr MC & \LGr MC \\

      % QUALIFYING CRITERIONS
      
      \hline
      
      % component instantiation covered?
      \multicolumn{1}{|l|}{\multirow{3}{*}{\begin{tabular}[c]{@{}l@{}}Qualifying\\ Criterions\end{tabular}}}
      & \begin{tabular}[c]{@{}l@{}}Component\\ Instantiation\end{tabular}
      & \YO T & \YO T & \YO T & \Gre N & \YO T & \YO T & \YO T & \YO T & \YO T & \YO T & \YO T & \Gre N \\

      % Synchronism expressibility status
      \cline{2-14} 
      \multicolumn{1}{|l|}{} & Synchronism & \R NE & \R NE & \R NE & \Gre Ex & \YO E & \YO E & \YO E & \YO E & \YO E & \YO E & \YO E & \R NE \\

      % Elaboration is formalized?
      \cline{2-14} 
      \multicolumn{1}{|l|}{}
      & Elaboration &\YO $\times$ &\YO $\times$ &\YO $\times$ & \Gre \checkmark &\YO $\times$ &\YO $\times$ & \Gre \checkmark &\YO $\times$ &\YO $\times$ &\YO $\times$ &\YO $\times$ & \Gre \checkmark \\ 
      
      % EXTRA. INFORMATION.

      \hline

      \multicolumn{1}{|c|}{\multirow{2}{*}{\begin{tabular}{@{}l@{}}Extra.\\ Informations.\end{tabular}}}
      
      % implementation technology
      & \begin{tabular}[c]{@{}l@{}} Impl.\\ Technology\end{tabular}
      & \LGr Focus \cite{Dederichs1993} & \LGr Prolog \cite{Colmerauer1990} & \LGr PVS \cite{Owre1994} & \LGr ? & \LGr Gofer \cite{Jones1994}
                    & \LGr ? & \LGr HOL \cite{Hutton1994} & \LGr ? & \LGr HOL \cite{Hutton1994} & \LGr ? & \LGr ? & \LGr ? \\
      
      % particular models or data types 
      \cline{2-14} 
      \multicolumn{1}{|c|}{}
      & \begin{tabular}[!htbp]{@{}l@{}}Particular\\ Model or\\ Data Types\end{tabular}
      & \LGr \begin{tabular}[c]{@{}c@{}}Stream \\ Processing \end{tabular}
      & \LGr No & \LGr \begin{tabular}[c]{@{}c@{}}Interval\\ Temporal \\ Logic \end{tabular}
      & \LGr No & \LGr \begin{tabular}[c]{@{}c@{}}Stream \\ Processing \end{tabular}
      & \LGr \begin{tabular}[c]{@{}c@{}}Evolving \\ Algebra \\ Machines \end{tabular}
      & \LGr \begin{tabular}[c]{@{}c@{}}Natural \\ Semantics \\ (big-step) \end{tabular}
      & \LGr \begin{tabular}[c]{@{}c@{}}Structural \\ Semantics \\ (small-step) \end{tabular}
      & \LGr \begin{tabular}[c]{@{}l@{}}Flow \\ Graphs \end{tabular}
      & \LGr \begin{tabular}[c]{@{}c@{}}Finite-State \\ Automatons \end{tabular}
      & \LGr \begin{tabular}[c]{@{}c@{}}Colored \\ Petri \\ Nets \end{tabular}
      & \LGr \begin{tabular}[c]{@{}c@{}}Abstract \\ Machines \\ and \\ Decision \\ Diagrams \end{tabular} \\
      \hline
    \end{tabular}%
  }
  \caption{A comparative summary on \vhdl{} formal semantics.}
  \label{tab:sum-vhdl-sem}

  \begin{itemize}[label=-]
    \fontsize{10}{12}\selectfont
  \item Kind : D (Denotational) - A (Axiomatic) - O (Operational) - T (Translational).
  \item Purpose : AR (Abstract Reasoning) - ATP (Automatic Theorem
    Proving) - SS (Simulator Specification) - ITP (Interactive Theorem
    Proving) - MC (Model Checking).
  \item Component Instantiation : T (statement is \emph{Transformed}
    into equivalent processes) - N (statement is \emph{Natively} taken into account in the semantics).
  \item Synchronism : E (Expressible within the semantics) - NE (Not
    Expressible within the semantics) - Ex (Explicitly built in the
    semantics).
  \end{itemize}
\end{table}

\end{landscape}

\newpage

To summarize, we are interested in a semantics built for the purpose
of interactive theorem proving (ideally, with an existing
implementation in the \coq{} proof assistant). Most important, the
formal semantics must be able to deal with the expression of
synchronous designs, that is, designs synchronized with a clock
signal. Therefore, a synchronous simulation cycle must be at least
expressible within the semantics. Moreover, the semantics must handle
component instantiation statements as they are, that is, without
transforming them into equivalent processes. As a bonus, the semantics
should formalize the elaboration part of \vhdl{} semantics.

In Table~\ref{tab:sum-vhdl-sem}, cells are colored in green when the
cell's content foster the adoption of the semantics, in yellow when
the content does not go towards the adoption of the semantics but is
not disqualifying, and red when the content is a disqualifying
criterion. Regarding the semantics adoption, cells are labelled in
light grey when their content is neutral. Now comparing the entries of
Table~\ref{tab:sum-vhdl-sem} with the expression of our needs, we can
discard the semantics with a cell labelled in red, that is, most of
the denotational semantics; moreover, all translational semantics are
disqualified for the previously mentioned reasons. The candidate
semantics are the operational semantics, plus the denotational
semantics by Borrione and Salem \cite{Borrione1995}, the only
semantics that formalizes an explicitly synchronous simulation
cycle. The semantics that is the most likely to be adopted is the
Borrione and Salem's semantics. However, we prefer an operational
setting for our semantics. To lower down the complexity of proofs, we
really need a semantics that builds the synchronism into its
simulation cycle, therefore putting aside all the intricacies of the
full-blown \vhdl{} simulation cycle. Moreover, the big-step style for
an operational semantics is more relevant to us; as stated before, we
are not interested in the intermediary states of computation that a
small-step style semantics considers.  Based on these observations, we
have decided to formalize our own \vhdl{} semantics inspired from the
semantics of Borrione and Salem's \cite{Borrione1995} and Van Tassel's
\cite{VanTassel1995}. The following sections are dedicated to the
presentation of the syntax and semantics of a subset of \vhdl{} that
we baptize \hvhdl{}. \hvhdl{} embeds the subset of \vhdl{} that we are
interested in when considering the \vhdl{} designs generated by the
\hilecop{} transformation.

%%% Local Variables:
%%% mode: latex
%%% TeX-master: "../../main"
%%% End:


\section{Abstract syntax of \hvhdl{}}
\label{sec:abstractSyntax}
In this section, we describe the abstract syntax of \hvhdl{}, a subset
of VHDL covering all the constructs present in the programs generated
by the \hilecop{} transformation. Terminals of the language are
written in \texttt{typewriter} font, or are enclosed in simple quotes
for symbols with no typewriter representation. The $a^{*}$ denotes a
possibly empty repetition of the element $a$; the $a^{+}$ denotes a
non-empty repetition of $a$.

\subsection{Design declaration}
\label{subsec:design-decl}

Similarly to \cite{VanTassel1995}, we define the \emph{design}
construct in the $\mathcal{H}$-VHDL's abstract syntax which has no
equivalent in the concrete syntax of VHDL.

\begin{table*}[!h]
  \begin{tabular}{lcl}
    design & ::= & \texttt{design} id$_e$ id$_a$ gens ports sigs cs \\
    gens & ::= & gdecl* \\
    ports & ::= & pdecl* \\
    sigs & ::= & sdecl* \\
  \end{tabular}
\end{table*}

% \subsection{Generic constant, port and internal signal declaration.}
% \label{subsec:ent-decl}
\begin{table*}[!h]
\begin{tabular}{lcl}
  gdecl & ::= & \texttt{(}id\texttt{,} $\tau$\texttt{,} e\texttt{)} \\
  pdecl & ::= & \texttt{(}(\vhdle|in||\vhdle|out|)\texttt{,} id\texttt{,} $\tau$\texttt{)}\\
  sdecl & ::= & \texttt{(}id, $\tau$\texttt{)} \\
\end{tabular}
\end{table*}

In the above entry, $\mathrm{id}_e$ indicates the entity identifier
and $\mathrm{id}_a$ the architecture identifier of the declared
design. The $\mathrm{gens}$ entry corresponds to the generic clause,
i.e. the declaration list for the generic constants of the design. A
generic constant is declared via the \textrm{gdecl} entry; a generic
constant declaration is a triplet composed of an identifier, a type
indication and an expression denoting the generic constant's default
value. The ports entry holds the declaration of the input and output
ports of the design. A port declaration (i.e. the pdecl entry) is a
triplet composed of a port type, i.e. \texttt{in} or \texttt{out}, an
identifier, and a type indication. The sigs entry is the list
declaring the internal signals of the design. An internal signal
declaration entry (i.e. sdecl) is a couple composed of an identifier
and a type indication. The cs entry represents the concurrent
statements composing the behavior of the design.

\subsection{Concurrent statements}
\label{subsec:conc-stmt}

\begin{table*}[!h]
\begin{tabular}{lcl}
  cs & ::= & psstmt | cistmt | cs \texttt{||} cs | \texttt{null} \\
\end{tabular}
\end{table*}
In \hvhdl{}, two kinds of concurrent statements are available to
describe the behavior of a design: process statements, represented by
the psstmt entry, and component instantiation statements, represented
by the cistmt entry. Concurrent statements are composable through the
\texttt{||} operator. We add the \texttt{null} statement to the
\hvhdl{} abstract syntax to help represent empty behaviors.

\subsubsection{Process statement}

\begin{table*}[!h]
\begin{tabular}{lcl}
  psstmt & ::= & \vhdle|process| \texttt{(}id$_p$\texttt{,} sl\texttt{,} vars\texttt{,}  ss\texttt{)} \\
  sl & ::= & id$^{*}$ \\
  vars & ::= & vdecl* \\
  vdecl & ::= & \texttt{(}id\texttt{,} $\tau$\texttt{)} \\
\end{tabular}
\end{table*}

A process statement declares a sensitivity list, i.e. the sl entry,
which is a possibly empty set of signal identifiers. In order to be
well-formed, the signals of a sensitivity list must be either internal
signals or input ports of the design, i.e. . The process possibly declares a
set internal variables, i.e. the vars entry. A variable declaration
entry is a couple composed of a variable identifier and a type
indication. The ss entry represents the sequence of statements
composing the body of the process, i.e. the part that will be executed
during the simulation.

\subsubsection{Component instantiation statement}

The VHDL LRM defines two kinds of component instantiation statement:
the instantiation of a component instance \cite[p.139]{VHDL2000} and
the instantiation of a design entity \cite[p.141]{VHDL2000}. The
component instantiation statement used in the \hvhdl{} abstract syntax
corresponds to the instantiation of a design entity.

\begin{table*}[!h]
\begin{tabular}{lcl}
  cistmt & ::= & \vhdle|comp| \texttt{(}id$_c$\texttt{,} id$_e$\texttt{,} gmap\texttt{,} ipmap\texttt{,} opmap\texttt{)} \\
  gmap & ::= & assoc$_g^{*}$ \\
  ipmap & ::= & assoc$_{ip}^{*}$ \\
  opmap & ::= & assoc$_{op}^{*}$\\
  assoc$_g$ & ::= & \texttt{(}id\texttt{,}e\texttt{)} \\
  assoc$_{ip}$ & ::= & \texttt{(}name\texttt{,}e\texttt{)} \\
  assoc$_{op}$ & ::= & \texttt{(}id\texttt{,}(name|\vhdle|open|)\texttt{)}|\texttt{(}id\texttt{(}e\texttt{)}\texttt{,}name\texttt{)}\\
\end{tabular}
\end{table*}

In the cistmt entry, the identifier $\mathrm{id}_c$ represents the
name of component instance. Identifier $\mathrm{id}_e$ points out the
name of the design, i.e. the entity identifier, being instantiated
here. The gmap entry describes the list of associations between
generic constant identifiers and expressions. The ipmap entry is the
list of associations between input port identifiers (or indexed
identifiers) and expressions. The opmap entry is the list of
associations between output port identifiers (or indexed identifiers)
and signal names, or the \texttt{open} keyword. Associating the
\texttt{open} keyword with an output port identifier indicates that
the port is not connected. The left element of an association is
called the \emph{formal} part, and the right element of an association
is called the \emph{actual} part.

\subsection{Sequential statements}

\begin{table*}[!h]
\begin{tabular}{lcl}
  ss & ::= & name $\mathtt{\Leftarrow}$ e | name \texttt{:=} e | \texttt{if} \texttt{(}e\texttt{)} ss [ss] | \texttt{for (}id\texttt{,}e\texttt{,}e\texttt{)} ss \\
     & &  | \texttt{falling} ss | \texttt{rising} ss | \texttt{rst} ss ss' | ss\texttt{;} ss | \texttt{null} \\
\end{tabular}
\end{table*}

The ss entry defines the sequential statements that compose the body
of processes. The signal assignment statement is represented with the
$\Leftarrow$ operator; the variable assignment statement with the
$\mathtt{:=}$ operator. Also, we devise three control flow statements
that have no equivalent in the VHDL syntax: the \texttt{falling} block
statement, the \texttt{rising} block statement and the \texttt{rst}
block (or reset) block statement.  The \texttt{falling} statement
(resp. \texttt{rising} ss) declares a block of sequential statements
to be executed only at the falling edge (resp. rising edge) of the
clock signal (see Section~\ref{sec:clk-phases-rules}). Also, the
\texttt{rst} statement declares two blocks, the first one must be
executed during the initialization phase of the simulation; otherwise,
the second one is executed (see Section~\ref{sec:init-rules}).  These
invented constructs are equivalent to specific if-else statements that
are commonly used in the body of a synchronous process (see
Section~\ref{sec:abs-syntax-examples} for an example of transcription
of a specific if-else statement into one of these constructs).

\subsection{Expressions, names and types}
\label{sec:expr-names}

\begin{table}[H]
  \begin{tabular}{lcl}

    e & ::= & e \texttt{and} e | e \texttt{or} e | \texttt{not} e | e \texttt{=} e | e $\neq$ e \\
      & & | e \texttt{<} e | e \texttt{<=} e | e \texttt{>} e | e \texttt{>=} e | e \texttt{+} e | e \texttt{-} e \\
      & & | name | natural | boolean | \texttt{(}e$^{+}$\texttt{)} \\
      & & \\
    name & ::= & id | id\texttt{(} e \texttt{)} \\
    boolean & ::= & \texttt{true} | \texttt{false} \\
    $\tau$ & ::= & \texttt{boolean} | \texttt{natural} \texttt{(}e\texttt{,} e\texttt{)} |
                   \texttt{array} \texttt{(}$\tau$\texttt{,} e\texttt{,} e\texttt{)} \\
  \end{tabular}
\end{table}

The expression entry, i.e. e, declares a set of operators over Boolean
expressions, and natural numbers expressions.  The natural
non-terminal represents the set of natural numbers ($\mathbb{N}$). The
id non-terminal represents the set of identifiers, comparable to the
set of strings, or any infinitly enumerable set. In the following
sections, concrete identifiers will be written in \texttt{typewriter}
font, e.g. the \texttt{place} and \texttt{transition} design
identifiers.

The $\tau$ entry corresponds to the type indication associated with
the declaration of a generic constant, a port or an internal
signal. The considered types are the \emph{Boolean} type, the
constrained natural type, and the array type. The constrained natural
type, i.e. \texttt{natural}(e,e), defines a finite interval of natural
numbers; the left-most expression of the range constraint denotes the
lower bound of the interval, and the second one denotes the upper
bound of the interval. The array type indication,
i.e. \texttt{array}($\tau$, e, e), denotes a non-empty set of elements
of type $\tau$. The elements are indexed with respect to the specified
\emph{index} constraint. The left-most expression of the index
constraint denotes the starting index (possibly different from 0) and
the right-most expression denotes the final index.

% \subsection{Examples of \hvhdl{} abstract syntax}
% \label{sec:abs-syntax-examples}

% Listing~\ref{lst:t-design-decl-part-abss} gives the translation in
% abstract \hvhdl{} syntax of the declarative part of the transition
% design presented in Listings~\ref{lst:trans-design-entity} and
% \ref{lst:trans-design-arch} in concrete VHDL syntax.

% \begin{lstlisting}[language=VHDL,label={lst:t-design-decl-part-abss},
% caption={[The transition design's declarative parts in abstract
%   \hvhdl{} syntax.]The transition design's declarative parts in
%   abstract \hvhdl{}
%   syntax.},framexleftmargin=1.5em,xleftmargin=2em,numbers=left,
% numberstyle=\tiny\ttfamily]
% design "transition" "transition_architecture"
%   -- Generic clause
%   (("transition_type", natural(0, 2), 0), 
%    ("input_arcs_number", natural(0, NATMAX), 1), 
%    ("conditions_number", natural(0, NATMAX), 1),
%    ("maximal_time_counter", natural(0, NATMAX), 1))

%   -- Port clause
%   ((in, "input_conditions", array(boolean, 0, conditions_number$-$1)),
%    (in, "time_A_value", natural(0, maximal_time_counter)),
%    (in, "time_B_value", natural(0, maximal_time_counter)),
%    (in, "input_arcs_valid", array(boolean, 0, input_arcs_number$-$1)),
%    (in, "reinit_time", array(boolean, 0, input_arcs_number$-$1)),
%    (in, "priority_authorizations", array(boolean, 0, input_arcs_number$-$1)),
%    (out, "fired", boolean))

%   -- Internal signals
%   (("s_condition_combination", boolean),
%    ("s_enabled", boolean),
%    #\dots#
%    ("s_time_counter", natural(0, maximal_time_counter)))

%   -- Concurrent statements
%   cs
% \end{lstlisting}

% In Listing~\ref{lst:t-design-decl-part-abss}, the type indication of
% the ``\texttt{transition\_type}'' generic constant has been
% transformed from the \texttt{temporal_t} enumeration type to the
% \texttt{natural(0,2)} type. The \texttt{temporal_t} enumeration type,
% defined through the three values \texttt{\{NOT\_TEMPORAL,
%   TEMPORAL\_A\_A, TEMPORAL\_A\_B\}}, is naturally transformed into a
% natural range. This transformation is only valid because the
% \texttt{temporal_t} type is only defined and used in the transition
% design which has a static behavior. Also, the \texttt{std_logic} type
% defined in the VHDL Std Logic 1164 library \cite{STDLOGIC} is
% transformed into the Boolean type, and the \texttt{std_logic_vector}
% is transformed into the array of Booleans type. This transformation is
% valid because, in the transition and place designs and also in the
% generated designs, we are only referring to the values '0' and '1'
% among all the values enumerated by the \texttt{std_logic} type (see
% \cite[p. 2]{STDLOGIC}). Also, note that the \texttt{clock} and
% \texttt{reset_n} input ports declared in the port clause of the
% transition design are removed in the hvhdl{} version. In the \hvhdl{}
% abstract syntax, the if statements that were testing the value of the
% \texttt{clock} and the \texttt{reset_n} signals have been turned into
% specific sequential statements (i.e \texttt{rst}, \texttt{falling} and
% \texttt{rising} blocks). Thus, we don't need the \texttt{clock} and
% \texttt{reset_n} signals in the port declaration list anymore.

% Listing~\ref{lst:t-design-pss-abss} presents the translation in
% \hvhdl{} abstract syntax of the two of the processes presented in
% Listing~\ref{lst:trans-design-arch}.

% \begin{lstlisting}[language=VHDL,label={lst:t-design-pss-abss},
% caption={[Two processes in \hvhdl{} abstract syntax.]The \texttt{condition_evaluation} and \texttt{firable} processes in \hvhdl{} abstract syntax. The two processes are defined in the behavior of the transition design.},framexleftmargin=1.5em,xleftmargin=2em,numbers=left,
% numberstyle=\tiny\ttfamily]
%   process ("condition_evaluation", ("input_conditions"), 
%           (("v_internal_condition", boolean)),
%     v_internal_condition := true;
%     (for ("i", 0, conditions_number $-$ 1) 
%       (v_internal_condition := v_internal_conditions and input_conditions(i)));
%     s_condition_combination <= v_internal_condition
%   ) ||
%   process ("firable", ("clk"), $\emptyset$,
%     rst (s_firable <= false)
%         (falling (s_firable <= s_firing_condition))
%   )
% \end{lstlisting}

% In Listing~\ref{lst:t-design-pss-abss}, inner blocks of sequential
% statements are enclosed between parentheses to ease the reading,
% albeit parentheses are not part of the \hvhdl{} syntax. The body of
% the \texttt{firable} process gives an example of the use of a
% \texttt{rst} block and a \texttt{falling} block. One can see, from the
% comparison of Listing~\ref{lst:trans-design-arch} and
% \ref{lst:t-design-pss-abss}, that a \texttt{rst} ss ss' statement is
% the translation of the concrete VHDL statement \texttt{if reset_n =
%   '0' then ss else ss' endif;}. Comparing the same listings, a
% \texttt{falling} ss statement is the translation of the concrete VHDL
% statement \texttt{if falling_edge(clock) then ss end if;}.

% Listing~\ref{lst:t-design-ci-abss} shows the translation in \hvhdl{}
% abstract syntax of the component instantiation statement laid out in
% Listing~\ref{lst:trans-design-instantiation}.

% \begin{lstlisting}[language=VHDL,label={lst:t-design-ci-abss},
% caption={[A component instantiation statement in \hvhdl{} abstract
%   syntax (transition component instance).] An example of instantiation
%   of the transition design in \hvhdl{} abstract syntax. The transition
%   component instance is named
%   $id_t$.},framexleftmargin=1.5em,xleftmargin=2em,numbers=left,
% numberstyle=\tiny\ttfamily]
%   comp ($id_t$, "transition", 
%     -- Generic map
%     (transition_type => 0,
%      input_arcs_number => 1,
%      conditions_number => 1,
%      maximal_time_counter => 1),
%     -- Input port map
%     (time_A_value => 0,
%      time_B_value => 0,
%      input_conditions(0) => $id_0$,
%      input_arcs_valid(0) => $id_1$,
%      priority_authorizations(0) => true,
%      reinit_time(0) => $id_2$),
%     -- Output port map
%     (fired => $id_3$)
% \end{lstlisting}

% The statement of Listing~\ref{lst:t-design-ci-abss} is almost similar
% to the one of Listing~\ref{lst:trans-design-instantiation}. The
% \texttt{NOT_TEMPORAL} value associated to the \texttt{transition_type}
% constant is turned into 0 (remember the \texttt{temporal_t}
% enumeration type is transformed into a natural range type).

%%% Local Variables:
%%% mode: latex
%%% TeX-master: "../../main"
%%% End:


\section{Preliminary definitions}
\label{sec:sem-rules}
\subsection{Semantics Domains}

Let $id$ denote the set of identifiers in the semantical domain. We
write $prefix\mhyphen{}id$ to denote arbitrary subsets of the $id$
set. The $type$ and $value$ semantical types are defined as follows:

\begin{table}[H]
  \begin{tabular}{lcl}
    $type$ & ::= & \texttt{bool} | \texttt{nat(}$nat$,$nat$\texttt{)} | \texttt{array} \texttt{(}$type$,$nat$,$nat$\texttt{)} \\
           & & \\
    $value$ & ::= & $bool$ | $nat$ | $array$ \\
    $bool$ & ::= & '$\mathbf{\top}$' | '$\mathbf{\bot}$' \\
    $nat$ & ::= & \texttt{0} | \texttt{1} | \dots | \texttt{NATMAX} \\
    $array$ & ::= & \texttt{(}$value^{+}$\texttt{)}\\

  \end{tabular}
  \caption{The $type$ and $value$ semantical types.}
  \label{tab:type-value}
\end{table}

In Table~\ref{tab:type-value}, the $type$ type is in any way similar
to the $\tau$ entry of the abstract syntax, however, all constraint
bounds in the \texttt{nat} and \texttt{array} types have been
evaluated to natural numbers. $\mathtt{NATMAX}$ denotes the maximum
value for a natural number.  The $\mathtt{NATMAX}$ value depends on
the implementation of the VHDL language; $\mathtt{NATMAX}$ must at
least be equal to $2^{31}-1$. Note that the $array$ value contains at
least one value as an array's index range contains at least one index
(that is index 0).

\begin{notation}[Partial functions]
  Here, we present our notations pertaining to partial functions:
  \begin{itemize}[label=-]
  \item The $\nrightarrow$ arrow denotes a partial function.
  \item The $\rightarrow$ denotes an application (i.e, a total
    function).
  \item For all $f\in{}A\nrightarrow{}B$, $x\in{}f$ states that x is in
    the domain of function $f$.
  \item For all $f\in{}A\nrightarrow{}B$ and $g\in{}A\nrightarrow{}C$,
    $f\subseteq{}g$ states that the domain of $f$ is a subset of the
    domain of $g$.
  \item For all $X\subset{}A$ and $f\in{}A\nrightarrow{}B$,
    $X\subseteq{}f$ states that $X$ is a subset of the domain of $f$.
  \end{itemize}
\end{notation}

Now, let us define the structure of an elaborated design which is a
structure bound to a given \hvhdl{} design and to a design store, i.e
a global environment mapping identifiers to \hvhdl{} designs. Only the
designs referenced into the global design store can be instantiated as
subcomponents of a given design. The elaborated design structure is
used in the expression of the elaboration relation presented in
Section~\nameref{sec:elab-rules}, as well as in the expression of the
simulation rules. Let $ElDesign(d,\mathcal{D})$ be the set of the
elaborated designs for a given \hvhdl{} design $d$ and a design store
$D$. An elaborated design is a composite environment built out of
multiple sub-environments.  Each sub-environment is a table,
represented as a partial function, mapping identifiers of a certain
category of constructs (e.g, input port identifiers) to their
declaration information (e.g, type indication for input ports). We
represent an elaborated design as a record where the fields are the
sub-environments. An elaborated design is defined as follows:

\begin{definition}[Elaborated Design] For a given \hvhdl{} design
  $d\in{}design$ s.t. $d=$ \vhdle|design| \textit{$id_{ent}$
    $id_{arch}$ gens ports sigs behavior} and a given design store
  $\mathcal{D}\in{}entity\mhyphen{}id\nrightarrow{}design$, an
  elaborated design $\Delta\in{}ElDesign(d,\mathcal{D})$ is a record
  ${<}Gens, Ins, Outs, Sigs, Ps, Comps{>}$ where:
  \begin{itemize}[label=$-$]
  \item
    $Gens\in{}generic\mhyphen{}id\nrightarrow{}(type\times{}value)$
    where
    $generic\mhyphen{}id=\{id~\vert~(id,\tau,e)\in{}gens\}$,
    is the partial function yielding the type and the value of generic
    constants.
  \item $Ins\in{}input\mhyphen{}id\nrightarrow{}type$ where
    $input\mhyphen{}id=\{id~\vert~(\mathtt{in},id,\tau)\in{}ports\}$,
    is the partial function yielding the type of input ports.
  \item $Outs\in{}output\mhyphen{}id\nrightarrow{}type$ where
    $output\mhyphen{}id=\{id~\vert~(\mathtt{out},id,\tau)\in{}ports\}$,
    the partial function yielding the type of output ports.
  \item
    $Sigs\in{}declared\mhyphen{}signal\mhyphen{}id\nrightarrow{}type$
    where
    $declared\mhyphen{}signal\mhyphen{}id=\{id~\vert~(id,\tau)\in{}sigs\}$,
    the partial function yelding the type of declared signals.
  \item
    $Ps\in{}process\mhyphen{}id\nrightarrow{}(variable\mhyphen{}id(id_p)\nrightarrow{}(type\times{}value))$
    where\\
    $process\mhyphen{}id=\{id_p~\vert~\mathtt{process}(id_p,sl,vars,ss)\in{}behavior\}$,
    the partial function associating processes to their local
    environment. Local environments are functions mapping local
    variable identifiers to their corresponding type and
    value. Therefore, each set of local variable identifiers
    $variable\mhyphen{}id(id_p)$ depends on the process identifier
    (represented by $id_p$) passed as the first argument of the $Ps$
    function.
  \item
    $Comps\in{}component\mhyphen{}id\nrightarrow{}ElDesign(d_e,\mathcal{D})$,
    where\\
    $component\mhyphen{}id=\{id_c~\vert~\mathtt{comp}(id_c,id_e,gm,ipm,opm)\in{}behavior\}$,
    the partial function mapping component instance ids to their
    elaborated design version. The set $ElDesign(d_e,\mathcal{D})$
    depends on the design $d_e$ from which the component identifier
    $id_c$, passed as the first argument of the $Comps$ function, is
    an instance. Design $d_e$ is retrieved from the design store
    $\mathcal{D}$ s.t. $d_e=\mathcal{D}(id_e)$.
  \end{itemize}
\end{definition}

We assume that there are no overlapping between the identifiers of the
sub-environments (i.e, an identifier belongs to at most one
sub-environment). We note $\Delta(x)$ to denote the value returned for
identifier $x$, where $x$ is looked up in the appropriate field of
$\Delta$. We note $x\in\Delta$ to state that identifier $x$ is defined
in one of $\Delta$'s fields. We note $\Delta(x)\leftarrow{}v$ the
overriding of the value associated to identifier $x$ with value $v$ in
the appropriate field of $\Delta$, $\Delta\cup{}(x,v)$ to note the
addition the mapping from identifier $x$ to value $v$ in the
appropriate field of $\Delta$, that assuming $x\notin\Delta$. We note
$x\in\mathcal{F}(\Delta)$, where $\mathcal{F}$ is a field of $\Delta$,
when more precision is needed regarding the lookup of identifier $x$
in the record $\Delta$.

Let $\Sigma(\Delta)$ be the set of design states for a given
elaborated design $\Delta$.  A design state of $\Delta$ is defined as
follows:

\begin{definition}[Design state]
  A design state $\sigma\in\Sigma(\Delta)$, for a given design
  $d\in{}design$, a given design store $\mathcal{D}$ and an elaborated
  design $\Delta\in{}ElDesign(d,\mathcal{D})$, is a record
  ${<}\mathcal{S},\mathcal{C},\mathcal{E}{>}$ where:
  \begin{itemize}[label=$-$]
  \item $\mathcal{S}\in{}signal\mhyphen{}id\rightarrow{}value$, the
    partial function yielding the current values of the design's
    signals (ports and declared signals).
  \item
    $\mathcal{C}\in{}component\mhyphen{}id\rightarrow{}\Sigma(\Delta_c)$,
    the partial function yielding the current state of design's
    component instances, where $\Delta_c=\Delta(id_c)$ and
    $id_c\in{}component\mhyphen{}id$ is the component identifier
    passed to function $\mathcal{C}$.
  \item
    $\mathcal{E}\subseteq{}signal\mhyphen{}id\cup{}component\mhyphen{}id$,
    the set of signal and component instance ids that generated an
    event at the current simulation cycle.
  \end{itemize}
\end{definition}

The $signal\mhyphen{}id$ subset is the disjoint union of
$input\mhyphen{}id$, $output\mhyphen{}id$ and
$declared\mhyphen{}signal\mhyphen{}id$. We use $\sigma(id)$ to denote
the value associated to an identifier in the signal store
$\mathcal{S}$ or in the component store $\mathcal{C}$
fields. $id\in_\mathcal{E}\sigma$ states that an identifier belongs to
the event set $\mathcal{E}$, whereas $id\in\sigma$ states that an
identifier is defined in either the signal store $\mathcal{S}$ or the
component store $\mathcal{C}$ fields. $\sigma\cup_\mathcal{E}\{id\}$
denotes a new design state, in all points similar to $\sigma$ but
enriched with the identifier $id$ in its events set.

\begin{notation}[No events design state]
  For a given \hvhdl{} design $d$, a design store $\mathcal{D}$, and
  an elaborated design $\Delta\in{}ElDesign(d,\mathcal{D})$, the
  function $NoEv\in\Sigma(\Delta)\rightarrow\Sigma(\Delta)$ returns a
  design state similar to the one passed in parameter only with an
  empty set of events. I.e, for all design state
  $\sigma\in\Sigma(\Delta)$
  s.t. $\sigma={<}\mathcal{S},\mathcal{C},\mathcal{E}{>}$,
  $NoEv(\sigma)={<}\mathcal{S},\mathcal{C},\emptyset{>}$.
\end{notation}

% \subsection{Design evaluation workflow.}
% \label{subsec:elab-sim-wflow}

% The workflow to evaluate a top-level VHDL design comprises two phases
% with some pre-requisites.

% \begin{figure}[H]
%   \centering
%   \begin{tikzpicture}

%     \node(elab) {\textcircled{1} Elaboration phase};
%     \node(sim) at ($(elab.east)+(2.5,0)$) {\textcircled{2} Simulation phase};

%     \draw (elab.east) edge[->] ($(sim.west)-(.1,0)$);
%   \end{tikzpicture}
%   \caption{Workflow for \emph{top-level} design evaluation.}
%   \label{fig:toplevel-eval}
% \end{figure}

% \begin{itemize}[label=-]
% \item Pre-requisites to the top-level design evaluation are as
%   follows:

%   \begin{enumerate}
%   \item All types and subtypes defined in VHDL standard library packages
%     are present in the environment, under the $\Gamma$ parameter.
%   \item All types and subtypes defined in user-defined packages are
%     present in the environment, under the $\Gamma$ parameter.
%   \item All globally defined constants will be considered as known
%     values in the elaboration/simulation rules.
%   \item No shared variables or signals are globally declared.
%   \end{enumerate}
  
% \item The \emph{elaboration} phase:

%   \begin{enumerate}
%   \item Builds a functor for each design of the project, including for
%     the top-level design.
%   \item Type-check designs' declarative parts.
%   \item Type-checks designs' behavioral parts.
%   \end{enumerate}

% \item The \emph{simulation} phase:
%   \begin{enumerate}
%   \item Instantiates the top-level design thanks to the $I_p$ function
%     yelding the primary input values.
%   \item Simulates the top-level design until $T_c$, the count of
%     simulation cycles, equals $0$.
%   \end{enumerate}
% \end{itemize}

% \subsection{Notation of rules.}
% \todo[inline]{Add examples of type-checking rules.}

% Semantics rules are expressed by judgments that take the following form: 

% \begin{figure}[H]

  
%   \centering
%   \begin{prooftree}
%     \hypo{env &\vdash \mathtt{a}\xrightarrow{rel}i}
%   \end{prooftree}
% \end{figure}

% Here is how to interpret the above judgment: some syntactic element
% \texttt{a} of the VHDL language is interpreted as \textit{i}
% with respect to the relation \textit{rel}, in the environment
% \textit{env}.

% Moreover, semantics rules can be with or without premises.

% \begin{figure}[H]
%   {\fontsize{8}{11}\selectfont \textsc{Axiom}}
  
%   \begin{prooftree}
%     \infer0{
%       env \vdash \mathtt{a}{}\xrightarrow{rel}i
%     }
%   \end{prooftree}
% \end{figure}

% Semantics rules without premises, as the one shown above, are called
% axioms.

% \begin{figure}[H]
%   {\fontsize{8}{11}\selectfont \textsc{WithPremises}}
  
%   \begin{prooftree}
%     \hypo{env &\vdash \mathtt{a}\xrightarrow{rel}i}
%     \hypo{env &\vdash \mathtt{b}\xrightarrow{rel}j}
%     \infer2{
%       env \vdash S\mathtt{(a,b)}\xrightarrow{rel}I(i,j)
%     }
%   \end{prooftree}
% \end{figure}

% The above example depicts a rule with premises. The premises are
% placed above the judgment bar, and the conclusion appears below.  The
% above rule expresses that if the syntactic element \texttt{a} is
% interpreted as \textit{i} and the syntactic element \texttt{b} is
% interpreted as \textit{j} both with respect to relation \textit{rel},
% then the syntactic element $S\mathtt{(a,b)}$ is interpreted as
% $I\mathtt{(i,j)}$ with respect to relation \textit{rel}. Here, $S$ is
% a relation appearing in the syntax of the VHDL language and bounding
% syntactic elements \texttt{a} and \texttt{b}, whereas $I$ is a
% relation belonging to the interpretation world, bounding elements $i$
% and $j$ which are both interpretation results.

%%% Local Variables:
%%% mode: latex
%%% TeX-master: "../../main"
%%% End:


\section{Elaboration rules}
\label{sec:elab-rules}
The goal of the elaboration phase is to build an elaborated design
$\Delta$ along with a \emph{default} state $\sigma_e$, out of a
\hvhdl{} design $d$. The elaboration relation also covers
type-checking operations for the declarative and behavioral parts of
design $d$.

\subsection{Design elaboration.}
\label{subsubsec:design-elab}

\begin{definition}[Design store]
  \label{def:design-store}
  $\mathcal{D}\in{}entity\mhyphen{}id\nrightarrow{}\mathrm{design}$ is
  the partial function mapping entity ids to their corresponding
  representation in abstract syntax. As a prerequisite to the
  elaboration of \hilecop{}-generated designs (i.e, resulting from the
  transformation of a SITPN into an \hvhdl{} design), a particular
  design store $\mathcal{D}_\mathcal{H}$ is defined. Design store
  $\mathcal{D}_\mathcal{H}$ holds the description of the
  \texttt{place} and \texttt{transition} designs which full definition
  in abstract syntax are given in appendices \ref{sec:place-abs-synt}
  and \ref{sec:trans-abs-synt}.
\end{definition}

At the beginning of the elaboration phase, a function
$\mathcal{M}_g\in{}generic\mhyphen{}id\nrightarrow{}value$ mapping the
top-level design's generic constants to values is passed as an element
of the environment. The $\mathcal{M}_g$ function is refered to as the
\emph{dimensioning} function.

\begin{table}[H]
  \centering
  \begin{tabular}{@{}l}
    {\fontsize{8}{11}\selectfont\textsc{DesignElab}} \\    
    {\begin{prooftree}

        % Elaborates gens, ports, sigs and cs.

        \hypo{\Delta_{\emptyset},\mathcal{M}_g&\vdash{}\mathrm{gens}\xrightarrow{egens}\Delta}
        \infer[no rule]1{\Delta,\sigma_\emptyset&\vdash{}
          \mathrm{ports}\xrightarrow{eports}\Delta',\sigma}
        \infer[no rule]1{\Delta',\sigma&\vdash{}\mathrm{sigs}\xrightarrow{esigs}\Delta'',\sigma'}
        \infer[no rule]1{\mathcal{D},\Delta'',\sigma'&\vdash
          \mathrm{cs}\xrightarrow{ebeh}\Delta''',\sigma''}
        
        % Conclusion
        
        \infer[template={\fontsize{10}{13}\selectfont\inserttext}]1
        {
          $\mathcal{D},\mathcal{M}_g\vdash$
          \vhdle|design| id$_e$ id$_a$ gens ports sigs cs
          $\xrightarrow{elab}$
          $\Delta''',\sigma''$
        }
      \end{prooftree}} \\
  \end{tabular}
\end{table}

where $\Delta_\emptyset$ denotes an empty elaborated design, that is
an elaborated design initialized with empty fields (empty tables). In
the same manner, $\sigma_\emptyset$ denotes an empty design state.
The effect of the $egens$, $eports$, $esigs$ and $ebeh$ that
respectively deal with the elaboration of the generic constants, the
ports, the architecture declarative part and the behavioral part of
the design, are explicited in the following sections.

\subsection{Generic clause elaboration.}
\label{subsubsec:gen-clause-elab}

\begin{premises}
  \begin{itemize}[label=-]
  \item $etype_g$ transforms a subtype indication, specifically
    attached to a generic constant declaration, into a $type$ instance
    and checks its well-formedness (see \ref{subsec:type-elab}).
  \item The $e$ relation evaluates expression $e$ to value $v$ (see
    \ref{subsec:expr-rules}).
  \item $SE_l$ states that an expression is $locally$ static (see
    \ref{subsubsec:loc-static}).
  \item $v\in_c{}T$ and $\mathcal{M}(\mathrm{id}_g)\in_c{}T$ checks
    that the default value and the value of yielded by the
    dimensioning function belongs to the type of the declared generic
    constant (see \ref{subsec:constr-satif-rel}).
  \end{itemize}  
\end{premises}

\begin{table}[H]
  \begin{tabular}{@{}l}
    {\fontsize{8}{11}\selectfont\textsc{GenElabDimen}} \\
    {\begin{prooftree}

        % Well-formed type and elaborates type.
        \hypo{\vdash\tau\xrightarrow{etype_g}T}
        
        % Evaluates expr. default.
        \hypo{\Delta_\emptyset,\sigma_\emptyset,\Lambda_\emptyset&\vdash
          \mathrm{e}\xrightarrow{e}v}

        % Checks static expr. default
        \hypo{&SE_l(\mathrm{e})}

        % Checks the type of the value yielded by the dimensioning fun.
        \hypo{v&\in_c{}T}
        \infer[no rule]1{\mathcal{M}(\mathrm{id}_g)&\in_c{}T}
        
        % Conclusion
        \infer[template={\inserttext}]4
        [{\renewcommand{\arraystretch}{1.5}
          \begin{tabular}{l}
            $\mathrm{id}_g\notin\Delta$ \\ 
            $\mathrm{id}_g\in\mathcal{M}$\\
          \end{tabular}
        }]
        {
          $\Delta,\mathcal{M}\vdash
          \mathtt{(}\mathrm{id}_g\mathtt{,}\tau\mathtt{,}\mathrm{e}\mathtt{)}
          \xrightarrow{egens}$
          $\Delta\cup{}(id_g,(T,\mathcal{M}(\mathrm{id}_g)))$
        }
      \end{prooftree}} \\
  \end{tabular}
\end{table}%

\begin{figure}[H]
  {\fontsize{8}{11}\selectfont\textsc{GenElabDefault}}
  
  \begin{prooftree}

    % Well-formed type and elaborates type.
    \hypo{\vdash\tau\xrightarrow{etype_g}T}

    % Evaluates default value.
    \hypo{
      \Delta_\emptyset,\sigma_\emptyset,\Lambda_\emptyset&
      \vdash{}\mathrm{e}\xrightarrow{e}v}

    % Checks static expr. default value.
    \hypo{&SE_l(\mathrm{e})}    

    % Well-typed default.
    \hypo{v\in_c{}T}
    
    % Conclusion
    \infer[template={\inserttext}]4
    [{\renewcommand{\arraystretch}{1.5}
      \begin{tabular}{@{}l}
        $\mathrm{id}_g\notin\Delta$ \\
        $\mathrm{id}_g\notin\mathcal{M}$\\
      \end{tabular}
    }]
    {
      $\Delta,\mathcal{M}\vdash
      \mathtt{(}\mathrm{id}_g\mathtt{,}\tau\mathtt{,}\mathrm{e}\mathtt{)}
      \xrightarrow{egens}$
      $\Delta\cup{}(id_g,(T,v))$
    }
  \end{prooftree}
\end{figure}

\begin{figure}[H]
  {\fontsize{8}{11}\selectfont\textsc{GenElabComp}}
    
  \begin{prooftree}

    % Evaluates GE1
    \hypo{\Delta,\mathcal{M}\vdash\mathrm{gdecl}\xrightarrow{egens}{}\Delta'}

    % Evaluates GE2.
    \hypo{\Delta',\mathcal{M}\vdash\mathrm{gens}\xrightarrow{egens}{}\Delta''}

    % Conclusion
    \infer[template={\inserttext}]2{
      $\Delta,\mathcal{M}\vdash
      \mathrm{gdecl}\mathtt{;}~\mathrm{gens}
      \xrightarrow{egens}\Delta''$
    }
  \end{prooftree}
\end{figure}

\subsection{Port clause elaboration.}
\label{subsubsec:port-clause-elab}

\begin{premises}
  The $defaultv$ relation yields the implicit \emph{default} value for
  a given type.
\end{premises}

% Elaborate IN port.

\begin{figure}[H]
  {\fontsize{8}{11}\selectfont\textsc{InPortElab}}
  
  \begin{prooftree}
        
    % Elaborates type.
    \hypo{\Delta\vdash\tau\xrightarrow{etype}T}

    % Retrieves leftmost value of type T.
    \hypo{\Delta\vdash{}T\xrightarrow{defaultv}v}
    
    % Conclusion
    \infer[template={\inserttext}]2
    [{
      \begin{tabular}{@{}l}
        $\mathrm{id}\notin\Delta$ \\
        $\mathrm{id}\notin\sigma$ \\
      \end{tabular}
    }]
    {
      $\Delta,\sigma\vdash
      ($\vhdle|in|$,\mathrm{id},\tau)$
      $\xrightarrow{eports}$
      $\Delta\cup{}(id,T),$
      $\sigma\cup(id,v)$
    }
  \end{prooftree}
\end{figure}

% Elaborate OUT port.

\begin{figure}[H]
  {\fontsize{8}{11}\selectfont\textsc{OutPortElab}}
  
  \begin{prooftree}

    % Elaborates type.
    \hypo{\Delta\vdash\tau\xrightarrow{etype}T}

    % Retrieves leftmost value of type T.
    \hypo{\Delta\vdash{}T\xrightarrow{defaultv}v}
    
    % Conclusion
    \infer[template={\inserttext}]2
    [{
      \begin{tabular}{@{}l}
        $\mathrm{id}\notin\Delta$ \\
        $\mathrm{id}\notin\sigma$ \\
      \end{tabular}
    }]
    {
      $\Delta,\sigma\vdash
      ($\vhdle|out|$,\mathrm{id},\tau)$
      $\xrightarrow{eports}$
      $\Delta\cup{}(id,T),$
      $\sigma\cup(id,v)$
    }
  \end{prooftree}
\end{figure}

% Inductive def. of port elab.

\begin{figure}[H]
  {\fontsize{8}{11}\selectfont\textsc{PortElabComp}}
  
  \begin{prooftree}

    % Evaluates PE1
    \hypo{\Delta,\sigma\vdash\mathrm{pdecl}\xrightarrow{eports}{}\Delta',\sigma'}

    % Evaluates PE2.
    \hypo{\Delta',\sigma'\vdash\mathrm{ports}\xrightarrow{eports}{}\Delta'',\sigma''}

    % Conclusion
    \infer[template={\inserttext}]2{
      $\Delta,\sigma\vdash
      \mathrm{pdecl}\mathtt{;}~\mathrm{ports}
      \xrightarrow{eports}\Delta'',\sigma''$
    }
  \end{prooftree}
\end{figure}

\subsection{Architecture declarative part elaboration.}
\label{subsubsec:arch-decl-part-elab}

% Signal elaboration rule.

\begin{figure}[H]
  {\fontsize{8}{11}\selectfont\textsc{SigElab}}
  
  \begin{prooftree}

    % Elaborates type.
    \hypo{\Delta\vdash\tau\xrightarrow{etype}T}
    
    % Retrieves leftmost value of type T.
    \hypo{\Delta\vdash{}T\xrightarrow{defaultv}v}
    
    % Conclusion
    \infer[template={\inserttext}]2
    [{
      \begin{tabular}{@{}l}
        $\mathrm{id}\notin\Delta$ \\
        $\mathrm{id}\notin\sigma$ \\
      \end{tabular}
    }]
    {
      $\Delta,\sigma\vdash$
      $(\mathrm{id},\tau)$
      $\xrightarrow{esigs}$
      $\Delta\cup{}(id,T),$
      $\sigma\cup(id,v)$
    }
  \end{prooftree}
\end{figure}


% Inductive elaboration rule for sequence of decl. entries.

\begin{figure}[H]
  {\fontsize{8}{11}\selectfont\textsc{SigElabComp}}
  
  \begin{prooftree}

    % Evaluates sdecl.
    \hypo{\Delta,\sigma\vdash\mathrm{sdecl}\xrightarrow{esigs}{}\Delta',\sigma'}

    % Evaluates sigs.
    \hypo{\Delta',\sigma'\vdash\mathrm{sigs}\xrightarrow{esigs}{}\Delta'',\sigma''}

    % Conclusion
    \infer[template={\inserttext}]2{
      $\Delta,\sigma\vdash
      \mathrm{sdecl}\mathtt{;}~\mathrm{sigs}
      \xrightarrow{esigs}\Delta'',\sigma''$
    }
  \end{prooftree}
\end{figure}

\subsection{Type indication elaboration.}
\label{subsec:type-elab}

The $etype$ relation checks the well-formedness of a type indication
$\tau$, and transforms it into a semantical $type$ (as defined in
Table~\ref{tab:type-value}). A subtype indication $\tau$ is well-formed
in the context $\Delta$ if $\tau$ denotes the \texttt{boolean} keyword
or the \texttt{nat} or \texttt{array} keywords with a
\emph{well-formed} constraint, and a well-formed element type in the
\texttt{array} case.

\begin{table}[H]
  \centering
  \begin{tabular}{@{}l}
    {\fontsize{8}{11}\selectfont\textsc{ETypeBool}} \\
    {\begin{prooftree}
        
        % Conclusion
        \infer[template={\inserttext}]0
        {
          $\Delta\vdash$
          $~\mathtt{boolean}$
          $\xrightarrow{etype}$
          $\mathtt{bool}$
        }
      \end{prooftree}} \\
  \end{tabular}
  \begin{tabular}{@{}l}
    {\fontsize{8}{11}\selectfont\textsc{ETypeNat}} \\
    {\begin{prooftree}
        
        % Well-formed constraint and evaluates constraint.
        \hypo{
          \Delta\vdash
          (\mathrm{e},\mathrm{e}')
          \xrightarrow{econstr}
          (v,v')
        }
        
        % Conclusion
        \infer[template={\inserttext}]1
        {
          $\Delta\vdash$
          $~\mathtt{natural}(\mathrm{e},\mathrm{e}')$
          $\xrightarrow{etype}$
          $\mathtt{nat}(v,v')$
        }
      \end{prooftree}} \\
  \end{tabular}
\end{table}

\begin{table}[H]
  \centering
  \begin{tabular}{@{}l}
    {\fontsize{8}{11}\selectfont\textsc{ETypeArray}} \\
    {\begin{prooftree}

        % Checks well-formednes and evaluates tau_ind.
        \hypo{\Delta\vdash\tau\xrightarrow{etype}T}

        % Checks well-formednes and evaluates index constraint.
        \hypo{\Delta\vdash(\mathrm{e},\mathrm{e'})\xrightarrow{econstr}(v,v')}
        
        % Conclusion
        \infer[template={\inserttext}]2
        {
          $\Delta\vdash
          ~\mathtt{array(}\tau,\mathrm{e},\mathrm{e'}\mathtt{)}$
          $\xrightarrow{etype}$
          $\mathtt{array(}T,v,v'\mathtt{)}$
        }
      \end{prooftree}} \\
  \end{tabular}
\end{table}

The $econstr$ relation checks that a constraint is well-formed and
evaluates the constraint bounds.  A constraint $c$ is well-formed in
the context $\Delta$ if:

\begin{itemize}[label=-]
\item its bounds are globally static expressions \cite[p.36]{VHDL00}
  of the \texttt{nat} type.
\item its lower bound value is inferior or equal to its upper bound
  value.
\end{itemize}

\begin{remark}[Type of constraints]
  As the VHDL language reference stays unclear about the type of range
  and index constraints \cite[p.33]{VHDL00}, we add the restriction
  that range and index constraints must have bounds of the
  \texttt{nat} type (i.e, value of type $nat$).
\end{remark}

\begin{premises}
  $SE_g$ states that an expression is \emph{globally} static (see
  \ref{subsubsec:glob-static}).
\end{premises}

\begin{table}[H]
  \centering
  \begin{tabular}{@{}l}
    {\fontsize{8}{11}\selectfont\textsc{EConstr}} \\
    {\begin{prooftree}
        
        
        % Globally static e and e'.
        \hypo{\Delta\vdash{}SE_g(\mathrm{e})}
        \infer[no rule]1{\Delta\vdash{}SE_g(\mathrm{e}')}
        
        % Evaluates e and e'
        \hypo{\Delta,\sigma_\emptyset,\Lambda_\emptyset&\vdash\mathrm{e}\xrightarrow{e}v}
        \infer[no rule]1{\Delta,\sigma_\emptyset,\Lambda_\emptyset&\vdash\mathrm{e}'\xrightarrow{e}v'}

        % Well-typed e and e'.
        \hypo{v&\in_c{}\mathtt{nat}(0,\mathtt{NATMAX})}
        \infer[no rule]1{v'&\in_c{}\mathtt{nat}(0,\mathtt{NATMAX})}
        
        % Conclusion
        \infer[template={\inserttext}]3
        [{
          \renewcommand{\arraystretch}{1.5}
          \begin{tabular}{@{}l}
            $v\leq{}v'$ \\
          \end{tabular}
        }]
        {
          $\Delta\vdash$
          $(\mathrm{e},\mathrm{e}')$
          $\xrightarrow{econstr}$
          $(v,v')$
        }
      \end{prooftree}} \\
  \end{tabular}
\end{table}

When considering a type indication in a generic constant declaration,
the definition of well-formedness differs slightly from the general
definition. A type indication $\tau$ associated to a generic constant
declaration is well-formed if $\tau$ denotes the \texttt{boolean}
keyword, or the \texttt{nat} keyword with a \emph{well-formed}
constraint.

The $etype_g$ relation is specially defined to check the
well-formedness of a subtype indication associated with a generic
constant declaration.

\begin{table}[H]
  \centering
  \begin{tabular}{@{}l}
    {\fontsize{8}{11}\selectfont\textsc{ETypeGBool}} \\
    {\begin{prooftree}
        
        % Conclusion
        \infer[template={\inserttext}]0
        {
          $\vdash$
          $~\mathtt{boolean}$
          $\xrightarrow{etype}$
          $\mathtt{bool}$
        }
      \end{prooftree}} \\
  \end{tabular}
    \begin{tabular}{@{}l}
    {\fontsize{8}{11}\selectfont\textsc{ETypeGNat}} \\
    {\begin{prooftree}
        
        % Well-formed constraint and evaluates constraint.
        \hypo{
          \Delta\vdash
          (\mathrm{e},\mathrm{e}')
          \xrightarrow{econstr_g}
          (v,v')
        }
        
        % Conclusion
        \infer[template={\inserttext}]1
        {
          $\vdash$
          $~\mathtt{natural}(\mathrm{e},\mathrm{e}')$
          $\xrightarrow{etype}$
          $\mathtt{nat}(v,v')$
        }
      \end{prooftree}} \\
  \end{tabular}
\end{table}

The $econstr_g$ relation checks that a \emph{generic} constraint (i.e,
a constraint appearing in a type indication associated with a generic
constant declaration) is well-formed and evaluates the constraint
bounds.

A \emph{generic} constraint $c$ is well-formed if:

\begin{itemize}[label=-]
\item its bounds are locally static expressions \cite[p.36]{VHDL00} of
  the \texttt{nat} type.
\item its lower bound value is inferior or equal to its upper bound
  value.
\end{itemize}

\begin{table}[H]
  \centering
  \begin{tabular}{@{}l}
    {\fontsize{8}{11}\selectfont\textsc{EConstrG}} \\
    {\begin{prooftree}
        
        % Locally static e and e'.
        \hypo{&SE_l(\mathrm{e})}
        \infer[no rule]1{&SE_l(\mathrm{e}')}
        
        % Evaluates e and e'
        \hypo{\Delta_\emptyset,\sigma_\emptyset,\Lambda_\emptyset&\vdash\mathrm{e}\xrightarrow{e}v}
        \infer[no rule]1{\Delta_\emptyset,\sigma_\emptyset,\Lambda_\emptyset&\vdash\mathrm{e}'\xrightarrow{e}v'}

        % Well-typed v and v'.
        \hypo{v&\in_c\mathtt{nat}(0,\mathtt{NATMAX})}
        \infer[no rule]1{v'&\in_c\mathtt{nat}(0,\mathtt{NATMAX})}
        
        % Conclusion
        \infer[template={\inserttext}]3
        [{
          \renewcommand{\arraystretch}{1.5}
          \begin{tabular}{@{}l}
            $v\leq{}v'$ \\
          \end{tabular}
        }]
        {
          $\vdash$
          $(\mathrm{e},\mathrm{e}')$
          $\xrightarrow{econstr_g}$
          $(v,v')$
        }
      \end{prooftree}} \\
  \end{tabular}
\end{table}

\subsection{Behavior elaboration.}
\label{subsec:beh-elab}

\subsubsection{Elaboration of concurrent statements.}
\label{subsubsec:conc-elab}

As it is, the elaboration of the composition of concurrent statements
is performed in a sequential manner.

\begin{table}[H]
  \begin{tabular}{@{}l}
  {\fontsize{8}{11}\selectfont\textsc{CsParElab}} \\
  {\begin{prooftree}

    % Elaborates first conc. stmt.
    \hypo{\mathcal{D},\Delta,\sigma\vdash\mathrm{cs}\xrightarrow{ebeh}\Delta',\sigma'}

    % Elaborates second conc. stmt.
    \hypo{\mathcal{D},\Delta',\sigma'\vdash\mathrm{cs'}\xrightarrow{ebeh}\Delta'',\sigma''}
    
    % Conclusion
    \infer[template={\inserttext}]2
    {
        $\mathcal{D},\Delta,\sigma\vdash$
        $~\mathrm{cs}~||~\mathrm{cs'}$
        $\xrightarrow{ebeh}$
        $\Delta'',\sigma''$
    }
  \end{prooftree}} \\
  \end{tabular}
  \begin{tabular}{l}
    {\fontsize{8}{11}\selectfont\textsc{CsNullElab}} \\
    {\begin{prooftree}

        % Conclusion
        \infer[template={\inserttext}]0
        {
          $\mathcal{D},\Delta,\sigma\vdash$
          $~\mathtt{null}$
          $\xrightarrow{ebeh}$
          $\Delta,\sigma$
        }
      \end{prooftree}} \\
  \end{tabular}
\end{table}

\subsubsection{Process elaboration.}
\label{subsubsec:ps-elab}

% Well-typed process with sensitivity list and vars.

\begin{premises}
  $\mathtt{valid}_{ss}$ states that a sequential statement is
  well-typed.
\end{premises}

\begin{sideconds}
  $\mathrm{sl}\subseteq{}Ins(\Delta)\cup{}Sigs(\Delta)$ indicates that
  the sensitivity list \emph{sl} must only contain signal identifiers
  that are readable, that is, \emph{input} ports and declared signals.
\end{sideconds}

\begin{figure}[H]
  {\fontsize{8}{11}\selectfont\textsc{PsElab}}
  \vspace{1em}
  
  \begin{prooftree}

    % Elaborates local context.
    \hypo{\Delta,\Lambda_\emptyset\vdash\mathrm{vars}\xrightarrow{evars}\Lambda}

    % Well-typed sequential statement.
    \hypo{\Delta,\sigma,\Lambda\vdash{}\mathtt{valid}_{ss}(\mathrm{ss})}
    
    % Conclusion
    \infer[template={\inserttext}]2
    [{\renewcommand{\arraystretch}{1.5}
      \begin{tabular}{@{}l}
        $\mathrm{id}_p\notin\Delta$ \\
        $\mathrm{sl}\subseteq{}Ins(\Delta)\cup{}Sigs(\Delta)$ \\
      \end{tabular}
    }]
    {
        $\mathcal{D},\Delta,\sigma\vdash$
        $~$\vhdle|process| \texttt{(}id$_p$\texttt{,} sl\texttt{,} vars\texttt{,} ss\texttt{)}
        $\xrightarrow{ebeh}$
        $\Delta\cup{}(id_p,\Lambda),\sigma$
    }
  \end{prooftree}
\end{figure}

\subsubsection{Process declarative part elaboration.}
\label{subsubsec:ps-decl-elab}

The $evars$ relation builds a local environment out of a process
declarative part.

% Variable declaration elaboration.

\begin{figure}[H]
  {\fontsize{8}{11}\selectfont\textsc{VarElab}}
  
  \begin{prooftree}
        
    % Elaborates type ind.
    \hypo{\Delta\vdash\tau\xrightarrow{etype}T}

    % Retrieves leftmost value of type T.
    \hypo{\vdash{}T\xrightarrow{defaultv}v}
    
    % Conclusion
    \infer[template={\inserttext}]2
    [$\mathrm{id}\notin{}\Lambda,\mathrm{id}\notin\Delta$]
    {
      $\Delta,\Lambda\vdash$
      $~(\mathrm{id},\tau)$
      $\xrightarrow{evars}$
      $\Lambda\cup{}(id,(T,v))$
    }
  \end{prooftree}
\end{figure}

% Inductive elab. rule for composition of variable declaration.

\begin{figure}[H]
  {\fontsize{8}{11}\selectfont\textsc{VarElabComp}}
  
  \begin{prooftree}

    % Evaluates VE1
    \hypo{\Delta,\Lambda\vdash\mathrm{vdecl}\xrightarrow{evars}{}\Lambda'}

    % Evaluates VE2.
    \hypo{\Delta,\Lambda'\vdash\mathrm{vars}\xrightarrow{evars}{}\Lambda''}

    % Conclusion
    \infer[template={\inserttext}]2{
      $\Delta,\Lambda\vdash
      ~\mathrm{vdecl}\mathtt{,}~\mathrm{vars}
      \xrightarrow{evars}\Lambda''$
    }
  \end{prooftree}
\end{figure}

\subsubsection{Component instantiation elaboration.}
\label{subsubsec:comp-inst-elab}

\begin{premises}
  \begin{itemize}[label=-]
  \item The $emapg$ relation binds a generic map to a function
    $\mathcal{M}\in{}id\nrightarrow{}value$ (see definition below).
  \item \texttt{valid}$_{ipm}$ (resp. \texttt{valid}$_{opm}$) states
    that a in port map (resp. out port map) is valid, i.e well-formed
    and well-typed (see \ref{subsec:valid-pm}).
  \end{itemize}
\end{premises}

\begin{sideconds}
  $\mathcal{M}\subseteq{}Gens(\Delta_c)$ checks that
  the generic map \emph{gmap} contains references to known generic
  constant identifiers only.
\end{sideconds}

\begin{table}[H]
  \begin{tabular}{@{}l}
    {\fontsize{8}{11}\selectfont\textsc{CompElab}} \\
    {\begin{prooftree}
        
        % Creates mapping and elaborates gens.
        \hypo{\mathcal{M}_\emptyset&\vdash\mathrm{gmap}\xrightarrow{emapg}\mathcal{M}}
        
        % Elaborates design.
        \hypo{
          \mathcal{D},\mathcal{M}&\vdash
          \mathcal{D}(\mathrm{id}_e)\xrightarrow{elab}\Delta_c,\sigma_c}

        % Checks in port map validity.
        \hypo{\Delta,\Delta_c,\sigma&\vdash\mathtt{valid}_{ipm}(\mathrm{ipmap})}

        % Checks out port map validity.
        \infer[no rule]1{\Delta,\Delta_c&\vdash\mathtt{valid}_{opm}(\mathrm{opmap})}
        
        % Conclusion
        \infer[template={\inserttext}]3
        [{
          \renewcommand{\arraystretch}{1.5}
          \begin{tabular}{@{}l}
            $\mathrm{id}_c\notin\Delta$, $\mathrm{id}_c\notin\sigma$ \\
            $\mathrm{id}_e\in{}\mathcal{D}$\\
            % $\mathcal{D}(\mathrm{id}_e)=$ \vhdlep{7}{9}|design| id$_{ce}$ id$_{ca}$ gens ports sigs cs \\
            $\mathcal{M}\subseteq{}Gens(\Delta_c)$ \\
          \end{tabular}
        }]
        {
          $\mathcal{D},\Delta,\sigma\vdash$
          \vhdle|comp| $(\mathrm{id}_c, \mathrm{id}_e, \mathrm{gmap}, \mathrm{ipmap}, \mathrm{opmap})\xrightarrow{ebeh}$
          $\Delta\cup{}(id_c,\Delta_c),$
          $\sigma\cup{}(id_c,\sigma_c)$
        }
      \end{prooftree}} \\
  \end{tabular}
\end{table}

A port map is a mapping between signals coming from an embedding
design ($\Delta$) and ports of an internal component instance
($\Delta_c$). The formal part of an port map entry (i.e, left of the
arrow) belongs to the internal component, whereas the actual part
(i.e, right of the arrow) refers to the embedding design. Therefore,
we need both $\Delta$ and $\Delta_c$ to verify if a port map is
well-typed leveraging the $\mathtt{valid}_{pm}$ predicate.

\begin{remark}[Valid generic map]
  Note that we are not checking the validity of the generic map.  In
  case of an ill-formed generic map, a inconsistent mapping
  $\mathcal{M}$ is generated by the \emph{emapg} that will make the
  \emph{elab} relation, taking $\mathcal{M}$ as a parameter, never
  derivable. Therefore, the \emph{elab} relation does an implicit
  validity check on the generic map.
\end{remark}

% Creates a mapping function from an assoc. list.

% Case formal part is not partial.

\begin{table}[H]
  \begin{tabular}{@{}l}
    {\fontsize{8}{11}\selectfont\textsc{AssocGElab}} \\
    
    {\begin{prooftree}

        % Locally static expression e.
        \hypo{SE_l(\mathrm{e})}
        
        % Evaluates e.
        \hypo{\Delta_\emptyset,\sigma_\emptyset,\Lambda_\emptyset\vdash\mathrm{e}\xrightarrow{e}v}

        % Conclusion
        \infer[template={\inserttext}]2
        [$\mathrm{id}_g\notin\mathcal{M}$]
        {
          $\mathcal{M}\vdash$
          $~\mathrm{id}_g\Rightarrow\mathrm{e}$
          $\xrightarrow{emapg}$
          $\mathcal{M}\cup{}(\mathrm{id}_g,v)$
        }
      \end{prooftree}} \\
  \end{tabular}
  \begin{tabular}{@{}l}
    {\fontsize{8}{11}\selectfont\textsc{GmapElabComp}} \\
    
    {\begin{prooftree}

        % Collects assoc_g.
        \hypo{\mathcal{M}\vdash\mathrm{assoc}_g\xrightarrow{emapg}\mathcal{M}'}

        % Collects lassoc_g.
        \hypo{\mathcal{M}'\vdash\mathrm{gmap}\xrightarrow{emapg}\mathcal{M}''}

        % Conclusion
        \infer[template={\inserttext}]2
        {
          $\mathcal{M}\vdash
          \mathrm{assoc}_g,~\mathrm{gmap}$
          $\xrightarrow{emapg}\mathcal{M}''$
        }
      \end{prooftree}} \\
  \end{tabular}
\end{table}

An $\mathrm{assoc}_g$ entry doesn't allow indexed id in its formal
part, due to the restriction of generic constants to scalar types.
Note that this restriction is not imposed by the VHDL language
reference; it is our stance towards a simplification of the VHDL
semantics.

\subsection{Implicit default value.}
\label{subsec:implicit-default-v}

According to the VHDL reference, when declaring a port, a signal or a
variable, these items must receive an implicit default value depending
on their types \cite[p.61, 64, 173]{VHDL00}. The $defaultv$ relation
determines the default value for a given type.

\begin{table}[H]
  \centering
  \begin{tabular}{@{}l}
    {\fontsize{10}{13}\selectfont\textsc{DefaultVBool}} \\
    {\begin{prooftree}

        % Conclusion
        \infer[template={\inserttext}]0
        {
          $\mathtt{bool}$
          $\xrightarrow{defaultv}\bot$
        }
      \end{prooftree}} \\
  \end{tabular}
  \begin{tabular}{@{}l}
    {\fontsize{10}{13}\selectfont\textsc{DefaultVCNat}} \\
    {\begin{prooftree}

        % Conclusion
        \infer[template={\inserttext}]0
        [$n\le{}m$]
        {
          $\mathtt{nat}(n,m)$
          $\xrightarrow{defaultv}n$
        }
      \end{prooftree}} \\
  \end{tabular}
\end{table}

\begin{table}[H]
  \centering
  \begin{tabular}{@{}l}
    {\fontsize{10}{13}\selectfont\textsc{DefaultVCArr}} \\
    {\begin{prooftree}

        % Retrieves default value for element subtype.
        \hypo{T\xrightarrow{defaultv}v}
        
        % Conclusion
        \infer[template={\inserttext}]1
        [
        \begin{tabular}{@{}l}
          $n\le{}m$ \\
          $size=(m-n)+1$ \\
        \end{tabular}
        ]
        {
          $\mathtt{array}(T,n,m)$
          $\xrightarrow{defaultv}$
          $\mathtt{create\_array}(size,T,v)$
        }
      \end{prooftree}} \\
  \end{tabular}
\end{table}

$\mathtt{create\_array}(size,T,v)$ creates an array of size $size$,
containing element of type $T$, where each element is initialized with
the value $v$.

%%% Local Variables:
%%% mode: latex
%%% TeX-master: "../../main"
%%% End:

\subsection{Typing relation}
\label{sec:constr-satif-rel}

The typing relation $\in_c$ checks that a given value conforms to a
given type.

\begin{table}[H]
  \centering
  \begin{tabular}{@{}l}
    {\fontsize{8}{11}\selectfont\textsc{IsBool}} \\
    {\begin{prooftree}

        % Conclusion.
        \infer0[$b\in{}\mathbb{B}$]{b\in_c{}bool}
      \end{prooftree}} \\
  \end{tabular}
  \begin{tabular}{@{}l}
    {\fontsize{8}{11}\selectfont\textsc{IsCNat}} \\
    {\begin{prooftree}

        % Conclusion.
        \infer0[$n\in{}[l,u]$]{n\in_c{}nat(l,u)}
      \end{prooftree}} \\
  \end{tabular}
  \begin{tabular}{@{}l}
    {\fontsize{8}{11}\selectfont\textsc{Array}} \\
    {\begin{prooftree}
        
        % Subelements satisfy subtype indication's constraint.
        \hypo{v_i\in_c{}T}
        
        % Conclusion.
        \infer1
        [{
          \begin{tabular}{@{}l}
            $i=1,\dots,n$ \\
            $n=(u-l)+1$ \\
          \end{tabular}
        }]
        {
          \Delta\vdash
          \mathtt{(}v_1,\dots,v_n\mathtt{)}
          \in_c
          array(T,l,u)}
      \end{prooftree}} \\
  \end{tabular}
\end{table}

\subsection{Static expressions}
\label{subsubsec:wfe}

Static expressions are either locally static or globally static; the
LRM defines locally static and globally static expressions as follows.

\subsubsection{Locally static expressions}
\label{subsubsec:loc-static}

An expression is \emph{locally} static if:

\begin{itemize}[label=-]
\item It is composed of operators and operands of a \emph{scalar} type
  (i.e, \texttt{natural} or \texttt{boolean}).
\item It is a \emph{literal} of a scalar type.
\end{itemize}

The $SE_l$ relation, defined by the following rules, states that an
expression is locally static.

\begin{table}[H]
  \begin{tabular}{@{}l}
    {\fontsize{8}{11}\selectfont\textsc{LSENat}} \\
    {\begin{prooftree}
        
        % Conclusion
        \infer[template={\inserttext}]0
        [$\mathrm{n}\in\mathbb{N}$]
        {$SE_l(\mathrm{n})$}
      \end{prooftree}} \\
  \end{tabular}
  \begin{tabular}{@{}l}
    {\fontsize{8}{11}\selectfont\textsc{LSEBool}} \\
    {\begin{prooftree}
        
        % Conclusion
        \infer[template={\inserttext}]0
        [$\mathrm{b}\in\mathbb{B}$]
        {$SE_l(\mathrm{b})$}
      \end{prooftree}} \\
  \end{tabular}
  \begin{tabular}{@{}l}
    {\fontsize{8}{11}\selectfont\textsc{LSENot}} \\
    {\begin{prooftree}

        % Locally static e.
        \hypo{SE_l(\mathrm{e})}
        
        % Conclusion
        \infer[template={\inserttext}]1
        {
          $SE_l(\mathtt{not}~\mathrm{e})$
        }
      \end{prooftree}} \\
  \end{tabular}
  \begin{tabular}{@{}l}
    {\fontsize{8}{11}\selectfont\textsc{LSEBinOp}}\\    
    {\begin{prooftree}

        % Locally static e.
        \hypo{SE_l(\mathrm{e})}

        % Locally static e'.
        \hypo{SE_l(\mathrm{e'})}
        
        % Conclusion
        \infer[template={\inserttext}]2
        [$\mathtt{op}\in\{~+,-,=,\neq,<,\le,>,\ge,\mathtt{and},\mathtt{or}~\}$]
        {
          $SE_l(\mathrm{e}~\mathtt{op}~\mathrm{e'})$
        }
      \end{prooftree}}\\
  \end{tabular}
\end{table}

\subsubsection{Globally static expressions}
\label{subsubsec:glob-static}

An expression is \emph{globally} static in the context $\Delta$ if:

\begin{itemize}[label=-]
\item It is a generic constant.
\item It is an array aggregate composed of globally static
  expressions.
\item It is a locally static expression.
\end{itemize}

The $SE_g$ relation, defined by the following rules, checks that an
expression is globally static is a given context $\Delta$.

\begin{table}[H]
  \centering
  \begin{tabular}{@{}l}
    {\fontsize{8}{11}\selectfont\textsc{GSELocal}} \\
    {\begin{prooftree}

      % Locally static e.
      \hypo{SE_l(\mathrm{e})}
      
      % Conclusion
      \infer[template={\inserttext}]1
      {
        $\Delta\vdash{}SE_g(\mathrm{e})$
      }
    \end{prooftree}} \\
  \end{tabular}
  \begin{tabular}{@{}l}
    {\fontsize{8}{11}\selectfont\textsc{GSEGen}} \\
    {\begin{prooftree}
      
      % Conclusion
      \infer[template={\inserttext}]0
      [{\renewcommand{\arraystretch}{1.5}
        \begin{tabular}{@{}l}
          $\mathrm{id}_g\in{}Gens(\Delta)$ \\
        \end{tabular}
      }]
      {
        $\Delta\vdash{}SE_g(\mathrm{id}_g)$
      }
    \end{prooftree}} \\
  \end{tabular}
  \begin{tabular}{@{}l}
    {\fontsize{8}{11}\selectfont\textsc{GSEAggregate}} \\
    {\begin{prooftree}

        % Locally static e.
        \hypo{\Delta\vdash{}SE_g(\mathrm{e}_i)}
        
        % Conclusion
        \infer[template={\inserttext}]1
        [$i=1,\dots,n$]
        {
          $\Delta\vdash{}SE_g($\texttt{(}$\mathrm{e}_1,\dots,\mathrm{e}_n$\texttt{)}$)$
        }
      \end{prooftree}} \\
  \end{tabular}
\end{table}

\subsection{Valid port map}
\label{subsec:valid-pm}

\paragraph{Valid input port map}

The $\mathtt{valid}_{ipm}$ predicate states that an input port map is
valid in the context $\Delta,\Delta_c$, where $\Delta$ is the
embedding design structure and $\Delta_c$ denotes the component
instance, owner of the input port map, if:

\begin{itemize}[label=-]
\item All ports defined in $\Delta_c$ are exactly mapped once in the
  input port map.
\item For each input port map entry, the formal and actual part are of
  the same type.
\end{itemize}

\begin{premises}
  \begin{itemize}[label=-]
  \item $list_{ipm}$ builds a set
    $\mathcal{L}\subset{}id\sqcup{}(id\times\mathbb{N})$ out of the
    input port map.
  \item $\mathtt{check}_{pm}$ checks the validity of a port map based
    on the corresponding port list (here, the input ports of
    $\Delta_c$) and the set built by the $list_{ipm}$ relation.
  \end{itemize}
\end{premises}

\begin{table}[H]
  \centering
  \begin{tabular}{l}
    {\fontsize{8}{11}\selectfont\textsc{ValidIPM}} \\
    {\begin{prooftree}

        % Lists formal parts.
        \hypo{\Delta,\Delta_c,\sigma,\mathcal{L}_\emptyset\vdash\mathrm{i}\xrightarrow{list_{ipm}}\mathcal{L}}

        % Well-formed port map.
        \hypo{\mathtt{check}_{pm}(Ins(\Delta_c),\mathcal{L})}
        
        % Conclusion
        \infer[template={\inserttext}]2
        {
          $\Delta,\Delta_c,\sigma\vdash\mathtt{valid}_{ipm}(\mathrm{i})$
        }
      \end{prooftree}} \\
  \end{tabular}
\end{table}

The $list_{ipm}$ relation builds a set composed of identifiers and/or
couples \textit{(identifier, natural number)} collected from the
identifiers and indexed identifiers found in the formal parts of an
input port map. It also checks, for each association of the input port
map, that the expression of the actual part is of the same type than
the identifier or indexed identifier of the formal part.

\begin{sideconds}
  \begin{itemize}
  \item $\mathrm{id}_f\in{}Ins(\Delta_c)$ checks that the identifier
    $\mathrm{id}_f$ is an input port identifier of $\Delta_c$.
  \item $\mathrm{id}_f\notin\mathcal{L}$ checks that the port
    identifier $\mathrm{id}_f$ is not already mapped, i.e. it is not
    already referenced in the $\mathcal{L}$ set.
  \item $\nexists{}v_i$ s.t. $(\mathrm{id}_f,v_i)\in\mathcal{L}$
    checks that a subelement of id$_f$ is not already map, that is, if
    id$_f$ denotes a signal identifier of the \texttt{array} type.
  \end{itemize}
\end{sideconds}

\begin{table}[H]
  \begin{tabular}{@{}l}
    {\fontsize{8}{11}\selectfont\textsc{ListIPMSimple}} \\
    {\begin{prooftree}

        % Evaluates actual.
        \hypo{\Delta,\sigma\vdash\mathrm{e}~\xrightarrow{e}v}

        % Well-typed actual.
        \hypo{v\in_c{}T}
        
        % Conclusion
        \infer[template={\inserttext}]
        2
        [{\renewcommand{\arraystretch}{1.5}
          \begin{tabular}{l}
            $\mathrm{id}_f\notin\mathcal{L}$, $\mathrm{id}_f\in{}Ins(\Delta_c)$\\
            $\nexists{}v_i$ s.t. $(\mathrm{id}_f,v_i)\in\mathcal{L}$\\
            $\Delta_c(\mathrm{id}_f)=T$ \\
          \end{tabular}
        }]
        {
          $\Delta,\Delta_c,\sigma,\mathcal{L}\vdash
          ~(\mathrm{id}_f,\mathrm{e})~$
          $\xrightarrow{list_{ipm}}\mathcal{L}\cup{}\{id_f\}$
        }
      \end{prooftree}} \\
  \end{tabular}
\end{table}

\begin{premises}
  $v_i\in_c{}nat(n,m)$ checks that the index value stays in the
  array bounds.
\end{premises}

\begin{sideconds}
  $\mathrm{id}_f\notin\mathcal{L}$ and
  $(\mathrm{id}_f, v_i)\notin\mathcal{L}$ checks that neither the port
  identifier $id_f$ nor the couple port identifier $id_f$ and index
  $v_i$ are already mapped.
\end{sideconds}

\begin{figure}[H]
  {\fontsize{8}{11}\selectfont\textsc{ListIPMPartial}}
  \vspace{1em}

  \begin{prooftree}

    % Static expr e_i.
    \hypo{SE_l(\mathrm{e}_i)}

    % Evaluates e_i and e.
    \hypo{&\mathrm{e}_i\xrightarrow{e}v_i}
    \infer[no rule]1{\Delta,\sigma&\vdash\mathrm{e}\xrightarrow{e}v}
    
    % Evaluates e.
    \hypo{v_i&\in_c{}nat(n,m)}
    \infer[no rule]1{v&\in_c{}T}
    
    
    % Conclusion
    \infer[template={\inserttext}]
    3
    [{\renewcommand{\arraystretch}{1.5}
      \begin{tabular}{l}
        $\mathrm{id}_f\notin\mathcal{L}$, $(\mathrm{id}_f, v_i)\notin\mathcal{L}$ \\
        $\mathrm{id}_f\in{}Ins(\Delta_c)$ \\
        $\Delta_c(\mathrm{id}_f)=array(T,n,m)$ \\
      \end{tabular}
    }]
    {
      $\Delta,\Delta_c,\sigma,\mathcal{L}\vdash
      ~(\mathrm{id}_f(\mathrm{e}_i),\mathrm{e})~$
      $\xrightarrow{list_{ipm}}
      \mathcal{L}\cup\{~(id_f,v_i)~\}$
    }
  \end{prooftree}
\end{figure}

\begin{figure}[H]
  {\fontsize{8}{11}\selectfont\textsc{ListIPMCons}}
  \vspace{1em}

  \begin{prooftree}

    % Lists assoc.
    \hypo{\Delta,\Delta_c,\sigma,\mathcal{L}\vdash\mathrm{assoc}_{ip}\xrightarrow{list_{ipm}}\mathcal{L}'}

    % Lists lassoc.
    \hypo{\Delta,\Delta_c,\sigma,\mathcal{L}'\vdash\mathrm{i}\xrightarrow{list_{ipm}}\mathcal{L}''}
    
    % Conclusion
    \infer[template={\inserttext}] 2 {
      $\Delta,\Delta_c,\sigma,\mathcal{L}\vdash
      ~\mathrm{assoc}_{ip}\mathtt{,}~\mathrm{i}~$
      $\xrightarrow{list_{ipm}}\mathcal{L}''$ }
  \end{prooftree}
\end{figure}

The $\mathtt{check}_{pm}(Ports,\mathcal{L})$ predicate states that all
port identifiers referenced in the domain of
$Ports\in{}id\nrightarrow{}type$ appear in $\mathcal{L}$ as a simple
identifier, or if the port identifier is of the $array$ type, then all
couples ($id$,$i$) must belong to $\mathcal{L}$, where $i$ denotes all
indexes of the index range and $id$ denotes the port identifier.

\begin{equation*}
  \begin{split}
    \mathtt{check}_{pm}(Ports,\mathcal{L})\equiv
    \forall{}\mathrm{id}_f\in\mathtt{dom}(Ports),~\mathrm{id}_f\in\mathcal{L}~\lor~&
    (Ports(\mathrm{id}_f)=\mathtt{array}(T,n,m)\land\\
    & \forall{}i\in[n,m],~(\mathrm{id}_f,i)\in\mathcal{L}) \\
  \end{split}
\end{equation*}

\paragraph{Valid output port map}

The $\mathtt{valid}_{opm}$ predicate states that an \emph{output} port
map is valid in the context $\Delta,\Delta_c$, where $\Delta$ is the
embedding design structure and $\Delta_c$ denotes the component
instance owner of the port map, if:

\begin{itemize}[label=-]
\item An output port identifier appears at most once in the output
  port map.
\item Two different output port identifiers cannot be connected to the
  same signal.
\item For each output port map entry, the formal and the actual part
  are of the exact same type.
\end{itemize}

We allow partially connected output port map; i.e, an output port map
where all output ports might not be present in the mapping. Such
output ports are \texttt{open} by default.

\begin{premises}
  $list_{opm}$ builds two sets
  $\mathcal{L},\mathcal{L}_{ids}\subseteq{}id\sqcup{}(id\times\mathbb{N})$
  out of the output port map $\textrm{opmap}$. $\mathcal{L}_{ids}$ is
  built incrementally to check that there are no multiply-driven
  signals resulting of the port map connection.
\end{premises}

\begin{table}[H]
  \centering
  \begin{tabular}{l}
    {\fontsize{8}{11}\selectfont\textsc{ValidOPM}} \\
    {\begin{prooftree}

        % Lists formal parts.
        \hypo{
          \Delta,\Delta_c,\mathcal{L}_\emptyset,\mathcal{L}_{ids\emptyset}\vdash
          \mathrm{o}
          \xrightarrow{list_{opm}}
          \mathcal{L},\mathcal{L}_{ids}
        }
        
        % Conclusion
        \infer[template={\inserttext}]
        1
        {
          $\Delta,\Delta_c\vdash\mathtt{valid}_{opm}(\mathrm{o})$
        }
      \end{prooftree}} \\
  \end{tabular}
\end{table}

\begin{sideconds}
  \begin{itemize}[label=-]
  \item $\mathrm{id}_f\notin\mathcal{L}$ checks that the port
    identifier $\mathrm{id}_f$ is not already mapped (i.e, is not
    already used in the formal part of a port map entry).
  \item $\mathrm{id}_a\notin\mathcal{L}_{ids}$ checks that the signal
    identifier $\mathrm{id}_a$ is not already mapped (i.e, is not
    already used in the actual part of a port map entry).
  \item $\mathrm{id}_f\in{}Outs(\Delta_c)$ checks that $id_f$ is an
    output port identifier of $\Delta_c$.
  \item $\mathrm{id}_a\in{}Sigs(\Delta)\cup{}Outs(\Delta)$ checks that
    $\mathrm{id}_a$ is either an output port or an internal signal
    identifier of $\Delta$.
  \item $\Delta_c(\mathrm{id}_f)=\Delta(\mathrm{id}_a)=T$ checks that
    $\mathrm{id}_f$ and $\mathrm{id}_a$ are exactly of the same type.
  \end{itemize}
\end{sideconds}

\begin{table}[H]
  \begin{tabular}{@{}l}
    {\fontsize{8}{11}\selectfont\textsc{ListOPMSimpleToSimple}} \\
    {\begin{prooftree}
        
        % Conclusion
        \infer[template={\inserttext}]
        0
        [{\renewcommand{\arraystretch}{1.5}
          \begin{tabular}{@{}l}
            $\mathrm{id}_f\notin\mathcal{L}$, $\mathrm{id}_a\notin\mathcal{L}_{ids}$ \\
            $\mathrm{id}_f\in{}Outs(\Delta_c)$\\
            $\mathrm{id}_a\in{}Sigs(\Delta)\cup{}Outs(\Delta)$ \\
            $\Delta_c(\mathrm{id}_f)=\Delta(\mathrm{id}_a)=T$ \\
          \end{tabular}
        }] {
          $\Delta,\Delta_c,\mathcal{L},\mathcal{L}_{ids}\vdash$
          $(\mathrm{id}_f,\mathrm{id}_a)$
          $\xrightarrow{list_{opm}}$
          $\mathcal{L}\cup{}\{id_f\},\mathcal{L}_{ids}\cup{}\{id_a\}$ }
      \end{prooftree}} \\
  \end{tabular}
\end{table}

\begin{sideconds}
  $Outs_c(\mathrm{id}_f)=T$ and
  $Sigs(\mathrm{id}_a)=\mathtt{array(}T,n,m\mathtt{)}$ checks that the
  type of $\mathrm{id}_f$ and the type of the elements of
  $\mathrm{id}_a$ are the same. Note that $\mathrm{id}_a$ be an signal
  identifier of the \texttt{array} type as $\mathrm{id}_f$ is mapped
  to one subelement of $\mathrm{id}_a$.
\end{sideconds}

\begin{table}[H]
  \begin{tabular}{@{}l}
    {\fontsize{8}{11}\selectfont\textsc{ListOPMSimpleToPartial}} \\
    {\begin{prooftree}

        % Static expr e_i and natural.
        \hypo{SE_l(\mathrm{e}_i)}

        % Evaluates e_i.
        \hypo{\mathrm{e}_i\xrightarrow{e}v_i}

        % Well-typed v_i (in array bounds).
        \hypo{v_i\in_c{}nat(n,m)}
        
        % Conclusion
        \infer[template={\inserttext}]    
        3
        [{\renewcommand{\arraystretch}{1.5}
          \begin{tabular}{@{}l}
            $\mathrm{id}_f\notin\mathcal{L}$, $\mathrm{id}_a$, $(\mathrm{id}_a, v_i)\notin\mathcal{L}_{ids}$ \\
            $\mathrm{id}_f\in{}Outs(\Delta_c)$ \\
            $\mathrm{id}_a\in{}Sigs(\Delta)\cup{}Outs(\Delta)$ \\
            $\Delta_c(\mathrm{id}_f)=T$ \\
            $\Delta(\mathrm{id}_a)=array(T,n,m)$ \\
          \end{tabular}
        }]
        {
          $\Delta,\Delta_c,\mathcal{L},\mathcal{L}_{ids}\vdash
          ~(\mathrm{id}_f,\mathrm{id}_a(\mathrm{e}_i))~$
          $\xrightarrow{list_{opm}}
          \mathcal{L}\cup{}\{id_f\},\mathcal{L}_{ids}\cup{}\{(id_a,v_i)\}$
        }
      \end{prooftree}} \\
  \end{tabular}
\end{table}

\begin{table}[H]
  \begin{tabular}{@{}l}
    {\fontsize{8}{11}\selectfont\textsc{ListOPMSimpleToOpen}} \\
    {\begin{prooftree}

        % Conclusion
        \infer[template={\inserttext}]
        0
        [{
          \begin{tabular}{l}
            $\mathrm{id}_f\notin\mathcal{L}$ \\
            $\mathrm{id}_f\in{}Outs(\Delta_c)$ \\
          \end{tabular}
        }]
        {
          $\Delta,\Delta_c,\mathcal{L},\mathcal{L}_{ids}\vdash
          ~(\mathrm{id}_f,\mathtt{open})~$
          $\xrightarrow{list_{opm}}\mathcal{L}\cup{}\{id_f\},\mathcal{L}_{ids}$
        }
      \end{prooftree}} \\
  \end{tabular}
\end{table}

\begin{remark}[Unconnected output port.]
  We forbid the case where an indexed formal part corresponding to the
  subelement of a composite output port is unconnected, i.e
  $(\mathrm{id}_f(e_i),\mathtt{open})$, as it could lead to the case
  where some subelements of a composite output port are connected
  while others are not (error case in \cite[p.7]{VHDL2000}).
\end{remark}

\begin{table}[H]
  \begin{tabular}{@{}l}
    {\fontsize{8}{11}\selectfont\textsc{ListOPMPartialToSimple}} \\
    {\begin{prooftree}

        % Static expr e_i and natural.
        \hypo{SE_l(\mathrm{e}_i)}

        % Evaluates e_i.
        \hypo{\mathrm{e}_i\xrightarrow{e}v_i}

        % Well-typed v_i (in array bounds).
        \hypo{v_i\in_c{}nat(n,m)}
        
        % Conclusion
        \infer[template={\inserttext}]
        3
        [{\renewcommand{\arraystretch}{1.5}
          \begin{tabular}{l}
            $\mathrm{id}_f,(\mathrm{id}_f,v_i)\notin\mathcal{L}$, $\mathrm{id}_a\notin\mathcal{L}_{ids}$ \\
            $\mathrm{id}_f\in{}Outs(\Delta_c)$ \\
            $\mathrm{id}_a\in{}Sigs(\Delta)\cup{}Outs(\Delta)$ \\
            $\Delta_c(\mathrm{id}_f)=array(T,n,m)$ \\
            $\Delta(\mathrm{id}_a)=T$ \\
          \end{tabular}
        }]
        {
          $\Delta,\Delta_c,\mathcal{L},\mathcal{L}_{ids}\vdash$
          $(\mathrm{id}_f(\mathrm{e}_i),\mathrm{id}_a)$
          $\xrightarrow{list_{opm}}$
          $\mathcal{L}\cup\{(id_f,v_i)\},\mathcal{L}_{ids}\cup\{id_a\}$
        }
      \end{prooftree}} \\
  \end{tabular}
\end{table}

\begin{table}[H]
  \begin{tabular}{@{}l}
    {\fontsize{8}{11}\selectfont\textsc{ListOPMPartialToPartial}} \\
    {\begin{prooftree}

        % Static expr. ei e'i.
        
        \hypo{&SE_l(\mathrm{e'}_i)}
        \infer[no rule]1{&SE_l(\mathrm{e}_i)}

        % Evaluates e_i.
        \hypo{&\mathrm{e'}_i\xrightarrow{e}v'_i}
        \infer[no rule]1{&\mathrm{e}_i\xrightarrow{e}v_i}

        % Well-typed v'_i and v_i.
        \hypo{v'_i&\in_c{}nat(n',m')}
        \infer[no rule]1{v_i&\in_c{}nat(n,m)}
        
        % Conclusion
        \infer[template={\inserttext}]
        3
        [{\renewcommand{\arraystretch}{1.5}
          \begin{tabular}{l}
            $\mathrm{id}_f,(\mathrm{id}_f,v_i)\notin\mathcal{L}$, $\mathrm{id}_a,(\mathrm{id}_a,v'_i)\notin\mathcal{L}_{ids}$ \\
            $\mathrm{id}_f\in{}Outs(\Delta_c)$ \\
            $\mathrm{id}_a\in{}Sigs(\Delta)\cup{}Outs(\Delta)$ \\
            $\Delta_c(\mathrm{id}_f)=array(T,n,m)$ \\
            $\Delta(\mathrm{id}_a)=array(T,n',m')$ \\ 
          \end{tabular}
        }]
        {
          $\Delta,\Delta_c,\mathcal{L},\mathcal{L}_{ids}\vdash$
          $(\mathrm{id}_f(\mathrm{e}_i),\mathrm{id}_a(\mathrm{e'}_i))$
          $\xrightarrow{list_{opm}}$
          \begin{tabular}{@{}l@{}}
            $\mathcal{L}\cup\{(id_f,v_i)\},$ \\
            $\mathcal{L}_{ids}\cup\{(id_a,v'_i)\}$\\
          \end{tabular}
        }
      \end{prooftree}} \\
  \end{tabular}
\end{table}

\begin{table}[H]
  \begin{tabular}{@{}l}
    {\fontsize{8}{11}\selectfont\textsc{ListOPMCons}} \\
    {\begin{prooftree}

        % Lists assoc.
        \hypo{
          \Delta,\Delta_c,\mathcal{L},\mathcal{L}_{ids}\vdash
          \mathrm{assoc}_{po}
          \xrightarrow{list_{opm}}
          \mathcal{L}',\mathcal{L}'_{ids}
        }

        % Lists lassoc.
        \hypo{
          \Delta,\Delta_c,\mathcal{L}',\mathcal{L}'_{ids}\vdash
          \mathrm{opmap}
          \xrightarrow{list_{opm}}
          \mathcal{L}'',\mathcal{L}''_{ids}
        }
        
        % Conclusion
        \infer[template={\inserttext}]
        2
        {
          $\Delta,\Delta_c,\mathcal{L},\mathcal{L}_{ids}\vdash
          ~\mathrm{assoc}_{po}\mathtt{,}~\mathrm{opmap}~$
          $\xrightarrow{list_{opm}}$
          $\mathcal{L}'',\mathcal{L}''_{ids}$
        }
      \end{prooftree}} \\
  \end{tabular}
\end{table}

\subsection{Valid sequential statements}
\label{subsubsec:valid-ss}

The $\mathtt{valid}_{ss}$ predicate states that a sequential statement
is well-typed in the context $\Delta,\sigma,\Lambda$.

\subsubsection{Well-typed signal assignment}

% Well-typed signal assignment.

\begin{premises}
  \begin{itemize}
  \item $\Delta,\sigma,\Lambda\vdash{}\mathrm{e}\xrightarrow{e}v$
    evaluates the expression assigned to signal $\mathrm{id}_s$ in the
    context $\Delta,\sigma,\Lambda$.% During the elaboration, $\sigma$
    % corresponds to the default design state, i.e. where each signal is
    % associated to its type default value.
  \item $v\in_c{}T$ checks that the value of expression $e$ conforms
    to the type of signal $\mathrm{id}_s$.
  \end{itemize}
\end{premises}

\begin{table}[H]
  \begin{tabular}{@{}l}
    {\fontsize{8}{11}\selectfont\textsc{WTSig}} \\
    {\begin{prooftree}

        % Evaluates e.
        \hypo{\Delta,\sigma,\Lambda\vdash{}\mathrm{e}\xrightarrow{e}v}
        
        % Well-typed v.
        \hypo{v\in_c{}T}
        
        % Conclusion
        \infer[template={\inserttext}]2
        [{
          \renewcommand{\arraystretch}{1.5}
          \begin{tabular}{@{}l}
            $\mathrm{id}_s\in{}Sigs(\Delta)\cup{}Outs(\Delta)$\\
            $\Delta(\mathrm{id}_s)=T$\\
          \end{tabular}
        }]
        {
          $\Delta,\sigma,\Lambda\vdash\mathtt{valid}_{ss}(\mathrm{id}_s\Leftarrow\mathrm{e})$
        }
      \end{prooftree}} \\
  \end{tabular}
\end{table}

% Well-typed indexed signal assignment.
\begin{table}[H]
  \begin{tabular}{l}
    {\fontsize{8}{11}\selectfont\textsc{WTIdxSig}} \\
    {\begin{prooftree}

        % Evaluates e and e_i.
        \hypo{\Delta,\sigma,\Lambda&\vdash\mathrm{e}\xrightarrow{e}v}
        \infer[no rule]1{\Delta,\sigma,\Lambda&\vdash\mathrm{e_i}\xrightarrow{e}v_i}
        
        % Well-typed v and v_i.
        \hypo{v&\in_c{}T}
        \infer[no rule]1{v_i&\in_c{}nat(n,m)}
        
        % Conclusion
        \infer[template={\inserttext}]2
        [{
          \renewcommand{\arraystretch}{1.5}
          \begin{tabular}{@{}l}
            $\mathrm{id}_s\in{}Sigs(\Delta)\cup{}Outs(\Delta)$ \\
            $\Delta(\mathrm{id}_s)=array(T,n,m)$ \\
          \end{tabular}
        }]
        {
          $\Delta,\sigma,\Lambda\vdash\mathtt{valid}_{ss}(\mathrm{id}_s(\mathrm{e}_i)\Leftarrow\mathrm{e})$
        }
      \end{prooftree}} \\
  \end{tabular}
\end{table}

\subsubsection{Well-typed variable assignment}

\begin{table}[H]
  \begin{tabular}{@{}l}
    {\fontsize{8}{11}\selectfont\textsc{WTVar}} \\
    {\begin{prooftree}
        
        % Evaluates e.
        \hypo{\Delta,\sigma,\Lambda\vdash\mathrm{e}\xrightarrow{e}v}

        % Well-typed v.
        \hypo{v\in_c{}T}
        
        % Conclusion
        \infer[template={\inserttext}]2
        [{
          \renewcommand{\arraystretch}{1.5}
          \begin{tabular}{@{}l}
            $\mathrm{id}_v\in\Lambda$\\
            $\Lambda(\mathrm{id}_v)=(T,val)$\\
          \end{tabular}
        }]
        {
          $\Delta,\sigma,\Lambda\vdash\mathtt{valid}_{ss}(\mathrm{id}_v := \mathrm{e})$
        }
      \end{prooftree}} \\
  \end{tabular}
\end{table}

\begin{table}[H]    
  \begin{tabular}{@{}l}
    {\fontsize{8}{11}\selectfont\textsc{WTIdxVar}} \\
    {\begin{prooftree}

        % Evaluates e and e_i.
        \hypo{\Delta,\sigma,\Lambda&\vdash\mathrm{e}\xrightarrow{e}v}
        \infer[no rule]1{\Delta,\sigma,\Lambda&\vdash\mathrm{e_i}\xrightarrow{e}v_i}
        
        % Well-typed v and v_i.
        \hypo{v&\in_c{}T}
        \infer[no rule]1{v_i&\in_c{}nat(n,m)}
        
        % Conclusion
        \infer[template={\inserttext}]2
        [{
          \renewcommand{\arraystretch}{1.5}
          \begin{tabular}{@{}l}
            $\mathrm{id}_v\in{}\Lambda$ \\
            $\Lambda(\mathrm{id}_v)=(array(T,n,m),val)$ \\
          \end{tabular}
        }]
        {
          $\Delta,\Lambda\vdash\mathtt{valid}_{ss}(\mathrm{id}_v(\mathrm{e}_i) := \mathrm{e})$
        }
      \end{prooftree}} \\
  \end{tabular}
\end{table}%

\subsubsection{Well-typed if statements}

\begin{table}[H]
  \begin{tabular}{@{}l}
    {\fontsize{8}{11}\selectfont\textsc{WTIf}} \\
    {\begin{prooftree}

        % Eval e.
        \hypo{\Delta,\sigma,\Lambda\vdash\mathrm{e}\xrightarrow{e}v}
        
        % Well-typed v.
        \hypo{v\in_c\mathtt{bool}}

        % Validss ss.
        \hypo{\Delta,\sigma,\Lambda\vdash\mathtt{valid}_{ss}(\mathrm{ss})}
        
        % Conclusion
        \infer[template={\inserttext}]3
        {
          $\Delta,\sigma,\Lambda\vdash\mathtt{valid}_{ss}($\vhdle|if| \texttt{(}e\texttt{)} ss$)$
        }
      \end{prooftree}} \\
  \end{tabular}
\end{table}%

\begin{table}[H]
  \begin{tabular}{@{}l}
    {\fontsize{8}{11}\selectfont\textsc{WTIfElse}} \\
    {\begin{prooftree}

        % Eval e.
        \hypo{\Delta,\sigma,\Lambda\vdash\mathrm{e}\xrightarrow{e}v}
        
        % Well-typed v.
        \hypo{v\in_c\mathtt{bool}}

        % Validss ss and ss'.
        \hypo{\Delta,\sigma,\Lambda\vdash\mathtt{valid}_{ss}(\mathrm{ss})}
        \infer[no rule]1{\Delta,\sigma,\Lambda\vdash\mathtt{valid}_{ss}(\mathrm{ss}')}
        
        % Conclusion
        \infer[template={\inserttext}]3
        {
          $\Delta,\sigma,\Lambda\vdash\mathtt{valid}_{ss}($\vhdle|if| \texttt{(}e\texttt{)} ss ss'$)$
        }
      \end{prooftree}} \\
  \end{tabular}
\end{table}%

\subsubsection{Well-typed loop statement}

\begin{table}[H]
  \begin{tabular}{@{}l}
    {\fontsize{8}{11}\selectfont\textsc{WTLoop}} \\
    {\begin{prooftree}

        % Eval e and e'.
        \hypo{\Delta,\sigma,\Lambda&\vdash\mathrm{e}\xrightarrow{e}v}
        \infer[no rule]1{\Delta,\sigma,\Lambda&\vdash\mathrm{e'}\xrightarrow{e}v'}

        % Well-typed v and v'.
        \hypo{v&\in_c{}{}nat(0,\mathtt{NATMAX})}
        \infer[no rule]1{v'&\in_c{}{}nat(0,\mathtt{NATMAX})}
        
        % Well-typed ss.
        \hypo{\Delta,\sigma,\Lambda'\vdash\mathtt{valid}_{ss}(\mathrm{ss})}
        
        % Conclusion
        \infer[template={\inserttext}]3
        [$\Lambda'=\Lambda\cup(\mathrm{id}_v,({}nat(v,v'), v))$]
        {
          $\Delta,\sigma,\Lambda\vdash\mathtt{valid}_{ss}($\vhdle|for| (id$_v$, e, e') ss$)$
        }
      \end{prooftree}} \\
  \end{tabular}
\end{table}%

\subsubsection{Well-typed rising and falling edge blocks}

\begin{table}[H]
  \begin{tabular}{@{}l}
    {\fontsize{8}{11}\selectfont\textsc{WTRising}} \\
    {\begin{prooftree}
        
        % Well-typed ss.
        \hypo{\Delta,\sigma,\Lambda\vdash\mathtt{valid}_{ss}(\mathrm{ss})}
        
        % Conclusion
        \infer[template={\inserttext}]1
        {
          $\Delta,\sigma,\Lambda\vdash\mathtt{valid}_{ss}($\vhdle|rising| ss$)$
        }
      \end{prooftree}} \\
  \end{tabular}
  \begin{tabular}{@{}l}
    {\fontsize{8}{11}\selectfont\textsc{WTFalling}} \\
    {\begin{prooftree}
        
        % Well-typed ss.
        \hypo{\Delta,\sigma,\Lambda\vdash\mathtt{valid}_{ss}(\mathrm{ss})}
        
        % Conclusion
        \infer[template={\inserttext}]1
        {
          $\Delta,\sigma,\Lambda\vdash\mathtt{valid}_{ss}($\vhdle|falling| ss$)$
        }
      \end{prooftree}} \\
  \end{tabular}
\end{table}%

\subsubsection{Well-typed rst blocks}

\begin{table}[H]
  \begin{tabular}{@{}l}
    {\fontsize{8}{11}\selectfont\textsc{WTRst}} \\
    {\begin{prooftree}
        
        % Well-typed ss.
        \hypo{\Delta,\sigma,\Lambda\vdash\mathtt{valid}_{ss}(\mathrm{ss})}

        % Well-typed ss'.
        \hypo{\Delta,\sigma,\Lambda\vdash\mathtt{valid}_{ss}(\mathrm{ss'})}
        
        % Conclusion
        \infer[template={\inserttext}]2
        {
          $\Delta,\sigma,\Lambda\vdash\mathtt{valid}_{ss}($\vhdle|rst| ss ss'$)$
        }
      \end{prooftree}} \\
  \end{tabular}
\end{table}%

\subsubsection{Well-typed null statement}
\label{sec:wt-null}

\begin{table}[H]
  \begin{tabular}{@{}l}
    {\fontsize{8}{11}\selectfont\textsc{WTNull}} \\
    {\begin{prooftree}
                
        % Conclusion
        \infer[template={\inserttext}]0
        {
          $\Delta,\sigma,\Lambda\vdash\mathtt{valid}_{ss}($\vhdle|null|$)$
        }
      \end{prooftree}} \\
  \end{tabular}
\end{table}


%%% Local Variables:
%%% mode: latex
%%% TeX-master: "../../main"
%%% End:


\section{Simulation rules}
\label{sec:sim-rules}
% \subsection{Well-formed environment.}
% \label{subsec:env-wf}

% To avoid the situation where an identifier met in an expression refers
% both to a variable and a signal at the same time, we need to ensure
% that identifiers referenced in the environment are unique.  We define
% the predicate $\mathcal{U}$ stating that identifier unicity is met at
% different environment levels.

% \begin{figure}[H]
%   {\fontsize{10}{13}\selectfont\textsc{WFTypedefEnv}}
  
%   \begin{prooftree}

%     % No intersection dom and val
%     \hypo{\mathtt{dom}(f)\cap\mathtt{val}(f)=\emptyset}

%     % Unique val.
%     \infer[no rule]1{
%       \forall{}id,id'\in\mathtt{dom}(f),~id\neq{}id'\wedge\mathtt{enum}(f(id))\wedge\mathtt{enum}(f(id'))
%       \Rightarrow{}f(id)\cap{}f(id')=\emptyset
%     }
    
%     % Conclusion
%     \infer1
%     [$f\in{}ident\nrightarrow{}typedef$]
%     {
%       \mathcal{U}(f)
%     }
%   \end{prooftree}
% \end{figure}

% where $\mathtt{enum}$ is a predicate stating that a $typedef$ instance
% is an enumeration of values.

% \begin{figure}[H]
%   {\fontsize{10}{13}\selectfont\textsc{WFDesign}}
  
%   \begin{prooftree}

%     % Well-formed Types.
%     \hypo{\mathcal{U}(Types)}

%     % No intersection val(Types).
%     \infer[no rule]1{
%       \mathtt{val}(Types)\cap\mathtt{dom}(\Delta)=\emptyset
%     }
    
%     % No intersection dom(Types).
%     \infer[no rule]1{
%       \mathtt{dom}(Types)\cap\big((\mathtt{dom}(\Delta)\setminus\mathtt{dom}(Types))\big)=\emptyset
%     }

%     % No intersection Gens.
%     \infer[no rule]1{
%       \mathtt{dom}(Gens)\cap\big((\mathtt{dom}(\Delta)\setminus\mathtt{dom}(Gens))\cup\mathtt{val}(Types)\big)=\emptyset
%     }

%     % No intersection Ins.
%     \infer[no rule]1{
%       \mathtt{dom}(Ins)\cap\big((\mathtt{dom}(\Delta)\setminus\mathtt{dom}(Ins))\cup\mathtt{val}(Types)\big)=\emptyset
%     }

%     % No intersection Outs.
%     \infer[no rule]1{
%       \mathtt{dom}(Outs)\cap\big((\mathtt{dom}(\Delta)\setminus\mathtt{dom}(Outs))\cup\mathtt{val}(Types)\big)=\emptyset
%     }

%     % No intersection Sigs.
%     \infer[no rule]1{
%       \mathtt{dom}(Sigs)\cap\big((\mathtt{dom}(\Delta)\setminus\mathtt{dom}(Sigs))\cup\mathtt{val}(Types)\big)=\emptyset
%     }

%     % No intersection Consts.
%     \infer[no rule]1{
%       \mathtt{dom}(Consts)\cap\big((\mathtt{dom}(\Delta)\setminus\mathtt{dom}(Consts))\cup\mathtt{val}(Types)\big)=\emptyset
%     }
    
%     % No intersection Ps.
%     \infer[no rule]1{
%       \mathtt{dom}(Ps)\cap\big((\mathtt{dom}(\Delta)\setminus\mathtt{dom}(Ps))\cup\mathtt{val}(Types)\big)=\emptyset
%     }

%     % No intersection Comps.
%     \infer[no rule]1{
%       \mathtt{dom}(Comps)\cap\big((\mathtt{dom}(\Delta)\setminus\mathtt{dom}(Comps))\cup\mathtt{val}(Types)\big)=\emptyset
%     }
    
%     % Conclusion
%     \infer1
%     [$\Delta\in{}Design$]
%     {
%       \mathcal{U}(\Delta)
%     }
%   \end{prooftree}
% \end{figure}

% where $\Delta=(Gens,Ins,Outs,Sigs,Consts,Ps,Comps,Types,Behavior)$.

% \begin{figure}[H]
%   {\fontsize{10}{13}\selectfont\textsc{WFGammaDeltaEnv}}
%   \vspace{2ex}
  
%   \begin{prooftree}

%     % Well-formed .
%     \hypo{\mathcal{U}()}

%     % Well-formed \Delta.
%     \hypo{\mathcal{U}(\Delta)}

%     % No intersection  and \Delta.
%     \hypo{(\mathtt{dom}()\cup\mathtt{val}())
%       \cap(\mathtt{dom}(\Delta)\cup\mathtt{val}(Types))=\emptyset}
    
%     % Conclusion
%     \infer3
%     [{
%       \begin{tabular}{@{}l}
%         $\in{}ident\nrightarrow{}typedef$ \\
%         $\Delta\in{}Design$ \\
%       \end{tabular}
%     }]
%     {
%       \mathcal{U}(\Delta)
%     }
%   \end{prooftree}
% \end{figure}

% \begin{figure}[H]
%   {\fontsize{10}{13}\selectfont\textsc{WFGammaDeltaLambdaEnv}}
%   \vspace{2ex}
  
%   \begin{prooftree}

%     % Well-formed  and \Delta.
%     \hypo{\mathcal{U}(\Delta)}

%     % No intersection \Lambda and .
%     \hypo{\mathtt{dom}(\Lambda)\cap(\mathtt{dom}()\cup\mathtt{val}())=\emptyset}
    
%     % No intersection \Lambda and \Delta.
%     \hypo{\mathtt{dom}(\Lambda)\cap(\mathtt{dom}(\Delta)\cup\mathtt{val}(Types))=\emptyset}
    
%     % Conclusion
%     \infer3
%     [{
%       \begin{tabular}{@{}l}
%         $\in{}ident\nrightarrow{}typedef$ \\
%         $\Delta\in{}Design$ \\
%         $\Lambda\in{}ident\nrightarrow{}(type\times{}value)$
%       \end{tabular}
%     }]
%     {
%       \mathcal{U}(\Delta,\Lambda)
%     }
%   \end{prooftree}
% \end{figure}

% To safely pursue the simulation phase for a given top-level design, we
% need to prove Theorem~\ref{thm:id-unicity} stating that if the
% elaboration of the top-level design succeeds then there is no
% overlaping of identifiers between the $$ and $\Delta$ structures
% and every local environment of processes present in the top-level
% design's behavior.

% \begin{theorem}
%   \label{thm:id-unicity}
%   $\forall{}\mathrm{design},\mathcal{M},\Delta,$
%   $\big(\mathcal{D},\mathcal{M}\vdash\mathrm{design}\xrightarrow{elab}\Delta\big)$
%   $\Rightarrow\forall\Lambda,\exists\mathrm{id}\in\mathtt{dom}(Ps)|Ps(\mathrm{id})=\Lambda,~\mathcal{U}(\Delta,\Lambda)$.
% \end{theorem}

% Note that in $$ and $\mathcal{D}$ are free variables in
% Theorem~\ref{thm:id-unicity} as they are known structures.  With
% Theorem~\ref{thm:id-unicity} proved, they can be no ambiguity due to
% identifiers while interpreting expressions during the simulation
% phase.

\subsection{Full Simulation}

The full simulation process is decomposed in two steps. The first step
is the elaboration phase that builds an elaborated version of a
\hvhdl{} design along with its default state, and type-checks the
design. Previous to the elaboration phase, the top-level design
receives a value for each of its generic constant; we refer to it as
the \emph{dimensioning} of the top-level design. The second step is
the simulation phase that executes the behavioral part of the
top-level design starting from an initial state. The simulation is
decomposed into simulation cycles. Each simulation cycle is divided in
four parts entailed by the \emph{synchronous} execution of
$\mathcal{H}$-VHDL top-level designs, i.e designs whose behavior
depend on a clock signal. The four parts are, first, the execution of
concurrent statements responding to the rising edge of the clock
signal, then, a phase of signal stabilization followed by the
execution of concurrent statements responding to the falling edge of
the clock signal, and finally another phase of signal
stabilization. At each clock event, the value of the primary inputs of
the design being currrently simulated are captured and injected in the
simulation; primary inputs receive values from the design
environment. Here, the environment is represented by a function
mapping input port identifiers to values depending on the current
count of simulation cycles and the considered clock event. This leads
to the following hypothesis:

\begin{hypothesis}[Stable primary inputs]
  \label{hyp:stable-pi}
  The values of primary inputs (i.e, input ports of the top-level
  design) are captured at each clock event, and therefore are stable
  (i.e, their values do not change) between two contiguous clock
  events.
\end{hypothesis}

Hypothesis~\nameref{hyp:stable-pi} arises from the fact that the clock
signal sample rate respects the Nyquist-Shannon sampling
theorem. Therefore, the sample rate of the design's clock is
sufficient to capture all events possibly arising in the environment.
We only need to settle the values of the primary inputs at the clock
edges.

Also, after each clock event phase follows a signal stabilization
phase in the proceedings of a simulation cycle. One more hypothesis is
needed here:

\begin{hypothesis}[Stabilization]
  \label{hyp:stabilization}
  All signals have enough time to stabilize during the signal
  stabilization phase that happens between two clock events.
\end{hypothesis}

As a \hvhdl{} design represents a physical circuit, one can assume
that the represented circuit is analyzed former to the simulation.
Therefore, one knows exactly how much time is needed to propagate
signal values through the longest physical path; as a consequence, a
proper clock frequency is set ensuring signal stabilization between
two clock events. Thus, Hypothesis~\nameref{hyp:stabilization} arises
from the previous facts.

%%% DESIGN ELABORATION AND SIMULATION.

The $full$ simulation relation takes in parameter a top-level design
d, a design store $\mathcal{D}\in{}id\nrightarrow{}design$, an
elaborated design $\Delta\in{}ElDesign(d)$, a dimensioning function
$\mathcal{M}_g\in{}Gens(\Delta)\nrightarrow{}value$, a primary input
environment
$E_p\in{}(\mathbb{N}\times{}Clk)\rightarrow{}(Ins(\Delta)\rightarrow{}value)$,
a simulation cycle count $\tau\in\mathbb{N}$, and a simulation trace
$\theta\in{}\mathtt{list}(\Sigma(\Delta))$, corresponding to the list
of states yielded by design d after $\tau$ simulation cycles. Note
that we use the pointed notation to access the behavioral part of
design d, written d.cs. It is this part of the design that is executed
during the simulation, and therefore is passed as a parameter of the
initialization and simulation relations. % The states in
% $\sigma_t$ are \emph{time}-ordered, that is, the first states of the
% list are the earliest states in the simulation history. To achieve
% conciseness, further on a singleton trace containing the only element
% $\sigma$ is written $\sigma$. The trace resulting of the concatenation
% of two traces $\sigma_t$ and $\sigma_t'$ is written
% $\sigma_t.\sigma_t'$.

\begin{table}[H]
  {\fontsize{10}{13}\selectfont\textsc{FullSim}}
  
  \begin{prooftree}[template=\inserttext]

    % Design elab.
    \hypo{$\mathcal{D},\mathcal{M}_g\vdash\mathrm{d}\xrightarrow{elab}\Delta,\sigma$}

    % Initialization.
    \hypo{$\mathcal{D},\Delta,\sigma\vdash{}\mathrm{d.cs}\xrightarrow{init}\sigma_0$}
       
    % Simulation loop.
    \hypo{$\mathcal{D},E_p,\Delta,\tau,\sigma_0\vdash{}\mathrm{d.cs}\rightarrow\theta$}
    
    \infer3 [] { $\mathcal{D},\Delta,\mathcal{M}_g,E_p,\tau\vdash$
      $\mathrm{d}\xrightarrow{full}(\sigma_0::\theta)$ }
  \end{prooftree}
\end{table}

where:

\begin{itemize}[label=-]
\item $\mathcal{M}_g\in{}Gens(\Delta)\nrightarrow{}value$, the
  function yelding the values of generic constants for a given
  top-level design, refered to as the \emph{dimensioning} function.
\item
  $E_p\in{}(\mathbb{N}\times{}Clk)\rightarrow{}(ident\nrightarrow{}value)$,
  the function yelding a mapping from primary inputs (i.e, input ports
  of the top-level design) to values at a given simulation cycle count
  (i.e, the $\mathbb{N}$ argument), and a given clock event (i.e, the
  $Clk$ argument, where $Clk=\{\uparrow,\downarrow\}$).
\item $\tau$, the number of simulation cycles to execute.  The value
  of $\tau$ is decremented at each clock cycle until it reaches zero
  (see Section~\ref{sec:sim-loop}).
\end{itemize}

\subsection{Simulation loop.}
\label{sec:sim-loop}

\begin{table}[H]

  \begin{tabular}{@{}l}
    %%% SIMULATION END.
    
    {\fontsize{10}{13}\selectfont\textsc{SimEnd}} \\
    
    {\begin{prooftree}[template=\inserttext]
        \infer0 {
          $\mathcal{D},E_p,\Delta,0,\sigma\vdash{}cs\rightarrow{}[~]$
        }
      \end{prooftree}} \\
  \end{tabular}
  \begin{tabular}{l}
    % SIMULATION LOOP
    
    {\fontsize{10}{13}\selectfont\textsc{SimLoop}} \\
    
    {\begin{prooftree}[template=\inserttext]
        
        % First column.
        \hypo{$\mathcal{D},E_p,\Delta,\tau,\sigma\vdash{}cs\xrightarrow{\uparrow,\downarrow}\sigma',\sigma''$}

        % Second column.
        \hypo{$\mathcal{D},E_p,\Delta,\tau-1,\sigma''\vdash{}cs\rightarrow\theta$}
        
        \infer2 [$\tau>0$] {
          $\mathcal{D},E_p,\Delta,\tau,\sigma\vdash{}cs\rightarrow(\sigma'
          :: \sigma'' :: \theta)$ }
      \end{prooftree}} \\
  \end{tabular}

\end{table}

\subsection{Simulation cycle.}

 To ease the reading of
forward simulation rules, we need to introduce two notations.

\begin{notation}[Overriding union]
  For all partial function $f,f'\in{}X\nrightarrow{}Y$, $f\ocup{}f'$
  denotes the overriding union of $f$ and $f'$ such that
  $f\ocup{}f'(x)=
  \begin{cases}
    f'(x) & if~x\in\mathtt{dom}(f') \\
    f(x) & otherwise \\
  \end{cases}
  $
\end{notation}

\begin{notation}[Differentiated intersection domain]
  For all partial function $f,f'\in{}X\nrightarrow{}Y$, $f\dcap{}f'$
  denotes the intersection of the domain of $f$ and $f'$ for which $f$
  and $f'$ yields different values. That is,
  $f\dcap{}f'=\{~x\in\mathtt{dom}(f)\cap\mathtt{dom}(f')~|~f(x)\neq{}f'(x)~\}$.
\end{notation}

% \begin{remark}[Initialization and signal events.]
%   In the definition of the $\dcap$ relation, adding the difference
%   between the domain of $f'$ and $f$ is mandatory, as it permits to
%   take into account elements that have a fresh value in $f'$.  This
%   helps to catch events on signals that happened to have no values in
%   a signal store $\mathcal{S}$ but have a fresh (initial) value in a
%   resulting signal store $\mathcal{S}'$.
% \end{remark}

\begin{definition}[Input port values update]
  Given an \hvhdl{} design $d\in{}design$, a design store
  $\mathcal{D}\in{}id\rightarrow{}design$, an elaborated design
  $\Delta\in{}ElDesign(d,\mathcal{D})$, a simulation environment
  $E_p\in{}(\mathbb{N}\times\{\uparrow,\downarrow\})\rightarrow(Ins(\Delta)\rightarrow{}value)$,
  let us define the relation expressing the update of the values of
  the input ports of $\Delta$ at a given design state
  $\sigma\in\Sigma(\Delta)$, clock cycle count $\tau\in{}\mathbb{N}$,
  and clock event $clk\in\{\uparrow,\downarrow\}$, and thus resulting
  in a new state $\sigma_i\in\Sigma(\Delta)$. The relation is written
  $\mathtt{Inject}_{clk}(\sigma,E_p,\tau,\sigma_i)$ and verifies that:
  $\sigma={<}\mathcal{S},\mathcal{C},\mathcal{E}{>}$ and
  $\sigma_i={<}\mathcal{S}\ocup{}E_p(\tau,clk),\mathcal{C},\mathcal{E}{>}$.
\end{definition}

The cycle relation states that the design state $\sigma''$ is the
result of the execution of concurrent statement $cs$ over one
simulation cycle, this starting from state $\sigma$.  As told in
Hypothesis~\ref{hyp:stable-pi}, we update the input port values at
each clock event. New input port values are coming from the
environment $E_p$. The updates are made in the definitions of
$\sigma_i$ and $\sigma'_i$. These definitions are expressed as side
conditions.

\begin{table}[H]
  {\fontsize{10}{13}\selectfont\textsc{SimCyc}}
  
  \begin{prooftree}[template=\inserttext]

    % Rising
    \hypo{$\mathcal{D},\Delta,\sigma_i\vdash\mathrm{cs}\xrightarrow{\uparrow}\sigma_\uparrow$}
    
    % Falling
    \infer[no rule]1{$\mathcal{D},\Delta,\sigma'_i\vdash\mathrm{cs}\xrightarrow{\downarrow}\sigma_\downarrow$}

    % Stabilize after rising.
    \hypo{$\mathcal{D},\Delta,\sigma_\uparrow\vdash\mathrm{cs}\xrightarrow{\rightsquigarrow}\sigma'$}

    % Stabilize after falling.
    \infer[no rule]1{$\mathcal{D},\Delta,\sigma_\downarrow\vdash\mathrm{cs}\xrightarrow{\rightsquigarrow}\sigma''$}
    
    \infer2
    [{
      \renewcommand{\arraystretch}{1.5}
      \begin{tabular}{@{}l}
        $\mathtt{Inject}_\uparrow(\sigma,E_p,\tau,\sigma_i)$\\
        $\mathtt{Inject}_\downarrow(\sigma',E_p,\tau,\sigma'_i)$\\
      \end{tabular}
    }] {
      $\mathcal{D},E_p,\Delta,\tau,\sigma\vdash\mathrm{cs}\xrightarrow{\uparrow,\downarrow}$
      $\sigma',\sigma''$ }
  \end{prooftree}
\end{table}

\begin{remark}[Input ports and signal store]
  For a given $\Delta\in{}Design$, $\sigma\in\Sigma(\Delta)$,
  $E_p\in{}\mathbb{N}\rightarrow{}Clk\rightarrow{}(Ins(\Delta)\rightarrow{}value)\mathbb{}$,
  $\tau\in\mathbb{N}$, $clk\in{}Clk$, we have
  $\mathtt{dom}(E_p(\tau,clk))\subseteq{}\mathtt{dom}(\mathcal{S})$,
  where $\sigma={<}\mathcal{S}, \mathcal{C}, \mathcal{E}{>}$. Indeed,
  the input ports of $\Delta$ that constitutes the domain of
  $E_p(\tau,clk)$ are a subset of the set of signals. The set of
  signals constitutes the domain of the signal store of $\sigma$ (i.e,
  $\mathcal{S}$); thus we have
  $\mathtt{dom}(E_p(\tau,clk))\subseteq{}\mathtt{dom}(\mathcal{S})$.
\end{remark}

\subsection{Initialization rules}

\begin{table}[H]
  {\fontsize{10}{13}\selectfont\textsc{Init}}
  
  \begin{prooftree}[template=\inserttext]

    % Run all processes once.
    \hypo{$\mathcal{D},\Delta,\sigma\vdash\mathrm{cs}\xrightarrow{runinit}{}\sigma'$}

    % Stabilization phase after runinit.
    \hypo{$\mathcal{D},\Delta,\sigma'\vdash\mathrm{cs}\xrightarrow{\rightsquigarrow}{}\sigma''$}

    % Conclusion.
    \infer2
    {
      $\mathcal{D},\Delta,\sigma\vdash\mathrm{cs}\xrightarrow{init}\sigma''$
    }

  \end{prooftree}
\end{table}

At the initialization phase, the block of sequential instructions of
all processes is executed exactly once ($runinit$), then a
stabilization phase follows ($stabilize$). It is during the
initialization phase that the first part of $\mathtt{rst}$ blocks is
executed. A block (\vhdle|rst| ss ss') is equivalent to (\vhdle|if|
(rst = false) \vhdle|then| ss \vhdle|else| ss') where \texttt{rst} is
a reserved signal identifier. Therefore, when considering a
\texttt{rst} block, the $runinit$ relation executes the ss block; at
every other moment of the simulation, the ss' block is executed. This
mimicks the conventional proceeding of a simulation where the
\texttt{rst} signal (for \textit{reset} signal) is set to false during
the initialization (only during the $runinit$ phase, not during the
stabilization phase), and then is set to true for the rest of the
simulation.

\subsubsection{Evaluation of a process statement}
\label{subsubsec:ps-stmt-init}

\begin{premises}
  \begin{itemize}
  \item The $i$ flag of the $ss_i$ relation indicates that all
    sequential instructions responding to the initialization phase
    (i.e, \texttt{rst} blocks) will be executed.
  \item The set of events of state $\sigma$ is emptied
    ($NoEv(\sigma)$) before the evaluation of the process statement
    body, and the resulting state is the starting state that will be
    written through the execution of the process statement body.
\end{itemize}
\end{premises}

\begin{table}[H]
  \centering
  \begin{tabular}{@{}l}
    {\fontsize{10}{13}\selectfont\textsc{PsRunInit}} \\
    {\begin{prooftree}[template=\inserttext]
        
        % Runs seq.
        \hypo{$\Delta,\sigma,NoEv(\sigma),\Lambda\vdash\mathrm{ss}\xrightarrow{ss_i}{}\sigma',\Lambda'$}

        % Conlcusion.
        \infer1
        [{
          \begin{tabular}{@{}l}
            $\Delta(\mathrm{id}_p)=\Lambda$\\
          \end{tabular}
        }]
        {
          $\mathcal{D},\Delta,\sigma\vdash$
          \vhdle|process| \texttt{(}id$_p$\texttt{,} sl\texttt{,} vars\texttt{,} ss\texttt{)}
          $\xrightarrow{runinit}\sigma'$
        }
      \end{prooftree}} \\
  \end{tabular}
\end{table}

\subsubsection{Evaluation of a component instantiation statement}
\label{subsubsec:ps-stmt-init}

\begin{sideconds}
  If $\sigma_c''$ has a non-empty set of events, then the state of
  component $\mathrm{id}_c$ must be overidden in the target state of
  the $runinit$ relation. Moreover, $\mathrm{id}_c$ must be added to
  the set of events of the embedding design. This is expressed by the
  side condition:
  $\sigma''=(\sigma'(\mathrm{id}_c)\leftarrow\sigma_c'')\cup_\mathcal{E}\{\mathrm{id}_c\}$
\end{sideconds}

\begin{table}[H]
  \centering
  \begin{tabular}{@{}l}
    {\fontsize{10}{13}\selectfont\textsc{CompRunInit}} \\
    {\begin{prooftree}
        
        % Builds mapping for in ports.
        \hypo{\Delta,\Delta_c,\sigma,\sigma_c\vdash\mathrm{i}\xrightarrow{mapip}\sigma'_c}
        
        % Executes runinit on component behavior.
        \infer[no rule]1{\mathcal{D},\Delta_c,\sigma_c'\vdash{}\mathcal{D}(\mathrm{id_e}).\mathrm{cs}\xrightarrow{runinit}\sigma_c''}
        
        % Builds mapping for out ports.
        \infer[no rule]1{
          \Delta,\Delta_c,NoEv(\sigma),\sigma_c''\vdash
          \mathrm{o}
          \xrightarrow{mapop}
          \sigma'
        }
        
        % Conclusion.
        \infer1
        [{\renewcommand{\arraystretch}{1.5}
          \begin{tabular}{@{}l}
            $id_e\in\mathcal{D}$ \\
            $\Delta(\mathrm{id}_c)=\Delta_c$, $\sigma(\mathrm{id}_c)=\sigma_c$ \\
            $\sigma''={<}\mathcal{S}',\mathcal{C}'(id_c)\leftarrow\sigma_c'',\mathcal{E}'\cup(\mathcal{C}\dcap\mathcal{C}'){>}$ \\
          \end{tabular}
        }] {
          \mathcal{D},\Delta,\sigma\vdash~
          $\vhdle|comp|$~(\mathrm{id}_c, \mathrm{id}_e, \mathrm{g},
          \mathrm{i}, \mathrm{o})
          \xrightarrow{runinit}{}\sigma''
        }
      \end{prooftree}} \\
  \end{tabular}
\end{table}

\subsubsection{Evaluation of the composition of concurrent statements}
\label{subsubsec:ps-stmt-init}

\begin{table}[H]
  \begin{tabular}{l}
    {\fontsize{10}{13}\selectfont\textsc{ParRunInit}} \\
    {\begin{prooftree}
        \hypo{\mathcal{D},\Delta,\sigma\vdash\mathrm{cs}\xrightarrow{runinit}\sigma'}
        \hypo{\mathcal{D},\Delta,\sigma\vdash\mathrm{cs'}\xrightarrow{runinit}\sigma''}
        \infer2
        [{\renewcommand{\arraystretch}{1.5}
          \begin{tabular}{@{}l@{}}
            $\mathcal{E}'\cap\mathcal{E}''=\emptyset$ \\
          \end{tabular}
        }] {
          \mathcal{D},\Delta,\sigma\vdash\mathrm{cs}~\mathtt{||}~\mathrm{cs'}\xrightarrow{runinit}\mathtt{merge}(\sigma,\sigma',\sigma'')
        }
      \end{prooftree}} \\
  \end{tabular}
  \begin{tabular}{l}
    {\fontsize{10}{13}\selectfont\textsc{NullRunInit}} \\
    {\begin{prooftree}
        \infer0
        {
          \Delta,\sigma\vdash\mathtt{null}\xrightarrow{runinit}NoEv(\sigma)
        }
      \end{prooftree}} \\
  \end{tabular}
\end{table}

The \texttt{merge} function computes a new state based on the original
state \textit{o}, and the states \textit{s} and \textit{s'} yielded by
the computation of two concurrent statements. In the resulting state,
the signal value store $\mathcal{S}_m$ is a function merging together
the signal store functions at state $o$, $s$ and $s'$. $S_m$ yields
values from the signal store $\mathcal{S}$ (resp. $\mathcal{S}'$) for
all signal that belongs to the set of events at state $s$
(resp. $s'$), and yields values from the original signal store
$\mathcal{S}_o$ for all unchanged signals. The same goes for the
resulting component instance state store $C_m$. The new set of events
$\mathcal{E}_m$ is the union between set of events at state $s$ and
$s'$. The \texttt{merge} correctly merges the state $o$, $s$ and $s'$
only if the set of events of $s$ and $s'$ are disjoint. Fortunately,
the \textsc{ParRunInit} rule that calls to the \texttt{merge} function
defines the condition of disjoint set of events as a side condition.

\begin{table}[H]
\begin{lstlisting}[language=PseudoCoq]
Definition merge(o,s,s') :=
   let o = ($\mathcal{S}_o$,$\mathcal{C}_o$,$\mathcal{E}_o$) in
   let s = ($\mathcal{S}$,$\mathcal{C}$,$\mathcal{E}$) in
   let s' = ($\mathcal{S}'$,$\mathcal{C}'$,$\mathcal{E}'$) in
   let $\mathcal{S}_m$ = $\lambda\mathrm{id}.$ if $\mathrm{id}\in\mathcal{E}$ then $\mathcal{S}(\mathrm{id})$ else if $\mathrm{id}\in\mathcal{E}'$ then $\mathcal{S}'(\mathrm{id})$ else $\mathcal{S}_o(\mathrm{id})$
   let $\mathcal{C}_m$ = $\lambda\mathrm{id}.$ if $\mathrm{id}\in\mathcal{E}$ then $\mathcal{C}(\mathrm{id})$ else if $\mathrm{id}\in\mathcal{E}'$ then $\mathcal{C}'(\mathrm{id})$ else $\mathcal{C}_o(\mathrm{id})$
   let $\mathcal{E}_m$ = $\mathcal{E}\cup\mathcal{E}'$ in ($\mathcal{S}_m$,$\mathcal{C}_m$,$\mathcal{E}_m$).
\end{lstlisting}
\end{table}

\begin{remark}[No multiply-driven signals]
  For all states $\sigma=(\mathcal{S},\mathcal{C},\mathcal{E})$ and
  $\sigma'=(\mathcal{S}',\mathcal{C}',\mathcal{E}')$ resulting from
  the execution of two concurrent statements cs and cs',
  $\mathcal{E}\cap\mathcal{E}'=\emptyset$. Otherwise, there exists
  some multiply-driven signals, which are forbidden in our semantics.
\end{remark}

\subsection{Clock phases rules}

The following rules express the evaluation of concurrent statements at
clock phases, i.e, the $\uparrow$ and $\downarrow$ phases. The clock
signal, trigerring the evaluation of synchronous process statements,
is represented by the reserved signal identifier \texttt{clk}. Thus,
synchronous processes are processes containing the \texttt{clk} in
their sensitivity list.

\subsubsection{Evaluation of a process statement}

\begin{table}[H]
  {\fontsize{10}{13}\selectfont\textsc{PsRENoClk}}
  
  \begin{prooftree}[template=\inserttext]
    \hypo{}

    \infer1
    [$\mathtt{clk}\notin\mathrm{sl}$]
    {
      $\mathcal{D},\Delta,\sigma\vdash$
      \vhdle|process| \texttt{(}id$_p$\texttt{,} sl\texttt{,} vars\texttt{,} ss\texttt{)}
      $\xrightarrow{\uparrow}$
      $\sigma$
    }
  \end{prooftree}
\end{table}

\begin{premises}
  The $\uparrow$ flag in the $ss_\uparrow$ relation indicates that
  $\mathtt{rising}$ blocks will be executed.
\end{premises}

\begin{table}[H]
  {\fontsize{10}{13}\selectfont\textsc{PsREClk}}
  
  \begin{prooftree}[template=\inserttext]

    \hypo{$\Delta,\sigma,NoEv(\sigma),\Lambda\vdash\mathrm{ss}\xrightarrow{ss_\uparrow}{}\sigma',\Lambda'$}
    
    \infer1
    [{
      \begin{tabular}{@{}l}
        $\mathtt{clk}\in\mathrm{sl}$ \\
        $\Delta(\mathrm{id}_p)=\Lambda$ \\
      \end{tabular}
    }]
    {
      $\mathcal{D},\Delta,\sigma\vdash$
      \vhdle|process| \texttt{(}id$_p$\texttt{,} sl\texttt{,} vars\texttt{,} ss\texttt{)}
      $\xrightarrow{\uparrow}$
      $\sigma'$
    }
  \end{prooftree}
\end{table}

\begin{table}[H]
  {\fontsize{10}{13}\selectfont\textsc{PsFENoClk}}
  
  \begin{prooftree}[template=\inserttext]
    \hypo{}

    \infer1
    [$\mathtt{clk}\notin\mathrm{sl}$]
    {
      $\mathcal{D},\Delta,\sigma
      \vdash$
      \vhdle|process| \texttt{(}id$_p$\texttt{,} sl\texttt{,} vars\texttt{,} ss\texttt{)}
      $\xrightarrow{\downarrow}\sigma$
    }
  \end{prooftree}
\end{table}

\begin{premises}
  The $\downarrow$ flag in the $ss_\downarrow$ relation indicates that
  $\mathtt{falling}$ blocks will be executed.
\end{premises}

\begin{table}[H]
  {\fontsize{10}{13}\selectfont\textsc{PsFEClk}}
  
  \begin{prooftree}[template=\inserttext]

    \hypo{$\Delta,\sigma,NoEv(\sigma),\Lambda\vdash\mathrm{ss}\xrightarrow{ss_\downarrow}{}\sigma',\Lambda'$}
    
    \infer1
    [{
      \begin{tabular}{@{}l}
        $\mathtt{clk}\in\mathrm{sl}$ \\
        $\Delta(\mathrm{id}_p)=\Lambda$ \\
      \end{tabular}
    }]
    {
      $\mathcal{D},\Delta,\sigma
      \vdash$
      \vhdle|process| \texttt{(}id$_p$\texttt{,} sl\texttt{,} vars\texttt{,} ss\texttt{)}
      $\xrightarrow{\downarrow}$
      $\sigma'$
    }
  \end{prooftree}
\end{table}

\subsubsection{Evaluation of a component instantiation statement}
\label{subsubsec:sync-comp-inst}

\begin{table}[H]
  \centering
  \begin{tabular}{@{}l}
    {\fontsize{10}{13}\selectfont\textsc{CompRE}} \\
    {\begin{prooftree}[template=\inserttext]
        
        % Builds mapping for in ports.
        \hypo{$\Delta,\Delta_c,\sigma,\sigma_c\vdash\mathrm{i}\xrightarrow{mapip}\sigma'_c$}
        
        % Executes rising on component behavior.
        \infer[no rule]1{$\mathcal{D},\Delta_c,\sigma_c'\vdash{}\mathcal{D}(id_e).\mathrm{cs}\xrightarrow{\uparrow}\sigma_c''$}
        
        % Builds mapping for out ports.
        \infer[no rule]1{
          $\Delta,\Delta_c,\sigma,\sigma_c''\vdash$
          $\mathrm{o}$
          $\xrightarrow{mapop}$
          $\sigma'$
        }
        
        % Conclusion.
        \infer1
        [{\renewcommand{\arraystretch}{1.5}
          \begin{tabular}{@{}l}
            $id_e\in\mathcal{D}$ \\
            $\Delta(\mathrm{id}_c)=\Delta_c$, $\sigma(\mathrm{id}_c)=\sigma_c$ \\
            $\sigma''={<}\mathcal{S}',\mathcal{C}'(id_c)\leftarrow\sigma_c'',\mathcal{E}'\cup(\mathcal{C}\dcap\mathcal{C}'){>}$ \\
          \end{tabular}
        }]
        {
          $\mathcal{D},\Delta,\sigma\vdash$
          \vhdle|comp| (id$_c$, id$_e$, g, i, o)
          $\xrightarrow{\uparrow}{}\sigma''$
        }
      \end{prooftree}} \\
  \end{tabular}
\end{table}

\begin{table}[H]
  \centering
  \begin{tabular}{@{}l}
    {\fontsize{10}{13}\selectfont\textsc{CompFE}} \\
    {\begin{prooftree}[template=\inserttext]
        
        % Builds mapping for in ports.
        \hypo{$\Delta,\Delta_c,\sigma,\sigma_c\vdash\mathrm{i}\xrightarrow{mapip}\sigma'_c$}
        
        % Executes rising on component behavior.
        \infer[no rule]1{$\mathcal{D},\Delta_c,\sigma_c'\vdash{}\mathcal{D}(id_e).\mathrm{cs}\xrightarrow{\downarrow}\sigma_c''$}
        
        % Builds mapping for out ports.
        \infer[no rule]1{
          $\Delta,\Delta_c,\sigma,\sigma_c''\vdash$
          $\mathrm{o}$
          $\xrightarrow{mapop}$
          $\sigma'$
        }
        
        % Conclusion.
        \infer1
        [{\renewcommand{\arraystretch}{1.5}
          \begin{tabular}{@{}l}
            $id_e\in\mathcal{D}$ \\
            $\Delta(\mathrm{id}_c)=\Delta_c$, $\sigma(\mathrm{id}_c)=\sigma_c$ \\
            $\sigma''={<}\mathcal{S}',\mathcal{C}'(id_c)\leftarrow\sigma_c'',\mathcal{E}'\cup(\mathcal{C}\dcap\mathcal{C}'){>}$ \\
          \end{tabular}
        }]
        {
          $\mathcal{D},\Delta,\sigma\vdash$
          \vhdle|comp| (id$_c$, id$_e$, g, i, o)
          $\xrightarrow{\downarrow}{}\sigma''$
        }
      \end{prooftree}} \\
  \end{tabular}
\end{table}

\subsubsection{Evaluation of the composition of concurrent statements}
\label{subsubsec:comp-of-cs-clk}

\begin{table}[H]
  \begin{tabular}{l}
    {\fontsize{10}{13}\selectfont\textsc{ParFE}} \\
    {\begin{prooftree}
        \hypo{\mathcal{D},\Delta,\sigma\vdash\mathrm{cs}\xrightarrow{\downarrow}\sigma'}
        \hypo{\mathcal{D},\Delta,\sigma\vdash\mathrm{cs'}\xrightarrow{\downarrow}\sigma''}
        \infer2
        [{\renewcommand{\arraystretch}{1.5}
          \begin{tabular}{@{}l@{}}
            $\mathcal{E}'\cap\mathcal{E}''=\emptyset$ \\
          \end{tabular}
        }] {
          \mathcal{D},\Delta,\sigma\vdash\mathrm{cs}~\mathtt{||}~\mathrm{cs'}\xrightarrow{\downarrow}
          \mathtt{merge}(\sigma,\sigma',\sigma'')
        }
      \end{prooftree}} \\
  \end{tabular}
  \begin{tabular}{l}
    {\fontsize{10}{13}\selectfont\textsc{NullFE}} \\
    {\begin{prooftree}
        \infer0
        {
          \Delta,\sigma\vdash\mathtt{null}\xrightarrow{\downarrow}\sigma
        }
      \end{prooftree}} \\
  \end{tabular}
\end{table}

\begin{table}[H]
  \begin{tabular}{l}
    {\fontsize{10}{13}\selectfont\textsc{ParRE}} \\
    {\begin{prooftree}
        \hypo{\mathcal{D},\Delta,\sigma\vdash\mathrm{cs}\xrightarrow{\uparrow}\sigma'}
        \hypo{\mathcal{D},\Delta,\sigma\vdash\mathrm{cs'}\xrightarrow{\uparrow}\sigma''}
        \infer2
        [{\renewcommand{\arraystretch}{1.5}
          \begin{tabular}{@{}l@{}}
            $\mathcal{E}'\cap\mathcal{E}''=\emptyset$ \\
          \end{tabular}
        }] {
          \mathcal{D},\Delta,\sigma\vdash\mathrm{cs}~\mathtt{||}~\mathrm{cs'}\xrightarrow{\uparrow}
          \mathtt{merge}(\sigma,\sigma',\sigma'')
        }
      \end{prooftree}} \\
  \end{tabular}
  \begin{tabular}{l}
    {\fontsize{10}{13}\selectfont\textsc{NullRE}} \\
    {\begin{prooftree}
        \infer0
        {\Delta,\sigma\vdash\mathtt{null}\xrightarrow{\uparrow}\sigma}
      \end{prooftree}} \\
  \end{tabular}
\end{table}

\subsection{Stabilization rules}
\label{sec:stab-rules}

\begin{table}[H]
  \begin{tabular}{@{}l}
    {\fontsize{10}{13}\selectfont\textsc{StabilizeEnd}} \\
    
    {\begin{prooftree}[template=\inserttext]

        \hypo{$\mathcal{D},\Delta,\sigma\vdash\mathrm{cs}\xrightarrow{comb}\sigma$}
        \infer1
        [$\mathcal{E}=\emptyset$]
        {
          $\mathcal{D},\Delta,\sigma\vdash\mathrm{cs}\xrightarrow{\rightsquigarrow}\sigma$
        }
      \end{prooftree}} \\
  \end{tabular}
\end{table}

\begin{table}[H]  
  \begin{tabular}{@{}l}
    {\fontsize{10}{13}\selectfont\textsc{StabilizeLoop}} \\
    {\begin{prooftree}[template=\inserttext]
        \hypo{$\mathcal{D},\Delta,\sigma\vdash\mathrm{cs}\xrightarrow{comb}\sigma'$}
        \hypo{$\mathcal{D},\Delta,\sigma'\vdash\mathrm{cs}\xrightarrow{\rightsquigarrow}\sigma''$}
        \infer2
        [{
          \begin{tabular}{@{}l}
            $\mathcal{E}\neq\emptyset$ \\
            $\mathcal{E}''=\emptyset$
          \end{tabular}
        }]
        {
          $\mathcal{D},\Delta,\sigma\vdash\mathrm{cs}\xrightarrow{\rightsquigarrow}\sigma''$
        }
      \end{prooftree}} \\
  \end{tabular}
\end{table}

\subsubsection{Evaluation of a process statement}

% \begin{table}[H]
%   {\fontsize{10}{13}\selectfont\textsc{PsCombStable}}
  
%   \begin{prooftree}[template=\inserttext]
%     \infer0
%     [{
%       \begin{tabular}{@{}l}
%         $\sigma={<}\mathcal{S},\mathcal{C},\mathcal{E}{>}$\\
%         $\mathrm{sl}\cap\mathcal{E}=\emptyset$
%       \end{tabular}
%     }]
%     {
%       $\mathcal{D},\Delta,\sigma\vdash$
%       \vhdle|process| \texttt{(}id$_p$\texttt{,} sl\texttt{,} vars\texttt{,} ss\texttt{)}
%       $\xrightarrow{comb}NoEv(\sigma)$
%     }
%   \end{prooftree}
% \end{table}

\begin{premises}
  The $c$ flag (for \textit{combinational}) on the $ss_c$ relation
  indicates that instructions responding to clock events
  (\texttt{falling} and \texttt{rising} blocks) and instructions
  executed during the initialization phase only (\texttt{rst} blocks)
  will not be considered.
\end{premises}

\begin{table}[H]
  {\fontsize{10}{13}\selectfont\textsc{PsComb}}
  
  \begin{prooftree}[template=\inserttext]
    
    % Executes ss.
    \hypo{$\Delta,\sigma,NoEv(\sigma),\Lambda\vdash\mathrm{ss}\xrightarrow{ss_c}{}\sigma',\Lambda'$}

    % Conclusion.
    \infer1
    [{
      \begin{tabular}{@{}l}
        $\Delta(\mathrm{id}_p)=\Lambda$ \\
      \end{tabular}
    }]
    {
      $\mathcal{D},\Delta,\sigma\vdash$
      \vhdle|process| \texttt{(}id$_p$\texttt{,} sl\texttt{,} vars\texttt{,} ss\texttt{)}
      $\xrightarrow{comb}\sigma'$
    }
  \end{prooftree}
\end{table}

\subsubsection{Evaluation of a component instantiation statement}

\begin{table}[H]
  \centering
  \begin{tabular}{@{}l}
    {\fontsize{10}{13}\selectfont\textsc{CompComb}} \\
    {\begin{prooftree}[template=\inserttext]
        
        % Builds mapping for in ports.
        \hypo{$\Delta,\Delta_c,\sigma,\sigma_c\vdash\mathrm{i}\xrightarrow{mapip}\sigma'_c$}
        
        % Executes comb on component behavior.
        \infer[no rule]1{$\mathcal{D},\Delta_c,\sigma_c'\vdash{}\mathcal{D}(id_e).\mathrm{cs}\xrightarrow{comb}\sigma_c''$}
        
        % Builds mapping for out ports.
        \infer[no rule]1{
          $\Delta,\Delta_c,NoEv(\sigma),\sigma_c''\vdash$
          $\mathrm{o}$
          $\xrightarrow{mapop}$
          $\sigma'$
        }
        
        % Conclusion.
        \infer1
        [{\renewcommand{\arraystretch}{1.5}
          \begin{tabular}{@{}l}
            $id_e\in\mathcal{D}$ \\
            $\Delta(\mathrm{id}_c)=\Delta_c$, $\sigma(\mathrm{id}_c)=\sigma_c$ \\
            $\sigma''={<}\mathcal{S}',\mathcal{C}'(id_c)\leftarrow\sigma_c'',\mathcal{E}'\cup(\mathcal{C}\dcap\mathcal{C}'){>}$ \\
          \end{tabular}
        }]
        {
          $\mathcal{D},\Delta,\sigma\vdash$
          \vhdle|comp| (id$_c$, id$_e$, g, i, o)
          $\xrightarrow{comb}$
          $\sigma''$
        }
      \end{prooftree}} \\
  \end{tabular}
\end{table}

\subsubsection{Evaluation of the composition of concurrent statements}

\begin{table}[H]
  \begin{tabular}{l}
    {\fontsize{10}{13}\selectfont\textsc{ParComb}} \\
    {\begin{prooftree}
        \hypo{\mathcal{D},\Delta,\sigma\vdash\mathrm{cs}\xrightarrow{comb}\sigma'}
        \hypo{\mathcal{D},\Delta,\sigma\vdash\mathrm{cs'}\xrightarrow{comb}\sigma''}
        \infer2
        [{\renewcommand{\arraystretch}{1.5}
          \begin{tabular}{@{}l@{}}
            $\mathcal{E}'\cap\mathcal{E}''=\emptyset$ \\
          \end{tabular}
        }] {
          \mathcal{D},\Delta,\sigma\vdash\mathrm{cs}~\mathtt{||}~\mathrm{cs'}\xrightarrow{comb}
          \mathtt{merge}(\sigma,\sigma',\sigma'')
        }
      \end{prooftree}} \\
  \end{tabular}
  \begin{tabular}{l}
    {\fontsize{10}{13}\selectfont\textsc{NullComb}} \\
    {\begin{prooftree}
        \infer0
        {
          \Delta,\sigma\vdash\mathtt{null}\xrightarrow{\downarrow}NoEv(\sigma)
        }
      \end{prooftree}} \\
  \end{tabular}
\end{table}

\subsection{Evaluation of input and output port maps}
\label{subsec:mapinout}

\begin{table}[H]
  \begin{tabular}{@{}l}
    {\fontsize{10}{13}\selectfont\textsc{MapipSimple}} \\
    {\begin{prooftree}

        % Evaluates e.
        \hypo{\Delta,\sigma\vdash\mathrm{e}\xrightarrow{e}v}

        % Checks that v complies with T.
        \hypo{v\in_c{}T}

        % Conclusion.
        \infer2
        [{\renewcommand{\arraystretch}{1.5}
          \begin{tabular}{@{}l}
            $\Delta_c(\mathrm{id}_s)=T$ \\
            $\sigma_c={<}\mathcal{S},\mathcal{C},\mathcal{E}{>}$ \\
            $\mathcal{S}'=\mathcal{S}(id_s)\leftarrow{}v$ \\
          \end{tabular}
        }]
        {
          \Delta,\Delta_c,\sigma,\sigma_c\vdash
          \mathrm{id}_s\Rightarrow\mathrm{e}
          \xrightarrow{mapip}
          {<}\mathcal{S}',\mathcal{C},\mathcal{E}{>}
        }
      \end{prooftree}} \\
  \end{tabular}
\end{table}

\begin{table}[H]
  \begin{tabular}{@{}l}
    {\fontsize{10}{13}\selectfont\textsc{MapipPartial}} \\
    {\begin{prooftree}

        % Evaluates e_i and e.
        \hypo{\Delta,\sigma&\vdash\mathrm{e}\xrightarrow{e}v}
        \infer[no rule]1{&\vdash\mathrm{e}_i\xrightarrow{e}v_i}        

        % Checks that v complies with T.
        \hypo{v&\in_c{}T}
        \infer[no rule]1{v_i&\in_c{}\mathtt{nat}(n,m)}

        % Conclusion.
        \infer2
        [{\renewcommand{\arraystretch}{1.5}
          \begin{tabular}{@{}l}
            $\Delta_c(\mathrm{id}_s)=\mathtt{array(}T,n,m\mathtt{)}$ \\
            $\sigma_c={<}\mathcal{S},\mathcal{C},\mathcal{E}{>}$ \\
            $\mathcal{S}'=\mathcal{S}(id_s)\leftarrow\mathtt{set\_at}(v,v_i,\mathcal{S}(id_s))$ \\
          \end{tabular}
        }]
        {
          \Delta,\Delta_c,\sigma,\sigma_c\vdash
          \mathrm{id}_s(\mathrm{e}_i)\Rightarrow\mathrm{e}
          \xrightarrow{mapip}
          {<}\mathcal{S}',\mathcal{C},\mathcal{E}{>}
        }
      \end{prooftree}} \\
  \end{tabular}
\end{table}

\begin{table}[H]
  {\fontsize{10}{13}\selectfont\textsc{MapipComp}}
  
  \begin{prooftree}    
    % Evaluates mapip on assoc_p.
    \hypo{
      \Delta,\Delta_c,\sigma,\sigma_c\vdash
      \mathrm{assoc}_{ip} 
      \xrightarrow{mapip}
      \sigma'_c
    }

    % Evaluates mapip on portmap.
    \hypo{
      \Delta,\Delta_c,\sigma,\sigma'_c\vdash
      \mathrm{ipmap}
      \xrightarrow{mapip}
      \sigma''_c
    }

    % Conclusion.
    \infer2
    {
      \Delta,\Delta_c,\sigma,\sigma_c\vdash
      \langle{}\mathrm{assoc}_{ip},~\mathrm{ipmap}\rangle
      \xrightarrow{mapip}
      \sigma''_c
    }
  \end{prooftree}
\end{table}

\begin{remark}[Out ports and $e$]
  We can not use the $e$ relation to interpret the values of output
  ports, because output ports are write-only constructs. We append the
  flag $o$ to the $e$ relation (i.e, $e_o$) to permit the
  evaluation of output port identifiers as regular signal identifier
  expressions.
\end{remark}

\begin{table}[H]
  \begin{tabular}{@{}l}
    {\fontsize{10}{13}\selectfont\textsc{MapopOpen}} \\
    {\begin{prooftree}

        % Conclusion
        \infer
        0
        {
          \Delta,\Delta_c,\sigma,\sigma_c\vdash
          \mathrm{id}_f\Rightarrow\mathtt{open}
          \xrightarrow{mapop}
          \sigma_c
        }
      \end{prooftree}} \\
  \end{tabular}
\end{table}

\begin{table}[H]
  \begin{tabular}{@{}l}
    {\fontsize{10}{13}\selectfont\textsc{MapopSimpleToSimple}} \\
    {\begin{prooftree}

        % Evaluates name.
        \hypo{\Delta_c,\sigma_c\vdash\mathrm{id}_f\xrightarrow{e_o}v}

        % Checks that v complies with T.
        \hypo{v\in_c{}T}
        
        % Conclusion.
        \infer2
        [{\renewcommand{\arraystretch}{1.5}
          \begin{tabular}{@{}l}
            $\mathrm{id}_a\in{}Sigs(\Delta)\cup{}Outs(\Delta)$ \\
            $\Delta(\mathrm{id}_a)=T$ \\
            $\sigma={<}\mathcal{S},\mathcal{C},\mathcal{E}{>}$ \\
            $\mathcal{S}'=\mathcal{S}(id_a)\leftarrow{}v$, $\mathcal{E}'=\mathcal{E}\cup(\mathcal{S}\dcap\mathcal{S}')$ \\
          \end{tabular}
        }]
        {
          \Delta,\Delta_c,\sigma,\sigma_c\vdash
          \mathrm{id}_f\Rightarrow\mathrm{id}_a
          \xrightarrow{mapop}
          {<}\mathcal{S}',\mathcal{C},\mathcal{E}'{>}
        }
      \end{prooftree}} \\
  \end{tabular}
\end{table}

\begin{table}[H]
  \begin{tabular}{@{}l}
    {\fontsize{10}{13}\selectfont\textsc{MapopSimpleToPartial}} \\
    {\begin{prooftree}

        % Evaluates id_f and e_i.
        \hypo{&\vdash\mathrm{e}_i\xrightarrow{e}v_i}
        \infer[no rule]1{&\Delta_c,\sigma_c\vdash\mathrm{id}_f\xrightarrow{e_o}v}

        % Checks that v complies with T.
        \hypo{v&\in_c{}T}
        \infer[no rule]1{v_i&\in_c{}\mathtt{nat}(n,m)}

        % Conclusion.
        \infer2
        [{\renewcommand{\arraystretch}{1.5}
          \begin{tabular}{l}
            $\mathrm{id}_a\in{}Sigs(\Delta)\cup{}Outs(\Delta)$ \\
            $\Delta(\mathrm{id}_a)=\mathtt{array(}T,n,m\mathtt{)}$ \\
            $\sigma={<}\mathcal{S},\mathcal{C},\mathcal{E}{>}$ \\
            $\mathcal{S}'=\mathcal{S}(id_a)\leftarrow\mathtt{set\_at}(v,v_i,\mathcal{S}(id_a))$ \\
            $\mathcal{E}'=\mathcal{E}\cup(\mathcal{S}\dcap\mathcal{S}')$ \\
          \end{tabular}
        }]
        {
          \Delta,\Delta_c,\sigma,\sigma_c\vdash
          \mathrm{id}_f\Rightarrow\mathrm{id}_a(\mathrm{e}_i)
          \xrightarrow{mapop}
          {<}\mathcal{S}',\mathcal{C},\mathcal{E}'{>}
        }
      \end{prooftree}} \\
  \end{tabular}
\end{table}

\begin{table}[H]
  \begin{tabular}{@{}l}
    {\fontsize{10}{13}\selectfont\textsc{MapopPartialToSimple}} \\
    {\begin{prooftree}

        % Evaluates name.
        \hypo{\Delta_c,\sigma_c\vdash\mathrm{id}_f(\mathrm{e}'_i)\xrightarrow{e_o}v}

        % Checks that v complies with T.
        \hypo{v\in_c{}T}
        
        % Conclusion.
        \infer2
        [{\renewcommand{\arraystretch}{1.5}
          \begin{tabular}{@{}l}
            $\mathrm{id}_a\in{}Sigs(\Delta)\cup{}Outs(\Delta)$ \\
            $\Delta(\mathrm{id}_a)=T$ \\
            $\sigma={<}\mathcal{S},\mathcal{C},\mathcal{E}{>}$ \\
            $\mathcal{S}'=\mathcal{S}(id_a)\leftarrow{}v$, $\mathcal{E}'=\mathcal{E}\cup(\mathcal{S}\dcap\mathcal{S}')$ \\
          \end{tabular}
        }]
        {
          \Delta,\Delta_c,\sigma,\sigma_c\vdash
          \mathrm{id}_f(\mathrm{e}'_i)\Rightarrow\mathrm{id}_a
          \xrightarrow{mapop}
          {<}\mathcal{S}',\mathcal{C},\mathcal{E}'{>}
        }
      \end{prooftree}} \\
  \end{tabular}
\end{table}

\begin{table}[H]
  \begin{tabular}{@{}l}
    {\fontsize{10}{13}\selectfont\textsc{MapopPartialToPartial}} \\
    {\begin{prooftree}

        % Evaluates id_f and e_i.
        \hypo{&\vdash\mathrm{e}_i\xrightarrow{e}v_i}
        \infer[no rule]1{&\Delta_c,\sigma_c\vdash\mathrm{id}_f(\mathrm{e}'_i)\xrightarrow{e_o}v}

        % Checks that v complies with T.
        \hypo{v&\in_c{}T}
        \infer[no rule]1{v_i&\in_c{}\mathtt{nat}(n,m)}

        % Conclusion.
        \infer2
        [{\renewcommand{\arraystretch}{1.5}
          \begin{tabular}{l}
            $\mathrm{id}_a\in{}Sigs(\Delta)\cup{}Outs(\Delta)$ \\
            $\Delta(\mathrm{id}_a)=\mathtt{array(}T,n,m\mathtt{)}$ \\
            $\sigma={<}\mathcal{S},\mathcal{C},\mathcal{E}{>}$ \\
            $\mathcal{S}'=\mathcal{S}(id_a)\leftarrow\mathtt{set\_at}(v,v_i,\mathcal{S}(id_a))$ \\
            $\mathcal{E}'=\mathcal{E}\cup(\mathcal{S}\dcap\mathcal{S}')$ \\
          \end{tabular}
        }]
        {
          \Delta,\Delta_c,\sigma,\sigma_c\vdash
          \mathrm{id}_f(\mathrm{e}'_i)\Rightarrow\mathrm{id}_a(\mathrm{e}_i)
          \xrightarrow{mapop}
          {<}\mathcal{S}',\mathcal{C},\mathcal{E}'{>}
        }
      \end{prooftree}} \\
  \end{tabular}
\end{table}


\begin{table}[H]
  {\fontsize{10}{13}\selectfont\textsc{MapopComp}}
  
  \begin{prooftree}    
    % Evaluates mapip on assoc.
    \hypo{\Delta,\Delta_c,\sigma,\sigma_c\vdash\mathrm{assoc}_{po}\xrightarrow{mapop}\sigma'}

    % Evaluates mapip on portmap.
    \hypo{\Delta,\Delta_c,\sigma',\sigma_c\vdash\mathrm{opmap}\xrightarrow{mapop}\sigma''}

    % Conclusion.
    \infer2
    {
      \Delta,\Delta_c,\sigma,\sigma_c\vdash
      \langle{}\mathrm{assoc}_{po},~\mathrm{opmap}\rangle
      \xrightarrow{mapop}\sigma''
    }
  \end{prooftree}
\end{table}

The $e_o$ relation is only defined to retrieve the value of out
ports from a store signal $\mathcal{S}$ under a design state
$\sigma={<}\mathcal{S},\mathcal{C},\mathcal{E}{>}$.

\begin{table}[H]
  \centering
  \begin{tabular}{@{}l}
    {\fontsize{10}{13}\selectfont\textsc{OutO}} \\
    {\begin{prooftree}
        \infer0
        [{
          \begin{tabular}{@{}l}
            $\mathrm{id}_s\in{}Outs(\Delta)$ \\
            $\mathrm{id}_s\in\sigma$ \\
          \end{tabular}
        }] {
          \Delta,\sigma\vdash
          \mathrm{id}_s
          \xrightarrow{e_o}
          \sigma(\mathrm{id}_s)
        }
      \end{prooftree}} \\
  \end{tabular}
\end{table}

\begin{table}[H]
  \centering
  \begin{tabular}{@{}l}
    {\fontsize{10}{13}\selectfont\textsc{IdxOutO}} \\
    {\begin{prooftree}

        % Evaluates e_i.
        \hypo{
          \vdash
          \mathrm{e}_i
          \xrightarrow{e}
          v_i
        }

        % Checks well-typed v_i.
        \hypo{v_i\in_c\mathtt{nat}(n,m)}
        
        % Conclusion.
        \infer2
        [{
          \begin{tabular}{@{}l}
            $\mathrm{id}_s\in{}Outs(\Delta)$ \\
            $\mathrm{id}_s\in\sigma$ \\
            $\Delta(\mathrm{id}_s)=\mathtt{array(}T,n,m\mathtt{)}$ \\
            $i=v_i~\mathtt{mod}~n$ \\
          \end{tabular}
        }]
        {
          \Delta,\sigma\vdash
          \mathrm{id}_s(\mathrm{e}_i)
          \xrightarrow{e_o}
          \mathtt{get\_at}(i,\sigma(\mathrm{id}_s))
        }
      \end{prooftree}} \\
  \end{tabular}
\end{table}

\subsection{Evaluation of sequential statements}
\label{subsec:seq-stmts}

The $ss$ symbol indicates that the evaluation of the considered
sequential statement does not depend on a specific flag (i.e, the $c$,
$i$, $\uparrow$ or $\downarrow$ flag). In the rules of the $ss$
relation, a $ss$ flag is tranferred from the conclusion to the
premises when an sequential statement is composed of inner sequential
blocks.

\subsubsection{Signal assigment statement} A signal assignment
generates a new design state with a modified signal store and a new
set of events. Note that there are two states on the left side of the
thesis symbol. $\sigma$ represents the state holding the current
values of signals, and $\sigma_w$ holds the new values of signals
(i.e. the \emph{written} state).

\begin{premises}
  The premise $\mathcal{S}(id_s)\in_c{}T$ checks that the value
  associated to signal $id_s$ in the signal store of $\sigma$ complies
  with type $T$, where $T$ is the type associated with signal $id_s$
  in $\Delta$.
\end{premises}

\begin{table}[H]
  \begin{tabular}{@{}l}
    {\fontsize{10}{13}\selectfont\textsc{SigAssign}} \\
    {\begin{prooftree}

        % Evaluates e.
        \hypo{\Delta,\sigma,\Lambda\vdash\mathrm{e}\xrightarrow{e}v}
        
        % Checks that v complies with T.
        \hypo{v\in_c{}T}
        
        % Conclusion.
        \infer2
        [{\renewcommand{\arraystretch}{1.5}
          \begin{tabular}{@{}l}
            $\mathrm{id}_s\in{}Sigs(\Delta)\cup{}Outs(\Delta)$ \\
            $\Delta(\mathrm{id}_s)=T$ \\
            % $\sigma={<}\mathcal{S},\mathcal{C},\mathcal{E}{>}$ \\
            $\mathcal{S}'_w=\mathcal{S}_w(id_s)\leftarrow{}v$ \\
            $\mathcal{E}'_w=\mathcal{E}_w\cup(\mathcal{S}_w\dcap\mathcal{S}'_w)$ \\
          \end{tabular}
        }]
        {
          \Delta,\sigma,\sigma_w,\Lambda\vdash
          \mathrm{id}_s\Leftarrow\mathrm{e}
          \xrightarrow{ss}
          {<}\mathcal{S}'_w,\mathcal{C}_w,\mathcal{E}'_w{>},\Lambda
        }
      \end{prooftree}} \\
  \end{tabular}
\end{table}

\begin{table}[H]
  \begin{tabular}{@{}l}
    {\fontsize{10}{13}\selectfont\textsc{IdxSigAssign}} \\
    {\begin{prooftree}

        % Evaluates e_i and e.
        \hypo{\Delta,\sigma,\Lambda&\vdash\mathrm{e}_i\xrightarrow{e}v_i}
        \infer[no rule]1{\Delta,\sigma,\Lambda&\vdash\mathrm{e}\xrightarrow{e}v}

        % Checks that v complies with T.

        \hypo{v&\in_c{}T}
        \infer[no rule]1{v_i&\in_c{}\mathtt{nat}(n,m)}
        
        % Conclusion.
        \infer2
        [{\renewcommand{\arraystretch}{1.5}
          \begin{tabular}{@{}l}
            $\mathrm{id}_s\in{}Sigs(\Delta)\cup{}Outs(\Delta)$ \\
            $\Delta(\mathrm{id}_s)=\mathtt{array(}T,n,m\mathtt{)}$ \\
            $\mathcal{S}'_w=\mathcal{S}_w(id_s)\leftarrow\mathtt{set\_at}(v,v_i,\mathcal{S}_w(id_s))$ \\
            $\mathcal{E}'_w=\mathcal{E}_w\cup(\mathcal{S}_w\dcap\mathcal{S}'_w)$ \\
          \end{tabular}
        }]
        {
          \Delta,\sigma,\sigma_w,\Lambda\vdash
          \mathrm{id}_s(\mathrm{e}_i)\Leftarrow\mathrm{e}
          \xrightarrow{ss}
          {<}\mathcal{S}'_w,\mathcal{C}_w,\mathcal{E}'_w{>},\Lambda
        }
      \end{prooftree}} \\
  \end{tabular}
\end{table}

\subsubsection{Variable assignment statement}

A variable assignment statement modifies the variable values in the
local environment.

\begin{table}[H]
  \begin{tabular}{@{}l}
    {\fontsize{10}{13}\selectfont\textsc{VarAssign}} \\    
    {\begin{prooftree}

        % Evaluates e.
        \hypo{\Delta,\sigma,\Lambda\vdash\mathrm{e}\xrightarrow{e}v}

        % Checks that v complies with T.
        \hypo{v\in_c{}T}

        % Conclusion.
        \infer2
        [{
          \begin{tabular}{@{}l}
            $\mathrm{id}_v\in\Lambda$ \\
            $\Lambda(\mathrm{id}_v)=(T,val)$ \\
          \end{tabular}
        }]
        {
          \Delta,\sigma,\sigma_w,\Lambda\vdash
          \mathrm{id}_v:=\mathrm{e}
          \xrightarrow{ss}
          \sigma_w,\Lambda(id_v)\leftarrow{}(T,v)
        }
      \end{prooftree}} \\
  \end{tabular}
\end{table}

\begin{table}[H]
  \begin{tabular}{@{}l}
    {\fontsize{10}{13}\selectfont\textsc{IdxVarAssign}} \\    
    {\begin{prooftree}

        % Evaluates e_i.
        \hypo{\Delta,\sigma,\Lambda&\vdash\mathrm{e}_i\xrightarrow{e}v_i}
        \infer[no rule]1{\Delta,\sigma,\Lambda&\vdash\mathrm{e}\xrightarrow{e}v}

        % Checks that v complies with T.
        \hypo{v_i&\in_c\mathtt{nat}(n,m)}
        \infer[no rule]1{v&\in_c{}T}

        % Conclusion.
        \infer2
        [{
          \begin{tabular}{@{}l}
            $\mathrm{id}_v\in\Lambda$ \\
            $\Lambda(\mathrm{id}_v)=(\mathtt{array(}T,n,m\mathtt{)},val)$ \\
          \end{tabular}
        }]
        {
          \Delta,\sigma,\sigma_w,\Lambda\vdash
          \mathrm{id}_v(\mathrm{e}_i):=\mathrm{e}
          \xrightarrow{ss}
          \sigma_w,\Lambda(id_v)\leftarrow{}(T,\mathtt{set\_at}(v,v_i,val))
        }
      \end{prooftree}} \\
  \end{tabular}
\end{table}

\subsubsection{If statement}
\label{subsubsec:if-stmt}

\begin{table}[H]
  \begin{tabular}{@{}l}
    {\fontsize{10}{13}\selectfont\textsc{If}$\top$} \\    
    {\begin{prooftree}

        % Evaluates condition.
        \hypo{\Delta,\sigma,\Lambda\vdash\mathrm{e}\xrightarrow{e}\top}

        % Evaluates ss.
        \hypo{\Delta,\sigma,\sigma_w,\Lambda\vdash\mathrm{ss}\xrightarrow{ss}\sigma'_w,\Lambda'}

        % Conclusion.
        \infer2
        {
          \Delta,\sigma,\sigma_w,\Lambda\vdash~
          $\vhdle|if|$~(\mathrm{e})~\mathrm{ss}
          \xrightarrow{ss}
          \sigma'_w,\Lambda'
        }
      \end{prooftree}} \\
  \end{tabular}
  \begin{tabular}{@{}l}
    {\fontsize{10}{13}\selectfont\textsc{If}$\bot$} \\
    {\begin{prooftree}

        % Evaluates condition.
        \hypo{\Delta,\sigma,\Lambda\vdash\mathrm{e}\xrightarrow{e}\bot}

        % Conclusion.
        \infer1
        {
          \Delta,\sigma,\sigma_w,\Lambda\vdash~
          $\vhdle|if|$~(\mathrm{e})~\mathrm{ss}
          \xrightarrow{ss}
          \sigma_w,\Lambda
        }
      \end{prooftree}} \\
  \end{tabular}
\end{table}

\begin{table}[H]
  \begin{tabular}{@{}l}
    {\fontsize{10}{13}\selectfont\textsc{IfElse}$\top$} \\    
    {\begin{prooftree}

        % Evaluates condition.
        \hypo{\Delta,\sigma,\Lambda\vdash\mathrm{e}\xrightarrow{expr}\top}

        % Evaluates ss.
        \hypo{\Delta,\sigma,\sigma_w,\Lambda\vdash\mathrm{ss}\xrightarrow{ss}\sigma'_w,\Lambda'}

        % Conclusion.
        \infer2
        {
          \Delta,\sigma,\sigma_w,\Lambda\vdash~
          $\vhdle|if|$~(\mathrm{e})~\mathrm{ss}~\mathrm{ss}'
          \xrightarrow{ss}
          \sigma'_w,\Lambda'
        }
      \end{prooftree}} \\
  \end{tabular}
  \begin{tabular}{@{}l}
    {\fontsize{10}{13}\selectfont\textsc{IfElse}$\bot$} \\  
    {\begin{prooftree}

        % Evaluates condition.
        \hypo{\Delta,\sigma,\Lambda\vdash\mathrm{e}\xrightarrow{e}\bot}

        % Evaluates ss.
        \hypo{\Delta,\sigma,\sigma_w,\Lambda\vdash\mathrm{ss}'\xrightarrow{ss}\sigma'_w,\Lambda'}

        % Conclusion.
        \infer2
        {
          \Delta,\sigma,\sigma_w,\Lambda\vdash~
          $\vhdle|if|$~(\mathrm{e})~\mathrm{ss}~\mathrm{ss}'
          \xrightarrow{ss}
          \sigma'_w,\Lambda'
        }
      \end{prooftree}} \\
  \end{tabular}
\end{table}

\subsubsection{Loop statement}

\begin{table}[H]
  \begin{tabular}{@{}l}
    {\fontsize{10}{13}\selectfont\textsc{Loop}$\bot$} \\
    {\begin{prooftree}

        % Evaluates upper bound check.
        \hypo{\Delta,\sigma,\Lambda_i\vdash\mathrm{id}_v=\mathrm{e}'\xrightarrow{e}\bot}

        % Evaluates ss and recursive call.
        \hypo{\Delta,\sigma,\sigma_w,\Lambda_i&\vdash\mathrm{ss}\xrightarrow{ss}\sigma'_w,\Lambda'}
        \infer[no rule]1{
          \Delta,\sigma,\sigma'_w,\Lambda'&\vdash~
          $\vhdle|for|$~(\mathrm{id}_v,\mathrm{e},\mathrm{e}')~\mathrm{ss}
          \xrightarrow{ss}\sigma''_w,\Lambda''
        }
        
        % Conclusion. 
        \infer2
        [{\renewcommand{\arraystretch}{1.5}
          \begin{tabular}{@{}l}
            $\mathrm{id}_v\in\Lambda$ \\
            $\Lambda(\mathrm{id}_v)=(T,val)$ \\
            $\Lambda_i=\Lambda(id_v)\leftarrow{}(T,val+1)$ \\
          \end{tabular}
        }]
        {
          \Delta,\sigma,\sigma_w,\Lambda\vdash~
          $\vhdle|for|$~(\mathrm{id}_v,\mathrm{e},\mathrm{e}')~\mathrm{ss}
          \xrightarrow{ss}
          \sigma''_w,\Lambda''
        }
      \end{prooftree}} \\
  \end{tabular}
\end{table}

\begin{table}[H]
  \begin{tabular}{@{}l}
    {\fontsize{10}{13}\selectfont\textsc{Loop}$\top$} \\    
    {\begin{prooftree}
                
        % Upper bound check true.
        \hypo{\Delta,\sigma,\Lambda_i\vdash\mathrm{id}_v=\mathrm{e}'\xrightarrow{e}\top}

        % Conclusion.
        \infer1
        [{\renewcommand{\arraystretch}{1.5}
          \begin{tabular}{@{}l}
            $\mathrm{id}_v\in\Lambda$ \\
            $\Lambda(\mathrm{id}_v)=(T,val)$ \\
            $\Lambda_i=\Lambda(id_v)\leftarrow(T,val+1)$ \\
          \end{tabular}
        }]
        {
          \Delta,\sigma,\sigma_w,\Lambda\vdash~
          $\vhdle|for|$~(\mathrm{id}_v,\mathrm{e},\mathrm{e}')~\mathrm{ss}
          \xrightarrow{ss}
          \sigma_w,\Lambda\setminus(id_v,\Lambda(id_v))
        }
      \end{prooftree}} \\
  \end{tabular}
\end{table}

\begin{table}[H]
  \begin{tabular}{@{}l}
    {\fontsize{10}{13}\selectfont\textsc{LoopInit}} \\
    {\begin{prooftree}
        
        % Evaluates e and e'.
        \hypo{\Delta,\sigma,\Lambda&\vdash\mathrm{e}\xrightarrow{e}v}
        \infer[no rule]1{\Delta,\sigma,\Lambda&\vdash\mathrm{e'}\xrightarrow{e}v'}
        
        % Upper bound check true.
        \hypo{\Delta,\sigma,\sigma_w,\Lambda_i\vdash~
          $\vhdle|for|$~(\mathrm{id}_v,\mathrm{e},\mathrm{e}')~\mathrm{ss}
          \xrightarrow{ss}
          \sigma'_w,\Lambda'}

        % Conclusion.
        \infer2
        [{\renewcommand{\arraystretch}{1.5}
          \begin{tabular}{@{}l}
            $\mathrm{id}_v\notin\Lambda$ \\
            $\Lambda_i=\Lambda\cup(id_v,(\mathtt{nat}(v,v'),v))$ \\
          \end{tabular}
        }]
        {
          \Delta,\sigma,\sigma_w,\Lambda\vdash~
          $\vhdle|for|$~(\mathrm{id}_v,\mathrm{e},\mathrm{e}')~\mathrm{ss}
          \xrightarrow{ss}
          \sigma'_w,\Lambda'
        }
      \end{prooftree}} \\
  \end{tabular}
\end{table}

\subsubsection{Rising and falling edge block statements}
\label{subsubsec:rise-and-fall-stmts}

\begin{table}[H]
  \begin{tabular}{l}
    {\fontsize{10}{13}\selectfont\textsc{RisingEdgeDefault}} \\    
    {\begin{prooftree}
        
        % Conclusion.
        \infer0
        [$f\neq\uparrow$]
        {
          \Delta,\sigma,\sigma_w,\Lambda\vdash~
          $\vhdle|rising|$~\mathrm{ss}
          \xrightarrow{ss_f}
          \sigma_w,\Lambda
        }
      \end{prooftree}} \\
  \end{tabular}
  \begin{tabular}{l}
    {\fontsize{10}{13}\selectfont\textsc{FallingEdgeDefault}} \\    
    {\begin{prooftree}
        
        % Conclusion.
        \infer0
        [$f\neq\downarrow$]
        {
          \Delta,\sigma,\sigma_w,\Lambda\vdash~
          $\vhdle|falling|$~\mathrm{ss}
          \xrightarrow{ss_f}
          \sigma_w,\Lambda
        }
      \end{prooftree}} \\
  \end{tabular}
\end{table}

\begin{table}[H]
  \begin{tabular}{l}
    {\fontsize{10}{13}\selectfont\textsc{RisingEdgeExec}} \\    
    {\begin{prooftree}

        % Evaluates ss.
        \hypo{\Delta,\sigma,\sigma_w,\Lambda\vdash\mathrm{ss}\xrightarrow{ss_\uparrow}\sigma'_w,\Lambda'}
        
        % Conclusion.
        \infer1
        {
          \Delta,\sigma,\sigma_w,\Lambda\vdash~
          $\vhdle|rising|$~\mathrm{ss}
          \xrightarrow{ss_\uparrow}
          \sigma'_w,\Lambda'
        }
      \end{prooftree}} \\
  \end{tabular}
  \begin{tabular}{l}
    {\fontsize{10}{13}\selectfont\textsc{FallingEdgeExec}} \\    
    {\begin{prooftree}

        % Evaluates ss.
        \hypo{\Delta,\sigma,\sigma_w,\Lambda\vdash\mathrm{ss}\xrightarrow{ss_\downarrow}\sigma'_w,\Lambda'}
        
        % Conclusion.
        \infer1
        {
          \Delta,\sigma,\sigma_w,\Lambda\vdash~
          $\vhdle|falling|$~\mathrm{ss}
          \xrightarrow{ss_\downarrow}
          \sigma'_w,\Lambda'
        }
      \end{prooftree}} \\
  \end{tabular}
\end{table}

\subsubsection{Rst block statement}

\begin{table}[H]
  \begin{tabular}{l}
    {\fontsize{10}{13}\selectfont\textsc{RstDefault}} \\    
    {\begin{prooftree}

        % Evaluates ss.
        \hypo{\Delta,\sigma,\sigma_w,\Lambda\vdash\mathrm{ss'}\xrightarrow{ss_f}\sigma'_w,\Lambda'}
        
        % Conclusion.
        \infer1
        [$f\neq{}i$]
        {
          \Delta,\sigma,\sigma_w,\Lambda\vdash~
          $\vhdle|rst|$~\mathrm{ss}~\mathrm{ss'}
          \xrightarrow{ss_f}
          \sigma'_w,\Lambda'
        }
      \end{prooftree}} \\
  \end{tabular}
  \begin{tabular}{l}
    {\fontsize{10}{13}\selectfont\textsc{RstExec}} \\    
    {\begin{prooftree}

        % Evaluates ss.
        \hypo{\Delta,\sigma,\sigma_w,\Lambda\vdash\mathrm{ss}\xrightarrow{ss_i}\sigma'_w,\Lambda'}
        
        % Conclusion.
        \infer1
        {
          \Delta,\sigma,\sigma_w,\Lambda\vdash~
          $\vhdle|rst|$~\mathrm{ss}~\mathrm{ss'}
          \xrightarrow{ss_i}
          \sigma'_w,\Lambda'
        }
      \end{prooftree}} \\
  \end{tabular}
\end{table}

\subsubsection{Composition of sequential statements and null statement}

\begin{table}[H]
  \begin{tabular}{l}
    {\fontsize{10}{13}\selectfont\textsc{SeqStmt}} \\
    {\begin{prooftree}
        
        % Evaluates ss.
        \hypo{\Delta,\sigma,\sigma_w,\Lambda\vdash\mathrm{ss}\xrightarrow{ss}\sigma'_w,\Lambda'}

        % Evaluates ss'.
        \hypo{\Delta,\sigma,\sigma'_w,\Lambda'\vdash\mathrm{ss'}\xrightarrow{ss}\sigma''_w,\Lambda''}

        % Conclusion.
        \infer2{
          \Delta,\sigma,\sigma_w,\Lambda\vdash\mathrm{ss}\mathtt{;}~\mathrm{ss}'\xrightarrow{ss}\sigma''_w,\Lambda''
        }
      \end{prooftree}} \\
  \end{tabular}
  \begin{tabular}{l}
    {\fontsize{10}{13}\selectfont\textsc{NullStmt}} \\
    {\begin{prooftree}        

        % Conclusion.
        \infer0{
          \Delta,\sigma,\sigma_w,\Lambda\vdash\mathtt{null}\xrightarrow{ss}\sigma_w,\Lambda
        }
      \end{prooftree}} \\
  \end{tabular}
\end{table}

\subsection{Evaluation of expressions}
\label{subsec:expr-rules}

% Nat and Bool literals.

\begin{table}[H]
  \begin{tabular}{l}
    {\fontsize{10}{13}\selectfont\textsc{Nat}} \\
    {\begin{prooftree}
        \infer0
        [{
          \begin{tabular}{@{}l}
            $\mathrm{n}\in\mathbb{N}$ \\
            $\mathrm{n}\le\mathtt{NATMAX}$ \\
          \end{tabular}
        }]
        {
          \Delta,\sigma,\Lambda\vdash\mathrm{n}\xrightarrow{e}n
        }
      \end{prooftree}} \\
  \end{tabular}
  \begin{tabular}{l}
    {\fontsize{10}{13}\selectfont\textsc{False}} \\
    {\begin{prooftree}
        \infer0
        {
          \Delta,\sigma,\Lambda\vdash\mathtt{false}\xrightarrow{e}\bot
        }
      \end{prooftree}} \\
  \end{tabular}
  \begin{tabular}{l}
    {\fontsize{10}{13}\selectfont\textsc{True}} \\
    {\begin{prooftree}
        \infer0
        {
          \Delta,\sigma,\Lambda\vdash\mathtt{true}\xrightarrow{e}\top
        }
      \end{prooftree}} \\
  \end{tabular}

\end{table}

% Aggregate.

\begin{table}[H]
  \centering
  \begin{tabular}{l}
    {\fontsize{10}{13}\selectfont\textsc{Aggreg}} \\
    {\begin{prooftree}

        % Evaluates e.
        \hypo{\Delta,\sigma,\Lambda\vdash\mathrm{e}_i\xrightarrow{e}v_i}

        % Conclusion.
        \infer1
        [$i=1,\dots,n$]
        {
          \Delta,\sigma,\Lambda\vdash\mathtt{(}\mathrm{e}_1,\dots,\mathrm{e}_n\mathtt{)}\xrightarrow{e}(v_1,\dots,v_n)
        }
      \end{prooftree}} \\
  \end{tabular}
\end{table}

% Name expressions.

\begin{table}[H]
  \centering
  \begin{tabular}{l}
    {\fontsize{10}{13}\selectfont\textsc{Sig}} \\
    {\begin{prooftree}
        \infer0
        [{
          \begin{tabular}{@{}l}
            $\mathrm{id}_s\in{}Sigs(\Delta)\cup{}Ins(\Delta)$ \\
          \end{tabular}
        }]
        {
          \Delta,\sigma,\Lambda\vdash\mathrm{id}_s\xrightarrow{e}\sigma(\mathrm{id}_s)
        }
      \end{prooftree}} \\
  \end{tabular}
  \begin{tabular}{@{}l}
    {\fontsize{10}{13}\selectfont\textsc{Var}} \\
    {\begin{prooftree}
        \infer0
        [{
          \begin{tabular}{@{}l}
            $\mathrm{id}_v\in\Lambda$ \\
            $\Lambda(\mathrm{id}_v)=(T,v)$ \\
          \end{tabular}
        }]
        {
          \Delta,\sigma,\Lambda\vdash\mathrm{id}_v\xrightarrow{e}v
        }
      \end{prooftree}} \\
    \end{tabular}
\end{table}

\begin{table}[H]
  \centering
  \begin{tabular}{@{}l}
    {\fontsize{10}{13}\selectfont\textsc{Gen}} \\
    {\begin{prooftree}
      \infer0
      [{
        \begin{tabular}{@{}l}
          $\mathrm{id}_g\in{}Gens(\Delta)$ \\
          $\Delta(\mathrm{id}_g)=(T,v)$ \\
        \end{tabular}
      }]
      {
        \Delta,\sigma,\Lambda\vdash\mathrm{id}_g\xrightarrow{e}v
      }
    \end{prooftree}} \\
  \end{tabular}
  % Idx name expressions.
  \begin{tabular}{@{}l}
    {\fontsize{10}{13}\selectfont\textsc{IdxSig}} \\
    {\begin{prooftree}

        % Evaluates e_i.
        \hypo{\Delta,\sigma,\Lambda\vdash\mathrm{e}_i\xrightarrow{e}v_i}

        % Well-typed v_i.
        \hypo{v_i\in_c{}\mathtt{nat}(n,m)}
        
        % Conclusion.
        \infer2
        [{
          \begin{tabular}{@{}l}
            $\mathrm{id}_s\in{}Sigs(\Delta)\cup{}Ins(\Delta)$ \\
            $\Delta(\mathrm{id}_s)=\mathtt{array(}T,n,m\mathtt{)}$ \\
            $i=v_i~\mathtt{mod}~n$ \\
          \end{tabular}
        }]
        {
          \Delta,\sigma,\Lambda\vdash
          \mathrm{id}_s(\mathrm{e}_i)
          \xrightarrow{e}
          \mathtt{get\_at}(i,\sigma(\mathrm{id}_s))
        }
      \end{prooftree}} \\
  \end{tabular}
\end{table}

\begin{table}[H]
  \centering
  \begin{tabular}{@{}l}
    {\fontsize{10}{13}\selectfont\textsc{IdxVar}} \\
    {\begin{prooftree}
        
        % Evaluates e_i.
        \hypo{\Delta,\sigma,\Lambda\vdash\mathrm{e}_i\xrightarrow{e}v_i}

        % Well-typed v_i.
        \hypo{v_i\in_c{}\mathtt{nat}(n,m)}
        
        % Conclusion.
        \infer2
        [{
          \begin{tabular}{@{}l}
            $\mathrm{id}_v\in\Lambda$ \\
            $\Lambda(\mathrm{id}_v)=(\mathtt{array(}T,n,m\mathtt{)},v)$ \\
            $i=v_i~\mathtt{mod}~n$ \\
          \end{tabular}
        }]
        {
          \Delta,\sigma,\Lambda\vdash
          \mathrm{id}_v(\mathrm{e}_i)
          \xrightarrow{e}
          \mathtt{get\_at}(i,v)
        }
      \end{prooftree}} \\
  \end{tabular}
\end{table}

where \texttt{get\_at}$(i,a)$ is a function returning the $i$-th
element of array $a$.

% Natural operators.

\begin{table}[H]
  \begin{tabular}{@{}l}
    {\fontsize{10}{13}\selectfont\textsc{NatAdd}} \\
    {\begin{prooftree}

        % Evaluates e and e'.
        \hypo{\Delta,\sigma,\Lambda&\vdash\mathrm{e}\xrightarrow{e}v}

        \hypo{\Delta,\sigma,\Lambda&\vdash\mathrm{e'}\xrightarrow{e}v'}

        % Conclusion.
        \infer2
        [$v+_{\mathbb{N}}v'\le\mathtt{NATMAX}$]
        {
          \Delta,\sigma,\Lambda\vdash\mathrm{e}~+~\mathrm{e'}\xrightarrow{e}v~+_{\mathbb{N}}~v'
        }
      \end{prooftree}} \\
  \end{tabular}
\end{table}

\begin{table}[H]
  \begin{tabular}{@{}l}
    {\fontsize{10}{13}\selectfont\textsc{NatSub}} \\
    {\begin{prooftree}

        % Evaluates e and e'.
        \hypo{\Delta,\sigma,\Lambda&\vdash\mathrm{e}\xrightarrow{e}v}
        \hypo{\Delta,\sigma,\Lambda&\vdash\mathrm{e'}\xrightarrow{e}v'}

        % Conclusion.
        \infer2
        [$v\ge{}v'$]
        {
          \Delta,\sigma,\Lambda\vdash\mathrm{e}~-~\mathrm{e'}\xrightarrow{e}v~-_{\mathbb{N}}~v'
        }
      \end{prooftree}} \\
  \end{tabular}
\end{table}

\begin{table}[H]
  \begin{tabular}{@{}l}
    {\fontsize{10}{13}\selectfont\textsc{OrdOp}} \\
    
    {\begin{prooftree}

        % Evaluates e and e'.
        \hypo{\Delta,\sigma,\Lambda&\vdash\mathrm{e}\xrightarrow{e}v}
        \hypo{\Delta,\sigma,\Lambda&\vdash\mathrm{e'}\xrightarrow{e}v'}

        % Conclusion.
        \infer2
        [$\mathrm{op}_{ordn}\in{}\{<,\le,>,\ge\}$]
        {
          \Delta,\sigma,\Lambda\vdash\mathrm{e}~\mathrm{op}_{ordn}~\mathrm{e'}\xrightarrow{e}v~op_{ord\mathbb{N}}~v'
        }
      \end{prooftree}} \\
  \end{tabular}
\end{table}

% Boolean operators.

\begin{table}[H]
  \begin{tabular}{@{}l}
    {\fontsize{10}{13}\selectfont\textsc{BoolBinOp}} \\
    {\begin{prooftree}

        % Evaluates e adn e'.
        \hypo{\Delta,\sigma,\Lambda&\vdash\mathrm{e}\xrightarrow{e}v}
        \hypo{\Delta,\sigma,\Lambda&\vdash\mathrm{e'}\xrightarrow{e}v'}

        % Conclusion.
        \infer2
        [$\mathrm{op}_{bool}\in{}\{\mathtt{and},\mathtt{or}\}$]
        {
          \Delta,\sigma,\Lambda\vdash
          \mathrm{e}~\mathrm{op}_{bool}~\mathrm{e'}
          \xrightarrow{e}
          v~op_{\mathbb{B}}~v'
        }
      \end{prooftree}} \\
  \end{tabular}
  \begin{tabular}{@{}l}
    {\fontsize{10}{13}\selectfont\textsc{NotOp}} \\    
    {\begin{prooftree}

        % Evaluates e.
        \hypo{\Delta,\sigma,\Lambda\vdash\mathrm{e}\xrightarrow{e}v}
        
        % Conclusion.
        \infer1
        {
          \Delta,\sigma,\Lambda\vdash~$\vhdle|not|$~\mathrm{e}\xrightarrow{e}\lnot{}v
        }
      \end{prooftree}} \\
  \end{tabular}
\end{table}

% Equality and difference.

\begin{table}[H]
  {\fontsize{10}{13}\selectfont\textsc{EqOp}}
  
  \begin{prooftree}

    % Evaluates e.
    \hypo{\Delta,\sigma,\Lambda\vdash\mathrm{e}\xrightarrow{e}v}

    % Evaluates e'.
    \hypo{\Delta,\sigma,\Lambda\vdash\mathrm{e'}\xrightarrow{e}v'}

    % Conclusion.
    \infer2
    {
      \Delta,\sigma,\Lambda\vdash\mathrm{e}~=~\mathrm{e'}\xrightarrow{e}eq(v,v')
    }
  \end{prooftree}
\end{table}

\begin{table}[H]
  {\fontsize{10}{13}\selectfont\textsc{DiffOp}}
  
  \begin{prooftree}

    % Evaluates e.
    \hypo{\Delta,\sigma,\Lambda\vdash\mathrm{e}~=~\mathrm{e'}\xrightarrow{e}v}

    % Conclusion.
    \infer1
    {
      \Delta,\sigma,\Lambda\vdash
      \mathrm{e}~\neq~\mathrm{e'}
      \xrightarrow{e}\lnot{}v
    }
  \end{prooftree}
\end{table}

where $eq$ is the equality relation established for all types defined
in the semantics.

\begin{table}[H]
  {\fontsize{10}{13}\selectfont\textsc{Parenth}}
  
  \begin{prooftree}

    % Evaluates e.
    \hypo{\Delta,\sigma,\Lambda\vdash\mathrm{e}\xrightarrow{e}v}

    % Conclusion.
    \infer1
    {
      \Delta,\sigma,\Lambda\vdash\mathtt{(}\mathrm{e}\mathtt{)}\xrightarrow{e}v
    }
  \end{prooftree}
\end{table}

%%% Local Variables:
%%% mode: latex
%%% TeX-master: "../../main"
%%% End:


\section{An example of full simulation}
\label{sec:ex-full-sim}
In this section, we will demonstrate the full simulation of a \hvhdl{}
top-level design on the example of Listing~\ref{lst:tl-design-ex}. The
aim here is to illustrate the formal rules of the \hvhdl{}
semantics. Listing~\ref{lst:tl-design-ex} is the result of the
transformation of the SITPN model presented in
Figure~\ref{fig:sitpn-ex-full-sim} into a \hvhdl{} design. To keep the
examples within a reasonable size, Listing~\ref{lst:tl-design-ex}, and
the other listings and derivation trees used in this section, refer to
the generic constants and ports of the transition and place designs by
their short names. See Table~\ref{tab:consts-and-sigs-ref} for a
correspondence between the short names and the full names of constants
and signals of the place and transition designs.

\begin{lstlisting}[language=VHDL,label={lst:tl-design-ex},
caption={[An example of \hvhdl{} design generated by \hilecop{}.] An example of \hvhdl{} top-level design generated by the \hilecop{} transformation.},framexleftmargin=1.5em,xleftmargin=2em,numbers=left,
numberstyle=\tiny\ttfamily]
design tl tla

-- Generic constants
$\emptyset$

-- Ports ($ports_{tl}$)
((in, $id_{c0}$, boolean), (out, $id_{f0}$, boolean), (out, $id_{a0}$, boolean))

-- Declared (internal) signals ($sigs_{tl}$)
(($id_{ft}$, boolean), ($id_{av}$, boolean), ($id_{rt}$, boolean),($id_m$, boolean))

-- Behavior ($cs_{tl}$)

-- Place component instance $id_p$
comp ($id_p$, place,
-- Generic map
((ian, 1), (oan, 1), (mm, 1)),

-- Input port map
((im, 1), (iaw(0), 1), (oat(0), 0), (oaw(0), 1), (itf(0), $id_{ft}$), (otf(0), $id_{ft}$))

-- Output port map
((oav(0), $id_{av}$), (pauths, open), (rtt(0), $id_{rt}$), (marked, $id_m$)))

||

-- Transition component instance $id_t$
comp ($id_t$, transition,
-- Generic map
((tt, 0), (ian, 1), (cn, 1)),

-- Input port map
((ic(0), $id_{c0}$), (A, 0), (B, 0), (iav(0), $id_{av}$), (rt(0), $id_{rt}$), (pauths(0), true)),

-- Output port map
((fired, $id_{ft}$)))

||

-- The marked process
process (marked, {clk}, $\emptyset$,
(rst ($id_{a0}$ <= false)
     (falling ($id_{a0}$ <= $id_m$ or false))))

||

-- The fired process
process (fired, {clk}, $\emptyset$,
(rst ($id_{f0}$ <= false)
     (rising ($id_{f0}$ <= $id_{ft}$ or false))))
\end{lstlisting}

\begin{figure}[H]
  \centering
  \includegraphics[keepaspectratio,width=.15\textwidth]{Figures/H-VHDL/sitpn-ex-full-sim}
  \caption[An SITPN model transformed into a \hvhdl{} top-level design.]{The SITPN model at the base of the generation of the top-level design presented in Listing~\ref{lst:tl-design-ex}.}
  \label{fig:sitpn-ex-full-sim}
\end{figure}

The following rule states that the full simulation of the \texttt{tl}
design (presented in Listing~\ref{lst:tl-design-ex}) over 1 clock
cycle yields the simulation trace $(\sigma_0::\sigma_1::\sigma_2)$.
The simulation over one clock cycle (the rightmost premise) yields a
trace composed of the two states: the state $\sigma_1$ at half the
clock period, and the state $\sigma_2$ at the end of the first
cycle. The full simulation happens in the context of the \hilecop's
design store $\mathcal{D}_\mathcal{H}$, the elaborated design
$\Delta$, an empty dimensioning function and an simulation environment
$E_p$. Here, $ports_{tl}$ is an alias for the list of ports of
\texttt{tl}, $sigs_{tl}$ for the list of internal signals of
\texttt{tl}, and $cs_{tl}$ for the behavior of \texttt{tl}. In what
follows, we will detail the premises of the \textsc{FullSim} rule.

\begin{figure}[H]
  \begin{prooftree}[template={\fontsize{11}{12}\selectfont\inserttext}]

    % Design elab.
    \hypo{}
    \ellipsis{}{}
    \infer1{$\mathcal{D}_\mathcal{H},\emptyset\vdash$ \vhdle|design tl| $\dots\xrightarrow{elab}\Delta,\sigma_e$}

    % Initialization.
    \hypo{}
    \ellipsis{}{}
    \infer1{$\mathcal{D}_\mathcal{H},\Delta,\sigma_e\vdash{}cs_{tl}\xrightarrow{init}\sigma_0$}
       
    % Simulation loop.
    \hypo{}
    \ellipsis{}{}
    \infer1{$\mathcal{D}_\mathcal{H},E_p,\Delta,1,\sigma_0\vdash{}cs_{tl}\rightarrow(\sigma_1::\sigma_2)$}
    
    \infer3[\fontsize{7}{10}\selectfont\textsc{FullSim}]{
        $\mathcal{D}_\mathcal{H},\Delta,\emptyset,E_p,1\vdash$      
        \vhdle|design tl tla| $\empty$ $ports_{tl}$ $sigs_{tl}$ $cs_{tl}$ $\xrightarrow{full}(\sigma_0::\sigma_1::\sigma_2)$
    }
  \end{prooftree}
  \label{fig:ex-full-sim-tl}
  \caption{The \textsc{FullSim} rule applied to the \texttt{tl} design.}
\end{figure}

\subsection{Elaboration of the \texttt{tl} design}
\label{sec:ex-elab-of-tl}

The following rule state the elaborated design $\Delta$ and the
default design state $\sigma_e$ are the result of the elaboration of
the \texttt{tl} design. From left to right in the premises of the
rule, the three premises pertain to the elaboration of the declarative
parts of the \texttt{tl} design, i.e. the generic constant declaration
list, the port declaration list and the internal signal declaration
list. The leftmost premise pertains to the elaboration of the behavior
of the \texttt{tl} design.

\begin{figure}[H]
  \begin{prooftree}[template={\fontsize{9}{12}\selectfont\inserttext}]

    % Gens elab.
    \hypo{}
    \ellipsis{}{}
    \infer1{$\emptyset,\emptyset\vdash{}gens_{tl}\xrightarrow{egens}\Delta_0$}

    % Ports elab.
    \hypo{}
    \ellipsis{}{}
    \infer1{ $\Delta_0,\emptyset\vdash$
      $ports_{tl}\xrightarrow{eports}\Delta_1,\sigma_{e1}$}

    % Sigs elab.
    \hypo{}
    \ellipsis{}{}
    \infer1{
      $\Delta_1,\sigma_{e1}\vdash$
      $sigs_{tl}\xrightarrow{esigs}\Delta_2,\sigma_{e2}$}

    % Behavior elab.
    \hypo{}
    \ellipsis{}{}
    \infer1{$\mathcal{D}_\mathcal{H},\Delta_2,\sigma_{e2}\vdash$
      $cs_{tl}\xrightarrow{ebeh}\Delta,\sigma_e$}
        
    \infer4[\fontsize{7}{10}\selectfont\textsc{DesignElab}]{
      {\fontsize{12}{12}\selectfont
      $\mathcal{D}_\mathcal{H},\emptyset\vdash$
      \vhdle|design tl tla| $gens_{tl}$ $ports_{tl}$ $sigs_{tl}$ $cs_{tl}$
      $\xrightarrow{elab}\Delta,\sigma_e$}
    }
  \end{prooftree}
  \caption{The \textsc{DesignElab} rule applied to the \texttt{tl} design.}
  \label{fig:ex-elab-tl}
\end{figure}

\subsubsection{Elaboration of the declarative parts}
\label{sec:ex-elab-decl}

The elaboration of the declarative parts populates the $Gens$, $Ins$,
$Outs$ and $Sigs$ sub-environments of the elaborated design
$\Delta$. Here is the content of the $Gens$, $Ins$, $Outs$ and $Sigs$
sub-environments of $\Delta_2$, where $\Delta_2$ is the partial
elaborated design after the elaboration of the declarative parts of
the \texttt{tl} design (passed as a parameter of third and the fourth
premises of the rule described in Figure~\ref{fig:ex-elab-tl}).

\begin{itemize}
\item $Gens(\Delta_2):=\emptyset$
\item $Ins(\Delta_2):=\{(id_{c0}, bool)\}$
\item $Outs(\Delta_2):=\{(id_{f0}, bool), (id_{a0}, bool)\}$
\item $Sigs(\Delta_2):=\{(id_{ft}, bool), (id_{av}, bool), (id_{rt}, bool), (id_m, bool)\}$
\end{itemize}

The top-level design generated by the \hilecop{} transformation all
have an empty list of generic constants (see
Chapter~\ref{chap:transformation} for more details about the
transformation). Also, all ports and internal signals are of the
Boolean type. Thus, there are no range constraint or index constraint
to solve here. The \texttt{boolean} type indication is simply
transformed into the $bool$ semantic type.

The elaboration of the declarative parts also gives a default value to
the signals in the signal store of the default design state
$\sigma_{e2}$, where $\sigma_{e2}$ is the partial default design state
at the end of the elaboration of the declarative parts of the
\texttt{tl} design (passed as a parameter of the third and the fourth
premises of the rule described in Figure~\ref{fig:ex-elab-tl}).  Here
is the content of the signal store $\mathcal{S}$ of $\sigma_{e2}$.

\begin{itemize}
\item
  $\mathcal{S}(\sigma_{e2}):=\{(id_{c0},\bot),(id_{f0},\bot),(id_{a0},\bot),(id_{ft},
  \bot), (id_{av}, \bot), (id_{rt}, \bot), (id_m, \bot)\}$
\end{itemize}

The default value associated to the $bool$ type is $\bot$, thus, all
signals of the \texttt{tl design} are initialized to $\bot$ in the
signal store of $\sigma_{e2}$.

\subsubsection{Elaboration of the behavioral part}
\label{sec:ex-elab-beh}

The behavior of the \texttt{tl} design contains two component
instantiation statements and two process statements. Each one of these
statements will be elaborated in sequence. First, we present the
elaboration of the \texttt{marked} process to illustrate the
elaboration of a process statement; then, we present the elaboration
of place component instance $id_p$ to illustrate the elaboration of a
component instantiation statement.

\paragraph{Elaboration of a process statement}

The following rule presents the elaboration of the \texttt{marked}
process defined in the behavior of the \texttt{tl} design.

\begin{figure}[H]
  \begin{prooftree}[template={\fontsize{11}{12}\selectfont\inserttext}]

    % Sigs elab.
    \infer0{
      $\Delta_2,\emptyset\vdash$
      $\emptyset\xrightarrow{evars}\emptyset$}

    % Behavior elab.
    \hypo{}
    \ellipsis{}{}
    \infer1[\fontsize{7}{10}\selectfont\textsc{WTRst}]{$\Delta_2,\sigma_{e2},\emptyset\vdash$
      $\mathtt{valid}_{ss}\Big($
      \begin{tabular}{@{}l@{}}
        $\mathtt{rst}~(id_{a0}\Leftarrow{}\mathtt{false})$ \\
        $(\mathtt{falling}(id_{a0}\Leftarrow{}id_m~\mathtt{or}~\mathtt{false}))$ \\
      \end{tabular}$\Big)$}
    
    \infer2[\fontsize{7}{10}\selectfont\textsc{PsElab}]{
      {\fontsize{12}{12}\selectfont
        $\mathcal{D}_\mathcal{H},\Delta_2,\sigma_{e2}\vdash$
        \vhdle|process(marked, {clk},| $\emptyset$, $\dots$)
        $\xrightarrow{ebeh}\Delta_2\cup(\mathtt{marked},\emptyset),\sigma_{e2}$}
    }
  \end{prooftree}
  \caption{The elaboration of the \texttt{marked} process defined in
    the behavior of the \texttt{tl} design.}
  \label{fig:ex-elab-marked-ps}
\end{figure}


The \texttt{marked} process is elaborated in the context
$\mathcal{D}_\mathcal{H},\Delta_2,\sigma_{e2}$ where $\Delta_2$ and
$\sigma_{e2}$ are the partially-built elaborated design and default
design state at a certain point of the elaboration of the behavioral
part of the \texttt{tl} design. The elaboration of a process statement
associates the process identifier to a local variable environment in
the $Ps$ sub-environment of the being-built elaborated design. The
local variable environment is built out of the variable declaration
list of the process. Here, the \texttt{marked} process has an empty
variable declaration list. Thus, the binding
(\texttt{marked},$\emptyset$) is added in the $Ps$ sub-environment of
$\Delta_2$.  % As other concurrent statements could have been
% elaborated before the \texttt{marked} process, the $\Delta_2$ and
% $\sigma_{e2}$ of the above rule are not necessarily equal the
% $\Delta_2$ and $\sigma_{e2}$ of the rightmost premise of
% Figure~\ref{fig:ex-elab-tl}.
% The elaboration of the behavior of a design only acts on the $Ps$ and
% $Comps$ sub-environments of the being-built elaborated design, and
% only of the component store of the being-built default design
% state. Thus, we have $Gens(\Delta_2)=Gens(Delta)$,
% $Ins(\Delta_2)=Ins(\Delta)$, $Outs(\Delta_2)=Outs(\Delta)$ and
% $Sigs(\Delta_2)=Sigs(\Delta)$ where $\Delta$ is the final elaborated
% design. Also, we have $\mathcal{S}(\sigma_{e2})=\mathcal{S}(\sigma_e)$
% where $\sigma_e$ is the final default design state.

The elaboration of process statement also performs static
type-checking on the process statement body leveraging the
$\mathtt{valid}_{ss}$ relation. The following rule details the static
type-checking of the statement body of the \texttt{marked} process
(rightmost premise of the rule presented in
Figure~\ref{fig:ex-elab-marked-ps}). To keep the example within a
reasonable size, we do not detail the context of all rules. We
annotate the rule names to describe the side conditions associated to
a derivation.

\begin{figure}[H]
  \begin{prooftree}[template={\fontsize{11}{12}\selectfont\inserttext}]

    % First rst block
    \infer0[\fontsize{7}{10}\selectfont\textsc{False}]{$\mathtt{false}\xrightarrow{e}\bot$}
    \infer0[\fontsize{7}{10}\selectfont\textsc{IsBool}]{$\bot\in_c{}bool$}
    \infer2[\fontsize{7}{10}\selectfont\textsc{WTSig}$^1$]{$\vdash{}\mathtt{valid}_{ss}(id_{a0}\Leftarrow{}\mathtt{false})$}

    % Second rst block
    \infer0[\fontsize{7}{10}\selectfont\textsc{Sig}$^{2}$]{$id_m\xrightarrow{e}\bot$}
    \infer0[\fontsize{7}{10}\selectfont\textsc{False}]{$\mathtt{false}\xrightarrow{e}\bot$}
    \infer2[\fontsize{7}{10}\selectfont\textsc{BoolBinOp}]{$\vdash{}id_m~\mathtt{or}~\mathtt{false}\xrightarrow{e}\bot$}
    \infer0[\fontsize{7}{10}\selectfont\textsc{IsBool}]{$\bot\in_c{}bool$}
    \infer2[\fontsize{7}{10}\selectfont\textsc{WTSig}$^{1}$]{$\vdash{}\mathtt{valid}_{ss}(id_{a0}\Leftarrow{}id_m~\mathtt{or}~\mathtt{false})$}
    \infer1[\fontsize{7}{10}\selectfont\textsc{WTFalling}]{$\vdash{}\mathtt{valid}_{ss}(\mathtt{falling}(id_{a0}\Leftarrow{}id_m~\mathtt{or}~\mathtt{false}))$}
    
    % Conclusion
    \infer2[\fontsize{7}{10}\selectfont\textsc{WTRst}]{$\Delta_2,\sigma_{e2},\emptyset\vdash$
      $\mathtt{valid}_{ss}\Big($
      \begin{tabular}{@{}l@{}}
        $\mathtt{rst}~(id_{a0}\Leftarrow{}\mathtt{false})$ \\
        $(\mathtt{falling}(id_{a0}\Leftarrow{}id_m~\mathtt{or}~\mathtt{false}))$ \\
      \end{tabular}$\Big)$}
    
  \end{prooftree}
  \caption{Static type-checking of the \texttt{marked} process
    statement body.}
  \label{fig:ex-wt-marked-ps}
\end{figure}

\noindent{}where:
\begin{enumerate}
\item $\Delta_2(id_{a0})=bool$
\item $\sigma_{e2}(id_m)=\bot$
\end{enumerate}

At the end of the elaboration of the \texttt{tl} design's behavior,
the $Ps$ sub-environment of $Delta$ is as follows:
$Ps(\Delta):=\{(\mathtt{marked},\emptyset),
(\mathtt{fired},\emptyset)\}$

\paragraph{Elaboration of a component instantiation statement}

The rule of Figure~\ref{fig:ex-elab-pci-idp} presents the elaboration
of the place component instance $id_p$ belonging to the behavior of
the \texttt{tl} design.

\begin{figure}[H]
  \begin{prooftree}[template={\fontsize{9}{10}\selectfont\inserttext}, separation=4pt]

    % Dimensioning function
    \hypo{}
    \ellipsis{}{}
    \infer1{$\emptyset\vdash{}g_p\xrightarrow{emapg}\mathcal{M}$}

    % Elaboration of the place design
    \hypo{}
    \ellipsis{}{}
    \infer1[\fontsize{7}{10}\selectfont]{
      $\mathcal{D}_\mathcal{H},\mathcal{M}\vdash\mathtt{design}~\mathtt{place}\dots\xrightarrow{elab}\Delta_p,\sigma_p$
    }

    % Valid ipmap
    \hypo{}
    \ellipsis{}{}
    \infer1{
      $\Delta_2,\Delta_p,\sigma_{e2}\vdash{}\mathtt{valid}_{ipm}(i_p)$
    }

    % Valid opmap
    \hypo{}
    \ellipsis{}{}
    \infer1{
      $\Delta_2,\Delta_p\vdash{}\mathtt{valid}_{opm}(o_p)$
    }
    
    % Conclusion
    \infer4[\fontsize{7}{10}\selectfont\textsc{CompElab}$^{1}$]{
      {\fontsize{11}{13}\selectfont
        $\mathcal{D}_\mathcal{H},\Delta_2,\sigma_{e2}\vdash$
        $\mathtt{comp}~(id_p,\mathtt{place},g_p,i_p,o_p)\xrightarrow{ebeh}$
        $\Delta_2\cup(id_p,\Delta_p),\sigma_{e2}\cup(id_p,\sigma_p)$}
    }
    
  \end{prooftree}
  \caption{The elaboration of the $id_p$ component instance defined in
    the behavior of the \texttt{tl} design.}
  \label{fig:ex-elab-pci-idp}
\end{figure}

\noindent{}where:

\begin{enumerate}
\item
  \begin{tabular}{@{}l@{}}
    $id_p\notin\Delta_2$ \\
    $id_p\notin{}\sigma_{e2}$ \\
    $\mathtt{place}\in\mathcal{D}_\mathcal{H}$ \\
    $\mathcal{M}\subseteq{}Gens(\Delta_p)$ \\
  \end{tabular}
\end{enumerate}

The elaboration of a component instantiation statement is divided in
three parts. First, a dimensioning function is built out of the
generic map of the component instance. Figure~\ref{fig:ex-emapg} shows
a part of the creation of the dimensioning functioning $\mathcal{M}$
from the generic map of the component instance $id_p$. Basically, the
elaboration of a generic map is a transformation from a syntactic
construct, i.e. the generic map, into a function, i.e. the
dimensioning function $\mathcal{M}$. For each association of the
generic map, the elaboration checks that the actual part of the
association is a locally static expression (see
Section~\ref{subsubsec:loc-static}).

\begin{figure}[H]
  \begin{prooftree}[template={\fontsize{10}{13}\selectfont\inserttext}, separation=10pt]

    % Elab of assocg
    \infer0[\fontsize{7}{10}\selectfont\textsc{LSENat}]{$SE_l(1)$}
    \infer0[\fontsize{7}{10}\selectfont\textsc{Nat}]{$1\xrightarrow{e}1$}
    \infer2[\fontsize{7}{10}\selectfont\textsc{AssocGElab}$^1$]{$\emptyset\vdash{}(\mathtt{ian},1)\xrightarrow{emapg}$
      $\{(\mathtt{ian},1)\}$}

    % Ind. call
    \hypo{}
    \ellipsis{}{}
    \infer1[\fontsize{7}{10}\selectfont\textsc{GMElab}]{$\{(\mathtt{ian},1)\}\vdash{}(\mathtt{oan},1),(\mathtt{mm},1)\xrightarrow{emapg}$
      $\{(\mathtt{ian},1),(\mathtt{oan},1),(\mathtt{mm},1)\}$}
    
    % Conclusion 
    \infer2[\fontsize{7}{10}\selectfont\textsc{GMElab}]{$\emptyset\vdash{}(\mathtt{ian},1),(\mathtt{oan},1),(\mathtt{mm},1)\xrightarrow{emapg}$
      $\{(\mathtt{ian},1),(\mathtt{oan},1),(\mathtt{mm},1)\}$}
    
  \end{prooftree}
  \caption{The elaboration of the generic map of the $id_p$ component
    instance defined in the behavior of the \texttt{tl} design.}
  \label{fig:ex-emapg}
\end{figure}

\noindent{}where:
\begin{enumerate}
\item $\mathtt{ian}\notin\emptyset$
\end{enumerate}

The second step of the elaboration of a component instance is to
retrieve from the design store the design associated with the
component instance, and to elaborate this design. Here, the design
store is the \hilecop{} design store $\mathcal{D}_\mathcal{H}$, and
the design associated with $id_p$ is the \texttt{place} design. The
dimensioning function $\mathcal{M}$ sets the value of the generic
constants declared in the \texttt{place} design.  The full code of
\texttt{place} design is available in
Appendix~\ref{app:place-design}. In
Figures~\ref{fig:ex-elab-version-of-idp} and
\ref{fig:ex-default-state-of-idp}, we give the elaborated design
$\Delta_p$ and the default design state $\sigma_p$ resulting of the
elaboration of the \texttt{place} design given the dimensioning
function $\mathcal{M}$.

\begin{figure}[H]
\begin{tabular}{@{}rl@{}}
  $\Delta_p:=\{$ & \\
               & \begin{tabular}{@{}ll@{}}
                   $Gens := $ & $\{(\mathtt{ian}, (nat(0,\mathtt{NATMAX}, 1)),$ \\
                              & $(\mathtt{oan}, (nat(0,\mathtt{NATMAX}), 1))$ \\
                              & $(\mathtt{mm}, (nat(0,\mathtt{NATMAX}), 1))\}$ \\
                 \end{tabular} \\
               & \\
               & \begin{tabular}{@{}ll@{}}
                   $Ins :=$ & $\{(\mathtt{im}, nat(0,1)),$ \\
                            & $(\mathtt{iaw}, array(nat(0,255),0,0)),$ \\
                            & $(\mathtt{oat}, array(nat(0,2),0,0)),$ \\
                            & $(\mathtt{oaw}, array(nat(0,255),0,0)),$ \\
                            & $(\mathtt{itf}, array(bool,0,0)),$ \\
                            & $(\mathtt{otf}, array(bool,0,0))\},$ \\
                 \end{tabular} \\
               & \\
               & \begin{tabular}{@{}ll@{}}
                   $Outs :=$ & $\{(\mathtt{oav}, array(bool,0,0)),$ \\
                             & $(\mathtt{pauths}, array(bool,0,0)),$ \\
                             & $(\mathtt{rtt}, array(bool,0,0))\}$ \\
                 \end{tabular} \\
                 & \\
               & \begin{tabular}{@{}ll@{}}
                   $Sigs :=$ & $\{(\mathtt{sits}, nat(0,1)),$ \\
                             & $(\mathtt{sm}, nat(0,1)),$ \\
                             & $(\mathtt{sots}, nat(0,1))\},$ \\
                 \end{tabular} \\
               & \\
               & \begin{tabular}{@{}ll@{}}
                   $Ps :=$ & $\{(\mathtt{input\_tokens\_sum}, \{("v\_internal\_its", (nat(0,1), 0))\}),$ \\
                           & $(\mathtt{output\_tokens\_sum}, \{("v\_internal\_ots", (nat(0,1), 0))\})\}$ \\
                           & $(\mathtt{priority\_evaluation}, \{("v\_saved\_ots", (nat(0,1), 0))\})\}$ \\
                 \end{tabular} \\
               & \\
               & $Comps := \emptyset$ \\
  $\}$ &  \\
\end{tabular}

\caption[An elaborated version of the place design with a given
dimensioning function.]{An elaborated version of the place design
  built with the dimensioning function deduced from the generic map of
  the component instance $id_p$.}
\label{fig:ex-elab-version-of-idp}
\end{figure}

In $\Delta_p$, all the types associated with ports and internal
signals of the \texttt{place} design have been \emph{resolved};
i.e. the expressions qualifying the bounds of the range and index
constraints in type indications have been evaluated.  For example,
\texttt{array(boolean, 0, input_arcs_number-1)} is the type indication
associated with the \texttt{input_transitions_fired} input port
(i.e. \texttt{itf}) defined in the port clause of the \texttt{place}
design. The dimensioning function $\mathcal{M}$ sets the value of the
\texttt{input_arcs_number} (i.e. \texttt{ian}) generic constant to
1. After the elaboration, the type indication \texttt{array(boolean,
  0, input_arcs_number-1)} is thus transformed into the semantic type
$array(bool, 0, 0)$. Thus, we have
$\Delta_p(\mathtt{itf})=array(bool,0,0)$ in the resulting $\Delta_p$.

Figure~\ref{fig:ex-default-state-of-idp} shows the default design
state $\sigma_p$ of $\Delta_p$.

\begin{figure}[H]
  \begin{tabular}{@{}rl@{}}
    $\sigma_p:=\{$ & \\
                   & \begin{tabular}{@{}ll@{}}
                       $\mathcal{S} :=$ & $\{(\mathtt{im}, (0)),$ $(\mathtt{iaw}, (0)),$ $(\mathtt{oat}, (0)),$\\
                                        & $(\mathtt{oaw}, (0)),$ $(\mathtt{itf}, (\bot)),$ $(\mathtt{otf}, (\bot)),$ \\
                                        & $(\mathtt{oav}, (\bot)),$ $(\mathtt{pauths},(\bot)),$ $(\mathtt{rtt},(\bot))$ \\
                                        & $(\mathtt{sits}, 0),$ $(\mathtt{sm}, 0),$ $(\mathtt{sots}, 0)\},$ \\
                     \end{tabular} \\
                   & \\
                   & $\mathcal{C}:=\emptyset$ \\
                   & $\mathcal{E}:=\emptyset$ \\
    $\}$ &  \\
  \end{tabular}

\caption[An example of default design state for the place design.]{The
  default design state $\sigma_p$ of the elaborated design $\Delta_p$.}
\label{fig:ex-default-state-of-idp}
\end{figure}

The component store of design state $\sigma_p$ is empty as there are
no component instantiation statements in the behavior of the place
design. The same stands for the $Comps$ sub-environment of $\Delta_p$.
Also, the set of events of a default design state is always empty.

The final step in the elaboration of a component instantiation
statement is to check the well-formedness and the well-typedness of
the input and output port maps. The $\mathtt{valid}_{ipm}$ and
$\mathtt{valid}_{opm}$ relations, defined in
Section~\ref{subsec:valid-pm}, state the validity of the port maps.
The rule of Figure presents a part of the construction of the
$valid_{opm}$ relation applied to the output port map of the place
component instance $id_p$. Note that $\Delta_p$ is necessary to check
the validity of the output port map of $id_p$, as it holds the
correspondence between port identifiers and port types.

\begin{figure}[H]
  \centering
  \begin{prooftree}[template={\fontsize{10}{13}\selectfont\inserttext}]

    % Elab of assocop
    \infer0[\fontsize{7}{10}\selectfont\textsc{LSENat}]{$SE_l(0)$}
    \infer0[\fontsize{7}{10}\selectfont\textsc{Nat}]{$0\xrightarrow{e}0$}
    \infer0[\fontsize{7}{10}\selectfont\textsc{IsCNat}]{$0\in_c{}nat(0,0)$}
    \infer3[$^1$]{
      $\Delta_2,\Delta_p,\emptyset,\emptyset\vdash$
      $(\mathtt{oav}(0), id_{av})$
      $\xrightarrow{list_{opm}}$
      $\{(\mathtt{oav}, 0)\},\{id_{av}\}$
    }

    % Ind. call
    \hypo{}
    \ellipsis{\fontsize{7}{10}\selectfont\textsc{ListOPMCons}$_B$}{}

    % List OPM
    \infer2[\fontsize{7}{10}\selectfont\textsc{ListOPMCons}$_A$]{
      $\Delta_2,\Delta_p,\emptyset,\emptyset\vdash$
      \begin{tabular}{@{}l@{}}
        $(\mathtt{oav}(0), id_{av}),$ $(\mathtt{pauths}, \mathtt{open})$ \\
        $(\mathtt{rtt}(0), id_{rt}),$ $(marked, id_m)$ \\
      \end{tabular}
      $\xrightarrow{list_{opm}}$
      \begin{tabular}{@{}l@{}}
        $\{(\mathtt{oav}, 0), \mathtt{pauths},$ \\
        $(\mathtt{rtt}, 0), \mathtt{marked}\}$ \\
      \end{tabular},
      \begin{tabular}{@{}l@{}}
        $\{id_{av}, id_{rt}, id_m\}$ \\
      \end{tabular}
    }
    
    % Conclusion 
    \infer1[\fontsize{7}{10}\selectfont\textsc{ValidOPM}]{
      $\Delta_2,\Delta_p\vdash$
      $\mathtt{valid}_{opm}\Big($
      \begin{tabular}{@{}l@{}}
        $(\mathtt{oav}(0), id_{av}),$ $(\mathtt{pauths}, \mathtt{open})$ \\
        $(\mathtt{rtt}(0), id_{rt}),$ $(\mathtt{marked}, id_m)$ \\
      \end{tabular}
      $\Big)$
    }
    
  \end{prooftree}
  
  \vspace{40pt}
  
  \begin{prooftree}[template={\fontsize{10}{13}\selectfont\inserttext}, separation=10pt]
    % Ind. call
    \hypo{}
    \ellipsis{}{}
    \infer1[\fontsize{7}{10}\selectfont\textsc{ListOPMCons}$_B$]{
      {$\Delta_2,\Delta_p,\{(\mathtt{oav},0)\},\{id_{av}\}\vdash$
        \begin{tabular}{@{}l@{}}
          $(\mathtt{pauths}, \mathtt{open}),$ \\
          $(\mathtt{rtt}(0), id_{rt}),$ \\
          $(marked, id_m)$ \\
        \end{tabular}
        $\xrightarrow{list_{opm}}$
        \begin{tabular}{@{}l@{}}
          $\{(\mathtt{oav}, 0),$ \\
          $\mathtt{pauths},$ \\
          $(\mathtt{rtt}, 0),$ \\
          $\mathtt{marked}\}$ \\
        \end{tabular},
        \begin{tabular}{@{}l@{}}
          $\{id_{av}, id_{rt}, id_m\}$ \\
        \end{tabular}}
    }

  \end{prooftree}
  
  \caption{An example of validity checking performed on the output
    port map of the place component instance $id_p$. The bottom proof
    tree represents the top-right premise of the top proof tree.}
  \label{fig:ex-valid-opm}
\end{figure}

\noindent{}where:
\begin{enumerate}
\item
  \begin{tabular}{@{}l@{}}
    $\Delta_p(\mathtt{oav})=array(bool,0,0)$ \\
    $\Delta_2(id_{av})=bool$ \\
    $\mathtt{oav}\notin\emptyset$ and $(\mathtt{oav},0)\notin\emptyset$ \\
    $id_{av}\notin\emptyset$ \\
  \end{tabular}
\end{enumerate}

At the end of the elaboration of the \texttt{tl} design's behavior,
the $Comps$ sub-environment of $\Delta$ is as follows:
$Comps(\Delta):=\{(id_p,\Delta_p), (id_t,\Delta_t)\}$. Here,
$\Delta_t$ represents the elaborated version of the transition design
obtained from the elaboration of the transition component instance
$id_t$.

Also, at the end of the elaboration, the component store of $\sigma_e$
is as follows:
$\mathcal{C}(\sigma_e):=\{(id_p,\sigma_p),(id_t,\sigma_t)\}$. Here,
$\sigma_t$ is the default design state of the transition component
instance $id_t$.

\subsection{Simulation of the tl design}
\label{sec:ex-tl-sim}

Let us now present the rules pertaining to the simulation of the
\texttt{tl} design, that is, pertaining to the execution of the
\texttt{tl} design's behavior with respect to our formal simulation
algorithm.

\subsubsection{Initialization}
\label{sec:ex-init-tl}

The rule of Figure~\ref{fig:ex-init-tl} presents the initialization
phase in the proceeding of the simulation of the \texttt{tl}
design. The initialization phase builds the initial state of the
simulation. The first step of the initialization, formalized by the
$runinit$ relation, runs the processes and the internal behavior of
component instances exactly once (with the execution of the first part
of reset blocks). Then, a stabilization phase follows.

\begin{figure}[H]
  \centering
  \begin{prooftree}[template={\fontsize{12}{13}\selectfont\inserttext}]

    % Runinit
    \hypo{}
    \ellipsis{}{}
    \infer1[]{
      $\mathcal{D}_\mathcal{H},\Delta,\sigma_e\vdash$
      $cs_{tl}$
      $\xrightarrow{runinit}$
      $\sigma'$
    }
    
    % Stabilization
    \hypo{}
    \ellipsis{}{}
    \infer1[]{
      $\mathcal{D}_\mathcal{H},\Delta,\sigma_e\vdash$
      $cs_{tl}$
      $\xrightarrow{\rightsquigarrow}$
      $\sigma_0$
    }
    
    % Conclusion 
    \infer2[\fontsize{7}{10}\selectfont\textsc{Init}]{
      $\mathcal{D}_\mathcal{H},\Delta,\sigma_e\vdash$
      $cs_{tl}$
      $\xrightarrow{init}$
      $\sigma_0$
    }
    
  \end{prooftree}
  
  \caption{The initialization phase, first step of the simulation of
    the \texttt{tl} design.}
  \label{fig:ex-init-tl}
\end{figure}

The rule in Figure~\ref{fig:ex-runinit-tl} presents the execution of
the \texttt{tl} design's behavior during the $runinit$ phase. The
\texttt{tl} design's behavior is defined by the composition of
concurrent statements. Here, the \texttt{marked} process is at the
head of the behavior, whereas it is not the case in
Listing~\ref{lst:tl-design-ex}. We formally proved, with the \coq{}
proof assistant, that the $\mathtt{||}$ composition operator for
concurrent statements is commutative and associative with respect to
the $runinit$ relation. In Figure, the \texttt{marked} process is
executed and yields the state $\sigma'_e$. Then, the rest of the
\texttt{tl} design's behavior is executed and yields the state
$\sigma''_e$. Finally, the starting state $\sigma_e$ and the two
states $\sigma'_e$ and $\sigma''_e$ are merged into one by the
\texttt{merge} function.

\begin{figure}[H] \centering
  \begin{prooftree}[template={\fontsize{12}{13}\selectfont\inserttext}]
    
    % Runinit marked process
    \hypo{}
    \ellipsis{}{}
    \infer1[]{
      $\mathcal{D}_\mathcal{H},\Delta,\sigma_e\vdash$
      $\mathtt{process}(\mathtt{marked},\dots)$ $\xrightarrow{runinit}$
      $\sigma'_e$ }

    % Runinit tail
    \hypo{}
    \ellipsis{}{}
    \infer1[]{
      $\mathcal{D}_\mathcal{H},\Delta,\sigma_e\vdash$ $cs'_{tl}$
      $\xrightarrow{runinit}$ $\sigma''_e$ }

    % Conclusion
    \infer2[\fontsize{7}{10}\selectfont\textsc{CompRunInit}$^1$]{
      $\mathcal{D}_\mathcal{H},\Delta,\sigma_e\vdash$
      $\mathtt{process}(\mathtt{marked},\dots)~\mathtt{||}~cs'_{tl}$
      $\xrightarrow{runinit}$
      $\mathtt{merge}(\sigma_e,\sigma'_e,\sigma''_e)$ }

  \end{prooftree}

  \caption{The $runinit$ phase applied to the concurrent statements
    composing the behavior of the \texttt{tl} design.}
  \label{fig:ex-runinit-tl}
\end{figure}

\noindent{}where:
\begin{enumerate}
\item $\mathcal{E}'_e\cap\mathcal{E}''_e$
\end{enumerate}

In what follows, we detail the execution of a process statement and of
a component instantiation statement during the first part of the
initialization, i.e. the $runinit$ phase.

\paragraph{Execution of a process statement with the $runinit$ relation}

The rule in Figure shows the execution of the \texttt{marked} process
during the $runinit$ phase. The first part of the reset block defining
the statement body of the \texttt{marked} process is executed. This
first part assigns the expression $\mathtt{false}$ to signal
$id_{a0}$.

\begin{figure}[H] \centering
  \begin{prooftree}[template={\fontsize{12}{13}\selectfont\inserttext}]

    \infer0[\fontsize{7}{10}\selectfont\textsc{False}]{$\vdash\mathtt{false}\xrightarrow{e}\bot$}
    \infer0[\fontsize{7}{10}\selectfont\textsc{IsBool}]{$\bot\in_c{}bool$}
    
    % Execute fst part of rst block
    \infer2[\fontsize{7}{10}\selectfont\textsc{SigAssign}$^2$]{
      $\mathcal{D}_\mathcal{H},\Delta,\sigma_e,\sigma_e\vdash$
      $id_{a0}\Leftarrow\mathtt{false}$
      $\xrightarrow{ss_i}$
      $\sigma'_e,\emptyset$}

    % Execute rst block
    \infer1[\fontsize{7}{10}\selectfont\textsc{RstExec}]{
      $\mathcal{D}_\mathcal{H},\Delta,\sigma_e\vdash$
      $\mathtt{rst}(id_{a0}\Leftarrow\mathtt{false})(\dots)$
      $\xrightarrow{ss_i}$
      $\sigma'_e,\emptyset$ }

    % Conclusion
    \infer1[\fontsize{7}{10}\selectfont\textsc{PsRunInit}$^1$]{
      $\mathcal{D}_\mathcal{H},\Delta,\sigma_e\vdash$
      $\mathtt{process}(\mathtt{marked},\dots)$
      $\xrightarrow{runinit}$
      $\sigma'_e$ }
  \end{prooftree}

  \caption{The $runinit$ phase applied to the concurrent statements
    composing the behavior of the \texttt{tl} design.}
  \label{fig:ex-runinit-ps}
\end{figure}

\noindent{}where:

\begin{enumerate}
\item $\Delta(\mathtt{marked})=\emptyset$
\item
  \begin{tabular}{@{}l@{}}
    $\Delta(id_{a0})=bool$ \\
    $\sigma'_e={<}\mathcal{S}'_e,\mathcal{C}'_e,\mathcal{E}'_e{>}$ \\
    $\mathcal{S}'_e=\mathcal{S}_e(id_{a0})\leftarrow\bot$ \\
    $\mathcal{E}'_e=\mathcal{E}_e\cup(\mathcal{S}_e\dcap\mathcal{S}'_e)$ \\
  \end{tabular}
\end{enumerate}

In the side conditions of the \textsc{SigAssign} rule, an new event
set $\mathcal{E}'_e$ is computed based on the event set
$\mathcal{E}_e$ joined to the expression
$\mathcal{S}_e\dcap\mathcal{S}'_e$. This expression returns the set of
signals with a different value between signal store $\mathcal{S}_e$
and singal store $\mathcal{S}'_e$. The only signal that possibly has a
different value from $\mathcal{S}_e$ to $\mathcal{S}'_e$ is the
assigned signal $id_{a0}$. Thus, this expression is a shorthand to
test if the value of signal $id_{a0}$ has changed after the execution
of the signal assignment statement. If it is the case, then the event
set receives the signal identifier $id_{a0}$; $id_{a0}$ is then an
eventful signal. In the present case, the value of signal $id_{a0}$
was $\bot$ at state $\sigma_e$ and is still $\bot$ after the execution
of the signal assignment statement. Therefore, no event is registered
on signal $id_{a0}$. When states $\sigma_e$, $\sigma'_e$ and
$\sigma''_e$ will be merged (cf. Figure~\ref{fig:ex-runinit-tl}), if
$id_{a0}$ is part of the event set of state $\sigma''_e$, then, the
merged state will return the value associated to $id_{a0}$ in state
$\sigma''_e$. We would have
$\mathtt{merge}(\sigma_e,\sigma'_e,\sigma''_e)(id_{a0})=\sigma''_e(id_{a0})$.
However, signal $id_{a0}$ would be a potentially multiply-driven
signal because both the \texttt{marked} process and the concurrent
statement $cs'_{tl}$ (cf. Figure~\ref{fig:ex-runinit-tl}) assigns the
signal value.

\paragraph{Execution of a component instantiation statement with the $runinit$ relation}

The rule of Figure presents the execution of the \texttt{place}
component instance $id_p$ during the $runinit$ phase. The execution of
a component instantiation statement is pretty much the same in all the
phases of the simulation algorithm. The difference lies in the choice
of the relation used to execute of the internal behavior of the
component instance. During the $runinit$ phase, it is the $runinit$
relation that executes the internal behavior of component instances;
during the falling edge phase, it is the $\downarrow$ relation that
executes the internal behaviors, etc.

\begin{figure}[H] \centering
  \begin{prooftree}[template={\fontsize{11}{13}\selectfont\inserttext}]
    
    % Execute mapip
    \hypo{}
    \ellipsis{}{}
    \infer1{
      $\Delta,\Delta_p,\sigma_e,\sigma_p\vdash$
      $i_p$
      $\xrightarrow{mapip}$
      $\sigma'_p$}

    % Execute internal behavior
    \hypo{}
    \ellipsis{}{}
    \infer1{
      $\mathcal{D}_\mathcal{H},\Delta_p,\sigma'_p\vdash$
      $cs_p$
      $\xrightarrow{runinit}$
      $\sigma''_p$ }

    % Execute mapop
    \hypo{}
    \ellipsis{}{}
    \infer1{
      $\Delta,\Delta_p,\sigma_e,\sigma''_p\vdash$
      $o_p$
      $\xrightarrow{mapop}$
      $\sigma'_e$}

    % Conclusion
    \infer3[\fontsize{7}{10}\selectfont\textsc{CompRunInit}$^1$]{
      $\mathcal{D}_\mathcal{H},\Delta,\sigma_e\vdash$
      $\mathtt{comp}(id_p,\mathtt{place},g_p,i_p,o_p)$
      $\xrightarrow{runinit}$
      $\sigma''_e$ }
  \end{prooftree}

  \caption[The execution of the \texttt{place} component instance
    $id_p$ during the $runinit$ phase.]{The execution of the \texttt{place} component instance
    $id_p$ during the $runinit$ phase (first part of the
    initialization).}
  \label{fig:ex-runinit-pci-idp}
\end{figure}

\noindent{}where:

\begin{enumerate}
\item
  \begin{tabular}{@{}l@{}}
    $\sigma'_e={<}\mathcal{S}'_e,\mathcal{C}'_e,\mathcal{E}'_e{>}$ \\
    $\sigma''_e={<}\mathcal{S}'_e,\mathcal{C}''_e,\mathcal{E}'_e\cup(\mathcal{C}_e\dcap\mathcal{C}''_e){>}$ \\
    $\mathcal{C}''_e=\mathcal{C}'_e(id_p)\leftarrow{}\sigma''_p$ \\
  \end{tabular}
\end{enumerate}



The execution of a component instantiation statement is decomposed in
four parts. First, the input ports of the component instance receive
new values through the evaluation of the input port map. Second, the
internal behavior of the component instance is executed. Thirdly, the
evaluation of the output port map propagates the values coming from
the output interface to the component to the signals of the embedding
design. Finally, the component instance is assigned to its new
internal state in the component store of the embedding design; here,
$\sigma''_p$ is assigned to $id_p$ in component store
$\mathcal{C}''_e$. Moreover, if the new internal state of the
component instance is different from its older internal state, then
the component instance identifier is added to the event set of the
embedding design. Here, the expression
$\mathcal{C}_e\dcap\mathcal{C}''_e$ performs the state comparison; we
have:

{\raggedleft
  \begin{tabular}{ll}
    $\mathcal{C}_e\dcap\mathcal{C}''_e$ & $=\mathcal{C}_e\dcap(\mathcal{C}'_e\leftarrow\sigma''_p)$ \\
                                      & $=\mathcal{C}_e\dcap(\mathcal{C}_e\leftarrow\sigma''_p)$ \\
                                      &$=
                                        \begin{cases}
                                          \{id_p\}~\mathtt{if}~\sigma_p\neq\sigma''_p \\
                                          \emptyset~otherwise \\
                                        \end{cases}$
  \\
\end{tabular}
}

In the second line, we have $\mathcal{C}_e=\mathcal{C}'_e$ because the
evaluation of the output port map (performed by the $mapop$ relation)
does not change the component store.

The rule of Figure~\ref{fig:ex-eval-ipm} gives a part of the
evaluation of the input port map of $id_p$.

\begin{figure}[H]
  \centering
  \begin{prooftree}[template={\fontsize{10}{13}\selectfont\inserttext}]

    % MapipSimple
    \infer0[\fontsize{7}{10}\selectfont\textsc{Nat}]{$\Delta,\sigma_e\vdash{}1\xrightarrow{e}1$}
    \infer0[\fontsize{7}{10}\selectfont\textsc{IsCNat}]{$1\in_c{}nat(0,1)$}
    \infer2[\fontsize{7}{10}\selectfont\textsc{MapipSimple}$^1$]{
      $\Delta,\Delta_p,\sigma_e,\sigma_p\vdash$
      $(\mathtt{im}, 1)\xrightarrow{mapip}\sigma'_{p0}$
    }

    % MapipComp
    \hypo{}
    \ellipsis{}{}
    \infer1{
      $\Delta,\Delta_p,\sigma_e,\sigma'_{p0}\vdash$
      \begin{tabular}{@{}l@{}}
        $(\mathtt{iaw}(0), 1)$ \\
        $(\mathtt{oat}(0), 0),$ $(\mathtt{oaw}(0), 1),$ \\
        $(\mathtt{itf}(0), id_{ft}),$ $(\mathtt{otf}(0), id_{ft})$ \\
      \end{tabular}
      $\xrightarrow{mapip}$
      $\sigma'_p$
    }
    
    % Conclusion 
    \infer2[\fontsize{7}{10}\selectfont\textsc{MapipComp}]{
      $\Delta,\Delta_p,\sigma_e,\sigma_p\vdash$
      \begin{tabular}{@{}l@{}}
        $(\mathtt{im}, 1),$ $(\mathtt{iaw}(0), 1)$ \\
        $(\mathtt{oat}(0), 0),$ $(\mathtt{oaw}(0), 1),$ \\
        $(\mathtt{itf}(0), id_{ft}),$ $(\mathtt{otf}(0), id_{ft})$ \\
      \end{tabular}
      $\xrightarrow{mapip}$
      $\sigma'_p$
    }
    
  \end{prooftree}
  
  \caption{The evaluation of the input port map of the \texttt{place}
    component instance $id_p$.}
  \label{fig:ex-eval-ipm}
\end{figure}

\noindent{}where:
\begin{enumerate}
\item
  \begin{tabular}{@{}l@{}}
    $\Delta_p(\mathtt{im})=nat(0,1)$ \\
    $\sigma_p={<}\mathcal{S},\mathcal{C},\mathcal{E}{>}$ \\
    $\mathcal{S}'=\mathcal{S}(\mathtt{im})\leftarrow{}1$ \\
    $\sigma'_{p0}={<}\mathcal{S}',\mathcal{C},\mathcal{E}{>}$ \\
  \end{tabular}
\end{enumerate}

The evaluation of the input port map of $id_p$ changes the value of
the \texttt{initial\_marking} input port (i.e. \texttt{im}). We have
$\sigma_p(\mathtt{im})=0$ and $\sigma'_p(\mathtt{im})=1$. As the value
of one of its input port has changed, the \texttt{place} component
instance $id_p$ will be registered as an eventful component instance.

The rule of Figure~\ref{fig:ex-eval-opm} gives a part of the
evaluation of the output port map of $id_p$.

\begin{figure}[H]
  \centering
  \begin{prooftree}[template={\fontsize{10}{13}\selectfont\inserttext}]

    % MapipSimple
    \infer0[]{$\vdash{}0\xrightarrow{e}0$}
    \infer0{$0\in_c{}nat(0,0)$}
    \infer2[\fontsize{7}{10}\selectfont\textsc{IdxSig}$^2$]{$\Delta_p,\sigma''_p\vdash\mathtt{oav}(0)\xrightarrow{e_o}\top$}
    \infer0[\fontsize{7}{10}\selectfont\textsc{IsBool}]{$\top\in_c{}bool$}
    \infer2[$^1$]{
      $\Delta,\Delta_p,\sigma_e,\sigma''_p\vdash$
      $(\mathtt{oav}(0), id_{av})$
      $\xrightarrow{mapop}\sigma'_{e0}$
    }

    % MapipComp
    \hypo{}
    \ellipsis{}{}
    \infer1{
      $\Delta,\Delta_p,\sigma'_{e0},\sigma''_p\vdash$
      \begin{tabular}{@{}l@{}}
        $(\mathtt{pauths}, \mathtt{open})$ \\
        $(\mathtt{rtt}(0), id_{rt}),$ $(\mathtt{marked}, id_m)$ \\
      \end{tabular}
      $\xrightarrow{mapop}$
      $\sigma'_e$
    }
    
    % Conclusion 
    \infer2[\fontsize{7}{10}\selectfont\textsc{MapopComp}]{
      $\Delta,\Delta_p,\sigma_e,\sigma''_p\vdash$
      \begin{tabular}{@{}l@{}}
        $(\mathtt{oav}(0), id_{av}),$ $(\mathtt{pauths}, \mathtt{open})$ \\
        $(\mathtt{rtt}(0), id_{rt}),$ $(\mathtt{marked}, id_m)$ \\
      \end{tabular}
      $\xrightarrow{mapop}$
      $\sigma'_e$
    }
    
  \end{prooftree}
  
  \caption{The evaluation of the output port map of the \texttt{place}
    component instance $id_p$.}
  \label{fig:ex-eval-opm}
\end{figure}

\noindent{}where:
\begin{enumerate}
\item
  \begin{tabular}{@{}l@{}}
    $\Delta(id_{av})=bool$ \\
    $\sigma_e={<}\mathcal{S},\mathcal{C},\mathcal{E}{>}$ \\
    $\mathcal{S}'=\mathcal{S}(id_{av})\leftarrow{}\top$ \\
    $\mathcal{E}'=\mathcal{E}\cup(\mathcal{S}\dcap\mathcal{S}')$ \\
    $\sigma'_{e0}={<}\mathcal{S}',\mathcal{C},\mathcal{E}'{>}$ \\
  \end{tabular} (Rule~\textsc{MapopPartialToSimple}).
\item
  \begin{tabular}{@{}l@{}}
    $\Delta_p(\mathtt{oav})=array(bool,0,0)$ \\
    $\sigma''_p(\mathtt{oav})=(\top)$ \\
    $\mathtt{get\_at}(0,(\top))=\top$ \\
  \end{tabular}
\end{enumerate}

\subsubsection{Stabilization}
\label{sec:ex-stabilize-tl}

A stabilization phase happens after the $runinit$ during the
initialization phase, but also after the rising edge phase and the
falling edge phase in the course of a simulation cycle. The
stabilization phase executes the combinational parts of the design's
behavior. The \texttt{tl} design holds no combinational processes in
its behavior. The \texttt{marked} and \texttt{fired} processes are
both synchronous. To illustrate the execution of a combinational
process during a stabilization phase, let us consider the
\texttt{fired_evaluation} process defined in the behavior of
\texttt{transition} design. The \texttt{fired_evaluation} process will
be executed with the internal behavior of the \texttt{transition}
component instance $id_t$ during the stabilization phase.  The rule of
Figure presents the execution of the internal behavior of the
\texttt{transition} component instance $id_t$. As shown, the internal
behavior $cs_t$ is executed three times before reaching a stable
state. Here, the number of execution before stabilization is
arbitrary. In Figure, $\sigma_{t0}$ corresponds to the state of $id_t$
after the $runinit$ phase and after the evaluation of its input port
map. Remember that the evaluation of the input port map of a component
instance always precedes the execution of the internal behavior of the
same component. Since $\sigma_{t0}$ and $\sigma_{t1}$ are not stable
states, it means that their event set is not empty. Thus, we have
$\mathcal{E}(\sigma_{t0})\neq\emptyset$ and
$\mathcal{E}(\sigma_{t1})\neq\emptyset$. On the contrary,
$\sigma_{t2}$ is a stable state, and thus,
$\mathcal{E}(\sigma_{t2})=\emptyset$.



%%% Local Variables:
%%% mode: latex
%%% TeX-master: "../../main"
%%% End:


\section{Implementation of the \hvhdl{} syntax and semantics}
\label{sec:hvhdl-impl}
This section presents the implementation of the \hvhdl{} abstract
syntax, and also of the elaboration and the simulation semantics of
\hvhdl{} designs with the \coq{} proof assistant. The full code is
available under the \texttt{hvhdl} folder of the following repository:
\url{https://github.com/viampietro/ver-hilecop}.

\subsection{Implementation of the \texorpdfstring{\hvhdl{}}{H-VHDL}
  abstract syntax, elaborated design and design state}
\label{sec:impl-abss-and-env}

\paragraph{\hvhdl{} abstract syntax}

The implementation of the \hvhdl{} abstract syntax is naturally done
leveraging the \texttt{Inductive} construct of the \coq{} proof
assistant. The result is strictly similar to the formal definition of
the abstract syntax given in Section~\ref{sec:abstractSyntax}. The
reader can refer to the \texttt{AbstractSyntax.v} under the
\texttt{hvhdl} folder for the details of the implementation.

\paragraph{Elaborated design}

Listing~\ref{lst:elab-design-struct} presents the implementation of
the elaborated design structure
(cf. Definition~\ref{def:elab-design}). Two definitions are involved
in the implementation of the elaborated design structure.  The first
one defines the \texttt{SemanticObject} inductive type. Each
constructor of this type corresponds to a sub-environment of the
elaborated design. For instance, the \texttt{Generic} constructor
corresponds to the couple $(type\times{}value)$ associated with a
generic constant identifier in the $Gens$ sub-environment of
Definition~\ref{def:elab-design}. The \texttt{Process} constructor
corresponds to the local variable environment associated with the
process identifiers in the $Ps$ sub-environment. A local variable
environment is implemented by the \texttt{LEnv} type. The
\texttt{LEnv} type is a map between identifiers and couples
$(type\times{}value)$. Identifiers are implemented by the
\texttt{ident} type, an alias of the \texttt{nat} type. The
\texttt{type} and \texttt{value} types are the implementation of the
semantic $type$ and $value$ presented in
Table~\ref{tab:type-value}. The \texttt{ElDesign} type implements the
elaborated design structure. It is an alias to the \texttt{IdMap
  SemanticObject} type. The \texttt{IdMap} is the type of maps from
identifiers (i.e. belonging to the \texttt{ident} type) to instances
of the type passed as an input. Here, the input is the
\texttt{SemanticObject} type. Thus, an elaborated design is
implemented as a map between identifiers and terms of the
\texttt{SemanticObject} type. We leverage the \texttt{FMaps} module
defined in the \coq{} standard library to implement the \texttt{IdMap}
type. The \texttt{IdMap} type ensures that an identifier is only
mapped once. Thus, the implementation of the elaborated design
structure verifies that there are no intersection between the domains
of sub-environments. For instance, a generic constant identifier can
not be an input port identifier, and, as it is implemented, an
identifier \texttt{id} can not be mapped to a \texttt{Generic} object
and to an \texttt{Input} object in the same instance of
\texttt{ElDesign}.
 
\begin{lstlisting}[language=coq,label={lst:elab-design-struct},
caption={[\coq{} implementation of the elaborated design structure.]The implementation of the elaborated design structure with the \coq{} proof assistant.},framexleftmargin=1.5em,xleftmargin=2em,numbers=left,
numberstyle=\tiny\ttfamily]
Inductive SemanticObject : Type :=
| Generic (t : type) (v : value)
| Input (t : type)
| Output (t : type)
| Declared (t : type)
| Process (lenv : LEnv)
| Component ($\Delta_c$ : IdMap SemanticObject).

Definition ElDesign := IdMap SemanticObject.
\end{lstlisting}

\paragraph{Design state}

Listing~\ref{lst:design-state-impl} gives the implementation of the
design state structure through the definition of the \texttt{DState}
inductive type. The constructor of the \texttt{DState} type defines
three fields: \texttt{sigstore}, implementing the signal store
$\mathcal{S}$ of the design state, \texttt{compstore}, implementing
the component store $C$, and \texttt{events}, implementing the set of
events $\mathcal{E}$ of the design state. The \texttt{sigstore} field
is a map from identifiers to values. The \texttt{compstore} field is a
map from identifiers to design states, justifying the inductive
definition of the \texttt{DState} type. The \texttt{events} field is
an instance of the \texttt{IdSet} type. The \texttt{IdSet} is the type
of sets of identifiers (i.e. sets of natural numbers). The
\texttt{IdSet} type is defined leveraging the \texttt{MSets} module of
the \coq{} standard library.

\begin{lstlisting}[language=coq,label={lst:design-state-impl},
  caption={[\coq{} implementation of the design state structure.]The implementation of the design state structure with the \coq{} proof assistant.},framexleftmargin=1.5em,xleftmargin=2em,numbers=left,
  numberstyle=\tiny\ttfamily]
Inductive DState : Type := MkDState {
  sigstore  : IdMap value;
  compstore : IdMap DState;
  events    : IdSet;
}.
\end{lstlisting}

\subsection{Implementation of the elaboration phase}
\label{sec:impl-elaboration}

The design elaboration relation, as presented in
Section~\ref{subsubsec:design-elab}, is implemented in \coq{} by the
\texttt{edesign} relation. Listing~\ref{lst:design-elab-rel-impl}
presents the definition of the \texttt{edesign} relation as an
inductive type. As usual, a n-ary relation is implemented in \coq{} by
a type defined with $n$ parameters and projecting to the \texttt{Prop}
type. The \texttt{edesign} relation as five parameters. The first
parameter is the design store $\mathcal{D}$ of type \texttt{IdMap
  design}, i.e. a map from identifiers to \hvhdl{} designs as defined
by the abstract syntax. The second parameter is the dimensioning
function $\mathcal{M}$ of type \texttt{IdMap value}, i.e. a map from
identifiers to values. The third parameter is the design being
elaborated, of type \texttt{design}. The fifth and sixth parameters
are the elaborated design (of type \texttt{ElDesign} and the default
design state (of type \texttt{DState}) resulting from the elaboration.
In Listing~\ref{lst:design-elab-rel-impl}, the \texttt{EDesign}
constructor implements the \textsc{DesignElab} rule presented in
Section~\ref{subsubsec:design-elab}. From Line~7 to Line~10, the
constructor defines the premises of Rule~\textsc{DesignElab}.  The
empty elaborated design structure, denoted $\Delta_\emptyset$, is
implemented by the \texttt{EmptyElDesign} definition, and the empty
design state structure, denoted by $\sigma_\emptyset$, is implemented
by the \texttt{EmptyDState} definition. Line~13 implements the
conclusion of Rule~\textsc{DesignElab}.

\begin{lstlisting}[language=coq,label={lst:design-elab-rel-impl},
  caption={[\coq{} implementation of the design elaboration relation.]The implementation of the design elaboration relation with the \coq{} proof assistant.},framexleftmargin=1.5em,xleftmargin=2em,numbers=left,
  numberstyle=\tiny\ttfamily]
Inductive edesign ($\mathcal{D}$ : IdMap design) : IdMap value -> design -> ElDesign -> DState -> Prop :=
| EDesign :
    forall $\mathcal{M}$ $id_e$ $id_a$ gens ports sigs behavior
           $\Delta$ $\Delta'$ $\Delta''$ $\Delta'''$ $\sigma$ $\sigma'$ $\sigma''$,

      (* Premises *)
      egens EmptyElDesign $\mathcal{M}$ gens $\Delta$ ->
      eports $\Delta$ EmptyDState ports $\Delta'$ $\sigma$ ->
      edecls $\Delta'$ $\sigma$ sigs $\Delta''$ $\sigma'$ ->
      ebeh $\mathcal{D}$ $\Delta''$ $\sigma'$ behavior $\Delta'''$ $\sigma''$ ->
      
      (* Conclusion *)
      edesign $\mathcal{D}$ $\mathcal{M}$ (design_ $id_e$ $id_a$ gens ports sigs behavior) $\Delta'''$ $\sigma''$

with ebeh ($\mathcal{D}$ : IdMap design) : ElDesign -> DState -> cs -> ElDesign -> DState -> Prop :=
$\dots$
\end{lstlisting}

The \texttt{edesign} relation requires a mutually recursive definition
with the \texttt{ebeh} relation. The mutually recursive definition is
performed leveraging the \texttt{with} clause at the end of
Listing~\ref{lst:design-elab-rel-impl}. The \texttt{ebeh} relation
needs the \texttt{edesign} relation to elaborate the component
instances found in the behavior of a
design. Listing~\ref{lst:ebeh-rel-impl} gives the details of the
\texttt{with} clause defining the \texttt{ebeh} relation.

At Line~2, the \texttt{EBehPs} constructor implements the
Rule~\textsc{PsElab} defining the elaboration of a process statement
(cf. Section~\ref{subsubsec:ps-elab}). Lines~6 and 7 implement the
premises of the rule; the \texttt{evars} relation implements the
elaboration of the local variable declaration list of the process; the
\texttt{validss} relation implements the relation that type-checks the
statement body of the process. Lines~10 to 14 implement the side
conditions of the rule. The term \coqe|~NatMap.In| $id_p$ $\Delta$
implements the side condition $id_p\notin{}\Delta$. The
\coqe|NatMap.In id m| relation states that a given identifier
\texttt{id} is a key of the \texttt{m} map. At Line~13, the
\coqe|NatSet.In $id_s$ sl| term states that $id_s$ belongs to the
identifier set \texttt{sl}. At Line~14, the term \texttt{MapsTo $id_s$
  (Input t) $\Delta$} states that the identifier $id_s$ is mapped to
\texttt{Input t} in the elaborated design $\Delta$,
i.e. $Ins(\Delta)(id_s)=\mathtt{t}$. More generally, \texttt{MapsTo}
is a ternary relation stating that a given key \texttt{k} of type
\texttt{nat}, is mapped to a value \texttt{v} of a type \texttt{A}, in
a given map \texttt{m}, i.e. \coqe|Mapsto k v m|.  Line~17 implements
the conclusion of Rule~\textsc{PsElab}. The \texttt{NatMap.add}
function binds the process identifier $id_p$ to the term
\texttt{Process $\Lambda$} in the elaborated design $\Delta$,
i.e. $\Delta\cup(id_p,\Lambda)$.

At Line~19, the \texttt{EBehComp} constructor implements the
Rule~\textsc{CompElab}
(cf. Section~\ref{subsubsec:comp-inst-elab}). This rule describes the
elaboration of a component instantiation statement. Lines~24 to 27
implement the premises of the rule. Line~25 appeals to the
\texttt{edesign} relation to elaborate the \texttt{cdesign} design
associated with the component instance $id_c$; thence, the mutually
recursive definition with the \texttt{ebeh} relation. As it is stated
at Line~32, the \texttt{cdesign} design is associated to identifier
$id_e$, i.e. the entity identifier of component instance $id_c$, in
the design store $\mathcal{D}$. Lines~30 to 33 implement the side
conditions of the rule. Line~30 checks that the identifier $id_c$ is
not already bound to a semantic object in the elaborated design
$\Delta$. Line~31 checks that the identifier $id_c$ is not already
bound in the component store of $\sigma$. Line~33 checks that all
identifiers defined in the domain of map $\mathcal{M}$, i.e. the
dimensioning function, are bound to generic constants in the
elaborated design $\Delta_c$
(i.e. $\mathcal{M}\subseteq{}Gens(\Delta_c)$). Lines~36 to 38
implement the conclusion of Rule~\textsc{CompElab}. At Line~39, the
\texttt{cstore_add} function binds $id_c$ to design state $\sigma_c$
in the component store of state $\sigma$ and returns the resulting
state.

At Line~41, the \texttt{EBehNull} constructor implements
Rule~\textsc{CsNullElab}. At Line~43, the \texttt{EBehPar} constructor
implements Rule~\textsc{CsParElab}.

\begin{lstlisting}[language=coq,label={lst:ebeh-rel-impl},
  caption={[\coq{} implementation of the behavior elaboration relation.]The implementation of the \texttt{ebeh} behavior elaboration relation with the \coq{} proof assistant.},framexleftmargin=1.5em,xleftmargin=2em,numbers=left,
  numberstyle=\tiny\ttfamily]
with ebeh ($\mathcal{D}$ : IdMap design) : ElDesign -> DState -> cs -> ElDesign -> DState -> Prop :=
| EBehPs :
    forall $id_p$ sl vars stmt $\Lambda$ $\Delta$ $\sigma$,

      (* Premises *)
      evars $\Delta$ EmptyLEnv vars $\Lambda$ ->
      validss $\Delta$ $\sigma$ $\Lambda$ stmt ->

      (* Side conditions *)
      ~NatMap.In $id_p$ $\Delta$ ->

      (forall $id_s$,
          NatSet.In $id_s$ sl ->
          exists t, MapsTo $id_s$ (Declared t) $\Delta$ \/ MapsTo $id_s$ (Input t) $\Delta$) ->

      (* Conclusion *)
      ebeh $\mathcal{D}$ $\Delta$ $\sigma$ (cs_ps $id_p$ sl vars stmt) (NatMap.add $id_p$ (Process $\Lambda$) $\Delta$) $\sigma$

| EBehComp :
    forall $\Delta$ $\sigma$ $id_c$ $id_e$ gmap ipmap opmap
           $\mathcal{M}$ $\Delta_c$ $\sigma_c$ formals actuals cdesign,

      (* Premises *)
      emapg (NatMap.empty value) gmap $\mathcal{M}$ ->
      edesign $\mathcal{D}$ $\mathcal{M}$ cdesign $\Delta_c$ $\sigma_c$ ->
      validipm $\Delta$ $\Delta_c$ $\sigma$ ipmap formals ->
      validopm $\Delta$ $\Delta_c$ opmap formals actuals ->
      
      (* Side conditions *)
      ~NatMap.In $id_c$ $\Delta$ ->
      ~NatMap.In $id_c$ (compstore $\sigma$) ->
      MapsTo $id_e$ cdesign $\mathcal{D}$ ->
      (forall g, NatMap.In g $\mathcal{M}$ -> exists t v, MapsTo g (Generic t v) $\Delta_c$) ->
      
      (* Conclusion *)
      ebeh $\mathcal{D}$ $\Delta$ $\sigma$ (cs_comp $id_c$ $id_e$ gmap ipmap opmap)
                  (NatMap.add $id_c$ (Component $\Delta_c$) $\Delta$)
                  (cstore_add $id_c$ $\sigma_c$ $\sigma$)
           
| EBehNull: forall $\Delta$ $\sigma$, ebeh $\mathcal{D}$ $\Delta$ $\sigma$ cs_null $\Delta$ $\sigma$

| EBehPar:
    forall $\Delta$ $\Delta'$ $\Delta''$ $\sigma$ $\sigma'$ $\sigma''$ cstmt cstmt',
      ebeh $\mathcal{D}$ $\Delta$ $\sigma$ cstmt $\Delta'$ $\sigma'$ ->
      ebeh $\mathcal{D}$ $\Delta'$ $\sigma'$ cstmt' $\Delta''$ $\sigma''$ ->
      ebeh $\mathcal{D}$ $\Delta$ $\sigma$ (cs_par cstmt cstmt') $\Delta''$ $\sigma''$.

\end{lstlisting}

\subsection{Implementation of the simulation algorithm}
\label{sec:impl-simulation}

The full simulation relation (cf. Section~\ref{sec:full-sim})
formalizes the \hvhdl{} simulation algorithm. The \coq{}
implementation of the full simulation relation, presented in
Listing~\ref{lst:full-sim-impl}, is a strict translation of
Rule~\textsc{FullSim}. At Lines~14 and 15, the term \texttt{(behavior
  d)} represents the concurrent statements defining the behavior of
the \hvhdl{} design \texttt{d} (i.e. d.cs in the formal rule). Line~13
corresponds to the elaboration phase, Line~14 to the initialization
phase, and Line~15 to the main simulation loop.

\begin{lstlisting}[language=coq,label={lst:full-sim-impl},
  caption={[\coq{} implementation of the full simulation relation.]The implementation of the full simulation relation with the \coq{} proof assistant.},framexleftmargin=1.5em,xleftmargin=2em,numbers=left,
  numberstyle=\tiny\ttfamily]
Inductive fullsim
          ($\mathcal{D}$ : IdMap design)
          ($\mathcal{M}$ : IdMap value)
          ($E_p$ : nat -> Clk -> IdMap value)
          ($\tau$ : nat)
          ($\Delta$ : ElDesign) 
          (d : design) : list DState -> Prop :=
  
| FullSim :
    forall $\sigma_e$ $\sigma_0$ $\theta$,
      
      (* * Premises * *)
      edesign $\mathcal{D}$ $\mathcal{M}$ d $\Delta$ $\sigma_e$ ->
      init $\mathcal{D}$ $\Delta$ $\sigma_e$ (behavior d) $\sigma_0$ ->
      simloop $\mathcal{D}$ $E_p$ $\Delta$ $\sigma_0$ (behavior d) $\tau$ $\theta$ ->
                    
      (* * Conclusion * *)
      fullsim $\mathcal{D}$ $\mathcal{M}$ $E_p$ $\tau$ $\Delta$ d ($\sigma_0$ :: $\theta$).
\end{lstlisting}

The \texttt{simloop} relation appeals to the \texttt{simcycle} that
implements the simulation cycle relation defined in
Section~\ref{sec:sim-cycle}. Listing~\ref{lst:sim-cycle-impl} presents
the implementation of the \texttt{simcycle} relation. The
\texttt{simcycle} relation is a strict transcription of the
\textsc{SimCyc} rule. At Line~13, the \texttt{vrising} relation
implements the $\uparrow$ relation, i.e. the rising edge phase of the
cycle. At Line 15, the \texttt{vfalling} relation implements the
$\downarrow$ relation, i.e. the falling edge phase of the cycle. At
Lines~14 and 16, the \texttt{stabilize} relation implements the
$\rightsquigarrow$ relation, i.e. the stabilization phases of the
simulation cycle. At Lines~18 and 19, the \texttt{IsInjectedDState}
relation implements the \texttt{Inject} relation. Line~18 states that
the $\sigma_i$ state is the result of the injection of the map
\coqe|($E_p$ $\tau$)| in the signal store of state $\sigma$.

\begin{lstlisting}[language=coq,label={lst:sim-cycle-impl},
caption={[\coq{} implementation of the simulation cycle relation.]The
  implementation of the simulation cycle relation with the \coq{}
  proof
  assistant.},framexleftmargin=1.5em,xleftmargin=2em,numbers=left,
numberstyle=\tiny\ttfamily]
Inductive simcycle ($\mathcal{D}$ : IdMap design) ($E_p$ : nat -> IdMap value) 
    ($\Delta$ : ElDesign) ($\tau$ : nat) ($\sigma$ : DState) (behavior : cs) 
    ($\sigma'$ $\sigma''$ : DState) : Prop := 
| SimCycle : forall $\sigma_i$ $\sigma_\uparrow$ $\sigma_\downarrow$,
      
      (* * Premises * *)      
      vrising $\mathcal{D}$ $\Delta$ $\sigma_i$ behavior $\sigma_\uparrow$ ->
      stabilize $\mathcal{D}$ $\Delta$ $\sigma_\uparrow$ behavior $\sigma'$ ->
      vfalling $\mathcal{D}$ $\Delta$ $\sigma'$ behavior $\sigma_\downarrow$ ->
      stabilize $\mathcal{D}$ $\Delta$ $\sigma_\downarrow$ behavior $\sigma$'' ->

      (* * Side conditions * *)
      IsInjectedDState $\sigma$ ($E_p$ $\tau$) $\sigma_i$ ->
      
      (* * Conclusion * *)
      simcycle $\mathcal{D}$ $E_p$ $\Delta$ $\tau$ $\sigma$ behavior $\sigma'$ $\sigma''$.
\end{lstlisting}


%%% Local Variables:
%%% mode: latex
%%% TeX-master: "../../main"
%%% End:


\section{Conclusion}
\label{sec:hvhdl-concl}
In this chapter, we gave an overview of the VHDL language and its
informal simulation semantics.  Then, considering our needs, that is
considering the content of the VHDL programs generated by the
\hilecop{} model-to-text transformation, we defined a synthesizable
and synchronous subset of the VHDL language called \hvhdl{}. We gave a
small-step semantics to \hvhdl{} by formalizing a simplified
simulation algorithm. The simulation algorithm yields a simulation
trace, i.e. time-ordered list of states, corresponding to the
execution of the behavior of a \hvhdl{} design over multiple clock
cycles. The formalization of the \hvhdl{} semantics also includes the
formalization of the design elaboration. The elaboration, prior to the
simulation, ensures the well-formedness and the well-typedness of a
\hvhdl{} design. Moreover, we have implemented the \hvhdl{} syntax and
semantics with the \coq{} proof assistant.

Ever since the mechanization of the proof of behavior preservation has
begun, the semantics of \hvhdl{} has been
evolving. Section~\ref{sec:abstractSyntax}, \ref{sec:sem-rules},
\ref{sec:elab-rules} and \ref{sec:sim-rules} present the most recent
version of the semantics. However, it will probably be evolving again.
With regards to our proof task, we realized that an operational
semantics close to the simulation algorithm carried a lot of elements
that were of no use. These elements could sometimes complexify the
proof. For instance, in the VHDL simulation algorithm, the body of a
process is executed during the stabilization phase only if one signal
of its sensitivity list is part of the current state's event
set. However, it is through the execution of the body of a process
with the rules of the \hvhdl{} semantics that we can determine the
\emph{combinational} equation associated with the value of a
signal. In the proceeding of the proof of semantic preservation, we
must often describe the value of an output signal with regards to the
value of its input, or \emph{source}, signals
(cf. Section~\ref{sec:detailled-proof}).  Due to the event-based
system of resuming a process activity, a combinational process could
sometimes never be executed during a stabilization phase. Then, we are
not able to determine the value of signals. We had to carry extra
hypotheses in the definition of our lemmas to deal with this
problem. Finally, our current semantics always executes the body
combinational processes during a stabilization phase, and this greatly
simplifies the proof. By doing this kind of simplification, we
realized that we were heading toward a semantics that was closer to
the ``synthesis'' semantics we talked about at the beginning of the
chapter. This semantics tends to get closer to the combinational logic
and the synchronous logic rules. These rules that a hardware system
designer has in mind when devising a model with a hardware description
language.

In the future, we contrive to improve the implementation of the
\hvhdl{} semantics with more dependent types. Especially, the
elaborated design and the design state structure are formally defined
with intentional subsets. These subsets could be easily implemented
with the \texttt{sig} type of the \coq{} proof assistant. Also, we
plan to improve the formalization of the elaboration phase with a
global lookup of multiply-driven signals.


%%% Local Variables:
%%% mode: latex
%%% TeX-master: "../../main"
%%% End:


%%% Local Variables:
%%% mode: latex
%%% TeX-master: "../main"
%%% End:
