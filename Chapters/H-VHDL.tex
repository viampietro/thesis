\chapter{\hvhdl{}: a target hardware description language}
\label{chap:hvhdl}

\textsc{Research question of the chapter:}\\
\fbox{\parbox{\textwidth}{\sc Which formal semantics of VHDL will most
    suit our needs?}}

\section{Presentation of the \vhdl{} language}
\label{sec:vhdl-presentation}

\begin{itemize}
\item Present the main concepts of \vhdl{}
\item Present the informal semantics of \vhdl{}, i.e as described in
  the \textit{Language Reference Manual} (LRM):
  \begin{itemize}
  \item elaboration
  \item simulation
  \end{itemize}
\item Present what use is made of \vhdl{} in \hilecop{}
  \begin{itemize}
  \item show our simulation cycle
  \item present the P and T designs, and maybe a very high-level view
    of the transformation
  \item show a figure of the P and T designs interfaces
  \end{itemize}
  \pnote{Don't go in too much details while presenting the use of
    \vhdl{} in \hilecop{}. Details of which \vhdl{} constructs are
    used in our programs will be given in the next section, and
    details on the transformation will be given in the next chapter.}
\end{itemize}

\section{Choosing a semantics for \vhdl{}}
\label{sec:choosing-vhdl-sem}

\begin{itemize}
\item Our needs for a formal semantics\\
  \ding{212} reminder of the final goal of the thesis\\
  \pnote{Do not describe what the semantics must look like, but rather
    remind the purpose of the semantics in relation to the PhD goal.}
\item Research question: which semantics should we consider?\\
  \ding{212} express the trade-off between
  \begin{enumerate}
  \item using existing tools
  \item creating our own tools
  \end{enumerate}
\item Formalization depends on our needs
  \begin{itemize}[label=\ding{212}]
  \item present our needs
  \item present the qualifying criterions, and talk about the
    mandatory hypotheses about stable signals\\
    \pnote{Talking about the mandatory hypotheses first here. Decide
      the level of precision we want our semantics to have. }
  \end{itemize}
\item Presentation of the literature works and relations to our needs\\
  \fbox{\sc Conclusion:} ``We choose to set up our own semantics.''
\end{itemize}

\section{A natural semantics for \hvhdl{}}
\label{sec:hvhdl-sem}

\pnote{Refer to appendix \nameref{app:nat-sem} for a presentation of
  natural semantics on a very simple imperative programming language.}

\begin{itemize}
\item Abstract syntax
\item Semantical domains
\item Elaboration
\item Simulation
\end{itemize}

\pnote{Add examples of derivation of elaboration and simulation rules
  + P and T design in concrete syntax and abstract syntax.}

\section{\coq{} implementation of \hvhdl{}}
\label{sec:hvhdl-coq-impl}

\begin{itemize}
\item Give some interesting parts of the code (maybe implementation of
  $\Delta$ and $\sigma$ environment)
\item Discussing improvements with dependent-types?
\item Give some inductive types, simulation relations maybe?
\item Give metrics about the code
\item Give link to Git repo
\end{itemize}

\textsc{CONCLUSION OF THE CHAPTER:}\\

Contributions:
\begin{itemize}
\item setting a semantics for \hvhdl{}, fitting our needs closely, but
  based on previous formalization works
\item implementing the simulation semantics + abs. syntax in \coq{}
\end{itemize}

%%% Local Variables:
%%% mode: latex
%%% TeX-master: "../main"
%%% End:
