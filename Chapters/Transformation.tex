\chapter{The \hilecop{} model-to-text transformation}
\label{chap:transformation}

\begin{todobox}
  \begin{itemize}
  \item Motivate the interest of the transformation in the
    introduction. Why the models are not hand-coded in VHDL directly?
    Because graphics help the design and communication, automatic
    generation because places and transitions interconnections to complex\dots
  \item Pciples of SITPNs implementation in VHDL: place pivot, keep
    place and transition design as simple as possible,
    interconnections via Boolean signals only
  \end{itemize}
\end{todobox}

The aim of this chapter is to present the details of the \hilecop{}
model-to-text transformation that we propose to verify as semantic
preserving. The chapter is structured as follows. First, we make an overall 
description of the \hilecop{} transformation. Then, we present,
in Section~\ref{sec:transf-lit-rev}, a literature review of the works
pertaining to transformation functions in the context of formal
verification. The literature review focuses on the expression of
transformation functions and on their implementation. In
Section~\ref{sec:trans-alg}, we thoroughly present the \hilecop{} transformation
function in the form of a pseudo-code algorithm. Finally, in
Section~\ref{sec:trans-coq-impl}, we describe the \coq{}
implementation of the algorithm.

\section{Informal presentation of the \hilecop{} transformation}
\label{sec:informal-transf}
This section outlines the main phases of the \hilecop{} model-to-text
transformation function. The goal is to give to the reader the means
to appreciate the differences and the similarities between the
\hilecop{} transformation and the other transformations presented in
the literature review of Section~\ref{sec:transf-lit-rev}. Then,
Section~\ref{sec:trans-alg} will enter the details of the
transformation by presenting the transformation algorithm.

The \hilecop{} model-to-text transformation function takes an SITPN
model as input; then, it generates a top-level \hvhdl{} design out of
the input model. We will illustrate each step of the \hilecop{}
model-to-text transformation through the transformation of the input
SITPN model presented in Figure~\ref{fig:transf-toplevel}.

\begin{figure}[H]
  \centering
  \includegraphics[keepaspectratio,width=.8\textwidth]{Figures/Transformation/transf-fun-toplevel}
  \caption[Transformation of an input SITPN model into a top-level
  \hvhdl{} design.]{Transformation of an input SITPN model into a
    top-level \hvhdl{} design.  The input model is composed of two
    places, $p_0$ and $p_1$, and two transitions, $t_0$ and
    $t_1$. The transition $t_0$ is associated with the time interval
    $[1,3]$ and the condition $c_0$. The transition $t_1$ is associated
    with the condition $c_1$, and its firing triggers the execution of
    the function $f_0$. The action $a_0$ is activated when the place $p_0$ is
    marked, and the action $a_1$ is activated when the place $p_1$ is marked.
  }
  \label{fig:transf-toplevel}
\end{figure}

The generated top-level design implements the structure of the input
SITPN. As a first step, the transformation generates, for each place
of the input SITPN, a component instance of the \texttt{place} design,
and, for each transition of the input SITPN, a component instance of
the \texttt{transition} design. These subcomponents constitute the
main part of the \hvhdl{} top-level design's architecture (i.e. its
internal behavior). Figure~\ref{fig:itfaces} shows a graphical
representation of the input and output port interfaces of the
\texttt{place} and \texttt{transition} designs. All PCIs (Place
Component Instances) and TCIs (Transition Component Instances)
generated during the first step of the \hilecop{} transformation
inherit the interface presented in Figure~\ref{fig:itfaces}.

\begin{figure}[H]
  \centering
  \includegraphics[keepaspectratio,width=\textwidth]{Figures/Transformation/itfaces}
  \caption[The interfaces of the \texttt{place} and
  \texttt{transition} designs.]{On the left, the \texttt{place} design
    interface and on the right the \texttt{transition} design
    interface. The indexes of composite ports are expressed at the
    inner extremity of the pins, while the name (abbreviated) of ports
    are expressed at the outer extremity.}
  \label{fig:itfaces}
\end{figure}


During the first generation step of the \hilecop{} transformation,
each PCI and TCI receive a value for each of their generic constants
through the creation of generic maps.  In the generic map of a TCI
$id_t$ (implementing a transition $t$), the \texttt{ian} constant is
associated with the number of input arcs of $t$, the \texttt{cn}
constant with the number of conditions attached to $t$, etc.  In the
generic map of a PCI $id_p$, the \texttt{ian} constant is associated
with the number of input arcs of $p$, the \texttt{oan} constant with
the number of output arcs of $p$, and the \texttt{mm} constant with
the maximal marking value of $p$. The maximal marking value associated
with a given place $p$ of the input SITPN is an information passed as
a parameter to the transformation function. This information comes
from the analysis of the input SITPN pertaining to the
\textit{boundedness} of the input model. In the definition of the
\hilecop{} methodology, this analysis takes place before the
transformation of the input SITPN into a \hvhdl{} design. The generic
constants do not appear as pins in the interfaces of the
\texttt{place} and \texttt{transition} designs presented in
Figure~\ref{fig:itfaces}. The generic constants have an impact of the
structure of the interface of each component instance. For example,
Figure~\ref{fig:itfaces} shows the dependency between the size
(i.e. the number of pins) of composite ports and the value of generic
constants, e.g. the size of \texttt{iaw} input port of the
\texttt{place} design depends on the \texttt{ian} generic constant.
Thus, the generation of generic maps during this first generation step
corresponds to the \textit{dimensioning} of the PCIs and TCIs; this is
when the number of pins of composite ports are determined.

Figure~\ref{fig:transf-arch} shows the architecture of the top-level
design resulting of the first generation step of the \hilecop{}
transformation.

\begin{figure}[H]
  \centering
  \includegraphics[keepaspectratio,width=\textwidth]{Figures/Transformation/transf-fun-arch}
  \caption[Generation of the \texttt{place} and \texttt{transition}
  component instances.]{Generation of the \texttt{place} and
    \texttt{transition} component instances based on the set of places
    and transitions of the input SITPN. The \texttt{PCI}
    $\mathtt{id}_{p_0}$ implements the place $p_0$, TCI
    $\mathtt{id}_{t_0}$ the transition $t_0$\dots{} In
    \textcolor{red}{red}, the internal signals connected to the
    \texttt{marked} port of PCIs and to the \texttt{fired} port of
    TCIs.}
  \label{fig:transf-arch}
\end{figure}

During the first transformation step, illustrated in
Figure~\ref{fig:transf-arch}, the input and output port maps of PCIs
and TCIs are also partly generated. In the manner of the generic
constants in generic maps, some input ports are associated with
constant values in the input port maps of PCIs and TCIs. All these
associations are generated during this first step. Also, the
\texttt{marked} output port of every PCI is associated with an
internal signal in the output port map of the PCI. The internal signal
will be connected later in the course of the transformation. The same
holds for the \texttt{fired} output port of every TCI.

After the first transformation step, the component instances are
interconnected through their port interfaces.
Figure~\ref{fig:transf-inter} illustrates the behavior of the
top-level design after the interconnection of PCIs and TCIs.

\begin{figure}[H]
  \centering
  \includegraphics[keepaspectratio,width=\textwidth]{Figures/Transformation/transf-fun-inter}
  \caption[Generation of the interconnections between the
  \texttt{place} and \texttt{transition} component
  instances.]{Generation of the interconnections between the
    \texttt{place} and \texttt{transition} component instances. In
    \textcolor{red}{red}, the internal signals interconnecting the
    PCIs and the TCIs. These signals are generated by the
    transformation. The arrows indicate the sense of propagation of
    the information.  In \textcolor{blue}{blue}, the constant associations (i.e. the generic maps and a part of the input port maps)
    produced during the previous transformation step.}
  \label{fig:transf-inter}
\end{figure}

The PCIs and TCIs interact through their interfaces to exchange
informations. For instance, a PCI $id_p$, implementing a given place
$p$, separately informs its output TCIs (i.e. the TCIs implementing
the output transitions of $p$) that its current marking enables
them. The marking of a PCI is represented by the value of its internal
signal \texttt{s_marking}. A PCI is the only one to have access to the
current value of its internal signals. Thus, a PCI must communicate to
its output TCIs their sensitization status.  To perform this exchange
of information, the transformation generates an internal signal to
connect a specific output port of a PCI (the \texttt{oav} port) to a
specific input port of the output TCIs (the \texttt{iav} port).
Likewise, a TCI informs its input and output PCIs about its firing
status. The transformation generates an internal signal to connect the
\texttt{fired} output port of a TCI to the \texttt{itf} and
\texttt{otf} input ports of the input and output PCIs.  These
interconnections are performed by adding new associations in the input
port map and output port map of PCIs and TCIs.  Through the execution
of the internal behavior of each PCI and TCI, and, through the
interconnection of component instances, the transformation aims at
generating a design's behavior that, by its inherent structure,
carries the rules of the SITPN semantics and conforms to the execution
of the input SITPN model.

To reduce the size of circuits after the synthesis on an FPGA or ASIC,
PCIs and TCIs only communicate with Boolean signals through their
interfaces. To restrict the interconnections to Boolean signals, the
\texttt{place} design, which is the mold of all PCIs, carries the arc
information (i.e. the weight and type of its input and output arcs) in
its interface; this approach of encoding the arc information is called
the \textit{place-pivot} approach. Figure~\ref{fig:arc-infos} points
out where the arc information is encoded in the interface of the
\texttt{place} design. Thus, a PCI has all the needed information to
compute the sensitization of its output TCIs by comparing the weight
of its output arcs to its current marking value.  A PCI can simply
communicate through a Boolean signal that it is currently enabling its
output TCIs. In the other approach, the \textit{transition-pivot}
approach, the \texttt{transition} design carries the arc
information. In that case, the TCIs compute their own sensitization
status. To be able to do so, the PCIs must communicate their current
marking value to the TCIs. As a marking value is a natural number, the
number of interconnecting signals between PCIs and TCIs greatly
increases in the \textit{transition-pivot} approach.  Eventually, the
\textit{place-pivot} approach has been retained in the current version
of \hilecop{}.

\begin{figure}[H]
  \centering
  \includegraphics[keepaspectratio,width=.4\textwidth]{Figures/Transformation/arcs-infos}
  \caption[The arcs information in the interface of the \texttt{place}
  design.]{Inside the \textcolor{red}{red} frame, the arc information
    encoded through the \texttt{iaw}, \texttt{oat} and \texttt{oaw}
    input ports in the interface of the \texttt{place} design.}
  \label{fig:arc-infos}
\end{figure}

The last part of the transformation deals with the interpretation
elements of the input SITPN, i.e. the conditions, the actions and the
functions. Each condition of the input SITPN leads to the declaration
of a Boolean input port in the port clause of the top-level design. As
it was pointed out in Chapter~\ref{chap:hilecop-models}
(cf. Section~\ref{subsec:pn-formalism}), the interpretation aspect has
been greatly simplified in the SITPN structure, and the generation and
the association of an input port to each condition of the input SITPN
is a consequence of the simplification. In the \textit{full} version
of the SITPN structure, a condition depends on a Boolean expression
that involves both the value of internal signals and input ports of
the top-level design. In our simplified version of the SITPN
structure, a condition value depends on the execution environment,
i.e. a function that updates the value of conditions at each falling
edge of the clock signal. Thus, we find it natural to transform each
condition into an input port of the top-level design, as the value of
both depends on the execution/simulation environment.  Then, each
input port representing a condition is connected to the \texttt{ic}
input port of TCIs. The interconnection of an input port of the
top-level design to the \texttt{ic} input port of a TCI reflects an
existing association between a transition and a condition of the input
SITPN model.

For each action and function of the input SITPN, the transformation
generates a Boolean output port, a.k.a. an \emph{action} or a
\emph{function} port. At runtime, the value of these output ports
represent the activation or execution status of the corresponding
actions and functions.  To determine the value of the action and
function ports, the transformation generates two processes: the
\texttt{action} process and the \texttt{function} process. The
\texttt{action} process is a synchronous process responding to the
falling edge of the clock signal. At the occurrence of the falling
edge of the clock signal, the \texttt{action} process sets the value
of the \emph{action} ports computed from the values of the multiple
\texttt{marked} output ports\footnote{As one action can be associated
  to multiple places, one action port can depend on the value of
  multiple \texttt{marked} output port.}. The \texttt{marked} port is
an output port of the \texttt{place} design. Through the
\texttt{marked} port, the PCIs inform the outside about their marking
status, i.e. if they possess at least one token or not.  Remember that
the transformation generated an association between the
\texttt{marked} output port and an internal signal in the output port
map of PCIs during the first transformation step. These internal
signals are read by the \texttt{action} process to assign a value to
the \textit{action} ports of the top-level design.  The
\texttt{function} process is a synchronous process responding to the
rising edge of the clock signal. At the occurrence of the rising edge
of the clock signal, the \texttt{function} process sets the value of
the \emph{function} ports computed from the values of the
\texttt{fired} output ports. The \texttt{fired} port is an output port
of the \texttt{transition} design. Through the \texttt{fired} port,
the TCIs inform the outside about their firing status, i.e. if they
are fired or not. Remember that, during the first transformation step,
the transformation generated an association between the \texttt{fired}
output port and an internal signal in the output port map of
TCIs. These internal signals are read by the \texttt{function} process
to assign a value to the \textit{function} ports of the top-level
design. Figure~\ref{fig:transf-ports} presents the top-level
$\mathcal{H}$-VHDL design at the end of the transformation.

\begin{figure}[H]
  \centering
  \includegraphics[keepaspectratio,width=\textwidth]{Figures/Transformation/transf-fun-ports}
  \caption[Generation of the input and output ports, and of the
  \texttt{action} and the \texttt{function} process in the \hvhdl{}
  top-level design.]{Generation of the input and output ports, and of
    the \texttt{action} and the \texttt{function} processes in the
    \hvhdl{} top-level design. The \emph{primary} input port
    $\mathtt{id}_{c_1}$ (resp. $\mathtt{id}_{c_0}$) implements the
    condition $c_1$ (resp. $c_0$). In \textcolor{ForestGreen}{green},
    the internal signals, generated by the transformation, connecting
    the input ports of the top-level design to the
    \texttt{input_conditions} input port of TCIs. The
    $\mathtt{id}_{a_0}$ and $\mathtt{id}_{a_1}$ output ports reflect
    the activation status of the actions $a_0$ and $a_1$. The
    $\mathtt{id}_{f_0}$ output port reflects the activation status of
    the function $f_0$. In \textcolor{orange}{orange}, the internal
    signals, generated by the transformation, connecting the
    \texttt{marked} and \texttt{fired} output ports of PCIs and TCIs
    to the \texttt{action} and \texttt{function} processes. In
    \textcolor{Purple}{purple}, the representation of the assignments
    performed by the \texttt{action} and \texttt{function} processes
    and that set the value of the action and function ports of the
    top-level design.}
  \label{fig:transf-ports}
\end{figure}

%%% Local Variables:
%%% mode: latex
%%% TeX-master: "../../main"
%%% End:


\section{Expressing transformation functions}
\label{sec:transf-lit-rev}
In this section, we present our literature review pertaining to
transformation functions in the context of formal verification. Here,
a transformation function is understood as any kind of mapping from a
source representation to a target representation, where the source and
target representations possess a behavior of their own (i.e. they are
executable). We use the same articles to perform our literature review
in this chapter and in the following chapter,
i.e. Chapter~\ref{chap:proof}. However, our research questions,
i.e. the questions we try to give an answer to while reading the
articles, and our presentation axis differ from one chapter to the
other.  Here, the following questions guide our reading:

\begin{itemize}
\item Is there a proper way to build a transformation function? Are
  there standards depending on the application domain?
\item How can we build a modular, extensible transformation function?
\item How can we build a transformation function that will ease the
  proof of semantic preservation?
\end{itemize}

The goal is to inspire ourselves with the works of the literature, and
to see how far the correspondence holds between our specific case of
transformation, and other cases of transformations. % The results of the
% literature review are presented in two parts. The two parts have
% been prepared based on the same material. The first part will be
% focusing on the expression of the transformation functions in the
% literature, and the second part will be focusing on the proof that
% these transformations are semantic preserving ones.
The material we used for the literature review is divided in three
categories. Each category covers a specific case of transformation
function. The three categories are:

\begin{itemize}
\item Compilers for generic programming languages
\item Compilers for hardware description languages
\item Model-to-model and model-to-text transformations
\end{itemize}

Note that, in the case of compilers for programming languages, the
term \textit{translation} is preferred over transformation to talk
about the generation of a target program from a source program.

\subsection{Building transformation functions}
\label{sec:build-transf}

As the authors state in \cite{Tan2016}, ``Although theoretically
possible, verifying a compiler that is not designed for verification
would be a prohibitive amount of work in practice.'' The question is
to know how to design such a compiler? How to anticipate the fact that
we will have to prove that the compiler is semantic preserving? Now,
let us consider these questions in the more general context of
transformation functions that map a source representation to a target
one.

\paragraph{Compilers for generic programming languages} In the context
of formally verified compilers for generic programming languages, the
translation from a source program to a target program is straight
forward. While descending recursively through the AST of the input
program, each construct of the source language is mapped to one or
many constructs of the target
language. Figure~\ref{fig:java-expr-to-java-bytecode} gives an example
of the translation from \java{} program expressions to \java{}
bytecode expressions, set in the context of a compiler for \java{}
programs written within the \isahol{} theorem prover
\cite{Strecker2002}. Here, the mapping between source and target
constructs is clearly defined.
\begin{figure}[H]
  \centering
  \includegraphics[keepaspectratio,width=.6\linewidth]{Figures/Proof/java-exprs-transl}
  \caption{Translation from \java{} expressions to \java{} bytecode expressions}
  \label{fig:java-expr-to-java-bytecode}
\end{figure}
In the works pertaining to the well-known \ccert{} project
\cite{Leroy2009, Blazy2006}, the many steps that compose the compiler
from C programs to assembly programs are also clearly mapping each
construct of source program to target program constructs.  Moreover,
the pattern matching possibilities offered by languages like \coq{},
\isa{}, \hol{} and other interactive theorem provers enable a clear
and concise implementation of compilers.

The cases of optimizing compilers like \cite{Leroy2009} and \cite{Tan}
show that, to avoid writing too complex functions when passing from a
source to a target program, the compilation is decomposed into many
passes. No more than 12 passes for the CakeML compiler, and up to 7
passes for \ccert{}. This is a way to keep the translation functions
simple enough in order to ease reasoning afterwards. Indeed, the more
the gap is important between the source representation and the target
one, the more the translation function will be complex.

Another point that is noticeable while expressing a translation
function is the necessity to keep a binding between the source and the
target representations. For instance, in \ccert, when passing from
transformed C programs to an RTL representation (based on registers
and control flow graphs), a binding function $\gamma$ links the
variables of a C program to the registers generated in the RTL
representation of the program. The binding is necessary for both the
translation and the proof of semantic preservation. During the
translation, it permits to replace the variables by their
corresponding registers in the RTL code. During the proof of semantic
preservation, the link that exists between a variable and a register
indicates which elements must be compared to prove that the execution
state of the source representation is similar to the execution state
of the target representation. The generation of this binding function
must be integrated to the design of the translation function.

In \cite{Leroy2009}, and \cite{Chlipala2010}, compilers are written
within the \coq{} proof assistant. Compilers are expressed using the
state-and-error monad, thus mimicking the traits of imperative
languages into a functional programming language setting. In
Section~\ref{sec:trans-alg}, we present the \hilecop{} transformation
in the form of an imperative pseudo-code algorithm. The
state-and-error monad is well-suited to the implementation of this
kind of algorithm with a functional language like \coq{}; thus, we
chose to apply this monad to our implementation of the transformation
algorithm (see Section~\ref{sec:trans-coq-impl}).

\paragraph{Compilers for hardware description languages}

The other category of compilers that we are interested in are
compilers for hardware description languages (HDL). The \hilecop{}
methodology's goal is the design of hardware circuits. For that
reason, we are interested in studying the case of compilers for
HDLs. However, one can notice that compiling an HDL program into a
lower level representation is one level of abstraction down compared
to the transformation we propose to verify. Indeed, it corresponds to
Step~3 in the \hilecop{} methodology
(cf. Figure~\ref{fig:hilecop-wf}), i.e. the transformation of VHDL
source code into an RTL representation.

In the context of formal verification applied to HDLs compilers, only
a few works describe the specificities of their translation function.

In \cite{Braibant2013}, the authors define the FeSi language (a
refinement of the BlueSpec language, a specification language for
hardware circuit behaviors), and its implementation within the \coq{}
proof assistant.  The authors present the syntax and semantics of the
FeSi language and of the RTL language which is the target language of
the compiler.  FeSi programs are composed of simple expressions, and
actions permitting to read or write from different types of memory
(registers). Therefore, the abstract syntax is divided into the
definition of expressions and the definition of actions, i.e: control
flow instructions and operations on memory. The RTL language is
composed of expressions and write operations to registers. The authors
are more interested in proving that a FeSi specification is
well-implemented by a given \coq{} program, than giving the details of
the translation from FeSi to RTL. However, the translation seems
straight-forward, and proceeds as usual by descending through the AST
of FeSi programs.

In \cite{Bourgeat2020}, the authors present a compiler for the
language Koîka, which is also a simpler version of the BlueSpec
language. A Koîka program is composed of a list of rules; each rule
describes actions that must be performed atomically. Actions are read
and write operations on registers. A Koîka program is accompanied by a
scheduler that specify an execution order for the rules. The described
compiler transforms Koîka programs into RTL descriptions of hardware
circuits. The translation function builds an RTL circuit by
descending recursively down the AST of rules. Each action is
translated into a specific RTL representation which are afterwards
composed together to get complex circuits. The translation becomes
trickier when it comes to decide the composition of RTL circuits to
respect the execution order prescribed by the scheduler.

In \cite{Bourke}, the authors present the verification of a compiler
toolchain from \textsf{Lustre} programs to an imperative language
(Obc), and from Obc to Clight.  The Clight target is the one defined
in \ccert{} \cite{Leroy2009}.  \textsf{Lustre} permits the definition
of programs composed of nodes that are executed synchronously.  Nodes
treat input streams and yield output streams of values.  A node body
is composed of sequence of equations that determine the values of
output streams based on the input.  Obc programs are composed of class
declarations. A class declaration has a vector of memory variables, a
vector of instances of other classes, and method declarations.  The
translation turns each node of a \textsf{Lustre} program into a class
of Obc accompanied by two methods: the reset method, for the
initialization of the streams, and the step method, for the update of
values resulting of a synchronous step.

In \cite{Loow2021}, the authors describe a compiler that transforms
Verilog programs into netlists targeting given FPGA models. Verilog
programs are a lot like VHDL programs; they describe a hardware
circuit behavior in terms of processes. A netlist is composed of
registers, variables and a list of cells corresponding to
combinational components. During the translation process, the
expressions of the Verilog programs are turned into netlist cells, and
the composition of statements leads to the creation of complex
circuits by means of cell composition.

\paragraph{Model transformations}

We will now present the works pertaining to model-to-model and
model-to-text transformations in the context of formal
verification. Because of the nature of the transformation we propose
to verify, i.e a model-to-text transformation, the following works are
of particular interest to us. We will focus here on the manner to
express transformations in the case of model-to-model and
model-to-text transformations. Also, we tried to find articles related
to model transformations involving Petri nets.

In \cite{Berramla2015}, the authors observe that Model-Driven
Engineering (MDE) is all about model transformation operations. They
propose to set a formal context within the \coq{} proof assistant to
verify that model transformations preserve the structure of the source
models into the target models. To illustrate their methodology, they
choose to transform UML state machine diagrams into Petri net
models. The translation rules from source to target models are
expressed within the setting of the OMG standard QVT language
(Query/View/Transform). The QVT language offers a formal way to
express model transformations, partly based on the Object Constraint
Language (OCL). The translation rules map the different kind of
structures that can be found in UML state diagrams to specific
structures of Petri nets. Even though the two models used as source
and target of transformations are executable, the authors leverage the
formal context provided by \coq{} to prove that the expressed
transformations preserve certain structural properties.

In \cite{Calegari2011}, the authors describe a process for model
transformation where transformation rules are expressed with the Atlas
Transformation Language (ATL). Transformation rules in ATL involve
both declarative (OCL) and imperative (match rules) instructions. The
authors show how the ATL rules can easily be translated into \coq{}
relations. An example is given on the kind of model-to-model
transformations that can be implemented that way. The example is a UML
class diagram to relational database model transformation.

In \cite{Combemale2009}, the authors explore the different ways to
give a formal semantics to a Domain-Specific Language (DSL) in the
context of MDE. Here, the syntax of a given DSL is expressed with a
meta-model.  An instantiation of this meta-model (a model) yields a
DSL program. The authors specify a transformation from a DSL model to
another executable model, thus providing an \textit{translational}
semantics to the DSL model.  The authors illustrate their approach
with a source DSL named xSPEM, which is a process description
language. The target models are timed PNs. The transformation is
expressed through a structural mapping; i.e, each element of an xSPEM
model is mapped to a particular PN: an activity is mapped to a subnet,
a resource to a single place, connection from activity to resource
through parameter is mapped to a connection of transitions and places
in the resulting PN\dots 

In \cite{Dyck2019}, the authors address the problem of expressing
model transformations by using transformation graphs. Precisely, the
kind of transformation graphs that are used are called Triple Graph
Grammar (TGG). A TGG is a triplet ${<}s,c,t{>}$ where the
``correspondence model $c$ explicitly stores correspondence
relationships between source model $s$ and target model $t$''.

The work described is \cite{Fronc2011} is really close to our own
verification task. The article describes how Coloured Petri Nets
(CPNs, specifically LLVM-labelled Petri nets) are transformed into
LLVM (Low Level Virtual Machine) programs representing the state space
(the graph of reachable markings) of these PNs. LLVM is a low-level
assembly language. The aim is to enable an efficient model-checking of
the CPNs.  LLVM-labelled PNs are CPNs whose places, transitions and
arcs have LLVM constructs for color domains. Places are labelled with
data types.  Transitions are labelled with boolean expressions that
correspond to the guard of the transition. Arcs are labelled by
multisets of expressions. A marking is a function that maps each place
to a multiset of values belonging to the place's type.  The authors
define data structures (multisets, sets, markings\dots) with
interfaces, i.e. sets of operations over structures, to represent the
Petri nets in LLVM.  They define interpretation functions that draw
equivalences between Petri nets objects and LLVM data structures.  The
authors define two algorithms: $\mathtt{fire\_t}$ and
$\mathtt{succ\_t}$ to compute the graph of reachable states.  These
are the functions that transform CPNs into concrete LLVM programs.

In \cite{Meghzili2017}, the author describes a transformation from UML
state machine diagrams to Coloured Petri Nets (CPNs). The aim is to
leverage the means of analysis provided by Petri nets to certify
certain properties over UML state machine diagrams. The authors want
to verify that the transformation preserve structural properties
between source and target models. The transformation function does not
use a standard setting as QVT or ATL, or transformation graphs. It is
expressed as a specific function written in \isahol.

In \cite{Yang2014}, the author presents a transformation from
Architecture Analysis and Design Language (AADL) models to Timed
Abstract State Machines (TASMs). AADL is a language widely used in
avionics to describe both hardware and software systems. AADL doesn't
have a lot of tools to analyze and simulate the designed systems;
therefore transforming AADL models into TASM enables the use of an
important toolbox for analysis, and simulation. The transformation
from AADL to TASMs are described with ATL rules.

\paragraph{Discussions on how to build transformation functions in the
  context of semantic preservation}

Transformation functions are mappings from a source representation to
a target representation. The more the mapping from source to target is
straight-forward the easier the comparison will be when proving that
the transformation is semantic preserving. Thus, in
\cite{Leroy2009,Tan,Chlipala2010} where complex case of optimizing
compilers are presented, the compilation is split into many simple
passes to ease the verification effort coming afterwards. In the case
of the \hilecop{} transformation, we are not yet concerned with the
optimization of the generated \vhdl{} code. Thus, our transformation
algorithm performs the generation of the target \hvhdl{} design in a
single pass. We do not need to use intermediary representations
between the input SITPN model and the generated \hvhdl{} design.

Also, while transforming source programs, the compiler must often
generate fresh constructs belonging to the target language (for
instance, generating a fresh RTL register for each variable referenced
in a source C program in \cite{Leroy2009}). The compiler must keep a
binding, that is, a memory of the mapping between the elements of the
source program and their mirror in the target program. This
consideration is of interest in our case of transformation where the
elements of SITPNs are also mirrored by elements in the generated
\hvhdl{} design.

It remains hard to establish a standard way to express a
transformation function as it really depends on the form of the input
and the output representations. Compilers for programming languages
tend to be a lot more compositional than model transformations. Here,
the word \textit{compositional} means that the translation rules can
be split into simple and independent cases of translation, e.g.
translation of expressions, then translation of statements, then
translation of function bodies,\dots This is a huge advantage to
perform the proof of semantic preservation. Indeed, this decomposition
of a translation function permits to reason on simple translation
cases; yet, each of these translations cases yields a piece of target
code that can be executed or interpreted in an independent manner. In
the case of the \hilecop{}, we tried as much as possible to express
the transformation in a compositional way. First, we tried to devise
the transformation by building up transformation functions for each
element of the SITPN structure, i.e.: a transformation function for
the places, another for the transitions\dots However, due to the
interconnections that exist between the component instances of the
generated \hvhdl{} design, it is impossible to define transformation
functions that would yield stand-alone executable code.

In the world of models, there exist some standard formalisms to
express transformation rules (QVT, ATL, transformation graphs\dots).
However, the complexity of the transformation rules depends on the
richness of the elements composing the source model, and the distance
to the concepts of the target model. In our case, we were not able to
grab the perks of using such formalisms as QVT or ATL to devise our
transformation.




%%% Local Variables:
%%% mode: latex
%%% TeX-master: "../../main"
%%% End:


\section{The transformation algorithm}
\label{sec:trans-alg}
In this section, we give the algorithm underlying the \hilecop{}
model-to-text transformation. This algorithm is the base of the \coq{}
implementation of the \hilecop{} transformation; the implementation is
presented in Section~\ref{sec:trans-coq-impl}. As stated in
Chapter~\ref{chap:hilecop}, there exists a Java implementation of the
\hilecop{} methodology. This implementation embeds the generation of
VHDL code from an SITPN model. However, the algorithm of the
transformation has never been devised, nor a formal specification
given. The following algorithm is one of the contribution of this
thesis. It has been devised through the examination of the code of the
existing Java implementation, and through the discussions with the
designers of the \hilecop{} methodology.

\subsection{The \texttt{sitpn_to_hvhdl} function}
\label{sec:sitpn-to-hvhdl}

The \hilecop{} transformation algorithm, presented in
Algorithm~\ref{alg:sitpn2hvhdl}, generates a \hvhdl{} design and a
SITPN-to-\hvhdl{} binder from an input SITPN. A SITPN-to-\hvhdl{}
design binder is a structure that binds the elements of an SITPN
(places, transitions, actions\dots) to the elements of a \hvhdl{}
design (component instances or signals). As it is generated along the
transformation, the binder links a SITPN element to its \hvhdl{}
\textit{implementation}, i.e. the \hvhdl{} element that will
supposedly behave similarly to the source SITPN element at
runtime. Thus, the SITPN-to-\hvhdl{} design binder is at the center of
the state similarity relation, presented in Chapter~\ref{chap:proof},
and that enables the comparison between an SITPN state and an \hvhdl{}
design state. The formal definition of an SITPN-to-\hvhdl{} design
binder is as follows.

\begin{definition}[SITPN-to-\hvhdl{} design binder]
  \label{def:sitpn-to-hvhdl-binder}
  Given a $sitpn\in{}SITPN$ and a \hvhdl{} design $d\in{}design$, a
  SITPN-to-\hvhdl{} design binder $\gamma\in{}WM(sitpn,d)$ is a tuple\\
  ${<}PMap,TMap,CMap,AMap,FMap{>}$
  where:
  \begin{itemize}
  \item
    $sitpn={<}P,T,pre,test,inhib,post,M_0,{\succ},\mathcal{A},\mathcal{C},\mathcal{F},
    \mathbb{A},\mathbb{C},\mathbb{F},{I_s}{>}$
  \item $d=$ \vhdle|design| \textit{$id_{e}$ $id_{a}$ gens ports sigs cs}
  \item $PMap\in{}P\rightarrow{}\{id~|~\mathtt{comp}(id,\mathtt{place},g,i,o)\in{}cs\}$
  \item $TMap\in{}T\rightarrow{}\{id~|~\mathtt{comp}(id,\mathtt{transition},g,i,o)\in{}cs\}$
  \item $CMap\in\mathcal{C}\rightarrow\{id~|~(\mathtt{in}, id, t)\in{}ports\}$
  \item $AMap\in\mathcal{A}\rightarrow\{id~|~(\mathtt{out}, id, t)\in{}ports\}$
  \item $FMap\in\mathcal{F}\rightarrow\{id~|~(\mathtt{out}, id, t)\in{}ports\}$        
  \end{itemize}
\end{definition}

As presented in Definition~\ref{def:sitpn-to-hvhdl-binder}, the binder
is composed of five sub-environments that map the different SITPN sets
to identifiers. The $PMap$ and $TMap$ sub-environments map the places
to their corresponding PCI identifiers, and the transitions to their
corresponding TCI identifiers. The $CMap$ sub-environment maps the
conditions to input port identifiers. The $AMap$ and $FMap$
sub-environments map the actions and functions to output port
identifiers. In what follows, for a given binder $\gamma$ and an
element of an SITPN structure
$e\in{}P\sqcup{}T\sqcup\mathcal{C}\sqcup\mathcal{A}\sqcup\mathcal{F}$,
we write $\gamma(e)$ where $e$ is looked up in the appropriate
function. For instance, for a given $f\in\mathcal{F}$, $\gamma(f)$ is
a shorthand for $FMap(f)$ where $\gamma={<}\dots,FMap{>}$.

Algorithm~\ref{alg:sitpn2hvhdl} is the algorithm of the \hilecop{}
model-to-text transformation. The algorithm as four parameters; the
first one is the input SITPN model; $id_e$ and $id_a$ are the entity
and the architecture identifiers for the generated \hvhdl{} design;
$mmf\in{}P\rightarrow\mathbb{N}$ is the function associating a maximal
marking value to each place of the input SITPN. This function is the
result of the analysis of the input SITPN.

\begin{remark}[Bounded SITPN]
  \label{rem:bounded-sitpn}
  A part of the analysis is interested in determining the maximal
  number of tokens that a place can hold during the execution of a
  SITPN. If each place of the SITPN can only hold a limited number of
  tokens during the execution, then the model is said to be
  bounded. In that case, a function associated the places with a
  maximal marking value can be computed. Thus, the presence of the
  $mmf$ function as a parameter of the \texttt{sitpn\_to\_hvhdl}
  function implies that the input SITPN model is bounded. In the case
  of an unbounded input model, there exists a place that can
  accumulate an infinite number of tokens during the model
  execution. In the world of hardware description, and especially when
  aiming at the hardware synthesis, every element must have a finite
  dimension. In the definition of the \texttt{place} design, the
  internal signal \texttt{s\_marking} represents the marking value of
  a place. The maximal value of the \texttt{s\_marking} signal is
  bounded by the generic constant \texttt{maximal\_marking}. Thus,
  when generating, a PCI from a place in the course of the
  transformation, we must be able to give a value to the
  \texttt{maximal\_marking} generic constant. However, even with a
  settled \texttt{maximal\_marking} value, the execution of a \hvhdl{}
  design, resulting from the transformation of an unbounded SITPN
  model, could lead to the overflow of the value of the
  \texttt{s\_marking} signals in the internal states of PCIs.  Thus,
  it is impossible to prove the equivalence between the behavior of an
  unbounded SITPN model and its corresponding \hvhdl{} design. 
\end{remark}

\begin{algorithm}[H]
  \DontPrintSemicolon

  \SetAlFnt{\fontsize{11}{13}\selectfont}
  
  \SetAlCapFnt{\fontsize{11}{13}\selectfont}
  \SetAlCapNameFnt{\fontsize{11}{13}\selectfont}
  % \NoCaptionOfAlgo

  \caption{\texttt{sitpn\_to\_hvhdl}($sitpn$, $id_e$, $id_a$, $mmf$)}
  \label{alg:sitpn2hvhdl}
  
  \AlFnt % overriding the new font

  $d\leftarrow\mathtt{design}$ $id_e$ $id_a$ $\emptyset$ $\emptyset$ $\emptyset$ $\mathtt{null}$\; \label{line:init-design}
  $\gamma\leftarrow\emptyset$\; \label{line:init-binder}
  \BlankLine

  \texttt{generate\_architecture}($sitpn$, $d$, $\gamma$, $mmf$)\; \label{line:genarch}
  \texttt{generate\_interconnections}($sitpn$, $d$, $\gamma$)\; \label{line:geninter}
  \texttt{generate\_ports}($sitpn$, $d$, $\gamma$)\; \label{line:genports}
  \BlankLine  

  \Return{($d$,$\gamma$)}\;
\end{algorithm}

In Algorithm~\ref{alg:sitpn2hvhdl}, Line~\ref{line:init-design}
creates the \hvhdl{} design. Initially, the design has an empty port
declaration set, an empty internal signal declaration set, and a
behavior defined by the \texttt{null} statement. The design generated
by the \texttt{sitpn\_to\_hvhdl} function has an empty set of generic
constant, even at the end of the
transformation. Line~\ref{line:init-binder} initializes the $\gamma$
binder with empty sub-environments. From Lines~\ref{line:genarch} to
\ref{line:genports}, the called procedures modify the design and the
binder structures. Each part of the sequence corresponds to one step
of the transformation, which were outlined in
Section~\ref{sec:informal-transf}. The content of the
\texttt{generate\_architecture} function is detailled in Algorithms,
the content of the \texttt{generate\_interconnections} function is
detailled in Algorithms, and the content of the
\texttt{generate\_ports} function is detailled in Algorithms.

\subsection{Primitive functions and sets}
\label{sec:prim-funs}

The description of further functions and algorithms appeals to some
primitive functions and set definitions that we introduce here. Below
are all the sets that we use in the description of the algorithms.

\begin{itemize}
\item
  \texttt{input}$(p)=\{t~\vert~\exists\omega~s.t.~post(t,p)=\omega\}$,
  the set of input transitions of a place $p$.
\item
  \texttt{output}$(p)=\{t~\vert~\exists{}\omega,a~s.t.~pre(p,t)=(\omega,a)\}$,
  the set of output transitions of a place $p$.

\item \texttt{acts}$(p)=\{a~\vert~\mathbb{A}(p,a)=\mathtt{true}\}$,
  the set of actions associated with a place $p$.
\item
  \texttt{input}$(t)=\{p~\vert~\exists\omega,a~s.t.~pre(p,t)=(\omega,a)\}$,
  the set of input places of a transition $t$.
\item
  \texttt{output}$(t)=\{p~\vert~\exists\omega~s.t.~post(t,p)=\omega\}$,
  the set of output places of a transition $t$.
\item
  \texttt{conds}$(t)=\{c~\vert~\mathbb{C}(t,c)=1\lor\mathbb{C}(t,c)=-1\}$,
  the set of conditions associated with a transition $t$.
\item
  \texttt{trs}$(c)=\{t~\vert~\mathbb{C}(t,c)=1\lor\mathbb{C}(t,c)=-1\}$,
  the set of transitions to which a condition $c$ is associated.
\item \texttt{pls}$(a)=\{p~\vert~\mathbb{A}(p,a)=\mathtt{true}\}$, the
  set of places to which an action $a$ is associated.
\item \texttt{trs}$(f)=\{t~\vert~\mathbb{F}(t,f)=\mathtt{true}\}$, the
  set of transitions to which a function $f$ is associated.
\end{itemize}

Every above set are unordered. However, we assume that, every time we
iterate over the elements of an unordered set with a \textbf{foreach}
statement, the iteration respects an arbitrary order. This order is
always the same through the multiple calls to \textbf{foreach}
statements.

Now, let us introduce some primitive functions and procedures that we
use in the description of the following algorithms.

\begin{itemize}
\item \texttt{output}$_c(p)$. The function operates the following
  sequence:
  \begin{enumerate}
  \item if all conflicts between the output transitions of $p$ are
    solved by mutual exclusion, or if the set of conflicting
    transitions of $p$ is a singleton, then returns an empty set.
  \item otherwise, tries to establish a total ordering over the set of
    conflicting transitions of $p$ w.r.t the firing priority relation:
    \begin{itemize}
    \item raises an error if no such ordering can be established (in
      that case, the firing priority relation is ill-formed, and the
      input SITPN is not well-defined).
    \item returns the ordered set, with the top-level priority
      transition at the head.
    \end{itemize}
  \end{enumerate}

\item \texttt{output}$_{nc}(p)$. If all conflicts between the output
  transitions of $p$ are solved by mutual exclusion, or if the set of
  conflicting transitions of $p$ is a singleton, then, the function
  returns the set of output transitions of $p$, i.e
  \texttt{output}$(p)$ as defined above. Otherwise, the function
  returns the set of output transitions of $p$ connected through a
  \texttt{test} or an \texttt{inhib} arc,
  i.e.
  $\{t~\vert~\exists\omega~s.t.~pre(p,t)=(\omega,\mathtt{test})\lor{}pre(p,t)=(\omega,\mathtt{inhib})\}$.

\item \texttt{cassoc}$(map,id,x)$ where $map$ is either a generic map,
  an input port map or an output port map, $id$ is an identifier, $x$
  is an expression, a name (i.e. an simple or indexed identifier) or
  the \texttt{open} keyword. The \texttt{cassoc} procedure adds an
  association of the form $(id(i),x)$ to the $map$ structure. The
  index $i$ is computed as follows based on the content of $map$:
  \begin{enumerate}
  \item looks up $id(j)$ with $max(j)$ in the formal parts of $map$
  \item if no such $j$, adds $(id(0),x)$ in $map$
  \item if such $j$, adds $(id(j+1),x)$ in $map$
  \end{enumerate}

  \noindent{}\textit{Examples:}
  \begin{itemize}
  \item
    \texttt{cassoc}$(~\mathtt{((s(0),\mathtt{true}),(s(1),\mathtt{false}))},~\mathtt{s},~\mathtt{true})$
    yields the resulting map \\
    $\mathtt{((s(0),\mathtt{true}),(s(1),\mathtt{false}),(s(2),\mathtt{true}))}$.
  \item
    \texttt{cassoc}$(~\mathtt{((s(0),\mathtt{true}),(s(1),\mathtt{false}))},~\mathtt{a},~\mathtt{open})$
    yields the resulting map \\
    $\mathtt{((s(0),\mathtt{true}),(s(1),\mathtt{false}),(a(0),\mathtt{open}))}$.
  \end{itemize}

\item \texttt{get\_comp}$(id_c, cstmt)$ where $id_c$ is an
  identifier, and $cstmt\in{}\mathrm{cs}$ is a \hvhdl{} concurrent
  statement. The \texttt{get\_comp} function looks up $cstmt$ for a
  component instantiation statement labelled with $id_c$ as a
  component instance identifier, and returns the component
  instantiation statement when found. The \texttt{get\_comp} function
  throws an error if no component instantiation statement with
  identifier $id_c$ exists in $cstmt$, or if there ecist multiple
  component instantiation statements with identifier $id_c$ in
  $cstmt$.
\item \texttt{put\_comp}$(id_c,cistmt,cstmt)$ where $id_c$ is an
  identifier, $cistmt$ is a component instantiation statement, and
  $cstmt\in{}\mathrm{cs}$ is a \hvhdl{} concurrent statement. The
  \texttt{put\_comp} procedure looks up in $cstmt$ for a component
  instantiation statement with identifier $id_c$, and replaces the
  statement with $cistmt$ in $cstmt$. If no CIS with identifier $id_c$
  exists in $cstmt$, then $cistmt$ is composed with $cstmt$ with the
  \texttt{||} operator. The \texttt{put\_comp} procedure throws an
  error if multiple CIS with identifier $id_c$ exist in $cstmt$.
\item \texttt{actual}$(id,map)$ where $id$ is an identifier and $map$
  is a generic, an input port or an output port map. The
  \texttt{actual} function returns the actual part associated with the
  formal part $id$ in $map$, i.e. returns $a$ if $(id,a)\in{}map$. The
  function throws an error if $id$ is not a formal part in $map$, or
  if there are multiple association with $id$ as a formal part in
  $map$.
\item \texttt{genid}$()$. The \texttt{genid} function returns a fresh
  and unique identifier. During the transformation, we appeal to it
  when a new internal signal, a new port or a new component instance
  must be declared or generated.
\end{itemize}

Algorithm~\ref{alg:connect} presents the \texttt{connect} procedure.
The procedure takes an output port map $o$, an input port map $i$, two
identifiers $id_1,id_2$ and a \hvhdl{} design $d$ as parameters. The
procedure generates an internal signal $id_s$ and adds it to the
internal signal declaration list of design $d$. Then, the procedure
connects a subelement of port $id_1$ (resp. $id_2$) to the internal
signal $id_s$ in the output port map $o$ (resp. input port map $i$).
As the result, a subelement of $id_1$ is connected to a subelement of
$id_2$ through the internal signal $id_s$.

\begin{algorithm}[H] \DontPrintSemicolon

  \SetAlFnt{\fontsize{11}{13}\selectfont}
  
  \SetAlCapFnt{\fontsize{11}{13}\selectfont}
  \SetAlCapNameFnt{\fontsize{11}{13}\selectfont} % \NoCaptionOfAlgo

  \caption{\texttt{connect}($o$, $i$, $n$, $id$, $d$)}
  \label{alg:connect}

  $id_s\leftarrow\mathtt{genid}()$\;
  $d.sigs\leftarrow{}d.sigs\cup(id_s,\mathtt{boolean})$\;
  $o\leftarrow{}o\cup(n,id_s)$\; \texttt{cassoc}$(i,id,id_s)$\;
\end{algorithm}

\paragraph{Complementary notations}
When there is no ambiguity, $id_p$ (resp. $id_t$) denotes the PCI
(resp. TCI) identifier associated to a given place $p$
(resp. transition $t$) through $\gamma(p)=id_p$
(resp. $\gamma(t)=id_t$), where $\gamma$ is the binder returned by the
transformation function. Similarly, $id_c$ (resp. $id_a$ and $id_f$)
denotes the input port (resp. output port) identifier associated to a
given condition $c$ (resp. action $a$ and function $f$) through
$\gamma(c)=id_c$.

\subsection{Generation of component instances and constant parts}
\label{sec:genarch}

The first step of the transformation generates the PCIs and TCIs,
their generic map, and the constant part of their input port maps, in
the behavior of the \hvhdl{} design. This is when places are bound to
PCI identifiers, and transitions are bound to TCI identifiers in the
$\gamma$ binder. Also, the \texttt{marked} output port and the
\texttt{fired} output port are connected in the output port map of
PCIs and TCIs during this first generation
step. Algorithm~\ref{alg:genarch} presents the content of the
\texttt{generate\_architecture} procedure that implements this first
part of code generation. The \texttt{generate\_architecture} procedure
is decomposed in two procedures: the \texttt{generate\_PCIs} and the
\texttt{generate\_TCIs} procedures.

\begin{algorithm}[H]
  \DontPrintSemicolon

  \SetAlFnt{\fontsize{11}{13}\selectfont}
  
  \SetAlCapFnt{\fontsize{11}{13}\selectfont}
  \SetAlCapNameFnt{\fontsize{11}{13}\selectfont}
  % \NoCaptionOfAlgo

  \caption{\texttt{generate\_architecture}($sitpn$, $d$, $\gamma$, $mmf$)}
  \label{alg:genarch}
  
  \AlFnt % overriding the new font

  \texttt{generate\_PCIs}($sitpn$, $d$, $\gamma$, $mmf$)\; \label{line:genpcis}
  \texttt{generate\_TCIs}($sitpn$, $d$, $\gamma$)\; \label{line:gentcis}

\end{algorithm}

Note that, in the following algorithms and explanations, we refer to
the generic constant, internal signal and port identifiers defined in
the \texttt{place} and \texttt{transition} designs through their
abbreviated names (see Table~\ref{tab:consts-and-sigs-ref}).

\begin{algorithm}[H]
  \DontPrintSemicolon

  \SetAlFnt{\fontsize{11}{13}\selectfont}
  
  \SetAlCapFnt{\fontsize{11}{13}\selectfont}
  \SetAlCapNameFnt{\fontsize{11}{13}\selectfont}
  % \NoCaptionOfAlgo

  \caption{\texttt{generate\_PCIs}($sitpn$, $d$, $\gamma$, $mmf$)}
  \label{alg:genpcis}
  
  \AlFnt % overriding the new font

  \ForEach{$p\in{}P$}{

    \lIf{$\mathtt{input}(p)=\emptyset$ and $\mathtt{output}(p)=\emptyset$}{\label{line:p-isolated}
      \texttt{err}$("p~is~an~isolated~place")$\label{line:p-isolated-err}
    }
    
    \BlankLine
    $g_p\leftarrow\{(\mathtt{mm},mmf(p))\}$; 
    $i_p\leftarrow\emptyset$; 
    $o_p\leftarrow\emptyset$\; \label{line:init-pcomp}

    \BlankLine
    \uIf{$\mathtt{input}(p)=\emptyset$}{ \label{line:genpcis-inp-p-empty}
      $g_p\leftarrow{}g_p\cup\{(\mathtt{ian},1)\}$\; \label{line:ian-empty}
      $i_p\leftarrow{}i_p\cup\{(\mathtt{iaw}(0),0),(\mathtt{itf}(0),\mathtt{false})\}$\; \label{line:iaw-itf-empty}
    }
    \Else{
      $g_p\leftarrow{}g_p\cup\{(\mathtt{ian},\vert\mathtt{input}(p)\vert)\}$\; \label{line:ian-not-empty}
      $i\leftarrow{}0$\;
      \ForEach{$t\in\mathtt{input}(p)$}{ \label{line:foreach-iaw}
        $i_p\leftarrow{}i_p\cup\{(\mathtt{iaw}(i),post(t,p))\}$\;
        $i\leftarrow{}i+1$\;
      }
    }

    \BlankLine
    \uIf{$\mathtt{output}(p)=\emptyset$}{\label{line:genpcis-out-p-empty}
      $g_p\leftarrow{}g_p\cup\{(\mathtt{oan},1)\}$\; \label{line:oan-empty}
      $i_p\leftarrow{}i_p\cup\{(\mathtt{oaw}(0),0),(\mathtt{oat}(0),\mathtt{basic}),(\mathtt{otf}(0),\mathtt{false})\}$\;
      \label{line:genpcis-out-p-empty-ip}
      $o_p\leftarrow{}o_p\cup\{(\mathtt{oav},\mathtt{open}),(\mathtt{pauths},\mathtt{open}),(\mathtt{rtt},\mathtt{open})\}$\;
      \label{line:genpcis-out-p-empty-op}
    }
    \Else{
      $i\leftarrow{}0$\;
      \ForEach{$t\in\mathtt{output}_c(p)\cup\mathtt{output}_{nc}(p)$}{\label{line:begin-oaw}
        $(\omega,a)\leftarrow{}pre(p,t)$\;
        $i_p\leftarrow{}i_p\cup\{(\mathtt{oaw}(i),\omega),(\mathtt{oat}(i),a)\}$\;
        $i\leftarrow{}i+1$\;
      }
    }

    \BlankLine
    \lIf{$\mathtt{acts}(p)=\emptyset$}{$o_p\leftarrow{}o_p\cup\{(\mathtt{marked},\mathtt{open})\}$}\label{line:genpcis-acts-empty}
    \Else{
      $id_s\leftarrow{}\mathtt{genid}()$\;
      $d.sigs\leftarrow{}d.sigs\cup\{(id_s,\mathtt{boolean})\}$\;
      $o_p\leftarrow{}o_p\cup\{(\mathtt{marked},id_s)\}$\; \label{line:end-op}
    }
    
    \BlankLine
    $id_p\leftarrow{}\mathtt{genid}()$\;\label{line:gen-pcomp-id}
    $d.cs\leftarrow{}d.cs~\mathtt{||}~\mathtt{comp}(\mathtt{id}_p,\mathtt{place},g_p,i_p,o_p)$\;\label{line:gen-pcomp}
    $\gamma\leftarrow\gamma\cup\{(p,\mathtt{id}_p)\}$\;\label{line:gen-bind-p}
  }
\end{algorithm}

The \texttt{generate\_PCIs} procedure, presented in
Algorithm~\ref{alg:genpcis}, has four parameters: $sitpn\in{}SITPN$,
the input SITPN model; $d\in{}design$, the \hvhdl{} design being
generated; $\gamma\in{}WM(sitpn,d)$, the binder between $sitpn$ and
$d$; $mmf\in{}P\rightarrow\mathbb{N}$, the function assigning a
maximal marking value to each place. The procedure iterates over the
set of places of the $sitpn$ parameter. For each place $p$ in the set,
the procedure produces a corresponding PCI $id_p$, and generates its
generic map $g_p$, and its partially-built input and output port maps
$i_p$ and $o_p$. As said before, the only associations generated in
the input and output port maps during this phase pertain to the
association of ports to constant values. At the end of the procedure
(Lines~\ref{line:gen-pcomp-id} to \ref{line:gen-bind-p}), a fresh and
unique component identifier $id_p$ is generated, and a new CI
statement, corresponding to the instantiation of PCI $id_p$, is
composed with the current behavior of design $d$. Finally, the
$\gamma$ binder receives a new couple corresponding to binding of
place $p$ to identifier $id_p$.

From Line~\ref{line:p-isolated} to Line~\ref{line:end-op}, the
procedure generates the generic map, the input port map, and the
output port map of the PCI. First, the procedure checks if the current
place $p$ is isolated, i.e. without input nor output transitions. An
error, with an associated message, is raised with the \texttt{err}
primitive if the test succeeds. The \hilecop{} transformation raises
errors in the presence of an input SITPN model that does not meet the
well-definition property (see Definition~\ref{def:wd-sitpn}). One part
of the well-definition property pertains to the absence of isolated
place in the input model. Line~\ref{line:init-pcomp} initializes the
variables $g_p$, $i_p$ and $o_p$, respectively holding the generic
map, the input port map and the output port map of the generated
PCI. The generic map $g_p$ initially takes a single association that
binds the \texttt{mm} constant to the maximal marking value returned
by the $mmf$ function for place $p$. The input port map $i_p$ and the
output port map $o_p$ are initialized with empty sets.

Line~\ref{line:genpcis-inp-p-empty} tests if the set of input
transitions of $p$ is empty. If the test succeeds, the \texttt{ian}
constant is associated to $1$ in the generic map $g_p$. The size of
the \texttt{iaw} and \texttt{itf} input ports, which are of the array
type, is equal to the value of the \texttt{ian} constant minus
one. Thus, the \texttt{iaw} and \texttt{itf} input ports are composed
of one subelement with index $0$. At Line~\ref{line:iaw-itf-empty},
the sole subelement of the \texttt{iaw} port is associated with $0$,
and the sole subelement of the \texttt{itf} port is associated with
\texttt{false} in the input port map $i_p$. If the set of input
transitions of $p$ is not empty, the \texttt{ian} constant is
associated with the size of the set in the generic map $g_p$. Then,
each subelement of the \texttt{iaw} port is associated with the weight
of the arc between place $p$ and an input transition $t$.  Note that,
in that case, the procedure does not deal with the connection of the
\texttt{itf} port. As the set of input transitions of $p$ is not
empty, the connection of the \texttt{itf} port will be performed by
the \texttt{connect_input_trs} procedure during the generation of the
interconnections between PCIs and TCIs.

Line~\ref{line:genpcis-out-p-empty} tests if the set of output
transitions of $p$ is empty. If the test succeeds, the \texttt{oan}
constant is associated to $1$ in the generic map $g_p$. The size of
the \texttt{oaw}, \texttt{oat} and \texttt{otf} input ports, which are
of the array type, is equal to the value of the \texttt{oan} constant
minus one. Thus, the \texttt{oaw}, \texttt{oat} and \texttt{otf} input
ports are composed of one subelement with index $0$. At
Line~\ref{line:genpcis-out-p-empty-ip}, the sole subelement of the
\texttt{oaw} port is associated with $0$, the sole subelement of the
\texttt{oat} port is associated with \texttt{basic}, and the sole
subelement of the \texttt{otf} port is associated with \texttt{false}
in the input port map $i_p$. Also, in the abscence of output
transitions, the \texttt{oav}, \texttt{pauths} and \texttt{rtt} output
ports are left unconnected, i.e. they are associated with the
\texttt{open} keyword of output port map $o_p$.

If the set of output transitions of $p$ is not empty, the \texttt{oan}
constant is associated with the size of this set in the generic map
$g_p$. Then, each subelement of the \texttt{oaw} (resp. the
\texttt{oat}) port is associated with the weight (resp. the type) of
the arc between place $p$ and an output transition $t$.  Note that, in
that case, the procedure does not handle the connection of the
\texttt{otf} input port, nor the connection of the \texttt{oav},
\texttt{pauths} and \texttt{rtt} output ports. As the set of output
transitions of $p$ is not empty, these connections will be performed
by the \texttt{connect_output_trs} procedure during the generation of
the interconnections between PCIs and TCIs.

From Line~\ref{line:genpcis-acts-empty} to Line~\ref{line:end-op}, the
procedure connects the \texttt{marked} output port in the output port
map $o_p$. If the place $p$ is not associated with any action, the
\texttt{marked} output port is left unconnected, i.e. connected to the
\texttt{open} keyword. Otherwise, the \texttt{marked} output port is
connected to a newly generated internal signal of the Boolean
type. This generated signal joins the internal signal declaration list
of design $d$. The connection between the \texttt{marked} output port
and the internal signal will be used later, during the generation of
the \texttt{action} process (see Section~\ref{sec:genports}).

\begin{algorithm}[H]
  \DontPrintSemicolon

  \SetAlFnt{\fontsize{11}{13}\selectfont}
  
  \SetAlCapFnt{\fontsize{11}{13}\selectfont}
  \SetAlCapNameFnt{\fontsize{11}{13}\selectfont}
  % \NoCaptionOfAlgo

  \caption{\texttt{generate\_TCIs}($sitpn$, $d$, $\gamma$)}
  \label{alg:gentcis}
  
  \AlFnt % overriding the new font

  \ForEach{$t\in{}T$}{

    \lIf{$\mathtt{input}(t)=\emptyset$ and $\mathtt{output}(t)=\emptyset$}{
      \texttt{err}$("t~is~an~isolated~transition")$\label{line:t-isolated}
    }

    \BlankLine
    $g_t\leftarrow\{(\mathtt{tt},\mathtt{get\_ttype}(t)),(\mathtt{mtc},\mathtt{get\_mtc}(t))\}$\;\label{line:init-gt}
    $i_t\leftarrow\{(\mathtt{A},
    \begin{cases}
      0~\mathtt{if}~t\notin{}\mathtt{dom}(I_s) \\
      lower(I_s(t))~otherwise\\
    \end{cases}), (\mathtt{B},
    \begin{cases}
      0~\mathtt{if}~t\notin{}\mathtt{dom}(I_s)\lor{}upper(I_s(t))=\infty \\
      upper(I_s(t))~otherwise\\
    \end{cases})\}$\;\label{line:init-it}

    $id_s\leftarrow{}\mathtt{genid()}$\;\label{line:genid-fired}
    $d.sigs\leftarrow{}d.sigs\cup(id_s,boolean)$\;\label{line:declid-fired}
    $o_t\leftarrow{}\{(\mathtt{fired},id_s)\}$\;\label{line:connect-fired}
    
    \BlankLine
    \uIf{$\mathtt{input}(t)=\emptyset$}{\label{line:gentcis-input-empty}
      $g_t\leftarrow{}g_t\cup\{(\mathtt{ian},1)\}$\;
      $i_t\leftarrow{}i_t\cup\{(\mathtt{iav}(0),\mathtt{true}),(\mathtt{pauths}(0),\mathtt{true}),(\mathtt{rt}(0),id_s)\}$\;
      \label{line:gentcis-in-empty-it}
    }
    \Else{
      $g_t\leftarrow{}g_t\cup\{(\mathtt{ian},\vert{}\mathtt{input}(t)\vert)\}$\;
    }
    
    \BlankLine
    \uIf{$\mathtt{conds}(t)=\emptyset$}{ \label{line:gentcis-conds-empty}
      $g_t\leftarrow{}g_t\cup\{(\mathtt{cn},1)\}$\;
      $i_t\leftarrow{}i_t\cup\{(\mathtt{ic}(0),\mathtt{true})\}$\;
    } 
    \Else{
      $g_t\leftarrow{}g_t\cup\{(\mathtt{cn},\vert{}\mathtt{conds}(t)\vert)\}$\;
    }
    
    \BlankLine
    $id_t\leftarrow{}\mathtt{genid}()$\;\label{line:gen-tcomp-id}
    $d.cs\leftarrow{}d.cs~\mathtt{||}~\mathtt{comp}(\mathtt{id}_t,\mathtt{transition},g_t,i_t,o_t)$\;\label{line:gen-tcomp}
    $\gamma\leftarrow\gamma\cup(t,\mathtt{id}_t)$\;\label{line:gen-bind-t}
  }
\end{algorithm}

The \texttt{generate\_TCIs} procedure, presented in
Algorithm~\ref{alg:gentcis}, iterates over the set of transitions $T$
of the $sitpn$ parameter. For each transition $t$ in the set, the
procedure produces a corresponding TCI $id_t$, and generates its
generic map $g_t$, and its partially-built input and output port maps
$i_t$ and $o_t$.  At the end of the procedure
(Lines~\ref{line:gen-tcomp-id} to \ref{line:gen-bind-t}), a fresh and
unique component identifier $id_t$ is generated, and a new CI
statement, corresponding to the instantiation of TCI $id_t$, is
composed with the current behavior of design $d$. Finally, the
$\gamma$ binder receives a new couple corresponding to binding of
transition $t$ to identifier $id_t$.

At Line~\ref{line:t-isolated}, the procedure checks if transition $t$
is isolated, and raises an error accordingly.
Lines~\ref{line:init-gt} to \ref{line:connect-fired} initialize the
variables $g_t$, $i_t$ and $o_t$, respectively holding the generic
map, the input port map and the output port map of the generated
TCI. The generic map $g_t$ initially takes two associations: the one
between the \texttt{tt} constant and the result of the function call
\texttt{get\_ttype}$(t)$, and the one between the \texttt{mtc}
constant and the result of the function call \texttt{get\_mtc}$(t)$.
The \texttt{get\_ttype} function returns the type of transition $t$,
i.e. either \texttt{NOT\_TEMPORAL}, \texttt{TEMPORAL\_A\_A},
\texttt{TEMPORAL\_A\_B} or \texttt{TEMPORAL\_A\_INFINITE}, based on
the form of the time interval associated with
$t$. Algorithm~\ref{alg:getttype} describes the \texttt{get\_ttype}
function. The \texttt{get\_mtc} function determines the maximal value
for the time counter of $t$ based on the form of the time interval
associated with transition $t$. Algorithm~\ref{alg:getmtc} describes
the \texttt{get\_mtc} function.

\begin{algorithm}[H]
  \DontPrintSemicolon

  \SetAlFnt{\fontsize{11}{13}\selectfont}
  
  \SetAlCapFnt{\fontsize{11}{13}\selectfont}
  \SetAlCapNameFnt{\fontsize{11}{13}\selectfont}
  % \NoCaptionOfAlgo

  \caption{\texttt{get\_ttype}($t$)}
  \label{alg:getttype}
  
  \AlFnt % overriding the new font

  \lIf{$t\notin{}\mathtt{dom}(I_s)$}{
    \Return{\texttt{NOT\_TEMPORAL}}
  }
  \lElseIf{$I_s(t)=[a,a]$}{
    \Return{\texttt{TEMPORAL\_A\_A}}
  }
  \lElseIf{$I_s(t)=[a,b]$}{
    \Return{\texttt{TEMPORAL\_A\_B}}
  }
  \lElseIf{$I_s(t)=[a,\infty]$}{\Return{\texttt{TEMPORAL\_A\_INFINITE}}}
  
\end{algorithm}

\begin{algorithm}[H]
  \DontPrintSemicolon

  \SetAlFnt{\fontsize{11}{13}\selectfont}
  
  \SetAlCapFnt{\fontsize{11}{13}\selectfont}
  \SetAlCapNameFnt{\fontsize{11}{13}\selectfont}
  % \NoCaptionOfAlgo

  \caption{\texttt{get\_mtc}($t$)}
  \label{alg:getmtc}
  
  \AlFnt % overriding the new font

  \lIf{$t\notin{}\mathtt{dom}(I_s)$}{
    \Return{$1$}
  }
  \lElseIf{$I_s(t)=[a,b]$}{
    \Return{$b$}
  }
  \lElseIf{$I_s(t)=[a,\infty]$}{
    \Return{$a$}
  }
  
\end{algorithm}

Line~\ref{line:init-it} sets the value of the \texttt{A} and
\texttt{B} input ports while initializing the input port map
$i_t$. The \texttt{A} port is associated with $0$ if the transition
$t$ is not a time transition (i.e. $t$ has no associated time
interval, it is not in the domain of function $I_s$); otherwise, the
\texttt{A} port is associated with the lower bound of the time
interval of $t$.  The \texttt{B} input port is associated with $0$ if
transition $t$ is not a time transition or if its time interval has an
infinite upper bound; otherwise, the \texttt{B} port is associated
with the upper bound of the time interval of $t$. From
Lines~\ref{line:genid-fired} to \ref{line:connect-fired}, the
procedure connects the \texttt{fired} output port to a newly generated
internal signal in the output port map $o_p$. This internal signal
will then be connected to the input port map of PCIs during the
interconnection phase of the transformation (see
Section~\ref{sec:geninter}).

Line~\ref{line:gentcis-input-empty} checks if the set of input places
of $t$ is empty. If the test succeeds, the \texttt{ian} constant is
associated with $1$ in the generic map $g_t$. The size of the
\texttt{iav}, \texttt{pauths} and \texttt{rt} input ports, which are
of the array type, is equal to the value of the \texttt{ian} constant
minus one. Thus, in the case where the set of input places of $t$ is
empty, the \texttt{iav}, \texttt{pauths} and \texttt{rt} input ports
are composed of one subelement with index $0$. At
Line~\ref{line:gentcis-in-empty-it}, the sole subelements of the
\texttt{iav} and the \texttt{pauths} ports are associated with
\texttt{true}, and the sole subelement of the \texttt{rt} port is
associated with the signal identifier $id_s$. Remember that the
\texttt{fired} output port has been previously connected to the
internal signal $id_s$ in the output port map $o_t$. Thus, the
\texttt{fired} output port is connected to the subelement of the
\texttt{rt} input port with index $0$ through the $id_s$ signal.  This
connection is mandatory to reset the value of the
\texttt{s\_time\_counter} signal (which is an internal signal of the
\texttt{transition} design) in the abscence of input places. If the
set of input places of $t$ is not empty, then the \texttt{ian}
constant is associated with the size of the set in the generic map
$g_t$.

Line~\ref{line:gentcis-conds-empty} checks if the set of conditions
attached to $t$ is empty. If the test succeeds, the \texttt{cn}
constant is associated with $1$ in the generic map $g_t$. The size of
the \texttt{ic} input port, which is of the array type, is equal to
the value of the \texttt{cn} constant minus one. Thus, in the case
where the set of conditions attached to $t$ is empty, the \texttt{ic}
input port is composed of one subelement with index $0$. Then, the
sole subelement of the \texttt{ic} port is associated with
\texttt{true} in the input port map $i_t$. If the set of conditions
attached to $t$ is not empty, the \texttt{cn} constant is associated
with the size of the set in the generic map $g_t$. In that case, the
\texttt{generate\_conds} procedure, presented in
Algorithm~\ref{alg:genconds}, will handle the connection of the
subelements of the \texttt{ic} input port.

\subsection{Interconnection of the place and transition component
  instances}
\label{sec:geninter}

After the generation of PCIs and TCIs, and of all constant
associations in their generic and input port maps, the next step of
the transformation performs the interconnections between the
interfaces of PCIs and TCIs. The \texttt{generate\_interconnections}
procedure, presented in Algorithm~\ref{alg:geninter}, produces these
interconnections.

\begin{algorithm}[H]
  \DontPrintSemicolon

  \SetAlFnt{\fontsize{11}{13}\selectfont}
  
  \SetAlCapFnt{\fontsize{11}{13}\selectfont}
  \SetAlCapNameFnt{\fontsize{11}{13}\selectfont}
  % \NoCaptionOfAlgo

  \caption{\texttt{generate\_interconnections}($sitpn$, $d$, $\gamma$)}
  \label{alg:geninter}
  
  \AlFnt % overriding the new font

  \ForEach{$p\in{}P$}{
    \texttt{comp}$(id_p,\mathtt{place},g_p,i_p,o_p)\leftarrow\mathtt{get\_comp}(\gamma(p),d.cs)$\;\label{line:geninter-getpcomp}

    \BlankLine
    $i\leftarrow{}0$\; \label{geninter-start}
    \ForEach{$t\in\mathtt{input}(p)$}{\label{line:foreach-input-cit}
      \texttt{comp}$(id_t,\mathtt{transition},g_t,i_t,o_t)\leftarrow\mathtt{get\_comp}(\gamma(t),d.cs)$\;
      $i_p\leftarrow\{(\mathtt{itf}(i),\mathtt{actual}(\mathtt{fired},o_t))\}$\;\label{line:connect-itf-fired}
      $i\leftarrow{}i+1$\;
    }\label{line:geninter-end-input}

    \BlankLine
    $i\leftarrow{}0$\;
    \ForEach{$t\in\mathtt{output}_c(p)$}{\label{line:foreach-outc}
      \texttt{comp}$(id_t,\mathtt{transition},g_t,i_t,o_t)\leftarrow\mathtt{get\_comp}(\gamma(t),d.cs)$\;
      $i_p\leftarrow{}i_p\cup\{(\mathtt{otf}(i),\mathtt{actual}(\mathtt{fired},o_t))\}$\; \label{line:otf-fired-1}
      \texttt{connect}$(o_p,i_t,\mathtt{oav}(i),\mathtt{iav},d)$\; \label{line:oav-iav-1}
      \texttt{connect}$(o_p,i_t,\mathtt{rtt}(i),\mathtt{rt},d)$\; \label{line:rtt-rt-1}
      \texttt{connect}$(o_p,i_t,\mathtt{pauths}(i),\mathtt{pauths},d)$\; \label{line:pauths-pauths-1}
      \texttt{put\_comp}$(id_t, \mathtt{comp}(id_t,\mathtt{transition},g_t,i_t,o_t), d.cs)$\; \label{geninter-puttcomp}
      $i\leftarrow{}i+1$\;
    }\label{line:end-foreach-outc}
    
    \ForEach{$t\in\mathtt{output}_{nc}(p)$}{
      \texttt{comp}$(id_t,\mathtt{transition},g_t,i_t,o_t)\leftarrow\mathtt{get\_comp}(\gamma(t),d.cs)$\;\label{line:foreach-outnc}
      $i_p\leftarrow{}i_p\cup\{(\mathtt{otf}(i),\mathtt{actual}(\mathtt{fired},o_t))\}$\; \label{line:otf-fired-2}
      \texttt{connect}$(o_p,i_t,\mathtt{oav}(i),\mathtt{iav},d)$\; \label{line:oav-iav-2}
      \texttt{connect}$(o_p,i_t,\mathtt{rtt}(i),\mathtt{rt},d)$\; \label{line:rtt-rt-2}

      $id_s\leftarrow\mathtt{genid}()$\;\label{line:genid-for-pauths}
      $d.sigs\leftarrow{}d.sigs\cup(id_s,\mathtt{boolean})$\;
      $o_p\leftarrow{}o_p\cup\{(\mathtt{pauths}(i),id_s)\}$\;\label{line:pauths-assoc}
      \texttt{cassoc}$(i_t,\mathtt{pauths},\mathtt{true})$\; \label{line:pauths-true}
      
      \texttt{put\_comp}$(id_t, \mathtt{comp}(id_t,\mathtt{transition},g_t,i_t,o_t), d.cs)$\;\label{line:end-foreach-outnc}
      $i\leftarrow{}i+1$\;
    } \label{geninter-end}
    
    \BlankLine
    \texttt{put\_comp}$(id_p, \mathtt{comp}(id_p,\mathtt{place},g_p,i_p,o_p), d.cs)$\;\label{line:putpcomp-cit}
  }
\end{algorithm}

The \texttt{generate\_interconnections} procedure iterates over the
set of places of the $sitpn$ parameter. For each place $p$, the
procedure generates the interconnections between the PCI $id_p$ and
the TCIs that implement the input and output transitions of $p$; we
will refer to them as the input and output TCIs of PCI $id_p$.

At Line~\ref{line:geninter-getpcomp}, the \texttt{get\_comp} function
returns the PCI associated with the identifier $\gamma(p)$ (i.e. the
PCI identifier associated with place $p$ in $\gamma$) by looking up
the behavior of the design $d$. At this step, we assume that all PCIs
and TCIs, and all bindings pertaining to places and transitions in the
$\gamma$ binder, have been previously generated by the
\texttt{generate\_architecture} procedure. Otherwise, the
\texttt{get\_comp} function raises an error if it is not able to find
the PCI $id_p$ in the behavior of design $d$.

Then, from Line~\ref{geninter-start} to Line~\ref{geninter-end}, the
procedure modifies the input and output port map of PCI $id_p$ and the
input port map of its input and output TCIs. Finally,
Line~\ref{line:putpcomp-cit} replaces the old PCI $id_p$ by the
modified one in the behavior of design $d$.

From Line~\ref{geninter-start} to Line~\ref{line:geninter-end-input},
the procedure iterates over the input transitions of place $p$. Note
that the iteration is performed in the same order than the iteration
performed by the \textbf{foreach} loop at Line~\ref{line:foreach-iaw}
of the \texttt{generate_PCIs} procedure; this is mandatory to preserve
a consistency between the index $i$ and the connection to a given
transition (see Remark~\ref{rem:connections-consistency}).  For each
input transition $t$ of $p$, the corresponding TCI $id_t$ is retrieved
from the behavior of design $d$. Then, the internal signal associated
with the \texttt{fired} output port in the output port map of TCI
$id_t$ is retrieved (i.e. \texttt{actual}$(\mathtt{fired},o_t)$), and
the signal is associated with the subelement of the \texttt{itf} input
port with index $i$. We know that the \texttt{generate_TCIs} function
has generated the association between the \texttt{fired} output port
and an internal signal in the output port map of all TCIs. Thus, the
\texttt{actual} function never raises an error.

\begin{remark}[Connections consistency]
  \label{rem:connections-consistency}
  In the behavior of the \texttt{place} design, some processes access
  to the subelements of composite ports through the use of
  indices. For instance, the \texttt{input\_tokens\_sum} process (see
  Appendix~\ref{app:place-design}) increments a local variable $i$ in
  range $0$ to $\mathtt{input\_arcs\_number}-1$ in a for loop. The
  process tests the value of the \texttt{itf} port's subelement with
  index $i$. If the test suceeds, the process adds the value of the
  \texttt{iaw} port's subelement with index $i$ to the local variable
  \texttt{v\_internal\_input\_token\_sum}. Thus, the subelement with
  index $i$ of the \texttt{itf} and \texttt{iaw} ports must refer to
  the connection to the same transition. Otherwise, the process does
  not compute a correct input tokens
  sum. Figure~\ref{fig:consistent-connections} illustrates the correct
  connection of the \texttt{itf} and \texttt{iaw} ports in the input
  port map of PCI $id_p$ w.r.t. to the connection between transitions
  $t_a$, $t_b$, $t_c$ and place $p$.

  \begin{figure}[H]
    \centering
    \includegraphics[keepaspectratio,width=.8\textwidth]{Figures/Transformation/consistent-indexes}
    \caption[An example of correct connections between several TCIs
    and a PCI.]{An example of correct connections between the PCI
      $id_p$ and TCIs $id_{t_a}$, $id_{t_b}$ and $id_{t_c}$. On the
      left, the input SITPN model showing the connections of the
      transitions $t_a$, $t_b$ and $t_c$ to the place $p$. The dots
      indicate that the place $p$ possibly has other input
      transitions. On the right, the TCIs and the PCI generated by the
      transformation. In the input port map of PCI $id_p$, the
      subelements of the \texttt{itf} input port are connected to the
      \texttt{fired} port of TCIs; the subelements of the \texttt{iaw}
      port are connected to constant values, i.e. the weight of the
      arcs between place $p$ and the input transitions of $p$.}
    \label{fig:consistent-connections}
  \end{figure}

  It is the rôle of the transformation function to ensure the
  consistency of the connections of the subelements in the input and
  output port maps of PCIs. The input and output port maps of TCIs are
  not subject to such a constraint. The fact that a \textbf{foreach}
  loop always iterates in the same order over the elements of a set
  ensures the consistency of the connections.
\end{remark}

From Line~\ref{line:foreach-outc} to Line~\ref{line:end-foreach-outc},
the procedure connects the PCI $id_p$ to the TCIs implementing the
conflicting output transitions of place $p$. For each conflicting
output transition $t$ of $p$, the corresponding TCI $id_t$ is
retrieved from the behavior of design $d$. The function call
\texttt{actual}$(\mathtt{fired},o_t)$ returns the internal signal
associated with the \texttt{fired} output port in the output port map
of TCI $id_t$. This internal signal is then connected to the
subelement of the \texttt{otf} input port with index $i$ in the input
port map of PCI $id_p$. At Line~\ref{line:oav-iav-1}, the
\texttt{connect} function generates an internal signal $id$ and adds
it to the internal signal declaration list of design $d$. Then, the
function associates the subelement \texttt{oav}$(i)$ (i.e. the
subelement of the \texttt{oav} input port with index $i$) with the
internal signal $id$ in the output port map $o_p$, and it associates
one subelement of the \texttt{iav} input port to the internal signal
$id$ in the input port map $i_t$. The \texttt{connect} function
operates similarly on the \texttt{rtt} output port and the \texttt{rt}
input port at Line~\ref{line:rtt-rt-1}, and on the \texttt{pauths}
input port and the \texttt{pauths} output port at
Line~\ref{line:pauths-pauths-1}. Finally, at
Line~\ref{geninter-puttcomp}, the old TCI $id_t$ is replaced by the
modified one in the behavior of design $d$.

From Line~\ref{line:foreach-outnc} to
Line~\ref{line:end-foreach-outnc}, the procedure connects the TCIs
implementing to the output transitions of $p$ that are not in
conflict. Note that the variable $i$ is not reset between the two
\textbf{foreach} loops to preserve the continuity of indices.  For
each non-conflicting output transition $t$ of $p$, the corresponding
TCI $id_t$ is retrieved from the behavior of design $d$. Then, the
interconnections between PCI $id_p$ and TCI $id_t$ are similarly to
the ones that have been performed for the conflicting transitions of
$p$. The difference lies in the connection the \texttt{pauths} ports.
Between the PCI $id_p$ and its \textit{non-conflicting} TCIs, the
\texttt{pauths} are not connected together; this to reflect the
independence of non-conflicting output transitions regarding the
priority authorizations. Instead, the subelement of the
\texttt{pauths} output port with index $i$ is connected to a newly
generated internal signal $id_s$ in the output port map $o_p$
(Line~\ref{line:genid-for-pauths} to
Line~\ref{line:pauths-assoc}). Also, one subelement of the
\texttt{pauths} input port is associated with \texttt{true} in the
input port map $i_t$ (Line~\ref{line:pauths-true}).

\subsection{Generation of ports, the \texttt{action} and the
  \texttt{function} process}
\label{sec:genports}

\begin{algorithm}[H]
  \DontPrintSemicolon

  \SetAlFnt{\fontsize{11}{13}\selectfont}
  
  \SetAlCapFnt{\fontsize{11}{13}\selectfont}
  \SetAlCapNameFnt{\fontsize{11}{13}\selectfont}
  % \NoCaptionOfAlgo

  \caption{\texttt{generate\_conds}($sitpn$, $d$, $\gamma$)}
  \label{alg:genconds}
  
  \AlFnt % overriding the new font
\end{algorithm}

%%% Local Variables:
%%% mode: latex
%%% TeX-master: "../../main"
%%% End:


\section{\coq{} implementation of the \hilecop{} model-to-text transformation}
\label{sec:trans-coq-impl}
This section presents the implementation of the \hilecop{}
model-to-text transformation with the \coq{} proof assistant.  The
full implementation is available under the \texttt{sitpn2hvhdl} folder
of the following Git repository:
\url{https://github.com/viampietro/ver-hilecop}

Listing~\ref{lst:sitpn2hvhdl} gives the \coq{} implementation of the
\texttt{sitpn\_to\_hvhdl} function presented in an imperative
pseudo-code version in Algorithm~\ref{alg:sitpn2hvhdl}.

\begin{lstlisting}[language=coq,label={lst:sitpn2hvhdl},
caption={[The \coq{} implementation of the \texttt{sitpn\_to\_hvhdl} function.] The \coq{} implementation of the \texttt{sitpn\_to\_hvhdl} function presented in Algorithm~\ref{alg:sitpn2hvhdl}.},framexleftmargin=1.5em,xleftmargin=2em,numbers=left,
numberstyle=\tiny\ttfamily]
Definition sitpn_to_hvhdl (sitpn : Sitpn)
   (decpr : forall x y : T sitpn, {pr x y} + {~pr x y})
   ($id_e$ $id_a$ : ident) (b : P sitpn -> nat) :
   (design * Sitpn2HVhdlMap sitpn) + string :=
  RedV 
    ((do _ <- generate_sitpn_infos sitpn decpr;
      do _ <- generate_architecture sitpn b;
      do _ <- generate_ports sitpn;
      do _ <- generate_comp_insts sitpn;
      generate_design_and_binder $id_e$ $id_a$)
      (InitS2HState sitpn Petri.ffid)).
\end{lstlisting}

In Listing~\ref{lst:sitpn2hvhdl}, the \texttt{sitpn\_to\_hvhdl}
function has five parameters: \texttt{sitpn}, the input SITPN model;
\texttt{decpr}, a proof that the \texttt{pr} relation (i.e. the
implementation of the firing priority relation) is decidable over the
set of transitions of \texttt{sitpn} (i.e. \texttt{T sitpn}); $id_e$
and $id_a$, the entity and architecture identifiers for the generated
\hvhdl{} design; the \texttt{b} function that maps the places of the
\texttt{sitpn} parameter to a maximal marking value, i.e. a natural
number. The \texttt{sitpn\_to\_hvhdl} function returns a couple
composed of the generated \hvhdl{} design, of type \texttt{design},
and the generated $\gamma$ binder, of type \texttt{Sitpn2HVhdlMap
  sitpn}; or, the \texttt{sitpn\_to\_hvhdl} function returns a
\texttt{string} corresponding to an error message.

In the body of the \texttt{sitpn\_to\_hvhdl} function, the
\texttt{RedV} is a notation that reduces a monadic function call to a
value. Our implementation of the \hilecop{} transformation function
relies on the state-and-error monad \cite{Wadler1992}. Each function
that implements a part of the transformation function takes a
\emph{compile-time} state as a parameter, and returns either a value
and a new compile-time state, or an error message. The \texttt{bind}
construct of the state-and-error monad permits to pipeline multiple
function calls, and, combined with the \texttt{do} notation, it
permits us to write functional programs in the style of imperative
languages. The sequence defined in the body of the
\texttt{sitpn\_to\_hvhdl} function gives an example of what can be
achieved with the combination of the state-and-error monad and the
\texttt{do} notation. This sequence constitutes a single monadic
function that takes a state of the \texttt{Sitpn2HVhdlState} type (see
Listing~\ref{lst:comp-time-state}) as input, and yields a value with a
new state, or an error message. Here, the \texttt{RedV} notation
retrieves only the value returned by the application of the monadic
function to the parameter \texttt{(InitS2HState sitpn Petri.ffid)}
(i.e. the initial compile-time state), or it retrieves the error
message.

In the \texttt{do} sequence of Listing~\ref{lst:sitpn2hvhdl}, the four
first function calls do not return values that are relevant; thus, we
use the underscore notation to notify that we are not interested in
the returned values.  Indeed, the \texttt{generate\_sitpn\_infos},
\texttt{generate\_architecture}, \texttt{generate\_ports} and
\texttt{generate\_comp\_insts} functions directly modify the
compile-time state without returning a value. They are the functional
implementation of the procedures described in the previous section.
 
Now, let us present the content of the compile-time state. As said
above, the compile-time state is carried from function to function and
modified all along the transformation.
Listing~\ref{lst:comp-time-state} gives the implementation of the
compile-time state structure.

\begin{lstlisting}[language=coq,label={lst:comp-time-state},
caption={[The compile-time state structure.] The compile-time state structure defined as the \coq{} \texttt{Sitpn2HVhdlState} record type.},framexleftmargin=1.5em,xleftmargin=2em,numbers=left,
numberstyle=\tiny\ttfamily]
Record Sitpn2HVhdlState (sitpn : Sitpn) : Type :=
  MkS2HState {
     lofPs : list (P sitpn); #\label{line:lofPs}#
     lofTs : list (T sitpn);
     lofCs : list (C sitpn);
     lofAs : list (A sitpn);
     lofFs : list (F sitpn); #\label{line:lofFs}#
     nextid : ident; #\label{line:nextid}#
     sitpninfos : SitpnInfos sitpn; #\label{line:sitpninfos}#
     iports : list pdecl; #\label{line:iports}#
     oports : list pdecl; #\label{line:oports}#
     arch : Architecture sitpn; #\label{line:arch}#
     beh : cs; #\label{line:beh}#
     $\gamma$ : Sitpn2HVhdlMap sitpn; #\label{line:gamma}#

  }.
\end{lstlisting}

The compile-time state structure is implemented by the
\texttt{Sitpn2HVhdlState} record type. This type depends on a given
\texttt{sitpn} passed as a parameter. It is composed of eleven
fields. The first five fields (Line~\ref{line:lofPs} to
\ref{line:lofFs}) are the list versions of the finite sets of places,
transitions, conditions, actions and functions of the \texttt{sitpn}
parameter. These fields are filled at the very beginning of the
transformation by the \texttt{generate\_sitpn\_infos} function, and
are convenient to write functions in the context of dependent
types. The \texttt{nextid} field (Line~\ref{line:nextid}) permits us to
generate fresh and unique identifiers all along the
transformation. The \texttt{sitpinfos} field
(Line~\ref{line:sitpninfos}) is an instance of the \texttt{SitpnInfos}
type that depends on the \texttt{sitpn} parameter. The
\texttt{sitpninfos} field is filled up by the
\texttt{generate\_sitpn\_infos} function. It is a convenient way to
represent the information associated with the places, transitions,
conditions, actions and functions of the \texttt{sitpn} parameter. The
\texttt{iports} (resp. \texttt{oports}) field, at
Line~\ref{line:iports} (resp. at Line~\ref{line:oports}), gathers the
list of input (resp. output) port declarations of the generated
\hvhdl{} design. The \texttt{arch} field (Line~\ref{line:arch}) is an
intermediary representation of the behavior of the generated \hvhdl{}
design. This representation is easier to modify and to handle than a
\hvhdl{} concurrent statement. The \texttt{beh} field
(Line~\ref{line:beh}) is the behavior of the generated \hvhdl{}
design; it is an instance of the \texttt{cs} type, i.e. the type of
concurrent statements defined in the abstract syntax of \hvhdl{}. The
$\gamma$ field (Line~\ref{line:gamma}) is the SITPN-to-\hvhdl{} binder
generated alongside the \hvhdl{} design, and returned at the end of
the transformation.

At the beginning of the transformation, an initial compile-time state
is built with the \texttt{Init\-S2HState} function. The
\texttt{InitS2HState} function gives an initial value to the fields of
the state structure; mostly, the fields are initialized with empty
lists, and the \texttt{beh} field is initialized with the
\texttt{null} statement. The \texttt{InitS2HState} function takes an
\texttt{Sitpn} instance and an identifier as inputs. The identifier
parameter represents the initial value of the \texttt{nextid}
field. In Listing~\ref{lst:sitpn2hvhdl}, the second parameter of the
\texttt{InitS2HState} function is \texttt{Petri.ffid}. It corresponds
to the \emph{first fresh} identifier that the transformation can use
to produce a \hvhdl{} design that respects the uniqueness of
identifiers.

Let us now present the functions composing the \texttt{do} sequence of
the \texttt{sitpn\_to\_hvhdl} function, and how they modify the
compile-time state to produce the final \hvhdl{} design and the
$\gamma$ binder.

\subsection{The \texttt{generate\_sitpn\_infos} function}
\label{sec:gen-sitpn-infos}

Listing~\ref{lst:gen-infos} presents a part of the
\texttt{generate\_sitpn\_infos}. The part that is let aside,
represented by little dots, pertains to the creation of the
dependently-typed lists constituting the first fields of the
compile-time state structure (Line~\ref{line:lofPs} to
\ref{line:lofFs} in Listing~\ref{lst:comp-time-state}).

\begin{lstlisting}[language=coq,label={lst:gen-infos},
caption={[The \texttt{generate\_sitpn\_infos} function.] A part of the \texttt{generate\_sitpn\_infos} function.},framexleftmargin=1.5em,xleftmargin=2em,numbers=left,
numberstyle=\tiny\ttfamily]
Definition generate_sitpn_infos
           (sitpn : Sitpn)
           (decpr : forall x y : T sitpn, {pr x y} + {~pr x y}) :
    Mon (Sitpn2HVhdlState sitpn) unit :=
  #\dots#
  do _ <- check_wd_sitpn sitpn decpr;
  do _ <- generate_trans_infos sitpn;
  do _ <- generate_place_infos sitpn decpr;
  do _ <- generate_cond_infos sitpn; 
  do _ <- generate_action_infos sitpn;
  generate_fun_infos sitpn.
\end{lstlisting}

The \texttt{generate\_sitpn\_infos} function takes an \texttt{Sitpn}
instance and a proof of decidability for the \texttt{pr} relation as
parameters. It returns a value of type \texttt{Mon (Sitpn2HVhdlState
  sitpn) unit}. A value of this type can either be a couple
$(state,value)$, where $state$ is of type \texttt{(Sitpn2HVhdlState
  sitpn)} and $value$ is of type \texttt{unit}, or an error
message. The \texttt{unit} type as only one possible value
\texttt{tt}. The \texttt{unit} type is used here to represent a
function that modifies the compile-time state without returning a
value.

The aim of the \texttt{generate\_sitpn\_infos} function is to fill the
\texttt{sitpninfos} field of the compile-time state; the
\texttt{sitpninfos} field is an instance of the \texttt{SitpnInfos}
record type. Listing~\ref{lst:infos-types} presents the definition of
the \texttt{SitpnInfos} record type, along with the definition of the
\texttt{PlaceInfo} and \texttt{TransInfo} record types.

\begin{lstlisting}[language=coq,label={lst:infos-types},
caption={[The \texttt{PlaceInfo}, \texttt{TransInfo} and \texttt{SitpnInfos} types.]The \texttt{PlaceInfo}, \texttt{TransInfo} and \texttt{SitpnInfos} record types.},framexleftmargin=1.5em,xleftmargin=2em,numbers=left,
numberstyle=\tiny\ttfamily]
Record PlaceInfo (sitpn : Sitpn) : Type :=
  MkPlaceInfo { tinputs : list (T sitpn);
                tconflict : list (T sitpn);
                toutputs : list (T sitpn) }.

Record TransInfo (sitpn : Sitpn) : Type :=
  MkTransInfo { pinputs : list (P sitpn); conds : list (C sitpn) }.
  
Record SitpnInfos (sitpn : Sitpn) : Type :=
  MkSitpnInfos {
      pinfos : list (P sitpn * PlaceInfo);
      tinfos : list (T sitpn * TransInfo);
      cinfos : list (C sitpn * list (T sitpn));
      ainfos : list (A sitpn * list (P sitpn));
      finfos : list (F sitpn * list (T sitpn));
    }.
\end{lstlisting}

The \texttt{PlaceInfo} record type is composed of three lists that
represent the input transitions, \texttt{tinputs}, the conflicting
output transitions, \texttt{tconflict}, and the non-conflicting output
transitions, \texttt{toutputs}, of a place. In the \texttt{SitpnInfos}
structure, the \texttt{pinfos} field maps the places of the
\texttt{sitpn} parameter to their respective information, i.e. an
instance of the \texttt{PlaceInfo} type. This mapping is built by the
\texttt{generate\_place\_infos} function called in the body of
\texttt{generate\_sitpn\_infos} function. While building an instance
of the \texttt{PlaceInfo} type for a given place $p$, the
\texttt{generate_place_infos} function computes the list of output
transitions of $p$ that are in conflict (in the manner of the
\texttt{output}$_c$ function described in
Section~\ref{sec:prim-funs}). First, it computes the list of output
transitions that are linked to the place $p$ through a \texttt{basic}
arc; then, the function checks if all conflicts between the
transitions of this list are solved by means of mutual exclusion. If
it is the case, the \texttt{tconflict} field is left empty, and all
transitions of the list join the \texttt{toutputs} list. Otherwise,
the function tries to establish a strict total order over the
transitions of the list, by decreasing level of priority.  If no such
order can be established, the function raises an error; otherwise, the
\texttt{tconflict} field is filled with the ordered list. This process
never fails if the input \texttt{sitpn} parameter is indeed
well-defined (cf. Definition~\ref{def:wd-sitpn}).

The \texttt{TransInfo} record type is composed of two lists that
represent the input places, \texttt{pinputs}, and the output places,
\texttt{poutputs}, of a transition. In the \texttt{SitpnInfos}
structure, the \texttt{tinfos} field maps the transitions of the
\texttt{sitpn} parameter to their respective information, i.e. an
instance of the \texttt{TransInfo} type. This mapping is built by the
\texttt{generate\_trans\_infos} function called in the body of
\texttt{generate\_sitpn\_infos} function.

In the \texttt{SitpnInfos} structure, the \texttt{cinfos}
(resp. \texttt{ainfos} and \texttt{finfos}) field maps the conditions
(resp. actions and functions) of the \texttt{sitpn} parameter to the
list of transitions (resp. places and transitions) they are attached
to. This mapping is built by the \texttt{generate\_cond\_infos}
(resp. \texttt{generate\_action\_infos} and
\texttt{generate\_fun\_infos}) function called in the body of
\texttt{generate\_sitpn\_infos} function.

At the beginning of the \texttt{generate\_sitpn\_infos} function, the
\texttt{check\_wd\_sitpn} function partly checks the well-definition
of the \texttt{sitpn} parameter. Precisely, it checks that the set of
places and transitions of the \texttt{sitpn} parameter are not empty,
and that the priority relation is a strict order, i.e. transitive and
reflexive, over the set of transitions. The other parts of the
well-definition checking are performed later during the
transformation. For instance, the \texttt{generate\_place\_infos}
function checks that, for each group of transitions in conflict, the
conflicts are either solved by means of mutual exclusion or that the
priority relation is a strict total order over this group. It also
checks that there are no isolated places in the input \texttt{sitpn}
parameter, etc.

\subsection{The \texttt{generate\_architecture} function}
\label{sec:impl-gen-arch}

Listing~\ref{lst:gen-arch} presents the
\texttt{generate\_architecture} function. The
\texttt{generate\_architecture} function implements the
\texttt{generate\_architecture} and the
\texttt{generate\_interconnections} procedures detailed in
Algorithms~\ref{alg:genarch} and \ref{alg:geninter}.  The composition
of the \texttt{generate\_place\_map} and the
\texttt{generate\_trans\_map} functions implements
\texttt{generate\_architecture} procedure of
Algorithm~\ref{alg:genarch}. Precisely, the
\texttt{generate\_place\_map} function implements the
\texttt{generate\_PCIs} procedure presented in
Algorithm~\ref{alg:genpcis}, and the \texttt{generate\_trans\_map}
function implements the \texttt{generate\_TCIs} procedure presented in
Algorithm~\ref{alg:gentcis}.

\begin{lstlisting}[language=coq,label={lst:gen-arch},
caption={[The \texttt{generate\_architecture} function.]The \texttt{generate\_architecture} function that implements the \texttt{generate\_architecture} procedure of Algorithm~\ref{alg:genarch}.},framexleftmargin=1.5em,xleftmargin=2em,numbers=left,
numberstyle=\tiny\ttfamily]
Definition generate_architecture (sitpn : Sitpn) (b : P sitpn -> nat) :
  Mon (Sitpn2HVhdlState sitpn) unit :=
  do _ <- generate_place_map sitpn b;
  do _ <- generate_trans_map sitpn;
  generate_interconnections.
\end{lstlisting}

The \texttt{generate\_architecture} function takes an \texttt{Sitpn}
instance and the \texttt{b} function as inputs, and modifies the
compile-time state. The \texttt{generate\_architecture} function fills
the \texttt{arch} field of the compile-time state; the \texttt{arch}
field is an instance of the \texttt{Architecture} record
type. Listing~\ref{lst:infos-types} presents the definition of the
\texttt{Architecture} record type, along with the definition of the
\texttt{InputMap}, \texttt{OutputMap} and \texttt{HComponent} type
aliases.

\begin{lstlisting}[language=coq,label={lst:arch-types}, caption={[The
\texttt{Architecture} record type.]The \texttt{Architecture} record
type, and the \texttt{InputMap}, \texttt{OutputMap} and
\texttt{HComponent} subsidiary
types. },framexleftmargin=1.5em,xleftmargin=2em,numbers=left,
numberstyle=\tiny\ttfamily]
Definition InputMap := list (ident * (expr + list expr)).
Definition OutputMap := list (ident * ((option name) + list name)).
Definition HComponent := (genmap * InputMap * OutputMap).

Record Architecture (sitpn : Sitpn) := MkArch {
  sigs  : list sdecl;
  plmap : list (P sitpn * HComponent);
  trmap : list (T sitpn * HComponent);
  fmap  : list (F sitpn * list expr);
  amap  : list (A sitpn * list expr) }.
\end{lstlisting}

The \texttt{HComponent} type is an intermediate representation of an
\hvhdl{} component instantiation statement. This type has been devised
to ease the construction of PCIs and TCIs, and of their generic, input
port and output port maps all along the transformation.  The
\texttt{HComponent} type is a triplet composed of a generic map as
defined in the \hvhdl{} abstract syntax, an instance of the
\texttt{InputMap} type, and an instance of the \texttt{OutputMap}
type.  The \texttt{InputMap} type maps an input port identifier to
either a simple expression or to a list of expressions, where the
\texttt{expr} type is the type of expressions defined in the \hvhdl{}
abstract syntax. In an \texttt{InputMap} instance, an input port
identifier of a scalar type (i.e. Boolean or constrained natural) is
mapped to a simple expression, whereas an input port identifier of the
array type is mapped to a list of expressions. Each expression of the
list represents the actual part associated with one subelement of the
input port. Similarly to the \texttt{InputMap} type, the
\texttt{OutputMap} type maps an output port identifier to either an
option to a signal (the \texttt{None} value representing the
connection to the \texttt{open} keyword) name, or to a list of signal
names. In the definition of the \texttt{OutputMap} type, the
\texttt{name} type represents the type of simple identifiers or
indexed identifiers defined in the \hvhdl{} abstract syntax.

The \texttt{Architecture} record type is an intermediary
representation of the behavioral and declarative part of an \hvhdl{}
design's architecture.  The \texttt{sigs} field of the
\texttt{Architecture} type represents the internal signal declaration
list constituting the declarative part of an \hvhdl{} design's
architecture. The transformation adds a new signal declaration entry
to the \texttt{sigs} field every time an internal signal must be
generated, for example, during the generation of interconnections
between PCIs and TCIs.  The \texttt{plmap} (resp. the \texttt{trmap})
field maps the places (resp. transitions) of the \texttt{sitpn}
parameter to their corresponding PCI (resp. TCI) implemented in an
intermediate format, i.e. an instance of the \texttt{HComponent}
type. The \texttt{fmap} field of the \texttt{Architecture} type maps
the functions of the \texttt{sitpn} parameter to a list of
expressions. For a given function $f$, the associated list of
expressions corresponds to the list of internal signals associated
with the \texttt{fired} port of the TCIs implementing the transitions
of the \texttt{trs}$(f)$ set (i.e. the set of transitions associated
with function $f$). The \texttt{fmap} field is filled by the
\texttt{generate\_ports} function described in
Listing~\ref{lst:impl-gen-ports}. The \texttt{amap} field is the twin
of the \texttt{fmap} field but on the side of the actions of the
\texttt{sitpn} parameter. Thus, in the \texttt{amap} field, the list
of expressions associated with an action $a$ corresponds to the list
of internal signals connected to the \texttt{marked} port of the PCIs
implementing the places of $a$.

In the body of the \texttt{generate\_architecture} function, the
\texttt{generate\_place\_map} function implements the
\texttt{generate\_PCIs} procedure described in
Algorithm~\ref{alg:genpcis}. For each place of the \texttt{sitpn}
parameter, the \texttt{generate\_place\_map} function builds an
instance of the \texttt{HComponent} type, and adds an association
between place and \texttt{HComponent} instance in the \texttt{plmap}
field. The \texttt{generate\_place\_map} function fills the generic,
input port and output port map of the \texttt{HComponent} instances as
described in the \texttt{generate\_PCIs} procedure. Following the
\texttt{generate\_place\_map} function, the
\texttt{generate\_trans\_map} function implements the
\texttt{generate\_TCIs} procedure described in
Algorithm~\ref{alg:gentcis}. For each transition of the \texttt{sitpn}
parameter, the \texttt{generate\_trans\_map} function builds an
instance of the \texttt{HComponent} type, and adds an association
between transition and \texttt{HComponent} instance in the
\texttt{trmap} field. The \texttt{generate\_trans\_map} function fills
the generic, input port and output port map of the \texttt{HComponent}
instances as described in the \texttt{generate\_TCIs} procedure.
Finally, the \texttt{generate\_interconnections} function modifies the
input and output port maps of the \texttt{HComponent} instances in the
\texttt{plmap} and \texttt{trmap} fields, and thus, implements the
interconnections described in the \texttt{generate\_interconnections}
procedure of Algorithm~\ref{alg:geninter}.

\subsection{The \texttt{generate\_ports} function}
\label{sec:impl-gen-ports}

Listing~\ref{lst:impl-gen-ports} presents the \texttt{generate\_ports}
function called in the body of the \texttt{sitpn\_to\_hvhdl} function
(see Listing~\ref{lst:sitpn2hvhdl}). The \texttt{generate\_ports}
function implements the \texttt{generate\_ports} procedure described
in Algorithm~\ref{alg:genports}. The \texttt{generate\_ports} function
calls three functions: the \texttt{generate\_action\_ports\_and\_ps}
function that implements the \texttt{generate_action_ports} procedure
of Algorithm~\ref{alg:genacts}, the
\texttt{generate\_fun\_ports\_and\_ps} function that implements the
\texttt{generate_function_ports} procedure of
Algorithm~\ref{alg:genfuns}, and the
\texttt{generate_and_connect_cond_ports} that implements the
\texttt{generate\_condition_ports} procedure of
Algorithm~\ref{alg:genconds}.

\begin{lstlisting}[language=coq,label={lst:impl-gen-ports},
caption={[The \texttt{generate\_ports} function.]The
  \texttt{generate\_ports} function implementing the
  \texttt{generate\_ports} procedure presented in
  Algorithm~\ref{alg:genports}. },framexleftmargin=1.5em,xleftmargin=2em,numbers=left,
numberstyle=\tiny\ttfamily]
Definition generate_ports (sitpn : Sitpn) : Mon (Sitpn2HVhdlState sitpn) unit :=
  do _ <- generate_action_ports_and_ps;
  do _ <- generate_fun_ports_and_ps;
  generate_and_connect_cond_ports.
\end{lstlisting}

For every action of the \texttt{sitpn} parameter, the
\texttt{generate\_action_ports_and_ps} function adds a port
declaration entry to the \texttt{oports} field of the compile-time
state, and adds a binding between action and output port identifier in
the $\gamma$ field. It also builds the \texttt{action} process as
described in the \texttt{generate_action_ports} procedure, and adds
the process to the \texttt{beh} field of the compile-time state.  The
\texttt{generate_fun_ports_and_ps} does the same for the functions of
the \texttt{sitpn} parameter, and similarly builds the
\texttt{function} process and adds it to the \texttt{beh} field. The
\texttt{generate_and_connect_cond_ports} function add a port
declaration entry for every condition of the \texttt{sitpn} parameter
to the \texttt{iports} field of the compile-time state. Then, it
modifies the input port map of \texttt{HComponent} instances in the
\texttt{trmap} of the compile-time state's \texttt{arch} field. The
modifications pertain to the connection of input ports to the
\texttt{ic} input port of TCIs, as described in the
\texttt{generate\_condition_ports} procedure (see
Algorithm~\ref{alg:genconds}).

\subsection{The \texttt{generate\_comp\_insts} and
  \texttt{generate\_design\_and\_binder} functions}
\label{sec:impl-gen-comp-insts}

At the end of the \texttt{sitpn\_to\_hvhdl} function (see
Listing~\ref{lst:sitpn2hvhdl}), the \texttt{generate\_comp_insts}
function transforms the \texttt{HComponent} instances, associated with
places and transitions in the compile-time state's \texttt{arch}
field, into real component instantiation statements as defined in the
\hvhdl{} abstract syntax. Then, the \texttt{generate_design_and_binder} builds
up the final \hvhdl{} design and the $\gamma$ binder, and returns the couple.
Listing~\ref{lst:gen-comp-insts} presents the 
\texttt{generate_comp_insts} function and the \texttt{generate_design_and_binder} 
function. 

\begin{lstlisting}[language=coq,label={lst:gen-comp-insts},
caption={[The \texttt{generate\_comp\_insts} and the \texttt{generate_design_and_binder} function.]The \texttt{generate\_comp\_insts} and the \texttt{generate_design_and_binder} function.},framexleftmargin=1.5em,xleftmargin=2em,numbers=left,numberstyle=\tiny\ttfamily]
Definition generate_comp_insts (sitpn : Sitpn) : Mon (Sitpn2HVhdlstate sitpn) unit :=
  do _ <- generate_place_comp_insts sitpn; generate_trans_comp_insts sitpn.

Definition generate_design_and_binder (sitpn : Sitpn) ($id_e$ $id_a$ : ident) :
    Mon (Sitpn2HVhdlstate sitpn) (design * Sitpn2HVhdlMap sitpn) :=
  do s <- Get;
  Ret ((design_ $id_e$ $id_a$ [] ((iports s) $\mdoubleplus$ (oports s)) (sigs (arch s)) (beh s)), ($\gamma$ s)).
\end{lstlisting}

The \texttt{generate\_comp\_insts} function is needed because we are
using an intermediary representation for the component instantiation
statements. Even though this representation is convenient to
manipulate data during the different phases of the transformation, it
also implies an extra generation step to complete the generation of
the \hvhdl{} design and the $\gamma$ binder.  The
\texttt{generate_comp_insts} function calls the
\texttt{generate_place_comp_insts} and the
\texttt{generate_trans_comp_insts} functions.  These two functions
being similar in all points, except for the type of their inputs, we
are only presenting the \texttt{generate_place_comp_insts} function
here.  The \texttt{generate_place_comp_insts} function calls the
\texttt{generate\_place\_comp\_inst} function for each place defined
in the set of places of the \texttt{sitpn}
parameter. Listing~\ref{lst:gen-pcomp-inst} presents the code the
\texttt{generate\_place\_comp\_inst} function.

\begin{lstlisting}[language=coq,label={lst:gen-pcomp-inst},
caption={[The \texttt{generate\_place\_comp\_inst} function.]The \texttt{generate\_place\_comp\_inst} function.},framexleftmargin=1.5em,xleftmargin=2em,numbers=left,numberstyle=\tiny\ttfamily]
Definition generate_place_comp_inst (sitpn : Sitpn) (p : P sitpn) :
      Mon (Sitpn2HVhdlstate sitpn) unit :=

      do $id_p$ <- get_nextid;
      do _      <- bind_place p $id_p$;
      do pcomp  <- get_pcomp p;
      do pci    <- HComponent_to_comp_inst $id_p$ place_entid pcomp;
      add_cs pci.
\end{lstlisting}

The \texttt{generate_place_comp_inst} function generates a fresh and unique PCI identifier 
by appealing to the \texttt{get_nextid} function. The \texttt{get_nextid} function returns
and increments the current value of the \texttt{nextid} field, defined in the compile-time state.
Then, the \texttt{bind_place} function adds a binding between the place \texttt{p} and the identifier
$id_p$ in the $\gamma$ field of the compile-time state. The \texttt{get_pcomp} function looks up the 
\texttt{plmap} field (defined under the \texttt{arch} field of the compile-time state) and returns
the \texttt{HComponent} instance associated with the place \texttt{p}, i.e. \texttt{pcomp}.
The \texttt{HComponent_to_comp_inst} function translates the \texttt{HComponent} instance \texttt{pcomp}
into a PCI with the identifier $id_p$. Finally, the \texttt{add_cs} function composes 
the returned PCI with the current \hvhdl{} design behavior, hold in the \texttt{beh} field of the 
compile-time state. 

The transformation of a \texttt{HComponent}
instance into a PCI implies the translation of the input and output port map,
which are instances of the \texttt{InputMap} and \texttt{OutputMap} types, into their equivalent
representation in \hvhdl{} abstract syntax. The translation especially concerns the association
between a port identifier of the array type and a list of expressions, or names.
For instance, let us consider an instance of
\texttt{InputMap} that is an intermediary representation of the input
port map of a PCI $id_p$. In this \texttt{InputMap} instance, the \texttt{itf}
port, which is a composite input port of the \texttt{place}
design, is associated with the list $[id_a, id_b, id_c]$. Then, based on the previous association, the
\texttt{HComponent_to_comp_inst} function generates the following
associations is the concrete input port map of PCI $id_p$: $(\mathtt{rt}(0), id_a)$,
$(\mathtt{rt}(1), id_b)$ and $(\mathtt{rt}(2), id_c)$. 

Getting back to Listing~\ref{lst:gen-comp-insts}, the
\texttt{generate_design_and_binder} function retrieves the current
compile-time state \texttt{s} with the \texttt{Get} function. Then,
based on the value of the different fields of the compile-time state,
the function builds an \hvhdl{} design and returns it along with the
$\gamma$ binder.  The \hvhdl{} design receives the $id_e$ and $id_a$
identifiers, passed as inputs, as the design's entity and architecture
identifiers.  The generic constant declaration list of the \hvhdl{}
design is empty, i.e. it receives the empty list value.  The port
declaration list of the \hvhdl{} design is built by concatenating the
content of the \texttt{iports} and \texttt{oports} fields defined in
state \texttt{s}. The internal signal declaration list is filled by
the \texttt{sigs} field, defined under the \texttt{arch} field of
state \texttt{s}. Finally, the \texttt{beh} field receives the
behavior of the \hvhdl{} design.







%%% Local Variables:
%%% mode: latex
%%% TeX-master: "../../main"
%%% End:


\section{Conclusion}
\label{sec:trans-concl}
The purpose of this chapter was to give to the reader a complete
understanding of the \hilecop{} model-to-text transformation function,
and of what makes it a very specific transformation case. We first
gave an informal presentation of the transformation function with a
high-level view of the transformation principles. Then, we presented
our literature review pertaining to transformation functions in the
context of formal verification, with a particular focus on the
expression and the implementation of transformation functions. Two
points, drawn out from the literature review, are of particular
interest. First, the review showed that it is important, during a
transformation, to keep the binding between the elements of the source
representation and their corresponding versions in the target
representation. This binding is the base of the comparison of the
run-time state of the source and target representation that permits to
express the theorem of semantic preservation. Second, if the distance
between the source and the target representation is too important, it
is easier, while aiming at proving a semantic preservation property,
to split the transformation into multiple simple transformation steps.
Then, to each transformation step will correspond an intermediary
representation, and a theorem of semantic preservation will be laid
out and proved for each one of them. In the case of the \hilecop{}
model-to-text transformation, even though the transformation has a lot
of tricky aspects pertaining to particular cases of input models,
there is no need to split the transformation into simple steps with
intermediary representations. Even though the verification task is
quite close, the \hilecop{} transformation is quite different from the
certified GPL or the HDL compilers presented in the literature
review. Indeed, the source representation is an input model not a
programming language. Moreover, due to the interconnection of the
component instances generated by the transformation function, devising
a transformation algorithm that generates modular and independently
executable code is impossible. As everything is connected, one has to
reason over the entire transformation process to get the overall
behavior of the generated \hvhdl{} design. This is also one of the
main difference between the \hilecop{} transformation and compilers
for programming languages. Despite all that, the transformation
algorithm, presented in this chapter, gets as close as possible to a
modular expression of the \hilecop{} transformation. % Our
% implementation of the transformation algorithm uses an intermediate
% format to represent the component instantiation statements. Even
% though this intermediate representation is convenient to handle datas
% during the transformation, it requires an extra transformation step to
% generate the behavior of the resulting \hvhdl{} design. Thus, this
% intermediate format complexifies the transformation function. In the
% future, we will implement the transformation function closer to the
% expression of the transformation algorithm. This will help reasoning
% over the \texttt{sitpn\_to\_hvhdl} function during the mechanization
% of the proof of semantic preservation.

%%% Local Variables:
%%% mode: latex
%%% TeX-master: "../../main"
%%% End:


%%% Local Variables:
%%% mode: latex
%%% TeX-master: "../main"
%%% End:
