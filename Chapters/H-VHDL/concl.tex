In this chapter, we gave an overview of the VHDL language and its
informal simulation semantics.  Then, considering our needs, that is
considering the content of the VHDL programs generated by the
\hilecop{} model-to-text transformation, we defined a synthesizable
and synchronous subset of the VHDL language called \hvhdl{}. We gave a
small-step semantics to \hvhdl{} by formalizing a simplified
simulation algorithm. The simulation algorithm yields a simulation
trace, i.e. time-ordered list of states, corresponding to the
execution of the behavior of a \hvhdl{} design over multiple clock
cycles. The formalization of the \hvhdl{} semantics also includes the
formalization of the design elaboration. The elaboration, prior to the
simulation, ensures the well-formedness and the well-typedness of a
\hvhdl{} design. Moreover, we have implemented the \hvhdl{} syntax and
semantics with the \coq{} proof assistant.

Ever since the mechanization of the proof of behavior preservation has
begun, the semantics of \hvhdl{} has been
evolving. Section~\ref{sec:abstractSyntax}, \ref{sec:sem-rules},
\ref{sec:elab-rules} and \ref{sec:sim-rules} present the most recent
version of the semantics. However, it will probably be evolving again.
With regards to our proof task, we realized that an operational
semantics close to the simulation algorithm carried a lot of elements
that were of no use. These elements could sometimes complexify the
proof. For instance, in the VHDL simulation algorithm, the body of a
process is executed during the stabilization phase only if one signal
of its sensitivity list is part of the current state's event
set. However, it is through the execution of the body of a process
with the rules of the \hvhdl{} semantics that we can determine the
\emph{combinational} equation associated with the value of a
signal. In the proceeding of the proof of semantic preservation, we
must often describe the value of an output signal with regards to the
value of its input, or \emph{source}, signals
(cf. Section~\ref{sec:detailled-proof}).  Due to the event-based
system of resuming a process activity, a combinational process could
sometimes never be executed during a stabilization phase. Then, we are
not able to determine the value of signals. We had to carry extra
hypotheses in the definition of our lemmas to deal with this
problem. Finally, our current semantics always executes the body
combinational processes during a stabilization phase, and this greatly
simplifies the proof. By doing this kind of simplification, we
realized that we were heading toward a semantics that was closer to
the ``synthesis'' semantics we talked about at the beginning of the
chapter. This semantics tends to get closer to the combinational logic
and the synchronous logic rules. These rules that a hardware system
designer has in mind when devising a model with a hardware description
language.

In the future, we contrive to improve the implementation of the
\hvhdl{} semantics with more dependent types. Especially, the
elaborated design and the design state structure are formally defined
with intentional subsets. These subsets could be easily implemented
with the \texttt{sig} type of the \coq{} proof assistant. Also, we
plan to improve the formalization of the elaboration phase with a
global lookup of multiply-driven signals.


%%% Local Variables:
%%% mode: latex
%%% TeX-master: "../../main"
%%% End:
