Terminals of the language will be written in \texttt{typewriter} font,
or will be enclosed in simple quotes for symbols with no typewriter
representation. The $a^{*}$ denotes a possibly empty repetition of the
element $a$; the $a^{+}$ denotes a non-empty repetition of $a$.

\subsection{Design declaration}
\label{subsec:design-decl}

As in \cite{VanTassel1995}, we define the \emph{design} construct in
the $\mathcal{H}$-VHDL's abstract syntax which has no equivalent in
the concrete syntax of VHDL.

\begin{tabular}{lcl}
  design & ::= & \texttt{design} id$_e$ id$_a$ gens ports sigs cs \\
  gens & ::= & gdecl* \\
  ports & ::= & pdecl* \\
  sigs & ::= & sdecl* \\
\end{tabular}


\subsection{Generic constant, port and internal signal declaration.}
\label{subsec:ent-decl}

\begin{tabular}{lcl}
  gdecl & ::= & \texttt{(}id\texttt{,} $\tau$\texttt{,} e\texttt{)} \\
  pdecl & ::= & \texttt{(}(\vhdle|in||\vhdle|out|)\texttt{,} id\texttt{,} $\tau$\texttt{)}\\
  sdecl & ::= & \texttt{(}id, $\tau$\texttt{)} \\
\end{tabular}

\subsection{Concurrent statements.}
\label{subsec:conc-stmt}

\begin{tabular}{lcl}
  cs & ::= & psstmt | cistmt | cs \texttt{||} cs | \texttt{null} \\
\end{tabular}

\subsubsection{Process statement.}

\begin{tabular}{lcl}
  psstmt & ::= & \vhdle|process| \texttt{(}id$_p$\texttt{,} sl\texttt{,} vars\texttt{,}  ss\texttt{)} \\
  sl & ::= & id$^{*}$ \\
  vars & ::= & vdecl* \\
  vdecl & ::= & \texttt{(}id\texttt{,} $\tau$\texttt{)} \\
\end{tabular}

\subsubsection{Component instantiation statement.}

\begin{tabular}{lcl}
  cistmt & ::= & \vhdle|comp| \texttt{(}id$_c$\texttt{,} id$_e$\texttt{,} gmap\texttt{,} ipmap\texttt{,} opmap\texttt{)} \\
  gmap & ::= & assoc$_g^{*}$ \\
  ipmap & ::= & assoc$_{ip}^{*}$ \\
  opmap & ::= & assoc$_{op}^{*}$\\
  assoc$_g$ & ::= & \texttt{(}id\texttt{,}e\texttt{)} \\
  assoc$_{ip}$ & ::= & \texttt{(}name\texttt{,}e\texttt{)} \\
  assoc$_{op}$ & ::= & \texttt{(}id\texttt{,}(name|\vhdle|open|)\texttt{)}|\texttt{(}id\texttt{(}e\texttt{)}\texttt{,}name\texttt{)}\\
\end{tabular}

\subsection{Sequential statement.}

\begin{tabular}{lcl}
  ss & ::= & name $\mathtt{\Leftarrow}$ e | name \texttt{:=} e | \texttt{if} \texttt{(}e\texttt{)} ss [ss] | \texttt{for (}id\texttt{,}e\texttt{,}e\texttt{)} ss \\
     & &  | \texttt{falling} ss | \texttt{rising} ss | \texttt{rst} ss ss' | ss\texttt{;} ss | \texttt{null} \\
\end{tabular}

\subsection{Expressions, names and types. }
\label{sec:expr-names}

\begin{table}[H]
  \begin{tabular}{lcl}

    e & ::= & e \texttt{and} e | e \texttt{or} e | \texttt{not} e | e \texttt{=} e | e $\neq$ e \\
      & & | e \texttt{<} e | e \texttt{<=} e | e \texttt{>} e | e \texttt{>=} e | e \texttt{+} e | e \texttt{-} e \\
      & & | name | natural | boolean | \texttt{(}e$^{+}$\texttt{)} \\
      & & \\
    name & ::= & id | id\texttt{(} e \texttt{)} \\
    boolean & ::= & \texttt{true} | \texttt{false} \\
    $\tau$ & ::= & \texttt{boolean} | \texttt{natural} \texttt{(}e\texttt{,} e\texttt{)} |
                   \texttt{array} \texttt{(}$\tau$\texttt{,} e\texttt{,} e\texttt{)} \\
  \end{tabular}
\end{table}

Under the expression entry, the natural non-terminal represents the
set of natural numbers ($\mathbb{N}$). The id non-terminal represents
the set of identifiers.

%%% Local Variables:
%%% mode: latex
%%% TeX-master: "../../main"
%%% End:
