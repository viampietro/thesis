The intent here is to give an overview of the VHDL language, its
purpose, its main syntactal constructs, and an informal description of
its semantics as presented in the Language Reference Manual (LRM)
\cite{VHDL2000}. As explained in Chapter~\ref{chap:hilecop}, the
hilecop{} transformation generates a VHDL design implementing the
input SITPN model. To do so, the transformation generates and connects
the component instances of two previously defined VHDL designs: the
place design that implements the concept of a SITPN place, and the
transition design that implements a SITPN transition. These designs
were defined by the INRIA team at the creation of the \hilecop{}
methodology. Here, we use excerpts of the definition of the place and
transition designs to illustrate the content of VHDL programs. The
reader will find the source code of the place and transition designs
in concrete and abstract syntax in Appendices~\ref{app:trans-design}
and \ref{app:trans-design}.

\subsection{Main concepts}
\label{sec:vhdl-main-concepts}

The VHDL acronym stands for Very high speed integrated circuit
Hardware Description Language. As its name indicates, the main purpose
of the VHDL language is to describe hardware circuits. In VHDL, the
concept of program is called \emph{design}. A design corresponds to
the description of a hardware circuit. A VHDL design is composed of
two descriptive parts. The first part is called the entity and
describes the interfaces of circuit, namely: the input and output
ports, and the generic constants. Listing is an excerpt of the
transition design's entity that defines the generic constants, the
input and output port interfaces of the design. Figure is a visual
representation of the interfaces of the transition design.

\begin{lstlisting}[language=VHDL,label={lst:trans-design-entity},
caption={The entity part of the transition design in concrete VHDL
  syntax.},framexleftmargin=1.5em,xleftmargin=2em,numbers=left,
numberstyle=\tiny\ttfamily]
entity transition is
  generic(
    transition_type      : transition_t := NOT_TEMPORAL;
    input_arcs_number    : natural := 1; 
    conditions_number    : natural := 1;
    maximal_time_counter : natural := 1
    );        
  port(
    clock                   : in std_logic;
    reset_n                 : in std_logic;
    input_conditions        : in std_logic_vector(conditions_number-1 downto 0);
    time_A_value            : in natural range 0 to maximal_time_counter;
    time_B_value            : in natural range 0 to maximal_time_counter;
    input_arcs_valid        : in std_logic_vector(input_arcs_number-1 downto 0);
    reinit_time             : in std_logic_vector(input_arcs_number-1 downto 0);
    priority_authorizations : in std_logic_vector(input_arcs_number-1 downto 0);
    fired                   : out std_logic 
    );
end transition;
\end{lstlisting}

The entity generic clause holds the declaration of the generic
constants.  The purpose of generic constants is either to represent
some dimensions of the design (e.g. the size of ports, internal
signals\dots) or to represent constant values used througout the
design. In Listing~\ref{lst:trans-design-entity}, one can see that the
\texttt{conditions_number} generic constant gives a dimensionin to the
type of the \texttt{input_conditions} input port, which is a array of
Boolean values with indexes ranging from 0 to
\texttt{conditions_number-1} (that is the meaning of
\texttt{std_logic_vector (conditions_number-1 downto 0)}. The port
clause holds the declaration of input and output ports of the
design. The \texttt{in} keyword indicates the declaration of an input
port and the \texttt{out} indicates the declaration of an output port.

\begin{figure}[H]
  \centering
  % \includegraphics[keepaspectratio=true, width=\textwidth]{Figures/H-VHDL/transition-entity}
  \caption[A representation of the transition design entity.]{A
    representation of the transition design entity. }}
  \label{fig:pn-example}
\end{figure}



The second part is called the architecture and corresponds
to the description of the internal behavior of the design.



%%% Local Variables:
%%% mode: latex
%%% TeX-master: "../../main"
%%% End:
