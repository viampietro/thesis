\chapter{Proving semantic preservation in \hilecop{}}
\label{chap:proof}

\textsc{Research question of the chapter:}\\
\fbox{\parbox{\textwidth}{\sc What is the strategy to prove that the
    \hilecop{} model-to-text transformation function is
    semantic-preserving?}}

\pnote{Refer to appendix \nameref{app:ind-pciples} for a presentation
  of induction principles that will be used through the proof.}

\section{Tranformation functions and proof strategies}
\label{sec:lit-rev-proof-strat}

\pnote{maybe explain the difference between compiler verification and
  compiler validation, and stress the fact that we do verification}
\pnote{maybe to close to the first
  part of X. Leroy's article on CompCert}

\begin{itemize}
\item Lit. review on proof strategies
  \begin{itemize}
  \item are there usual proof strategies?
  \item do they apply in our case?
  \end{itemize}
\item Present the main results (mostly coming from the first part of X. Leroy's article on CompCert):
  \begin{itemize}
  \item proofs are based on execution steps, and state comparison
  \item \dots
  \end{itemize}
\end{itemize}

\section{Behavior preservation theorem}
\label{sec:beh-pres-thm}

\begin{itemize}
\item preliminary definitions for the proof (+ notations to follow the proof)\\
  \pnote{illustrate the state similarity relation}
\item definitions of main thms + proofs (proofs refer to lemmas that
  will be presented in the next section)
\item illustrating the main thms (execution steps figure)
\end{itemize}

\section{Proving semantic preservation}
\label{sec:sem-pres-proof}

\begin{itemize}
\item Give the proof (the whole proof? maybe a bit long, some part as
  appendices?)
  \begin{enumerate}
  \item Initial states
  \item First step\\
    \pnote{not sure it is really relevant, as it uses most of the
      material coming from the ``Initial states'' part and the ``Rising
      edge'' part}
  \item Rising edge
  \item Falling edge
  \end{enumerate}
\item Illustrate some point of the proof (determine which points) with
  figures\\
  \pnote{illustrate the part that says ``by construction'',
    linking PN parts to \hvhdl{} parts}
\end{itemize}

\section{A detailled proof: equivalence of fired transitions}
\label{sec:detailled-proof}

\bnote{don't know if this is a separate section or if it must be a
  part of the previous section}

\begin{itemize}
\item Informal presentation of the proof
\item formal proof
\item bug detection (illustrated, show the figure)
\end{itemize}

\section{Verification of the proof of behavior preservation}
\label{sec:proof-verif}

\begin{itemize}
\item Present the interesting points in the verification of the proof
  with \coq{}
\end{itemize}

\textsc{Conclusion of the chapter:}
\begin{itemize}
\item Remind of the main proof strategy
\item Remind of the bug detection, the changes on the SITPN and
  \hvhdl{} semantics that are consequences of the proof
\item State of the proof verification. How far am I right now?
\end{itemize}

%%% Local Variables:
%%% mode: latex
%%% TeX-master: "../main"
%%% End:
