\chapter{Conclusion}
\label{chap:concl}

% \section{Contributions}
% \label{sec:contribs}

% \begin{itemize}
% \item Contributions to SITPN
% \item Contributions to VHDL
% \item Contributions to transformation functions
% \item Proof
%   \begin{itemize}
%   \item Bug detections + changes in semantics
%   \item How far are we to complete the verification?
%   \end{itemize}
% \end{itemize}

% \section{Improvements and perspectives}
% \label{sec:improvs-and-perspect}

% \pnote{if proof not yet verified, first thing is not complete this job!}

% \begin{itemize}
% \item improvements on the implementation of \hvhdl{}, get closer to
%   the formal def., more use of dependent types
% \item proving the semantic preservation thm in its existential version, i.e:
%   \begin{itemize}
%   \item the generated design is always elaborable
%   \item the generated design is always simulable, i.e, it will never
%     produce errors as long as the input model verifies some properties
%     (liveness, boundedness\dots)
%   \end{itemize}
% \item consider the whole \hilecop{} high-level model
%   \begin{itemize}
%   \item macroplaces
%   \item GALS
%   \end{itemize}
% \end{itemize}

In the thesis, we were interested in the formal verification of a part
of the \hilecop{} methodology. The \hilecop{} methodology has been
devised to help the engineers to design and to produce critical
digital systems. To summarize the content of
Chapter~\ref{chap:hilecop}, the \hilecop{} methodology proposes a
high-level graphical modelling formalism; the aim of the formalism is
to provide to the engineers a framework to model critical digital
systems in a way that foster communication around designs and formal
analysis ensuring the soundness of the produced models. The formalism
is based on component diagrams and Petri nets. Thus, these models
obtain an executable semantics through the Petri net formalism. The
high-level formalism of \hilecop{} is supposed to talk to the humans;
however the ultimate goal of the methodology is to obtain a physical
version of the critical digital system designed by the engineers on a
FPGA card. Thus, from the state of high-level model to its concrete
implementation on a FPGA card, the designed critical digital system
goes through multiple transformations in the \hilecop{} methodology.
In this thesis, we tackled the formal verification of one of these
transformations: the \textit{model-to-text} transformation from a
flattened Petri net version of the high-level model of the critical
digital system to its implementation into a VHDL design. This
transformation is, of course, performed by a computer program. It was
our purpose to formally verifying that this transformation is
\textit{semantic preserving}; that is, for any input model of the
transformation the corresponding output model behaves in the same way
as the input model. Pragmatically, the research question that we
formalized in the introduction of this thesis was:

\begin{center}
  Can we prove that the model-to-text transformation described in the
  \hilecop{} methodology is semantic preserving?
\end{center}

%%% Local Variables:
%%% mode: latex
%%% TeX-master: "../main"
%%% End:
