Here, we give an overview of the \hilecop{} transformation
function. The goal is to give to the reader the means to appreciate
the difference and the similarities between the \hilecop{}
transformation and the other transformations presented in the
literature review of Section~\ref{sec:transf-lit-rev}. Then,
Section~\ref{sec:trans-alg} will enter the details of the
transformation by presenting the transformation algorithm.

The \hilecop{} model-to-text transformation function takes an SITPN
model as an input; then, it generates a top-level \hvhdl{} design out
of the input model. We will illustrate the \hilecop{} model-to-text
transformation on the input SITPN model presented in
Figure~\ref{fig:transf-toplevel}.

\begin{figure}[H]
  \centering
  \includegraphics[keepaspectratio,width=.8\textwidth]{Figures/Transformation/transf-fun-toplevel}
  \caption[Transformation of an input SITPN model into a top-level
  \hvhdl{} design.]{Transformation of an input SITPN model into a
    top-level \hvhdl{} design.  The input model is composed of two
    places, $p_0$ and $p_1$, and two transitions, $t_0$ and
    $t_1$. Transition $t_0$ is associated with the time interval
    $[1,3]$ and the condition $c_0$. Transition $t_1$ is associated
    with condition $c_1$, and its firing triggers the execution of
    function $f_0$. Action $a_0$ is activated when place $p_0$ is
    marked, and action $a_1$ is activated when place $p_1$ is marked.
  }
  \label{fig:transf-toplevel}
\end{figure}

The generated top-level design implements the structure of the input
SITPN. As a first step, the transformation generates for each place of
the input SITPN a component instance of the \texttt{place} design, and
for each transition of the input SITPN a component instance of the
\texttt{transition} design. These subcomponents constitute the main
part of the \hvhdl{} top-level design's
behavior. Figure~\ref{fig:transf-arch} shows the content of the
behavior of the top-level design after this first generation step.  In
addition to the generation of PCIs and TCIs, all constant values are
produced in the generic map and the input port map of component
instances. The constant values pertain to all the information related
to the structure of the input SITPN. The generic map of TCIS receive
the number of conditions, the type of time interval, and the maximal
value for the time counter associated with their corresponding
transitions. The generic map of PCIs receive the number of input arcs,
the number of output arcs, and the maximal marking associated with
their corresponding places. The maximal marking associated with each
place of the input SITPN is an information passed as a parameter to
the transformation function. This information comes from the analysis
of the input SITPN pertaining to the boundedness of the input
model. This analysis takes place before the transformation into a
\hvhdl{} design in the proceeding of the \hilecop{} methodology. For
now, a global maximal marking is passed as a parameter to the
transformation function; therefore, all PCIs receive the same value
for the \texttt{maximal_marking} generic constant in the first phase
of the transformation. However, we can easily convert the global
maximal marking into a function mapping the places of the input SITPN
to a specific maximal marking value. Thus, each PCI would be
associated with their own maximal marking value at the generation of
their generic map. The input port map of PCIs receive the weight and
type of input arcs, and the weight of output arcs for each input and
output arc of their corresponding places.

\begin{figure}[H]
  \centering
  \includegraphics[keepaspectratio,width=\textwidth]{Figures/Transformation/transf-fun-arch}
  \caption[Generation of the \texttt{place} and \texttt{transition}
  component instances.]{Generation of the \texttt{place} and
    \texttt{transition} component instances based on the set of places
    and transitions of the input SITPN. The \texttt{PCI}
    $\mathtt{id}_{p_0}$ implements the place $p_0$, TCI
    $\mathtt{id}_{t_0}$ the transition $t_0$,\dots}
  \label{fig:transf-arch}
\end{figure}

After the generation of the PCIs, the TCIs, and of the constant parts
of the generic and input port maps, the component instances are
interconnected through their port interfaces. The PCIs and TCIs
interact through their interfaces to exchange
informations. Figure~\ref{fig:transf-inter} illustrates the behavior
of the top-level design after the interconnection of PCIs and TCIs.
For instance, a PCI, implementing a given place $p$, informs its
output TCIS (i.e. the TCIs implementing the output transitions of $p$)
that its current marking enables them. The marking of a PCI is
represented by the value of its internal signal
\texttt{s_marking}. The PCI is the only one to have access to to the
current value of its internal signals. Thus, a PCI must communicate to
its output TCIs about their sensitization status.  To perform this
exchange of information, the transformation generates an internal
signal to connect the subelements of the \texttt{output_arcs_valid}
output port\footnote{The \texttt{output_arcs_valid} output port is a
  composite port, i.e. of the \textit{array} type.}, where
\texttt{output_arcs_valid} is defined in the output port map of the
PCI. Each subelement of the \texttt{output_arcs_valid} port is
connected to one subelement of the \texttt{input_arcs_valid} input
port, where \texttt{input_arcs_valid} is defined in the input port map
of TCIS. Likewise, a TCI informs its input and output PCIs about its
firing status. The transformation generates an internal signal to
connect the \texttt{fired} output port, defined in the output port map
of the TCI, to the \texttt{input_transitions_fired} and
\texttt{output_transitions_fired} input ports, defined in the input
port map of the input and output PCIS. Through the execution of the
internal behavior of each PCI and TCI, and, through the
interconnection of component instances, the transformation aims at
generating a design's behavior that, by its very structure, carries
the rules of the SITPN semantics and conforms to the input SITPN
model. To reduce the size of circuits after the synthesis on an FPGA
card, PCIs and TCIs only communicate with Boolean signals through
their interfaces.

\begin{figure}[H]
  \centering
  \includegraphics[keepaspectratio,width=\textwidth]{Figures/Transformation/transf-fun-inter}
  \caption[Generation of the interconnections between the
  \texttt{place} and \texttt{transition} component
  instances.]{Generation of the interconnections between the
    \texttt{place} and \texttt{transition} component instances. In
    \textcolor{red}{red}, the internal signals interconnecting the
    PCIs and the TCIs. These signals are generated by the
    transformation. The arrows indicate the sense of propagation of
    the information.  In \textcolor{blue}{blue}, the constant values
    produced at the previous transformation step.}
  \label{fig:transf-inter}
\end{figure}

The last part of the transformation deals with the interpretation
elements of the input SITPN, i.e. the conditions, the actions and the
functions. Each condition of the input SITPN leads to the declaration
of a Boolean input port in the port clause of the top-level
design. Then, in the design's behavior, each input port is connected
to the \texttt{input_conditions} input port of TCIs. The
interconnection of an input port of the top-level design to the
\texttt{input_conditions} input port of a TCI reflects an existing
association between a transition and a condition of the input SITPN
model. For each action and function of the input SITPN, the
transformation generates a Boolean output port, a.k.a. an
\emph{action} or a \emph{function} port. At runtime, the value of
these output ports represent the activation or execution status of the
corresponding actions and functions.  To determine the value of the
action and function ports, the transformation generates the
\texttt{action} and the \texttt{function}) process. The
\texttt{action} process is a synchronous process responding to the
falling edge of the clock signal. At the occurrence of the falling
edge of the clock signal, the \texttt{action} process sets the value
of the \emph{action} ports computed from the values of the
\texttt{marked} output ports. The \texttt{marked} output port belongs
to the output port map of PCIs.  Through the Boolean \texttt{marked}
output port, the PCIs inform the outside about their marking status,
i.e. if they possess at least one token or not. The \texttt{function}
process is a synchronous process responding to the rising edge of the
clock signal. At the occurrence of the rising edge of the clock
signal, the \texttt{function} process sets the value of the
\emph{function} ports computed from the values of the \texttt{fired}
output ports. The \texttt{fired} output port belongs to the output
port map of TCIs.  Through the Boolean \texttt{fired} output port, the
TCIs inform the outside about their firing status, i.e. if they are
fired or not. Figure gives the top-level $\mathcal{H}$-VHDL design at
the end of the transformation.

\begin{figure}[H]
  \centering
  \includegraphics[keepaspectratio,width=\textwidth]{Figures/Transformation/transf-fun-ports}
  \caption[Generation of the input and output ports, and of the
  \texttt{action} and the \texttt{function} process in the \hvhdl{}
  top-level design.]{Generation of the input and output ports, and of
    the \texttt{action} and the \texttt{function} process in the
    \hvhdl{} top-level design. The \emph{primary} input port
    $\mathtt{id}_{c_1}$ (resp. $\mathtt{id}_{c_0}$) implements the
    condition $c_1$ (resp. $c_0$). In \textcolor{ForestGreen}{green},
    the internal signals, generated by the transformation, connecting
    the input ports of the top-level design to the
    \texttt{input_conditions} input port of TCIs. The
    $\mathtt{id}_{a_0}$ and $\mathtt{id}_{a_1}$ output ports reflect
    the activation status of actions $a_0$ and $a_1$. The
    $\mathtt{id}_{f_0}$ output port reflect the activation status of
    function $f_0$. In \textcolor{orange}{orange}, the internal
    signals, generated by the transformation, connecting the
    \texttt{marked} and \texttt{fired} output ports of PCIs and TCIs
    to the \texttt{action} and \texttt{function} processes. In
    \textcolor{Purple}{purple}, the representation of the assignment
    performed by the \texttt{action} and \texttt{function} processes
    the action and function ports.}
  \label{fig:transf-ports}
\end{figure}


%%% Local Variables:
%%% mode: latex
%%% TeX-master: "../../main"
%%% End:
