Here, we give an overview of the \hilecop{} transformation
function. The goal is to give to the reader the means to appreciate
the difference and the similarities between the \hilecop{}
transformation and the other transformations presented in the
literature review of Section~\ref{sec:transf-lit-rev}. Then,
Section~\ref{sec:trans-alg} will enter the details of the
transformation by presenting the transformation algorithm.

The \hilecop{} model-to-text transformation function takes an SITPN
model as an input; then, it generates a top-level \hvhdl{} design out
of the input model. We will illustrate the \hilecop{} model-to-text
transformation on the input SITPN model presented in
Figure~\ref{fig:transf-toplevel}.

\begin{figure}[H]
  \centering
  \includegraphics[keepaspectratio,width=\textwidth]{Figures/Transformation/transl-fun-toplevel}
  \caption[Transformation of an input SITPN model.]{Transformation of an input SITPN model into a top-level \hvhdl{} design.}
  \label{fig:transf-toplevel}
\end{figure}

The generated top-level design implements the structure of the input
SITPN. The places and transitions of the SITPN are transformed into
instances of the \texttt{place} and \texttt{transition} designs; the
instances define the behavior of the top-level design.



Also, the \texttt{place} and \texttt{transition} component instances
are connected together through their input and output port
interfaces. These interconnections implement the arc interconnection
between the places and transitions of the input SITPN.


%%% Local Variables:
%%% mode: latex
%%% TeX-master: "../../main"
%%% End:
