This section outlines the main phases of the \hilecop{} model-to-text
transformation function. The goal is to give to the reader the means
to appreciate the differences and the similarities between the
\hilecop{} transformation and the other transformations presented in
the literature review of Section~\ref{sec:transf-lit-rev}. Then,
Section~\ref{sec:trans-alg} will enter the details of the
transformation by presenting the transformation algorithm.

The \hilecop{} model-to-text transformation function takes an SITPN
model as input; then, it generates a top-level \hvhdl{} design out of
the input model. We will illustrate each step of the \hilecop{}
model-to-text transformation through the transformation of the input
SITPN model presented in Figure~\ref{fig:transf-toplevel}.

\begin{figure}[H]
  \centering
  \includegraphics[keepaspectratio,width=.8\textwidth]{Figures/Transformation/transf-fun-toplevel}
  \caption[Transformation of an input SITPN model into a top-level
  \hvhdl{} design.]{Transformation of an input SITPN model into a
    top-level \hvhdl{} design.  The input model is composed of two
    places, $p_0$ and $p_1$, and two transitions, $t_0$ and
    $t_1$. The transition $t_0$ is associated with the time interval
    $[1,3]$ and the condition $c_0$. The transition $t_1$ is associated
    with the condition $c_1$, and its firing triggers the execution of
    the function $f_0$. The action $a_0$ is activated when the place $p_0$ is
    marked, and the action $a_1$ is activated when the place $p_1$ is marked.
  }
  \label{fig:transf-toplevel}
\end{figure}

The generated top-level design implements the structure of the input
SITPN. As a first step, the transformation generates, for each place
of the input SITPN, a component instance of the \texttt{place} design,
and, for each transition of the input SITPN, a component instance of
the \texttt{transition} design. These subcomponents constitute the
main part of the \hvhdl{} top-level design's architecture (i.e. its
internal behavior). Figure~\ref{fig:itfaces} shows a graphical
representation of the input and output port interfaces of the
\texttt{place} and \texttt{transition} designs. All PCIs (Place
Component Instances) and TCIs (Transition Component Instances)
generated during the first step of the \hilecop{} transformation
inherit the interface presented in Figure~\ref{fig:itfaces}.

\begin{figure}[H]
  \centering
  \includegraphics[keepaspectratio,width=\textwidth]{Figures/Transformation/itfaces}
  \caption[The interfaces of the \texttt{place} and
  \texttt{transition} designs.]{On the left, the \texttt{place} design
    interface and on the right the \texttt{transition} design
    interface. The indexes of composite ports are expressed at the
    inner extremity of the pins, while the name (abbreviated) of ports
    are expressed at the outer extremity.}
  \label{fig:itfaces}
\end{figure}


During the first generation step of the \hilecop{} transformation,
each PCI and TCI receive a value for each of their generic constants
through the creation of generic maps.  In the generic map of a TCI
$id_t$ (implementing a transition $t$), the \texttt{ian} constant is
associated with the number of input arcs of $t$, the \texttt{cn}
constant with the number of conditions attached to $t$, etc.  In the
generic map of a PCI $id_p$, the \texttt{ian} constant is associated
with the number of input arcs of $p$, the \texttt{oan} constant with
the number of output arcs of $p$, and the \texttt{mm} constant with
the maximal marking value of $p$. The maximal marking value associated
with a given place $p$ of the input SITPN is an information passed as
a parameter to the transformation function. This information comes
from the analysis of the input SITPN pertaining to the
\textit{boundedness} of the input model. In the definition of the
\hilecop{} methodology, this analysis takes place before the
transformation of the input SITPN into a \hvhdl{} design. The generic
constants do not appear as pins in the interfaces of the
\texttt{place} and \texttt{transition} designs presented in
Figure~\ref{fig:itfaces}. The generic constants have an impact of the
structure of the interface of each component instance. For example,
Figure~\ref{fig:itfaces} shows the dependency between the size
(i.e. the number of pins) of composite ports and the value of generic
constants, e.g. the size of \texttt{iaw} input port of the
\texttt{place} design depends on the \texttt{ian} generic constant.
Thus, the generation of generic maps during this first generation step
corresponds to the \textit{dimensioning} of the PCIs and TCIs; this is
when the number of pins of composite ports are determined.

Figure~\ref{fig:transf-arch} shows the architecture of the top-level
design resulting of the first generation step of the \hilecop{}
transformation.

\begin{figure}[H]
  \centering
  \includegraphics[keepaspectratio,width=\textwidth]{Figures/Transformation/transf-fun-arch}
  \caption[Generation of the \texttt{place} and \texttt{transition}
  component instances.]{Generation of the \texttt{place} and
    \texttt{transition} component instances based on the set of places
    and transitions of the input SITPN. The \texttt{PCI}
    $\mathtt{id}_{p_0}$ implements the place $p_0$, TCI
    $\mathtt{id}_{t_0}$ the transition $t_0$\dots{} In
    \textcolor{red}{red}, the internal signals connected to the
    \texttt{marked} port of PCIs and to the \texttt{fired} port of
    TCIs.}
  \label{fig:transf-arch}
\end{figure}

During the first transformation step, illustrated in
Figure~\ref{fig:transf-arch}, the input and output port maps of PCIs
and TCIs are also partly generated. In the manner of the generic
constants in generic maps, some input ports are associated with
constant values in the input port maps of PCIs and TCIs. All these
associations are generated during this first step. Also, the
\texttt{marked} output port of every PCI is associated with an
internal signal in the output port map of the PCI. The internal signal
will be connected later in the course of the transformation. The same
holds for the \texttt{fired} output port of every TCI.

After the first transformation step, the component instances are
interconnected through their port interfaces.
Figure~\ref{fig:transf-inter} illustrates the behavior of the
top-level design after the interconnection of PCIs and TCIs.

\begin{figure}[H]
  \centering
  \includegraphics[keepaspectratio,width=\textwidth]{Figures/Transformation/transf-fun-inter}
  \caption[Generation of the interconnections between the
  \texttt{place} and \texttt{transition} component
  instances.]{Generation of the interconnections between the
    \texttt{place} and \texttt{transition} component instances. In
    \textcolor{red}{red}, the internal signals interconnecting the
    PCIs and the TCIs. These signals are generated by the
    transformation. The arrows indicate the sense of propagation of
    the information.  In \textcolor{blue}{blue}, the constant associations (i.e. the generic maps and a part of the input port maps)
    produced during the previous transformation step.}
  \label{fig:transf-inter}
\end{figure}

The PCIs and TCIs interact through their interfaces to exchange
informations. For instance, a PCI $id_p$, implementing a given place
$p$, separately informs its output TCIs (i.e. the TCIs implementing
the output transitions of $p$) that its current marking enables
them. The marking of a PCI is represented by the value of its internal
signal \texttt{s_marking}. A PCI is the only one to have access to the
current value of its internal signals. Thus, a PCI must communicate to
its output TCIs their sensitization status.  To perform this exchange
of information, the transformation generates an internal signal to
connect a specific output port of a PCI (the \texttt{oav} port) to a
specific input port of the output TCIs (the \texttt{iav} port).
Likewise, a TCI informs its input and output PCIs about its firing
status. The transformation generates an internal signal to connect the
\texttt{fired} output port of a TCI to the \texttt{itf} and
\texttt{otf} input ports of the input and output PCIs.  These
interconnections are performed by adding new associations in the input
port map and output port map of PCIs and TCIs.  Through the execution
of the internal behavior of each PCI and TCI, and, through the
interconnection of component instances, the transformation aims at
generating a design's behavior that, by its inherent structure,
carries the rules of the SITPN semantics and conforms to the execution
of the input SITPN model.

To reduce the size of circuits after the synthesis on an FPGA or ASIC,
PCIs and TCIs only communicate with Boolean signals through their
interfaces. To restrict the interconnections to Boolean signals, the
\texttt{place} design, which is the mold of all PCIs, carries the arc
information (i.e. the weight and type of its input and output arcs) in
its interface; this approach of encoding the arc information is called
the \textit{place-pivot} approach. Figure~\ref{fig:arc-infos} points
out where the arc information is encoded in the interface of the
\texttt{place} design. Thus, a PCI has all the needed information to
compute the sensitization of its output TCIs by comparing the weight
of its output arcs to its current marking value.  A PCI can simply
communicate through a Boolean signal that it is currently enabling its
output TCIs. In the other approach, the \textit{transition-pivot}
approach, the \texttt{transition} design carries the arc
information. In that case, the TCIs compute their own sensitization
status. To be able to do so, the PCIs must communicate their current
marking value to the TCIs. As a marking value is a natural number, the
number of interconnecting signals between PCIs and TCIs greatly
increases in the \textit{transition-pivot} approach.  Eventually, the
\textit{place-pivot} approach has been retained in the current version
of \hilecop{}.

\begin{figure}[H]
  \centering
  \includegraphics[keepaspectratio,width=.4\textwidth]{Figures/Transformation/arcs-infos}
  \caption[The arcs information in the interface of the \texttt{place}
  design.]{Inside the \textcolor{red}{red} frame, the arc information
    encoded through the \texttt{iaw}, \texttt{oat} and \texttt{oaw}
    input ports in the interface of the \texttt{place} design.}
  \label{fig:arc-infos}
\end{figure}

The last part of the transformation deals with the interpretation
elements of the input SITPN, i.e. the conditions, the actions and the
functions. Each condition of the input SITPN leads to the declaration
of a Boolean input port in the port clause of the top-level design. As
it was pointed out in Chapter~\ref{chap:hilecop-models}
(cf. Section~\ref{subsec:pn-formalism}), the interpretation aspect has
been greatly simplified in the SITPN structure, and the generation and
the association of an input port to each condition of the input SITPN
is a consequence of the simplification. In the \textit{full} version
of the SITPN structure, a condition depends on a Boolean expression
that involves both the value of internal signals and input ports of
the top-level design. In our simplified version of the SITPN
structure, a condition value depends on the execution environment,
i.e. a function that updates the value of conditions at each falling
edge of the clock signal. Thus, we find it natural to transform each
condition into an input port of the top-level design, as the value of
both depends on the execution/simulation environment.  Then, each
input port representing a condition is connected to the \texttt{ic}
input port of TCIs. The interconnection of an input port of the
top-level design to the \texttt{ic} input port of a TCI reflects an
existing association between a transition and a condition of the input
SITPN model.

For each action and function of the input SITPN, the transformation
generates a Boolean output port, a.k.a. an \emph{action} or a
\emph{function} port. At runtime, the value of these output ports
represent the activation or execution status of the corresponding
actions and functions.  To determine the value of the action and
function ports, the transformation generates two processes: the
\texttt{action} process and the \texttt{function} process. The
\texttt{action} process is a synchronous process responding to the
falling edge of the clock signal. At the occurrence of the falling
edge of the clock signal, the \texttt{action} process sets the value
of the \emph{action} ports computed from the values of the multiple
\texttt{marked} output ports\footnote{As one action can be associated
  to multiple places, one action port can depend on the value of
  multiple \texttt{marked} output port.}. The \texttt{marked} port is
an output port of the \texttt{place} design. Through the
\texttt{marked} port, the PCIs inform the outside about their marking
status, i.e. if they possess at least one token or not.  Remember that
the transformation generated an association between the
\texttt{marked} output port and an internal signal in the output port
map of PCIs during the first transformation step. These internal
signals are read by the \texttt{action} process to assign a value to
the \textit{action} ports of the top-level design.  The
\texttt{function} process is a synchronous process responding to the
rising edge of the clock signal. At the occurrence of the rising edge
of the clock signal, the \texttt{function} process sets the value of
the \emph{function} ports computed from the values of the
\texttt{fired} output ports. The \texttt{fired} port is an output port
of the \texttt{transition} design. Through the \texttt{fired} port,
the TCIs inform the outside about their firing status, i.e. if they
are fired or not. Remember that, during the first transformation step,
the transformation generated an association between the \texttt{fired}
output port and an internal signal in the output port map of
TCIs. These internal signals are read by the \texttt{function} process
to assign a value to the \textit{function} ports of the top-level
design. Figure~\ref{fig:transf-ports} presents the top-level
$\mathcal{H}$-VHDL design at the end of the transformation.

\begin{figure}[H]
  \centering
  \includegraphics[keepaspectratio,width=\textwidth]{Figures/Transformation/transf-fun-ports}
  \caption[Generation of the input and output ports, and of the
  \texttt{action} and the \texttt{function} process in the \hvhdl{}
  top-level design.]{Generation of the input and output ports, and of
    the \texttt{action} and the \texttt{function} processes in the
    \hvhdl{} top-level design. The \emph{primary} input port
    $\mathtt{id}_{c_1}$ (resp. $\mathtt{id}_{c_0}$) implements the
    condition $c_1$ (resp. $c_0$). In \textcolor{ForestGreen}{green},
    the internal signals, generated by the transformation, connecting
    the input ports of the top-level design to the
    \texttt{input_conditions} input port of TCIs. The
    $\mathtt{id}_{a_0}$ and $\mathtt{id}_{a_1}$ output ports reflect
    the activation status of the actions $a_0$ and $a_1$. The
    $\mathtt{id}_{f_0}$ output port reflects the activation status of
    the function $f_0$. In \textcolor{orange}{orange}, the internal
    signals, generated by the transformation, connecting the
    \texttt{marked} and \texttt{fired} output ports of PCIs and TCIs
    to the \texttt{action} and \texttt{function} processes. In
    \textcolor{Purple}{purple}, the representation of the assignments
    performed by the \texttt{action} and \texttt{function} processes
    and that set the value of the action and function ports of the
    top-level design.}
  \label{fig:transf-ports}
\end{figure}

%%% Local Variables:
%%% mode: latex
%%% TeX-master: "../../main"
%%% End:
