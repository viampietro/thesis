\section{Conclusion}
\label{sec:concl}

Le but de cette thèse a été de vérifier formellement une partie de la
méthodologie \hilecop{}, usitée dans le cadre de la conception de
circuits numériques critiques.  Spécifiquement, le travail de
vérification porte sur la phase transformant un modèle de conception,
à base de RdPs, en \emph{design} VHDL. Au final, nous avons fait la
démonstration d’un théorème de préservation de comportement pour cette
phase de transformation. La preuve a été effectuée informellement,
mais représente tout de même un volume d'une centaine de pages. La
mécanisation complète de cette preuve avec l'assistant de preuves
\coq{} est en cours de réalisation.

Comme autres perspectives de travail, les modèles d'entrée de la
transformation doivent inclure de nouveaux éléments.  Deux éléments
déjà existant dans ce formalisme restent à prendre en compte : les
macroplaces, qui permettent d'exprimer la gestion d'exceptions dans
les SITPNs, ainsi que la possibilité de spécifier des domaines
d'horloge différents au sein d'un même modèle; c'est le cas des
systèmes Globalement Asynchrones Localement Synchrones (GALS).

Le code \coq{} de la thèse, comprenant la formalisation des
sémantiques des SITPNs et du langage \hvhdl{}, l'implémentation de la
transformation, l'expression du théorème de préservation sémantique
ainsi qu'une partie de la preuve formelle, est accessible à l'adresse:

\begin{center}
  \url{https://github.com/viampietro/ver-hilecop}
\end{center}

%%% Local Variables:
%%% mode: latex
%%% TeX-master: "main"
%%% End:
