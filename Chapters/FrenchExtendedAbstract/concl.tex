\section{Conclusion}
\label{sec:concl}

Le but de la thèse est de vérifier formellement une partie de la
méthodologie \hilecop{}, usitée dans le cadre de la conception de
circuits numériques critiques.  Spécifiquement, le travail de
vérification porte sur la phase transformant un modèle de conception,
à base de RdPs, en \emph{design} VHDL. La finalité de ce travail sera
la spécification et la démonstration d’un théorème de préservation de
comportement pour cette phase de transformation.

Jusqu'ici, la sémantique des SITPNs, modèles de haut niveau de
\hilecop{}, et la sémantique de $\mathcal{H}$-VHDL ont été implantées
à l'aide du langage \coq{}. Concernant les SITPNs, deux éléments déjà
existant dans ce formalisme restent à prendre en compte : les
macroplaces, qui permettent d'exprimer la gestion d'exceptions dans
les SITPNs, ainsi que la possibilité de spécifier des domaines
d'horloge différents au sein d'un même modèle; c'est le cas des
systèmes Globalement Asynchrones Localement Synchrones (GALS).

La transformation d'un SITPN en un modèle VHDL est en cours de
programmation avec le langage \coq{}. Enfin, la dernière étape de ce
travail sera d'établir la preuve de préservation de comportement.

%%% Local Variables:
%%% mode: latex
%%% TeX-master: "main"
%%% End:
