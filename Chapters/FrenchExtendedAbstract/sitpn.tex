\section{Un formalisme de haut-niveau : les réseaux de Petri}

Du fait de leur statut de modèles formels et des possibilités
d'analyse qui en résultent, les RdPs ont été retenus comme modèles de
haut niveau de la méthodologie \hilecop{}. Le but de la méthodologie
étant la conception et la production de circuits numériques
\emph{critiques}, les modèles se doivent d'être validés par analyse
formelle. Afin d'augmenter l'expressivité des modèles, les RdPs
\hilecop{} combinent plusieurs classes connues de RdPs (présentées
ci-après), mais leur particularité réside dans leur exécution
synchrone. Les RdPs \hilecop{} sont nommés SITPNs pour Synchronously
executed Interpreted Time Petri Nets with priorities.

Les SITPNs sont des RdP interprétés; des actions peuvent être
associées aux places d'un réseau, et des fonctions/conditions peuvent
être associées aux transitions. % Actions, fonctions et conditions
% décrivent l'interaction du système modélisé avec son environnement.
% Une action est exécutée tant qu'au moins une place à laquelle elle
% est associée possède un marquage non nul. Une fonction, liée à une
% ou plusieurs transitions, est exécutée lors du tir d'une de ces
% transitions.
Actions et fonctions définissent des opérations sur une ensemble de
variables, ici, des \textit{signaux} \vhdl{}.  Les conditions
associées aux transitions sont des expressions Booléennes sûr la
valeur des signaux. Dans un RdP interprété, une transition est
franchissable si elle est sensibilisée et que toutes les conditions
qui lui sont associées sont \emph{vraies}. La Figure~\ref{fig:ipn}
donne un exemple de RdP interprété.

\begin{figure}[H]
  \centering
  \includegraphics[keepaspectratio=true, width=.6\textwidth]{Figures/SITPN/interpreted-pn}
  \caption[Un exemple de réseau de Petri interprété.]{Un exemple de
    réseau de Petri interprété; sur le côté gauche, le RdP; sur le
    côté droit, les expressions Booléennes associées aux conditions et
    les opérations associées aux actions et fonctions.}
  \label{fig:ipn}
\end{figure}%

Les RdP utilisés dans \hilecop{} sont temporels; une fenêtre de tir,
i.e un intervalle de temps, peut être associée à une transition. Un
compteur de temps est lancé lorsqu'une transition devient
sensibilisée; celle-ci devient franchissable lorsque son compteur de
temps a atteint l'intervalle de tir. La Figure~\ref{fig:tpn} donne un
exemple de RdP temporel. La valeur courante des compteurs de temps est
représentée entre chevrons en dessous des intervalles temporels
associés. En résumé, une transition d'un SITPN est franchissable si
elle est sensibilisée, si toutes les conditions qui lui sont associées
sont vraies et si son compteur de temps est dans l'intervalle défini.

\begin{figure}[H]
  \centering
  \includegraphics[keepaspectratio=true, width=0.2\textwidth]{Figures/SITPN/time-pn}
  \caption[Un exemple de RdP temporel.]{Un exemple de RdP temporel. La
    valeur des compteurs de temps apparaît en rouge.}
  \label{fig:tpn}
\end{figure}

Contrairement au cas général, les SITPNs ont une politique de tir
(i.e, une sémantique) \emph{synchrone}. Fondamentalement, le tir des
transitions d'un RdP est un phéno\-mène indéterministe (si deux
transitions sont franchissables au même instant, tous les ordres de
tirs sont possibles), et asynchrone (dès qu'une transition est
franchissable, elle peut être tirée sans attente). A contrario,
l'évolution d'un SITPN est rythmée par le front montant et le front
descendant d'un signal d'horloge, comme montré dans la
Figure~\ref{fig:sync-exec}. Sur le front descendant (\textcircled{1}
de la Figure~\ref{fig:sync-exec}), toutes les transitions devant être
tirées sont déterminées, ce après mise à jour des conditions et
intervalles de temps; sur le front montant (\textcircled{2} de la
Figure~\ref{fig:sync-exec}), les précédentes transitions sont tirées,
entraînant la mis à jour du marquage du réseau et l'exécution de
fonctions. La sémantique d'évolution d'un tel réseau est synchrone et
déterministe.

% \begin{figure}[!htbp]
%   \centering
%   \includegraphics[keepaspectratio,width=.75\linewidth]{pictures/sync-exec}
%   \caption{Evolution synchrone d'un SITPN.}
%   \label{fig:sync-exec}
% \end{figure}

\begin{figure}[H]
  \centering
  \includegraphics[keepaspectratio=true, width=.9\textwidth]{Figures/SITPN/sync-exec}
  \caption{Evolution d'un SITPN synchronisée avec un signal
    d'horloge.}
  \label{fig:sync-exec}
\end{figure}

La structure et la sémantique des SITPNs ont été formalisées dans
\cite{Leroux2014, Merzoug2018}. La sémantique est exprimée comme un
système états-transitions où les transitions sont étiquetées par les
évènements d'un signal d'horloge. Il y a deux évènements possibles :
le front montant et le front descendant du signal. L'état d'un SITPN
décrit, entre autres, le marquage courant du SITPN, la valeur des
compteurs de temps et des conditions associés aux transitions, la
liste des transitions à tirer\dots La sémantique des SITPNs fixe les
règles de changement d'état en fonction des évènements d'horloge. Par
exemple, sur le front descendant d'horloge, a lliste des transitions à
tirer au prochain front montant est calculée; une règle stipule qu'une
aucune transition non franchissable au front descendant n’appartient à
l’ensemble des transitions à tirer.

La première contribution de la thèse est l'implantation en \coq{} de
la structure et de la sémantique des SITPNs.  La sémantique a été
implantée comme une relation inductive paramétrée par un SITPN, deux
états (i.e, avant et après transition), et un évènement d'horloge. La
relation présente deux cas de construction, un pour chaque évènement
d'horloge considéré.  Afin de tester notre implantation de la
sémantique des SITPNs, un interprète a été conçu, i.e un programme qui
simule les changements d'état d'un SITPN pour $n$ cycles d'horloge, en
partant de l'état initial du réseau. Cet interprète est prouvé correct
et complet vis à vis de la sémantique des SITPNs pour une évolution
sur un cycle d'horloge. L'intégralité de la formalisation et de la
mécanisation est mise à disposition du
lecteur\footnote{\url{https://github.com/viampietro/sitpns}}. Cependant,
nous avons utilisées une autre version de l'implémentation des SITPNs
en \coq{} pour effectuer la preuve de préservation sémantique. La
dernière version est plus élégante et utilisent les types
dépendants\footnote{\url{https://github.com/viampietro/ver-hilecop/tree/master/sitpn/dp}}.

%%% Local Variables:
%%% mode: latex
%%% TeX-master: "../../main"
%%% End:
