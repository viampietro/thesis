\section{Introduction}

Pour répondre aux contraintes liées à la conception de circuits
numériques critiques, et à l'augmentation constante de la complexité
des systèmes, le domaine de l'Ingénierie Système à Base de Modèles
(ISBM) a été développé. L'intérêt est de travailler sur des modèles de
haut niveau avec un pouvoir d'expression et des qualités de
compréhension et de lisibilité qui facilitent les interactions entre
les acteurs de la conception du circuit (i.e, les ingénieurs).
Plusieurs formalismes existent~: le langage SysML
\cite{Friedenthal2014}, des variantes du langage C \cite{Yankova2007},
ou encore les réseaux de Petri (RdPs) \cite{Yakovlev2006}, pour citer
les plus répandus.  Une fois la conception terminée, les modèles sont
physiquement synthétisés en suivant un procédé manuel ou
automatique. Il reste alors à prouver que la phase de transformation
préserve le comportement du modèle de conception.
% Dans le cadre de la production de circuits numériques critiques, les
% méthodes formelles servent notamment à des fins d'analyse des
% modèles de conception. L'analyse permet de vérifier certaines
% propriétés comportementales des circuits modélisés.  Après analyse
% et correction, les modèles sont physiquement synthétisés en suivant
% un procédé manuel ou automatique. Dans le cas d'une approche
% automatique de la synthèse, il reste à prouver que la phase de
% transformation préserve le comportement du modèle de conception.
La présente thèse s'intéresse à la vérification d'un processus d'aide
à la modélisation et à la production de circuits numériques critiques
: la méthodologie \hilecop{} (HIgh LEvel hardware COmponent
Programming).  Cette méthodologie est mise en œuvre dans le cadre de
la création de micro-contrôleurs intégrés à des dispositifs médicaux
de type neuroprothèses.  La Figure \ref{fig:hilecop-wf-french} en décrit les
principales étapes.

\begin{figure}[H]
\centering
\includegraphics[keepaspectratio=true,width=\textwidth]{Figures/Hilecop/hilecop-wf}
\caption[Principe de la méthodologie \hilecop{}.]{Principe de la
  méthodologie \hilecop{}; les double flêches horizontales
  représentent des phases de transformation; les simple flêches
  indiquent les autres types d'opérations ayant cours à une étape
  précise, ou entre étapes.}
\label{fig:hilecop-wf-french}
\end{figure}

Le concepteur de systèmes électroniques esquisse premièrement un
modèle graphique de haut niveau de son circuit (\textcircled{1}). Ce
modèle s'appuie sur le formalisme des diagrammes à composants, avec
l'addition des RdPs pour décrire le comportement interne des parties
du circuit. Dans un deuxième temps, les parties du modèle sont
assemblées et la structure des composants est effacée. Le résultat
obtenu est un réseau de Petri global décrivant le système modélisé
(\textcircled{2}).  Des outils d'analyse exploités par la méthodologie
permettent alors de vérifier certaines propriétés du modèle (caractère
borné, vivacité\dots) et présentent un compte rendu au
concepteur. Après plusieurs itérations du cycle analyse-correction, du
code VHDL est généré à partir du modèle d'implémentation
(\textcircled{3}). Dès lors, la dernière étape de la méthodologie, qui
opère la synthèse du circuit électronique depuis le code source VHDL,
est prise en charge par un compilateur/synthétiseur industriel
propriétaire (\textcircled{4}).

L'objectif de la thèse est de prouver que la transformation du modèle
d'implémentation en code VHDL (i.e, de \textcircled{2} vers
\textcircled{3} dans la Figure~\ref{fig:hilecop-wf-french})
n'introduit pas de divergences de
comportement. % Pour ce faire, la sémantique des
% modèles initiaux et des programmes cibles est considérée. 
Dans cette optique, il sera nécessaire de formaliser la sémantique des
modèles de haut niveau (RdP), du langage cible (VHDL), et de décrire
la transformation. Ensuite, la preuve de similarité comportementale
devra être établie. L'intégralité de la démarche sera mécanisée avec
l'assistant à la preuve \coq{} \cite{Coq2021}. Même si cette démarche
a été éprouvée pour la vérification de compilateurs, son application à
la conception de circuits numériques est bien moins
fréquente. L'intérêt scientifique provient de la distance qui existe
entre le modèle d'exécution du formalisme source (SITPN) et celui du
langage cible (VHDL). Cette distance devra être prise en compte lors
de la preuve de préservation de comportement.

%%% Local Variables:
%%% mode: latex
%%% TeX-master: "../../main"
%%% End:
