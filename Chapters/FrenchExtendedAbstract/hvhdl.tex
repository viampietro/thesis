\section{Un langage cible : VHDL}

% \begin{wrapfigure}{H}{.5\textwidth}
%   \centering 
%   \includegraphics[keepaspectratio,width=.5\textwidth]{pictures/design}
%   \caption{Exemple de design VHDL, un couple entité-architecture.}
%   \label{fig:design}
% \end{wrapfigure}

Il existe plusieurs techniques permettant la synthèse physique d'un
RdP. Cependant, la technique la plus étudiée est la transformation
vers la langage VHDL. Cette technique a donc été retenue par la
méthodologie \hilecop{}. Le langage VHDL permet les descriptions
structurelle et comportementale de circuits électroniques, à des fins
de simulation ou de synthèse physique. En VHDL, un \emph{design}
décrit un composant électronique en termes d'interface entrée-sortie
(l'\emph{entité}) et de comportement interne (l'\emph{architecture}).
Le comportement d'un design s'exprime de deux manières : via
l'interconnexion d'in\-stances d'autres designs (des sous-composants),
ou à l'aide de \emph{processus}.  La spécificité du langage VHDL tient
à l'exécution concurrente des processus et des sous-composants
décrivant une architecture de design. Un processus définit un bloc
d'instructions séquentielles; il observe un certain nombre de signaux
qui composent sa liste de sensibilité. Le changement d'état d'un
signal de cette liste entraîne l'exécution du bloc d'instructions du
processus.  Conceptuellement, un signal VHDL représente une connexion
physique sur un circuit électronique. Les signaux sont les principaux
véhicules des valeurs dans les programmes
VHDL. % Il est possible de les
% confondre jusqu'à une certain point avec les variables de la
% programmation impérative.

La sémantique de VHDL est décrite dans une prose informelle dans le
manuel de référence du langage (MRL). De fait, interpréter un
programme VHDL, qui décrit un \emph{design} de circuit, revient à
simuler le design décrit. Dans le MRL, la sémantique de VHDL est donc
définie sous la forme d'une boucle de simulation. La boucle de
simulation spécifie la dynamique d'exécution des blocs concurrents qui
composent une architecture de design, ainsi que la propagation des
valeurs au travers des signaux.

La littérature propose de nombreuses formalisations de la sémantique
de VHDL \cite{KB12}. Certaines formalisations expriment la boucle de
simulation telle qu'exhibée dans le MRL; d'autres choisissent de
s'abstraire de cette boucle, et optent pour une formalisation
alternative basée sur des modèles permettant la gestion de la
concurrence et du temps (automates temporels, réseaux de Petri,
logique d'intervalles temporels\dots).

La méthodologie \hilecop{} opère la génération d'un design VHDL dans
l'optique de sa synthèse physique. Dès lors, nous ne considérons
qu'une partie \emph{synthétisable} du langage que nous définissons et
nommons $\mathcal{H}$-VHDL. De plus, les designs VHDL générés par la
méthodologie \hilecop{} décrivent des circuits synchrones, i.e, dont
l'exécution est rythmée par un signal d'horloge.  La prise en compte
d'une sous-partie synthétisable et du synchronisme nous a permis
d'exprimer la sémantique des programmes $\mathcal{H}$-VHDL en termes
d'une boucle de simulation bien plus simple en comparaison de celle
exprimée dans le MRL.  L'Algorithme~\ref{alg:sim-loop} décrit notre
boucle spécifique de simulation pour un design $\mathcal{H}$-VHDL.

\begin{figure}[t]
\begin{algorithm}[H]
  \DontPrintSemicolon
  \SetAlFnt{\fontsize{8}{10}\selectfont}

  \AlFnt % overriding the new font
  
  % Beginning of the algorithm.

  \Begin{
    
    \BlankLine

    \tcp{\textcolor{red}{Initialization phase.}}
    $\sigma_1$ = \texttt{RunAllProcessesOnce($\Delta$,$\sigma_{init}$)}\;
    $\sigma_2$ = \texttt{Stabilize($\Delta$,$\sigma_1$)}\;

    \BlankLine
    
    \tcp{\textcolor{red}{Main loop.}}
    $T_c\leftarrow{}0$\;

    \BlankLine

    \While{$T_c\le{}nbCycles$}{
      $\sigma_3$ = \texttt{ExecuteFallingEdgePss($\Delta$,$\sigma_2$)}\;
      $\sigma_4$ = \texttt{Stabilize($\Delta$,$\sigma_3$)}\;
      $\sigma_5$ = \texttt{ExecuteRisingEdgePss($\Delta$,$\sigma_4$)}\;
      $\sigma_2$ = \texttt{Stabilize($\Delta$,$\sigma_5$)}\;
      $T_c\leftarrow{}T_c+1$\;
      
    }
  }
  \caption{SimulationLoop($\Delta$, $\sigma_{init}$, $nbCycles$)}
  \label{alg:sim-loop}
\end{algorithm}
\end{figure}
La boucle de simulation de l'Algorithme~\ref{alg:sim-loop} est
paramétrée par un design VHDL ($\Delta$), l'état initial du design
($\sigma_{init}$) qui contient les valeurs des signaux et les états
courants des sous-composants du design $\Delta$, et le nombre de
cycles de simulation à effectuer ($nbCycles$). Durant la phase
d'initialisation, tous les processus décrivant le comportement du
design sont exécutés une fois. Cette phase est suivie d'une phase de
stabilisation de la valeur des signaux.  La phase de stabilisation
correspond à la propagation des valeurs entre signaux interconnectés,
ce jusqu'à ce que la propagation n'induisent plus aucun changement. La
boucle principale de simulation décrit l'alternance entre l'exécution
des processus dits \emph{séquentiels}, i.e qui sont sensibles aux
évènements d'horloge, et des processus \emph{combinatoires}, qui
s'exécutent de manière continue jusqu'à stabilisation des signaux.

Au stade actuel des travaux, une formalisation de la sémantique de
$\mathcal{H}$-VHDL a été effectuée sous la forme d'une sémantique
opérationnelle à grands pas, et sa mécanisation en \coq{} a été
réalisée. Cette sémantique s'inspire des travaux de formalisation
esquissés dans \cite{VanTassel95, Borrione95}. La sémantique
formalisée prend également en compte la phase d'élaboration du design,
préliminaire à la simulation. L'élaboration génère l'environnement de
simulation, i.e un couple $\Delta,\sigma_{init}$ qui se trouve en
paramètre de la boucle de simulation (voir
Algorithme~\ref{alg:sim-loop}). Durant la phase d'élaboration, une
vérification de type est effectué sur le code VHDL. La vérification de
type s'assure que la partie déclarative et la partie comportementale
du design VHDL respectent certaines règles de typage définies par le
MRL. Par exemple, pour une instruction d'affectation de valeur à un
signal, l'expression affectée doit être du même type que le signal
cible.

%%% Local Variables:
%%% mode: latex
%%% TeX-master: "main"
%%% End:
