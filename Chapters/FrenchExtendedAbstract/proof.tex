\section{Preuve de préservation sémantique}

Le but de cette thèse a été de prouver que la transformation
modèle-vers-texte de \hilecop{} préserve le comportement de ses
modèles d'entrée. Plus précisément, pour un modèle d'entrée de la
transformation, nous voulons prouver que le design de top-niveau
\hvhdl{} résultant se comporte de la même manière. Il est donc d'abord
important de définir la relation nous permettant de comparer un état
d'un SITPN avec un état d'un design de top-niveau \hvhdl{}. Nous avons
défini une relation de similarité entre l'état d'un SITPN et l'état
d'un design \hvhdl{}. C'est à travers cette relation de similarité que
notre théorème de préservation de comportement pourra être exprimé. La
relation de similarité relie les valeurs présentes dans l'état d'un
SITPN aux valeurs de certains éléments, principalement les valeurs de
signaux, présents dans l'état d'un design \hvhdl{}. Pour un état $s$
de SITPN et un état $\sigma$ de design \hvhdl{}, $s$ et $\sigma$ sont
similaires si :

\begin{itemize}
\item Pour toute place $p$, le marquage de $p$ est égal à la valeur du
  signal interne \texttt{s\_marking} d'un composant de type
  \texttt{place} d'identifiant $id_p$ (où $p$ et $id_p$ sont liés par
  la transformation).
\item Pour toute transition $t$, la valeur du compteur de temps
  associé à $t$ est égale à la valeur du signal interne
  \texttt{s_time_counter} d'un composant de type \texttt{transition}
  d'identifiant $id_t$ (où $t$ et $id_t$ sont liés par la
  transformation).
\item Pour toute transition $t$, la valeur de l'ordre de reset associé
  à une transition $t$ est égale à la valeur du signal interne
  \texttt{s_reinit_time_counter} d'un composant de type
  \texttt{transition} d'identifiant $id_t$ (où $t$ et $id_t$ sont liés
  par la transformation).
\item Pour toute condition $c$, la valeur d'une condition $c$ est
  égale à la valeur du port d'entrée $id_c$ représentant la condition
  dans le design de top-niveau \hvhdl{}.
\item Pour toute action $a$, la valeur d'une action $a$ est égale à la
  valeur du port de sortie $id_a$ représentant l'action dans le design
  de top-niveau \hvhdl{}.
\item Pour toute fonction $f$, la valeur d'une fonction $f$ est égale
  à la valeur du port de sortie $id_f$ représentant la fonction dans
  le design de top-niveau \hvhdl{}.
\end{itemize}

Notre théorème de préservation de comportement prend donc la forme
suivante. Pour un modèle SITPN d'entrée et le design de top-niveau
\hvhdl{} résultant de la transformation, si le SITPN renvoie la trace
d'exécution $\theta$, et le design renvoie la trace de simulation
$\theta'$ en s'exécutant pendant $\tau$ cycle d'horloges, alors chaque
couple d'états, considéré dans les traces à un même instant temporel,
vérifie la relation de similarité.

Pour prouver ce théorème, nous avons raisonné par induction sur la
structure des traces d'exécution. Le lemme fondamental pour la preuve
déclare qu'à états de départ similaires, un SITPN et un design
\hvhdl{} liés par la transformation, et qui s'exécutent pendant un
cycle d'horloge, arrivent en fin de cycle à deux états similaires.  La
Figure~\ref{fig:one-cc} exprime graphiquement ce lemme.

\begin{figure}[H]
  \centering
  \includegraphics[keepaspectratio=true, width=.9\textwidth]{Figures/Proof/one-cc}
  \caption{Représentation graphique du lemme déclarant que la
    transformation \hilecop{} préserve la sémantique des modèles
    initiaux pour une exécution sur un cycle d'horloge.}
  \label{fig:one-cc}
\end{figure}

La partie supérieure de la Figure~\ref{fig:one-cc} représente
l'exécution d'un cycle d'horloge pour un modèle SITPN ; la partie
inférieure représente l'exécution d'un cycle d'horloge pour le design
de top-niveau \hvhdl{} résultant de la transformation \hilecop{}
(i.e. $\mathtt{design}=\lfloor{}sitpn\rfloor$ où le symbole de
plancher représente la transformation). Il y a deux types de relation
de similarité entre états, représentés par deux symboles
différents. Le premier symbole $\stackrel{\downarrow}{\approx}$
représente la relation de similarité après front descendant ; le
deuxième symbole $\stackrel{\uparrow}{\approx}$ représente la relation
de similarité après front montant. Selon la phase du cycle d'horloge
considérée, la relation de similarité varie quelque peu.

Nous avons prouvé que la transformation modèle-vers-texte \hilecop{}
vérifie bien la propriété de préservation sémantique. La preuve a été
effectuée informellement sur papier. Elle s'étale sur une centaine de
pages. La mécanisation de la preuve avec l'assistant de preuves \coq{}
est en cours de réalisation.

%%% Local Variables:
%%% mode: latex
%%% TeX-master: "../../main"
%%% End:
