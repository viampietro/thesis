\subsection{Falling Edge and marking}
\label{sec:fe-marking}

%%%%%%%%%%%%%%%%%%%%%%%%%%%%%%%%%%%%%%%%%%%%%%%%%%%%%%
%%%%%%%%%% FALLING EDGE EQUAL MARKING LEMMA %%%%%%%%%%
%%%%%%%%%%%%%%%%%%%%%%%%%%%%%%%%%%%%%%%%%%%%%%%%%%%%%%

\begin{lemma}[Falling edge equal marking]
  \label{lem:fe-equal-marking}
  \fehyps{} then
  $\forall{}p\in{}P,id_p\in{}Comps(\Delta)~s.t.~\gamma(p)=id_p,$
  $~s'.M(p)=\sigma'(id_p)("s\_marking")$.
\end{lemma}

\begin{niproof}
  Given a $p\in{}P$ and an $id\in{}Comps(\Delta)$
  s.t. $\gamma(p)=id_p$, let us show\\
  \fbox{$s'.M(p)=\sigma'(id_p)("s\_marking")$.}

  By definition of
  $E_c,\tau\vdash{}sitpn,s\xrightarrow{\downarrow}s'$, we can deduce
  $s.M(p)=s'.M(p)$.

  By property of the $\mathtt{Inject}_\downarrow$ relation, the
  \hvhdl{} falling edge relation, the stabilize relation and
  \InCsCompP, and through the examination of the \texttt{marking}
  process defined in the place design architecture, we can deduce
  $\sigma'(id_p)("s\_marking")=\sigma(id_p)("s\_marking")$.

  Rewriting the goal with $s.M(p)=s'.M(p)$ and
  $\sigma'(id_p)("sm")=\sigma(id_p)("sm")$:
  \fbox{$s.M(p)=\sigma(id_p)("s\_marking")$.}

  \noindent{}By definition of
  $\gamma,E_c,\tau\vdash{}s\stackrel{\downarrow}{\approx}\sigma$:
  \qedbox{$s.M(p)=\sigma(id_p)("s\_marking")$.}
  
\end{niproof}

%%%%%%%%%%%%%%%%%%%%%%%%%%%%%%%%%%%%%%%%%%%%%%%%%%%%%%%%%%%%%%%
%%%%%%%%%% FALLING EDGE EQUAL OUTPUT TOKEN SUM LEMMA %%%%%%%%%%
%%%%%%%%%%%%%%%%%%%%%%%%%%%%%%%%%%%%%%%%%%%%%%%%%%%%%%%%%%%%%%%

\begin{lemma}[Falling Edge Equal Output Token Sum]
  \label{lem:fe-equal-ots}
  \fehyps{} then $\forall{}p,id_p~s.t.~\gamma(p)=id_p$,
  $\sum\limits_{t\in{}Fired(s')}pre(p,t)=\sigma'(id_p)("s\_output\_token\_sum")$.
\end{lemma}

\begin{niproof}
  Given a $p\in{}P$ and an $id_p\in{}Comps(\Delta)$, let us show\\
  \fbox{$\sum\limits_{t\in{}Fired(s')}pre(p,t)=\sigma'(id_p)("s\_output\_token\_sum")$.}

  \exP{}
  
  By property of the stabilize relation, \InCsCompP{}, and through the
  examination of the \texttt{output_tokens_sum} process defined in the
  place design architecture:
  \begin{equation}
    \label{eq:sots-at-fe}
    \sigma'(id_p)("sots")=\sum\limits_{i=0}^{\Delta(id_p)("oan")-1}
    \begin{cases}
      \sigma'(id_p)("oaw")[i]~\mathtt{if}~(\sigma'(id_p)("otf")[i]~ \\
      \hspace{19.5ex}.~\sigma'(id_p)("oat")[i]=\mathtt{basic}) \\
      0~otherwise \\
    \end{cases}
  \end{equation}

  Rewriting the goal with \eqref{eq:sots-at-fe}:\\
  \begin{equation*}
    \fbox{$
      \sum\limits_{t\in{}Fired(s')}pre(p,t)=\sum\limits_{i=0}^{\Delta(id_p)("oan")-1}
      \begin{cases}
        \sigma'(id_p)("oaw")[i]~\mathtt{if}~(\sigma'(id_p)("otf")[i]~ \\
        \hspace{19.5ex}.~\sigma'(id_p)("oat")[i]=\mathtt{basic}) \\
        0~otherwise \\
      \end{cases}$}
  \end{equation*}

  \noindent{}Let us unfold the definition of the left sum term:\\
  \fbox{
    {\begin{tabular}{c}
       $\sum\limits_{t\in{}Fired(s')}
       \begin{cases}
         \omega~\mathtt{if}~pre(p,t)=(\omega,\mathtt{basic}) \\
         0~otherwise
       \end{cases}$ \\
       $=$ \\
       $\sum\limits_{i=0}^{\Delta(id_p)("oan")-1}
       \begin{cases}
         \sigma'(id_p)("oaw")[i]~\mathtt{if}~(\sigma'(id_p)("otf")[i]~ \\
         \hspace{19.5ex}.~\sigma'(id_p)("oat")[i]=\mathtt{basic}) \\
         0~otherwise \\
       \end{cases}$ \\
     \end{tabular}}
 }\\

 \noindent{}To ease the reading, let us define functions
 $f\in{}Fired(s')\rightarrow\mathbb{N}$ and
 $g\in[0,\vert{}output(p)\vert-1]\rightarrow\mathbb{N}$ s.t.
 $f(t)=\begin{cases}
   \omega~\mathtt{if}~pre(p,t)=(\omega,\mathtt{basic}) \\
   0~otherwise
 \end{cases}$ and $g(i)=\begin{cases}
   \sigma'(id_p)("oaw")[i]~\mathtt{if}~(\sigma'(id_p)("otf")[i]~ \\
   \hspace{19.5ex}.~\sigma'(id_p)("oat")[i]=\mathtt{basic}) \\
   0~otherwise \\
 \end{cases}$

 \noindent{}Then, the goal is: \fbox{$\sum\limits_{t\in{}Fired(s')}f(t)=\sum\limits_{i=0}^{\Delta(id_p)("oan")-1}g(i)$}\\

 \noindent{}Let us perform case analysis on $output(p)$; there are two cases:

 \begin{itemize}
 \item \textbf{CASE} $output(p)=\emptyset$:
   
   By construction,
   ${<}$\texttt{output\_arcs\_number}$\mathtt{\Rightarrow{}1}{>}\in{}gm_p$,
   ${<}$\texttt{output\_arcs\_types(0)}$\Rightarrow{}\mathtt{basic}{>}\in{}ipm_p$,
   ${<}$\texttt{output\_transitions\_fired(0)}$\Rightarrow{}\mathtt{true}{>}\in{}ipm_p$,
   and
   ${<}$\texttt{output\_arcs\_weights(0)}$\mathtt{\Rightarrow{}0}{>}\in{}ipm_p$.

   By property of the elaboration relation and \InCsCompP, we can
   deduce $\Delta(id_p)("oan")=1$.

   By property of the stabilize relation and \InCsCompP, we can deduce
   $\sigma'(id_p)("oat")[0]=\mathtt{basic}$,
   $\sigma'(id_p)("otf")[0]=\mathtt{true}$ and
   $\sigma'(id_p)("oaw")[0]=0$.

   By property of $output(p)=\emptyset$, we can deduce\\
   $\sum\limits_{t\in{}Fired(s')}
   \begin{cases}
     \omega~\mathtt{if}~pre(p,t)=(\omega,\mathtt{basic}) \\
     0~otherwise
   \end{cases}=0$

   \noindent{}Rewriting the goal with $\Delta(id_p)("oan")=1$,
   $\sigma'(id_p)("oat")[0]=\mathtt{basic}$,
   $\sigma'(id_p)("otf")[0]=\mathtt{true}$,
   $\sigma'(id_p)("oaw")[0]=0$ and $\sum\limits_{t\in{}Fired(s')}
   \begin{cases}
     \omega~\mathtt{if}~pre(p,t)=(\omega,\mathtt{basic}) \\
     0~otherwise
   \end{cases}=0$, \qedbox{tautology.}
   
 \item \textbf{CASE} $output(p)\neq\emptyset$:\\

   \noindent{}By construction,
   ${<}\mathtt{output\_arcs\_number\Rightarrow{}}\vert{}output(p)\vert{>}\in{}gm_p$,
   and by property of the elaboration relation, we can deduce
   $\Delta(id_p)("oan")=\vert{}output(p)\vert$.
   
   Rewriting the goal with
   $\Delta(id_p)("oan")=\vert{}output(p)\vert$:
   \fbox{$\sum\limits_{t\in{}Fired(s')}f(t)=\sum\limits_{i=0}^{\vert{}output(p)\vert-1}g(i)$.}

   \noindent{}Let us reason by induction on the right sum term of the goal.

   \begin{itemize}
   \item \textbf{BASE CASE}:

     In that case, $0>\vert{}output\vert-1$ and
     $\sum\limits_{i=0}^{\vert{}output(p)\vert-1}g(i)=0$.

     \noindent{}As $0>\vert{}output\vert-1$, then
     $\vert{}output(p)\vert=0$, thus \qedbox{contradicting
       $output(p)\neq\emptyset$.}

   \item \textbf{INDUCTION CASE}:

     In that case, $0\le\vert{}output(p)\vert-1$.
     
     \begin{ih}
       $\forall{}F\subseteq{}Fired(s'),$
       $g(0)+\sum\limits_{t\in{}F}f(t)=g(0)+\sum\limits_{i=1}^{\vert{}output(p)\vert-1}g(i)$
     \end{ih}

     \fbox{$\sum\limits_{t\in{}Fired(s')}f(t)=g(0)+\sum\limits_{i=1}^{\vert{}output(p)\vert-1}g(i)$}

     By definition of $g$:
     \begin{equation}
       \label{eq:31}g(0)=\begin{cases}
         \sigma'(id_p)("oaw")[0]~\mathtt{if}~(\sigma'(id_p)("otf")[0] \\
         \hspace{19.5ex}.~\sigma'(id_p)("oat")[0]=\mathtt{basic}) \\
         0~otherwise \\
       \end{cases}
     \end{equation}

     \noindent{}Let us perform case analysis on the value of
     $\sigma'(id_p)("otf")[0]~.~\sigma'(id_p)("oat")[0]=\mathtt{basic}$; there are two cases:
     \begin{enumerate}
     \item $(\sigma'(id_p)("otf")[0]~.~\sigma'(id_p)("oat")[0]=\mathtt{basic})=\mathtt{false}$:\\
       In that case, $g(0)=0$, and then we can apply the induction
       hypothesis with $F=Fired(s')$ to solve the goal:
       \qedbox{$\sum\limits_{t\in{}Fired(s')}f(t)=\sum\limits_{i=1}^{\vert{}output(p)\vert-1}g(i)$.}

     \item $(\sigma'(id_p)("otf")[0]~.~\sigma'(id_p)("oat")[0]=\mathtt{basic})=\mathtt{true}$:\\
       In that case, $g(0)=\sigma'(id_p)("oaw")[0]$,
       $\sigma'(id_p)("otf")[0]=\mathtt{true}$ and
       $\sigma'(id_p)("oat")[0]=\mathtt{basic}$.

       \noindent{}By construction, there exist a $t\in{}output(t)$,
       $id_t\in{}Comps(\Delta)$ s.t. $\gamma(t)=id_t$, and there exist
       $gm_t$, $ipm_t$, $opm_t$ s.t. $\mathtt{comp}(id_t,$
       $"transition",$ $gm_t,$ $ipm_t,$ $opm_t)\in{}d.cs$, and there
       exist an $\omega\in\mathbb{N}^{*}$, an
       $a\in\{\mathtt{basic},\mathtt{test},\mathtt{inhib}\}$ and an
       $id_{ft}\in{}Sigs(\Delta)$ such that:
       \begin{itemize}
       \item $pre(p,t)=(\omega,a)$
       \item ${<}\mathtt{output\_arcs\_types(0)\Rightarrow{}}a{>}\in{}ipm_p$
       \item ${<}\mathtt{output\_arcs\_weights(0)\Rightarrow{}}\omega{>}\in{}ipm_p$
       \item ${<}\mathtt{fired\Rightarrow}id_{ft}{>}\in{}opm_t$
       \item
         ${<}\mathtt{output\_transitions\_fired(0)\Rightarrow{}id_{ft}}{>}\in{}ipm_p$
       \end{itemize}
       
       \noindent{}By property of the stabilize relation,
       $\sigma'(id_p)("oat")[0]=\mathtt{basic}$ and
       ${<}$\texttt{output\_arcs\_types(0)}$\mathtt{\Rightarrow{}a}{>}\in{}ipm_p$,
       we can deduce $pre(p,t)=(\omega,\mathtt{basic})$.

       \noindent{}By property of the stabilize relation,
       ${<}\mathtt{fired\Rightarrow{}id_{ft}}{>}\in{}opm_t$,\\
       ${<}\mathtt{output\_transitions\_fired(0)\Rightarrow{}id_{ft}}{>}\in{}ipm_p$
       and $\sigma'(id_p)("otf")[0]=\mathtt{true}$, we can deduce
       $\sigma'(id_t)("fired")=\mathtt{true}$.

       Appealing to Lemma~\ref{lem:fe-equal-fired}, and thanks to
       $\sigma'(id_t)("fired")=\mathtt{true}$, we can deduce
       $t\in{}Fired(s')$.

       With $t\in{}Fired(s')$, we can rewrite the left sum
       term of the goal as follows:\\
       \fbox{$f(t)+\sum\limits_{t'\in{}Fired(s')\setminus\{t\}}f(t')=g(0)+\sum\limits_{i=1}^{\vert{}output(p)\vert-1}g(i)$}

       \noindent{}We know that $g(0)=\sigma'(id_p)("oaw")[0]$, and by
       property of the stabilize relation and
       ${<}$\texttt{output\_arcs\_weights(0)}$\Rightarrow\omega{>}\in{}ipm_p$,
       we can deduce $\sigma'(id_p)("oaw")[0]=\omega$.

       Rewriting the goal with $\sigma'(id_p)("oaw")[0]=\omega$:\\
       \fbox{$f(t)+\sum\limits_{t'\in{}Fired(s')\setminus\{t\}}f(t')=\omega+\sum\limits_{i=1}^{\vert{}output(p)\vert-1}g(i)$}

       By definition of $f$, and as $pre(p,t)=(\omega,\mathtt{basic})$, then $f(t)=\omega$; thus, rewriting the goal:\\
       \fbox{$\omega+\sum\limits_{t'\in{}Fired(s')\setminus\{t\}}f(t')=\omega+\sum\limits_{i=1}^{\vert{}output(p)\vert-1}g(i)$}

       Then, knowing that $g(0)=\omega$, we can apply the induction
       hypothesis with $F=Fired(s')\setminus\{t\}$:
       \qedbox{$g(0)+\sum\limits_{t'\in{}Fired(s')\setminus\{t\}}f(t')=g(0)+\sum\limits_{i=1}^{\vert{}output(p)\vert-1}g(i)$.}
     \end{enumerate}
     
   \end{itemize}

 \end{itemize}
 
\end{niproof}


%%%%%%%%%%%%%%%%%%%%%%%%%%%%%%%%%%%%%%%%%%%%%%%%%%%%%%%%%%%%%%
%%%%%%%%%% FALLING EDGE EQUAL INPUT TOKEN SUM LEMMA %%%%%%%%%%
%%%%%%%%%%%%%%%%%%%%%%%%%%%%%%%%%%%%%%%%%%%%%%%%%%%%%%%%%%%%%%

\begin{lemma}[Falling Edge Equal Input Token Sum]
  \label{lem:fe-equal-its}
  \fehyps{} then $\forall{}p,id_p~s.t.~\gamma(p)=id_p$,
  $~\sum\limits_{t\in{}Fired(s')}post(t,p)=\sigma'_p("s\_input\_token\_sum")$.
\end{lemma}

\begin{niproof}
  Given a $p\in{}P$ and an $id_p\in{}Comps(\Delta)$, let us show\\
  \fbox{$\sum\limits_{t\in{}Fired(s')}post(t,p)=\sigma'(id_p)("s\_input\_token\_sum")$.}\\

  \exP{}
  
  By property of the stabilize relation, \InCsCompP, and through the
  examination of the \texttt{input_tokens_sum} process defined in the
  place design architecture:
  \begin{equation}
    \label{eq:sits-at-fe}
    \sigma'(id_p)("sits")=\sum\limits_{i=0}^{\Delta(id_p)("ian")-1}
    \begin{cases}
      \sigma'(id_p)("iaw")[i]~\mathtt{if}~\sigma'(id_p)("itf")[i]\\
      0~otherwise \\
    \end{cases}
  \end{equation}

  Rewriting the goal with \eqref{eq:sits-at-fe}:\\
  \begin{equation*}
    \fbox{$
      \sum\limits_{t\in{}Fired(s')}post(t,p)=\sum\limits_{i=0}^{\Delta(id_p)("ian")-1}
      \begin{cases}
        \sigma'(id_p)("iaw")[i]~\mathtt{if}~\sigma'(id_p)("otf")[i]\\
        0~otherwise \\
      \end{cases}$}
  \end{equation*}

  Let us unfold the definition of the left sum term:\\
  \begin{frameb}
    \begin{tabular}{c} $\sum\limits_{t\in{}Fired(s')}
      \begin{cases}
        \omega~\mathtt{if}~post(t,p)=\omega \\
        0~otherwise
      \end{cases}$ \\
      $=$ \\
      $\sum\limits_{i=0}^{\Delta(id_p)("ian")-1}
      \begin{cases}
        \sigma'(id_p)("iaw")[i]~\mathtt{if}~\sigma'(id_p)("itf")[i]\\
        0~otherwise \\
      \end{cases}$ \\
    \end{tabular}
  \end{frameb}
  
  Let us perform case analysis on $input(p)$; there are two cases:
  
  \begin{itemize}
  \item \textbf{CASE} $input(p)=\emptyset$:\\
    
    By construction,
    ${<}$\texttt{input\_arcs\_number}$\mathtt{\Rightarrow{}1}{>}\in{}gm_p$,
    ${<}$\texttt{input\_transitions\_fired(0)}$\Rightarrow{}\mathtt{true}{>}\in{}ipm_p$,
    and
    ${<}$\texttt{input\_arcs\_weights(0)}$\mathtt{\Rightarrow{}0}{>}\in{}ipm_p$.

    By property of the elaboration relation and \InCsCompP, we can
    deduce $\Delta(id_p)("ian")=1$.
    
    By property of the stabilize relation and \InCsCompP, we can deduce
    $\sigma'(id_p)("itf")[0]=\mathtt{true}$ and
    $\sigma'(id_p)("iaw")[0]=0$.

    By property of $input(p)=\emptyset$, we can deduce
    $\sum\limits_{t\in{}Fired(s')}
    \begin{cases}
      \omega~\mathtt{if}~post(t,p)=\omega \\
      0~otherwise
    \end{cases}=0$.

    \noindent{}Rewriting the goal with $\Delta(id_p)("ian")=1$,
    $\sigma'(id_p)("itf")[0]=\mathtt{true}$,
    $\sigma'(id_p)("iaw")[0]=0$, and $\sum\limits_{t\in{}Fired(s')}
    \begin{cases}
      \omega~\mathtt{if}~post(t,p)=\omega \\
      0~otherwise
    \end{cases}=0$, and simplifying the goal: \qedbox{tautology.}
    
  \item \textbf{CASE} $input(p)\neq\emptyset$:

    \noindent{}By construction,
    ${<}\mathtt{input\_arcs\_number\Rightarrow{}}\vert{}input(p)\vert{>}\in{}gm_p$,
    and by property of the elaboration relation, we can deduce
    \label{eq:24}$\Delta(id_p)("ian")=\vert{}input(p)\vert$.

    \noindent{}To ease the reading, let us define functions
    $f\in{}Fired(s')\rightarrow\mathbb{N}$ and
    $g\in[0,\vert{}input(p)\vert-1]\rightarrow\mathbb{N}$ s.t.
    $f(t)=\begin{cases}
      \omega~\mathtt{if}~post(t,p)=\omega \\
      0~otherwise
    \end{cases}$ and $g(i)=\begin{cases}
      \sigma'(id_p)("iaw")[i]~\mathtt{if}~\sigma'(id_p)("itf")[i]\\
      0~otherwise \\
    \end{cases}$

    \noindent{}Then, the goal is: \fbox{$\sum\limits_{t\in{}Fired(s')}f(t)=\sum\limits_{i=0}^{\Delta(id_p)("ian")-1}g(i)$}\\
    
    Rewriting the goal with $\Delta(id_p)("ian")=\vert{}input(p)\vert$:
    \fbox{$\sum\limits_{t\in{}Fired(s')}f(t)=\sum\limits_{i=0}^{\vert{}input(p)\vert-1}g(i)$.}

    \noindent{}Let us reason by induction on the right sum term of
    the goal.

    \begin{itemize}
    \item \textbf{BASE CASE}: In that case,
      $0>\vert{}input(p)\vert-1$ and
      $\sum\limits_{i=0}^{\vert{}input(p)\vert-1}g(i)=0$.

      As $0>\vert{}input(p)\vert-1$, then $\vert{}input(p)\vert=0$,
      thus \qedbox{contradicting $input(p)\neq\emptyset$.}
      
    \item \textbf{INDUCTION CASE}: In that case,
      $0\le\vert{}input(p)\vert-1$.
      
      \begin{ih}
        $\forall{}F\subseteq{}Fired(s'),~$
        $g(0)+\sum\limits_{t\in{}F}f(t)=g(0)+\sum\limits_{i=1}^{\vert{}input(p)\vert-1}g(i)$
      \end{ih}

      \fbox{$\sum\limits_{t\in{}Fired(s')}f(t)=g(0)+\sum\limits_{i=1}^{\vert{}input(p)\vert-1}g(i)$}\\

      By definition of $g$, we can deduce 
      $g(0)=\begin{cases}
        \sigma'(id_p)("iaw")[0]~\mathtt{if}~\sigma'(id_p)("itf")[0]\\
        0~otherwise \\
      \end{cases}$

      \noindent{}Let us perform case analysis on the value of
      $\sigma'(id_p)("itf")[0]$; there are two cases:
      \begin{enumerate}
      \item $\sigma'(id_p)("itf")[0]=\mathtt{false}$:\\
        In that case, $g(0)=0$, and then we can apply the induction
        hypothesis with $F=Fired(s')$ to solve the goal:
        \qedbox{$\sum\limits_{t\in{}Fired(s')}f(t)=\sum\limits_{i=1}^{\vert{}input(p)\vert-1}g(i)$.}

      \item $\sigma'(id_p)("itf")[0]=\mathtt{true}$:\\
        In that case, $g(0)=\sigma'(id_p)("iaw")[0]$ and
        $\sigma'(id_p)("itf")[0]=\mathtt{true}$ .

        \noindent{}By construction, there exist a $t\in{}input(t)$,
        an $id_t\in{}Comps(\Delta)$ s.t. $\gamma(t)=id_t$, $gm_t$,
        $ipm_t$, $opm_t$ s.t. $\mathtt{comp}(id_t,$ $"transition",$
        $gm_t,$ $ipm_t,$ $opm)\in{}d.cs$, an
        $\omega\in\mathbb{N}^{*}$ and an $id_{ft}\in{}Sigs(\Delta)$
        such that:
        \begin{itemize}
        \item $post(t,p)=\omega$
        \item
          ${<}\mathtt{input\_arcs\_weights(0)\Rightarrow{}}\omega{>}\in{}ipm_p$
        \item ${<}\mathtt{fired\Rightarrow}id_{ft}{>}\in{}opm_t$
        \item
          ${<}\mathtt{input\_transitions\_fired(0)\Rightarrow{}id_{ft}}{>}\in{}ipm_p$
        \end{itemize}
        
        By property of the stabilize relation,
        ${<}$\texttt{fired}$\Rightarrow{}\mathtt{id_{ft}}{>}\in{}opm_t$,
        ${<}$\texttt{input\_transitions\_fired(0)}$\mathtt{\Rightarrow{}id_{ft}}{>}\in{}ipm_p$
        and $\sigma'(id_p)("itf")[0]=\mathtt{true}$, we can deduce
        $\sigma'(id_t)("fired")=\mathtt{true}$.

        Appealing to Lemma~\ref{lem:fe-equal-fired} and
        $\sigma'(id_t)("fired")=\mathtt{true}$, we can deduce
        $t\in{}Fired(s')$.

        As $t\in{}Fired(s')$, we can rewrite the left sum
        term of the goal as follows:\\
        \fbox{$f(t)+\sum\limits_{t'\in{}Fired(s')\setminus\{t\}}f(t')=g(0)+\sum\limits_{i=1}^{\vert{}input(p)\vert-1}g(i)$}

        We know that $g(0)=\sigma'(id_p)("iaw")[0]$, and by property
        of the stabilize relation and
        ${<}$\texttt{input\_arcs\_weights(0)}$\Rightarrow\omega{>}\in{}ipm_p$,
        we can deduce $\sigma'(id_p)("iaw")[0]=\omega$.

        Rewriting the goal with $\sigma'(id_p)("iaw")[0]=\omega$:\\
        \fbox{$f(t)+\sum\limits_{t'\in{}Fired(s')\setminus\{t\}}f(t')=\omega+\sum\limits_{i=1}^{\vert{}input(p)\vert-1}g(i)$}

        By definition of $f$, and as $post(t,p)=\omega$, then $f(t)=\omega$; thus, rewriting the goal:\\
        \fbox{$\omega+\sum\limits_{t'\in{}Fired(s')\setminus\{t\}}f(t')=\omega+\sum\limits_{i=1}^{\vert{}input(p)\vert-1}g(i)$}

        Then, knowing that $g(0)=\omega$, we can apply the induction
        hypothesis with $F=Fired(s')\setminus\{t\}$:
        \qedbox{$g(0)+\sum\limits_{t'\in{}Fired(s')\setminus\{t\}}f(t')=g(0)+\sum\limits_{i=1}^{\vert{}input(p)\vert-1}g(i)$.}
      \end{enumerate}
      
    \end{itemize}
    
  \end{itemize}
  
\end{niproof}

%%%%%%%%%%%%%%%%%%%%%%%%%%%%%%%%%%%%%%%%%%%%%%%%%%%%%%
%%%%%%%%%% FALLING EDGE EQUAL TIME COUNTERS %%%%%%%%%%
%%%%%%%%%%%%%%%%%%%%%%%%%%%%%%%%%%%%%%%%%%%%%%%%%%%%%%

\subsection{Falling edge and time counters}
\label{sec:fe-equal-tc}

\begin{lemma}[Falling edge equal time counters]
  \label{lem:fe-equal-tc}
  \fehyps{} then $\forall{}t\in{}T_i,id_t\in{}Comps(\Delta)~s.t.~\gamma(t)=id_t,$\\
  $\big(upper(I_s(t))=\infty\land{}s'.I(t)\le{}lower(I_s(t))\Rightarrow{}s'.I(t)=\sigma'(id_t)("s\_time\_counter")\big)$\\
  $\land\big(upper(I_s(t))=\infty\land{}s'.I(t)>{}lower(I_s(t))\Rightarrow{}\sigma'(id_t)("s\_time\_counter")=lower(I_s(t))\big)$\\
  $\land\big(upper(I_s(t))\neq\infty\land{}s'.I(t)>{}upper(I_s(t))\Rightarrow{}\sigma'(id_t)("s\_time\_counter")=upper(I_s(t))\big)$\\
  $\land\big(upper(I_s(t))\neq\infty\land{}s'.I(t)\le{}upper(I_s(t))\Rightarrow{}s'.I(t)=\sigma'(id_t)("s\_time\_counter")\big)$.
\end{lemma}

\begin{niproof}
  Given a $t\in{}T_i$ and an $id_t\in{}Comps(\Delta)$ s.t. $\gamma(t)=id_t$, let us show\\
  \fbox{\parbox{\lwidth}{$\big(upper(I_s(t))=\infty\land{}s'.I(t)\le{}lower(I_s(t))\Rightarrow$
      $s'.I(t)=\sigma'(id_t)("s\_time\_counter")\big)$\\
      $\land\big(upper(I_s(t))=\infty\land{}s'.I(t)>{}lower(I_s(t))\Rightarrow$
      $\sigma'(id_t)("s\_time\_counter")=lower(I_s(t))\big)$\\
      $\land\big(upper(I_s(t))\neq\infty\land{}s'.I(t)>{}upper(I_s(t))\Rightarrow$
      $\sigma'(id_t)("s\_time\_counter")=upper(I_s(t))\big)$\\
      $\land\big(upper(I_s(t))\neq\infty\land{}s'.I(t)\le{}upper(I_s(t))\Rightarrow$
      $s'.I(t)=\sigma'(id_t)("s\_time\_counter")\big)$}}\\

  \exT{}

  \noindent{}By property of the elaboration,
  $\mathtt{Inject}_\downarrow$, \hvhdl{} rising edge and stabilize
  relations, \InCsCompT{}, and through the examination of the
  \texttt{time_counter} process defined in the transition design
  architecture, we can deduce:
  \begin{equation}
    \begin{split}
      \sigma(id_t)("se")=\mathtt{true}\land\Delta(id_t)("tt")\neq\mathtt{NOT\_TEMPORAL}
      \land\sigma(id_t)("srtc")=\mathtt{false}\\
      \land\sigma(id_t)("stc")<\Delta(id_t)("mtc")\Rightarrow
      \sigma'(id_t)("stc")=\sigma(id_t)("stc")+1
    \end{split}
    \label{eq:etnrlt}
  \end{equation}

  \begin{equation}
    \begin{split}
      \sigma(id_t)("se")=\mathtt{true}\land\Delta(id_t)("tt")\neq\mathtt{NOT\_TEMPORAL}
      \land\sigma(id_t)("srtc")=\mathtt{false}\\
      \land\sigma(id_t)("stc")\ge\Delta(id_t)("mtc")\Rightarrow
      \sigma'(id_t)("stc")=\sigma(id_t)("stc")
    \end{split}
    \label{eq:etnrge}
  \end{equation}

  \begin{equation}
    \begin{split}
      \sigma(id_t)("se")=\mathtt{true}\land\Delta(id_t)("tt")\neq\mathtt{NOT\_TEMPORAL}\\
      \land\sigma(id_t)("srtc")=\mathtt{true}\Rightarrow
      \sigma'(id_t)("stc")=1
    \end{split}
    \label{eq:etr}
  \end{equation}

  \begin{equation}
    \begin{split}
      \sigma(id_t)("se")=\mathtt{false}\lor\Delta(id_t)("tt")=\mathtt{NOT\_TEMPORAL}\Rightarrow
      \sigma'(id_t)("stc")=0
    \end{split}
    \label{eq:ne-or-nt}
  \end{equation}
  
  \noindent{}Then, there are 4 points to show:

  \begin{enumerate}
  \item\label{it:fe-eq-tc-fst}
    \fbox{$upper(I_s(t))=\infty\land{}s'.I(t)\le{}lower(I_s(t))\Rightarrow{}s'.I(t)=\sigma'(id_t)("s\_time\_counter")$}\\
    
    Assuming $upper(I_s(t))=\infty$ and
    $s'.I(t)\le{}lower(I_s(t))$, let us show\\
    \fbox{$s'.I(t)=\sigma'(id_t)("s\_time\_counter")$.}

    Let us perform case analysis on $t\in{}Sens(s.M)$; there are two
    cases:

    \begin{enumerate}
    \item \textbf{CASE} $t\notin{}Sens(s.M)$:\\
      By definition of \upSim, we can deduce
      $\sigma(id_t)("se")=\mathtt{false}$.

      Appealing to \eqref{eq:ne-or-nt} and
      $\sigma(id_t)("se")=\mathtt{false}$, we can deduce
      $\sigma'(id_t)("stc")=0$.

      By definition of \dwSitpn{} (Rule~\ref{it:reset-counters}), we
      can deduce $s'.I(t)=0$.

      Rewriting the goal with $\sigma'(id_t)("stc")=0$ and
      $s'.I(t)=0$: \qedbox{tautology.}
      
    \item \textbf{CASE} $t\in{}Sens(s.M)$:
      
      By definition of \upSim, we can deduce
      $\sigma(id_t)("se")=\mathtt{true}$.

      By construction, and as $upper(I_s(t))=\infty$,
      ${<}\mathtt{transition\_type\Rightarrow{}TEMP\_A\_INF}{>}\in{}gm_t$. By
      property of the elaboration relation, we have
      $\Delta(id_t)("tt")=\mathtt{TEMP\_A\_INF}$.

      Let us perform case analysis on $s.reset_t(t)$; there are two
      cases:
      \begin{enumerate}
      \item \textbf{CASE} $s.reset_t(t)=\mathtt{true}$:

        By definition of \upSim, $\sigma(id_t)("srtc")=\mathtt{true}$.

        Appealing to \eqref{eq:etr},
        $\sigma(id_t)("se")=\mathtt{true}$,
        $\Delta(id_t)("tt")=\mathtt{TEMP\_A\_INF}$ and
        $\sigma(id_t)("srtc")=\mathtt{true}$, we can deduce
        $\sigma'(id_t)("stc")=1$.

        By definition of \dwSitpn (Rule~\ref{it:reset-counters}), we
        can deduce $s'.I(t)=1$.

        Rewriting the goal with $\sigma'(id_t)("stc")=1$ and
        $s'.I(t)=1$: \qedbox{tautology.}

      \item \textbf{CASE} $s.reset_t(t)=\mathtt{false}$:
        
        By definition of \upSim, we have
        $\sigma(id_t)("srtc")=\mathtt{false}$.

        As $upper(I_s(t))=\infty$, there exists an
        $a\in\mathbb{N}^{*}$ s.t. $I_s(t)=[a,\infty]$. Let us take
        such an $a\in\mathbb{N}^{*}$. By construction,
        ${<}\mathtt{maximal\_time\_counter\Rightarrow}~a{>}\in{}gm_t$,
        and by property of the elaboration relation, we have
        $\Delta(id_t)("mtc")=a$.

        By definition of \dwSitpn (Rule~\ref{it:inc-counters}), and
        knowing that $t\in{}Sens(s.M)$, $s.reset_t(t)=\mathtt{false}$
        and $upper(I_s(t))=\infty$, we can deduce $s'.I(t)=s.I(t)+1$.

        Rewriting the goal with $s'.I(t)=s.I(t)+1$:
        \fbox{$s.I(t)+1=\sigma'(id_t)("stc")$.}
        
        We assumed that $s'.I(t)\le{}lower(I_s(t))$, and as
        $s'.I(t)=s.I(t)+1$, then $s.I(t)+1\le{}lower(I_s(t))$, then
        $s.I(t)<lower(I_s(t))$, then $s.I(t)<a$ since
        $a=lower(I_s(t))$.

        \noindent{}By definition of
        $\gamma,E_c,\tau\vdash{}s\stackrel{\uparrow}{\approx}\sigma$,
        and knowing that $s.I(t)<lower(I_s(t))$ and
        $upper(I_s(t))=\infty$, we can deduce
        $s.I(t)=\sigma(id_t)("stc")$.

        Appealing to $\Delta(id_t)("mtc")=a$,
        $s.I(t)=\sigma(id_t)("stc")$ and $s.I(t)<a$, we can deduce
        $\sigma(id_t)("stc")<\Delta(id_t)("mtc")$.
        
        Appealing to \eqref{eq:etnrlt},
        $\sigma(id_t)("stc")<\Delta(id_t)("mtc")$,
        $\sigma(id_t)("srtc")=\mathtt{false}$ and\\
        $\sigma(id_t)("se")=\mathtt{true}$, we can deduce:
        $\sigma'(id_t)("stc")=\sigma(id_t)("stc")+1$.

        Rewriting the goal with
        $\sigma'(id_t)("stc")=\sigma(id_t)("stc")+1$ and
        $s.I(t)=\sigma(id_t)("stc")$: \qedbox{tautology.}
      \end{enumerate}
    \end{enumerate}
    
  \item
    \fbox{$upper(I_s(t))=\infty\land{}s'.I(t)>{}lower(I_s(t))\Rightarrow$
      $\sigma'(id_t)("s\_time\_counter")=lower(I_s(t)$.}

    Assuming that $upper(I_s(t))=\infty$ and
    $s'.I(t)>{}lower(I_s(t))$, let us show\\
    \fbox{$\sigma'(id_t)("s\_time\_counter")=lower(I_s(t))$.}

    As $upper(I_s(t))=\infty$, there exists an $a\in\mathbb{N}^{*}$
    s.t. $I_s(t)=[a,\infty]$. Let us take such an
    $a\in\mathbb{N}^{*}$.

    By construction,
    ${<}$\texttt{maximal\_time\_counter}$\Rightarrow{}a{>}\in{}gm_t$,
    and
    ${<}$\texttt{transition\_type}$\Rightarrow{}$\texttt{TEMP\_A\_INF}${>}\in{}gm_t$
    by property of the elaboration relation, we can deduce
    $\Delta(id_t)("mtc")=a$ and
    $\Delta(id_t)("tt")=\mathtt{TEMP\_A\_INF}$.
    
    Let us perform case analysis on $t\in{}Sens(s.M)$:
    \begin{enumerate}
    \item \textbf{CASE} $t\notin{}Sens(s.M)$:
      
      By definition of $E_c,\tau\vdash{}s\xrightarrow{\downarrow}s'$
      (Rule~\ref{it:reset-not-sens}), and knowing that
      $t\in{}Sens(s.M)$, we can deduce $s'.I(t)=0$. Since
      $lower(I_s(t))\in\mathbb{N}^{*}$, then $lower(I_s(t))>0$.
      
      \qedbox{Contradicts $s'.I(t)>lower(I_s(t))$.}
      
    \item \textbf{CASE} $t\in{}Sens(s.M)$:
      
      By definition of
      $\gamma,E_c,\tau\vdash{}s\stackrel{\uparrow}{\sim}\sigma$ and
      $t\in{}Sens(s.M)$, we can deduce
      $\sigma(id_t)("se")=\mathtt{true}$.

      Let us perform case analysis on $s.reset_t(t)$; there are two
      cases:
      \begin{enumerate}
      \item \textbf{CASE} $s.reset_t(t)=\mathtt{true}$:
        
        By definition of
        $E_c,\tau\vdash{}s\xrightarrow{\downarrow}s'$: $s'.I(t)=1$.

        We assumed that $s'.I(t)>lower(I_s(t))$, then
        $1>lower(I_s(t))$.

        \qedbox{Contradicts $lower(I_s(t))>0$.}

      \item \textbf{CASE} $s.reset_t(t)=\mathtt{false}$:
        
        By property of
        $\gamma,E_c,\tau\vdash{}s\stackrel{\uparrow}{\approx}\sigma$
        and $s.reset_t(t)=\mathtt{false}$, we can deduce
        $\sigma(id_t)("srtc")=\mathtt{false}$.

        By definition of $E_c,\tau\vdash{}s\xrightarrow{\downarrow}s'$
        (Rule~\ref{it:inc-counters}), and knowing that
        $s'.I(t)>lower(I_s(t))$, we can deduce
        \begin{equation*}
          \begin{split}
            s'.I(t)=s.I(t)+1&\Rightarrow{}s.I(t)+1>lower(I_s(t))\\
            & \Rightarrow{}s.I(t)\ge{}lower(I_s(t))\\
          \end{split}
        \end{equation*}

        Let us perform case analysis on $s.I(t)\ge{}lower(I_s(t))$:
        \begin{enumerate}
        \item \textbf{CASE} $s.I(t)>lower(I_s(t))$: \fbox{$\sigma'(id_t)("stc")=lower(I_s(t))$.}\\
          By definition of
          $\gamma,E_c,\tau\vdash{}s\stackrel{\uparrow}{\approx}\sigma$,
          we can deduce $\sigma(id_t)("stc")=lower(I_s(t))$.

          Appealing to \eqref{eq:etnrge}, we can deduce
          $\sigma'(id_t)("stc")=\sigma(id_t)("stc")$.

          Rewriting the goal with
          $\sigma'(id_t)("stc")=\sigma(id_t)("stc")$ and
          $\sigma(id_t)("stc")=lower(I_s(t))$: \qedbox{tautology.}
          
        \item \textbf{CASE} $s.I(t)=lower(I_s(t))$: \fbox{$\sigma'(id_t)("stc")=lower(I_s(t))$.}\\
          By definition of
          $\gamma,E_c,\tau\vdash{}s\stackrel{\uparrow}{\approx}\sigma$,
          we can deduce $s.I(t)=\sigma(id_t)("stc")$.

          Appealing to \eqref{eq:etnrge}, we can deduce
          $\sigma'(id_t)("stc")=\sigma(id_t)("stc")$.

          Rewriting the goal with
          $\sigma'(id_t)("stc")=\sigma(id_t)("stc")$,
          $s.I(t)=\sigma(id_t)("stc")$ and $s.I(t)=lower(I_s(t))$:
          \qedbox{tautology.}
        \end{enumerate}
      \end{enumerate}
    \end{enumerate}
  \item
    \fbox{$upper(I_s(t))\neq\infty\land{}s'.I(t)>{}upper(I_s(t))\Rightarrow$
      $\sigma'(id_t)("s\_time\_counter")=upper(I_s(t))$.}

    Assuming that $upper(I_s(t))\neq\infty$ and
    $s'.I(t)>{}upper(I_s(t))$, let us show \\
    \fbox{$\sigma'(id_t)("s\_time\_counter")=upper(I_s(t))$.}

    As $upper(I_s(t))\neq\infty$, there exists an
    $a\in\mathbb{N}^{*}$, and a $b\in\mathbb{N}^{*}$
    s.t. $I_s(t)=[a,b]$. Let us take such an $a$ and $b$.

    By construction,
    ${<}\mathtt{maximal\_time\_counter\Rightarrow}b{>}\in{}gm_t$ and
    there exists $tt\in\{\mathtt{TEMP\_A\_A,TEMP\_A\_B\}}$ s.t.
    ${<}\mathtt{transition\_type\Rightarrow}tt{>}\in{}gm_t$.

    By property of the elaboration relation and \InCsCompT, we can
    deduce $\Delta(id_t)("mtc")=b=upper(I_s(t))$ and
    $\Delta(id_t)("tt")\neq\mathtt{NOT\_TEMP}$.

    Let us perform case analysis on $t\in{}Sens(s.M)$:
      \begin{enumerate}
      \item \textbf{CASE} $t\notin{}Sens(s.M)$:\\
        By definition of $E_c,\tau\vdash{}s\xrightarrow{\downarrow}s'$
        (Rule~\ref{it:reset-not-sens}), and knowing that
        $t\in{}Sens(s.M)$, then $s'.I(t)=0$. Since
        $upper(I_s(t))\in\mathbb{N}^{*}$, then $upper(I_s(t))>0$.
        
        \qedbox{Contradicts $s'.I(t)>upper(I_s(t))$.}
        
      \item \textbf{CASE} $t\in{}Sens(s.M)$:\\
        By definition of
        $\gamma,E_c,\tau\vdash{}s\stackrel{\uparrow}{\approx}\sigma$
        and $t\in{}Sens(s.M)$, we can deduce
        $\sigma(id_t)("se")=\mathtt{true}$.

        Let us perform case analysis on $s.reset_t(t)$; there are two
        cases:
        \begin{enumerate}
        \item \textbf{CASE} $s.reset_t(t)=\mathtt{true}$:\\
          By definition of
          $E_c,\tau\vdash{}s\xrightarrow{\downarrow}s'$
          (Rule~\ref{it:reset-counters}), we can deduce $s'.I(t)=1$.

          \noindent{}We assumed that $s'.I(t)>upper(I_s(t))$, then we
          can deduce $1>upper(I_s(t))$.

          \qedbox{Contradicts $upper(I_s(t))>0$.}

        \item \textbf{CASE} $s.reset_t(t)=\mathtt{false}$:\\
          By property of
          $\gamma,E_c,\tau\vdash{}s\stackrel{\uparrow}{\approx}\sigma$
          and $s.reset_t(t)=\mathtt{false}$, we can deduce
          $\sigma(id_t)("srtc")=\mathtt{false}$.
          
          Let us perform case analysis on $s.I(t)>{}upper(I_s(t))$ or
          $s.I(t)\le{}upper(I_s(t))$:
          \begin{enumerate}
          \item \textbf{CASE} $s.I(t)>upper(I_s(t))$: \fbox{$\sigma'(id_t)("stc")=upper(I_s(t))$.}\\
            By definition of \dwSitpn (Rule~\ref{it:locked-counters}),
            we can deduce $s'.I(t)=s.I(t)$.
            
            By definition of
            $\gamma,E_c,\tau\vdash{}s\stackrel{\uparrow}{\approx}\sigma$,
            we can deduce $\sigma(id_t)("stc")=upper(I_s(t))$.

            Appealing to \eqref{eq:etnrge}, we have
            $\sigma'(id_t)("stc")=\sigma(id_t)("stc")$.

            Rewriting the goal with
            $\sigma'(id_t)("stc")=\sigma(id_t)("stc")$ and
            $\sigma(id_t)("stc")=upper(I_s(t))$: \qedbox{tautology.}
            
          \item \textbf{CASE} $s.I(t)\le{}upper(I_s(t))$: \fbox{$\sigma'(id_t)("stc")=upper(I_s(t))$.}\\

            By definition of \upSim, we can deduce
            $s.I(t)=\sigma(id_t)("stc")$.
            
            Let us perform case analysis on
            $s.I(t)\le{}upper(I_s(t))$; there are two cases:
            \begin{itemize}
            \item \textbf{CASE} $s.I(t)=upper(I_s(t))$:\\

              Appealing to $\Delta(id_t)("mtc")=b=upper(I_s(t))$,
              $s.I(t)=\sigma(id_t)("stc")$ and $s.I(t)=upper(I_s(t))$,
              we can deduce
              $\Delta(id_t)("mtc")\le\sigma(id_t)("stc")$.

              Appealing to $\Delta(id_t)("mtc")\le\sigma(id_t)("stc")$
              and \eqref{eq:etnrge}, we can deduce
              $\sigma'(id_t)("stc")=\sigma(id_t)("stc")$.

              Rewriting the goal with
              $\sigma'(id_t)("stc")=\sigma(id_t)("stc")$,
              $s.I(t)=\sigma(id_t)("stc")$ and $s.I(t)=upper(I_s(t))$:
              \qedbox{tautology.}
              
            \item \textbf{CASE} $s.I(t)<upper(I_s(t))$:\\
              
              By definition of \dwSitpn (Rule~\ref{it:inc-counters}),
              we can deduce $s'.I(t)=s.I(t)+1$.

              From $s'.I(t)=s.I(t)+1$ and $s.I(t)<upper(I_s(t))$, we
              can deduce $s'.I(t)\le{}upper(I_s(t))$;
              \qedbox{contradicts $s'.I(t)>upper(I_s(t))$.}
            \end{itemize}
          \end{enumerate}
        \end{enumerate}
      \end{enumerate}
      
    \item
      \fbox{$upper(I_s(t))\neq\infty\land{}s'.I(t)\le{}upper(I_s(t))\Rightarrow$
        $s'.I(t)=\sigma'(id_t)("s\_time\_counter")$.}

      Assuming that $upper(I_s(t))\neq\infty$ and
      $s'.I(t)\le{}upper(I_s(t))$, let us show\\
      \fbox{$s'.I(t)=\sigma'(id_t)("s\_time\_counter")$.}

      As $upper(I_s(t))\neq\infty$, there exists an
      $a\in\mathbb{N}^{*}$, and a $b\in\mathbb{N}^{*}$
      s.t. $I_s(t)=[a,b]$. Let us take such an $a$ and $b$.

      By construction,
      ${<}\mathtt{maximal\_time\_counter\Rightarrow}b{>}\in{}gm_t$ and
      there exists $tt\in\{$\texttt{TEMP\_A\_A,TEMP\_A\_B}$\}$ s.t.
      ${<}\mathtt{transition\_type\Rightarrow}tt{>}\in{}gm_t$; by
      property of the elaboration relation, we can deduce
      $\Delta(id_t)("mtc")=b=upper(I_s(t))$ and
      $\Delta(id_t)("tt")\neq\mathtt{NOT\_TEMP}$.
      
      Let us perform case analysis on $t\in{}Sens(s.M)$:
      \begin{enumerate}
      \item \textbf{CASE} $t\notin{}Sens(s.M)$:\\
        
        By definition of \upSim, we have
        $\sigma(id_t)("se")=\mathtt{false}$.

        \noindent{}Appealing \eqref{eq:ne-or-nt} and
        $\sigma(id_t)("se")=\mathtt{false}$, we have
        $\sigma'(id_t)("stc")=0$.

        By definition of \dwSitpn (Rule~\ref{it:reset-not-sens}), we
        have $s'.I(t)=0$.

        Rewriting the goal with $\sigma'(id_t)("stc")=0$ and
        $s'.I(t)=0$: \qedbox{tautology.}
        
      \item \textbf{CASE} $t\in{}Sens(s.M)$:\\
        
        By definition of \upSim, we have
        $\sigma(id_t)("se")=\mathtt{true}$.

        Let us perform case analysis on $s.reset_t(t)$:
        \begin{enumerate}
        \item \textbf{CASE} $s.reset_t(t)=\mathtt{true}$:\\
          
          By definition of \upSim, we have
          $\sigma(id_t)("srtc")=\mathtt{true}$.

          Appealing to \eqref{eq:etr},
          $\Delta(id_t)("tt")\neq\mathtt{NOT\_TEMP}$,
          $\sigma(id_t)("se")=\mathtt{true}$ and
          $\sigma(id_t)("srtc")=\mathtt{true}$, we have
          $\sigma'(id_t)("stc")=1$.

          By definition of \dwSitpn (Rule~\ref{it:reset-counters}), we
          have $s'.I(t)=1$.

          Rewriting the goal with $\sigma'(id_t)("stc")=1$ and
          $s'.I(t)=1$, \qedbox{tautology.}
          
        \item \textbf{CASE} $s.reset_t(t)=\mathtt{false}$:\\
          
          By definition of \upSim, we have
          $\sigma(id_t)("srtc")=\mathtt{false}$.

          Let us perform case analysis on $s.I(t)>upper(I_s(t))$ or
          $s.I(t)\le{}upper(I_s(t))$:
          \begin{enumerate}
          \item \textbf{CASE} $s.I(t)>upper(I_s(t))$:\\
            
            By definition of \dwSitpn, we have $s.I(t)=s'.I(t)$, and
            thus, $s'.I(t)>upper(I_s(t))$. \qedbox{Contradicts
              $s'.I(t)\le{}upper(I_s(t))$.}
            
          \item \textbf{CASE} $s.I(t)\le{}upper(I_s(t))$:\\
            
            By definition of \upSim, we have
            $s.I(t)=\sigma(id_t)("stc")$.

            \begin{itemize}
            \item \textbf{CASE} $s.I(t)<upper(I_s(t))$:\\
              From $s.I(t)<upper(I_s(t))$,
              $s.I(t)=\sigma(id_t)("stc")$ and
              $\Delta(id_t)("mtc")=b=upper(I_s(t))$, we can deduce
              $\sigma(id_t)("stc")<\Delta(id_t)("mtc")$.

              From \eqref{eq:etnrlt},
              $\sigma(id_t)("se")=\mathtt{true}$,
              $\Delta(id_t)("tt")\neq\mathtt{NOT\_TEMP}$,
              $\sigma(id_t)("srtc")=\mathtt{false}$ and
              $\sigma(id_t)("stc")<\Delta(id_t)("mtc")$, we can deduce
              $\sigma'(id_t)("stc")=\sigma(id_t)("stc")+1$.

              By definition of \dwSitpn (Rule~\ref{it:inc-counters}),
              we can deduce $s'.I(t)=s.I(t)+1$.

              Rewriting the goal with
              $\sigma'(id_t)("stc")=\sigma(id_t)("stc")+1$ and
              $s'.I(t)=s.I(t)+1$, \qedbox{tautology.}
              
            \item \textbf{CASE} $s.I(t)=upper(I_s(t))$:\\
              By definition of \dwSitpn (Rule~\ref{it:inc-counters}),
              we know that $s'.I(t)=s.I(t)+1$. We assumed that
              $s'.I(t)\le{}upper(I_s(t))$; thus,
              $s.I(t)+1\le{}upper(I_s(t))$.

              \qedbox{Contradicts $s.I(t)=upper(I_s(t))$.}
              
            \end{itemize}
            
          \end{enumerate}
          
        \end{enumerate}
        
      \end{enumerate}
    \end{enumerate}
\end{niproof}

\subsection{Falling edge and condition values}
\label{sec:fe-cond-values}

\begin{lemma}[Falling edge equal condition values]
  \label{lem:fe-equal-cond-values}
  \fehyps{} then
  $\forall{}c\in\mathcal{C},id_c\in{}Ins(\Delta)~s.t.~\gamma(c)=id_c,~s'.cond(c)=\sigma'(id_c)$.
\end{lemma}

\begin{niproof}
  Given a $c\in\mathcal{C}$ and an $id_c\in{}Ins(\Delta)$
  s.t. $\gamma(c)=id_c$, let us show
  \fbox{$s'.cond(c)=\sigma'(id_c)$.}

  By definition of \dwSitpn (Rule~\ref{it:cond-env}), we have
  $s'.cond(c)=E_c(\tau,c)$.

  By property of the $\mathtt{Inject_\downarrow}$, the \hvhdl{}
  falling edge, the stabilize relations and $id_c\in{}Ins(\Delta)$, we
  have $\sigma'(id_c)=E_p(\tau,\downarrow)(id_c)$.

  \noindent{}Rewriting the goal with $s'.cond(c)=E_c(\tau,c)$ and
  $\sigma'(id_c)=E_p(\tau,\downarrow)(id_c)$:
  \fbox{$E_c(\tau,c)=E_p(\tau,\downarrow)(id_c)$}
  
  \noindent{}By definition of $\gamma\vdash{}E_p\stackrel{env}{=}E_c$:
  \qedbox{$E_c(\tau,c)=E_p(\tau,\downarrow)(id_c)$.}
  
\end{niproof}

\subsection{Falling and action executions}
\label{sec:fe-equal-act-exec}

\begin{lemma}[Falling edge equal action executions]
  \label{lem:fe-equal-act-exec}
  \fehyps{} then
  $\forall{}a\in\mathcal{A},id_a\in{}Outs(\Delta)~s.t.~\gamma(a)=id_a,~s'.ex(a)=\sigma'(id_a)$.
\end{lemma}

\begin{niproof}
  Given an $a\in\mathcal{A}$ and an $id_a\in{}Outs(\Delta)$
  s.t. $\gamma(a)=id_a$, let us show \fbox{$s'.ex(a)=\sigma'(id_a)$.}

  \noindent{}By property of \dwSitpn (Rule~\ref{it:activate-actions}):
  \begin{equation}
    s'.ex(a)=\sum\limits_{p\in{}marked(s.M)}\mathbb{A}(p,a)\label{eq:fe-eq-exa}
  \end{equation}
  
  By construction, the generated \texttt{action} process is a part of
  design $d$'s behavior, i.e there exist an
  $sl\subseteq{}Sigs(\Delta)$ and an $ss_a\in{}ss$ s.t.
  $\mathtt{ps}("action", \emptyset, sl, ss)\in{}d.cs$.
  
  By construction $id_a$ is only assigned in the body of the
  \texttt{action} process during the initialization or a falling edge
  phase.

  Let $pls(a)$ be the set of actions associated to action $a$, i.e
  $pls(a)=\{p\in{}P~\vert~\mathbb{A}(p,a)=true\}$. Then, depending on
  $pls(a)$, there are two cases of assignment of output port $id_a$:
  
  \begin{itemize}
  \item \textbf{CASE} $pls(a)=\emptyset$:\\
    \noindent{}By construction,
    $\mathtt{id_a\Leftarrow{}false}\in{}ss_{a\downarrow}$ where
    $ss_{a\downarrow}$ is the part of the \texttt{``action''} process
    body executed during a falling edge phase.

    \noindent{}By property of the \hvhdl{} falling edge relation, the
    stabilize relation and
    $\mathtt{ps}("action", \emptyset, sl, ss_a)\in{}d.cs$, we can
    deduce $\sigma'(id_a)=false$.
    
    \noindent{}By property of
    $\sum\limits_{p\in{}marked(s.M)}\mathbb{A}(p,a)$ and
    $pls(a)=\emptyset$, we can deduce
    $\sum\limits_{p\in{}marked(s.M)}\mathbb{A}(p,a)=\mathtt{false}$.

    \noindent{}Rewriting the goal with \eqref{eq:fe-eq-exa},
    $\sigma'(id_a)=false$ and
    $\sum\limits_{p\in{}marked(s.M)}\mathbb{A}(p,a)=\mathtt{false}$,
    \qedbox{tautology.}
    
  \item \textbf{CASE} $pls(a)\neq\emptyset$:\\
    \noindent{}By construction,
    $\mathtt{id_a\Leftarrow{}id_{mp_0}+\dots+id_{mp_n}}\in{}ss_{a\downarrow}$,
    where $id_{mp_i}\in{}Sigs(\Delta)$, $ss_{a\downarrow}$ is the part
    of the \texttt{action} process body executed during the falling
    edge phase, and $n=\vert{}pls(a)\vert-1$.

    \noindent{}By property of the $\mathtt{Inject}_\downarrow$, the
    \hvhdl{} falling edge relation, the stabilize relation, and\\
    $\mathtt{ps}("action",$ $\emptyset,$ $sl,$ $ss)\in{}d.cs$:
    \begin{equation}
      \sigma'(id_a)=\sigma(id_{mp_0})+\dots+\sigma(id_{mp_n})\label{eq:fe-eq-ida-sum}
    \end{equation}

    Rewriting the goal with \eqref{eq:fe-eq-exa} and
    \eqref{eq:fe-eq-ida-sum}:
    \fbox{$\sum\limits_{p\in{}marked(s.M)}\mathbb{A}(p,a)=\sigma(id_{mp_0})+\dots+\sigma(id_{mp_n})$.}

    Let us reason on the value of $\sigma(id_{mp_0})+\dots+\sigma(id_{mp_n})$; there are two cases:

    \begin{itemize}
    \item \textbf{CASE} $\sigma(id_{mp_0})+\dots+\sigma(id_{mp_n})=\mathtt{true}$:\\
      \noindent{}Then, we can rewrite the goal as follows:
      \fbox{$\sum\limits_{p\in{}marked(s.M)}\mathbb{A}(p,a)=\mathtt{true}$.}

      \noindent{}To prove the above goal, let us show
      \fbox{$\exists{}p\in{}marked(s.M)~s.t.~\mathbb{A}(p,a)=\mathtt{true}$.}

      \noindent{}From
      $\sigma(id_{mp_0})+\dots+\sigma(id_{mp_n})=\mathtt{true}$, we
      can deduce that
      $\exists{}id_{mp_i}~s.t.~\sigma(id_{mp_i})=\mathtt{true}$. Let
      us take an $id_{mp_i}$ s.t. $\sigma(id_{mp_i})=\mathtt{true}$.

      \noindent{}By construction, there exist a $p\in{}pls(a)$, an
      $id_{p}\in{}Comps(\Delta)$, $gm_{p}$, $ipm_{p}$ and
      $opm_{p}$ such that:
      \begin{itemize}
      \item $\gamma(p)=id_{p}$
      \item \InCsCompP
      \item ${<}\mathtt{marked\Rightarrow{}id_{mp_i}}{>}\in{}opm_{p}$
      \end{itemize}
      Let us take such a $p$, $id_{p}$, $gm_{p}$, $ipm_{p}$ and
      $opm_{p}$.

      By property of stable $\sigma$ and \InCsCompP, we can deduce
      $\sigma(id_{mp_i})=\sigma(id_{p})("marked")$.

      By property of stable $\sigma$, \InCsCompP, and through the
      examination of the \texttt{determine_marked} process defined in
      the place design architecture, we can deduce:
      \begin{equation}
        \sigma(id_{p})("marked")=\sigma(id_{p})("sm")>0\label{eq:fe-gt-sm-zero}
      \end{equation}

      \noindent{}From $\sigma(id_{mp_i})=\sigma(id_{p})("marked")$,
      \eqref{eq:fe-gt-sm-zero} and $\sigma(id_{mp_i})=\mathtt{true}$,
      we can deduce that $\sigma(id_{p})("mar\-ked")=\mathtt{true}$
      and $(\sigma(id_{p})("sm")>0)=\mathtt{true}$.
      
      \noindent{}By property of \upSim, we have
      $s.M(p)=\sigma(id_{p})("sm")$.
      
      \noindent{}From $s.M(p)=\sigma(id_{p})("sm")$ and
      $(\sigma(id_{p})("sm")>0)=\mathtt{true}$, we can deduce
      $p\in{}marked(s.M)$, i.e $s.M(p)>0$.

      Let us use $p$ to prove the goal:
      \fbox{$\mathbb{A}(p,a)=\mathtt{true}$.}

      By definition of $p\in{}pls(a)$,
      \qedbox{$\mathbb{A}(p,a)=\mathtt{true}$.}

    \item \textbf{CASE} $\sigma(id_{mp_0})+\dots+\sigma(id_{mp_n})=\mathtt{false}$:\\
      
      Then, we can rewrite the goal as follows:
      \fbox{$\sum\limits_{p\in{}marked(s.M)}\mathbb{A}(p,a)=\mathtt{false}$.}

      \noindent{}To prove the above goal, let us show
      \fbox{$\forall{}p\in{}marked(s.M)~s.t.~\mathbb{A}(p,a)=\mathtt{false}$.}

      \noindent{}Given a $p\in{}marked(s.M)$, let us show
      \fbox{$\mathbb{A}(p,a)=\mathtt{false}$.}

      \noindent{}Let us perform case analysis on $\mathbb{A}(p,a)$;
      there are 2 cases:

      \begin{itemize}
      \item \textbf{CASE} \qedbox{$\mathbb{A}(p,a)=\mathtt{false}$.}
      \item \textbf{CASE} $\mathbb{A}(p,a)=\mathtt{true}$:\\
        By construction, there exist an $id_{p}\in{}Comps(\Delta)$,
        $gm_{tp}$, $ipm_{p}$, $opm_{p}$ and
        $id_{mp_i}\in{}Sigs(\Delta)$ such that:
        \begin{itemize}
        \item $\gamma(p)=id_{p}$
        \item \InCsCompP
        \item ${<}\mathtt{marked\Rightarrow{}id_{mp_i}}{>}\in{}opm_{p}$
        \end{itemize}

        Let us take such a $id_{p}$, $gm_{p}$, $ipm_{p}$, $opm_{p}$
        and $id_{mp_i}$.

        By property of stable $\sigma$, \InCsCompP, and
        ${<}\mathtt{marked\Rightarrow{}id_{mp_i}}{>}\in{}opm_{p}$, we
        can deduce $\sigma(id_{mp_i})=\sigma(id_{p})("marked")$.

        By property of stable $\sigma$, \InCsCompP, and through the
        examination of the \texttt{determine_marked} process defined
        in the place design architecture, we can deduce:
        \begin{equation}
          \sigma(id_{p})("marked")=(\sigma(id_{p})("sm")>0)\label{eq:fe-eq-marked-gt-sm-zero}
        \end{equation}

        From
        $\sigma(id_{mp_0})+\dots+\sigma(id_{mp_n})=\mathtt{false}$, we
        can deduce $\sigma(id_{mp_i})=\mathtt{false}$.

        From $\sigma(id_{p})("marked")=\mathtt{false}$, we can deduce
        $(\sigma(id_{p})("sm")>0)=\mathtt{false}$.

        By definition of \upSim, we have $s.M(p)=\sigma(id_p)("sm")$,
        and thus, we can deduce that $s.M(p)=0$ (equivalent to
        $(s.M(p)>0)=\mathtt{false}$).

        \noindent{}Contradicts \qedbox{$p\in{}marked(s.M)$} (i.e,
        $s.M(p)>0$).
      \end{itemize}
    \end{itemize}
  \end{itemize}
\end{niproof}

\subsection{Falling edge and function executions}
\label{sec:fe-fun-exec}

\begin{lemma}[Falling edge equal function executions]
  \label{lem:fe-equal-fun-exec}
  \fehyps{} then
  $\forall{}f\in\mathcal{F},id_f\in{}Outs(\Delta)~s.t.~\gamma(f)=id_f,~s'.ex(f)=\sigma'(id_f)$.
\end{lemma}

\begin{niproof}
  Given an $f\in\mathcal{F}$ and an
  $id_f\in{}Outs(\Delta)~s.t.~\gamma(f)=id_f$, let us show
  \fbox{$s'.ex(f)=\sigma'(id_f)$.}\\

  \noindent{}By property of \dwSitpn, we can deduce
  $s.ex(f)=s'.ex(f)$.
  
  \noindent{}By construction, $id_f$ is an output port identifier of
  boolean type in the \hvhdl{} design $d$ assigned by the
  \texttt{function} process only during the initialization or during a
  rising edge phase.

  \noindent{}By property of the \hvhdl{} $\mathtt{Inject_{\uparrow}}$,
  rising edge, stabilize relations, and the \texttt{function} process,
  we can deduce $\sigma(id_f)=\sigma'(id_f)$.

  \noindent{}Rewriting the goal with $s.ex(f)=s'.ex(f)$ and
  $\sigma(id_f)=\sigma'(id_f)$, \fbox{$s.ex(f)=\sigma(id_f)$.}

  \noindent{}By definition of \upSim, \qedbox{$s.ex(f)=\sigma(id_f)$.}
\end{niproof}

\subsection{Falling edge and firable transitions}
\label{sec:fe-firable}

%%%%%%%%%%%%%%%%%%%%%%%%%%%%%%%%%%%%%%%%%%%%%%%%
%%%%%%%%%% FALLING EDGE EQUAL FIRABLE %%%%%%%%%%
%%%%%%%%%%%%%%%%%%%%%%%%%%%%%%%%%%%%%%%%%%%%%%%%

\begin{lemma}[Falling edge equal firable]
  \label{lem:fe-equal-firable}
  \fehyps{} then
  $\forall{}t\in{}T,id_t\in{}Comps(\Delta)~s.t.~\gamma(t)=id_t,$
  $t\in{}Firable(s')\Leftrightarrow\sigma'(id_t)("s\_firable")=\mathtt{true}$.
\end{lemma}

\begin{niproof}
  Given a $t\in{}T$ and $id_t\in{}Comps(\Delta)$
  s.t. $\gamma(t)=id_t$, let us show that\\
  \fbox{$t\in{}Firable(s')\Leftrightarrow\sigma'(id_t)("s\_firable")=\mathtt{true}$.}\\

  The proof is in two parts:
  \begin{enumerate}
  \item Assuming that $t\in{}Firable(s')$, let us show
    \fbox{$\sigma'(id_t)("s\_firable")=\mathtt{true}$.}
    
    Appealing to Lemma~\ref{lem:fe-equal-firable-1}:
    \qedbox{$\sigma'(id_t)("s\_firable")=\mathtt{true}$.}
    
  \item Assuming that $\sigma'(id_t)("s\_firable")=\mathtt{true}$, let
    us show \fbox{$t\in{}Firable(s')$.}
    
    Appealing to Lemma~\ref{lem:fe-equal-firable-2}:
    \qedbox{$t\in{}Firable(s')$.}
  \end{enumerate}
  
\end{niproof}

\begin{lemma}[Falling edge equal firable 1]
  \label{lem:fe-equal-firable-1}
  \fehyps{} then
  $\forall{}t\in{}T,id_t\in{}Comps(\Delta)~s.t.~\gamma(t)=id_t,$
  $t\in{}Firable(s')\Rightarrow\sigma'(id_t)("s\_firable")=\mathtt{true}$.
\end{lemma}

\begin{niproof}
  Given a $t\in{}T$ and $id_t\in{}Comps(\Delta)$
  s.t. $\gamma(t)=id_t$, and assuming that $t\in{}Firable(s')$, let us
  show \fbox{$\sigma'(id_t)("s\_firable")=\mathtt{true}$.}\\
  
  \exT{}

  By property of the $\mathtt{Inject}_\downarrow$ relation, the
  \hvhdl{} falling edge relation, the stabilize relation, \InCsCompT,
  and through the examination of the \texttt{firable} process defined
  in the transition design architecture, we can deduce:
  \begin{equation}
    \label{eq:fe-eq-sfa}
    \sigma'(id_t)("sfa")=\sigma(id_t)("se")~.~\sigma(id_t)("scc")~.~\mathtt{checktc}(\Delta(id_t),\sigma(id_t))
  \end{equation}

  Term $\mathtt{checktc}(\Delta(id_t),\sigma(id_t))$ is defined as
  follows:
  \begin{equation}
    \label{eq:checktc}
    \begin{split}
      \mathtt{checktc}(\Delta(id_t),\sigma(id_t))=&\bigg(\mathtt{not}~\sigma(id_t)("srtc")~.~\\
      & \quad\begin{split}
        \big[\big(\Delta(id_t)("tt")=\mathtt{TEMP\_A\_B}~&.~(\sigma(id_t)("stc")\ge{}\sigma(id_t)("A")-1)~\\
        &.~(\sigma(id_t)("stc")\le{}\sigma(id_t)("B")-1)\big)\\
      \end{split} \\
      & \quad+(\Delta(id_t)("tt")=\mathtt{TEMP\_A\_A}~.~(\sigma(id_t)("stc")={}\sigma(id_t)("A")-1))\\
      &
      \quad+(\Delta(id_t)("tt")=\mathtt{TEMP\_A\_INF}~.~(\sigma(id_t)("stc")\ge{}\sigma(id_t)("A")-1))\big]\bigg) \\
      & +\big(\sigma(id_t)("srtc")~.~\Delta(id_t)("tt")\neq\mathtt{NOT\_TEMP}~.~\sigma(id_t)("A")=1\big)\\
      & +\Delta(id_t)("tt")=\mathtt{NOT\_TEMP}\\
    \end{split}
  \end{equation}

  Rewriting the goal with \eqref{eq:fe-eq-sfa}:
  \fbox{$\sigma(id_t)("se")~.~\sigma(id_t)("scc")~.~\mathtt{checktc}(\Delta(id_t),\sigma(id_t))=\mathtt{true}$.}

  Then, there are three points to prove:
  \begin{enumerate}
  \item \fbox{$\sigma(id_t)("se")=\mathtt{true}$}:\\

    From $t\in{}Firable(s')$, we can deduce
    $t\in{}Sens(s'.M)$. By definition of \dwSitpn, we have $s.M=s'.M$,
    and thus, we can deduce $t\in{}Sens(s.M)$.

    By definition of \upSim, we know that $t\in{}Sens(s.M)$
    implies \qedbox{$\sigma(id_t)("se")=\mathtt{true}$.}
    
  \item \fbox{$\sigma(id_t)("scc")=\mathtt{true}$}:\\

    By definition of \upSim:
    \begin{equation}
      \sigma(id_t)("scc")=\prod\limits_{c\in{}conds(t)}
      \begin{cases}
        E_c(\tau,c) & if~\mathbb{C}(t,c)=1 \\
        \mathtt{not}(E_c(\tau,c)) & if~\mathbb{C}(t,c)=-1 \\
      \end{cases}
      \label{eq:fe-eq-scc-prod}
    \end{equation}
    where
    $conds(t)=\{c\in\mathcal{C}~\vert~\mathbb{C}(t,c)=1\lor\mathbb{C}(t,c)=-1\}$.

    Rewriting the goal with \eqref{eq:fe-eq-scc-prod}:
    \fbox{$\prod\limits_{c\in{}conds(t)}
      \begin{cases}
        E_c(\tau,c) & if~\mathbb{C}(t,c)=1 \\
        \mathtt{not}(E_c(\tau,c)) & if~\mathbb{C}(t,c)=-1 \\
      \end{cases}=\mathtt{true}$.}

    To ease the reading, let us define $f(c)=\begin{cases}
          E_c(\tau,c) & if~\mathbb{C}(t,c)=1 \\
          \mathtt{not}(E_c(\tau,c)) & if~\mathbb{C}(t,c)=-1 \\
        \end{cases}$.
    
    Let us reason by induction on the left term of the
    goal:

    \begin{itemize}
    \item \textbf{BASE CASE}: \qedbox{$\mathtt{true}=\mathtt{true}$.}
    \item \textbf{INDUCTION CASE}:
      \begin{ih}
        $\prod\limits_{c'\in{}conds(t)\setminus\{c\}}f(c')=\mathtt{true}$
      \end{ih}

      \fbox{$f(c)~.~\prod\limits_{c'\in{}conds(t)\setminus\{c\}}f(c')=\mathtt{true}$.}

      Rewriting the goal with the induction hypothesis, simplifying
      the goal, and unfolding the definition of $f(c)$:
      \fbox{$\begin{cases}
          E_c(\tau,c) & if~\mathbb{C}(t,c)=1 \\
          \mathtt{not}(E_c(\tau,c)) & if~\mathbb{C}(t,c)=-1 \\
        \end{cases}=\mathtt{true}$.}

      As $c\in{}conds(t)$, let us perform case analysis on
      $\mathbb{C}(t,c)=1\lor\mathbb{C}(t,c)=-1$:
      \begin{enumerate}
      \item \textbf{CASE} $\mathbb{C}(t,c)=1$: \fbox{$E_c(\tau,c)=\mathtt{true}$.}\\
        
        By definition of $t\in{}Firable(s')$, we can deduce that
        $s'.cond(c)=\mathtt{true}$. By definition of \dwSitpn{}
        (Rule~\ref{it:cond-env}), we have
        $s'.cond(c)=E_c(\tau,c)$. Thus,
        \qedbox{$E_c(\tau,c)=\mathtt{true}$.}
        
      \item $\mathbb{C}(t,c)=-1$: \fbox{$\mathtt{not}~E_c(\tau,c)=\mathtt{true}$.}\\

        By definition of $t\in{}Firable(s')$, we can deduce that
        $s'.cond(c)=\mathtt{false}$. By definition of \dwSitpn{}
        (Rule~\ref{it:cond-env}), we have
        $s'.cond(c)=E_c(\tau,c)$. Thus,
        \qedbox{$\mathtt{not}~E_c(\tau,c)=\mathtt{true}$.}
      \end{enumerate}
    \end{itemize}

  \item \fbox{$\mathtt{checktc}(\Delta(id_t),\sigma(id_t))=\mathtt{true}$}:\\

    By definition of $t\in{}Firable(s')$, we have
    $t\notin{}T_i\lor{}s'.I(t)\in{}I_s(t)$. Let us perform case
    analysis on $t\notin{}T_i\lor{}s'.I(t)\in{}I_s(t)$:\\

    \begin{enumerate}
    \item \textbf{CASE} $t\notin{}T_i$: \fbox{$\mathtt{checktc}(\Delta(id_t),\sigma(id_t))=\mathtt{true}$}\\

      By construction,
      ${<}\mathtt{transition\_type\Rightarrow{}NOT\_TEMP}{>}\in{}gm_t$,
      and by property of the elaboration relation, we have
      $\Delta(id_t)("tt")=\mathtt{NOT\_TEMP}$.

      From $\Delta(id_t)("tt")=\mathtt{NOT\_TEMP}$, and by definition
      of $\mathtt{checktc}(\Delta(id_t),\sigma(id_t))$, we can deduce
      \qedbox{$\mathtt{checktc}(\Delta(id_t),\sigma(id_t))=\mathtt{true}$.}

    \item \textbf{CASE} $s'.I(t)\in{}I_s(t)$: \fbox{$\mathtt{checktc}(\Delta(id_t),\sigma(id_t))=\mathtt{true}$}\\

      From $s'.I(t)\in{}I_s(t)$, we can deduce that $t\in{}T_i$. Thus,
      by construction, there exists
      $tt\in\{\mathtt{TEMP\_A\_B},\mathtt{TEMP\_A\_A},\mathtt{TEMP\_A\_INF}\}$
      s.t. ${<}\mathtt{transition\_type\Rightarrow}tt{>}\in{}gm_t$. By
      property of the elaboration relation, we have
      $\Delta(id_t)("tt")=tt$, and thus, we know
      $\Delta(id_t)("tt")\neq{}\mathtt{NOT\_TEMP}$. Therefore, we can
      simplfy the term $\mathtt{checktc}(\Delta(id_t),\sigma(id_t))$
      as follows:
      \begin{equation}        
        \label{eq:eq-checktc-minus-not-temp}
        \begin{split}
          \mathtt{checktc}(\Delta(id_t),\sigma(id_t))=&\bigg(\mathtt{not}~\sigma(id_t)("srtc")~.~\\
          & \quad\begin{split}
            \big[\big(\Delta(id_t)("tt")=\mathtt{TEMP\_A\_B}~&.~(\sigma(id_t)("stc")\ge{}\sigma(id_t)("A")-1)~\\
            &.~(\sigma(id_t)("stc")\le{}\sigma(id_t)("B")-1)\big)\\
          \end{split} \\
          & \quad{\begin{array}{l}
                    +(\Delta(id_t)("tt")=\mathtt{TEMP\_A\_A}~.\\
                    \quad~(\sigma(id_t)("stc")={}\sigma(id_t)("A")-1)) \\
                  \end{array}}\\
          &
          \quad{\begin{array}{l}
                  +(\Delta(id_t)("tt")=\mathtt{TEMP\_A\_INF}~. \\
                  \quad~(\sigma(id_t)("stc")\ge{}\sigma(id_t)("A")-1))\big]\bigg) \\
                \end{array}}\\
          & +\big(\sigma(id_t)("srtc")~.~\sigma(id_t)("A")=1\big)\\
        \end{split}
      \end{equation}
      
      By definition of \upSim, we have
      $s.reset_t(t)=\sigma(id_t)("srtc")$.

      Let us perform case analysis on the value
      $s.reset_t(t)$:

      \begin{enumerate}
      \item \textbf{CASE} $s.reset_t(t)=\mathtt{true}$: \fbox{$\mathtt{checktc}(\Delta(id_t),\sigma(id_t))=\mathtt{true}$}\\

        From $s.reset_t(t)=\sigma(id_t)("srtc")$, we can deduce that
        $\sigma(id_t)("srtc")=\mathtt{true}$.

        From $\sigma(id_t)("srtc")=\mathtt{true}$, we can
        simplify the term
        $\mathtt{checktc}(\Delta(id_t),\sigma(id_t))$ as follows:
        \begin{equation}
          \label{eq:eq-checktc-a-eq-1}
            \mathtt{checktc}(\Delta(id_t),\sigma(id_t))=\big(\sigma(id_t)("A")=1\big)
        \end{equation}

        Rewriting the goal with \eqref{eq:eq-checktc-a-eq-1}, and
        simplifying the goal: \fbox{$\sigma(id_t)("A")=1$.}

        By definition of \dwSitpn{} (Rule~\ref{it:reset-counters}), from
        $t\in{}Sens(s.M)$ and $s.reset_t(t)=\mathtt{true}$, we can
        deduce $s'.I(t)=1$. We know that $s'.I(t)\in{}I_s(t)$, and
        thus, we have $1\in{}I_s(t)$.

        By definition of $1\in{}I_s(t)$, there exist an
        $a\in\mathbb{N}^{*}$ and a
        $ni\in{}\mathbb{N}^{*}\sqcup\{\infty\}$ s.t. $I_s(t)=[a,ni]$
        and $1\in[a,ni]$.

        By definition of $1\in[a,ni]$, we have $a\le{}1$, and
        since $a\in\mathbb{N}^{*}$, we can deduce $a=1$.

        By construction,
        ${<}\mathtt{time\_A\_value\Rightarrow}{}a{>}\in{}ipm_t$, and
        by property of stable $\sigma$, we have
        \qedbox{$\sigma(id_t)("A")=a=1$.}

      \item \textbf{CASE} $s.reset_t(t)=\mathtt{false}$: \fbox{$\mathtt{checktc}(\Delta(id_t),\sigma(id_t))=\mathtt{true}$}\\

        From $s.reset_t(t)=\sigma(id_t)("srtc")$, we can deduce
        $\sigma(id_t)("srtc")=\mathtt{false}$.

        From $\sigma(id_t)("srtc")=\mathtt{false}$, we can simplify
        the term $\mathtt{checktc}(\Delta(id_t),\sigma(id_t))$ as
        follows:
        \begin{equation}
          \label{eq:eq-checktc-srtc-false}
          \begin{array}{l}
            \multicolumn{1}{c}{\mathtt{checktc}(\Delta(id_t),\sigma(id_t))}\\
            \multicolumn{1}{c}{=}\\
            \begin{array}{ll}
              \big(\Delta(id_t)("tt")=\mathtt{TEMP\_A\_B}~&.~(\sigma(id_t)("stc")\ge{}\sigma(id_t)("A")-1)~\\
                                                          &.~(\sigma(id_t)("stc")\le{}\sigma(id_t)("B")-1)\big)\\
            \end{array}\\
            +(\Delta(id_t)("tt")=\mathtt{TEMP\_A\_A}~.~(\sigma(id_t)("stc")={}\sigma(id_t)("A")-1))\\
            +(\Delta(id_t)("tt")=\mathtt{TEMP\_A\_INF}~.~(\sigma(id_t)("stc")\ge{}\sigma(id_t)("A")-1)) \\
          \end{array}
        \end{equation}

        Let us perform case analysis on $I_s(t)$; there are two cases:
        \begin{itemize}
        \item \textbf{CASE} $I_s(t)=[a,b]$ where
          $a,b\in\mathbb{N}^{*}$; then, either $a=b$ or $a\neq{}b$:
          
          \begin{itemize}
          \item \textbf{CASE} $a=b$:\\
            Then, we have $I_s(t)=[a,a]$, and by construction
            ${<}$\texttt{transition\_type}$\Rightarrow{}$
            \texttt{TEMP\_A\_A}${>}\in{}gm_t$. By property of the
            elaboration relation, we have
            $\Delta(id_t)$$("tt")=$\texttt{TEMP\_A\_A}; thus we can
            simplify the $\mathtt{checktc}$ term as follows:
            \begin{equation}
              \label{eq:eq-checktc-temp-a-a}
              \mathtt{checktc}(\Delta(id_t),\sigma(id_t))=(\sigma(id_t)("stc")={}\sigma(id_t)("A")-1)
            \end{equation}

            Rewriting the goal with \eqref{eq:eq-checktc-temp-a-a},
            and simplifying the goal:\\
            \fbox{$\sigma(id_t)("stc")={}\sigma(id_t)("A")-1$.}\\
            
            From $s'.I(t)\in[a,a]$, we can deduce that
            $s'.I(t)=a$. Let us perform case analysis on
            $s.I(t)<upper(I_s(t))$ or $s.I(t)\ge{}upper(I_s(t))$:
            
            \begin{itemize}
            \item \textbf{CASE} $s.I(t)<upper(I_s(t))$:\\
              By definition of \upSim, we have
              $s.I(t)=\sigma(id_t)("stc")$. By definition of
              \dwSitpn{} (Rule~\ref{it:inc-counters}), we have
              $s'.I(t)=s.I(t)+1$. From $s'.I(t)=a$ and
              $s'.I(t)=s.I(t)+1$, we can deduce $a-1=s.I(t)$.

              By construction,
              ${<}\mathtt{time\_A\_value\Rightarrow}{}a{>}\in{}ipm_t$,
              and by property of stable $\sigma$, we have
              $\sigma(id_t)("A")=a$.

              Rewriting the goal with $\sigma(id_t)("A")=a$,
              $s.I(t)=\sigma(id_t)("stc")$, and $a-1=s.I(t)$:
              \qedbox{tautology.}
              
            \item \textbf{CASE} $s.I(t)\ge{}upper(I_s(t))$:\\
              In the case where $s.I(t)>upper(I_s(t))$, then
              $s.I(t)>a$.  By definition of \dwSitpn{}
              (Rule~\ref{it:locked-counters}), we have
              $s.I(t)=s'.I(t)=a$. Then, \qedbox{$a>a$ is a
                contradiction.}\\

              In the case where $s.I(t)=upper(I_s(t))$, then
              $s.I(t)=a$. By definition of \dwSitpn{}
              (Rule~\ref{it:inc-counters}), we have
              $s'.I(t)=s.I(t)+1$. Then, we have $s'.I(t)=a$ and
              $s'.I(t)=a+1$. Then, \qedbox{$a=a+1$ is a
                contradiction.}
            \end{itemize}
            
          \item \textbf{CASE} $a\neq{}b$: \fbox{$\mathtt{checktc}(\Delta(id_t),\sigma(id_t))=\mathtt{true}$}\\
            Then, we have $I_s(t)=[a,b]$, and by construction
            ${<}$\texttt{transition\_type}$\Rightarrow{}$
            \texttt{TEMP\_A\_B}${>}\in{}gm_t$. By property of the
            elaboration relation, we have
            $\Delta(id_t)$$("tt")=$\texttt{TEMP\_A\_B}; thus we can
            simplify the term $\mathtt{checktc}$ as follows:
            \begin{equation}
              \label{eq:eq-checktc-temp-a-b}
              \begin{array}{c}
                \multicolumn{1}{c}{\mathtt{checktc}(\Delta(id_t),\sigma(id_t))}\\
                \multicolumn{1}{c}{=}\\
                (\sigma(id_t)("stc")\ge{}\sigma(id_t)("A")-1)~.~(\sigma(id_t)("stc")\le{}\sigma(id_t)("B")-1)\\
              \end{array}\\
            \end{equation}

            Rewriting the goal with \eqref{eq:eq-checktc-temp-a-b},
            and simplifying the goal:\\
            \fbox{$(\sigma(id_t)("stc")\ge{}\sigma(id_t)("A")-1)\land(\sigma(id_t)("stc")\le{}\sigma(id_t)("B")-1)$.}\\
            
            Let us perform case analysis on $s.I(t)<upper(I_s(t))$ or
            $s.I(t)\ge{}upper(I_s(t))$:
            \begin{itemize}
            \item \textbf{CASE} $s.I(t)<upper(I_s(t))$:\\
              By definition of \upSim, we have
              $s.I(t)=\sigma(id_t)("stc")$. By definition of \dwSitpn
              (Rule~\ref{it:inc-counters}), we have
              $s'.I(t)=s.I(t)+1$. By definition of $s'.I(t)\in[a,b]$:
              
              $\Rightarrow{}a\le{}s'.I(t)\le{}b$.
              
              $\Rightarrow{}a\le{}s'.I(t)\land{}s'.I(t)\le{}b$

              $\Rightarrow{}a\le{}s.I(t)+1\land{}s.I(t)+1\le{}b$

              $\Rightarrow{}a-1\le{}s.I(t)\land{}s.I(t)\le{}b-1$

              By construction,
              ${<}\mathtt{time\_A\_value\Rightarrow}{}a{>}\in{}ipm_t$
              and
              ${<}\mathtt{time\_B\_value\Rightarrow}{}b{>}\in{}ipm_t$,
              and by property of stable $\sigma$, we have
              $\sigma(id_t)("A")=a$ and
              $\sigma(id_t)("B")=b$.

              Rewriting the goal with
              $\sigma(id_t)("A")=a$, $\sigma(id_t)("B")=b$ and
              $s.I(t)=\sigma(id_t)("stc")$:
              \qedbox{$a-1\le{}s.I(t)\land{}s.I(t)\le{}b-1$.}
              
            \item \textbf{CASE} $s.I(t)\ge{}upper(I_s(t))$:\\
              
              In the case where $s.I(t)>upper(I_s(t))$, then
              $s.I(t)>b$. By definition of \dwSitpn{}
              (Rule~\ref{it:locked-counters}), we have
              $s.I(t)=s'.I(t)=b$. Then, \qedbox{$b>b$ is a
                contradiction.}\\

              In the case where $s.I(t)=upper(I_s(t))$, then
              $s.I(t)=b$. By definition of \dwSitpn{}
              (Rule~\ref{it:inc-counters}), we have
              $s'.I(t)=s.I(t)+1$.

              By definition of $s'.I(t)\in[a,b]$, we have
              $s'.I(t)\le{}b$:

              $\Rightarrow{}s.I(t)+1\le{}b$

              $\Rightarrow{}$ \qedbox{$b+1\le{}b$ is contradiction.}
            \end{itemize}
          \end{itemize}
          
        \item \textbf{CASE} $I_s(t)=[a,\infty]$ where
          $a\in{}\mathbb{N}^{*}$: \fbox{$\mathtt{checktc}(\Delta(id_t),\sigma(id_t))=\mathtt{true}$}
 
          By construction ${<}\mathtt{transition\_type\Rightarrow{}}$
          $\mathtt{TEMP\_A\_INF}{>}\in{}gm_t$. By property of the
          elaboration relation, we have
          $\Delta(id_t)$$("tt")=\mathtt{TEMP\_A\_INF}$; thus we can
          simplify the term $\mathtt{checktc}$ as follows:
          \begin{equation}
            \label{eq:eq-checktc-temp-a-inf}
            \mathtt{checktc}(\Delta(id_t),\sigma(id_t))=(\sigma(id_t)("stc")\ge{}\sigma(id_t)("A")-1))
          \end{equation}

          Rewriting the goal with \eqref{eq:eq-checktc-temp-a-inf},
          and simplifying the goal:\\
          \fbox{$\sigma(id_t)("stc")\ge{}\sigma(id_t)("A")-1$.}\\

          From $s'.I(t)\in[a,\infty]$, we can deduce
          $a\le{}s'.I(t)$. Then, let us perform case analysis on
          $s.I(t)\le{}lower(I_s(t))$ or $s.I(t)>lower(I_s(t))$:
          \begin{itemize}
          \item \textbf{CASE} $s.I(t)\le{}lower(I_s(t))$:

            By definition of \upSim, we have
            $s.I(t)=\sigma(id_t)("stc")$.

            By definition of \dwSitpn{} (Rule~\ref{it:inc-counters}),
            we have $s'.I(t)=s.I(t)+1$:

            $\Rightarrow{}s'.I(t)\ge{}a$

            $\Rightarrow{}s.I(t)+1\ge{}a$

            $\Rightarrow{}s.I(t)\ge{}a-1$

            By construction,
            ${<}\mathtt{time\_A\_value\Rightarrow}{}a{>}\in{}ipm_t$,
            and by property of stable $\sigma$, we have
            $\sigma(id_t)("A")=a$.

            Rewriting the goal with $\sigma(id_t)("A")=a$ and
            $s.I(t)=\sigma(id_t)("stc")$:\\ \qedbox{$s.I(t)\ge{}a-1$.}
            
          \item \textbf{CASE} $s.I(t)>lower(I_s(t))$:

            By definition of \upSim, we have
            $\sigma(id_t)("stc")=lower(I_s(t))=a$.

            By construction,
            ${<}\mathtt{time\_A\_value\Rightarrow}{}a{>}\in{}ipm_t$,
            and by property of stable $\sigma$, we have
            $\sigma(id_t)("A")=a$.

            Rewriting the goal with $\sigma(id_t)("stc")=a$ and
            $\sigma(id_t)("A")=a$: \qedbox{$a\ge{}a-1$.}
          \end{itemize}
          
        \end{itemize}
      \end{enumerate}
    \end{enumerate}
  \end{enumerate}
\end{niproof}

\begin{lemma}[Falling Edge Equal Firable 2]
  \label{lem:fe-equal-firable-2}
  \fehyps{} then
  $\forall{}t\in{}T,id_t\in{}Comps(\Delta)~s.t.~\gamma(t)=id_t,$
  $\sigma'(id_t)("s\_firable")=\mathtt{true}\Rightarrow{}t\in{}Firable(s')$.
\end{lemma}

\begin{niproof}

  Given a $t\in{}T$ and $id_t\in{}Comps(\Delta)$
  s.t. $\gamma(t)=id_t$, and assuming that
  $\sigma'(id_t)("s\_firable")=\mathtt{true}$, let us
  show \fbox{$t\in{}Firable(s')$.}\\
  
  \exT{}

  By property of the $\mathtt{Inject}_\downarrow$ relation, the
  \hvhdl{} falling edge relation, the stabilize relation, \InCsCompT,
  and through the examination of the \texttt{firable} process defined
  in the transition design architecture, we can deduce:
  \begin{equation}
    \label{eq:fe-eq-sfa-true}
    \sigma'(id_t)("sfa")=\sigma(id_t)("se")~.~\sigma(id_t)("scc")~.~\mathtt{checktc}(\Delta(id_t),\sigma(id_t))=\mathtt{true}
  \end{equation}

  From \eqref{eq:fe-eq-sfa-true}, we can deduce:
  \begin{eqnarray}
    \label{eq:fe-eq-se-true}\sigma(id_t)("se")=\mathtt{true}\\
    \label{eq:fe-eq-scc-true}\sigma(id_t)("scc")=\mathtt{true}\\
    \label{eq:fe-eq-checktc-true}\mathtt{checktc}(\Delta(id_t),\sigma(id_t))=\mathtt{true}
  \end{eqnarray}

  Term $\mathtt{checktc}(\Delta(id_t),\sigma(id_t))$ as the same
  definition as in Lemma~\nameref{lem:fe-equal-firable-1}.

  By definition of $t\in{}Firable(s')$, there are three points to prove:
  \begin{enumerate}
  \item \fbox{$t\in{}Sens(s'.M)$}
  \item
    \fbox{\parbox{\lwidth}{$\forall{}c\in{}\mathcal{C},~\mathbb{C}(t,c)=1\Rightarrow{}s'.cond(c)=\mathtt{true}$
        and $\mathbb{C}(t,c)=-1\Rightarrow{}s'.cond(c)=\mathtt{false}$}}
  \item \fbox{$t\notin{}T_i\lor{}s'.I(t)\in{}I_s(t)$}
  \end{enumerate}

  Let us prove these three points:
  \begin{enumerate}
  \item \fbox{$t\in{}Sens(s'.M)$}:

    By definition of \dwSitpn, we have $s.M=s'.M$. Rewriting the goal
    with $s.M=s'.M$: \fbox{$t\in{}Sens(s.M)$.}

    By definition of \upSim, we have
    $\sigma(id_t)("se")=\mathtt{true}\Leftrightarrow{}t\in{}Sens(s.M)$.

    From $\sigma(id_t)("se")=\mathtt{true}$, we can deduce: \qedbox{$t\in{}Sens(s.M)$.}
    
  \item
    \fbox{\parbox{\lwidth}{$\forall{}c\in{}\mathcal{C},~\mathbb{C}(t,c)=1\Rightarrow{}s'.cond(c)=\mathtt{true}$
        and
        $\mathbb{C}(t,c)=-1\Rightarrow{}s'.cond(c)=\mathtt{false}$}}

    Given a $c\in\mathcal{C}$, there are two points to prove:
    \begin{enumerate}
    \item
      \fbox{$\mathbb{C}(t,c)=1\Rightarrow{}s'.cond(c)=\mathtt{true}$.}
    \item
      \fbox{$\mathbb{C}(t,c)=-1\Rightarrow{}s'.cond(c)=\mathtt{false}$.}
    \end{enumerate}

    Let us prove these two points:
    \begin{enumerate}
    \item Assuming that $\mathbb{C}(t,c)=1$, let us show
      \fbox{$s'.cond(c)=\mathtt{true}$.}

      By definition of \upSim, we have:
      \begin{equation}
        \sigma(id_t)("scc")=\prod\limits_{c'\in{}conds(t)}
        \begin{cases}
          E_c(\tau,c') & if~\mathbb{C}(t,c')=1 \\
          \mathtt{not}(E_c(\tau,c')) & if~\mathbb{C}(t,c')=-1 \\
        \end{cases}=\mathtt{true}
        \label{eq:fe-eq-scc-prod-true}
      \end{equation}
      where
      $conds(t)=\{c_i\in\mathcal{C}~\vert~\mathbb{C}(t,c_i)=1\lor\mathbb{C}(t,c_i)=-1\}$.

      From $\mathbb{C}(t,c)=1$, we can deduce $c\in{}conds(t)$. By
      definition of the product expression, we have:
      \begin{equation}
        \label{eq:fe-eq-scc-prod-true-decomp}
        E_c(\tau,c)~.~\prod\limits_{c'\in{}conds(t)\setminus\{c\}}
        \begin{cases}
          E_c(\tau,c') & if~\mathbb{C}(t,c')=1 \\
          \mathtt{not}(E_c(\tau,c')) & if~\mathbb{C}(t,c')=-1 \\
        \end{cases}=\mathtt{true}
      \end{equation}

      From \eqref{eq:fe-eq-scc-prod-true-decomp}, we can deduce that
      $E_c(\tau,c)=\mathtt{true}$.

      By definition of \dwSitpn{} (Rule~\ref{it:cond-env}), we have
      $s'.cond(c)=E_c(\tau,c)$.
      
      Rewriting the goal with $s'.cond(c)=E_c(\tau,c)$ and
      $E_c(\tau,c)=\mathtt{true}$: \qedbox{tautology.}
      
    \item Assuming that $\mathbb{C}(t,c)=-1$, let us show
      \fbox{$s'.cond(c)=\mathtt{false}$.}

      By definition of \upSim, we have:
      \begin{equation}
        \sigma(id_t)("scc")=\prod\limits_{c'\in{}conds(t)}
        \begin{cases}
          E_c(\tau,c') & if~\mathbb{C}(t,c')=1 \\
          \mathtt{not}(E_c(\tau,c')) & if~\mathbb{C}(t,c')=-1 \\
        \end{cases}=\mathtt{true}
        \label{eq:fe-eq-scc-prod-false}
      \end{equation}
      where
      $conds(t)=\{c'\in\mathcal{C}~\vert~\mathbb{C}(t,c')=1\lor\mathbb{C}(t,c')=-1\}$.

      From $\mathbb{C}(t,c)=-1$, we can deduce $c\in{}conds(t)$. By
      definition of the product expression, we have:
      \begin{equation}
        \label{eq:fe-eq-scc-prod-false-decomp}
        \mathtt{not}~E_c(\tau,c)~.~\prod\limits_{c'\in{}conds(t)\setminus\{c\}}
        \begin{cases}
          E_c(\tau,c') & if~\mathbb{C}(t,c')=1 \\
          \mathtt{not}(E_c(\tau,c')) & if~\mathbb{C}(t,c')=-1 \\
        \end{cases}=\mathtt{true}
      \end{equation}

      From \eqref{eq:fe-eq-scc-prod-false-decomp}, we can deduce that
      $E_c(\tau,c)=\mathtt{false}$.

      By definition of \dwSitpn{} (Rule~\ref{it:cond-env}), we have
      $s'.cond(c)=E_c(\tau,c)$.
      
      Rewriting the goal with $s'.cond(c)=E_c(\tau,c)$ and
      $E_c(\tau,c)=\mathtt{false}$: \qedbox{tautology.}
    \end{enumerate}
    
  \item \fbox{$t\notin{}T_i\lor{}s'.I(t)\in{}I_s(t)$}

    Reasoning on
    $\mathtt{checktc}(\Delta(id_t),\sigma(id_t))=\mathtt{true}$, there
    are 3 cases:
    \begin{enumerate}
    \item
      $\big(\mathtt{not}~\sigma(id_t)("srtc")~.~[\dots]\big)=\mathtt{true}$\footnote{See
        equation~\eqref{eq:checktc} for the full definition.}
    \item
      $\big(\sigma(id_t)("srtc")~.~\Delta(id_t)("tt")\neq\mathtt{NOT\_TEMP}~.~\sigma(id_t)("A")=1\big)=\mathtt{true}$
    \item
      $\big(\Delta(id_t)("tt")=\mathtt{NOT\_TEMP}\big)=\mathtt{true}$
    \end{enumerate}
    
    \begin{enumerate}
    \item \textbf{CASE}
      $\big(\mathtt{not}~\sigma(id_t)("srtc")~.~[\dots]\big)=\mathtt{true}$:

      Then, we can deduce
      $\mathtt{not}~\sigma(id_t)("srtc")=\mathtt{true}$ and
      $[\dots]=\mathtt{true}$. From
      $\mathtt{not}~\sigma(id_t)("srtc")=\mathtt{true}$, we can deduce
      $\sigma(id_t)("srtc")=\mathtt{false}$, and from
      $[\dots]=\mathtt{true}$, we have three other cases:

      \begin{enumerate}
      \item \textbf{CASE}
        $\big(\Delta(id_t)("tt")=\mathtt{TEMP\_A\_B}~.~(\sigma(id_t)("stc")\ge{}\sigma(id_t)("A")-1)~.~(\sigma(id_t)("stc")\le{}\sigma(id_t)("B")-1)\big)=\mathtt{true}$
      \item \textbf{CASE}
        $(\Delta(id_t)("tt")=\mathtt{TEMP\_A\_A}~.~(\sigma(id_t)("stc")={}\sigma(id_t)("A")-1))=\mathtt{true}$
      \item \textbf{CASE}
        $(\Delta(id_t)("tt")=\mathtt{TEMP\_A\_INF}~.~(\sigma(id_t)("stc")\ge{}\sigma(id_t)("A")-1))=\mathtt{true}$
      \end{enumerate}

      Let us prove the goal is these three contexts:
      \begin{enumerate}
      \item \textbf{CASE}
        $\big(\Delta(id_t)("tt")=\mathtt{TEMP\_A\_B}~.~(\sigma(id_t)("stc")\ge{}\sigma(id_t)("A")-1)~.~(\sigma(id_t)("stc")\le{}\sigma(id_t)("B")-1)\big)=\mathtt{true}$:

        Then, converting boolean equalities into intuitionistic
        predicates, we have:
        \begin{itemize}
        \item $\Delta(id_t)("tt")=\mathtt{TEMP\_A\_B}$
        \item $\sigma(id_t)("stc")\ge{}\sigma(id_t)("A")-1$
        \item $\sigma(id_t)("stc")\le{}\sigma(id_t)("B")-1$
        \end{itemize}

        By property of the elaboration relation, and
        $\Delta(id_t)("tt")=\mathtt{TEMP\_A\_B}$, there exist
        $a,b\in\mathbb{N}^{*}$ s.t. $I_s(t)=[a,b]$. Let us take such
        an $a$ and $b$. Then, let us show \fbox{$s'.I(t)\in{}I_s(t)$.}

        Rewriting the goal with $I_s(t)=[a,b]$:
        \fbox{$s'.I(t)\in[a,b]$.}
        
        By construction, ${<}\mathtt{time\_A\_value\Rightarrow}{}a{>}$
        and ${<}\mathtt{time\_B\_value\Rightarrow}{}b{>}$, and by
        property of stable $\sigma$, we have $\sigma(id_t)("A")=a$ and
        $\sigma(id_t)("B")=b$.

        Rewriting the goal with $\sigma(id_t)("A")=a$ and
        $\sigma(id_t)("B")=b$, and by definition of $\in$:
        \fbox{$\sigma(id_t)("A")\le{}s'.I(t)\le{}\sigma(id_t)("B")$.}

        Now, let us perform case analysis on
        $s.I(t)\le{}upper(I_s(t))$ or $s.I(t)>upper(I_s(t))$:
        \begin{itemize}
        \item \textbf{CASE} $s.I(t)\le{}upper(I_s(t))$:
          
          By definition of \upSim, we have $s.I(t)=\sigma(id_t)("stc")$.

          From $\sigma(id_t)("se")=\mathtt{true}$, we can deduce
          $t\in{}Sens(s.M)$, and from
          $\sigma(id_t)("srtc")=\mathtt{false}$, we can deduce
          $s.reset_t(t)=\mathtt{false}$. Then, by definition of
          \dwSitpn{} (Rule~\ref{it:inc-counters}), we have
          $s'.I(t)=s.I(t)+1$.

          $\Rightarrow{}$
          \fbox{$\sigma(id_t)("A")\le{}s.I(t)+1\le{}\sigma(id_t)("B")$}
          (by $s'.I(t)=s.I(t)+1$)

          $\Rightarrow{}$
          \fbox{$\sigma(id_t)("A")\le{}\sigma(id_t)("stc")+1\le{}\sigma(id_t)("B")$}
          (by $s.I(t)=\sigma(id_t)("stc")$)
          
          $\Rightarrow{}$
          \fbox{$\sigma(id_t)("A")-1\le{}\sigma(id_t)("stc")\le{}\sigma(id_t)("B")-1$}
          
          We assumed $\sigma(id_t)("stc")\ge{}\sigma(id_t)("A")-1$ and
          $\sigma(id_t)("stc")\le{}\sigma(id_t)("B")-1$, and thus we can deduce:
          \qedbox{$\sigma(id_t)("A")-1\le{}\sigma(id_t)("stc")\le{}\sigma(id_t)("B")-1$}
          
        \item \textbf{CASE} $s.I(t)>upper(I_s(t))$:

          By definition of \upSim, we have
          $\sigma(id_t)("stc")=upper(I_s(t))=b$.

          Then, from $\sigma(id_t)("stc")\le{}\sigma(id_t)("B")-1$,
          $\sigma(id_t)("stc")=upper(I_s(t))=b$ and
          $\sigma(id_t)("B")=b$, we can deduce the following
          contradiction:\\
          \qedbox{$\sigma(id_t)("B")\le{}\sigma(id_t)("B")-1$.}
        \end{itemize}
      \item
        $(\Delta(id_t)("tt")=\mathtt{TEMP\_A\_A}~.~(\sigma(id_t)("stc")={}\sigma(id_t)("A")-1))=\mathtt{true}$:

        Then, converting boolean equalities into intuitionistic
        predicates, we have:
        \begin{itemize}
        \item $\Delta(id_t)("tt")=\mathtt{TEMP\_A\_A}$
        \item $\sigma(id_t)("stc")=\sigma(id_t)("A")-1$
        \end{itemize}

        By property of the elaboration relation, and
        $\Delta(id_t)("tt")=\mathtt{TEMP\_A\_A}$, there exist
        $a\in\mathbb{N}^{*}$ s.t. $I_s(t)=[a,a]$. Let us take such an
        $a$. Then, let us show \fbox{$s'.I(t)\in{}I_s(t)$.}

        Rewriting the goal with $I_s(t)=[a,a]$:
        \fbox{$s'.I(t)\in[a,a]$.}
        
        By construction,
        ${<}\mathtt{time\_A\_value\Rightarrow}{}a{>}$, and by property
        of stable $\sigma$, we have $\sigma(id_t)("A")=a$.

        Rewriting the goal with $\sigma(id_t)("A")=a$, unfolding the
        definition of $\in$, and simplifying the goal:
        \fbox{$s'.I(t)=\sigma(id_t)("A")$.}

        Now, let us perform case analysis on
        $s.I(t)\le{}upper(I_s(t))$ or $s.I(t)>upper(I_s(t))$:
        \begin{itemize}
        \item \textbf{CASE} $s.I(t)\le{}upper(I_s(t))$:
          
          By definition of \upSim, we have $s.I(t)=\sigma(id_t)("stc")$.

          From $\sigma(id_t)("se")=\mathtt{true}$, we can deduce
          $t\in{}Sens(s.M)$, and from
          $\sigma(id_t)("srtc")=\mathtt{false}$, we can deduce
          $s.reset_t(t)=\mathtt{false}$. Then, by definition of
          \dwSitpn{} (Rule~\ref{it:inc-counters}), we have
          $s'.I(t)=s.I(t)+1$.

          $\Rightarrow{}$
          \fbox{$s.I(t)+1=\sigma(id_t)("A")$} (by $s'.I(t)=s.I(t)+1$)

          $\Rightarrow{}$
          \fbox{$\sigma(id_t)("stc")+1=\sigma(id_t)("A")$}
          (by $s.I(t)=\sigma(id_t)("stc")$)

          $\Rightarrow{}$
          \qedbox{$\sigma(id_t)("stc")=\sigma(id_t)("A")-1$} (assumption)
          
        \item \textbf{CASE} $s.I(t)>upper(I_s(t))$:

          By definition of \upSim, we have
          $\sigma(id_t)("stc")=upper(I_s(t))=a$.

          Then, from $\sigma(id_t)("stc")={}\sigma(id_t)("A")-1$,
          $\sigma(id_t)("stc")=upper(I_s(t))=a$,
          $\sigma(id_t)("A")=a$, and $a\in\mathbb{N}^{*}$, we can
          derive the following
          contradiction:\\
          \qedbox{$\sigma(id_t)("A")=\sigma(id_t)("A")-1$.}
        \end{itemize}
        
      \item
        $(\Delta(id_t)("tt")=\mathtt{TEMP\_A\_INF}~.~(\sigma(id_t)("stc")\ge\sigma(id_t)("A")-1))=\mathtt{true}$:

        Then, converting boolean equalities into intuitionistic
        predicates, we have:
        \begin{itemize}
        \item $\Delta(id_t)("tt")=\mathtt{TEMP\_A\_INF}$
        \item $\sigma(id_t)("stc")\ge\sigma(id_t)("A")-1$
        \end{itemize}

        By property of the elaboration relation, and
        $\Delta(id_t)("tt")=\mathtt{TEMP\_A\_INF}$, there exist
        $a\in\mathbb{N}^{*}$ s.t. $I_s(t)=[a,\infty]$. Let us take
        such an $a$. Then, let us show \fbox{$s'.I(t)\in{}I_s(t)$.}

        Rewriting the goal with $I_s(t)=[a,\infty]$:
        \fbox{$s'.I(t)\in[a,\infty]$.}
        
        By construction,
        ${<}\mathtt{time\_A\_value\Rightarrow}{}a{>}$, and by property
        of stable $\sigma$, we have $\sigma(id_t)("A")=a$.

        Rewriting the goal with $\sigma(id_t)("A")=a$, unfolding the
        definition of $\in$, and simplifying the goal:
        \fbox{$\sigma(id_t)("A")\le{}s'.I(t)$.}

        Now, let us perform case analysis on
        $s.I(t)\le{}lower(I_s(t))$ or $s.I(t)>lower(I_s(t))$:
        \begin{itemize}
        \item \textbf{CASE} $s.I(t)\le{}lower(I_s(t))$:
          
          By definition of \upSim, we have $s.I(t)=\sigma(id_t)("stc")$.

          From $\sigma(id_t)("se")=\mathtt{true}$, we can deduce
          $t\in{}Sens(s.M)$, and from
          $\sigma(id_t)("srtc")=\mathtt{false}$, we can deduce
          $s.reset_t(t)=\mathtt{false}$. Then, by definition of
          \dwSitpn{} (Rule~\ref{it:inc-counters}), we have
          $s'.I(t)=s.I(t)+1$.

          $\Rightarrow{}$
          \fbox{$\sigma(id_t)("A")\le{}s.I(t)+1$} (by $s'.I(t)=s.I(t)+1$)

          $\Rightarrow{}$
          \fbox{$\sigma(id_t)("A")\le\sigma(id_t)("stc")+1$} (by
          $s.I(t)=\sigma(id_t)("stc")$)

          $\Rightarrow{}$
          \qedbox{$\sigma(id_t)("A")-1\le\sigma(id_t)("stc")$}
          (assumption)
          
        \item \textbf{CASE} $s.I(t)>lower(I_s(t))$:

          By definition of \upSim, we have
          $\sigma(id_t)("stc")=lower(I_s(t))=a$.

          From $\sigma(id_t)("se")=\mathtt{true}$, we can deduce
          $t\in{}Sens(s.M)$, and from
          $\sigma(id_t)("srtc")=\mathtt{false}$, we can deduce
          $s.reset_t(t)=\mathtt{false}$. Then, by definition of
          \dwSitpn{} (Rule~\ref{it:inc-counters}), we have
          $s'.I(t)=s.I(t)+1$.

          $\Rightarrow{}$ \fbox{$\sigma(id_t)("A")\le{}s.I(t)+1$} (by
          $s'.I(t)=s.I(t)+1$)

          $\Rightarrow{}$ \fbox{$a\le{}s.I(t)+1$} (by
          $\sigma(id_t)("A")=a$)
          
          $\Rightarrow{}$ \fbox{$a<s.I(t)$}

          $\Rightarrow{}$ \qedbox{$lower(I_s(t))<s.I(t)$} (assumption)
        \end{itemize}
      \end{enumerate}

    \item
      $\big(\sigma(id_t)("srtc")~.~\Delta(id_t)("tt")\neq\mathtt{NOT\_TEMP}~.~\sigma(id_t)("A")=1\big)=\mathtt{true}$

      Then, converting boolean equalities into intuitionistic predicates, we have:
      \begin{itemize}
      \item $\sigma(id_t)("srtc")=\mathtt{true}$
      \item $\Delta(id_t)("tt")\neq\mathtt{NOT\_TEMP}$
      \item $\sigma(id_t)("A")=1$
      \end{itemize}

      By property of the elaboration relation, and
      $\Delta(id_t)("tt")\neq\mathtt{NOT\_TEMP}$, there exist an
      $a\in\mathbb{N}^{*}$ and a $ni\in\mathbb{N}^{*}\sqcup\{\infty\}$
      s.t. $I_s(t)=[a,ni]$. Let us take such an $a$ and $ni$.

      By construction,
      ${<}\mathtt{time\_A\_value\Rightarrow}{}a{>}\in{}ipm_t$, and by
      property of stable $\sigma$, we have
      $\sigma(id_t)("A")=a$. Thus, we can deduce $a=1$ and
      $I_s(t)=[1,ni]$.

      By definition of \upSim, from
      $\sigma(id_t)("se")=\mathtt{true}$, we can deduce
      $t\in{}Sens(s.M)$, and from
      $\sigma(id_t)("srtc")=\mathtt{true}$, we can deduce
      $s.reset_t(t)=\mathtt{true}$.

      By definition of \dwSitpn{} (Rule~\ref{it:reset-counters}),
      $t\in{}Sens(s.M)$ and $s.reset_t(t)=\mathtt{true}$, we have
      $s'.I(t)=1$.

      Now, let us show \fbox{$s'.I(t)\in{}I_s(t)$}.

      Rewriting the goal with $s'.I(t)=1$ and $I_s(t)=[1,ni]$:
      \qedbox{$1\in[1,ni]$.}
    \item
      $\big(\Delta(id_t)("tt")=\mathtt{NOT\_TEMP}\big)=\mathtt{true}$

      Let us show \fbox{$t\notin{}T_i$.}
      
      By property of the elaboration relation and
      $\Delta(id_t)("tt")=\mathtt{NOT\_TEMP}$, we have
      \qedbox{$t\notin{}T_i$.}
    \end{enumerate}
  \end{enumerate}
\end{niproof}

\begin{lemma}[Falling edge equal not firable]
  \label{lem:fe-equal-not-firable}
  \fehyps{} then
  $\forall{}t\in{}T,id_t\in{}Comps(\Delta)~s.t.~\gamma(t)=id_t,$
  $t\notin{}Firable(s')\Leftrightarrow\sigma'(id_t)("s\_firable")=\mathtt{false}$.
\end{lemma}

\begin{proof}
  Proving the above lemma is trivial by appealing to
  Lemma~\ref{lem:fe-equal-firable} and by reasoning on
  contrapositives.
\end{proof}

\subsection{Falling edge and fired transitions}
\label{sec:fe-fired}

%%%%%%%%%%%%%%%%%%%%%%%%%%%%%%%%%%%%%%%%%%%%%%%%%%%%%%%%
%%%%%%%%%% FALLING EDGE EQUAL FIRED SET LEMMA %%%%%%%%%%
%%%%%%%%%%%%%%%%%%%%%%%%%%%%%%%%%%%%%%%%%%%%%%%%%%%%%%%%

\begin{lemma}[Falling Edge Equal Fired Set]
  \label{lem:fe-equal-fset}
  \fehyps{} then
  $\forall{}t\in{}T,~id_t\in{}Comps(\Delta)~s.t.~\gamma(t)=id_t$,
  $\forall{}Fset\subseteq{}T,~s.t.~IsFiredSet(s',Fset),~t\in{}Fset\Leftrightarrow\sigma'(id_t)(\texttt{fired})=\mathtt{true}$.
\end{lemma}

\begin{niproof}
  Given a $t\in{}T$, and $id_t\in{}Comps(\Delta)$, and a
  $Fset\subseteq{}T$ s.t. $IsFiredSet(s',Fset)$, let us show
  \fbox{$t\in{}Fset\Leftrightarrow\sigma'(id_t)(\texttt{fired})=\mathtt{true}$.}\\

  By definition of $IsFiredSet(s',Fset)$, we have
  $IsFiredSetAux(s',T,\emptyset,Fset)$.

  Then, we can appeal to Lemma~\ref{lem:fe-equal-fset-aux} to solve
  the goal, but first we must prove the following \emph{extra
    hypothesis} (i.e, one of the premise of
  Lemma~\nameref{lem:fe-equal-fset-aux}):

  \fbox{\parbox{\lwidth}{$\forall{}t'\in{}T,id_{t'}\in{}Comps(\Delta)$
      s.t. $\gamma(t')=id_{t'},$\\
      $(t'\in{}\emptyset\Rightarrow\sigma'(id_{t'})(\texttt{fired})=\mathtt{true})$
      $\land$
      $(\sigma'(id_{t'})(\texttt{fired})=\mathtt{true}\Rightarrow{}t'\in{}\emptyset~\lor~{}t'\in{}T)$.}}\\

  Given a $t'\in{}T$ and an $id_{t'}\in{}Comps(\Delta)$
  s.t. $\gamma(t')=id_{t'}$, there are two points to prove:
  \begin{enumerate}
  \item
    \fbox{$t'\in{}\emptyset\Rightarrow\sigma'(id_{t'})(\texttt{fired})=\mathtt{true}$}
  \item
    \fbox{$\sigma'(id_{t'})(\texttt{fired})=\mathtt{true}\Rightarrow{}t'\in{}\emptyset~\lor~{}t'\in{}T$}
  \end{enumerate}

  Let us show these two points:
  \begin{enumerate}
  \item Assuming $t'\in{}\emptyset$, let us show
    \fbox{$\sigma'(id_{t'})(\texttt{fired})=\mathtt{true}$.}

    \qedbox{$t'\in{}\emptyset$ is a contradiction.}
  \item Assuming $\sigma'(id_{t'})(\texttt{fired})=\mathtt{true}$, let us
    show \fbox{$t'\in{}\emptyset~\lor~{}t'\in{}T$.}

    By definition, \qedbox{$t'\in{}T$.}
  \end{enumerate}
  
\end{niproof}

%%%%%%%%%%%%%%%%%%%%%%%%%%%%%%%%%%%%%%%%%%%%%%%%%%%%%%%%%%%%
%%%%%%%%%% FALLING EDGE EQUAL FIRED SET AUX LEMMA %%%%%%%%%%
%%%%%%%%%%%%%%%%%%%%%%%%%%%%%%%%%%%%%%%%%%%%%%%%%%%%%%%%%%%%

\begin{lemma}[Falling edge equal fired set aux]
  \label{lem:fe-equal-fset-aux}
  \fehyps{} then
  $\forall{}t\in{}T,id_t\in{}Comps(\Delta)~s.t.~\gamma(t)=id_t$,
  $\forall{}F\subseteq{}T,~T_s\subseteq{}T,~Fset\subseteq{}T,$
  assume that:
  \begin{itemize}
  \item $IsFiredSetAux(s',T_s,F,Fset)$
  \item EH (Extra. Hypothesis):\\
    $\forall{}t'\in{}T,~id_{t'}\in{}Comps(\Delta)$
    s.t. $\gamma(t')=id_{t'}$,\\
    $(t'\in{}F\Rightarrow\sigma'(id_{t'})(\texttt{fired})=\mathtt{true})$
    $\land$
    $(\sigma'(id_{t'})(\texttt{fired})=\mathtt{true}\Rightarrow{}t'\in{}F~\lor~{}t'\in{}T_s)$.
  \end{itemize}
  then
  $t\in{}Fset\Leftrightarrow\sigma'(id_t)(\texttt{fired})=\mathtt{true}$.
\end{lemma}

\begin{niproof}
  Given a $t\in{}T$, an $id_t\in{}Comps(\Delta)$, a
  $T_s,F,Fset\subseteq{}T$, and assuming\\
  $IsFiredSetAux(s',T_s,F,Fset)$, let us show
  \begin{frameb}
    $\big(\forall{}t'\in{}T,~id_{t'}\in{}Comps(\Delta)$
    s.t. $\gamma(t')=id_{t'}$,\\
    $(t'\in{}F\Rightarrow\sigma'(id_{t'})(\texttt{fired})=\mathtt{true})$
    $\land$
    $(\sigma'(id_{t'})(\texttt{fired})=\mathtt{true}\Rightarrow{}t'\in{}F~\lor~{}t'\in{}T_s)\big)\Rightarrow$
    $t\in{}Fset\Leftrightarrow\sigma'(id_t)(\texttt{fired})=\mathtt{true}$.
  \end{frameb}

  Let us use rule induction on $IsFiredSetAux(s',T_s,F,Fset)$. Let us
  define the property $P$ taken into account in the induction scheme
  as follows

  \begin{center}
    \begin{tabular}{c}
      $P(s',T_s,F,Fset)$ \\
      $\equiv$ \\
      $(t'\in{}F\Rightarrow\sigma'(id_{t'})(\texttt{fired})=\mathtt{true})$
      $\land$
      $\big(\sigma'(id_{t'})(\texttt{fired})=\mathtt{true}\Rightarrow{}t'\in{}F~\lor~{}t'\in{}T_s)\big)\Rightarrow$ \\
      $t\in{}Fset\Leftrightarrow\sigma'(id_t)(\texttt{fired})=\mathtt{true}$ \\
    \end{tabular}
  \end{center}

  \begin{itemize}
  \item \textbf{CASE} \textsc{FSetEmp}: we must show $P(s',\emptyset,F,F)$, i.e.
    \begin{frameb}
      $\big(\forall{}t'\in{}T,~id_{t'}\in{}Comps(\Delta)$
      s.t. $\gamma(t')=id_{t'}$,\\
      $(t'\in{}F\Rightarrow\sigma'(id_{t'})(\texttt{fired})=\mathtt{true})$
      $\land$
      $(\sigma'(id_{t'})(\texttt{fired})=\mathtt{true}\Rightarrow{}t'\in{}F~\lor~{}t'\in{}\emptyset)\big)\Rightarrow$
      $t\in{}F\Leftrightarrow\sigma'(id_t)(\texttt{fired})=\mathtt{true}$.
    \end{frameb}
    
    Assuming
    \begin{center}
      \begin{tabular}{l}
        $\forall{}t'\in{}T,~id_{t'}\in{}Comps(\Delta)$
        s.t. $\gamma(t')=id_{t'}$,\\
        $(t'\in{}F\Rightarrow\sigma'(id_{t'})(\texttt{fired})=\mathtt{true})$
        $\land$
        $(\sigma'(id_{t'})(\texttt{fired})=\mathtt{true}\Rightarrow{}t'\in{}F~\lor~{}t'\in{}\emptyset)$ \\
      \end{tabular}
    \end{center}
    we can easily show
    \qedbox{$t\in{}F\Leftrightarrow\sigma'(id_t)(\texttt{fired})=\mathtt{true}$.}
    
  \item \textbf{CASE} \textsc{FSetFired}:

    Assuming
    \begin{itemize}
    \item $t\in{}Firable(s')$
    \item $t\in{}Sens(s'.M-\sum\limits_{t_i\in{}Pr(t,F)}pre(t_i))$
    \item $IsFiredSetAux(s',T_s,F\cup\{t\},Fset)$
    \item $\nexists{}t'\in{}T_s~s.t.~t'\succ{}t$
    \item $Pr(t,F)=\{t'~\vert~t'\succ{}t\land{}t'\in{}F\}$
    \end{itemize}
    and the induction hypothesis (i.e. $P(s',T_s,F\cup\{t\},Fset)$)
    \begin{ih}
      \begin{tabular}{l}
        $\big(\forall{}t'\in{}T,~id_{t'}\in{}Comps(\Delta)$
        s.t. $\gamma(t')=id_{t'}$,\\
        $(t'\in{}F\cup\{t\}\Rightarrow\sigma'(id_{t'})(\texttt{fired})=\mathtt{true})$\\
        $\land$
        $(\sigma'(id_{t'})(\texttt{fired})=\mathtt{true}\Rightarrow{}t'\in{}F\cup\{t\}~\lor~{}t'\in{}T_s)\big)\Rightarrow$ \\
        $t\in{}Fset\Leftrightarrow\sigma'(id_t)(\texttt{fired})=\mathtt{true}$
      \end{tabular}
    \end{ih}

    we must show

    \begin{frameb}
      \begin{tabular}{l}
        $\big(\forall{}t'\in{}T,~id_{t'}\in{}Comps(\Delta)$
        s.t. $\gamma(t')=id_{t'}$,\\
        $(t'\in{}F\Rightarrow\sigma'(id_{t'})(\texttt{fired})=\mathtt{true})$\\
        $\land$
        $(\sigma'(id_{t'})(\texttt{fired})=\mathtt{true}\Rightarrow{}t'\in{}F~\lor~{}t'\in{}T_s\cup\{t\})\big)\Rightarrow$ \\
        $t\in{}Fset\Leftrightarrow\sigma'(id_t)(\texttt{fired})=\mathtt{true}$
      \end{tabular}
    \end{frameb}

    Assuming the following hypothesis that we will call EH (for Extra
    Hypothesis)

    \begin{center}
      \begin{tabular}{l}
        $\forall{}t'\in{}T,~id_{t'}\in{}Comps(\Delta)$
        s.t. $\gamma(t')=id_{t'}$,\\
        $(t'\in{}F\Rightarrow\sigma'(id_{t'})(\texttt{fired})=\mathtt{true})$
        $\land$
        $(\sigma'(id_{t'})(\texttt{fired})=\mathtt{true}\Rightarrow{}t'\in{}F~\lor~{}t'\in{}T_s\cup\{t\})$ \\
      \end{tabular}
    \end{center}

    we must show
    \begin{frameb}
      $t\in{}Fset\Leftrightarrow\sigma'(id_t)(\texttt{fired})=\mathtt{true}$
    \end{frameb}
    
    Appealing to the induction hypothesis, to prove the current goal,
    it is sufficient to prove that

    \begin{frameb}
      \begin{tabular}{l}
        $\forall{}t'\in{}T,~id_{t'}\in{}Comps(\Delta)$
        s.t. $\gamma(t')=id_{t'}$,\\
        $(t'\in{}F\cup\{t\}\Rightarrow\sigma'(id_{t'})(\texttt{fired})=\mathtt{true})$\\
        $\land$
        $(\sigma'(id_{t'})(\texttt{fired})=\mathtt{true}\Rightarrow{}t'\in{}F\cup\{t\}~\lor~{}t'\in{}T_s)$ \\
      \end{tabular}
    \end{frameb}

    Given a $t'\in{}T$, an $id_{t'}\in{}Comps(\Delta)$ s.t. $\gamma(t')=id_{t'}$, we must show that
    \begin{frameb}
      \begin{tabular}{l}
        $(t'\in{}F\cup\{t\}\Rightarrow\sigma'(id_{t'})(\texttt{fired})=\mathtt{true})$\\
        $\land$
        $(\sigma'(id_{t'})(\texttt{fired})=\mathtt{true}\Rightarrow{}t'\in{}F\cup\{t\}~\lor~{}t'\in{}T_s)$ \\
      \end{tabular}
    \end{frameb}

    There are two points to prove
    \begin{enumerate}
    \item Assuming $t'\in{}F\cup\{t\}$, then
      $\sigma'(id_{t'})(\texttt{fired})=\mathtt{true}$
    \item Assuming $\sigma'(id_{t'})(\texttt{fired})=\mathtt{true}$,
      then $t'\in{}F\cup\{t\}~\lor~{}t'\in{}T_s$
    \end{enumerate}

    \begin{enumerate}
    \item Assuming $t'\in{}F\cup\{t\}$, let us show \fbox{$\sigma'(id_{t'})(\texttt{fired})=\mathtt{true}$}.
      Let us perform case analysis on $t'\in{}F\cup\{t\}$; there are 2 cases:
      \begin{itemize}
      \item \textbf{CASE} $t'\in{}F$: Appealing to EH, the goal is trivially proved.
      \item \textbf{CASE} $t'=t$: Then, $id_t=id_{t'}$, and we must
        show \fbox{$\sigma'(id_t)(\texttt{fired})=\mathtt{true}$}.

        By definition of $id_{t}$, there exist a $g_{t}$, $i_{t}$,
        $o_{t}$ s.t.  $\mathtt{comp}(id_{t},\texttt{transition},$
        $g_{t},$ $i_{t}$, $o_{t})\in{}d.cs$.

        By property of the stabilize relation and
        $\mathtt{comp}(id_{t},\texttt{transition},$ $g_{t},$ $i_{t}$,
        $o_{t})\in{}d.cs$, and through the examination of the
        \texttt{fired_evaluation} process defined in the
        \texttt{transition} design architecture:
        \begin{equation*}
          \sigma(id_{t})(\texttt{fired})=\sigma(id_{t})(\texttt{sfa})~.~\sigma(id_{t})(\texttt{spc})
        \end{equation*}

        Rewriting the goal with the above equation:
        \fbox{$\sigma(id_{t})(\texttt{sfa})~.~\sigma(id_{t})(\texttt{spc})=\mathtt{true}$.}
        
        Then, there are two points to prove: 
        \begin{enumerate}
        \item \fbox{$\sigma(id_{t})(\texttt{sfa})=\mathtt{true}$.}
          
          Appealing to Lemma~\ref{lem:fe-equal-firable}, and since
          $t\in{}Firable(s')$, we can deduce
          \qedbox{$\sigma(id_{t})(\texttt{sfa})=\mathtt{true}$.}
        \item \fbox{$\sigma(id_{t})(\texttt{spc})=\mathtt{true}$.}
          
          Appealing to Lemma~\ref{lem:stab-compute-pcomb}, and since
          $t\in{}Sens(s'M-\sum\limits_{t_i\in{}Pr(t,F)}pre(t_i))$, we
          can deduce
          \qedbox{$\sigma(id_{t})(\texttt{spc})=\mathtt{true}$.}
        \end{enumerate}
      \end{itemize}
      
    \item Assuming $\sigma'(id_{t'})(\texttt{fired})=\mathtt{true}$,
      let us show
      \fbox{$t'\in{}F\cup\{t\}~\lor~{}t'\in{}T_s$}. Appealing to EH,
      we can deduce that $t'\in{}F~\lor~{}t'\in{}T_s\cup\{t\}$. Then,
      the goal is trivially shown.
    \end{enumerate}
    
  \item \textbf{CASE} \textsc{FSetNotFirable}:
    Assuming
    \begin{itemize}
    \item $t\notin{}Firable(s')$
    \item $IsFiredSetAux(s',T_s,F,Fset)$
    \item $\nexists{}t'\in{}T_s~s.t.~t'\succ{}t$
    \end{itemize}
    and the induction hypothesis (i.e. $P(s',T_s,F,Fset)$)
    \begin{ih}
      \begin{tabular}{l}
        $\big(\forall{}t'\in{}T,~id_{t'}\in{}Comps(\Delta)$
        s.t. $\gamma(t')=id_{t'}$,\\
        $(t'\in{}F\Rightarrow\sigma'(id_{t'})(\texttt{fired})=\mathtt{true})$\\
        $\land$
        $(\sigma'(id_{t'})(\texttt{fired})=\mathtt{true}\Rightarrow{}t'\in{}F~\lor~{}t'\in{}T_s)\big)\Rightarrow$ \\
        $t\in{}Fset\Leftrightarrow\sigma'(id_t)(\texttt{fired})=\mathtt{true}$
      \end{tabular}
    \end{ih}

    we must show

    \begin{frameb}
      \begin{tabular}{l}
        $\big(\forall{}t'\in{}T,~id_{t'}\in{}Comps(\Delta)$
        s.t. $\gamma(t')=id_{t'}$,\\
        $(t'\in{}F\Rightarrow\sigma'(id_{t'})(\texttt{fired})=\mathtt{true})$\\
        $\land$
        $(\sigma'(id_{t'})(\texttt{fired})=\mathtt{true}\Rightarrow{}t'\in{}F~\lor~{}t'\in{}T_s\cup\{t\})\big)\Rightarrow$ \\
        $t\in{}Fset\Leftrightarrow\sigma'(id_t)(\texttt{fired})=\mathtt{true}$
      \end{tabular}
    \end{frameb}

    Assuming the following hypothesis that we will call EH (for Extra
    Hypothesis)

    \begin{center}
      \begin{tabular}{l}
        $\forall{}t'\in{}T,~id_{t'}\in{}Comps(\Delta)$
        s.t. $\gamma(t')=id_{t'}$,\\
        $(t'\in{}F\Rightarrow\sigma'(id_{t'})(\texttt{fired})=\mathtt{true})$
        $\land$
        $(\sigma'(id_{t'})(\texttt{fired})=\mathtt{true}\Rightarrow{}t'\in{}F~\lor~{}t'\in{}T_s\cup\{t\})$ \\
      \end{tabular}
    \end{center}

    we must show
    \begin{frameb}
      $t\in{}Fset\Leftrightarrow\sigma'(id_t)(\texttt{fired})=\mathtt{true}$
    \end{frameb}
    
    Appealing to the induction hypothesis, to prove the current goal,
    it is sufficient to prove that

    \begin{frameb}
      \begin{tabular}{l}
        $\forall{}t'\in{}T,~id_{t'}\in{}Comps(\Delta)$
        s.t. $\gamma(t')=id_{t'}$,\\
        $(t'\in{}F\Rightarrow\sigma'(id_{t'})(\texttt{fired})=\mathtt{true})$
        $\land$
        $(\sigma'(id_{t'})(\texttt{fired})=\mathtt{true}\Rightarrow{}t'\in{}F~\lor~{}t'\in{}T_s)$ \\
      \end{tabular}
    \end{frameb}

    Given a $t'\in{}T$, an $id_{t'}\in{}Comps(\Delta)$ s.t. $\gamma(t')=id_{t'}$, we must show that
    \begin{frameb}
      \begin{tabular}{l}
        $(t'\in{}F\Rightarrow\sigma'(id_{t'})(\texttt{fired})=\mathtt{true})$
        $\land$
        $(\sigma'(id_{t'})(\texttt{fired})=\mathtt{true}\Rightarrow{}t'\in{}F~\lor~{}t'\in{}T_s)$ \\
      \end{tabular}
    \end{frameb}

    There are two points to prove
    \begin{enumerate}
    \item Assuming $t'\in{}F$, then
      $\sigma'(id_{t'})(\texttt{fired})=\mathtt{true}$
    \item Assuming $\sigma'(id_{t'})(\texttt{fired})=\mathtt{true}$,
      then $t'\in{}F~\lor~{}t'\in{}T_s$
    \end{enumerate}

    \begin{enumerate}
    \item Assuming $t'\in{}F$, let us show
      \fbox{$\sigma'(id_{t'})(\texttt{fired})=\mathtt{true}$}.

      Appealing to EH, the goal is trivially shown.
      
    \item Assuming $\sigma'(id_{t'})(\texttt{fired})=\mathtt{true}$,
      let us show \fbox{$t'\in{}F~\lor~{}t'\in{}T_s$}.

      Appealing to EH, we can deduce $t'\in{}F~\lor~{}t'\in{}T_s\cup\{t\}$.
      Let us perform case analysis on $t'\in{}F~\lor~{}t'\in{}T_s\cup\{t\}$; there are 2 cases:
      \begin{itemize}
      \item \textbf{CASE} $t'\in{}F$: trivially shown, as it is an assumption.
      \item \textbf{CASE} $t'\in{}T_s\cup\{t\}$: In the case where
        $t'\in{}T_s$, the goal is trivially shown. In the case where
        $t'=t$, we can prove a contradiction based on
        $t\notin{}Firable(s')$ and
        $\sigma'(id_{t'})(\texttt{fired})=\mathtt{true}$.

        Since $t=t'$, then $id_t=id_{t'}$, and we know that
        $\sigma'(id_t)(\texttt{fired})=\mathtt{true}$.

        By definition of $id_{t}$, there exist a $g_{t}$, $i_{t}$,
        $o_{t}$ s.t.  $\mathtt{comp}(id_{t},\texttt{transition},$
        $g_{t},$ $i_{t}$, $o_{t})\in{}d.cs$.

        By property of the stabilize relation and
        $\mathtt{comp}(id_{t},\texttt{transition},$ $g_{t},$ $i_{t}$,
        $o_{t})\in{}d.cs$, and through the examination of the
        \texttt{fired_evaluation} process defined in the
        \texttt{transition} design architecture, we can deduce
        \begin{equation*}
          \sigma(id_{t})(\texttt{fired})=\sigma(id_{t})(\texttt{sfa})~.~\sigma(id_{t})(\texttt{spc})=\mathtt{true}
        \end{equation*}
        
        Thus, we have
        \begin{equation*}
          \sigma(id_{t})(\texttt{sfa})=\mathtt{true}
        \end{equation*}
        and, appealing to Lemma~\ref{lem:fe-equal-firable}, we can
        deduce $t\in{}Firable(s')$, which directly contradicts
        \qedbox{ $t\notin{}Firable(s')$.}
      \end{itemize}
    \end{enumerate}
    
  \item \textbf{CASE} \textsc{FSetNotSens}:
    Assuming
    \begin{itemize}
    \item $t\notin{}Sens(s'.M-\sum\limits_{t_i\in{}Pr(t,F)}pre(t_i))$
    \item $IsFiredSetAux(s',T_s,F,Fset)$
    \item $\nexists{}t'\in{}T_s~s.t.~t'\succ{}t$
    \item $Pr(t,F)=\{t'~\vert~t'\succ{}t\land{}t'\in{}F\}$
    \end{itemize}
    and the induction hypothesis (i.e. $P(s',T_s,F,Fset)$)
    \begin{ih}
      \begin{tabular}{l}
        $\big(\forall{}t'\in{}T,~id_{t'}\in{}Comps(\Delta)$
        s.t. $\gamma(t')=id_{t'}$,\\
        $(t'\in{}F\Rightarrow\sigma'(id_{t'})(\texttt{fired})=\mathtt{true})$\\
        $\land$
        $(\sigma'(id_{t'})(\texttt{fired})=\mathtt{true}\Rightarrow{}t'\in{}F~\lor~{}t'\in{}T_s)\big)\Rightarrow$ \\
        $t\in{}Fset\Leftrightarrow\sigma'(id_t)(\texttt{fired})=\mathtt{true}$
      \end{tabular}
    \end{ih}

    we must show

    \begin{frameb}
      \begin{tabular}{l}
        $\big(\forall{}t'\in{}T,~id_{t'}\in{}Comps(\Delta)$
        s.t. $\gamma(t')=id_{t'}$,\\
        $(t'\in{}F\Rightarrow\sigma'(id_{t'})(\texttt{fired})=\mathtt{true})$\\
        $\land$
        $(\sigma'(id_{t'})(\texttt{fired})=\mathtt{true}\Rightarrow{}t'\in{}F~\lor~{}t'\in{}T_s\cup\{t\})\big)\Rightarrow$ \\
        $t\in{}Fset\Leftrightarrow\sigma'(id_t)(\texttt{fired})=\mathtt{true}$
      \end{tabular}
    \end{frameb}

    Assuming the following hypothesis, which we will call EH (for
    Extra Hypothesis)

    \begin{center}
      \begin{tabular}{l}
        $\forall{}t'\in{}T,~id_{t'}\in{}Comps(\Delta)$
        s.t. $\gamma(t')=id_{t'}$,\\
        $(t'\in{}F\Rightarrow\sigma'(id_{t'})(\texttt{fired})=\mathtt{true})$
        $\land$
        $(\sigma'(id_{t'})(\texttt{fired})=\mathtt{true}\Rightarrow{}t'\in{}F~\lor~{}t'\in{}T_s\cup\{t\})$ \\
      \end{tabular}
    \end{center}

    we must show
    \begin{frameb}
      $t\in{}Fset\Leftrightarrow\sigma'(id_t)(\texttt{fired})=\mathtt{true}$
    \end{frameb}
    
    Appealing to the induction hypothesis, to prove the current goal,
    it is sufficient to prove that

    \begin{frameb}
      \begin{tabular}{l}
        $\forall{}t'\in{}T,~id_{t'}\in{}Comps(\Delta)$
        s.t. $\gamma(t')=id_{t'}$,\\
        $(t'\in{}F\Rightarrow\sigma'(id_{t'})(\texttt{fired})=\mathtt{true})$
        $\land$
        $(\sigma'(id_{t'})(\texttt{fired})=\mathtt{true}\Rightarrow{}t'\in{}F~\lor~{}t'\in{}T_s)$ \\
      \end{tabular}
    \end{frameb}

    Given a $t'\in{}T$, an $id_{t'}\in{}Comps(\Delta)$ s.t. $\gamma(t')=id_{t'}$, we must show that
    \begin{frameb}
      \begin{tabular}{l}
        $(t'\in{}F\Rightarrow\sigma'(id_{t'})(\texttt{fired})=\mathtt{true})$
        $\land$
        $(\sigma'(id_{t'})(\texttt{fired})=\mathtt{true}\Rightarrow{}t'\in{}F~\lor~{}t'\in{}T_s)$ \\
      \end{tabular}
    \end{frameb}

    There are two points to prove
    \begin{enumerate}
    \item Assuming $t'\in{}F$, then
      $\sigma'(id_{t'})(\texttt{fired})=\mathtt{true}$
    \item Assuming $\sigma'(id_{t'})(\texttt{fired})=\mathtt{true}$,
      then $t'\in{}F~\lor~{}t'\in{}T_s$
    \end{enumerate}

    \begin{enumerate}
    \item Assuming $t'\in{}F$, let us show
      \fbox{$\sigma'(id_{t'})(\texttt{fired})=\mathtt{true}$}.

      Appealing to EH, the goal is trivially shown.
      
    \item Assuming $\sigma'(id_{t'})(\texttt{fired})=\mathtt{true}$,
      let us show \fbox{$t'\in{}F~\lor~{}t'\in{}T_s$}.

      Appealing to EH, we can deduce $t'\in{}F~\lor~{}t'\in{}T_s\cup\{t\}$.
      Let us perform case analysis on $t'\in{}F~\lor~{}t'\in{}T_s\cup\{t\}$; there are 2 cases:
      \begin{itemize}
      \item \textbf{CASE} $t'\in{}F$: trivially shown, as it is an assumption.
      \item \textbf{CASE} $t'\in{}T_s\cup\{t\}$: In the case where
        $t'\in{}T_s$, the goal is trivially shown. In the case where
        $t'=t$, we can prove a contradiction based on
        $t\notin{}Sens(s'.M-\sum\limits_{t_i\in{}Pr(t,F)}pre(t_i))$
        and $\sigma'(id_{t'})(\texttt{fired})=\mathtt{true}$.

        Since $t=t'$, then $id_t=id_{t'}$, and we know that
        $\sigma'(id_t)(\texttt{fired})=\mathtt{true}$.

        By definition of $id_{t}$, there exist a $g_{t}$, $i_{t}$,
        $o_{t}$ s.t.  $\mathtt{comp}(id_{t},\texttt{transition},$
        $g_{t},$ $i_{t}$, $o_{t})\in{}d.cs$.

        By property of the stabilize relation and
        $\mathtt{comp}(id_{t},\texttt{transition},$ $g_{t},$ $i_{t}$,
        $o_{t})\in{}d.cs$, and through the examination of the
        \texttt{fired_evaluation} process defined in the
        \texttt{transition} design architecture, we can deduce
        \begin{equation*}
          \sigma(id_{t})(\texttt{fired})=\sigma(id_{t})(\texttt{sfa})~.~\sigma(id_{t})(\texttt{spc})=\mathtt{true}
        \end{equation*}
        
        Thus, we have
        \begin{equation*}
          \sigma(id_{t})(\texttt{spc})=\mathtt{true}
        \end{equation*}
        and, appealing to Lemma~\ref{lem:stab-compute-pcomb}, we can
        deduce
        $t\in{}Sens(s'.M-\sum\limits_{t_i\in{}Pr(t,F)}pre(t_i))$,
        which directly contradicts \qedbox{
          $t\notin{}Sens(s'.M-\sum\limits_{t_i\in{}Pr(t,F)}pre(t_i))$.}
      \end{itemize}
    \end{enumerate}
  \end{itemize}
  
\end{niproof}

%%%%%%%%%%%%%%%%%%%%%%%%%%%%%%%%%%%%%%%%%%%%%%%%%%%%%%%%%%%%%%%%%%%%%%%%%%%%%%%
%%%%%%%%%% STABILIZE COMPUTE PRIORITY COMBINATION AFTER FALLING EDGE %%%%%%%%%%
%%%%%%%%%%%%%%%%%%%%%%%%%%%%%%%%%%%%%%%%%%%%%%%%%%%%%%%%%%%%%%%%%%%%%%%%%%%%%%%

\begin{lemma}[Stabilize compute priority combination after falling edge]
  \label{lem:stab-compute-pcomb}
  \fehyps{} then
  $\forall{}t\in{}T,id_t\in{}Comps(\Delta)$ s.t. $\gamma(t)=id_t$,
  $\forall{}T_s,F,~Fset\subseteq{}T$ assume that:
  \begin{itemize}
  \item $t\in{}Firable(s')$
  \item $\nexists{}t'\in{}T_s~s.t.~t'\succ{}t$
  \item EH: $\forall{}t'\in{}T,id_{t'}\in{}Comps(\Delta)$
    s.t. $\gamma(t')=id_{t'}$,\\
    $(t'\in{}F\Rightarrow\sigma'(id_{t'})(\texttt{fired})=\mathtt{true})$
    $\land$
    $(\sigma'(id_{t'})(\texttt{fired})=\mathtt{true}\Rightarrow{}t'\in{}F~\lor~{}t'\in{}T_s)$.

  \end{itemize}
  then
  $t\in{}Sens(s'.M-\sum\limits_{t_i\in{}Pr(t,F)}pre(t_i))\Leftrightarrow\sigma'(id_t)(\texttt{spc})=\mathtt{true}$
\end{lemma}

\begin{niproof}

  Given a $t\in{}T$ and an $id_t\in{}Comps(\Delta)$
  s.t. $\gamma(t)=id_t$, a $T_s$, $F$, $Fset\subseteq{}T$ and assuming
  \begin{itemize}
  \item $t\in{}Firable(s')$
  \item $\nexists{}t'\in{}T_s~s.t.~t'\succ{}t$
  \item EH: $\forall{}t'\in{}T,id_{t'}\in{}Comps(\Delta)$
    s.t. $\gamma(t')=id_{t'}$,\\
    $(t'\in{}F\Rightarrow\sigma'(id_{t'})(\texttt{fired})=\mathtt{true})$
    $\land$
    $(\sigma'(id_{t'})(\texttt{fired})=\mathtt{true}\Rightarrow{}t'\in{}F~\lor~{}t'\in{}T_s)$.
  \end{itemize}
  
  let us show
  \begin{frameb}
    $t\in{}Sens(s'.M-\sum\limits_{t_i\in{}Pr(t,F)}pre(t_i))\Leftrightarrow\sigma'(id_t)(\texttt{spc})=\mathtt{true}$.
  \end{frameb}

  \exT{}
  
  By property of the stabilize relation, \InCsCompT, and through the
  examination of the \texttt{priority_authorization_evaluation}
  process defined in the \texttt{transition} design architecture, we
  can deduce:
  \begin{equation*}
    \sigma'(id_t)("spc")=\prod\limits_{i=0}^{\Delta(id_t)(\texttt{ian})-1}\sigma'(id_t)(\texttt{pauths})[i]\label{eq:frd-eq-spc-prod-pauths}
  \end{equation*}

  Rewriting the goal with the above equation:  
  \begin{frameb}
    $t\in{}Sens(s'.M-\sum\limits_{t_i\in{}Pr(t,F)}pre(t_i))\Leftrightarrow\prod\limits_{i=0}^{\Delta(id_t)(\texttt{ian})-1}\sigma'(id_t)(\texttt{pauths})[i]=\mathtt{true}$.
  \end{frameb}
  
  Then, the proof is in two parts:
  \begin{enumerate}
  \item
    $t\in{}Sens(s'.M-\sum\limits_{t_i\in{}Pr(t,F)}pre(t_i))\Rightarrow\prod\limits_{i=0}^{\Delta(id_t)(\texttt{ian})-1}\sigma'(id_t)(\texttt{pauths})[i]=\mathtt{true}$
  \item
    $\prod\limits_{i=0}^{\Delta(id_t)(\texttt{ian})-1}\sigma'(id_t)(\texttt{pauths})[i]=\mathtt{true}\Rightarrow{}t\in{}Sens(s'.M-\sum\limits_{t_i\in{}Pr(t,F)}pre(t_i))$
  \end{enumerate}

  Let us prove both sides of the equivalence:
  \begin{enumerate}
  \item\label{item:stab-comp-spc-fst-case} Assuming that
    $t\in{}Sens(s'.M-\sum\limits_{t_i\in{}Pr(t,F)}pre(t_i))$, let us
    show\\
    \fbox{$\prod\limits_{i=0}^{\Delta(id_t)(\texttt{ian})-1}\sigma'(id_t)(\texttt{pauths})[i]=\mathtt{true}$.}

    Let us perform case analysis on $input(t)$; there are 2 cases:
    \begin{itemize}
    \item \textbf{CASE} $input(t)=\emptyset$:

      By construction,
      ${<}\mathtt{input\_arcs\_number\Rightarrow}{}1{>}\in{}g_t$ and
      ${<}$\texttt{priority\_authorizations(0)}$\Rightarrow{}\mathtt{true}{>}\in{}i_t$.

      By property of the elaboration relation, we have
      $\Delta(id_t)(\texttt{ian})=1$, and by property of the stabilize
      relation, we have
      $\sigma'(id_t)(\texttt{pauths})[0]=\mathtt{true}$.
      
      Rewriting the goal with $\Delta(id_t)(\texttt{ian})=1$ and
      $\sigma'(id_t)(\texttt{pauths})[0]=\mathtt{true}$, and
      simplifying the goal: \qedbox{tautology.}
      
    \item \textbf{CASE} $input(t)\neq{}\emptyset$:

      Then, let us show an equivalent goal:\\
      \fbox{$\forall{}i\in[0,\Delta(id_t)(\texttt{ian})-1],~\sigma'(id_t)(\texttt{pauths})[i]=\mathtt{true}$.}

      Given an $i\in{}[0,\Delta(id_t)(\texttt{ian})-1]$, let us show
      \fbox{$\sigma'(id_t)(\texttt{pauths})[i]=\mathtt{true}$.}

      By construction,
      ${<}\mathtt{input\_arcs\_number\Rightarrow}{}\vert{}input(t)\vert{>}\in{}g_t$.

      By property of the elaboration relation, we have
      $\Delta(id_t)(\texttt{ian})=\vert{}input(t)\vert$. Then, we can deduce
      $i\in{}[0,\vert{}input(t)\vert-1]$.
      
      By construction, for all $i\in{}[0,\vert{}input(t)\vert-1]$,
      there exist a $p\in{}input(t)$ and an $id_p\in{}Comps(\Delta)$
      s.t. $\gamma(p)=id_p$, there exist a $g_p$, $i_p$, $o_p$
      s.t. $\mathtt{comp}(id_p,$ $\texttt{place},$ $g_p,$ $i_p,$
      $o_p)\in{}d.cs$, and there exist a
      $j\in{}[0,\vert{}output(p)\vert]$ and an
      $id_{ji}\in{}Sigs(\Delta)$ s.t.\\
      ${<}\mathtt{input\_arcs\_valid(i)\Rightarrow{}id_{ji}}{>}\in{}i_t$
      and
      ${<}\mathtt{output\_arcs\_valid(j)\Rightarrow{}id_{ji}}{>}\in{}o_t$.
      Let us take such a $p\in{}input(t)$, $id_p\in{}Comps(\Delta)$,
      $g_p$, $i_p$, $o_p$, $j\in{}[0,$ $\vert{}output(p)\vert]$ and
      $id_{ji}\in{}Sigs(\Delta)$.\\

      Now, let us perform case analysis on the nature of the arc
      connecting $p$ and $t$; there are 2 cases:
      
      \begin{itemize}
      \item \textbf{CASE} $pre(p,t)=(\omega,\mathtt{test})$ or
        $pre(p,t)=(\omega,\mathtt{inhib})$:

        By construction,
        ${<}\mathtt{priority\_authorizations(i)\Rightarrow{}true}{>}\in{}i_t$,
        and by property of the stabilize relation:
        \qedbox{$\sigma'(id_t)(\texttt{pauths})[i]=\mathtt{true}$.}

      \item \textbf{CASE} $pre(p,t)=(\omega,\mathtt{basic})$:

        Let us define
        $output_c(p)=\{t\in{}T~\vert~\exists{}\omega,~pre(p,t)=(\omega,\mathtt{basic})\}$,
        the set of output transitions of $p$ that are in
        conflict. Then, there are two cases, one for each way to solve
        the conflicts between the output transitions of $p$:

        \begin{itemize}
        \item \textbf{CASE} For all pair of transitions in
          $output_c(p)$, all conflicts are solved by mutual exclusion:

          By construction,
          ${<}\mathtt{priority\_authorizations(i)\Rightarrow{}true}{>}\in{}i_t$,
          and by property of the stabilize relation:
          \qedbox{$\sigma'(id_t)(\texttt{pauths})[i]=\mathtt{true}$.}
        \item \textbf{CASE} The priority relation is a strict total
          order over the set $output_c(p)$:
        \end{itemize}
        By construction, there exists an $id'_{ji}\in{}Sigs(\Delta)$
        s.t.\\
        ${<}\mathtt{priority\_authorizations(i)\Rightarrow{}id'_{ji}}{>}\in{}i_t$
        and\\
        ${<}\mathtt{priority\_authorizations(j)\Rightarrow{}id'_{ji}}{>}\in{}o_p$.

        By property of the stabilize relation, \InCsCompT{} and
        \InCsCompP, we can deduce:
        \begin{equation*}
          \label{eq:frd-eq-tpauthsi-ppauthsj}\sigma'(id_t)(\texttt{pauths})[i]=\sigma'(id'_{ji})=\sigma'(id_p)(\texttt{pauths})[j]\\
        \end{equation*}

        Rewriting the goal with the above equation:
        \fbox{$\sigma'(id_p)(\texttt{pauths})[j]=\mathtt{true}$.}

        By property of the stabilize relation, \InCsCompP, and
        through the examination of the \texttt{priority_evaluation}
        process defined in the \texttt{place} design behavior, we
        can deduce:
        \begin{equation}
          \label{eq:frd-eq-pauthsj}
          \sigma'(id_p)(\texttt{pauths})[j]=(\sigma'(id_p)(\texttt{sm})\ge{}\mathtt{vsots}+\sigma'(id_p)(\texttt{oaw})[j])
        \end{equation}

        Let us define the $\mathtt{vsots}$ term as follows:
        \begin{equation}
          \label{eq:frd-vsots}
          \mathtt{vsots}=\sum\limits_{i=0}^{j-1}
          \begin{cases}
            \sigma'(id_p)(\texttt{oaw})[i]~\mathtt{if}~\sigma'(id_p)(\texttt{otf})[i].\\
            \hspace{19ex}\sigma'(id_p)(\texttt{oat})[i]=\mathtt{basic}\\
            0~otherwise\\
          \end{cases}
        \end{equation}

        Rewriting the goal with \eqref{eq:frd-eq-pauthsj}:
        \fbox{$\sigma'(id_p)(\texttt{sm})\ge{}\mathtt{vsots}+\sigma'(id_p)(\texttt{oaw})[j]$}

        By definition of
        $t\in{}Sens(s'.M-\sum\limits_{t_i\in{}Pr(t,F)}pre(t_i))$, we
        can deduce:\\
        $s'.M(p)\ge{}\sum\limits_{t_i\in{}Pr(t,F)}pre(p,t_i)+\omega$.
        
        Then, there are three points to prove:
        \begin{enumerate}
        \item \fbox{$s'.M(p)=\sigma'(id_p)(\texttt{sm})$}
        \item \fbox{$\omega=\sigma'(id_p)(\texttt{oaw})[j]$}
        \item \fbox{$\sum\limits_{t_i\in{}Pr(t,F)}pre(p,t_i)=\mathtt{vsots}$}
        \end{enumerate}

        Let us prove these three points:
        \begin{enumerate}
        \item \fbox{$s'.M(p)=\sigma'(id_p)(\texttt{sm})$}

          Appealing to Lemma~\ref{lem:fe-equal-marking},
          \qedbox{$s'.M(p)=\sigma'(id_p)(\texttt{sm})$.}
        \item \fbox{$\omega=\sigma'(id_p)(\texttt{oaw})[j]$}

          By construction, and as
          $pre(p,t)=(\omega,\mathtt{basic})$, we know that
          ${<}$\texttt{output\_arcs\_weights(j)}$\Rightarrow\omega{>}\in{}i_p$.

          By property of the stabilize relation and \InCsCompP:\\
          \qedbox{$\omega=\sigma'(id_p)(\texttt{oaw})[j]$.}
          
        \item
          \fbox{$\sum\limits_{t_i\in{}Pr(t,F)}pre(p,t_i)=\mathtt{vsots}$}
        \end{enumerate}
        
        Let us replace the left and right term of the equality by
        their full definition:

        \begin{frameb}
          \begin{tabular}{c}
            $\sum\limits_{t_i\in{}Pr(t,F)}
            \begin{cases}
              \omega~\mathtt{if}~pre(p,t_i)=(\omega,\mathtt{basic})\\
              0~otherwise
            \end{cases}$ \\
            $=$ \\
            $\sum\limits_{i=0}^{j-1}
            \begin{cases}
              \sigma'(id_p)(\texttt{oaw})[i]~\mathtt{if}~\sigma'(id_p)(\texttt{otf})[i].\\
              \hspace{19ex}\sigma'(id_p)(\texttt{oat})[i]=\mathtt{basic}\\
              0~otherwise\\
            \end{cases}$ \\
          \end{tabular}
        \end{frameb}

        Now, we must reason on the priority status of transition
        $t$ regarding the group of conflicting output transitions
        of $p$. There 2 cases:
          
        \begin{itemize}
        \item \textbf{CASE} $t$ is the top-priority transition in
          the group of conflicting output transitions of $p$:

          In that case, the set $Pr(t,F)$ is empty and, by
          construction, $j=0$. Thus, the goal is a tautology
          \qedbox{$0=0$.}
          
        \item \textbf{CASE} $t$ is not the top-priority transition
          in the group of conflicting output transitions of $p$:
        \end{itemize}
        In that case, we know that there is a least one element
        in $Pr(t,F)$ and the index $j>0$.

        Let us replace the sum terms in the goal by equivalent terms:
        \begin{frameb}
          \begin{tabular}{c}
            $\sum\limits_{t_i\in{}Pr_p}
            \begin{cases}
              \omega~\mathtt{if}~pre(p,t_i)=(\omega,\mathtt{basic})~\mathtt{and}~t_i\in{}F\\
              0~otherwise
            \end{cases}$ \\
            $=$ \\
            $\sum\limits_{i\in{}IPr_p}
            \begin{cases}
              \sigma'(id_p)(\texttt{oaw})[i]~\mathtt{if}~\sigma'(id_p)(\texttt{otf})[i]\\
              0~otherwise\\
            \end{cases}$ \\
          \end{tabular}
        \end{frameb}

        Let us define the set $Pr_p$ as
        \begin{center}
          $Pr_p=\{t_i~\vert~t_i\succ{}t\land\exists{}\omega~s.t.~pre(p,t_i)=(\omega,\mathtt{basic})\}$
        \end{center}
        and set $IPr_p$ as
        \begin{center}
          $IPr_p=\{i~\vert~i\in[0,j-1]\land\sigma'(id_p)(\texttt{oat})[i]=\texttt{basic}\}$
        \end{center}

        Let us define $f(t_i)$ as
        \begin{center}
          $f(t_i)=\begin{cases}
            \omega~\mathtt{if}~pre(p,t_i)=(\omega,\mathtt{basic})~\mathtt{and}~t_i\in{}F\\
            0~otherwise
          \end{cases}$
        \end{center}
        and $g(i)$ as              
        \begin{center}
          $g(i)=\begin{cases}
            \sigma'(id_p)(\texttt{oaw})[i]~\mathtt{if}~\sigma'(id_p)(\texttt{otf})[i]\\
            0~otherwise\\
          \end{cases}$
        \end{center}

        then, we must prove
        \fbox{$\sum\limits_{t_i\in{}Pr_p}f(t_i)=\sum\limits_{i\in{}IPr_p}g(i)$.}

        To prove the above equality, it is sufficient to prove that
        there exists a bijection $\beta$ from $Pr_p$ to $IPr_p$ such
        that for all $t_i\in{}Pr_p$, $f(t_i)=g(\beta(t_i))$. Let us
        use the function $\beta$ that takes a $t_i\in{}Pr_p$ and
        yields the index denoting the position of $t_i$ in the
        priority-ordered version of set $Pr_p$. Since we assumed that
        a total order existed over the conflicting output transitions
        of place $p$, then there exists a total ordering of the
        transitions of set $Pr_p$. By property of the \hilecop{}
        transformation function, we know that the index returned by
        the function $\beta$ belongs to the interval $[0,j-1]$ and
        verifies $\sigma'(id_p)(\texttt{oat})[i]=\texttt{basic}$.
        Given a $t_i\in{}Pr_p$, we must show
        \fbox{$f(t_i)=g(\beta(t_i))$.}

        Let us unfold terms $f(t_i)$ and $g(\beta(t_i))$ to their full definition:
        \begin{frameb}
          \begin{tabular}{c}
            $\begin{cases}
              \omega~\mathtt{if}~pre(p,t_i)=(\omega,\mathtt{basic})~\mathtt{and}~t_i\in{}F\\
              0~otherwise
            \end{cases}$ \\
            $=$ \\
            $\begin{cases}
              \sigma'(id_p)(\texttt{oaw})[\beta(t_i)]~\mathtt{if}~\sigma'(id_p)(\texttt{otf})[\beta(t_i)]\\
              0~otherwise\\
            \end{cases}$ \\
          \end{tabular}

          By construction, there exists an
          $id_{t_i}\in{}Comps(\Delta)$ such that
          $\gamma(t_i)=id_{t_i}$, and there exist $g_{t_i}$, $i_{t_i}$
          and $o_{t_i}$ such that
          $\mathtt{comp}(id_{t_i},\texttt{transition},g_{t_i},i_{t_i},o_{t_i})\in{}d.cs$.
          
          By property of the function $\beta$ and by construction, we
          can deduce that the element of index $\beta(t_i)$ of the
          \texttt{otf} output port of PCI $id_p$ is connected the
          \texttt{fired} output port of TCI $id_{t_i}$.
        \end{frameb}
      \end{itemize}
    \end{itemize}
    
  \item Assuming that
    $\prod\limits_{i=0}^{\Delta(id_t)("ian")-1}\sigma'(id_t)("pauths")[i]=\mathtt{true}$,
    let us show\\
    \fbox{$t\in{}Sens(s'.M-\sum\limits_{t_i\in{}Pr(t,fired)}pre(t_i))$.}
    
    By definition of $t\in{}Sens(s'.M-\sum\limits_{t_i\in{}Pr(t,fired)}pre(t_i))$:

    \begin{frameb}
      $\forall{}p\in{}P,\omega\in\mathbb{N}^{*},~$\\
      $\big((pre(p,t)=(\omega,\mathtt{basic})\lor{}pre(p,t)=(\omega,\mathtt{test}))\Rightarrow$
      ${}s'.M(p)-\sum\limits_{t_i\in{}Pr(t,fired)}pre(p,t_i)\ge\omega\big)$\\
      $\land{}$ $\big(pre(p,t)=(\omega,\mathtt{inhib})\Rightarrow$
      ${}s'.M(p)-\sum\limits_{t_i\in{}Pr(t,fired)}pre(p,t_i)<\omega\big)$
    \end{frameb}

    Given a $p\in{}P$ and an $\omega\in{}\mathbb{N}^{*}$, let us show 
    \begin{frameb}
      $\big((pre(p,t)=(\omega,\mathtt{basic})\lor{}pre(p,t)=(\omega,\mathtt{test}))\Rightarrow$
      ${}s'.M(p)-\sum\limits_{t_i\in{}Pr(t,fired)}pre(p,t_i)\ge\omega\big)$\\
      $\land{}$ $\big(pre(p,t)=(\omega,\mathtt{inhib})\Rightarrow$
      ${}s'.M(p)-\sum\limits_{t_i\in{}Pr(t,fired)}pre(p,t_i)<\omega\big)$
    \end{frameb}

    By construction, there exists an $id_p\in{}Comps(\Delta)$
    s.t. $\gamma(p)=id_p$. \exP{}
    
    There are three different cases:
    \begin{enumerate}
    \item Assuming that $pre(p,t)=(\omega,\mathtt{test})$, let us show
      \fbox{$s'.M(p)-\sum\limits_{t_i\in{}Pr(t,fired)}pre(p,t_i)\ge\omega$.}

      Then, assuming that the priority relation is well-defined, there
      exists no transition $t_i$ connected by a $\mathtt{basic}$ arc
      to $p$ that verifies $t_i\succ{}t$. This is because $t$ is
      connected to $p$ by a $\mathtt{test}$ arc; thus, $t$ is not in
      conflict with the other output transitions of $p$; thus, there
      is no relation of priority between $t$ and the output of $p$.

      Then, we can deduce that
      $\sum\limits_{t_i\in{}Pr(t,fired)}pre(p,t_i)=0$.

      Then, the new goal is $s'.M(p)\ge\omega$.

      Knowing that $t\in{}Firable(s')$, thus, $t\in{}Sens(s'.M)$,
      thus, we have \qedbox{$s'.M(p)\ge\omega$.}
      
    \item Assuming that $pre(p,t)=(\omega,\mathtt{inhib})$, let us
      show
      \fbox{$s'.M(p)-\sum\limits_{t_i\in{}Pr(t,fired)}pre(p,t_i)<\omega$.}

      Use the same strategy as above.
      
    \item Assuming that $pre(p,t)=(\omega,\mathtt{basic})$, let us
      show
      \fbox{$s'.M(p)-\sum\limits_{t_i\in{}Pr(t,fired)}pre(p,t_i)\ge\omega$.}

      Then, there are two cases:

      \begin{enumerate}
      \item \textbf{CASE} For all pair of transitions in
        $output_c(p)$, all conflicts are solved by mutual exclusion.

        Then, assuming that the priority relation is well-defined, it
        must not be defined over the set $output_c(t)$, and we know
        that $t\in{}output_c(p)$ since
        $pre(p,t)=(\omega,\mathtt{basic})$.

        Then, there exists no transition $t_i$ connected to $p$ by a
        $\mathtt{basic}$ arc that verifies $t_i\succ{}t$.

        Then, we can deduce $\sum\limits_{t_i\in{}Pr(t,fired)}pre(p,t_i)=0$.
        
        Then, the new goal is $s'.M(p)\ge\omega$.

        We know $t\in{}Firable(s')$, thus, $t\in{}Sens(s'.M)$, thus,
        \qedbox{$s'.M(p)\ge\omega$.}
        
      \item \textbf{CASE} The priority relation is a strict total
        order over the set $output_c(p)$.
        
        By construction, there exists $id_t\in{}Comps(\Delta)$
        s.t. $\gamma(t)=id_t$. \exT{}

        By construction, there exist
        $j\in{}[0,\vert{}input(t)\vert-1]$,
        $k\in[0,\vert{}output(t)\vert-1]$, and
        $id_{kj}\in{}Sigs(\Delta)$ s.t.
        ${<}\mathtt{priority\_authorizations(j)\Rightarrow{}id_{kj}}{>}\in{}ipm_t$
        and\\
        ${<}\mathtt{priority\_authorizations(k)\Rightarrow{}id_{kj}}{>}\in{}opm_p$.
        Let us take such an $j$, $k$ and $id_{kj}$.

        From
        $\prod\limits_{i=0}^{\Delta(id_t)("ian")-1}\sigma'(id_t)("pauths")[i]=\mathtt{true}$,
        we can deduce that for all $i\in{}[0,$ $\Delta(id_t)$
        $("ian")-1]$, $\sigma'(id_t)("pauths")[i]=\mathtt{true}$.

        By construction,
        ${<}\mathtt{input\_arcs\_number\Rightarrow}{}\vert{}input(t)\vert{>}\in{}gm_t$,
        and by property of the elaboration relation, we have
        $\Delta(id_t)("ian")=\vert{}input(t)\vert$. Then, from
        $j\in{}[0,\vert{}input(t)\vert-1]$, we can deduce
        $j\in[0,\Delta(id_t)("ian")-1]$. And, from
        $\forall{}i\in{}[0,\Delta(id_t)("ian")-1],~\sigma'(id_t)$ $("pauths")[i]=$ $\mathtt{true}$,
        we can deduce $\sigma'(id_t)("pauths")[j]=\mathtt{true}$.
        
        By property of the stabilize relation, \InCsCompP{} and
        \InCsCompT{}:
        \begin{equation}
          \label{eq:frd-eq-pauthsk-pauthsj}
          \sigma'(id_p)("pauths")[k]=\sigma'(id_{kj})=\sigma'(id_t)("pauths")[j]=\mathtt{true}
        \end{equation}

        By property of the stabilize relation and \InCsCompP:
        \begin{equation}
          \label{eq:frd-eq-pauthsk}
          \sigma'(id_p)("pauths")[k]=(\sigma'(id_p)("sm")\ge{}\mathtt{vsots}+\sigma'(id_p)("oaw")[k])
        \end{equation}

        Let us define the $\mathtt{vsots}$ term as follows:
        \begin{equation}
          \label{eq:frd-vsots-snd-case}
          \mathtt{vsots}=\sum\limits_{i=0}^{k-1}
          \begin{cases}
            \sigma'(id_p)("oaw")[i]~\mathtt{if}~\sigma'(id_p)("otf")[i].\\
            \hspace{19ex}\sigma'(id_p)("oat")[i]=\mathtt{basic}\\
            0~otherwise\\
          \end{cases}
        \end{equation}
        
        From \eqref{eq:frd-eq-pauthsk-pauthsj} and
        \eqref{eq:frd-eq-pauthsk}, we can deduce that
        $\sigma'(id_p)("sm")\ge{}\mathtt{vsots}+\sigma'(id_p)("oaw")[k]$.

        Then, there are three points to prove:
        \begin{enumerate}
        \item \fbox{$s'.M(p)=\sigma'(id_p)("sm")$}
        \item \fbox{$\omega=\sigma'(id_p)("oaw")[k]$}
        \item \fbox{$\sum\limits_{t_i\in{}Pr(t,fired)}pre(p,t_i)=\mathtt{vsots}$}
        \end{enumerate}
        
        See \ref{item:stab-comp-spc-fst-case} for the remainder of the
        proof.
        
      \end{enumerate}
      
    \end{enumerate}
  \end{enumerate}
  
\end{niproof}

%%%%%%%%%%%%%%%%%%%%%%%%%%%%%%%%%%%%%%%%%%%%%%%%%%%%%%%%
%%%%%%%%%% FALLING EDGE NOT EQUAL FIRED LEMMA %%%%%%%%%%
%%%%%%%%%%%%%%%%%%%%%%%%%%%%%%%%%%%%%%%%%%%%%%%%%%%%%%%%

\begin{lemma}[Falling Edge Equal Not Fired]
  \label{lem:fe-equal-not-fired}
  \fehyps{} then $\forall{}t,id_t~s.t.~\gamma(t)=id_t,$
  $~t\notin{}Fired(s')\Leftrightarrow\sigma'_t("fired")=\mathtt{false}$.
\end{lemma}

\begin{proof}
  Proving the above lemma is trivial by appealing to
  Lemma~\nameref{lem:fe-equal-fired} and by reasoning on
  contrapositives.
\end{proof}

%%% Local Variables:
%%% mode: latex
%%% TeX-master: "../../main"
%%% End:
