\subsection{Falling Edge and marking}
\label{sec:fe-marking}

%%%%%%%%%%%%%%%%%%%%%%%%%%%%%%%%%%%%%%%%%%%%%%%%%%%%%%
%%%%%%%%%% FALLING EDGE EQUAL MARKING LEMMA %%%%%%%%%%
%%%%%%%%%%%%%%%%%%%%%%%%%%%%%%%%%%%%%%%%%%%%%%%%%%%%%%

\begin{lemma}[Falling Edge Equal Marking]
  \label{lem:fe-equal-marking}
  \fehyps{} then
  $\forall{}p\in{}P,id_p\in{}Comps(\Delta)~s.t.~\gamma(p)=id_p,$
  $~s'.M(p)=\sigma'(id_p)("s\_marking")$.
\end{lemma}

\begin{proof}
  Given a $p\in{}P$ and an $id\in{}Comps(\Delta)$
  s.t. $\gamma(p)=id_p$, let us show\\
  \fbox{$s'.M(p)=\sigma'(id_p)("s\_marking")$.}

  \noindent{}By definition of
  $E_c,\tau\vdash{}sitpn,s\xrightarrow{\downarrow}s'$:
  \begin{equation}
    \label{eq:42}s.M(p)=s'.M(p)
  \end{equation}

  \noindent{}By property of the $\mathtt{Inject}_\downarrow$ relation,
  the \hvhdl{} falling edge relation, the stabilize relation and
  \InCsCompP:
  \begin{equation}
    \label{eq:43}\sigma'(id_p)("s\_marking")=\sigma(id_p)("s\_marking")
  \end{equation}

  \noindent{}Rewriting the goal with \eqref{eq:42} and \eqref{eq:43}:
  \fbox{$s.M(p)=\sigma(id_p)("s\_marking")$.}

  \noindent{}By definition of
  $\gamma,E_c,\tau\vdash{}s\stackrel{\downarrow}{\sim}\sigma$:
  \qedbox{$s.M(p)=\sigma(id_p)("s\_marking")$.}
  
\end{proof}

%%%%%%%%%%%%%%%%%%%%%%%%%%%%%%%%%%%%%%%%%%%%%%%%%%%%%%%%%%%%%%%
%%%%%%%%%% FALLING EDGE EQUAL OUTPUT TOKEN SUM LEMMA %%%%%%%%%%
%%%%%%%%%%%%%%%%%%%%%%%%%%%%%%%%%%%%%%%%%%%%%%%%%%%%%%%%%%%%%%%

\begin{lemma}[Falling Edge Equal Output Token Sum]
  \label{lem:fe-equal-ots}
  \fehyps{} then $\forall{}p,id_p~s.t.~\gamma(p)=id_p$,
  $\sum\limits_{t\in{}Fired(s')}pre(p,t)=\sigma'(id_p)("s\_output\_token\_sum")$.
\end{lemma}

\begin{proof}
  Given a $p\in{}P$ and an $id_p\in{}Comps(\Delta)$, let us show\\
  \fbox{$\sum\limits_{t\in{}Fired(s')}pre(p,t)=\sigma'(id_p)("s\_output\_token\_sum")$.}

  \exP{}
  
  \noindent{}By property of the stabilize relation and \InCsCompP{}:
  \begin{equation}
    \label{eq:16}
    \sigma'(id_p)("sots")=\sum\limits_{i=0}^{\Delta(id_p)("oan")-1}
    \begin{cases}
      \sigma'(id_p)("oaw")[i]~\mathtt{if}~(\sigma'(id_p)("otf")[i]~ \\
      \hspace{19.5ex}.~\sigma'(id_p)("oat")[i]=\mathtt{BASIC}) \\
      0~otherwise \\
    \end{cases}
  \end{equation}

  \noindent{}Rewriting the goal with \eqref{eq:16}:\\
  \begin{equation*}
    \fbox{$
      \sum\limits_{t\in{}Fired(s')}pre(p,t)=\sum\limits_{i=0}^{\Delta(id_p)("oan")-1}
      \begin{cases}
        \sigma'(id_p)("oaw")[i]~\mathtt{if}~(\sigma'(id_p)("otf")[i]~ \\
        \hspace{19.5ex}.~\sigma'(id_p)("oat")[i]=\mathtt{BASIC}) \\
        0~otherwise \\
      \end{cases}$}
  \end{equation*}

  \noindent{}Let us unfold the definition of the left sum term:\\
  \fbox{
    {\begin{tabular}{c}
       $\sum\limits_{t\in{}Fired(s')}
       \begin{cases}
         \omega~\mathtt{if}~pre(p,t)=(\omega,\mathtt{basic}) \\
         0~otherwise
       \end{cases}$ \\
       $=$ \\
       $\sum\limits_{i=0}^{\Delta(id_p)("oan")-1}
       \begin{cases}
         \sigma'(id_p)("oaw")[i]~\mathtt{if}~(\sigma'(id_p)("otf")[i]~ \\
         \hspace{19.5ex}.~\sigma'(id_p)("oat")[i]=\mathtt{BASIC}) \\
         0~otherwise \\
       \end{cases}$ \\
     \end{tabular}}
 }\\

 \noindent{}To ease the reading, let us define functions
 $f\in{}Fired(s')\rightarrow\mathbb{N}$ and
 $g\in[0,\vert{}output(p)\vert-1]\rightarrow\mathbb{N}$ s.t.
 $f(t)=\begin{cases}
   \omega~\mathtt{if}~pre(p,t)=(\omega,\mathtt{basic}) \\
   0~otherwise
 \end{cases}$ and $g(i)=\begin{cases}
   \sigma'(id_p)("oaw")[i]~\mathtt{if}~(\sigma'(id_p)("otf")[i]~ \\
   \hspace{19.5ex}.~\sigma'(id_p)("oat")[i]=\mathtt{BASIC}) \\
   0~otherwise \\
 \end{cases}$

 \noindent{}Then, the goal is: \fbox{$\sum\limits_{t\in{}Fired(s')}f(t)=\sum\limits_{i=0}^{\Delta(id_p)("oan")-1}g(i)$}\\

 \noindent{}Let us perform case analysis on $output(p)$; there are two cases:

 \begin{enumerate}
 \item $output(p)=\emptyset$:\\
   
   \noindent{}By construction,
   ${<}\mathtt{output\_arcs\_number\Rightarrow{}1}{>}\in{}gm_p$,
   ${<}\mathtt{output\_arcs\_types(0)\Rightarrow{}\mathtt{BASIC}}{>}\in{}ipm_p$,
   ${<}\mathtt{output\_transitions\_fired(0)\Rightarrow{}\mathtt{true}}{>}\in{}ipm_p$,
   and
   ${<}\mathtt{output\_arcs\_weights(0)\Rightarrow{}0}{>}\in{}ipm_p$.

   \noindent{}By property of the elaboration relation and \InCsCompP:
   \begin{equation}
     \label{eq:25}\Delta(id_p)("oan")=1
   \end{equation}

   \noindent{}By property of the stabilize relation and \InCsCompP:
   \begin{eqnarray}
     \label{eq:26}\sigma'(id_p)("oat")[0]&=&\mathtt{BASIC}\\
     \label{eq:27}\sigma'(id_p)("otf")[0]&=&\mathtt{true}\\
     \label{eq:28}\sigma'(id_p)("oaw")[0]&=&0
   \end{eqnarray}

   By property of $output(p)=\emptyset$:
   \begin{equation}\label{eq:29}
     \sum\limits_{t\in{}Fired(s')}
     \begin{cases}
       \omega~\mathtt{if}~pre(p,t)=(\omega,\mathtt{basic}) \\
       0~otherwise
     \end{cases}=0
   \end{equation}

   \noindent{}Rewriting the goal with \eqref{eq:25}, \eqref{eq:26},
   \eqref{eq:27}, \eqref{eq:28} and \eqref{eq:29}, \qedbox{tautology.}
   
 \item $output(p)\neq\emptyset$:\\

   \noindent{}By construction,
   ${<}\mathtt{output\_arcs\_number\Rightarrow{}}\vert{}output(p)\vert{>}\in{}gm_p$,
   and by property of the elaboration relation:
   \begin{equation}
     \label{eq:30}\Delta(id_p)("oan")=\vert{}output(p)\vert
   \end{equation}
   
   Rewriting the goal with \eqref{eq:30}:
   \fbox{$\sum\limits_{t\in{}Fired(s')}f(t)=\sum\limits_{i=0}^{\vert{}output(p)\vert-1}g(i)$.}

   \noindent{}Let us reason by induction on the right sum term of the goal.

   \begin{itemize}
   \item \textbf{BASE CASE}:\\

     \noindent{}In that case, $0>\vert{}output\vert-1$ and
     $\sum\limits_{i=0}^{\vert{}output(p)\vert-1}g(i)=0$.

     \noindent{}As $0>\vert{}output\vert-1$, then
     $\vert{}output(p)\vert=0$, thus \qedbox{contradicting
       $output(p)\neq\emptyset$.}

   \item \textbf{INDUCTION CASE}:\\

     \noindent{}In that case, $0\le\vert{}output(p)\vert-1$.
     
     \begin{ih}
       $\forall{}F\subseteq{}Fired(s'),$
       $g(0)+\sum\limits_{t\in{}F}f(t)=g(0)+\sum\limits_{i=1}^{\vert{}output(p)\vert-1}g(i)$
     \end{ih}

     \fbox{$\sum\limits_{t\in{}Fired(s')}f(t)=g(0)+\sum\limits_{i=1}^{\vert{}output(p)\vert-1}g(i)$}

     By definition of $g$:
     \begin{equation}
       \label{eq:31}g(0)=\begin{cases}
         \sigma'(id_p)("oaw")[0]~\mathtt{if}~(\sigma'(id_p)("otf")[0] \\
         \hspace{19.5ex}.~\sigma'(id_p)("oat")[0]=\mathtt{BASIC}) \\
         0~otherwise \\
       \end{cases}
     \end{equation}

     \noindent{}Let us perform case analysis on the value of
     $\sigma'(id_p)("otf")[0]~.~\sigma'(id_p)("oat")[0]=\mathtt{BASIC}$; there are two cases:
     \begin{enumerate}
     \item $(\sigma'(id_p)("otf")[0]~.~\sigma'(id_p)("oat")[0]=\mathtt{BASIC})=\mathtt{false}$:\\
       In that case, $g(0)=0$, and then we can apply the induction
       hypothesis with $F=Fired(s')$ to solve the goal:
       \qedbox{$\sum\limits_{t\in{}Fired(s')}f(t)=\sum\limits_{i=1}^{\vert{}output(p)\vert-1}g(i)$.}

     \item $(\sigma'(id_p)("otf")[0]~.~\sigma'(id_p)("oat")[0]=\mathtt{BASIC})=\mathtt{true}$:\\
       In that case, $g(0)=\sigma'(id_p)("oaw")[0]$,
       $\sigma'(id_p)("otf")[0]=\mathtt{true}$ and\\
       $\sigma'(id_p)("oat")[0]=\mathtt{BASIC}$.

       \noindent{}By construction, there exist a $t\in{}output(t)$,
       $id_t\in{}Comps(\Delta)$ s.t. $\gamma(t)=id_t$. Let us take
       such a $t\in{}output(p)$.

       \exT{}

       \noindent{}As $t\in{}output(p)$, there exist
       $\omega\in\mathbb{N}^{*}$ and
       $a\in\{\mathtt{BASIC},\mathtt{TEST},\mathtt{INHIB}\}$
       s.t. $pre(p,t)=(\omega,a)$. Let us take an $\omega$ and $a$
       s.t. $pre(p,t)=(\omega,a)$.

       \noindent{}By construction,
       ${<}\mathtt{output\_arcs\_types(0)\Rightarrow{}}a{>}\in{}ipm_p$,\\
       ${<}\mathtt{output\_arcs\_weights(0)\Rightarrow{}}\omega{>}\in{}ipm_p$,
       and there exists $id_{ft}\in{}Sigs(\Delta)$ s.t.
       ${<}\mathtt{fired\Rightarrow}id_{ft}{>}\in{}opm_t$ and
       ${<}\mathtt{output\_transitions\_fired(0)\Rightarrow{}id_{ft}}{>}\in{}ipm_p$

       \noindent{}By property of the stabilize relation,
       $\sigma'(id_p)("oat")[0]=\mathtt{BASIC}$ and\\
       ${<}\mathtt{output\_arcs\_types(0)\Rightarrow{}a}{>}\in{}ipm_p$:
       \begin{equation}
         \label{eq:32}pre(p,t)=(\omega,\mathtt{basic})
       \end{equation}

       \noindent{}By property of the stabilize relation,
       ${<}\mathtt{fired\Rightarrow{}id_{ft}}{>}\in{}opm_t$,\\
       ${<}\mathtt{output\_transitions\_fired(0)\Rightarrow{}id_{ft}}{>}\in{}ipm_p$
       and $\sigma'(id_p)("otf")[0]=\mathtt{true}$:
       \begin{equation}
         \label{eq:33}\sigma'(id_t)("fired")=\mathtt{true}
       \end{equation}

       \noindent{}Appealing to Lemma~\ref{lem:fe-equal-fired}, we know
       $t\in{}Fired(s')$.

       \noindent{}As $t\in{}Fired(s')$, we can rewrite the left sum
       term of the goal as follows:\\
       \fbox{$f(t)+\sum\limits_{t'\in{}Fired(s')\setminus\{t\}}f(t')=g(0)+\sum\limits_{i=1}^{\vert{}output(p)\vert-1}g(i)$}

       \noindent{}We know that $g(0)=\sigma'(id_p)("oaw")[0]$, and by
       property of the stabilize relation and
       ${<}\mathtt{output\_arcs\_weights(0)\Rightarrow}\omega{>}\in{}ipm_p$:
       \begin{equation}
         \label{eq:21}\sigma'(id_p)("oaw")[0]=\omega
       \end{equation}

       Rewriting the goal with \eqref{eq:21}:\\
       \fbox{$f(t)+\sum\limits_{t'\in{}Fired(s')\setminus\{t\}}f(t')=\omega+\sum\limits_{i=1}^{\vert{}output(p)\vert-1}g(i)$}

       By definition of $f$, and as $pre(p,t)=(\omega,\mathtt{basic})$, then $f(t)=\omega$; thus, rewriting the goal:\\
       \fbox{$\omega+\sum\limits_{t'\in{}Fired(s')\setminus\{t\}}f(t')=\omega+\sum\limits_{i=1}^{\vert{}output(p)\vert-1}g(i)$}

       Then, knowing that $g(0)=\omega$, we can apply the induction
       hypothesis with $F=Fired(s')\setminus\{t\}$:
       \qedbox{$g(0)+\sum\limits_{t'\in{}Fired(s')\setminus\{t\}}f(t')=g(0)+\sum\limits_{i=1}^{\vert{}output(p)\vert-1}g(i)$.}
     \end{enumerate}
     
   \end{itemize}
   

 \end{enumerate}
 
\end{proof}


%%%%%%%%%%%%%%%%%%%%%%%%%%%%%%%%%%%%%%%%%%%%%%%%%%%%%%%%%%%%%%
%%%%%%%%%% FALLING EDGE EQUAL INPUT TOKEN SUM LEMMA %%%%%%%%%%
%%%%%%%%%%%%%%%%%%%%%%%%%%%%%%%%%%%%%%%%%%%%%%%%%%%%%%%%%%%%%%

\begin{lemma}[Falling Edge Equal Input Token Sum]
  \label{lem:fe-equal-its}
  \fehyps{} then $\forall{}p,id_p~s.t.~\gamma(p)=id_p$,
  $~\sum\limits_{t\in{}Fired(s')}post(t,p)=\sigma'_p("s\_input\_token\_sum")$.
\end{lemma}

\begin{proof}
  Given a $p\in{}P$ and an $id_p\in{}Comps(\Delta)$, let us show\\
  \fbox{$\sum\limits_{t\in{}Fired(s')}post(t,p)=\sigma'(id_p)("s\_input\_token\_sum")$.}\\

  \exP{}
  
  \noindent{}By property of the stabilize relation and \InCsCompP{}:
  \begin{equation}
    \label{eq:34}
    \sigma'(id_p)("sits")=\sum\limits_{i=0}^{\Delta(id_p)("ian")-1}
    \begin{cases}
      \sigma'(id_p)("iaw")[i]~\mathtt{if}~\sigma'(id_p)("itf")[i]\\
      0~otherwise \\
    \end{cases}
  \end{equation}

  \noindent{}Rewriting the goal with \eqref{eq:34}:\\
  \begin{equation*}
    \fbox{$
      \sum\limits_{t\in{}Fired(s')}post(t,p)=\sum\limits_{i=0}^{\Delta(id_p)("ian")-1}
      \begin{cases}
        \sigma'(id_p)("iaw")[i]~\mathtt{if}~\sigma'(id_p)("otf")[i]\\
        0~otherwise \\
      \end{cases}$}
  \end{equation*}

  \noindent{}Let us unfold the definition of the left sum term:\\
  \fbox{
    {\begin{tabular}{c}
       $\sum\limits_{t\in{}Fired(s')}
       \begin{cases}
         \omega~\mathtt{if}~post(t,p)=\omega \\
         0~otherwise
       \end{cases}$ \\
       $=$ \\
       $\sum\limits_{i=0}^{\Delta(id_p)("ian")-1}
       \begin{cases}
         \sigma'(id_p)("iaw")[i]~\mathtt{if}~\sigma'(id_p)("itf")[i]\\
         0~otherwise \\
       \end{cases}$ \\
     \end{tabular}}
 }\\

 \noindent{}Let us perform case analysis on $input(p)$; there are two cases:

 \begin{enumerate}
 \item $input(p)=\emptyset$:\\
   
   \noindent{}By construction,
   ${<}\mathtt{input\_arcs\_number\Rightarrow{}1}{>}\in{}gm_p$,
   ${<}\mathtt{input\_transitions\_fired(0)\Rightarrow{}\mathtt{true}}{>}\in{}ipm_p$,
   and
   ${<}\mathtt{input\_arcs\_weights(0)\Rightarrow{}0}{>}\in{}ipm_p$.

   \noindent{}By property of the elaboration relation and \InCsCompP:
   \begin{equation}
     \label{eq:35}\Delta(id_p)("ian")=1
   \end{equation}

   \noindent{}By property of the stabilize relation and \InCsCompP:
   \begin{eqnarray}
     \label{eq:36}\sigma'(id_p)("itf")[0]&=&\mathtt{true}\\
     \label{eq:37}\sigma'(id_p)("iaw")[0]&=&0
   \end{eqnarray}

   By property of $input(p)=\emptyset$:
   \begin{equation}\label{eq:23}
     \sum\limits_{t\in{}Fired(s')}
     \begin{cases}
       \omega~\mathtt{if}~post(t,p)=\omega \\
       0~otherwise
     \end{cases}=0
   \end{equation}

   \noindent{}Rewriting the goal with \eqref{eq:35}, \eqref{eq:36},
   \eqref{eq:37}, and \eqref{eq:23}, and simplifying the goal,
   \qedbox{tautology.}
   
 \item $input(p)\neq\emptyset$:\\

   \noindent{}By construction,
   ${<}\mathtt{input\_arcs\_number\Rightarrow{}}\vert{}input(p)\vert{>}\in{}gm_p$,
   and by property of the elaboration relation:
   \begin{equation}
     \label{eq:24}\Delta(id_p)("ian")=\vert{}input(p)\vert
   \end{equation}

   \noindent{}To ease the reading, let us define functions
   $f\in{}Fired(s')\rightarrow\mathbb{N}$ and
   $g\in[0,\vert{}input(p)\vert-1]\rightarrow\mathbb{N}$ s.t.
   $f(t)=\begin{cases}
     \omega~\mathtt{if}~post(t,p)=\omega \\
     0~otherwise
   \end{cases}$ and \\ $g(i)=\begin{cases}
     \sigma'(id_p)("iaw")[i]~\mathtt{if}~\sigma'(id_p)("itf")[i]\\
     0~otherwise \\
   \end{cases}$

   \noindent{}Then, the goal is: \fbox{$\sum\limits_{t\in{}Fired(s')}f(t)=\sum\limits_{i=0}^{\Delta(id_p)("ian")-1}g(i)$}\\
   
   Rewriting the goal with \eqref{eq:24}:
   \fbox{$\sum\limits_{t\in{}Fired(s')}f(t)=\sum\limits_{i=0}^{\vert{}input(p)\vert-1}g(i)$.}

   \noindent{}Let us reason by induction on the right sum term of the goal.

   \begin{itemize}
   \item \textbf{BASE CASE}:\\
     \noindent{}In that case, $0>\vert{}input(p)\vert-1$ and
     $\sum\limits_{i=0}^{\vert{}input(p)\vert-1}g(i)=0$.

     \noindent{}As $0>\vert{}input(p)\vert-1$, then
     $\vert{}input(p)\vert=0$, thus \qedbox{contradicting
       $input(p)\neq\emptyset$.}

   \item \textbf{INDUCTION CASE}:\\
     \noindent{}In that case, $0\le\vert{}input(p)\vert-1$.
     
     \begin{ih}
       $\forall{}F\subseteq{}Fired(s'),~$
       $g(0)+\sum\limits_{t\in{}F}f(t)=g(0)+\sum\limits_{i=1}^{\vert{}input(p)\vert-1}g(i)$
     \end{ih}

     \fbox{$\sum\limits_{t\in{}Fired(s')}f(t)=g(0)+\sum\limits_{i=1}^{\vert{}input(p)\vert-1}g(i)$}\\

     By definition of $g$:
     \begin{equation}
       \label{eq:38}g(0)=\begin{cases}
         \sigma'(id_p)("iaw")[0]~\mathtt{if}~\sigma'(id_p)("itf")[0]\\
         0~otherwise \\
       \end{cases}
     \end{equation}

     \noindent{}Let us perform case analysis on the value of
     $\sigma'(id_p)("itf")[0]$; there are two cases:
     \begin{enumerate}
     \item $\sigma'(id_p)("itf")[0]=\mathtt{false}$:\\
       In that case, $g(0)=0$, and then we can apply the induction
       hypothesis with $F=Fired(s')$ to solve the goal:
       \qedbox{$\sum\limits_{t\in{}Fired(s')}f(t)=\sum\limits_{i=1}^{\vert{}input(p)\vert-1}g(i)$.}

     \item $\sigma'(id_p)("itf")[0]=\mathtt{true}$:\\
       In that case, $g(0)=\sigma'(id_p)("iaw")[0]$ and
       $\sigma'(id_p)("itf")[0]=\mathtt{true}$ .

       \noindent{}By construction, there exist a $t\in{}input(t)$,
       $id_t\in{}Comps(\Delta)$ s.t. $\gamma(t)=id_t$. Let us take
       such a $t\in{}input(p)$.

       \exT{}

       \noindent{}As $t\in{}input(p)$, there exist
       $\omega\in\mathbb{N}^{*}$ s.t. $post(t,p)=\omega$. Let us take
       an $\omega$ s.t. $post(t,p)=\omega$.

       \noindent{}By construction,
       ${<}\mathtt{input\_arcs\_weights(0)\Rightarrow{}}\omega{>}\in{}ipm_p$,
       and there exists $id_{ft}\in{}Sigs(\Delta)$ s.t.
       ${<}\mathtt{fired\Rightarrow}id_{ft}{>}\in{}opm_t$ and
       ${<}\mathtt{input\_transitions\_fired(0)\Rightarrow{}id_{ft}}{>}\in{}ipm_p$

       \noindent{}By property of the stabilize relation and
       ${<}\mathtt{input\_arcs\_types(0)\Rightarrow{}a}{>}\in{}ipm_p$:
       \begin{equation}
         \label{eq:39}post(t,p)=\omega
       \end{equation}

       \noindent{}By property of the stabilize relation,
       ${<}\mathtt{fired\Rightarrow{}id_{ft}}{>}\in{}opm_t$,\\
       ${<}\mathtt{input\_transitions\_fired(0)\Rightarrow{}id_{ft}}{>}\in{}ipm_p$
       and $\sigma'(id_p)("itf")[0]=\mathtt{true}$:
       \begin{equation}
         \label{eq:40}\sigma'(id_t)("fired")=\mathtt{true}
       \end{equation}

       \noindent{}Appealing to Lemma~\ref{lem:fe-equal-fired} and
       \eqref{eq:40}, we know $t\in{}Fired(s')$.

       \noindent{}As $t\in{}Fired(s')$, we can rewrite the left sum
       term of the goal as follows:\\
       \fbox{$f(t)+\sum\limits_{t'\in{}Fired(s')\setminus\{t\}}f(t')=g(0)+\sum\limits_{i=1}^{\vert{}input(p)\vert-1}g(i)$}

       \noindent{}We know that $g(0)=\sigma'(id_p)("iaw")[0]$, and by
       property of the stabilize relation and
       ${<}\mathtt{input\_arcs\_weights(0)\Rightarrow}\omega{>}\in{}ipm_p$:
       \begin{equation}
         \label{eq:41}\sigma'(id_p)("iaw")[0]=\omega
       \end{equation}

       Rewriting the goal with \eqref{eq:41}:\\
       \fbox{$f(t)+\sum\limits_{t'\in{}Fired(s')\setminus\{t\}}f(t')=\omega+\sum\limits_{i=1}^{\vert{}input(p)\vert-1}g(i)$}

       By definition of $f$, and as $post(t,p)=\omega$, then $f(t)=\omega$; thus, rewriting the goal:\\
       \fbox{$\omega+\sum\limits_{t'\in{}Fired(s')\setminus\{t\}}f(t')=\omega+\sum\limits_{i=1}^{\vert{}input(p)\vert-1}g(i)$}

       Then, knowing that $g(0)=\omega$, we can apply the induction
       hypothesis with $F=Fired(s')\setminus\{t\}$:
       \qedbox{$g(0)+\sum\limits_{t'\in{}Fired(s')\setminus\{t\}}f(t')=g(0)+\sum\limits_{i=1}^{\vert{}input(p)\vert-1}g(i)$.}
     \end{enumerate}
     
   \end{itemize}
   
 \end{enumerate}
 
\end{proof}


%%%%%%%%%%%%%%%%%%%%%%%%%%%%%%%%%%%%%%%%%%%%%%%%%%%%%%
%%%%%%%%%% FALLING EDGE EQUAL TIME COUNTERS %%%%%%%%%%
%%%%%%%%%%%%%%%%%%%%%%%%%%%%%%%%%%%%%%%%%%%%%%%%%%%%%%

\subsection{Falling edge and time counters}
\label{sec:fe-equal-tc}

\begin{lemma}[Falling Edge Equal Time Counters]
  \label{lem:fe-equal-tc}
  \fehyps{} then $\forall{}t\in{}T_i,id_t\in{}Comps(\Delta)~s.t.~\gamma(t)=id_t,$\\
  $\big(upper(I_s(t))=\infty\land{}s'.I(t)\le{}lower(I_s(t))\Rightarrow{}s'.I(t)=\sigma'(id_t)("s\_time\_counter")\big)$\\
  $\land\big(upper(I_s(t))=\infty\land{}s'.I(t)>{}lower(I_s(t))\Rightarrow{}\sigma'(id_t)("s\_time\_counter")=lower(I_s(t))\big)$\\
  $\land\big(upper(I_s(t))\neq\infty\land{}s'.I(t)>{}upper(I_s(t))\Rightarrow{}\sigma'(id_t)("s\_time\_counter")=upper(I_s(t))\big)$\\
  $\land\big(upper(I_s(t))\neq\infty\land{}s'.I(t)\le{}upper(I_s(t))\Rightarrow{}s'.I(t)=\sigma'(id_t)("s\_time\_counter")\big)$.
\end{lemma}

\begin{proof}
  Given a $t\in{}T_i$ and an $id_t\in{}Comps(\Delta)$ s.t. $\gamma(t)=id_t$, let us show\\
  \noindent\fbox{\parbox{\lwidth}{$\big(upper(I_s(t))=\infty\land{}s'.I(t)\le{}lower(I_s(t))\Rightarrow$
      $s'.I(t)=\sigma'(id_t)("s\_time\_counter")\big)$\\
      $\land\big(upper(I_s(t))=\infty\land{}s'.I(t)>{}lower(I_s(t))\Rightarrow$
      $\sigma'(id_t)("s\_time\_counter")=lower(I_s(t))\big)$\\
      $\land\big(upper(I_s(t))\neq\infty\land{}s'.I(t)>{}upper(I_s(t))\Rightarrow$
      $\sigma'(id_t)("s\_time\_counter")=upper(I_s(t))\big)$\\
      $\land\big(upper(I_s(t))\neq\infty\land{}s'.I(t)\le{}upper(I_s(t))\Rightarrow$
      $s'.I(t)=\sigma'(id_t)("s\_time\_counter")\big)$}}\\

  \exT\\

  \noindent{}By property of the elaboration,
  $\mathtt{Inject}_\downarrow$, \hvhdl{} rising edge and stabilize
  relations, and \InCsCompT:
  \begin{equation}
    \begin{split}
      \sigma(id_t)("se")=\mathtt{true}\land\Delta(id_t)("tt")\neq\mathtt{NOT\_TEMPORAL}
      \land\sigma(id_t)("srtc")=\mathtt{false}\\
      \land\sigma(id_t)("stc")<\Delta(id_t)("mtc")\Rightarrow
      \sigma'(id_t)("stc")=\sigma(id_t)("stc")+1
    \end{split}
    \label{eq:etnrlt}
  \end{equation}

  \begin{equation}
    \begin{split}
      \sigma(id_t)("se")=\mathtt{true}\land\Delta(id_t)("tt")\neq\mathtt{NOT\_TEMPORAL}
      \land\sigma(id_t)("srtc")=\mathtt{false}\\
      \land\sigma(id_t)("stc")\ge\Delta(id_t)("mtc")\Rightarrow
      \sigma'(id_t)("stc")=\sigma(id_t)("stc")
    \end{split}
    \label{eq:etnrge}
  \end{equation}

  \begin{equation}
    \begin{split}
      \sigma(id_t)("se")=\mathtt{true}\land\Delta(id_t)("tt")\neq\mathtt{NOT\_TEMPORAL}\\
      \land\sigma(id_t)("srtc")=\mathtt{true}\Rightarrow
      \sigma'(id_t)("stc")=1
    \end{split}
    \label{eq:etr}
  \end{equation}

  \begin{equation}
    \begin{split}
      \sigma(id_t)("se")=\mathtt{false}\lor\Delta(id_t)("tt")=\mathtt{NOT\_TEMPORAL}\Rightarrow
      \sigma'(id_t)("stc")=0
    \end{split}
    \label{eq:ne-or-nt}
  \end{equation}
  
  \noindent{}Then, there are 4 points to show:

  \begin{enumerate}
  \item\label{it:re-eq-tc-fst}
    \fbox{$upper(I_s(t))=\infty\land{}s'.I(t)\le{}lower(I_s(t))\Rightarrow{}s'.I(t)=\sigma'(id_t)("s\_time\_counter")$}\\
    
    \noindent{}Assuming $upper(I_s(t))=\infty$ and
    $s'.I(t)\le{}lower(I_s(t))$, let us show\\
    \fbox{$s'.I(t)=\sigma'(id_t)("s\_time\_counter")$.}

    Case analysis on $t\in{}Sens(s.M)$; there are two cases:

    \begin{enumerate}
    \item $t\notin{}Sens(s.M)$:\\
      By definition of \upSim, we have
      \inlineequation[eq:50]{\sigma(id_t)("se")=\mathtt{false}}.

      Appealing to \eqref{eq:ne-or-nt} and \eqref{eq:50}, we have
      \inlineequation[eq:45]{\sigma'(id_t)("stc")=0}.

      By definition of \dwSitpn, we have
      \inlineequation[eq:46]{s'.I(t)=0}.

      Rewriting the goal with \eqref{eq:45} and \eqref{eq:46}:
      \qedbox{tautology.}
      
    \item $t\in{}Sens(s.M)$:\\
      By definition of \upSim, we have
      \inlineequation[eq:51]{\sigma(id_t)("se")=\mathtt{true}}.

      By construction, and as $upper(I_s(t))=\infty$,
      ${<}\mathtt{transition\_type\Rightarrow{}TEMP\_A\_INF}{>}\in{}gm_t$. By
      property of the elaboration relation, we have
      \inlineequation[eq:52]{\Delta(id_t)("tt")=\mathtt{TEMP\_A\_INF}}.

      Case analysis on $s.reset_t(t)$; there are two cases:
      \begin{enumerate}
      \item $s.reset_t(t)=\mathtt{true}$:\\
        By definition of \upSim,
        \inlineequation[eq:53]{\sigma(id_t)("srtc")=\mathtt{true}}.

        Appealing to \eqref{eq:etr}, \eqref{eq:51}, \eqref{eq:52} and
        \eqref{eq:53}, we have
        \inlineequation[eq:54]{\sigma'(id_t)("stc")=1}.

        By definition of \dwSitpn, we have
        \inlineequation[eq:55]{s'.I(t)=1}.

        Rewriting the goal with \eqref{eq:54} and \eqref{eq:55}:
        \qedbox{tautology.}

      \item $s.reset_t(t)=\mathtt{false}$:\\
        By definition of \upSim, we have
        \inlineequation[eq:56]{\sigma(id_t)("srtc")=\mathtt{false}}.

        As $upper(I_s(t))=\infty$, there exists an
        $a\in\mathbb{N}^{*}$ s.t. $I_s(t)=[a,\infty]$. Let us take
        such an $a\in\mathbb{N}^{*}$. By construction,
        ${<}\mathtt{maximal\_time\_counter\Rightarrow}~a{>}\in{}gm_t$,
        and by property of the elaboration relation, we have
        \inlineequation[eq:57]{\Delta(id_t)("mtc")=a}.

        By definition of \dwSitpn, and knowing that $t\in{}Sens(s.M)$,
        $s.reset_t(t)=\mathtt{false}$ and $upper(I_s(t))=\infty$:
        \begin{equation}
          \label{eq:58}s'.I(t)=s.I(t)+1
        \end{equation}

        Rewriting the goal with \eqref{eq:58}:
        \fbox{$s.I(t)+1=\sigma'(id_t)("stc")$.}
        
        We assumed that $s'.I(t)\le{}lower(I_s(t))$, and as
        $s'.I(t)=s.I(t)+1$, then $s.I(t)+1\le{}lower(I_s(t))$, then
        $s.I(t)<lower(I_s(t))$, then $s.I(t)<a$ since
        $a=lower(I_s(t))$.

        \noindent{}By definition of
        $\gamma,E_c,\tau\vdash{}s\stackrel{\uparrow}{\sim}\sigma$, and
        knowing that $s.I(t)<lower(I_s(t))$ and $upper(I_s(t))=\infty$:
        \begin{equation}
          \label{eq:59}s.I(t)=\sigma(id_t)("stc")
        \end{equation}

        Appealing to \eqref{eq:57}, \eqref{eq:59} and $s.I(t)<a$:
        \begin{equation}
          \label{eq:60}\sigma(id_t)("stc")<\Delta(id_t)("mtc")
        \end{equation}
        
        Appealing to \eqref{eq:etnrlt}, \eqref{eq:60}, \eqref{eq:56} and
        \eqref{eq:51}:
        \begin{equation}
          \label{eq:61}\sigma'(id_t)("stc")=\sigma(id_t)("stc")+1
        \end{equation}

        Rewriting the goal with \eqref{eq:61} and \eqref{eq:59}: \qedbox{tautology.}        
      \end{enumerate}
    \end{enumerate}
    
  \item
    \fbox{$upper(I_s(t))=\infty\land{}s'.I(t)>{}lower(I_s(t))\Rightarrow$
      $\sigma'(id_t)("s\_time\_counter")=lower(I_s(t)$.}

    Assuming that $upper(I_s(t))=\infty$ and
    $s'.I(t)>{}lower(I_s(t))$, let us show\\
    \fbox{$\sigma'(id_t)("s\_time\_counter")=lower(I_s(t))$.}

    As $upper(I_s(t))=\infty$, there exists an $a\in\mathbb{N}^{*}$
    s.t. $I_s(t)=[a,\infty]$. Let us take such an
    $a\in\mathbb{N}^{*}$. By construction,
    ${<}\mathtt{maximal\_time\_counter\Rightarrow}a{>}\in{}gm_t$, and
    ${<}\mathtt{transition\_type\Rightarrow{}TEMP\_A\_INF}{>}\in{}gm_t$
    by property of the elaboration relation:
    \begin{eqnarray}
      \label{eq:62}\Delta(id_t)("mtc")&=&a\\
      \label{eq:63}\Delta(id_t)("tt")&=&\mathtt{TEMP\_A\_INF}
    \end{eqnarray}
    
    Case analysis on $t\in{}Sens(s.M)$:
    \begin{enumerate}
    \item $t\notin{}Sens(s.M)$:\\
      \noindent{}By definition of
      $E_c,\tau\vdash{}s\xrightarrow{\downarrow}s'$, and knowing that
      $t\in{}Sens(s.M)$, then $s'.I(t)=0$. Since
      $lower(I_s(t))\in\mathbb{N}^{*}$, then $lower(I_s(t))>0$.
      
      \qedbox{Contradicts $s'.I(t)>lower(I_s(t))$.}
      
    \item $t\in{}Sens(s.M)$:\\
      By definition of
      $\gamma,E_c,\tau\vdash{}s\stackrel{\uparrow}{\sim}\sigma$ and
      $t\in{}Sens(s.M)$:
      \begin{equation}
        \label{eq:64}\sigma(id_t)("se")=\mathtt{true}
      \end{equation}

      Case analysis on $s.reset_t(t)$; there are two cases:
      \begin{enumerate}
      \item $s.reset_t(t)=\mathtt{true}$:\\
        By definition of
        $E_c,\tau\vdash{}s\xrightarrow{\downarrow}s'$: $s'.I(t)=1$.

        \noindent{}We assumed that $s'.I(t)>lower(I_s(t))$, then
        $1>lower(I_s(t))$.

        \qedbox{Contradicts $lower(I_s(t))>0$.}

      \item $s.reset_t(t)=\mathtt{false}$:\\
        \noindent{}By property of
        $\gamma,E_c,\tau\vdash{}s\stackrel{\uparrow}{\sim}\sigma$ and
        $s.reset_t(t)=\mathtt{false}$:
        \begin{equation}
          \sigma(id_t)("srtc")=\mathtt{false}\label{eq:68}
        \end{equation}

        By definition of
        $E_c,\tau\vdash{}s\xrightarrow{\downarrow}s'$, and knowing
        that $s'.I(t)>lower(I_s(t))$:
        \begin{equation}
          \begin{split}
            s'.I(t)=s.I(t)+1&\Rightarrow{}s.I(t)+1>lower(I_s(t))\\
            & \Rightarrow{}s.I(t)\ge{}lower(I_s(t))\\
          \end{split}
        \end{equation}

        Case analysis on $s.I(t)\ge{}lower(I_s(t))$:
        \begin{enumerate}
        \item $s.I(t)>lower(I_s(t))$: \fbox{$\sigma'(id_t)("stc")=lower(I_s(t))$.}\\
          By definition of
          $\gamma,E_c,\tau\vdash{}s\stackrel{\uparrow}{\sim}\sigma$:
          \begin{equation}
            \label{eq:69}\sigma(id_t)("stc")=lower(I_s(t))
          \end{equation}

          Appealing to \eqref{eq:etnrge}:
          \begin{equation}
            \label{eq:70}\sigma'(id_t)("stc")=\sigma(id_t)("stc")
          \end{equation}

          Rewriting the goal with \eqref{eq:69} and \eqref{eq:70}:
          \qedbox{tautology.}
          
        \item $s.I(t)=lower(I_s(t))$: \fbox{$\sigma'(id_t)("stc")=lower(I_s(t))$.}\\
          By definition of
          $\gamma,E_c,\tau\vdash{}s\stackrel{\uparrow}{\sim}\sigma$:
          \begin{equation}
            \label{eq:65}s.I(t)=\sigma(id_t)("stc")
          \end{equation}

          Appealing to \eqref{eq:etnrge}:
          \begin{equation}
            \label{eq:66}\sigma'(id_t)("stc")=\sigma(id_t)("stc")
          \end{equation}

          Rewriting the goal with \eqref{eq:66}, \eqref{eq:65} and
          $s.I(t)=lower(I_s(t))$: \qedbox{tautology.}
        \end{enumerate}
      \end{enumerate}
    \end{enumerate}
  \item
    \fbox{$upper(I_s(t))\neq\infty\land{}s'.I(t)>{}upper(I_s(t))\Rightarrow$
      $\sigma'(id_t)("s\_time\_counter")=upper(I_s(t))$.}

    Assuming that $upper(I_s(t))\neq\infty$ and
    $s'.I(t)>{}upper(I_s(t))$, let us show \\
    \fbox{$\sigma'(id_t)("s\_time\_counter")=upper(I_s(t))$.}

    As $upper(I_s(t))\neq\infty$, there exists an
    $a\in\mathbb{N}^{*}$, and a $b\in\mathbb{N}^{*}$
    s.t. $I_s(t)=[a,b]$. Let us take such an $a$ and $b$. By
    construction, there exists
    $tt\in\{\mathtt{TEMP\_A\_A,TEMP\_A\_B\}}$ s.t.
      ${<}\mathtt{maximal\_time\_counter\Rightarrow}b{>}\in{}gm_t$,
      and ${<}\mathtt{transition\_type\Rightarrow}tt{>}\in{}gm_t$; by
      property of the elaboration relation:
      \begin{eqnarray}
        \label{eq:67}\Delta(id_t)("mtc")&=&b=upper(I_s(t))\\
        \label{eq:71}\Delta(id_t)("tt")&\neq&\mathtt{NOT\_TEMP}
      \end{eqnarray}

      Case analysis on $t\in{}Sens(s.M)$:
      \begin{enumerate}
      \item $t\notin{}Sens(s.M)$:\\
        \noindent{}By definition of
        $E_c,\tau\vdash{}s\xrightarrow{\downarrow}s'$, and knowing that
        $t\in{}Sens(s.M)$, then $s'.I(t)=0$. Since
        $upper(I_s(t))\in\mathbb{N}^{*}$, then $upper(I_s(t))>0$.
        
        \qedbox{Contradicts $s'.I(t)>upper(I_s(t))$.}
        
      \item $t\in{}Sens(s.M)$:\\
        By definition of
        $\gamma,E_c,\tau\vdash{}s\stackrel{\uparrow}{\sim}\sigma$ and
        $t\in{}Sens(s.M)$:
        \begin{equation}
          \label{eq:72}\sigma(id_t)("se")=\mathtt{true}
        \end{equation}

        Case analysis on $s.reset_t(t)$; there are two cases:
        \begin{enumerate}
        \item $s.reset_t(t)=\mathtt{true}$:\\
          By definition of
          $E_c,\tau\vdash{}s\xrightarrow{\downarrow}s'$: $s'.I(t)=1$.

          \noindent{}We assumed that $s'.I(t)>upper(I_s(t))$, then
          $1>upper(I_s(t))$.

          \qedbox{Contradicts $upper(I_s(t))>0$.}

        \item $s.reset_t(t)=\mathtt{false}$:\\
          \noindent{}By property of
          $\gamma,E_c,\tau\vdash{}s\stackrel{\uparrow}{\sim}\sigma$ and
          $s.reset_t(t)=\mathtt{false}$:
          \begin{equation}
            \sigma(id_t)("srtc")=\mathtt{false}\label{eq:73}
          \end{equation}
          
          Case analysis on $s.I(t)>{}upper(I_s(t))$ or
          $s.I(t)\le{}upper(I_s(t))$:
          \begin{enumerate}
          \item $s.I(t)>upper(I_s(t))$: \fbox{$\sigma'(id_t)("stc")=upper(I_s(t))$.}\\
            By definition of \upSim:
            \begin{equation}
              \label{eq:75}s'.I(t)=s.I(t)
            \end{equation}
            
            By definition of
            $\gamma,E_c,\tau\vdash{}s\stackrel{\uparrow}{\sim}\sigma$:
            \begin{equation}
              \label{eq:74}\sigma(id_t)("stc")=upper(I_s(t))
            \end{equation}

            Appealing to \eqref{eq:etnrge}, we have
            $\sigma'(id_t)("stc")=\sigma(id_t)("stc")$.

            Rewriting the goal with $\sigma'(id_t)("stc")=\sigma(id_t)("stc")$ and \eqref{eq:74}:
            \qedbox{tautology.}
            
          \item $s.I(t)\le{}upper(I_s(t))$: \fbox{$\sigma'(id_t)("stc")=upper(I_s(t))$.}\\

            By definition of \upSim:
            \begin{equation}
              \label{eq:77}s.I(t)=\sigma(id_t)("stc")
            \end{equation}
            
            \noindent{}Case analysis on $s.I(t)\le{}upper(I_s(t))$;
            there are two cases:
            \begin{itemize}
            \item $s.I(t)=upper(I_s(t))$:\\

              Appealing to \eqref{eq:67}, \eqref{eq:77} and
              $s.I(t)=upper(I_s(t))$:
              \begin{equation}
                \label{eq:44}\Delta(id_t)("mtc")\le\sigma(id_t)("stc")
              \end{equation}

              Appealing to \eqref{eq:44} and \eqref{eq:etnrge}:
              \begin{equation}
                \label{eq:47}\sigma'(id_t)("stc")=\sigma(id_t)("stc")
              \end{equation}

              Rewriting the goal with \eqref{eq:47}, \eqref{eq:77} and
              $s.I(t)=upper(I_s(t))$: \qedbox{tautology.}
              
            \item $s.I(t)<upper(I_s(t))$:\\
              
              By definition of \dwSitpn:
              \begin{equation}
                \label{eq:76} s'.I(t)=s.I(t)+1
              \end{equation}

              From \eqref{eq:76} and $s.I(t)<upper(I_s(t))$, we can
              deduce $s'.I(t)\le{}upper(I_s(t))$; \qedbox{contradicts
                $s'.I(t)>upper(I_s(t))$.}
            \end{itemize}
          \end{enumerate}
        \end{enumerate}
      \end{enumerate}
      
    \item
      \fbox{$upper(I_s(t))\neq\infty\land{}s'.I(t)\le{}upper(I_s(t))\Rightarrow$
        $s'.I(t)=\sigma'(id_t)("s\_time\_counter")$.}

      Assuming that $upper(I_s(t))\neq\infty$ and
      $s'.I(t)\le{}upper(I_s(t))$, let us show\\
      \fbox{$s'.I(t)=\sigma'(id_t)("s\_time\_counter")$.}

      As $upper(I_s(t))\neq\infty$, there exists an
      $a\in\mathbb{N}^{*}$, and a $b\in\mathbb{N}^{*}$
      s.t. $I_s(t)=[a,b]$. Let us take such an $a$ and $b$. By
      construction, there exists
      $tt\in\{\mathtt{TEMP\_A\_A,TEMP\_A\_B\}}$ s.t.\\
      ${<}\mathtt{maximal\_time\_counter\Rightarrow}b{>}\in{}gm_t$,
      and ${<}\mathtt{transition\_type\Rightarrow}tt{>}\in{}gm_t$; by
      property of the elaboration relation:
      \begin{eqnarray}
        \label{eq:eq-mtc-upper}\Delta(id_t)("mtc")&=&b=upper(I_s(t))\\
        \label{eq:neq-tt-not-temp}\Delta(id_t)("tt")&\neq&\mathtt{NOT\_TEMP}
      \end{eqnarray}
      
      Case analysis on $t\in{}Sens(s.M)$:
      \begin{enumerate}
      \item $t\notin{}Sens(s.M)$:\\
        By definition of \upSim, we have
        \inlineequation[eq:eq-rst-false]{\sigma(id_t)("se")=\mathtt{false}}.

        \noindent{}Appealing \eqref{eq:ne-or-nt} and
        \eqref{eq:eq-rst-false}, we have
        \inlineequation[eq:eq-stc-zero]{\sigma'(id_t)("stc")=0}.

        \noindent{}By definition of \dwSitpn, we have
        \inlineequation[eq:eq-Itp-zero]{s'.I(t)=0}.

        Rewriting the goal with \eqref{eq:eq-stc-zero} and
        \eqref{eq:eq-Itp-zero}: \qedbox{tautology.}
        
      \item $t\in{}Sens(s.M)$:\\
        By definition of \upSim, we have
        \inlineequation[eq:eq-se-true]{\sigma(id_t)("se")=\mathtt{true}}.

        \noindent{}Case analysis on $s.reset_t(t)$:
        \begin{enumerate}
        \item $s.reset_t(t)=\mathtt{true}$:\\
          By definition of \upSim, we have
          \inlineequation[eq:eq-srtc-true]{\sigma(id_t)("srtc")=\mathtt{true}}.

          Appealing to \eqref{eq:etr}, \eqref{eq:neq-tt-not-temp},
          \eqref{eq:eq-se-true} and \eqref{eq:eq-srtc-true}, we have
          \inlineequation[eq:eq-stc-one]{\sigma'(id_t)("stc")=1}.

          By definition of \dwSitpn, we have
          \inlineequation[eq:eq-Itp-one]{s'.I(t)=1}.

          Rewriting the goal with \eqref{eq:eq-stc-one} and
          \eqref{eq:eq-Itp-one}, \qedbox{tautology.}
          
        \item $s.reset_t(t)=\mathtt{false}$:\\
          By definition of \upSim, we have
          \inlineequation[eq:eq-srtc-false]{\sigma(id_t)("srtc")=\mathtt{false}}.

          Case analysis on $s.I(t)>upper(I_s(t))$ or $s.I(t)\le{}upper(I_s(t))$:
          \begin{enumerate}
          \item $s.I(t)>upper(I_s(t))$:\\
            By definition of \dwSitpn, we have $s.I(t)=s'.I(t)$, and
            thus, $s'.I(t)>upper(I_s(t))$. \qedbox{Contradicts
              $s'.I(t)\le{}upper(I_s(t))$.}
            
          \item $s.I(t)\le{}upper(I_s(t))$:\\
            By definition of \upSim, we have
            \inlineequation[eq:eq-It-stc]{s.I(t)=\sigma(id_t)("stc")}.

            \begin{itemize}
            \item $s.I(t)<upper(I_s(t))$:\\
              From $s.I(t)<upper(I_s(t))$, \eqref{eq:eq-It-stc} and
              \eqref{eq:eq-mtc-upper}, we can deduce\\
              \inlineequation[eq:lt-mtc-stc]{\sigma(id_t)("stc")<\Delta(id_t)("mtc")}.

              From \eqref{eq:etnrlt}, \eqref{eq:eq-se-true},
              \eqref{eq:neq-tt-not-temp}, \eqref{eq:eq-srtc-false} and
              \eqref{eq:lt-mtc-stc}, we can deduce:
              \begin{equation}
                \label{eq:eq-stc-stc-plus-one}
                \sigma'(id_t)("stc")=\sigma(id_t)("stc")+1
              \end{equation}

              By definition of \dwSitpn:
              \begin{equation}
                \label{eq:eq-Itp-It-plus-one}
                s'.I(t)=s.I(t)+1
              \end{equation}

              Rewriting the goal with \eqref{eq:eq-stc-stc-plus-one}
              and \eqref{eq:eq-Itp-It-plus-one}, \qedbox{tautology.}
              
            \item $s.I(t)=upper(I_s(t))$:\\
              By definition of \dwSitpn, we know that
              $s'.I(t)=s.I(t)+1$. We assumed that
              $s'.I(t)\le{}upper(I_s(t))$; thus,
              $s.I(t)+1\le{}upper(I_s(t))$.

              \qedbox{Contradicts $s.I(t)=upper(I_s(t))$.}
              
            \end{itemize}
            
          \end{enumerate}
          
        \end{enumerate}
        
      \end{enumerate}
    \end{enumerate}
\end{proof}

\subsection{Falling edge and condition values}
\label{sec:fe-cond-values}

\begin{lemma}[Falling Edge Equal Condition Values]
  \label{lem:fe-equal-cond-values}
  \fehyps{} then
  $\forall{}c\in\mathcal{C},id_c\in{}Ins(\Delta)~s.t.~\gamma(c)=id_c,~s'.cond(c)=\sigma'(id_c)$.
\end{lemma}

\begin{proof}
  Given a $c\in\mathcal{C}$ and an $id_c\in{}Ins(\Delta)$
  s.t. $\gamma(c)=id_c$, let us show
  \fbox{$s'.cond(c)=\sigma'(id_c)$.}

  \noindent{}By definition of \dwSitpn, we have
  \inlineequation[eq:fe-eq-cond]{s'.cond(c)=E_c(\tau,c)}.

  \noindent{}By property of the $\mathtt{Inject_\downarrow}$, the
  \hvhdl{} falling edge, the stabilize relations and
  $id_c\in{}Ins(\Delta)$, we have
  \inlineequation[eq:fe-eq-idc-Ep]{\sigma'(id_c)=E_p(\tau,\downarrow)(id_c)}.

  \noindent{}Rewriting the goal with \eqref{eq:fe-eq-cond} and
  \eqref{eq:fe-eq-idc-Ep}:
  \fbox{$E_c(\tau,c)=E_p(\tau,\downarrow)(id_c)$}
  
  \noindent{}By definition of $\gamma\vdash{}E_p\stackrel{env}{=}E_c$:
  \qedbox{$E_c(\tau,c)=E_p(\tau,\downarrow)(id_c)$.}
  
\end{proof}

\subsection{Falling and action executions}
\label{sec:fe-equal-act-exec}

\begin{lemma}[Falling Edge Equal Action Executions]
  \label{lem:fe-equal-act-exec}
  \fehyps{} then
  $\forall{}a\in\mathcal{A},id_a\in{}Outs(\Delta)~s.t.~\gamma(a)=id_a,~s'.ex(a)=\sigma'(id_a)$.
\end{lemma}

\begin{proof}
  Given an $a\in\mathcal{A}$ and an $id_a\in{}Outs(\Delta)$
  s.t. $\gamma(a)=id_a$, let us show \fbox{$s'.ex(a)=\sigma'(id_a)$.}

  \noindent{}By property of \dwSitpn:
  \begin{equation}
    s'.ex(a)=\sum\limits_{p\in{}marked(s.M)}\mathbb{A}(p,a)\label{eq:fe-eq-exa}
  \end{equation}
  
  \noindent{}By construction, the \texttt{``action''} process is a
  part of design $d$'s behavior, i.e there exist an
  $sl\subseteq{}Sigs(\Delta)$ and an $ss_a\in{}ss$ s.t.
  $\mathtt{ps}("action", \emptyset, sl, ss)\in{}d.cs$.
  
  \noindent{}By construction $id_a$ is only assigned in the body of
  the \texttt{``action''} process. Let $pls(a)$ be the set of actions
  associated to action $a$, i.e
  $pls(a)=\{p\in{}P~\vert~\mathbb{A}(p,a)=true\}$. Then, depending on
  $pls(a)$, there are two cases of assignment of output port $id_a$:
  
  \begin{itemize}
  \item \textbf{CASE} $pls(a)=\emptyset$:\\
    \noindent{}By construction,
    $\mathtt{id_a\Leftarrow{}false}\in{}ss_{a\downarrow}$ where
    $ss_{a\downarrow}$ is the part of the \texttt{``action''} process
    body executed during the falling edge phase.

    \noindent{}By property of the \hvhdl{} falling edge, the stabilize
    relations and $\mathtt{ps}("action", \emptyset, sl, ss_a)\in{}d.cs$:
    \begin{equation}
      \sigma'(id_a)=false\label{eq:fe-eq-ida-false}
    \end{equation}
    
    \noindent{}By property of
    $\sum\limits_{p\in{}marked(s.M)}\mathbb{A}(p,a)$ and
    $pls(a)=\emptyset$:
    \begin{equation}
      \sum\limits_{p\in{}marked(s.M)}\mathbb{A}(p,a)=\mathtt{false}\label{eq:fe-eq-suma-false}
    \end{equation}

    \noindent{}Rewriting the goal with \eqref{eq:fe-eq-exa},
    \eqref{eq:fe-eq-ida-false} and \eqref{eq:fe-eq-suma-false},
    \qedbox{tautology.}
    
  \item \textbf{CASE} $pls(a)\neq\emptyset$:\\
    \noindent{}By construction,
    $\mathtt{id_a\Leftarrow{}id_{mp_0}+\dots+id_{mp_n}}\in{}ss_{a\downarrow}$,
    where $id_{mp_i}\in{}Sigs(\Delta)$, $ss_{a\downarrow}$ is the part
    of the \texttt{``action''} process body executed during the
    falling edge phase, and $n=\vert{}pls(a)\vert-1$.

    \noindent{}By property of the $\mathtt{Inject}_\downarrow$, the
    \hvhdl{} falling edge, the stabilize relations, and\\
    $\mathtt{ps}("action", \emptyset, sl, ss)\in{}d.cs$:
    \begin{equation}
      \sigma'(id_a)=\sigma(id_{mp_0})+\dots+\sigma(id_{mp_n})\label{eq:fe-eq-ida-sum}
    \end{equation}

    Rewriting the goal with \eqref{eq:fe-eq-exa} and
    \eqref{eq:fe-eq-ida-sum},
    \fbox{$\sum\limits_{p\in{}marked(s.M)}\mathbb{A}(p,a)=\sigma(id_{mp_0})+\dots+\sigma(id_{mp_n})$.}

    Let us reason on the value of $\sigma(id_{mp_0})+\dots+\sigma(id_{mp_n})$; there are two cases:

    \begin{itemize}
    \item \textbf{CASE} $\sigma(id_{mp_0})+\dots+\sigma(id_{mp_n})=\mathtt{true}$:\\
      \noindent{}Then, we can rewrite the goal as follows:
      \fbox{$\sum\limits_{p\in{}marked(s.M)}\mathbb{A}(p,a)=\mathtt{true}$.}

      \noindent{}To prove the above goal, let us show
      \fbox{$\exists{}p\in{}marked(s.M)~s.t.~\mathbb{A}(p,a)=\mathtt{true}$.}

      \noindent{}From
      $\sigma(id_{mp_0})+\dots+\sigma(id_{mp_n})=\mathtt{true}$, we
      can deduce that
      $\exists{}id_{mp_i}~s.t.~\sigma(id_{mp_i})=\mathtt{true}$. Let
      us take an $id_{mp_i}$ s.t. $\sigma(id_{mp_i})=\mathtt{true}$.

      \noindent{}By construction, for all $id_{mp_i}$, there exist a
      $p_i\in{}pls(a)$, an $id_{p_i}\in{}Comps(\Delta)$, $gm_{p_i}$,
      $ipm_{p_i}$ and $opm_{p_i}$ s.t. $\gamma(p_i)=id_{p_i}$ and
      $\mathtt{comp}(id_{p_i}, "place", gm_{p_i}, ipm_{p_i},
      opm_{p_i})\in{}d.cs$ and
      ${<}\mathtt{marked\Rightarrow{}id_{mp_i}}{>}\in{}opm_{p_i}$. Let
      us take such a $p_i$, $id_{p_i}$, $gm_{p_i}$, $ipm_{p_i}$ and
      $opm_{p_i}$.

      \noindent{}By property of stable $\sigma$, and
      $\mathtt{comp}(id_{p_i}, "place", gm_{p_i}, ipm_{p_i},
      opm_{p_i})\in{}d.cs$:
      \begin{eqnarray}
        \sigma(id_{mp_i})=\sigma(id_{p_i})("marked")\label{eq:fe-eq-marked-idmpi}\\
        \sigma(id_{p_i})("marked")=\sigma(id_{p_i})("sm")>0\label{eq:fe-gt-sm-zero}
      \end{eqnarray}

      \noindent{}From \eqref{eq:fe-eq-marked-idmpi},
      \eqref{eq:fe-gt-sm-zero} and $\sigma(id_{mp_i})=\mathtt{true}$,
      we can deduce that $\sigma(id_{p_i})("marked")=\mathtt{true}$
      and $(\sigma(id_{p_i})("sm")>0)=\mathtt{true}$.

      
      \noindent{}By property of \upSim:
      \begin{equation}
        \label{eq:fe-eq-m-sm}s.M(p_i)=\sigma(id_{p_i})("sm")
      \end{equation}
      
      \noindent{}From \eqref{eq:fe-eq-m-sm} and
      $(\sigma(id_{p_i})("sm")>0)=\mathtt{true}$, we can deduce
      $p_i\in{}marked(s.M)$, i.e $s.M(p_i)>0$.

      Let us use $p_i$ to prove the goal:
      \fbox{$\mathbb{A}(p,a)=\mathtt{true}$.}

      \noindent{}By definition of $p_i\in{}pls(a)$,
      \qedbox{$\mathbb{A}(p,a)=\mathtt{true}$.}

    \item \textbf{CASE} $\sigma(id_{mp_0})+\dots+\sigma(id_{mp_n})=\mathtt{false}$:\\
      \noindent{}Then, we can rewrite the goal as follows:
      \fbox{$\sum\limits_{p\in{}marked(s.M)}\mathbb{A}(p,a)=\mathtt{false}$.}

      \noindent{}To prove the above goal, let us show
      \fbox{$\forall{}p\in{}marked(s.M)~s.t.~\mathbb{A}(p,a)=\mathtt{false}$.}

      \noindent{}Given a $p\in{}marked(s.M)$, let us show
      \fbox{$\mathbb{A}(p,a)=\mathtt{false}$.}

      \noindent{}Let us perform case analysis on $\mathbb{A}(p,a)$;
      there are 2 cases:

      \begin{itemize}
      \item \textbf{CASE} \qedbox{$\mathbb{A}(p,a)=\mathtt{false}$.}
      \item \textbf{CASE} $\mathbb{A}(p,a)=\mathtt{true}$:\\
        \noindent{}By construction, for all $p\in{}P$
        s.t. $\mathbb{A}(p,a)=\mathtt{true}$, there exist an
        $id_{p}\in{}Comps(\Delta)$, $gm_{tp}$, $ipm_{p}$, $opm_{p}$
        and $id_{mp_i}\in{}Sigs(\Delta)$ s.t. $\gamma(p)=id_{p}$ and
        $\mathtt{comp}(id_{p},$ $"place",$ $gm_{p},$ $ipm_{p},$
        $opm_{p})\in{}d.cs$ and
        ${<}\mathtt{marked\Rightarrow{}id_{mp_i}}{>}\in{}opm_{p}$. Let
        us take such a $id_{p}$, $gm_{p}$, $ipm_{p}$, $opm_{p}$ and
        $id_{mp_i}$.

        \noindent{}By property of stable $\sigma$ and \InCsCompP:
        \begin{eqnarray}
          \sigma(id_{mp_i})=\sigma(id_{p})("marked")\label{eq:fe-eq-marked-idmp}\\
          \sigma(id_{p})("marked")=\sigma(id_{p})("sm")>0\label{eq:fe-eq-marked-gt-sm-zero}
        \end{eqnarray}

        From
        $\sigma(id_{mp_0})+\dots+\sigma(id_{mp_n})=\mathtt{false}$, we
        can deduce $\sigma(id_{p})("marked")=\mathtt{false}$, and thus
        that $(\sigma(id_{p})("sm")>0)=\mathtt{false}$.

        By definition of \upSim, we have $s.M(p)=\sigma(id_p)("sm")$,
        and thus, we can deduce that $s.M(p)=0$ (equivalent to
        $(s.M(p)>0)=\mathtt{false}$).

        \noindent{}Contradicts \qedbox{$p\in{}marked(s.M)$} (i.e,
        $s.M(p)>0$).
      \end{itemize}
    \end{itemize}
  \end{itemize}
\end{proof}

\subsection{Falling edge and function executions}
\label{sec:fe-fun-exec}

\begin{lemma}[Falling Edge Equal Function Executions]
  \label{lem:fe-equal-fun-exec}
  \fehyps{} then
  $\forall{}f\in\mathcal{F},id_f\in{}Outs(\Delta)~s.t.~\gamma(f)=id_f,~s'.ex(f)=\sigma'(id_f)$.
\end{lemma}

\begin{proof}
  Given an $f\in\mathcal{F}$ and an
  $id_f\in{}Outs(\Delta)~s.t.~\gamma(f)=id_f$, let us show
  \fbox{$s'.ex(f)=\sigma'(id_f)$.}\\

  \noindent{}By property of \dwSitpn:
  \begin{equation}
    s.ex(f)=s'.ex(f)\label{eq:fe-eq-exf}
  \end{equation}
  
  \noindent{}By construction, $id_f$ is an output port identifier of
  boolean type in the \hvhdl{} design $d$ assigned by the
  \texttt{``function''} process only during a rising edge phase.

  \noindent{}By property of the \hvhdl{} $\mathtt{Inject_{\uparrow}}$,
  rising edge, stabilize relations, and the \texttt{``function''}
  process:
  \begin{equation}
    \sigma(id_f)=\sigma'(id_f)\label{eq:fe-eq-idf}
  \end{equation}

  \noindent{}Rewriting the goal with \eqref{eq:fe-eq-exf} and
  \eqref{eq:fe-eq-idf}, \fbox{$s.ex(f)=\sigma(id_f)$.}

  \noindent{}By definition of \upSim, \qedbox{$s.ex(f)=\sigma(id_f)$.}
\end{proof}

\subsection{Falling edge and firable transitions}
\label{sec:fe-firable}

%%%%%%%%%%%%%%%%%%%%%%%%%%%%%%%%%%%%%%%%%%%%%%%%
%%%%%%%%%% FALLING EDGE EQUAL FIRABLE %%%%%%%%%%
%%%%%%%%%%%%%%%%%%%%%%%%%%%%%%%%%%%%%%%%%%%%%%%%

\begin{lemma}[Falling Edge Equal Firable]
  \label{lem:fe-equal-firable}
  \fehyps{} then
  $\forall{}t\in{}T,id_t\in{}Comps(\Delta)~s.t.~\gamma(t)=id_t,$
  $t\in{}Firable(s')\Leftrightarrow\sigma'(id_t)("s\_firable")=\mathtt{true}$.
\end{lemma}

\begin{proof}
  Given a $t\in{}T$ and $id_t\in{}Comps(\Delta)$
  s.t. $\gamma(t)=id_t$, let us show that\\
  \fbox{$t\in{}Firable(s')\Leftrightarrow\sigma'(id_t)("s\_firable")=\mathtt{true}$.}\\

  The proof is in two parts:
  \begin{enumerate}
  \item Assuming that $t\in{}Firable(s')$, let us show
    \fbox{$\sigma'(id_t)("s\_firable")=\mathtt{true}$.}
    
    Apply Lemma~\nameref{lem:fe-equal-firable-1} to solve the goal.
  \item Assuming that $\sigma'(id_t)("s\_firable")=\mathtt{true}$, let
    us show \fbox{$t\in{}Firable(s')$.}
    
    Apply Lemma~\nameref{lem:fe-equal-firable-2} to solve the goal.
  \end{enumerate}
  
\end{proof}

\begin{lemma}[Falling Edge Equal Firable 1]
  \label{lem:fe-equal-firable-1}
  \fehyps{} then
  $\forall{}t\in{}T,id_t\in{}Comps(\Delta)~s.t.~\gamma(t)=id_t,$
  $t\in{}Firable(s')\Rightarrow\sigma'(id_t)("s\_firable")=\mathtt{true}$.
\end{lemma}

\begin{niproof}
  Given a $t\in{}T$ and $id_t\in{}Comps(\Delta)$
  s.t. $\gamma(t)=id_t$, and assuming that $t\in{}Firable(s')$, let us
  show \fbox{$\sigma'(id_t)("s\_firable")=\mathtt{true}$.}\\
  
  \exT{}

  By property of the $Inject_\downarrow$, the \hvhdl{}
  falling edge, the stabilize relations and \InCsCompT:
  \begin{equation}
    \label{eq:fe-eq-sfa}
    \sigma'(id_t)("sfa")=\sigma(id_t)("se")~.~\sigma(id_t)("scc")~.~\mathtt{checktc}(\Delta(id_t),\sigma(id_t))
  \end{equation}

  Let us define term
  $\mathtt{checktc}(\Delta(id_t),\sigma(id_t))$ as follows:
  \begin{equation}
    \label{eq:checktc}
    \begin{split}
      \mathtt{checktc}(\Delta(id_t),\sigma(id_t))=&\bigg(\mathtt{not}~\sigma(id_t)("srtc")~.~\\
      & \quad\begin{split}
        \big[\big(\Delta(id_t)("tt")=\mathtt{TEMP\_A\_B}~&.~(\sigma(id_t)("stc")\ge{}\sigma(id_t)("A")-1)~\\
        &.~(\sigma(id_t)("stc")\le{}\sigma(id_t)("B")-1)\big)\\
      \end{split} \\
      & \quad+(\Delta(id_t)("tt")=\mathtt{TEMP\_A\_A}~.~(\sigma(id_t)("stc")={}\sigma(id_t)("A")-1))\\
      &
      \quad+(\Delta(id_t)("tt")=\mathtt{TEMP\_A\_INF}~.~(\sigma(id_t)("stc")\ge{}\sigma(id_t)("A")-1))\big]\bigg) \\
      & +\big(\sigma(id_t)("srtc")~.~\Delta(id_t)("tt")\neq\mathtt{NOT\_TEMP}~.~\sigma(id_t)("A")=1\big)\\
      & +\Delta(id_t)("tt")=\mathtt{NOT\_TEMP}\\
    \end{split}
  \end{equation}

  Rewriting the goal with \eqref{eq:fe-eq-sfa}:
  \fbox{$\sigma(id_t)("se")~.~\sigma(id_t)("scc")~.~\mathtt{checktc}(\Delta(id_t),\sigma(id_t))=\mathtt{true}$.}

  Then, there are three points to prove:
  \begin{enumerate}
  \item \fbox{$\sigma(id_t)("se")=\mathtt{true}$}:\\

    From $t\in{}Firable(s')$, we can deduce
    $t\in{}Sens(s'.M)$. By definition of \dwSitpn, we have $s.M=s'.M$,
    and thus, we can deduce $t\in{}Sens(s.M)$.

    By definition of \upSim, we know that $t\in{}Sens(s.M)$
    implies \qedbox{$\sigma(id_t)("se")=\mathtt{true}$.}
    
  \item \fbox{$\sigma(id_t)("scc")=\mathtt{true}$}:\\

    By definition of \upSim:
    \begin{equation}
      \sigma(id_t)("scc")=\prod\limits_{c\in{}conds(t)}
      \begin{cases}
        E_c(\tau,c) & if~\mathbb{C}(t,c)=1 \\
        \mathtt{not}(E_c(\tau,c)) & if~\mathbb{C}(t,c)=-1 \\
      \end{cases}
      \label{eq:fe-eq-scc-prod}
    \end{equation}
    where
    $conds(t)=\{c\in\mathcal{C}~\vert~\mathbb{C}(t,c)=1\lor\mathbb{C}(t,c)=-1\}$.

    Rewriting the goal with \eqref{eq:fe-eq-scc-prod}:
    \fbox{$\prod\limits_{c\in{}conds(t)}
      \begin{cases}
        E_c(\tau,c) & if~\mathbb{C}(t,c)=1 \\
        \mathtt{not}(E_c(\tau,c)) & if~\mathbb{C}(t,c)=-1 \\
      \end{cases}=\mathtt{true}$.}

    To ease the reading, let us define $f(c)=\begin{cases}
          E_c(\tau,c) & if~\mathbb{C}(t,c)=1 \\
          \mathtt{not}(E_c(\tau,c)) & if~\mathbb{C}(t,c)=-1 \\
        \end{cases}$.
    
    Let us reason by induction on the left term of the
    goal:

    \begin{itemize}
    \item \textbf{BASE CASE}: \qedbox{$true=true$.}
    \item \textbf{INDUCTION CASE}:
      \begin{ih}
        $\prod\limits_{c'\in{}conds(t)\setminus\{c\}}f(c')=\mathtt{true}$
      \end{ih}

      \fbox{$f(c)~.~\prod\limits_{c'\in{}conds(t)\setminus\{c\}}f(c')=\mathtt{true}$.}

      Rewriting the goal with the induction hypothesis, and
      simplifying the goal, and unfolding the definition of $f(c)$:
      \fbox{$\begin{cases}
          E_c(\tau,c) & if~\mathbb{C}(t,c)=1 \\
          \mathtt{not}(E_c(\tau,c)) & if~\mathbb{C}(t,c)=-1 \\
        \end{cases}=\mathtt{true}$.}

      As $c\in{}conds(t)$, let us perform case analysis on
      $\mathbb{C}(t,c)=1\lor\mathbb{C}(t,c)=-1$:
      \begin{enumerate}
      \item $\mathbb{C}(t,c)=1$: \fbox{$E_c(\tau,c)=\mathtt{true}$.}\\
        
        By definition of $t\in{}Firable(s')$, we can deduce
        that $s'.cond(c)=\mathtt{true}$. By definition of \dwSitpn, we
        have $s'.cond(c)=E_c(\tau,c)$. Thus,
        \qedbox{$E_c(\tau,c)=\mathtt{true}$.}
        
      \item $\mathbb{C}(t,c)=-1$: \fbox{$\mathtt{not}~E_c(\tau,c)=\mathtt{true}$.}\\

        By definition of $t\in{}Firable(s')$, we can deduce
        that $s'.cond(c)=\mathtt{false}$. By definition of \dwSitpn,
        we have $s'.cond(c)=E_c(\tau,c)$. Thus,
        \qedbox{$\mathtt{not}~E_c(\tau,c)=\mathtt{true}$.}
      \end{enumerate}
    \end{itemize}

  \item \fbox{$\mathtt{checktc}(\Delta(id_t),\sigma(id_t))=\mathtt{true}$}:\\

    By definition of $t\in{}Firable(s')$, we have
    $t\notin{}T_i\lor{}s'.I(t)\in{}I_s(t)$. Let us perform case
    analysis on $t\notin{}T_i\lor{}s'.I(t)\in{}I_s(t)$:\\

    \begin{enumerate}
    \item $t\notin{}T_i$:\\

      By construction,
      ${<}\mathtt{transition\_type\Rightarrow{}NOT\_TEMP}{>}\in{}gm_t$,
      and by property of the elaboration relation, we have
      $\Delta(id_t)("tt")=\mathtt{NOT\_TEMP}$.

      From $\Delta(id_t)("tt")=\mathtt{NOT\_TEMP}$, and the
      definition of $\mathtt{checktc}(\Delta(id_t),\sigma(id_t))$, we
      can deduce
      \qedbox{$\mathtt{checktc}(\Delta(id_t),\sigma(id_t))=\mathtt{true}$.}

    \item $s'.I(t)\in{}I_s(t)$:\\

      From $s'.I(t)\in{}I_s(t)$, we can deduce that $t\in{}T_i$. Thus,
      by construction, there exists
      $tt\in\{\mathtt{TEMP\_A\_B},\mathtt{TEMP\_A\_A},\mathtt{TEMP\_A\_INF}\}$
      s.t. ${<}\mathtt{transition\_type\Rightarrow}tt{>}\in{}gm_t$. By
      property of the elaboration relation, we have
      $\Delta(id_t)("tt")=tt$, and thus, we know
      $\Delta(id_t)("tt")\neq{}\mathtt{NOT\_TEMP}$. Therefore, we can
      simplfy the term $\mathtt{checktc}(\Delta(id_t),\sigma(id_t))$
      as follows:
      \begin{equation}        
        \label{eq:eq-checktc-minus-not-temp}
        \begin{split}
          \mathtt{checktc}(\Delta(id_t),\sigma(id_t))=&\bigg(\mathtt{not}~\sigma(id_t)("srtc")~.~\\
          & \quad\begin{split}
            \big[\big(\Delta(id_t)("tt")=\mathtt{TEMP\_A\_B}~&.~(\sigma(id_t)("stc")\ge{}\sigma(id_t)("A")-1)~\\
            &.~(\sigma(id_t)("stc")\le{}\sigma(id_t)("B")-1)\big)\\
          \end{split} \\
          & \quad{\begin{array}{l}
                    +(\Delta(id_t)("tt")=\mathtt{TEMP\_A\_A}~.\\
                    \quad~(\sigma(id_t)("stc")={}\sigma(id_t)("A")-1)) \\
                  \end{array}}\\
          &
          \quad{\begin{array}{l}
                  +(\Delta(id_t)("tt")=\mathtt{TEMP\_A\_INF}~. \\
                  \quad~(\sigma(id_t)("stc")\ge{}\sigma(id_t)("A")-1))\big]\bigg) \\
                \end{array}}\\
          & +\big(\sigma(id_t)("srtc")~.~\sigma(id_t)("A")=1\big)\\
        \end{split}
      \end{equation}
      
      By definition of \upSim, we have
      $s.reset_t(t)=\sigma(id_t)("srtc")$.

      Let us perform case analysis on the value
      $s.reset_t(t)$:

      \begin{enumerate}
      \item $s.reset_t(t)=\mathtt{true}$:\\

        Then, from $s.reset_t(t)=\sigma(id_t)("srtc")$, we can deduce
        that $\sigma(id_t)("srtc")=\mathtt{true}$.

        From $\sigma(id_t)("srtc")=\mathtt{true}$, we can
        simplify the term
        $\mathtt{checktc}(\Delta(id_t),\sigma(id_t))$ as follows:
        \begin{equation}
          \label{eq:eq-checktc-a-eq-1}
            \mathtt{checktc}(\Delta(id_t),\sigma(id_t))=\big(\sigma(id_t)("A")=1\big)
        \end{equation}

        Rewriting the goal with \eqref{eq:eq-checktc-a-eq-1}, and
        simplifying the goal: \fbox{$\sigma(id_t)("A")=1$.}

        By definition of \dwSitpn, from $t\in{}Sens(s.M)$ and
        $s.reset_t(t)=\mathtt{true}$, we can deduce $s'.I(t)=1$. We
        know that $s'.I(t)\in{}I_s(t)$, and thus, we have
        $1\in{}I_s(t)$. By definition of $1\in{}I_s(t)$, there exist
        an $a\in\mathbb{N}^{*}$ and a
        $ni\in{}\mathbb{N}^{*}\sqcup\{\infty\}$ s.t. $I_s(t)=[a,ni]$
        and $1\in[a,ni]$.

        By definition of $1\in[a,ni]$, we have $a\le{}1$, and
        since $a\in\mathbb{N}^{*}$, we can deduce $a=1$.

        By construction,
        ${<}\mathtt{time\_A\_value\Rightarrow}{}a{>}\in{}ipm_t$, and
        by property of stable $\sigma$, we have
        \qedbox{$\sigma(id_t)("A")=a=1$.}

      \item $s.reset_t(t)=\mathtt{false}$:\\

        Then, from $s.reset_t(t)=\sigma(id_t)("srtc")$, we can deduce
        that $\sigma(id_t)("srtc")=\mathtt{false}$.

        From $\sigma(id_t)("srtc")=\mathtt{false}$, we can simplify
        the term $\mathtt{checktc}(\Delta(id_t),\sigma(id_t))$ as
        follows:
        \begin{equation}
          \label{eq:eq-checktc-srtc-false}
          \begin{array}{l}
            \multicolumn{1}{c}{\mathtt{checktc}(\Delta(id_t),\sigma(id_t))}\\
            \multicolumn{1}{c}{=}\\
            \begin{array}{ll}
              \big(\Delta(id_t)("tt")=\mathtt{TEMP\_A\_B}~&.~(\sigma(id_t)("stc")\ge{}\sigma(id_t)("A")-1)~\\
                                                          &.~(\sigma(id_t)("stc")\le{}\sigma(id_t)("B")-1)\big)\\
            \end{array}\\
            +(\Delta(id_t)("tt")=\mathtt{TEMP\_A\_A}~.~(\sigma(id_t)("stc")={}\sigma(id_t)("A")-1))\\
            +(\Delta(id_t)("tt")=\mathtt{TEMP\_A\_INF}~.~(\sigma(id_t)("stc")\ge{}\sigma(id_t)("A")-1)) \\
          \end{array}
        \end{equation}

        Let us perform case analysis on $I_s(t)$; there are two cases:
        \begin{itemize}
        \item $I_s(t)=[a,b]$ where $a,b\in\mathbb{N}^{*}$; then, either $a=b$ or $a\neq{}b$:
          \begin{itemize}
          \item $a=b$:\\
            Then, we have $I_s(t)=[a,a]$, and by construction
            ${<}\mathtt{transition\_type\Rightarrow{}}$
            $\mathtt{TEMP\_A\_A}{>}\in{}gm_t$. By property of the
            elaboration relation, we have\\
            $\Delta(id_t)$$("tt")=\mathtt{TEMP\_A\_A}$; thus we can
            simplify the term $\mathtt{checktc}$ as follows:
            \begin{equation}
              \label{eq:eq-checktc-temp-a-a}
              \mathtt{checktc}(\Delta(id_t),\sigma(id_t))=(\sigma(id_t)("stc")={}\sigma(id_t)("A")-1)
            \end{equation}

            Rewriting the goal with \eqref{eq:eq-checktc-temp-a-a},
            and simplifying the goal:\\
            \fbox{$\sigma(id_t)("stc")={}\sigma(id_t)("A")-1$.}
            
            From $s'.I(t)\in[a,a]$, we can deduce that
            $s'.I(t)=a$. Let us perform case analysis on
            $s.I(t)<upper(I_s(t))$ or $s.I(t)\ge{}upper(I_s(t))$:
            \begin{itemize}
            \item $s.I(t)<upper(I_s(t))$:\\
              By definition of \upSim, we have
              $s.I(t)=\sigma(id_t)("stc")$. By definition of \dwSitpn,
              we have $s'.I(t)=s.I(t)+1$. From $s'.I(t)=a$ and
              $s'.I(t)=s.I(t)+1$, we can deduce $a-1=s.I(t)$.

              By construction,
              ${<}\mathtt{time\_A\_value\Rightarrow}{}a{>}\in{}ipm_t$,
              and by property of stable $\sigma$, we have
              $\sigma(id_t)("A")=a$.

              Rewriting the goal with $\sigma(id_t)("A")=a$ and
              $s.I(t)=\sigma(id_t)("stc")$:\\
              \qedbox{$\sigma(id_t)("stc")={}\sigma(id_t)("A")-1$.}
              
            \item $s.I(t)\ge{}upper(I_s(t))$:\\
              In the case where $s.I(t)>upper(I_s(t))$, then
              $s.I(t)>a$. By definition of \dwSitpn, we have
              $s.I(t)=s'.I(t)=a$. Then, \qedbox{$a>a$ is a
                contradiction.}\\

              In the case where $s.I(t)=upper(I_s(t))$, then
              $s.I(t)=a$. By definition of \dwSitpn, we have
              $s'.I(t)=s.I(t)+1$. Then, we have $s'.I(t)=a$ and
              $s'.I(t)=a+1$. Then, \qedbox{$a=a+1$ is a
                contradiction.}
            \end{itemize}
            
          \item $a\neq{}b$:\\
            Then, we have $I_s(t)=[a,b]$, and by construction
            ${<}\mathtt{transition\_type\Rightarrow{}}$
            $\mathtt{TEMP\_A\_B}{>}\in{}gm_t$. By property of the
            elaboration relation, we have\\
            $\Delta(id_t)$$("tt")=\mathtt{TEMP\_A\_B}$; thus we can
            simplify the term $\mathtt{checktc}$ as follows:
            \begin{equation}
              \label{eq:eq-checktc-temp-a-b}
              \begin{array}{c}
                \multicolumn{1}{c}{\mathtt{checktc}(\Delta(id_t),\sigma(id_t))}\\
                \multicolumn{1}{c}{=}\\
                (\sigma(id_t)("stc")\ge{}\sigma(id_t)("A")-1)~.~(\sigma(id_t)("stc")\le{}\sigma(id_t)("B")-1)\\
              \end{array}\\
            \end{equation}

            Rewriting the goal with \eqref{eq:eq-checktc-temp-a-b},
            and simplifying the goal:\\
            \fbox{$(\sigma(id_t)("stc")\ge{}\sigma(id_t)("A")-1)\land(\sigma(id_t)("stc")\le{}\sigma(id_t)("B")-1)$.}\\
            
            Let us perform case analysis on $s.I(t)<upper(I_s(t))$ or
            $s.I(t)\ge{}upper(I_s(t))$:
            \begin{itemize}
            \item $s.I(t)<upper(I_s(t))$:\\
              By definition of \upSim, we have
              $s.I(t)=\sigma(id_t)("stc")$. By definition of \dwSitpn,
              we have $s'.I(t)=s.I(t)+1$. By definition of
              $s'.I(t)\in[a,b]$:
              
              $\Rightarrow{}a\le{}s'.I(t)\le{}b$.
              
              $\Rightarrow{}a\le{}s'.I(t)\land{}s'.I(t)\le{}b$

              $\Rightarrow{}a\le{}s.I(t)+1\land{}s.I(t)+1\le{}b$

              $\Rightarrow{}a-1\le{}s.I(t)\land{}s.I(t)\le{}b-1$

              By construction,
              ${<}\mathtt{time\_A\_value\Rightarrow}{}a{>}\in{}ipm_t$
              and
              ${<}\mathtt{time\_B\_value\Rightarrow}{}b{>}\in{}ipm_t$,
              and by property of stable $\sigma$, we have
              $\sigma(id_t)("A")=a$ and
              $\sigma(id_t)("B")=b$.

              Rewriting the goal with
              $\sigma(id_t)("A")=a$, $\sigma(id_t)("B")=b$ and
              $s.I(t)=\sigma(id_t)("stc")$:
              \qedbox{$a-1\le{}s.I(t)\land{}s.I(t)\le{}b-1$.}
              
            \item $s.I(t)\ge{}upper(I_s(t))$:\\
              In the case where $s.I(t)>upper(I_s(t))$, then
              $s.I(t)>b$. By definition of \dwSitpn, we have
              $s.I(t)=s'.I(t)=b$. Then, \qedbox{$b>b$ is a
                contradiction.}\\

              In the case where $s.I(t)=upper(I_s(t))$, then
              $s.I(t)=b$. By definition of \dwSitpn, we have
              $s'.I(t)=s.I(t)+1$.

              By definition of $s'.I(t)\in[a,b]$, we have
              $s'.I(t)\le{}b$:

              $\Rightarrow{}s.I(t)+1\le{}b$

              $\Rightarrow{}$ \qedbox{$b+1\le{}b$ is contradiction.}
            \end{itemize}
          \end{itemize}
          
        \item $I_s(t)=[a,\infty]$ where $a\in{}\mathbb{N}^{*}$:

          By construction ${<}\mathtt{transition\_type\Rightarrow{}}$
          $\mathtt{TEMP\_A\_INF}{>}\in{}gm_t$. By property of the
          elaboration relation, we have
          $\Delta(id_t)$$("tt")=\mathtt{TEMP\_A\_INF}$; thus we can
          simplify the term $\mathtt{checktc}$ as follows:
          \begin{equation}
            \label{eq:eq-checktc-temp-a-inf}
            \mathtt{checktc}(\Delta(id_t),\sigma(id_t))=(\sigma(id_t)("stc")\ge{}\sigma(id_t)("A")-1))
          \end{equation}

          Rewriting the goal with \eqref{eq:eq-checktc-temp-a-inf},
          and simplifying the goal:\\
          \fbox{$\sigma(id_t)("stc")\ge{}\sigma(id_t)("A")-1$.}

          From $s'.I(t)\in[a,\infty]$, we can deduce
          $a\le{}s'.I(t)$. Then, let us perform case analysis on
          $s.I(t)\le{}lower(I_s(t))$ or $s.I(t)>lower(I_s(t))$:
          \begin{itemize}
          \item $s.I(t)\le{}lower(I_s(t))$:

            By definition of \upSim, we have
            $s.I(t)=\sigma(id_t)("stc")$.

            By definition of \dwSitpn, we have $s'.I(t)=s.I(t)+1$:

            $\Rightarrow{}a\le{}s'.I(t)$

            $\Rightarrow{}a\le{}s.I(t)+1$

            $\Rightarrow{}a-1\le{}s.I(t)$

            By construction,
            ${<}\mathtt{time\_A\_value\Rightarrow}{}a{>}\in{}ipm_t$,
            and by property of stable $\sigma$, we have
            $\sigma(id_t)("A")=a$.

            Rewriting the goal with $\sigma(id_t)("A")=a$ and
            $s.I(t)=\sigma(id_t)("stc")$:\\ \qedbox{$a-1\le{}s.I(t)$.}
            
          \item $s.I(t)>lower(I_s(t))$:

            By definition of \upSim, we have
            $\sigma(id_t)("stc")=lower(I_s(t))=a$.

            By construction,
            ${<}\mathtt{time\_A\_value\Rightarrow}{}a{>}\in{}ipm_t$,
            and by property of stable $\sigma$, we have
            $\sigma(id_t)("A")=a$.

            Rewriting the goal with $\sigma(id_t)("stc")=a$ and
            $\sigma(id_t)("A")=a$: \qedbox{$a-1\le{}a$.}
          \end{itemize}
          
        \end{itemize}
      \end{enumerate}
    \end{enumerate}
  \end{enumerate}
\end{niproof}

\begin{lemma}[Falling Edge Equal Firable 2]
  \label{lem:fe-equal-firable-2}
  \fehyps{} then
  $\forall{}t\in{}T,id_t\in{}Comps(\Delta)~s.t.~\gamma(t)=id_t,$
  $\sigma'(id_t)("s\_firable")=\mathtt{true}\Rightarrow{}t\in{}Firable(s')$.
\end{lemma}

\begin{niproof}

  Given a $t\in{}T$ and $id_t\in{}Comps(\Delta)$
  s.t. $\gamma(t)=id_t$, and assuming that
  $\sigma'(id_t)("s\_firable")=\mathtt{true}$, let us
  show \fbox{$t\in{}Firable(s')$.}\\
  
  \exT{}

  By property of the $Inject_\downarrow$, the \hvhdl{}
  falling edge, the stabilize relations and \InCsCompT:
  \begin{equation}
    \label{eq:fe-eq-sfa-true}
    \sigma'(id_t)("sfa")=\sigma(id_t)("se")~.~\sigma(id_t)("scc")~.~\mathtt{checktc}(\Delta(id_t),\sigma(id_t))=\mathtt{true}
  \end{equation}

  From \eqref{eq:fe-eq-sfa-true}, we can deduce:
  \begin{eqnarray}
    \label{eq:fe-eq-se-true}\sigma(id_t)("se")=\mathtt{true}\\
    \label{eq:fe-eq-scc-true}\sigma(id_t)("scc")=\mathtt{true}\\
    \label{eq:fe-eq-checktc-true}\mathtt{checktc}(\Delta(id_t),\sigma(id_t))=\mathtt{true}
  \end{eqnarray}

  Term $\mathtt{checktc}(\Delta(id_t),\sigma(id_t))$ as the same
  definition as in Lemma~\nameref{lem:fe-equal-firable-1}.

  By definition of $t\in{}Firable(s')$, there are three points to prove:
  \begin{enumerate}
  \item \fbox{$t\in{}Sens(s'.M)$}
  \item \fbox{$t\notin{}T_i\lor{}s'.I(t)\in{}I_s(t)$}
  \item
    \fbox{\parbox{\lwidth}{$\forall{}c\in{}\mathcal{C},~\mathbb{C}(t,c)=1\Rightarrow{}s'.cond(c)=\mathtt{true}$
    and $\mathbb{C}(t,c)=-1\Rightarrow{}s'.cond(c)=\mathtt{false}$}}
  \end{enumerate}

  Let us prove these three points:
  \begin{enumerate}
  \item \fbox{$t\in{}Sens(s'.M)$}:

    By definition of \dwSitpn, we have $s.M=s'.M$. Rewriting the goal
    with $s.M=s'.M$: \fbox{$t\in{}Sens(s.M)$.}

    By definition of \upSim, we have
    $\sigma(id_t)("se")=\mathtt{true}\Leftrightarrow{}t\in{}Sens(s.M)$.

    \qedbox{$t\in{}Sens(s.M)$.}
    
  \item
    \fbox{\parbox{\lwidth}{$\forall{}c\in{}\mathcal{C},~\mathbb{C}(t,c)=1\Rightarrow{}s'.cond(c)=\mathtt{true}$
        and
        $\mathbb{C}(t,c)=-1\Rightarrow{}s'.cond(c)=\mathtt{false}$}}

    Given a $c\in\mathcal{C}$, there are two points to prove:
    \begin{enumerate}
    \item
      \fbox{$\mathbb{C}(t,c)=1\Rightarrow{}s'.cond(c)=\mathtt{true}$.}
    \item
      \fbox{$\mathbb{C}(t,c)=-1\Rightarrow{}s'.cond(c)=\mathtt{false}$.}
    \end{enumerate}

    Let us prove these two points:
    \begin{enumerate}
    \item Assuming that $\mathbb{C}(t,c)=1$, let us show
      \fbox{$s'.cond(c)=\mathtt{true}$.}

      By definition of \upSim, we have:
      \begin{equation}
        \sigma(id_t)("scc")=\prod\limits_{c\in{}conds(t)}
        \begin{cases}
          E_c(\tau,c) & if~\mathbb{C}(t,c)=1 \\
          \mathtt{not}(E_c(\tau,c)) & if~\mathbb{C}(t,c)=-1 \\
        \end{cases}=\mathtt{true}
        \label{eq:fe-eq-scc-prod-true}
      \end{equation}
      where
      $conds(t)=\{c\in\mathcal{C}~\vert~\mathbb{C}(t,c)=1\lor\mathbb{C}(t,c)=-1\}$.

      As $c\in{}conds(t)$ and $\mathbb{C}(t,c)=1$, and by definition
      of the product expression, we have:
      \begin{equation}
        \label{eq:fe-eq-scc-prod-true-decomp}
        E_c(\tau,c)~.~\prod\limits_{c'\in{}conds(t)\setminus\{c\}}
        \begin{cases}
          E_c(\tau,c') & if~\mathbb{C}(t,c')=1 \\
          \mathtt{not}(E_c(\tau,c')) & if~\mathbb{C}(t,c')=-1 \\
        \end{cases}=\mathtt{true}
      \end{equation}

      From \eqref{eq:fe-eq-scc-prod-true-decomp}, we can deduce that
      $E_c(\tau,c)=\mathtt{true}$.

      By definition of \dwSitpn, we have $s'.cond(c)=E_c(\tau,c)$.
      
      Rewriting the goal with $s'.cond(c)=E_c(\tau,c)$ and
      $E_c(\tau,c)=\mathtt{true}$: \qedbox{tautology.}
      
    \item Assuming that $\mathbb{C}(t,c)=-1$, let us show
      \fbox{$s'.cond(c)=\mathtt{false}$.}

      By definition of \upSim, we have:
      \begin{equation}
        \sigma(id_t)("scc")=\prod\limits_{c\in{}conds(t)}
        \begin{cases}
          E_c(\tau,c) & if~\mathbb{C}(t,c)=1 \\
          \mathtt{not}(E_c(\tau,c)) & if~\mathbb{C}(t,c)=-1 \\
        \end{cases}=\mathtt{true}
        \label{eq:fe-eq-scc-prod-false}
      \end{equation}
      where
      $conds(t)=\{c\in\mathcal{C}~\vert~\mathbb{C}(t,c)=1\lor\mathbb{C}(t,c)=-1\}$.

      As $c\in{}conds(t)$ and $\mathbb{C}(t,c)=-1$, and by definition
      of the product expression, we have:
      \begin{equation}
        \label{eq:fe-eq-scc-prod-false-decomp}
        \mathtt{not}~E_c(\tau,c)~.~\prod\limits_{c'\in{}conds(t)\setminus\{c\}}
        \begin{cases}
          E_c(\tau,c') & if~\mathbb{C}(t,c')=1 \\
          \mathtt{not}(E_c(\tau,c')) & if~\mathbb{C}(t,c')=-1 \\
        \end{cases}=\mathtt{true}
      \end{equation}

      From \eqref{eq:fe-eq-scc-prod-false-decomp}, we can deduce that
      $E_c(\tau,c)=\mathtt{false}$.

      By definition of \dwSitpn, we have $s'.cond(c)=E_c(\tau,c)$.
      
      Rewriting the goal with $s'.cond(c)=E_c(\tau,c)$ and
      $E_c(\tau,c)=\mathtt{false}$: \qedbox{tautology.}
    \end{enumerate}
    
  \item \fbox{$t\notin{}T_i\lor{}s'.I(t)\in{}I_s(t)$}

    Reasoning on
    $\mathtt{checktc}(\Delta(id_t),\sigma(id_t))=\mathtt{true}$, there
    are 3 cases:
    \begin{enumerate}
    \item
      $\bigg(\mathtt{not}~\sigma(id_t)("srtc")~.~[\dots]\bigg)=\mathtt{true}$\footnote{See
        equation~\eqref{eq:checktc} for the full definition}
    \item
      $\big(\sigma(id_t)("srtc")~.~\Delta(id_t)("tt")\neq\mathtt{NOT\_TEMP}~.~\sigma(id_t)("A")=1\big)=\mathtt{true}$
    \item
      $\big(\Delta(id_t)("tt")=\mathtt{NOT\_TEMP}\big)=\mathtt{true}$
    \end{enumerate}
    
    \begin{enumerate}
    \item $\bigg(\mathtt{not}~\sigma(id_t)("srtc")~.~[\dots]\bigg)=\mathtt{true}$:

      Then, we can deduce
      $\mathtt{not}~\sigma(id_t)("srtc")=\mathtt{true}$ and
      $[\dots]=\mathtt{true}$. From
      $\mathtt{not}~\sigma(id_t)("srtc")=\mathtt{true}$, we can deduce
      $\sigma(id_t)("srtc")=\mathtt{false}$, and from
      $[\dots]=\mathtt{true}$, we have three other cases:

      \begin{enumerate}
      \item
        $\big(\Delta(id_t)("tt")=\mathtt{TEMP\_A\_B}~.~(\sigma(id_t)("stc")\ge{}\sigma(id_t)("A")-1)~.~(\sigma(id_t)("stc")\le{}\sigma(id_t)("B")-1)\big)=\mathtt{true}$
      \item
        $(\Delta(id_t)("tt")=\mathtt{TEMP\_A\_A}~.~(\sigma(id_t)("stc")={}\sigma(id_t)("A")-1))=\mathtt{true}$
      \item
        $(\Delta(id_t)("tt")=\mathtt{TEMP\_A\_INF}~.~(\sigma(id_t)("stc")\ge{}\sigma(id_t)("A")-1))=\mathtt{true}$
      \end{enumerate}

      Let us prove the goal is these three contexts:
      \begin{enumerate}
      \item
        $\big(\Delta(id_t)("tt")=\mathtt{TEMP\_A\_B}~.~(\sigma(id_t)("stc")\ge{}\sigma(id_t)("A")-1)~.~(\sigma(id_t)("stc")\le{}\sigma(id_t)("B")-1)\big)=\mathtt{true}$:

        Then, converting boolean equalities into intuitionistic
        predicates, we have:
        \begin{itemize}
        \item $\Delta(id_t)("tt")=\mathtt{TEMP\_A\_B}$
        \item $\sigma(id_t)("stc")\ge{}\sigma(id_t)("A")-1$
        \item $\sigma(id_t)("stc")\le{}\sigma(id_t)("B")-1$
        \end{itemize}

        By property of the elaboration relation, and
        $\Delta(id_t)("tt")=\mathtt{TEMP\_A\_B}$, there exist
        $a,b\in\mathbb{N}^{*}$ s.t. $I_s(t)=[a,b]$. Let us take such
        an $a$ and $b$. Then, let us show \fbox{$s'.I(t)\in{}I_s(t)$.}

        Rewriting the goal with $I_s(t)=[a,b]$:
        \fbox{$s'.I(t)\in[a,b]$.}
        
        By construction, ${<}\mathtt{time\_A\_value\Rightarrow}{}a{>}$
        and ${<}\mathtt{time\_B\_value\Rightarrow}{}b{>}$, and by
        property of stable $\sigma$, we have $\sigma(id_t)("A")=a$ and
        $\sigma(id_t)("B")=b$.

        Rewriting the goal with $\sigma(id_t)("A")=a$ and
        $\sigma(id_t)("B")=b$, and by definition of $\in$:
        \fbox{$\sigma(id_t)("A")\le{}s'.I(t)\le{}\sigma(id_t)("B")$.}

        Now, let us perform case analysis on
        $s.I(t)\le{}upper(I_s(t))$ or $s.I(t)>upper(I_s(t))$:
        \begin{itemize}
        \item $s.I(t)\le{}upper(I_s(t))$:
          
          By definition of \upSim, we have $s.I(t)=\sigma(id_t)("stc")$.

          From $\sigma(id_t)("se")=\mathtt{true}$, we can deduce
          $t\in{}Sens(s.M)$, and from
          $\sigma(id_t)("srtc")=\mathtt{false}$, we can deduce
          $s.reset_t(t)=\mathtt{false}$. Then, by definition of
          \dwSitpn, we have $s'.I(t)=s.I(t)+1$.

          $\Rightarrow{}$
          \fbox{$\sigma(id_t)("A")\le{}s.I(t)+1\le{}\sigma(id_t)("B")$}
          (by $s'.I(t)=s.I(t)+1$)

          $\Rightarrow{}$
          \fbox{$\sigma(id_t)("A")\le{}\sigma(id_t)("stc")+1\le{}\sigma(id_t)("B")$}
          (by $s.I(t)=\sigma(id_t)("stc")$)

          $\Rightarrow{}$
          \qedbox{$\sigma(id_t)("A")-1\le{}\sigma(id_t)("stc")\le{}\sigma(id_t)("B")-1$}
          
        \item $s.I(t)>upper(I_s(t))$:

          By definition of \upSim, we have
          $\sigma(id_t)("stc")=upper(I_s(t))=b$.

          Then, from $\sigma(id_t)("stc")\le{}\sigma(id_t)("B")-1$,
          $\sigma(id_t)("stc")=upper(I_s(t))=b$ and
          $\sigma(id_t)("B")=b$, we can deduce the following
          contradiction:\\
          \qedbox{$\sigma(id_t)("B")\le{}\sigma(id_t)("B")-1$.}
        \end{itemize}
      \item
        $(\Delta(id_t)("tt")=\mathtt{TEMP\_A\_A}~.~(\sigma(id_t)("stc")={}\sigma(id_t)("A")-1))=\mathtt{true}$:

        Then, converting boolean equalities into intuitionistic
        predicates, we have:
        \begin{itemize}
        \item $\Delta(id_t)("tt")=\mathtt{TEMP\_A\_A}$
        \item $\sigma(id_t)("stc")=\sigma(id_t)("A")-1$
        \end{itemize}

        By property of the elaboration relation, and
        $\Delta(id_t)("tt")=\mathtt{TEMP\_A\_A}$, there exist
        $a\in\mathbb{N}^{*}$ s.t. $I_s(t)=[a,a]$. Let us take such an
        $a$. Then, let us show \fbox{$s'.I(t)\in{}I_s(t)$.}

        Rewriting the goal with $I_s(t)=[a,a]$:
        \fbox{$s'.I(t)\in[a,a]$.}
        
        By construction,
        ${<}\mathtt{time\_A\_value\Rightarrow}{}a{>}$, and by property
        of stable $\sigma$, we have $\sigma(id_t)("A")=a$.

        Rewriting the goal with $\sigma(id_t)("A")=a$, unfolding the
        definition of $\in$, and simplifying the goal:
        \fbox{$s'.I(t)=\sigma(id_t)("A")$.}

        Now, let us perform case analysis on
        $s.I(t)\le{}upper(I_s(t))$ or $s.I(t)>upper(I_s(t))$:
        \begin{itemize}
        \item $s.I(t)\le{}upper(I_s(t))$:
          
          By definition of \upSim, we have $s.I(t)=\sigma(id_t)("stc")$.

          From $\sigma(id_t)("se")=\mathtt{true}$, we can deduce
          $t\in{}Sens(s.M)$, and from
          $\sigma(id_t)("srtc")=\mathtt{false}$, we can deduce
          $s.reset_t(t)=\mathtt{false}$. Then, by definition of
          \dwSitpn, we have $s'.I(t)=s.I(t)+1$.

          $\Rightarrow{}$
          \fbox{$s.I(t)+1=\sigma(id_t)("A")$} (by $s'.I(t)=s.I(t)+1$)

          $\Rightarrow{}$
          \fbox{$\sigma(id_t)("stc")+1=\sigma(id_t)("A")$}
          (by $s.I(t)=\sigma(id_t)("stc")$)

          $\Rightarrow{}$
          \qedbox{$\sigma(id_t)("stc")=\sigma(id_t)("A")-1$}
          
        \item $s.I(t)>upper(I_s(t))$:

          By definition of \upSim, we have
          $\sigma(id_t)("stc")=upper(I_s(t))=a$.

          Then, from $\sigma(id_t)("stc")={}\sigma(id_t)("A")-1$,
          $\sigma(id_t)("stc")=upper(I_s(t))=a$,
          $\sigma(id_t)("A")=a$, and $a\in\mathbb{N}^{*}$, we can
          deduce the following
          contradiction:\\
          \qedbox{$\sigma(id_t)("A")=\sigma(id_t)("A")-1$.}
        \end{itemize}
        
      \item
        $(\Delta(id_t)("tt")=\mathtt{TEMP\_A\_INF}~.~(\sigma(id_t)("stc")\ge\sigma(id_t)("A")-1))=\mathtt{true}$:

        Then, converting boolean equalities into intuitionistic
        predicates, we have:
        \begin{itemize}
        \item $\Delta(id_t)("tt")=\mathtt{TEMP\_A\_INF}$
        \item $\sigma(id_t)("stc")\ge\sigma(id_t)("A")-1$
        \end{itemize}

        By property of the elaboration relation, and
        $\Delta(id_t)("tt")=\mathtt{TEMP\_A\_INF}$, there exist
        $a\in\mathbb{N}^{*}$ s.t. $I_s(t)=[a,\infty]$. Let us take
        such an $a$. Then, let us show \fbox{$s'.I(t)\in{}I_s(t)$.}

        Rewriting the goal with $I_s(t)=[a,\infty]$:
        \fbox{$s'.I(t)\in[a,\infty]$.}
        
        By construction,
        ${<}\mathtt{time\_A\_value\Rightarrow}{}a{>}$, and by property
        of stable $\sigma$, we have $\sigma(id_t)("A")=a$.

        Rewriting the goal with $\sigma(id_t)("A")=a$, unfolding the
        definition of $\in$, and simplifying the goal:
        \fbox{$\sigma(id_t)("A")\le{}s'.I(t)$.}

        Now, let us perform case analysis on
        $s.I(t)\le{}lower(I_s(t))$ or $s.I(t)>lower(I_s(t))$:
        \begin{itemize}
        \item $s.I(t)\le{}lower(I_s(t))$:
          
          By definition of \upSim, we have $s.I(t)=\sigma(id_t)("stc")$.

          From $\sigma(id_t)("se")=\mathtt{true}$, we can deduce
          $t\in{}Sens(s.M)$, and from
          $\sigma(id_t)("srtc")=\mathtt{false}$, we can deduce
          $s.reset_t(t)=\mathtt{false}$. Then, by definition of
          \dwSitpn, we have $s'.I(t)=s.I(t)+1$.

          $\Rightarrow{}$
          \fbox{$\sigma(id_t)("A")\le{}s.I(t)+1$} (by $s'.I(t)=s.I(t)+1$)

          $\Rightarrow{}$
          \fbox{$\sigma(id_t)("A")\le\sigma(id_t)("stc")+1$} (by
          $s.I(t)=\sigma(id_t)("stc")$)

          $\Rightarrow{}$
          \qedbox{$\sigma(id_t)("A")-1\le\sigma(id_t)("stc")$}
          
        \item $s.I(t)>lower(I_s(t))$:

          By definition of \upSim, we have
          $\sigma(id_t)("stc")=lower(I_s(t))=a$.

          From $\sigma(id_t)("se")=\mathtt{true}$, we can deduce
          $t\in{}Sens(s.M)$, and from
          $\sigma(id_t)("srtc")=\mathtt{false}$, we can deduce
          $s.reset_t(t)=\mathtt{false}$. Then, by definition of
          \dwSitpn, we have $s'.I(t)=s.I(t)+1$.

          $\Rightarrow{}$ \fbox{$\sigma(id_t)("A")\le{}s.I(t)+1$} (by
          $s'.I(t)=s.I(t)+1$)

          $\Rightarrow{}$ \fbox{$a\le{}s.I(t)+1$} (by
          $\sigma(id_t)("A")=a$)
          
          $\Rightarrow{}$ \fbox{$a<s.I(t)$}

          $\Rightarrow{}$ \qedbox{$lower(I_s(t))<s.I(t)$}
        \end{itemize}
      \end{enumerate}

    \item
      $\big(\sigma(id_t)("srtc")~.~\Delta(id_t)("tt")\neq\mathtt{NOT\_TEMP}~.~\sigma(id_t)("A")=1\big)=\mathtt{true}$

      Then, converting boolean equalities into intuitionistic predicates, we have:
      \begin{itemize}
      \item $\sigma(id_t)("srtc")=\mathtt{true}$
      \item $\Delta(id_t)("tt")\neq\mathtt{NOT\_TEMP}$
      \item $\sigma(id_t)("A")=1$
      \end{itemize}

      By property of the elaboration relation, and
      $\Delta(id_t)("tt")\neq\mathtt{NOT\_TEMP}$, there exist an
      $a\in\mathbb{N}^{*}$ and a $ni\in\mathbb{N}^{*}\sqcup\{\infty\}$
      s.t. $I_s(t)=[a,ni]$. Let us take such an $a$ and $ni$.

      By construction,
      ${<}\mathtt{time\_A\_value\Rightarrow}{}a{>}\in{}ipm_t$, and by
      property of stable $\sigma$, we have
      $\sigma(id_t)("A")=a$. Thus, we can deduce $a=1$ and
      $I_s(t)=[1,ni]$.

      By definition of \upSim, from
      $\sigma(id_t)("se")=\mathtt{true}$, we can deduce
      $t\in{}Sens(s.M)$, and from
      $\sigma(id_t)("srtc")=\mathtt{true}$, we can deduce
      $s.reset_t(t)=\mathtt{true}$.

      By definition of \dwSitpn, $t\in{}Sens(s.M)$ and
      $s.reset_t(t)=\mathtt{true}$, we have $s'.I(t)=1$.

      Now, let us show \fbox{$s'.I(t)\in{}I_s(t)$}.

      Rewriting the goal with $s'.I(t)=1$ and $I_s(t)=[1,ni]$:
      \qedbox{$1\in[1,ni]$.}
    \item
      $\big(\Delta(id_t)("tt")=\mathtt{NOT\_TEMP}\big)=\mathtt{true}$

      Let us show \fbox{$t\notin{}T_i$.}
      
      By property of the elaboration relation and
      $\Delta(id_t)("tt")=\mathtt{NOT\_TEMP}$, we have
      \qedbox{$t\notin{}T_i$.}
    \end{enumerate}
  \end{enumerate}
\end{niproof}

\begin{lemma}[Falling Edge Equal Not Firable]
  \label{lem:fe-equal-not-firable}
  \fehyps{} then
  $\forall{}t\in{}T,id_t\in{}Comps(\Delta)~s.t.~\gamma(t)=id_t,$
  $t\in{}Firable(s')\Leftrightarrow\sigma'(id_t)("s\_firable")=\mathtt{true}$.
\end{lemma}

\begin{proof}
  Proving the above lemma is trivial by appealing to
  Lemma~\nameref{lem:fe-equal-firable} and by reasoning on
  contrapositives.
\end{proof}

%%% Local Variables:
%%% mode: latex
%%% TeX-master: "../../main"
%%% End:
