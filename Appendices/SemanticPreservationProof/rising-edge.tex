%%%%%%%%%%%%%%%%%%%%%%%%%%%%%%%%%%%%%%%%%%%%
%%%%%%%%%% RISING EDGE HYPOTHESES %%%%%%%%%%
%%%%%%%%%%%%%%%%%%%%%%%%%%%%%%%%%%%%%%%%%%%%

\begin{definition}[Rising edge hypotheses]
  \label{def:re-hyps}
  Given an $sitpn\in{}SITPN$, $b\in{}P\rightarrow\mathbb{N}$,
  $d\in{}design$, $\gamma\in{}WM(sitpn,d)$,
  $E_c\in\mathbb{N}\rightarrow\mathcal{C}\rightarrow\mathbb{B}$,
  $\Delta\in{}ElDesign$,
  $E_p\in\mathbb{N}\rightarrow{}Ins(\Delta)\rightarrow{}value$,
  $\tau\in\mathbb{N}$, $s,s'\in{}S(sitpn)$,
  $\sigma_e,\sigma,\sigma_i,\sigma_\uparrow,\sigma'\in\Sigma$, assume
  that:
  \begin{itemize}
  \item $\lfloor{}sitpn\rfloor_b=(d,\gamma)$ and
    $\gamma\vdash{}E_p\stackrel{env}{=}E_c$ and
    $\mathcal{D}_\mathcal{H},\emptyset\vdash\mathrm{d}\srarrow{elab}{\fontsize{7}{7}\selectfont}\Delta,\sigma_e$
  \item $\gamma\vdash{}s\stackrel{\downarrow}{\approx}\sigma$
  \item
    $E_c,\tau\vdash{}s\srarrow{\uparrow}{\fontsize{7}{7}\selectfont}s'$
  \item $\mathtt{Inject}(\sigma, E_p, \tau, \sigma_i)$ and
    $\mathcal{D}_{\mathcal{H}},\Delta,\sigma_i\vdash\mathrm{d.cs}\xrightarrow{\uparrow}\sigma_\uparrow$
    and
    $\mathcal{D}_{\mathcal{H}},\Delta,\sigma_\uparrow\vdash\mathrm{d.cs}\xrightarrow{\rightsquigarrow}\sigma'$
  \item State $\sigma$ is a stable design state:
    $\mathcal{D}_{\mathcal{H}},\Delta,\sigma\vdash\mathrm{d.cs}\xrightarrow{comb}\sigma$
  \end{itemize}
\end{definition}

\def\rehyps{For all $sitpn$, $b$, $d$, $\gamma$, $E_c$, $E_p$, $\tau$,
  $\Delta$, $\sigma_e$, $s$, $s'$, $\sigma$, $\sigma_i$,
  $\sigma_\uparrow$, $\sigma'$ that verify the hypotheses of
  Definition~\ref{def:re-hyps},}

\subsection{Rising edge and Marking}
\label{sec:re-marking}

%%%%%%%%%%%%%%%%%%%%%%%%%%%%%%%%%%%%%%%%%%%%%%%%%%%%%
%%%%%%%%%% RISING EDGE EQUAL MARKING LEMMA %%%%%%%%%%
%%%%%%%%%%%%%%%%%%%%%%%%%%%%%%%%%%%%%%%%%%%%%%%%%%%%%

\begin{lemma}[Rising edge equal marking]
  \label{lem:re-equal-marking}
  \rehyps{} then $\forall{}p,id_p~s.t.~\gamma(p)=id_p$,
  $s'.M(p)=\sigma'(id_p)(\texttt{s\_marking})$.
\end{lemma}

\begin{niproof}

  Given a $p\in{}P$, let us show
  \fbox{$s'.M(p)=\sigma'(id_p)(\texttt{s\_marking})$.}

  \exP{}
  
  By definition of the SITPN state transition relation on rising edge:
  \begin{equation}\label{eq:re-eq-marking-eqmp}
    s'.M(p)=s.M(p)-\sum\limits_{t\in{}Fired(s)}pre(p,t)+\sum\limits_{t\in{}Fired(s)}post(t,p)
  \end{equation}

  By property of the $\mathtt{Inject}$, the \hvhdl{} rising
  edge and the stabilize relations, \InCsCompP{}, and through the
  examination of the \texttt{marking} process defined in the place
  design architecture, we can deduce:
  \begin{equation}\label{eq:re-eq-marking-eqsm}
    \begin{split}
      \sigma'(id_p)(\texttt{sm})=\sigma(id_p)(\texttt{sm})-\sigma(id_p)(\texttt{s\_output\_token\_sum})\\
      +\sigma(id_p)(\texttt{s\_input\_token\_sum})
    \end{split}
  \end{equation}

  Rewriting the goal with \ref{eq:re-eq-marking-eqmp} and
  \ref{eq:re-eq-marking-eqsm},
  
  \fbox{
    \begin{tabular}{@{}c@{}}
      $s.M(p)-\sum\limits_{t\in{}Fired(s)}pre(p,t)+\sum\limits_{t\in{}Fired(s)}post(t,p)$\\
      $=$ \\
      $\sigma(id_p)(\texttt{sm})-\sigma(id_p)(\texttt{sots})
      +\sigma(id_p)(\texttt{sits})$ \\
    \end{tabular}
  }\\
  
  By definition of the \nameref{def:full-post-fe-state-sim} relation,
  we can deduce $s.M(p)=\sigma(id_p)(\texttt{sm})$,
  $\sum\limits_{t\in{}Fired(s)}pre(p,t)=\sigma(id_p)(\texttt{sots})$ and
  $\sum\limits_{t\in{}Fired(s)}post(t,p)=\sigma(id_p)(\texttt{sits})$, and thus, 

  \qedbox{
    \begin{tabular}{c}
      $s.M(p)-\sum\limits_{t\in{}Fired(s)}pre(p,t)+\sum\limits_{t\in{}Fired(s)}post(t,p)$ \\
      $=$ \\
      $\sigma(id_p)(\texttt{sm})-\sigma(id_p)(\texttt{sots})+\sigma(id_p)(\texttt{sits})$ \\
    \end{tabular}
  }
\end{niproof}

%%%%%%%%%%%%%%%%%%%%%%%%%%%%%%%%%%%%%%%%%%%%%%%%%%%%%%%%%%%%%%
%%%%%%%%%% RISING EDGE EQUAL CONDITION COMBINATION  %%%%%%%%%%
%%%%%%%%%%%%%%%%%%%%%%%%%%%%%%%%%%%%%%%%%%%%%%%%%%%%%%%%%%%%%%

\subsection{Rising edge and conditions}
\label{sec:re-cond-comb}

\begin{lemma}[Rising edge equal condition combination]
  \label{lem:re-equal-cond-comb}
  \rehyps{} then\\
  $\forall{}t\in{}T,id_t\in{}Comps(\Delta)~s.t.~\gamma(t)=id_t,$\\
  $\sigma'(id_t)(\texttt{s\_condition\_combination})=
  \prod\limits_{c\in{}conds(t)}
  \begin{cases}
    E_c(\tau,c) & if~\mathbb{C}(t,c)=1 \\
    \mathtt{not}(E_c(\tau,c)) & if~\mathbb{C}(t,c)=-1 \\
  \end{cases}$\\
  where
  $conds(t)=\{c\in\mathcal{C}~\vert~\mathbb{C}(t,c)=1\lor\mathbb{C}(t,c)=-1\}$.
\end{lemma}

\begin{niproof}
  Given a $t$ and an $id_t$ s.t. $\gamma(t)=id_t$, let us show\\
  \fbox{$\sigma'(id_t)(\texttt{s\_condition\_combination})=
    \prod\limits_{c\in{}conds(t)}
    \begin{cases}
      E_c(\tau,c) & if~\mathbb{C}(t,c)=1 \\
      \mathtt{not}(E_c(\tau,c)) & if~\mathbb{C}(t,c)=-1 \\
    \end{cases}$.}\\

  \exT

  \noindent By property of the \hvhdl{} stabilize relation,
  \InCsCompT{}, and through the examination of the
  \texttt{condition_evaluation} process defined in the transition
  design architecture, we can deduce:
  \begin{equation}
    \sigma'(id_t)(\texttt{scc})=\prod\limits_{i=0}^{\Delta(id_t)(\texttt{conditions\_number})-1}\sigma'(id_t)(\texttt{input\_conditions})[i]\label{eq:re-eq-cc-eqscc}
  \end{equation}

  \noindent{}Rewriting the goal with \ref{eq:re-eq-cc-eqscc},\\
  \fbox{$\prod\limits_{i=0}^{\Delta(id_t)(\texttt{cn})-1}\sigma'(id_t)(\texttt{ic})[i]=
    \prod\limits_{c\in{}conds(t)}
    \begin{cases}
      E_c(\tau,c) & if~\mathbb{C}(t,c)=1 \\
      \mathtt{not}(E_c(\tau,c)) & if~\mathbb{C}(t,c)=-1 \\
    \end{cases}$.}\\

  \noindent{}Let us perform case analysis on $conds(t)$; there are two cases:

  \begin{itemize}
  \item \textbf{CASE} $conds(t)=\emptyset$:
    \fbox{$\prod\limits_{i=0}^{\Delta(id_t)(\texttt{cn})-1}\sigma'(id_t)(\texttt{ic})[i]=\mathtt{true}$.}\\
    
    By construction, ${<}\mathtt{cn\Rightarrow{}1}{>}\in{}g_t$ and
    ${<}\mathtt{ic(0)\Rightarrow{}true}{>}\in{}i_t$.

    By property of the stabilize relation,
    ${<}\mathtt{cn\Rightarrow{}1}{>}\in{}g_t$ and
    ${<}$\texttt{ic(0)}$\mathtt{\Rightarrow{}true}{>}\in{}i_t$, we can
    deduce $\Delta(id_t)(\texttt{cn})=1$ and
    $\sigma'(id_t)(\texttt{ic})[0]=\mathtt{true}$.

    Rewriting the goal with $\Delta(id_t)(\texttt{cn})=1$ and
    $\sigma'(id_t)(\texttt{ic})[0]=\mathtt{true}$, \qedbox{tautology.}
    
  \item \textbf{CASE} $conds(t)\neq\emptyset$:\\
    By construction,
    ${<}\mathtt{cn\Rightarrow{}\vert{}conds(t)\vert}{>}\in{}g_t$, and
    by property of the stabilize relation, we can deduce
    $\Delta(id_t)(\texttt{cn})=\vert{}conds(t)\vert$.
    
    Rewriting the goal with $\Delta(id_t)(\texttt{cn})=\vert{}conds(t)\vert$:\\
    \fbox{$\prod\limits_{i=0}^{\vert{}conds(t)\vert-1}\sigma'(id_t)(\texttt{ic})[i]=
      \prod\limits_{c\in{}conds(t)}
      \begin{cases}
        E_c(\tau,c) & if~\mathbb{C}(t,c)=1 \\
        \mathtt{not}(E_c(\tau,c)) & if~\mathbb{C}(t,c)=-1 \\
      \end{cases}$}\\

    There exists a mapping, given by the transformation function,
    between the set $conds(t)$ and the indexes of
    $[0,\vert{}conds(t){}\vert-1]$.

    Let $\beta\in{}conds(t)\rightarrow[0,\vert{}conds(t){}\vert-1]$ be
    this mapping.

    To prove the current goal, it suffices to prove that for all
    condition $c\in{}conds(t)$, we have

    \begin{frameb}
      $\bigg(\begin{cases}
        E_c(\tau,c) & if~\mathbb{C}(t,c)=1 \\
        \mathtt{not}(E_c(\tau,c)) & if~\mathbb{C}(t,c)=-1 \\
      \end{cases}\bigg)=\sigma'(id_t)(\texttt{ic})[\beta(c)]$
    \end{frameb}

    Given a $c\in{}conds(t)$, let us show the above goal.
    
    By construction, for all $c\in{}conds(t)$, there exists an
    $id_c\in{}Ins(\Delta)$ such that
    \begin{itemize}
    \item $\gamma(c)=id_c$
    \item $\mathbb{C}(t,c)=1$ implies
      ${<}\mathtt{ic(}\beta(c)\mathtt{)\Rightarrow{}id_c}{>}\in{}i_t$
    \item $\mathbb{C}(t,c)=-1$ implies
      ${<}\mathtt{ic(}\beta(c)\mathtt{)\Rightarrow{}not~id_c}{>}\in{}i_t$
    \end{itemize}

    Let us take such an $id_c$ with the above properties.

    By definition of $c\in{}conds(t)$, we have
    $\mathbb{C}(t,c)=1\lor\mathbb{C}(t,c)=-1$. Let us perform case
    analysis on $\mathbb{C}(t,c)=1\lor\mathbb{C}(t,c)=-1$:

    \begin{itemize}
    \item \textbf{CASE} $\mathbb{C}(t,c)=1$:

      In that case, we must show:
      \fbox{$E_c(\tau,c)=\sigma'(id_t)(\texttt{ic})[\beta(c)]$}

      By assumption, we have
      ${<}\mathtt{ic(}\beta(c)\mathtt{)\Rightarrow{}id_c}{>}\in{}i_t$
      and by property of the stabilize relation, we can deduce
      $\sigma(id_t)(\texttt{ic})[\beta(c)]=\sigma'(id_c)$.

      Rewriting the goal with
      $\sigma(id_t)(\texttt{ic})[\beta(c)]=\sigma'(id_c)$:

      \fbox{$E_c(\tau,c)=\sigma'(id_c)$}

      By property of the $\mathtt{Inject}$ relation and
      $id_c\in{}Ins(\Delta)$, we can deduce
      $\sigma'(id_c)=E_p(\tau)(id_c)$.

      By property of $\gamma\vdash{}E_p\stackrel{env}{=}E_c$, we can
      deduce $E_p(\tau)(id_c)=E_c(\tau,c)$.

      Rewriting the goal with $\sigma'(id_c)=E_p(\tau)(id_c)$ and
      $E_p(\tau)(id_c)=E_c(\tau,c)$:
      \fbox{$E_c(\tau,c)=E_c(\tau,c)$}, then \qedbox{tautology.}

    \item \textbf{CASE} $\mathbb{C}(t,c)=-1$:

      In that case, we must show:
      \fbox{$\mathtt{not}~E_c(\tau,c)=\sigma'(id_t)(\texttt{ic})[\beta(c)]$}

      By assumption, we have
      ${<}\mathtt{ic(}\beta(c)\mathtt{)\Rightarrow{}not~id_c}{>}\in{}i_t$
      and by property of the stabilize relation, we can deduce
      $\sigma(id_t)(\texttt{ic})[\beta(c)]=\mathtt{not}~\sigma'(id_c)$.

      Rewriting the goal with
      $\sigma(id_t)(\texttt{ic})[\beta(c)]=\mathtt{not}~\sigma'(id_c)$:

      \fbox{$\mathtt{not}~E_c(\tau,c)=\mathtt{not}~\sigma'(id_c)$}

      By property of the $\mathtt{Inject}$ relation and
      $id_c\in{}Ins(\Delta)$, we can deduce
      $\sigma'(id_c)=E_p(\tau)(id_c)$.

      By property of $\gamma\vdash{}E_p\stackrel{env}{=}E_c$, we can
      deduce $E_p(\tau)(id_c)=E_c(\tau,c)$.

      Rewriting the goal with $\sigma'(id_c)=E_p(\tau)(id_c)$ and
      $E_p(\tau)(id_c)=E_c(\tau,c)$:
      \fbox{$\mathtt{not}~E_c(\tau,c)=\mathtt{not}~E_c(\tau,c)$}, then
      \qedbox{tautology.}
      
    \end{itemize}
    
  \end{itemize}
  
\end{niproof}

\begin{lemma}[Rising edge equal conditions]
  \label{lem:re-equal-cond}
  \rehyps{} then\\
  $\forall{}c\in\mathcal{C},id_c\in{}Ins(\Delta)$
  s.t. $\gamma(c)=id_c$, $\sigma'(id_c)=E_c(\tau,c)$.
\end{lemma}

\begin{niproof}
  Given a $c\in\mathcal{C}$ and an $id_c\in{}Ins(\Delta)$ such that
  $\gamma(c)=id_c$, let us show
  \begin{frameb}
    $\sigma'(id_c)=E_c(\tau,c)$
  \end{frameb}

  By property of the $\mathtt{Inject}$ relation and
  $id_c\in{}Ins(\Delta)$, we can deduce
  $\sigma'(id_c)=E_p(\tau)(id_c)$.

  By property of $\gamma\vdash{}E_p\stackrel{env}{=}E_c$, we can
  deduce $E_p(\tau)(id_c)=E_c(\tau,c)$.

  Rewriting the goal with $\sigma'(id_c)=E_p(\tau)(id_c)$ and
  $E_p(\tau)(id_c)=E_c(\tau,c)$, \qedbox{tautology}.
\end{niproof}

\subsection{Rising edge and time counters}
\label{sec:re-tc}

\begin{lemma}[Rising edge equal time counters]
  \label{lem:re-equal-tc}
  \rehyps{} then\\
  $\forall{}t\in{}T_i,id_t\in{}Comps(\Delta)~s.t.~\gamma(t)=id_t$,\\
  $\big(u(I_s(t))=\infty\land{}s'.I(t)\le{}l(I_s(t))\Rightarrow$
  $s'.I(t)=\sigma'(id_t)(\texttt{s\_time\_counter})\big)$\\
  $\land\big(u(I_s(t))=\infty\land{}s'.I(t)>{}l(I_s(t))\Rightarrow$
  $\sigma'(id_t)(\texttt{s\_time\_counter})=l(I_s(t))\big)$\\
  $\land\big(u(I_s(t))\neq\infty\land{}s'.I(t)>{}u(I_s(t))\Rightarrow$
  $\sigma'(id_t)(\texttt{s\_time\_counter})=u(I_s(t))\big)$\\
  $\land\big(u(I_s(t))\neq\infty\land{}s'.I(t)\le{}u(I_s(t))\Rightarrow$
  $s'.I(t)=\sigma'(id_t)(\texttt{s\_time\_counter})\big)$.
\end{lemma}

\begin{niproof}
  Given a $t\in{}T_i$ and an $id_t\in{}Comps(\Delta)$ s.t. $\gamma(t)=id_t$, let us show\\
  \noindent\fbox{\parbox{\lwidth}{$\big(u(I_s(t))=\infty\land{}s'.I(t)\le{}l(I_s(t))\Rightarrow$
      $s'.I(t)=\sigma'(id_t)(\texttt{s\_time\_counter})\big)$\\
      $\land\big(u(I_s(t))=\infty\land{}s'.I(t)>{}l(I_s(t))\Rightarrow$
      $\sigma'(id_t)(\texttt{s\_time\_counter})=l(I_s(t))\big)$\\
      $\land\big(u(I_s(t))\neq\infty\land{}s'.I(t)>{}u(I_s(t))\Rightarrow$
      $\sigma'(id_t)(\texttt{s\_time\_counter})=u(I_s(t))\big)$\\
      $\land\big(u(I_s(t))\neq\infty\land{}s'.I(t)\le{}u(I_s(t))\Rightarrow$
      $s'.I(t)=\sigma'(id_t)(\texttt{s\_time\_counter})\big)$}}\\

  \exT{}

  Then, there are 4 points to show:

  \begin{enumerate}
  \item\label{it:re-eq-tc-fst}
    \fbox{$u(I_s(t))=\infty\land{}s'.I(t)\le{}l(I_s(t))\Rightarrow{}s'.I(t)=\sigma'(id_t)(\texttt{s\_time\_counter})$}\\
    
    Assuming that $u(I_s(t))=\infty$ and
    $s'.I(t)\le{}l(I_s(t))$, let us show\\
    \fbox{$s'.I(t)=\sigma'(id_t)(\texttt{s\_time\_counter})$.}

    By property of the $\mathtt{Inject}$, \hvhdl{} rising edge and
    stabilize relations, \InCsCompT, and through the examination of
    the \texttt{time_counter} process defined in the transition design
    architecture, we can deduce
    $\sigma'(id_t)(\texttt{stc})=\sigma(id_t)(\texttt{stc})$.

    By property of
    $\gamma\vdash{}s\stackrel{\downarrow}{\approx}\sigma$, we can
    deduce $s.I(t)=\sigma(id_t)(\texttt{stc})$.

    Rewriting the goal with $\sigma'(id_t)(\texttt{stc})=\sigma(id_t)(\texttt{stc})$
    and $s.I(t)=\sigma(id_t)(\texttt{stc})$, \qedbox{tautology.}
    
  \item
    \fbox{$u(I_s(t))=\infty\land{}s'.I(t)>{}l(I_s(t))\Rightarrow
      \sigma'(id_t)(\texttt{s\_time\_counter})=l(I_s(t)$.}

    \noindent{}Proved in the same fashion as \ref{it:re-eq-tc-fst}.
  \item
    \fbox{$u(I_s(t))\neq\infty\land{}s'.I(t)>{}u(I_s(t))\Rightarrow
      \sigma'(id_t)(\texttt{s\_time\_counter})=u(I_s(t)$.}

    \noindent{}Proved in the same fashion as \ref{it:re-eq-tc-fst}.
    
  \item
    \fbox{$u(I_s(t))\neq\infty\land{}s'.I(t)\le{}u(I_s(t))\Rightarrow{}s'.I(t)=\sigma'(id_t)(\texttt{s\_time\_counter})$}
    
    \noindent{}Proved in the same fashion as \ref{it:re-eq-tc-fst}.
  \end{enumerate}
  
\end{niproof}

\subsection{Rising edge and reset orders}
\label{sec:re-reset-orders}

\begin{lemma}[Rising edge equal reset orders]
  \label{lem:re-equal-reset-orders}
  \rehyps{} then\\
  $\forall{}t\in{}T_i,id_t\in{}Comps(\Delta)~s.t.~\gamma(t)=id_t,$
  $s'.reset_t(t)=\sigma'(id_t)(\texttt{s\_reinit\_time\_counter})$
\end{lemma}

\begin{niproof}
  Given a $t\in{}T_i$ and an $id_t\in{}Comps(\Delta)$
  s.t. $\gamma(t)=id_t$, let us show\\
  \fbox{$s'.reset_t(t)=\sigma'(id_t)(\texttt{s\_reinit\_time\_counter})$.}

  \exT

  By property of the \hvhdl{} stabilize relation, \InCsCompT, and
  through the examination of the
  \texttt{reinit_time_counter_evaluation} process defined in the
  transition design architecture, we can deduce:
  \begin{equation}
    \sigma'(id_t)(\texttt{srtc})=\sum\limits_{i=0}^{\Delta(id_t)(\texttt{input\_arcs\_number})-1}\sigma'(id_t)(\texttt{reinit\_time})[i]\label{eq:eq-srtc-prod}
  \end{equation}

  Rewriting the goal with \eqref{eq:eq-srtc-prod},
  \fbox{$s'.reset_t(t)=\sum\limits_{i=0}^{\Delta(id_t)(\texttt{ian})-1}\sigma'(id_t)(\texttt{rt})[i]$.}
  
  Let us perform case analysis on $input(t)$; there are two cases:

  \begin{itemize}
  \item \textbf{CASE} $input(t)=\emptyset$:

    By construction,
    ${<}\mathtt{input\_arcs\_number\Rightarrow{}1}{>}\in{}g_t$, and
    by property of the elaboration relation, we can deduce
    $\Delta(id_t)(\texttt{ian})=1$.

    By construction, there exists an $id_{ft}\in{}Sigs(\Delta)$ s.t.
    ${<}\mathtt{reinit\_time(0)\Rightarrow{}id_{ft}}{>}\in{}i_t$ and
    ${<}\mathtt{fired\Rightarrow{}id_{ft}}{>}\in{}o_t$, and by
    property of the stabilize relation and \InCsCompT, we can deduce
    $\sigma'(id_t)(\texttt{rt})[0]=\sigma'(id_{ft})=\sigma'(id_t)(\texttt{fired})$.

    Rewriting the goal with $\Delta(id_t)(\texttt{ian})=1$ and
    $\sigma'(id_t)(\texttt{rt})[0]=\sigma'(id_{ft})=\sigma'(id_t)(\texttt{fired})$:
    \fbox{$s'.reset_t(t)=\sigma'(id_t)(\texttt{fired})$.}
    
    By property of the stabilize relation, \InCsCompT{}, and through
    the examination of the \texttt{fired_evaluation} process, we can deduce:
    \begin{equation}
      \sigma'(id_t)(\texttt{fired})=\sigma'(id_t)(\texttt{s\_firable})~.~\sigma'(id_t)(\texttt{s\_priority\_combination})\label{eq:fired-at-re}
    \end{equation}

    Rewriting the goal with \eqref{eq:fired-at-re}:\\
    \fbox{$s'.reset_t(t)=\sigma'(id_t)(\texttt{s\_firable})~.~\sigma'(id_t)(\texttt{s\_priority\_combination})$.}

    By property of the stabilize relation, \InCsCompT, and through the
    examination of the \texttt{priority_authorization_evaluation}
    process defined in the transition design architecture, we can
    deduce:
    \begin{equation}
      \sigma'(id_t)(\texttt{spc})=\prod\limits_{i=0}^{\Delta(id_t)(\texttt{ian})-1}\sigma'(id_t)(\texttt{priority\_authorizations})[i]\label{eq:eq-spc-sum}
    \end{equation}

    As $\Delta(id_t)(\texttt{ian})=1$, we can deduce
    $\prod\limits_{i=0}^{\Delta(id_t)(\texttt{ian})-1}\sigma'(id_t)(\texttt{pauths})[i]=\sigma'(id_t)(\texttt{pauths})[0]$.

    Rewriting the goal with \eqref{eq:eq-spc-sum} and
    $\prod\limits_{i=0}^{\Delta(id_t)(\texttt{ian})-1}\sigma'(id_t)(\texttt{pauths})[i]=\sigma'(id_t)(\texttt{pauths})[0]$:\\
    \fbox{$s'.reset_t(t)=\sigma'(id_t)(\texttt{s\_firable})~.~\sigma'(id_t)(\texttt{pauths})[0]$.}
    
    By construction,
    ${<}\mathtt{priority\_authorizations(0)\Rightarrow{}true}{>}\in{}i_t$,
    and by property of the stabilize relation and \InCsCompT, we can
    deduce\\ $\sigma'(id_t)(\texttt{pauths})[0]=\mathtt{true}$.
    
    Rewriting the goal with $\sigma'(id_t)(\texttt{pauths})[0]=\mathtt{true}$
    , and simplifying the equation:\\
    \fbox{$s'.reset_t(t)=\sigma'(id_t)(\texttt{s\_firable})$.}
    
    Let us perform case analysis on $t\in{}Fired(s)$ or
    $t\notin{}Fired(s)$:

    \begin{itemize}
    \item \textbf{CASE} $t\in{}Fired(s)$:
      
      By property of
      $E_c,\tau\vdash{}s\srarrow{\uparrow}{\fontsize{7}{7}\selectfont}s'$
      (Rule~\ref{it:reset-order}), we can deduce
      $s'.reset_t(t)=\mathtt{true}$.

      Rewriting the goal with $s'.reset_t(t)=\mathtt{true}$:
      \fbox{$\sigma'(id_t)(\texttt{s\_firable})=\mathtt{true}$.}

      By property of the stabilize, the \hvhdl{} rising edge and the
      $\mathtt{Inject}$ relations, \InCsCompT, and through the
      examination of the \texttt{firable} process defined in the
      transition design architecture, we can deduce\\
      $\sigma(id_t)(\texttt{s\_firable})=\sigma'(id_t)(\texttt{s\_firable})$.

      Rewriting the goal with
      $\sigma(id_t)(\texttt{s\_firable})=\sigma'(id_t)(\texttt{s\_firable})$,
      we have \\
      \fbox{$\sigma(id_t)(\texttt{s\_firable})=\mathtt{true}$.}

      By property of
      $\gamma\vdash{}s\stackrel{\downarrow}{\approx}\sigma$, we can
      deduce
      $t\in{}Firable(s)\Leftrightarrow\sigma(id_t)(\texttt{sfa})=\mathtt{true}$.

      Rewriting the goal with
      $t\in{}Firable(s)\Leftrightarrow\sigma(id_t)(\texttt{sfa})=\mathtt{true}$,
      \fbox{$t\in{}Firable(s)$.}

      By property of $t\in{}Fired(s)$, \qedbox{$t\in{}Firable(s)$.}
      
    \item \textbf{CASE} $t\notin{}Fired(s)$:

      \noindent{}By property of $input(t)=\emptyset$, there does not
      exist any input place connected to $t$ by a $\mathtt{basic}$ or
      $\mathtt{test}$ arc. Thus, by property of
      $E_c,\tau\vdash{}s\srarrow{\uparrow}{\fontsize{7}{7}\selectfont}s'$
      (Rule~\ref{it:reset-order}), we can deduce
      $s'.reset_t(t)=\mathtt{false}$.

      Rewriting the goal with $s'.reset_t(t)=\mathtt{false}$:
      \fbox{$\sigma'(id_t)(\texttt{s\_firable})=\mathtt{false}$.}

      By property of the stabilize, the \hvhdl{} rising edge and the
      $\mathtt{Inject}$ relations, \InCsCompT, and through
      the examination of the \texttt{firable} process defined in the
      transition design architecture, we can deduce
      $\sigma(id_t)(\texttt{sfa})=\sigma'(id_t)(\texttt{sfa})$.

      Rewriting the goal with
      $\sigma(id_t)(\texttt{sfa})=\sigma'(id_t)(\texttt{sfa})$,
      \fbox{$\sigma(id_t)(\texttt{sfa})=\mathtt{false}$.}

      By property of
      $\gamma\vdash{}s\stackrel{\downarrow}{\approx}\sigma$, we can
      deduce
      $t\notin{}Firable(s)\Leftrightarrow\sigma(id_t)(\texttt{sfa})=\mathtt{false}$.
      
      By property of $t\notin{}Fired(s)$ and $input(t)=\emptyset$,
      \qedbox{$t\notin{}Firable(s)$}.
    \end{itemize}
    
  \item \textbf{CASE} $input(t)\neq{}\emptyset$:

    By construction,
    ${<}\mathtt{input\_arcs\_number\Rightarrow{}}\vert{}input(t)\vert{>}\in{}g_t$,
    and by property of the elaboration relation, we can deduce
    $\Delta(id_t)(\texttt{ian})=\vert{}input(t)\vert$.

    Rewriting the goal with
    $\Delta(id_t)(\texttt{ian})=\vert{}input(t)\vert$,
    \fbox{$s'.reset_t(t)=\sum\limits_{i=0}^{\vert{}input(t)\vert-1}\sigma'(id_t)(\texttt{rt})[i]$.}
    
    Let us perform case analysis on $t\in{}Fired(s)$ or
    $t\notin{}Fired(s)$:
    
    \begin{itemize}
    \item \textbf{CASE} $t\in{}Fired(s)$:
      
      By property of
      $E_c,\tau\vdash{}s\srarrow{\uparrow}{\fontsize{7}{7}\selectfont}s'$
      (Rule~\ref{it:reset-order}), we can deduce
      $s'.reset_t(t)=\mathtt{true}$.

      Rewriting the goal with $s'.reset_t(t)=\mathtt{true}$,
      \fbox{$\sum\limits_{i=0}^{\vert{}input(t)\vert-1}\sigma'(id_t)(\texttt{rt})[i]=\mathtt{true}$.}

      To prove the goal, let us show
      \fbox{$\exists{}i\in[0,\vert{}input(t)\vert-1]$
        s.t. $\sigma'(id_t)(\texttt{rt})[i]=\mathtt{true}$.}

      By construction, and $input(t)\neq{}\emptyset$, there exist
      ${}p\in{}input(t)$ and $id_p\in{}Comps(\Delta)$
      s.t. $\gamma(p)=id_p$.

      \exP{}

      By construction, there exist an
      $i\in{}[0,\vert{}input(t)\vert-1],$ a
      $j\in[0,\vert{}output(p)\vert-1]$ and $id_{ji}\in{}Sigs(\Delta)$
      s.t.
      ${<}\mathtt{reinit\_transition\_time(j)\Rightarrow{}id_{ji}}{>}\in{}o_p$
      and\\
      ${<}\mathtt{reinit\_time(i)\Rightarrow{}id_{ji}}{>}\in{}i_t$. Let
      us take such an $i$, $j$ and $id_{ji}$, and let us use $i$ to
      prove the goal: \fbox{$\sigma'(id_t)(\texttt{rt})[i]=\mathtt{true}$.}

      \noindent{}By property of the stabilize relation,
      ${<}\mathtt{reinit\_transition\_time(j)\Rightarrow{}id_{ji}}{>}\in{}o_p$
      and
      ${<}\mathtt{reinit\_time(i)\Rightarrow{}id_{ji}}{>}\in{}i_t$,
      we can deduce
      $\sigma'(id_t)(\texttt{rt})[i]=\sigma'(id_{ji})=\sigma'(id_p)(\texttt{rtt})[j]$.

      Rewriting the goal with
      $\sigma'(id_t)(\texttt{rt})[i]=\sigma'(id_{ji})=\sigma'(id_p)(\texttt{rtt})[j]$,
      \fbox{$\sigma'(id_p)(\texttt{rtt})[j]=\mathtt{true}$.}

      By property of the $\mathtt{Inject}$, the \hvhdl{}
      rising edge and the stabilize relations, \InCsCompP, and through
      the examination of the
      \texttt{reinit_transitions_time_evaluation} process defined in
      the place design architecture, we can deduce:
      \begin{equation}
        \label{eq:eq-rtt-j}
        \begin{split}
          \sigma'(id_p)(\texttt{rtt})[j]=& \big((\sigma(id_p)(\texttt{oat})[j]=\mathtt{basic}+\sigma(id_p)(\texttt{oat})[j]=\mathtt{test}) \\
          & .(\sigma(id_p)(\texttt{sm})-\sigma(id_p)(\texttt{sots})<\sigma(id_p)(\texttt{oaw})[j])\\
          & .(\sigma(id_p)(\texttt{sots})>0)\big)\\
          & +\sigma(id_p)(\texttt{otf})[j] \\
        \end{split}
      \end{equation}

      Rewriting the goal with \eqref{eq:eq-rtt-j},
      \begin{equation*}
        \fbox{$\begin{split}
            \mathtt{true}=& ((\sigma(id_p)(\texttt{oat})[j]=\mathtt{basic}+\sigma(id_p)(\texttt{oat})[j]=\mathtt{test}) \\
            & .(\sigma(id_p)(\texttt{sm})-\sigma(id_p)(\texttt{sots})<\sigma(id_p)(\texttt{oaw})[j])\\
            & .(\sigma(id_p)(\texttt{sots})>0))\\
            & +(\sigma(id_p)(\texttt{otf})[j]) \\
          \end{split}$}
      \end{equation*}

      By construction, there exists $id_{ft}\in{}Sigs(\Delta)$ such that\\
      ${<}\mathtt{output\_transitions\_fired(j)\Rightarrow{}id_{ft}}{>}\in{}i_p$
      and ${<}\mathtt{fired\Rightarrow{}id_{ft}}{>}\in{}o_t$. By
      property of state $\sigma$, which is a stable state, we have
      $\sigma(id_t)(\texttt{fired})=\sigma(id_{ft})=\sigma(id_p)(\texttt{otf})[j]$.

      Rewriting the goal with
      $\sigma(id_t)(\texttt{fired})=\sigma(id_{ft})=\sigma(id_p)(\texttt{otf})[j]$,
      \begin{equation*}
        \fbox{$\begin{split}
            \mathtt{true}=& ((\sigma(id_p)(\texttt{oat})[j]=\mathtt{basic}+\sigma(id_p)(\texttt{oat})[j]=\mathtt{test}) \\
            & .(\sigma(id_p)(\texttt{sm})-\sigma(id_p)(\texttt{sots})<\sigma(id_p)(\texttt{oaw})[j])\\
            & .(\sigma(id_p)(\texttt{sots})>0))\\
            & +\sigma(id_t)(\texttt{fired}) \\
          \end{split}$}
      \end{equation*}

      By property of
      $\gamma\vdash{}s\stackrel{\downarrow}{\approx}\sigma$, we can
      deduce
      $t\in{}Fired(s)\Leftrightarrow\sigma(id_t)(\texttt{fired})=\mathtt{true}$.

      Rewriting the goal with
      $t\in{}Fired(s)\Leftrightarrow\sigma(id_t)(\texttt{fired})=\mathtt{true}$
      and simplify the goal, then \qedbox{tautology}.

    \item \textbf{CASE} $t\notin{}Fired(s)$: Then, there are two cases
      that will determine the value of $s'.reset_t(t)$. Either there
      exists a place $p$ with an output token sum greater than zero,
      that is connected to $t$ by an $\mathtt{basic}$ or
      $\mathtt{test}$ arc, and such that the transient marking of $p$
      disables $t$; or such a place does not exist (the predicate is
      decidable).

      \begin{itemize}
      \item \textbf{CASE} there exists such a place $p$ as described
        above:

        Then, let us take such a place $p$ and $\omega\in\mathbb{N}^{*}$ s.t.:
        \begin{enumerate}
        \item $\sum\limits_{t_i\in{}Fired(s)}pre(p,t_i)>0$\label{item:1}
        \item
          $pre(p,t)=(\omega,\mathtt{basic})\lor{}pre(p,t)=(\omega,\mathtt{test})$\label{item:2}
        \item
          $s.M(p)-\sum\limits_{t_i\in{}Fired(s)}pre(p,t_i)<\omega$\label{item:3}
        \end{enumerate}

        We will only consider the case where
        $pre(p,t)=(\omega,\mathtt{basic})$; the proof is the similar
        when $pre(p,t)=(\omega,\mathtt{test})$.

        Assuming that $p$ exists, and by property of
        $E_c,\tau\vdash{}s\srarrow{\uparrow}{\fontsize{7}{7}\selectfont}s'$
        (Rule~\ref{it:reset-order}), we can deduce
        $s'.reset_t(t)=\mathtt{true}$.

        Rewriting the goal with $s'.reset_t(t)=\mathtt{true}$,
        \fbox{$\sum\limits_{i=0}^{\vert{}input(t)\vert-1}\sigma'(id_t)(\texttt{rt})[i]=\mathtt{true}$.}

        \noindent{}To prove the goal, let us show
        \fbox{$\exists{}i\in[0,\vert{}input(t)\vert-1]$
          s.t. $\sigma'(id_t)(\texttt{rt})[i]=\mathtt{true}$.}

        By construction, there exists $id_p\in{}Comps(\Delta)$
        s.t. $\gamma(p)=id_p$.

        \exP{}

        By construction, there exist an
        $i\in{}[0,\vert{}input(t)\vert-1],$ a
        $j\in[0,\vert{}output(p)\vert-1]$ and
        $id_{ji}\in{}Sigs(\Delta)$ s.t.
        ${<}\mathtt{reinit\_transition\_time(j)\Rightarrow{}id_{ji}}{>}\in{}o_p$
        and\\
        ${<}\mathtt{reinit\_time(i)\Rightarrow{}id_{ji}}{>}\in{}i_t$. Let
        us take such an $i$, $j$ and $id_{ji}$, and let us use $i$ to
        prove the goal: \fbox{$\sigma'(id_t)(\texttt{rt})[i]=\mathtt{true}$.}

        By property of the stabilize relation,
        ${<}\mathtt{reinit\_transition\_time(j)\Rightarrow{}id_{ji}}{>}\in{}o_p$
        and
        ${<}\mathtt{reinit\_time(i)\Rightarrow{}id_{ji}}{>}\in{}i_t$,
        we have
        $\sigma'(id_t)(\texttt{rt})[i]=\sigma'(id_{ji})=\sigma'(id_p)(\texttt{rtt})[j]$.

        Rewriting the goal with
        $\sigma'(id_t)(\texttt{rt})[i]=\sigma'(id_{ji})=\sigma'(id_p)(\texttt{rtt})[j]$, we have\\
        \fbox{$\sigma'(id_p)(\texttt{rtt})[j]=\mathtt{true}$.}

        By property of the $\mathtt{Inject}$, the \hvhdl{}
        rising edge and the stabilize relation, and through the
        examination of the \texttt{reinit_transitions_time_evaluation}
        process defined in the place design architecture, we can
        deduce:
        \begin{equation}
          \label{eq:eq-rtt-j-2}
          \begin{split}
            \sigma'(id_p)(\texttt{rtt})[j]=& \big((\sigma(id_p)(\texttt{oat})[j]=\mathtt{basic}+\sigma(id_p)(\texttt{oat})[j]=\mathtt{test}) \\
            & .(\sigma(id_p)(\texttt{sm})-\sigma(id_p)(\texttt{sots})<\sigma(id_p)(\texttt{oaw})[j])\\
            & .(\sigma(id_p)(\texttt{sots})>0)\big)\\
            & +\sigma(id_p)(\texttt{otf})[j] \\
          \end{split}
        \end{equation}

        Rewriting the goal with \eqref{eq:eq-rtt-j-2},
        \begin{equation*}
          \fbox{$\begin{split}
              \mathtt{true}=& ((\sigma(id_p)(\texttt{oat})[j]=\mathtt{basic}+\sigma(id_p)(\texttt{oat})[j]=\mathtt{test}) \\
              & .(\sigma(id_p)(\texttt{sm})-\sigma(id_p)(\texttt{sots})<\sigma(id_p)(\texttt{oaw})[j])\\
              & .(\sigma(id_p)(\texttt{sots})>0))\\
              & +\sigma(id_p)(\texttt{otf})[j] \\
            \end{split}$}
        \end{equation*}

        By construction,
        ${<}\mathtt{output\_arcs\_types(j)\Rightarrow{}basic}{>}\in{}i_p$
        and\\
        ${<}\mathtt{output\_arcs\_weights(j)\Rightarrow{}}\omega{>}\in{}i_p$.
        
        By property of the stabilize relation and \InCsCompP, we can
        deduce $\sigma'(id_p)(\texttt{oat})[j]=\mathtt{basic}$ and
        $\sigma'(id_p)(\texttt{oaw})[j]=\omega$.

        By property of
        $\gamma\vdash{}s\stackrel{\downarrow}{\approx}\sigma$, we can
        deduce $\sigma(id_p)(\texttt{sm})=s.M(p)$ and
        $\sigma(id_p)(\texttt{sots})=\sum\limits_{t_i\in{}Fired(s)}pre(p,t_i)$.

        \noindent{}Rewriting the goal with
        $\sigma'(id_p)(\texttt{oat})[j]=\mathtt{basic}$,
        $\sigma'(id_p)(\texttt{oaw})[j]=\omega$, $\sigma(id_p)(\texttt{sm})=s.M(p)$
        and
        $\sigma(id_p)(\texttt{sots})=\sum\limits_{t_i\in{}Fired(s)}pre(p,t_i)$,
        and simplifying the goal:
        \begin{equation*}
          \fbox{
            \begin{tabular}{@{}c@{}}
              $\big((s.M(p)-\sum\limits_{t_i\in{}Fired(s)}pre(p,t_i)<\omega)~.~(\sum\limits_{t_i\in{}Fired(s)}pre(p,t_i)>0)\big)+\sigma(id_t)(\texttt{fired})$ \\
              $=$\\
              $\mathtt{true}$\\
            \end{tabular}
          }
        \end{equation*}

        We assumed that
        $s.M(p)-\sum\limits_{t_i\in{}Fired(s)}pre(p,t_i)<\omega$ and
        $\sum\limits_{t_i\in{}Fired(s)}pre(p,t_i)>0$.

        Thus, by assumption: \\
        \qedbox{
          \begin{tabular}{@{}c@{}}
            $\big((s.M(p)-\sum\limits_{t_i\in{}Fired(s)}pre(p,t_i)<\omega)~.~(\sum\limits_{t_i\in{}Fired(s)}pre(p,t_i)>0)\big)+\sigma(id_t)(\texttt{fired})$ \\
            $=$ \\
            $\mathtt{true}$\\
          \end{tabular}
        }
        
      \item \textbf{CASE} such a place does not exist:\\
        Then, let us assume that, for all place $p\in{}P$
        \begin{enumerate}
        \item $\sum\limits_{t_i\in{}Fired(s)}pre(p,t_i)=0$\label{item:4}
        \item or
          $\forall{}\omega\in\mathbb{N}^{*},~pre(p,t)=(\omega,\mathtt{basic})\lor{}pre(p,t)=(\omega,\mathtt{test})
          \Rightarrow{}$
          
          \hspace{3ex}$s.M(p)-\sum\limits_{t_i\in{}Fired(s)}pre(p,t_i)\ge\omega$.\label{item:5}
        \end{enumerate}
        
        In that case, by property of
        $E_c,\tau\vdash{}s\srarrow{\uparrow}{\fontsize{7}{7}\selectfont}s'$
        (Rule~\ref{it:reset-order}), we can deduce
        $s'.reset_t(t)=\mathtt{false}$.
        
        Rewriting the goal with $s'.reset_t(t)=\mathtt{false}$:
        \fbox{$\sum\limits_{i=0}^{\vert{}input(t)\vert-1}\sigma'(id_t)(\texttt{rt})[i]=\mathtt{false}$.}

        To prove the goal, let us show
        \fbox{$\forall{}i\in[0,\vert{}input(t)\vert-1],~\sigma'(id_t)(\texttt{rt})[i]=\mathtt{false}$.}\\

        Given an $i\in[0,\vert{}input(t)\vert-1]$, let us show
        \fbox{$\sigma'(id_t)(\texttt{rt})[i]=\mathtt{false}$.}\\

        By construction, there exist a $p\in{}input(t)$, an
        $id_p\in{}Comps(\Delta)$, $g_p$, $i_p$, $o_p$, \\ a
        $j\in[0,\vert{}output(p)\vert-1]$, an
        $id_{ji}\in{}Sigs(\Delta)$ s.t. $\gamma(p)=id_p$ and
        \InCsCompP{} and
        ${<}\mathtt{reinit\_transition\_time(j)\Rightarrow{}id_{ji}}{>}\in{}o_p$
        and
        ${<}$\texttt{reinit\_time(i)}$\Rightarrow{}\mathtt{id_{ji}}{>}\in{}i_t$. Let
        us take such a $p$, $id_p$, $g_p$, $i_p$, $o_p$, $j$ and
        $id_{ji}$.\\

        By property of the stabilize relation,
        ${<}\mathtt{reinit\_transition\_time(j)\Rightarrow{}id_{ji}}{>}\in{}o_p$
        and
        ${<}\mathtt{reinit\_time(i)\Rightarrow{}id_{ji}}{>}\in{}i_t$,
        we have
        $\sigma'(id_t)(\texttt{rt})[i]=\sigma'(id_{ji})=\sigma'(id_p)(\texttt{rtt})[j]$.

        Rewriting the goal with
        $\sigma'(id_t)(\texttt{rt})[i]=\sigma'(id_{ji})=\sigma'(id_p)(\texttt{rtt})[j]$:\\
        \fbox{$\sigma'(id_p)(\texttt{rtt})[j]=\mathtt{false}$.}

        By property of the $\mathtt{Inject}$, the \hvhdl{}
        rising edge and the stabilize relations, \InCsCompP, and
        through the examination of the
        \texttt{reinit_transitions_time_evaluation} process defined in
        the place design architecture, we can deduce:
        \begin{equation}
          \label{eq:10}
          \begin{split}
            \sigma'(id_p)(\texttt{rtt})[j]=& \big((\sigma(id_p)(\texttt{oat})[j]=\mathtt{basic}+\sigma(id_p)(\texttt{oat})[j]=\mathtt{test}) \\
            & .(\sigma(id_p)(\texttt{sm})-\sigma(id_p)(\texttt{sots})<\sigma(id_p)(\texttt{oaw})[j])\\
            & .(\sigma(id_p)(\texttt{sots})>0)\big)\\
            & +\sigma(id_p)(\texttt{otf})[j] \\
          \end{split}
        \end{equation}

        Rewriting the goal with \eqref{eq:10},
        \begin{equation*}
          \fbox{$\begin{split}
              \mathtt{false}=& ((\sigma(id_p)(\texttt{oat})[j]=\mathtt{basic}+\sigma(id_p)(\texttt{oat})[j]=\mathtt{test}) \\
              & .(\sigma(id_p)(\texttt{sm})-\sigma(id_p)(\texttt{sots})<\sigma(id_p)(\texttt{oaw})[j])\\
              & .(\sigma(id_p)(\texttt{sots})>0))\\
              & +\sigma(id_p)(\texttt{otf})[j]) \\
            \end{split}$}
        \end{equation*}
        
        By construction, there exists $id_{ft}\in{}Sigs(\Delta)$ such that\\
        ${<}\mathtt{output\_transitions\_fired(j)\Rightarrow{}id_{ft}}{>}\in{}i_p$
        and ${<}\mathtt{fired\Rightarrow{}id_{ft}}{>}\in{}o_t$. By
        property of state $\sigma$ as being a stable state, we have
        $\sigma(id_t)(\texttt{fired})=\sigma(id_{ft})=\sigma(id_p)(\texttt{otf})[j]$.

        Rewriting the goal with
        $\sigma(id_t)(\texttt{fired})=\sigma(id_{ft})=\sigma(id_p)(\texttt{otf})[j]$:
        \begin{equation*}
          \fbox{$\begin{split}
              \mathtt{false}=& ((\sigma(id_p)(\texttt{oat})[j]=\mathtt{basic}+\sigma(id_p)(\texttt{oat})[j]=\mathtt{test}) \\
              & .(\sigma(id_p)(\texttt{sm})-\sigma(id_p)(\texttt{sots})<\sigma(id_p)(\texttt{oaw})[j])\\
              & .(\sigma(id_p)(\texttt{sots})>0))\\
              & +\sigma(id_t)(\texttt{fired}) \\
            \end{split}$}
        \end{equation*}

        By property of
        $\gamma\vdash{}s\stackrel{\downarrow}{\approx}\sigma$, we can
        deduce
        $t\notin{}Fired(s)\Leftrightarrow\sigma(id_t)(\texttt{fired})=\mathtt{false}$

        Rewriting the goal with
        $t\notin{}Fired(s)\Leftrightarrow\sigma(id_t)(\texttt{fired})=\mathtt{false}$
        and simplifying the goal:
        \begin{equation*}
          \fbox{$\begin{split}
              \mathtt{false}=& ((\sigma(id_p)(\texttt{oat})[j]=\mathtt{basic}+\sigma(id_p)(\texttt{oat})[j]=\mathtt{test}) \\
              & .(\sigma(id_p)(\texttt{sm})-\sigma(id_p)(\texttt{sots})<\sigma(id_p)(\texttt{oaw})[j])\\
              & .(\sigma(id_p)(\texttt{sots})>0))\\
            \end{split}$}
        \end{equation*}

        Then, based on the assumptions made at the beginning of case,
        there are two cases:
        \begin{enumerate}
        \item \textbf{CASE}
          $\sum\limits_{t_i\in{}Fired(s)}pre(p,t_i)=0$:\\
          
          By property of
          $\gamma\vdash{}s\stackrel{\downarrow}{\approx}\sigma$, we
          can deduce
          $\sum\limits_{t_i\in{}Fired(s)}pre(p,t_i)=\sigma(id_p)(\texttt{sots})$.

          Rewriting the goal with
          $\sum\limits_{t_i\in{}Fired(s)}pre(p,t_i)=\sigma(id_p)(\texttt{sots})$
          and $\sum\limits_{t_i\in{}Fired(s)}pre(p,t_i)=0$, and
          simplifying the goal: \qedbox{tautology.}
          
        \item \textbf{CASE}
          $\forall{}\omega\in\mathbb{N}^{*},~pre(p,t)=(\omega,\mathtt{basic})\lor{}pre(p,t)=(\omega,\mathtt{test})
          \Rightarrow{}$

          \hspace{7ex}$s.M(p)-\sum\limits_{t_i\in{}Fired(s)}pre(p,t_i)\ge\omega$:

          Let us perform case analysis on $pre(p,t)$; there are two
          cases:

          \begin{enumerate}
          \item \textbf{CASE} $pre(p,t)=(\omega,\mathtt{basic})$ or $pre(p,t)=(\omega,\mathtt{basic})$:\\
            \noindent{}By construction,
            ${<}\mathtt{output\_arcs\_weights(j)\Rightarrow{}}\omega{>}\in{}i_p$.
            
            \noindent{}By property of stable state $\sigma$ and
            \InCsCompP, we can deduce $\sigma(id_p)(\texttt{oaw})[j]=\omega$.

            By property of
            $\gamma\vdash{}s\stackrel{\downarrow}{\approx}\sigma$, we
            can deduce $\sigma(id_p)(\texttt{sm})=s.M(p)$ and
            $\sigma(id_p)(\texttt{sots})=\sum\limits_{t_i\in{}Fired(s)}pre(p,t_i)$.

            Rewriting the goal with
            $\sigma(id_p)(\texttt{oaw})[j]=\omega$,
            $\sigma(id_p)(\texttt{sm})=s.M(p)$ and\\
            $\sigma(id_p)(\texttt{sots})=\sum\limits_{t_i\in{}Fired(s)}pre(p,t_i)$:
            \begin{equation*}
              \fbox{$\begin{split}
                  \mathtt{false}=& ((\sigma(id_p)(\texttt{oat})[j]=\mathtt{basic}+\sigma(id_p)(\texttt{oat})[j]=\mathtt{test}) \\
                  & .(s.M(p)-\sum\limits_{t_i\in{}Fired(s)}pre(p,t_i)<\omega)\\
                  & .(\sum\limits_{t_i\in{}Fired(s)}pre(p,t_i)>0))\\
                \end{split}$}
            \end{equation*}
            
            We assumed that
            $s.M(p)-\sum\limits_{t_i\in{}Fired(s)}pre(p,t_i)\ge\omega$,
            and then we can deduce
            $s.M(p)-\sum\limits_{t_i\in{}Fired(s)}pre(p,t_i)<\omega=\mathtt{false}$.
            

            Rewriting the goal with
            $s.M(p)-\sum\limits_{t_i\in{}Fired(s)}pre(p,t_i)<\omega=\mathtt{false}$,
            and simplifying the goal, \qedbox{tautology.}

          \item \textbf{CASE} $pre(p,t)=(\omega,\mathtt{inhib})$:
            
            By construction,
            ${<}\mathtt{output\_arcs\_types(j)\Rightarrow{}inhib}{>}\in{}i_p$.
            
            By property of stable state $\sigma$ and \InCsCompP, we
            can deduce $\sigma(id_p)(\texttt{oat})[j]=\mathtt{inhib}$.

            Rewriting the goal with
            $\sigma(id_p)(\texttt{oat})[j]=\mathtt{inhib}$, and simplifying
            the goal, we have a \qedbox{tautology.}
          \end{enumerate}
        \end{enumerate}
      \end{itemize}
    \end{itemize}
  \end{itemize}
\end{niproof}

\subsection{Rising edge and action executions}
\label{sec:re-action-exec}

\begin{lemma}[Rising edge equal action executions]
  \label{lem:re-equal-action-exec}
  \rehyps{} then\\
  $\forall{}a\in\mathcal{A},id_a\in{}Outs(\Delta)~s.t.~\gamma(a)=id_a,~s'.ex(a)=\sigma'(id_a)$.
\end{lemma}

\begin{niproof}
  Given an $a\in\mathcal{A}$ and an
  $id_a\in{}Outs(\Delta)~s.t.~\gamma(a)=id_a$, let us show
  \fbox{$s'.ex(a)=\sigma'(id_a)$.}

  By property of
  $E_c,\tau\vdash{}s\srarrow{\uparrow}{\fontsize{7}{7}\selectfont}s'$,
  we can deduce $s.ex(a)=s'.ex(a)$.
  
  By construction, $id_a$ is an output port identifier of Boolean type
  in the \hvhdl{} design $d$. The generated \texttt{``action''}
  process is responsible for the assignment of the $id a$ only during
  the initialization phase or during a falling edge phase.

  By property of the \hvhdl{} $\mathtt{Inject}$, rising
  edge, stabilize relations, and the \texttt{``action''} process, we
  can deduce $\sigma(id_a)=\sigma'(id_a)$.

  Rewriting the goal with $s.ex(a)=s'.ex(a)$ and
  $\sigma(id_a)=\sigma'(id_a)$, \fbox{$s.ex(a)=\sigma(id_a)$.}

  By property of
  $\gamma\vdash{}s\stackrel{\downarrow}{\approx}\sigma$,
  \qedbox{$s.ex(a)=\sigma(id_a)$.}
\end{niproof}

\subsection{Rising edge and function executions}
\label{sec:re-fun-exec}

\begin{lemma}[Rising edge equal function executions]
  \label{lem:re-equal-fun-exec}
  \rehyps{} then\\
  $\forall{}f\in\mathcal{F},id_f\in{}Outs(\Delta)~s.t.~\gamma(f)=id_f,~s'.ex(f)=\sigma'(id_f)$.
\end{lemma}

\begin{niproof}
  Given an $f\in\mathcal{F}$ and an $id_f\in{}Outs(\Delta)$
  s.t. $\gamma(f)=id_f$, let us show \fbox{$s'.ex(f)=\sigma'(id_f)$.}

  By property of
  $E_c,\tau\vdash{}s\srarrow{\uparrow}{\fontsize{7}{7}\selectfont}s'$
  (Rule~\ref{it:exec-fun}):
  \begin{equation}
    s'.ex(f)=\sum\limits_{t\in{}Fired(s)}\mathbb{F}(t,f)\label{eq:eq-exf}
  \end{equation}

  By construction, $id_f$ is an output port identifier of Boolean type
  in the \hvhdl{} design $d$. The generated \texttt{function} process
  assigns a value to the output port $id_f$ only during the
  initialization phase or during a rising edge phase.
  
  By construction, the \texttt{function} process is defined in the
  behavior of design $d$, i.e.\\
  $\mathtt{ps}(\texttt{function}, \emptyset, sl, ss)\in{}d.cs$.
  
  Let $trs(f)$ be the set of transitions associated to function $f$,
  i.e. $trs(f)=\{t\in{}T~\vert~\mathbb{F}(t,f)=\mathtt{true}\}$.

  Let us perform case analysis on $trs(f)$; there are two cases:
  
  \begin{itemize}
  \item \textbf{CASE} $trs(f)=\emptyset$:
    
    By construction,
    $\mathtt{id_f\Leftarrow{}false}\in{}ss_{\uparrow}$ where
    $ss_\uparrow$ is the part of the \texttt{function} process body
    executed during a rising edge phase.

    By property of the \hvhdl{} rising edge, the stabilize relations
    and $\mathtt{ps}(\texttt{function}, \emptyset, sl, ss)\in{}d.cs$, we can
    deduce $\sigma'(id_f)=\mathtt{false}$.
    
    \noindent{}By property of
    $\sum\limits_{t\in{}Fired(s)}\mathbb{F}(t,f)$ and
    $trs(f)=\emptyset$, we can deduce
    $\sum\limits_{t\in{}Fired(s)}\mathbb{F}(t,f)=\mathtt{false}$.

    \noindent{}Rewriting the goal with \eqref{eq:eq-exf},
    $\sigma'(id_f)=\mathtt{false}$ and
    $\sum\limits_{t\in{}Fired(s)}\mathbb{F}(t,f)=\mathtt{false}$:
    \qedbox{tautology.}
    
  \item \textbf{CASE} $trs(f)\neq\emptyset$:
    
    By construction,
    $\mathtt{id_f\Leftarrow{}id_{ft_0}+\dots+id_{ft_n}}\in{}ss_\uparrow$,
    where $id_{ft_i}\in{}Sigs(\Delta)$, $ss_\uparrow$ is the part of
    the \texttt{function} process body executed during a rising edge
    phase, and $n=\vert{}trs(f)\vert-1$.

    By property of the $\mathtt{Inject}$, the
    \hvhdl{} rising edge, the stabilize relations, and\\
    $\mathtt{ps}(\texttt{function}, \emptyset, sl, ss)\in{}d.cs$, we can
    deduce:
    \begin{equation}
      \sigma'(id_f)=\sigma(id_{ft_0})+\dots+\sigma(id_{ft_n})\label{eq:eq-idf-prod}
    \end{equation}

    Rewriting the goal with \eqref{eq:eq-exf} and
    \eqref{eq:eq-idf-prod},
    \fbox{$\sum\limits_{t\in{}Fired(s)}\mathbb{F}(t,f)=\sigma(id_{ft_0})+\dots+\sigma(id_{ft_n})$.}

    Let us reason on the value of
    $\sigma(id_{ft_0})+\dots+\sigma(id_{ft_n})$; there are two cases:

    \begin{itemize}
    \item \textbf{CASE} $\sigma(id_{ft_0})+\dots+\sigma(id_{ft_n})=\mathtt{true}$:
      
      Then, we can rewrite the goal as follows:
      \fbox{$\sum\limits_{t\in{}Fired(s)}\mathbb{F}(t,f)=\mathtt{true}$.}

      To prove the above goal, let us show
      \fbox{$\exists{}t\in{}Fired(s)~s.t.~\mathbb{F}(t,f)=\mathtt{true}$.}

      From $\sigma(id_{ft_0})+\dots+\sigma(id_{ft_n})=\mathtt{true}$,
      we can deduce
      $\exists{}id_{ft_i}~s.t.~\sigma(id_{ft_i})=\mathtt{true}$. Let
      us take such an $id_{ft_i}$.
      
      By construction, there exist a $t\in{}trs(f)$, an
      $id_t\in{}Comps(\Delta)$, $g_t$, $i_t$, $o_t$ such that:
      \begin{itemize}
      \item $\gamma(t)=id_t$
      \item \InCsCompT{}
      \item ${<}\mathtt{fired\Rightarrow{id_{ft_i}}}{>}\in{}o_{t}$
      \end{itemize}
      
      \noindent{}By property of $\sigma$ as being a stable design
      state, and \InCsCompT, we can deduce
      $\sigma(id_{t})(\texttt{fired})=\sigma(id_{ft_i})$, and thus that
      $\sigma(id_{t})(\texttt{fired})=\mathtt{true}$.

      By property of
      $\gamma\vdash{}s\stackrel{\downarrow}{\approx}\sigma$, we can
      deduce $t\in{}Fired(s)$.
      
      Let us use $t$ to prove the goal:
      \fbox{$\mathbb{F}(t,f)=\mathtt{true}$.}

      By definition of $t\in{}trs(f)$,
      \qedbox{$\mathbb{F}(t,f)=\mathtt{true}$.}

    \item \textbf{CASE} $\sigma(id_{ft_0})+\dots+\sigma(id_{ft_n})=\mathtt{false}$:\\
      \noindent{}Then, we can rewrite the goal as follows:
      \fbox{$\sum\limits_{t\in{}Fired(s)}\mathbb{F}(t,f)=\mathtt{false}$.}

      \noindent{}To prove the above goal, let us show
      \fbox{$\forall{}t\in{}Fired(s)~s.t.~\mathbb{F}(t,f)=\mathtt{false}$.}

      \noindent{}Given a $t\in{}Fired(s)$, let us show
      \fbox{$\mathbb{F}(t,f)=\mathtt{false}$.}

      \noindent{}Let us perform case analysis on $\mathbb{F}(t,f)$; there are 2 cases:

      \begin{itemize}
      \item \textbf{CASE} \qedbox{$\mathbb{F}(t,f)=\mathtt{false}$.}
      \item \textbf{CASE} $\mathbb{F}(t,f)=\mathtt{true}$:\\
        
        By construction, there exist an $id_{t}\in{}Comps(\Delta)$,
        $g_{t}$, $i_{t}$, $o_{t}$ and
        $id_{ft_i}\in{}Sigs(\Delta)$ such that:
        \begin{itemize}
        \item $\gamma(t)=id_{t}$
        \item \InCsCompT
        \item ${<}\mathtt{fired\Rightarrow{id_{ft_i}}}{>}\in{}o_{t}$
        \end{itemize}

        By property of stable design state $\sigma$ and \InCsCompT, we
        can deduce $\sigma(id_t)(\texttt{fired})=\sigma(id_{ft_i})$.

        By property of
        $\gamma\vdash{}s\stackrel{\downarrow}{\approx}\sigma$, we can
        deduce
        $t\in{}Fired(s)\Leftrightarrow{}\sigma(id_t)(\texttt{fired})=\mathtt{true}$.

        \noindent{}Since $t\in{}Fired(s)$, we can deduce
        $\sigma(id_t)(\texttt{fired})=\mathtt{true}$, and from
        $\sigma(id_t)(\texttt{fired})=\sigma(id_{ft_i})$, we can deduce
        $\sigma(id_{ft_i})=\mathtt{true}$.

        Then, \qedbox{$\sigma(id_{ft_i})=\mathtt{true}$ contradicts
          $\sigma(id_{ft_0})+\dots+\sigma(id_{ft_n})=\mathtt{false}$.}
      \end{itemize}
    \end{itemize}
  \end{itemize}
\end{niproof}

\subsection{Rising edge and sensitization}
\label{sec:re-sens}

\begin{lemma}[Rising edge equal sensitized]
  \label{lem:re-equal-sens}
  \rehyps{} then\\
  $\forall{}t\in{}T,id_t\in{}Comps(\Delta)~s.t.~\gamma(t)=id_t,$
  $t\in{}Sens(s'.M)\Leftrightarrow\sigma'(id_t)(\texttt{s\_enabled})=\mathtt{true}$.
\end{lemma}

\begin{niproof}
  Given a $t\in{}T$ and an $id_t\in{}Comps(\Delta)$
  s.t. $\gamma(t)=id_t$, let us show\\
  \fbox{$t\in{}Sens(s'.M)\Leftrightarrow\sigma'(id_t)(\texttt{s\_enabled})=\mathtt{true}$.}\\
  
  \exT{}. Then, the proof is in two parts:

  \begin{enumerate}
  \item Assuming that $t\in{}Sens(s'.M)$, let us show
    \fbox{$\sigma'(id_t)(\texttt{s\_enabled})=\mathtt{true}$.}

    \noindent{}By property of the stabilize relation, \InCsCompT{},
    and through the examination of the \texttt{enable_evaluation}
    process defined in the transition design architecture:
    \begin{equation}
      \sigma'(id_t)(\texttt{se})=\prod\limits_{i=0}^{\Delta(id_t)(\texttt{ian})-1}\sigma'(id_t)(\texttt{input\_arcs\_valid})[i]\label{eq:eq-senabled-prod}
    \end{equation}

    Rewriting the goal with \eqref{eq:eq-senabled-prod},
    \fbox{$\prod\limits_{i=0}^{\Delta(id_t)(\texttt{ian})-1}\sigma'(id_t)(\texttt{iav})[i]=\mathtt{true}$.}
    
    \noindent{}To prove the goal, let us show that
    \fbox{$\forall{}i\in[0,\Delta(id_t)(\texttt{ian})-1],~\sigma'(id_t)(\texttt{iav})[i]=\mathtt{true}$.}

    \noindent{}Given an $i\in[0,\Delta(id_t)(\texttt{ian})-1]$, let us show
    \fbox{$\sigma'(id_t)(\texttt{iav})[i]=\mathtt{true}$.}
    
    \noindent{}Let us perform case analysis on $input(t)$.

    \begin{itemize}
    \item \textbf{CASE} $input(t)=\emptyset$:

      By construction,
      ${<}\mathtt{input\_arcs\_number\Rightarrow{}1}{>}\in{}g_t$ and\\
      ${<}\mathtt{input\_arcs\_valid(0)\Rightarrow{}true}{>}\in{}i_t$.

      By property of the elaboration and stabilize relations and
      \InCsCompT{}, we can deduce $\Delta(id_t)(\texttt{ian})=1$ and
      $\sigma'(id_t)(\texttt{iav})[0]=\mathtt{true}$.

      Thanks to $\Delta(id_t)(\texttt{ian})=1$, we can deduce that $i=0$.

      Rewriting the goal with $\sigma'(id_t)(\texttt{iav})[0]=\mathtt{true}$,
      \qedbox{tautology.}
      
    \item \textbf{CASE} $input(t)\neq\emptyset$:\\
      \noindent{}By construction,
      ${<}\mathtt{input\_arcs\_number\Rightarrow{}}\vert{}input(t)\vert{>}\in{}g_t$.

      By property of the elaboration relation and \InCsCompT, we can
      deduce $\Delta(id_t)(\texttt{ian})=\vert{}input(t)\vert$.
      
      Thanks to $\Delta(id_t)(\texttt{ian})=\vert{}input(t)\vert$, we know
      that $i\in[0,\vert{}input(t)\vert-1]$.

      By construction, there exist a $p\in{}input(t)$,
      $id_p\in{}Comps(\Delta)$, $g_p$, $i_p$, $o_p$,
      $j\in{}[0,\vert{}output(p)\vert-1]$ and
      $id_{ji}\in{}Sigs(\Delta)$ s.t. $\gamma(p)=id_p$ and\\
      \InCsCompP{} and
      ${<}\mathtt{output\_arcs\_valid(j)\Rightarrow{}id_{ji}}{>}\in{}o_p$
      and
      ${<}\mathtt{input\_arcs\_valid(i)\Rightarrow{}id_{ji}}{>}\in{}i_t$.

      By property of the stabilize relation, \InCsCompT{} and
      \InCsCompP, we can deduce
      $\sigma'(id_t)(\texttt{iav})[i]=\sigma'(id_{ji})=\sigma'(id_p)(\texttt{oav})[j]$.

      Rewriting the goal with
      $\sigma'(id_t)(\texttt{iav})[i]=\sigma'(id_{ji})=\sigma'(id_p)(\texttt{oav})[j]$:\\
      \fbox{$\sigma'(id_p)(\texttt{oav})[j]=\mathtt{true}$.}

      By property of the stabilize relation, \InCsCompP, and through
      the examination of the \texttt{marking_validation_evaluation}
      process defined in the place design architecture, we can deduce:
      \begin{equation}
        \label{eq:eq-oav-sens}
        \begin{split}
          \sigma'(id_p)(\texttt{oav})[j]=& \big((\sigma'(id_p)(\texttt{oat})[j]=\mathtt{basic}+\sigma'(id_p)(\texttt{oat})[j]=\mathtt{test}) \\
          & \quad.~\sigma'(id_p)(\texttt{sm})\ge\sigma'(id_p)(\texttt{oaw})[j]\big)\\
          & +\big(\sigma'(id_p)(\texttt{oat})[j]=\mathtt{inhib}~.~\sigma'(id_p)(\texttt{sm})<\sigma'(id_p)(\texttt{oaw})[j]\big)\\
        \end{split}
      \end{equation}

      Rewriting the goal with \eqref{eq:eq-oav-sens},
      \begin{equation*}
        \fbox{$
          \begin{split}
            \mathtt{true}=& \big((\sigma'(id_p)(\texttt{oat})[j]=\mathtt{basic}+\sigma'(id_p)(\texttt{oat})[j]=\mathtt{test}) \\
            & \quad.~\sigma'(id_p)(\texttt{sm})\ge\sigma'(id_p)(\texttt{oaw})[j]\big)\\
            & +\big(\sigma'(id_p)(\texttt{oat})[j]=\mathtt{inhib}~.~\sigma'(id_p)(\texttt{sm})<\sigma'(id_p)(\texttt{oaw})[j]\big)\\
          \end{split}
          $}
      \end{equation*}

      Let us perform case analysis on $pre(p,t)$; there are 3 cases:
      \begin{itemize}
      \item \textbf{CASE} $pre(p,t)=(\omega,\mathtt{basic})$:\\

        \noindent{}By construction,
        ${<}\mathtt{output\_arcs\_types(j)\Rightarrow{}basic}{>}\in{}i_p$
        and\\
        ${<}\mathtt{output\_arcs\_weights(j)\Rightarrow{}}\omega{>}\in{}i_p$.

        By property of the stabilize relation and \InCsCompP,

        we can deduce $\sigma'(id_p)(\texttt{oat})[j]=\mathtt{basic}$ and
        $\sigma'(id_p)(\texttt{oaw})[j]=\omega$.

        Rewriting the goal with
        $\sigma'(id_p)(\texttt{oat})[j]=\mathtt{basic}$
        and $\sigma'(id_p)(\texttt{oaw})[j]=\omega$, and simplifying the goal:\\
        \fbox{$\sigma'(id_p)(\texttt{sm})\ge\omega=\mathtt{true}$.}

        \noindent{}Appealing to Lemma~\ref{lem:re-equal-marking}, we
        can deduce $s'.M(p)=\sigma'(id_p)(\texttt{sm})$.

        \noindent{}Rewriting the goal with
        $s'.M(p)=\sigma'(id_p)(\texttt{sm})$:
        \fbox{$s'.M(p)\ge\omega=\mathtt{true}$.}

        \noindent{}By definition of $t\in{}Sens(s'.M)$,
        \qedbox{$s'.M(p)\ge\omega=\mathtt{true}$.}
        
      \item \textbf{CASE} $pre(p,t)=(\omega,\mathtt{test})$: same as
        above.
      \item \textbf{CASE} $pre(p,t)=(\omega,\mathtt{inhib})$:

        \noindent{}By construction,
        ${<}\mathtt{output\_arcs\_types(j)\Rightarrow{}inhib}{>}\in{}i_p$
        and\\
        ${<}\mathtt{output\_arcs\_weights(j)\Rightarrow{}}\omega{>}\in{}i_p$.

        \noindent{}By property of the stabilize relation and
        \InCsCompP, we can deduce
        $\sigma'(id_p)(\texttt{oat})[j]=\mathtt{inhib}$ and
        $\sigma'(id_p)(\texttt{oaw})[j]=\omega$.

        \noindent{}Rewriting the goal with
        $\sigma'(id_p)(\texttt{oat})[j]=\mathtt{inhib}$ and
        $\sigma'(id_p)(\texttt{oaw})[j]=\omega$, and simplifying the goal:
        \fbox{$\sigma'(id_p)(\texttt{sm})<\omega=\mathtt{true}$.}

        \noindent{}Appealing to Lemma~\ref{lem:re-equal-marking}, we
        can deduce $s'.M(p)=\sigma'(id_p)(\texttt{sm})$.

        \noindent{}Rewriting the goal with
        $s'.M(p)=\sigma'(id_p)(\texttt{sm})$:
        \fbox{$s'.M(p)<\omega=\mathtt{true}$.}

        \noindent{}By definition of $t\in{}Sens(s'.M)$,
        \qedbox{$s'.M(p)<\omega=\mathtt{true}$.}
        
      \end{itemize}
    \end{itemize}
    
  \item Assuming that $\sigma'(id_t)(\texttt{s\_enabled})=\mathtt{true}$, let
    us show \fbox{$t\in{}Sens(s'.M)$.}

    \noindent{}By definition of $t\in{}Sens(s'.M)$, let us show
    \begin{equation*}
      \fbox{\parbox{\lwidth}{$
          \forall{}p\in{}P,\omega\in\mathbb{N}^{*},
          ~\big(pre(p,t)=(\omega,\mathtt{basic})\lor{}pre(p,t)=(\omega,\mathtt{test})\Rightarrow{}s'.M(p)\ge\omega\big)\land
          \big(pre(p,t)=(\omega,\mathtt{inhib})\Rightarrow{}s'.M(p)<\omega\big)
          $}}
    \end{equation*}

    Given a $p\in{}P$ and an $\omega\in\mathbb{N}^{*}$, let us show\\
    \fbox{$pre(p,t)=(\omega,\mathtt{basic})\lor{}pre(p,t)=(\omega,\mathtt{test})\Rightarrow{}s'.M(p)\ge\omega$}
    and \\
    \fbox{$pre(p,t)=(\omega,\mathtt{inhib})\Rightarrow{}s'.M(p)<\omega$.}

    \begin{enumerate}
    \item Assuming
      $pre(p,t)=(\omega,\mathtt{basic})\lor{}pre(p,t)=(\omega,\mathtt{test})$,
      let us show \fbox{$s'.M(p)\ge\omega$.}

      \noindent{}The proceeding is the same for
      $pre(p,t)=(\omega,\mathtt{basic})$ and
      $pre(p,t)=(\omega,\mathtt{test})$. Therefore, we will only cover
      the case where $pre(p,t)=(\omega,\mathtt{basic})$.
      
      \noindent{}By property of the stabilize relation and \InCsCompT,
      equation \eqref{eq:eq-senabled-prod} holds.
      
      Rewriting $\sigma'(id_t)(\texttt{se})=\mathtt{true}$ with
      \eqref{eq:eq-senabled-prod}, we can deduce:\\
      $\prod\limits_{i=0}^{\Delta(id_t)(\texttt{ian})-1}\sigma'(id_t)(\texttt{iav})[i]=\mathtt{true}$.

      \noindent{}Then, we can deduce that
      $\forall{}i\in[0,\Delta(id_t)(\texttt{ian})-1],~\sigma'(id_t)(\texttt{iav})[i]=\mathtt{true}$.

      By construction, there exist an $id_p\in{}Comps(\Delta)$,
      $g_p$, $i_p$, $o_p$, $i\in[0,\vert{}input(t)\vert-1]$,
      $j\in{}[0,\vert{}output(p)\vert-1]$ and
      $id_{ji}\in{}Sigs(\Delta)$ s.t.  $\gamma(p)=id_p$ and\\
      \InCsCompP{} and
      ${<}\mathtt{output\_arcs\_valid(j)\Rightarrow{}id_{ji}}{>}\in{}o_p$
      and
      ${<}\mathtt{input\_arcs\_valid(i)\Rightarrow{}id_{ji}}{>}\in{}i_t$. Let
      us take such an $id_p\in{}Comps(\Delta)$, $g_p$, $i_p$,
      $o_p$, $i\in[0,\vert{}input(t)\vert-1]$,
      $j\in{}[0,\vert{}output(p)\vert-1]$ and
      $id_{ji}\in{}Sigs(\Delta)$.

      \noindent{}By construction,
      ${<}\mathtt{input\_arcs\_number\Rightarrow{}}\vert{}input(t)\vert{>}\in{}g_t$.

      \noindent{}By property of the elaboration relation and
      \InCsCompT, we can deduce
      $\Delta(id_t)(\texttt{ian})=\vert{}input(t)\vert$.

      \noindent{}Thanks to $\Delta(id_t)(\texttt{ian})=\vert{}input(t)\vert$,
      we can deduce that
      $\forall{}i\in[0,\vert{}input(t)\vert-1],$\\
      $\sigma'(id_t)(\texttt{iav})[i]=\mathtt{true}$.

      \noindent{}Having such an $i\in[0,\vert{}input(t)\vert-1]$, we
      can deduce that $\sigma'(id_t)(\texttt{iav})[i]=\mathtt{true}$.

      \noindent{}By property of the stabilize relation, \InCsCompT{}
      and \InCsCompP, we can deduce
      $\sigma'(id_t)(\texttt{iav})[i]=\sigma'(id_{ji})=\sigma'(id_p)(\texttt{oav})[j]$.

      \noindent{}Thanks to
      $\sigma'(id_t)(\texttt{iav})[i]=\sigma'(id_{ji})=\sigma'(id_p)(\texttt{oav})[j]$,
      we have $\sigma'(id_p)(\texttt{oav})[j]=\mathtt{true}$.

      By property of the stabilize relation and \InCsCompP, equation
      \eqref{eq:eq-oav-sens} holds. Thanks to \eqref{eq:eq-oav-sens},
      we can deduce that:
      \begin{equation}
        \label{eq:sens-true}
        \begin{split}
          \mathtt{true}=& \big((\sigma'(id_p)(\texttt{oat})[j]=\mathtt{basic}+\sigma'(id_p)(\texttt{oat})[j]=\mathtt{test}) \\
          & \quad.~\sigma'(id_p)(\texttt{sm})\ge\sigma'(id_p)(\texttt{oaw})[j]\big)\\
          & +\big(\sigma'(id_p)(\texttt{oat})[j]=\mathtt{inhib}~.~\sigma'(id_p)(\texttt{sm})<\sigma'(id_p)(\texttt{oaw})[j]\big)\\
        \end{split}
      \end{equation}

      \noindent{}By construction,
      ${<}\mathtt{output\_arcs\_types(j)\Rightarrow{}basic}{>}\in{}i_p$
      and\\
      ${<}\mathtt{output\_arcs\_weights(j)\Rightarrow{}}\omega{>}\in{}i_p$.

      \noindent{}By property of the stabilize relation and \InCsCompP,
      we can deduce $\sigma'(id_p)(\texttt{oat})[j]=\mathtt{basic}$ and
      $\sigma'(id_p)(\texttt{oaw})[j]=\omega$.

      \noindent{}Thanks to $\sigma'(id_p)(\texttt{oat})[j]=\mathtt{basic}$,
      $\sigma'(id_p)(\texttt{oaw})[j]=\omega$, and simplifying
      Equation~\eqref{eq:sens-true}, we can deduce
      $\sigma'(id_p)(\texttt{sm})\ge\omega=\mathtt{true}$.

      \noindent{}Appealing to Lemma~\ref{lem:re-equal-marking},
      \qedbox{$s'.M(p)\ge\omega$.}

    \item Assuming $pre(p,t)=(\omega,\mathtt{inhib})$, let us show
      \fbox{$s'.M(p)<\omega$.}

      The proceeding is the same as in the preceding case. Here, we
      will start the proof where the two cases are diverging, i.e:

      \noindent{}By construction,
      ${<}\mathtt{output\_arcs\_types(j)\Rightarrow{}inhib}{>}\in{}i_p$
      and\\
      ${<}\mathtt{output\_arcs\_weights(j)\Rightarrow{}}\omega{>}\in{}i_p$.

      \noindent{}By property of the stabilize relation and \InCsCompP,
      we can deduce $\sigma'(id_p)(\texttt{oat})[j]=\mathtt{inhib}$ and
      $\sigma'(id_p)(\texttt{oaw})[j]=\omega$.

      \noindent{}Thanks to $\sigma'(id_p)(\texttt{oat})[j]=\mathtt{inhib}$
      and $\sigma'(id_p)(\texttt{oaw})[j]=\omega$, and simplifying
      Equation~\eqref{eq:sens-true}, we can deduce
      $\sigma'(id_p)(\texttt{sm})<\omega=\mathtt{true}$.

      \noindent{}Appealing to Lemma~\ref{lem:re-equal-marking},
      \qedbox{$s'.M(p)<\omega$.}
    \end{enumerate}

  \end{enumerate}
\end{niproof}

\begin{lemma}[Rising edge equal not sensitized]
  \label{lem:re-equal-not-sens}
  \rehyps{} then\\
  $\forall{}t\in{}T,id_t\in{}Comps(\Delta)~s.t.~\gamma(t)=id_t,$
  $t\notin{}Sens(s'.M)\Leftrightarrow\sigma'(id_t)(\texttt{s\_enabled})=\mathtt{false}$.
\end{lemma}

\begin{niproof}
  Proving the above lemma is trivial by appealing to
  Lemma~\ref{lem:re-equal-sens} and by reasoning on
  contrapositives.
\end{niproof}

%%% Local Variables:
%%% mode: latex
%%% TeX-master: "../../main"
%%% End:
