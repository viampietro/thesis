%%%%%%%%%%%%%%%%%%%%%%%%%%%%%%%%%%%%%%%%%%%%%%%
%%%%%% SIMILAR INITIAL STATES HYPOTHESES %%%%%%
%%%%%%%%%%%%%%%%%%%%%%%%%%%%%%%%%%%%%%%%%%%%%%%

\begin{definition}[Initial state hypotheses]
  \label{def:init-states-hyps}
  Given an $sitpn\in{}SITPN$, $d\in{}design$,
  $\gamma\in{}WM(sitpn,d)$,
  $\Delta\in{}ElDesign(d,\mathcal{D}_\mathcal{H}),\sigma_{e},\sigma_0\in{}\Sigma(\Delta)$,
  assume that:
  \begin{itemize}
  \item SITPN $sitpn$ translates into design $d$:
    $\lfloor{}sitpn\rfloor_\mathcal{H}=(d,\gamma)$
  \item $\Delta$ is the elaborated version of $d$, $\sigma_e$ is the
    default state of $\Delta$, i.e, state of $\Delta$ where all signals have their default value:\\
    $\mathcal{D}_\mathcal{H},\emptyset\vdash{}d\srarrow{elab}{\fontsize{6}{8}\selectfont}(\Delta,\sigma_{e})$
    
  \item $\sigma_0$ is the initial state of $\Delta$: 
    $\Delta,\sigma_{e}\vdash{}d.cs\srarrow{init}{\fontsize{6}{8}\selectfont}\sigma_0$
  \end{itemize}
\end{definition}

\def\inithyps{For all $sitpn\in{}SITPN$, $d\in{}design$,
  $\gamma\in{}WM(sitpn,d)$,
  $\Delta\in{}ElDesign(d,\mathcal{D}_\mathcal{H}),\sigma_{e},\sigma_0\in{}\Sigma(\Delta)$
  that verify the hypotheses of Definition~\ref{def:init-states-hyps},}

%%%%%%%%%%%%%%%%%%%%%%%%%%%%%%%%%%%%%%%%%%
%%%%%% SIMILAR INITIAL STATES LEMMA %%%%%%
%%%%%%%%%%%%%%%%%%%%%%%%%%%%%%%%%%%%%%%%%%

\begin{lemma}[Similar Initial States]
  \label{lem:sim-init-states}
  \inithyps{} then $\gamma\vdash{}s_0\sim\sigma_0$.
\end{lemma}

\begin{proof}
  By definition of the \nameref{def:state-sim} relation, there are 6
  points to prove.
  \begin{frameb}
    \begin{enumerate}
    \item\label{item:init-sim-mark} $\forall{}p\in{}P,id_p\in{}Comps(\Delta)~s.t.~\gamma(p)=id_p,$
      $~s_0.M(p)=\sigma_0(id_p)("s\_marking")$.
    \item\label{item:init-sim-tc}
      $\forall{}t\in{}T_i,id_t\in{}Comps(\Delta)~s.t.~\gamma(t)=id_t,$\\
      $\big(upper(I_s(t))=\infty\land{}s_0.I(t)\le{}lower(I_s(t))\Rightarrow{}s_0.I(t)=\sigma_0(id_t)("s\_time\_counter")\big)$\\
      $\land\big(upper(I_s(t))=\infty\land{}s_0.I(t)>{}lower(I_s(t))\Rightarrow{}\sigma_0(id_t)("s\_time\_counter")=lower(I_s(t))\big)$\\
      $\land\big(upper(I_s(t))\neq\infty\land{}s_0.I(t)>{}upper(I_s(t))\Rightarrow{}\sigma_0(id_t)("s\_time\_counter")=upper(I_s(t))\big)$\\
      $\land\big(upper(I_s(t))\neq\infty\land{}s_0.I(t)\le{}upper(I_s(t))\Rightarrow{}s_0.I(t)=\sigma_0(id_t)("s\_time\_counter")\big)$.
    \item\label{item:init-sim-reset}
      $\forall{}t\in{}T_i,id_t\in{}Comps(\Delta)~s.t.~\gamma(t)=id_t,$
      $s_0.reset_t(t)=\sigma_0(id_t)("s\_reinit\_time\_counter")$.
    \item\label{item:init-sim-cond}
      $\forall{}c\in\mathcal{C},id_c\in{}Ins(\Delta)~s.t.~\gamma(c)=id_c,~s_0.cond(c)=\sigma_0(id_c)$.
    \item\label{item:init-sim-act}
      $\forall{}a\in\mathcal{A},id_a\in{}Outs(\Delta)~s.t.~\gamma(a)=id_a,~s_0.ex(a)=\sigma_0(id_a)$.
    \item\label{item:init-sim-fun}
      $\forall{}f\in\mathcal{F},id_f\in{}Outs(\Delta)~s.t.~\gamma(f)=id_f,~s_0.ex(f)=\sigma_0(id_f)$.
    \end{enumerate}
  \end{frameb}

  \begin{itemize}
  \item Apply the \nameref{lem:init-states-eq-marking} lemma to solve \ref{item:init-sim-mark}.
  \item Apply the \nameref{lem:init-states-eq-tc} lemma to solve \ref{item:init-sim-tc}.
  \item Apply the \nameref{lem:init-states-eq-rorders} lemma to solve \ref{item:init-sim-reset}.
  \item Apply the \nameref{lem:init-states-cond-vals} lemma to solve \ref{item:init-sim-cond}.
  \item Apply the \nameref{lem:init-states-act-exec} lemma to solve \ref{item:init-sim-act}.
  \item Apply the \nameref{lem:init-states-fun-exec} lemma to solve \ref{item:init-sim-fun}.
  \end{itemize}
\end{proof}

\subsection{Initial states and marking}
\label{sec:init-states-marking}

\begin{lemma}[Initial States Equal Marking]
  \label{lem:init-states-eq-marking}
  \inithyps{} then
  $\forall{}p\in{}P,id_p\in{}Comps(\Delta)~s.t.~\gamma(p)=id_p$, then $~s_0.M(p)=\sigma_0(id_p)("s\_marking")$.
\end{lemma}

\begin{niproof}
  Given a $p\in{}P$ and an $id_p\in{}Comps(\Delta)$
  s.t. $\gamma(p)=id_p$, let us show that\\
  \framebox{$s_0.M(p)=\sigma_0(id_p)("s\_marking")$.}

  By construction and by definition of $id_p$, there exist
  $gm_p,ipm_p,opm_p~s.t.~\mathtt{comp}(id_p,"place",gm_p,ipm_p,opm_p)\in{}d.cs$.\\

  By property of the \hvhdl{} initialization relation, \InCsCompP{},
  and through the examination of the \texttt{marking} process defined
  in the place design architecture, we can deduce
  $\sigma_0(id_p)("s\_marking")=\sigma_0(id_p)("initial\_marking")$.

  \noindent{}Rewriting $\sigma_0(id_p)("s\_marking")$ as $\sigma_0(id_p)("initial\_marking")$,
  \framebox{$\sigma_p^0("initial\_marking")=s_0.M(p)$.}
  
  By construction,
  ${<}\mathtt{initial\_marking\Rightarrow}M_0(p){>}\in{}ipm_p$.

  By property of the \hvhdl{} initialization relation, and \InCsCompP,
  then $\sigma_0(id_p)("initial\_marking")=M_0(p)$.  Rewriting
  $\sigma_0(id_p)("initial\_marking")$ as $M_0(p)$ in the current
  goal: \framebox{$M_0(p)=s_0.M(p)$.}
  
  By definition of $s_0$, we can rewrite $s_0.M(p)$ as $M_0(p)$ in the
  current goal, \qedbox{tautology.}
  
\end{niproof}

\subsection{Initial states and time counters}
\label{sec:init-states-tc}

\begin{lemma}[Initial States Equal Time Counters]
  \label{lem:init-states-eq-tc}
  \inithyps{} then
  $\forall{}t\in{}T_i,id_t\in{}Comps(\Delta)~s.t.~\gamma(t)=id_t$,\\
  $upper(I_s(t))=\infty\land{}s_0.I(t)\le{}lower(I_s(t))\Rightarrow{}s_0.I(t)=\sigma_0(id_t)("s\_time\_counter")\land{}$\\
  $upper(I_s(t))=\infty\land{}s_0.I(t)>{}lower(I_s(t))\Rightarrow{}\sigma_0(id_t)("s\_time\_counter")=lower(I_s(t))\land{}$\\
  $upper(I_s(t))\neq\infty\land{}s_0.I(t)>{}upper(I_s(t))\Rightarrow{}\sigma_0(id_t)("s\_time\_counter")=upper(I_s(t))\land{}$\\
  $upper(I_s(t))\neq\infty\land{}s_0.I(t)\le{}upper(I_s(t))\Rightarrow{}s_0.I(t)=\sigma_0(id_t)("s\_time\_counter")$.
\end{lemma}

\begin{niproof}
  Given a $t\in{}T_i$ and an $id_t\in{}Comps(\Delta)$
  s.t. $\gamma(t)=id_t$, let us show that:
  \begin{enumerate}
  \item \framebox{$upper(I_s(t))=\infty\land{}s_0.I(t)\le{}lower(I_s(t))\Rightarrow{}s_0.I(t)=\sigma_0(id_t)("s\_time\_counter")$}
  \item \framebox{$upper(I_s(t))=\infty\land{}s_0.I(t)>{}lower(I_s(t))\Rightarrow{}\sigma_0(id_t)("s\_time\_counter")=lower(I_s(t))$}
  \item \framebox{$upper(I_s(t))\neq\infty\land{}s_0.I(t)>{}upper(I_s(t))\Rightarrow{}\sigma_0(id_t)("s\_time\_counter")=upper(I_s(t))$}
  \item \framebox{$upper(I_s(t))\neq\infty\land{}s_0.I(t)\le{}upper(I_s(t))\Rightarrow{}s_0.I(t)=\sigma_0(id_t)("s\_time\_counter")$}
  \end{enumerate}

  \exP

  Then, let us show the 4 previous points.
  
  \begin{enumerate}
  \item Assuming that $upper(I_s(t))=\infty\land{}s_0.I(t)\le{}lower(I_s(t))$, then let us show\\
    \framebox{${}s_0.I(t)=\sigma_0(id_t)("s\_time\_counter")$.}
    
    Rewriting $s_0.I(t)$ as $0$, by definition of $s_0$,
    \framebox{$\sigma_0(id_t)("s\_time\_counter")=0$.}

    \noindent By property of the \hvhdl{} initialization relation,
    \InCsCompT, and through the examination of the
    \texttt{time\_counter} process defined in the transition design
    architecture, we can deduce
    \qedbox{$\sigma_0(id_t)("s\_time\_counter")=0$.}
    
  \item Assuming that $upper(I_s(t))=\infty$ and
    $s_0.I(t)>{}lower(I_s(t))$, let us show\\
    \framebox{$\sigma_0(id_t)("s\_time\_counter")=lower(I_s(t))$}.

    By definition, $lower(I_s(t))\in\mathbb{N}^{*}$ and
    $s_0.I(t)=0$. Then, \qedbox{$lower(I_s(t)){}<0$ is a
      contradiction.}
    
  \item Assuming that $upper(I_s(t))\neq\infty$ and
    $s_0.I(t)>{}upper(I_s(t))$, let us show\\
    \framebox{$\sigma_0(id_t)("s\_time\_counter")=upper(I_s(t))$}.

    By definition, $upper(I_s(t))\in\mathbb{N}^{*}$ and
    $s_0.I(t)=0$. Then, \qedbox{$upper(I_s(t)){}<0$ is a
      contradiction.}
    
  \item Assuming that $upper(I_s(t))\neq\infty$ and
    $s_0.I(t)\le{}upper(I_s(t))$, let us
    show\\ \framebox{$s_0.I(t)=\sigma_0(id_t)("s\_time\_counter")$}.\\
 
    Rewriting $s_0.I(t)$ as $0$, by definition of $s_0$,
    \framebox{$\sigma_0(id_t)("s\_time\_counter")=0$.}

    By property of the \hvhdl{} initialization relation, \InCsCompT,
    and through the examination of the \texttt{time\_counter} process
    defined in the transition design architecture, we can deduce
    \qedbox{$\sigma_0(id_t)("s\_time\_counter")=0$.}
  \end{enumerate}
\end{niproof}

\subsection{Initial states and reset orders}
\label{sec:init-states-rorders}

\begin{lemma}[Initial States Equal Reset Orders]
  \label{lem:init-states-eq-rorders}
  \inithyps{} then
  $\forall{}t\in{}T_i,id_t\in{}Comps(\Delta)~s.t.~\gamma(t)=id_t$,
  $s_0.reset_t(t)=\sigma_0(id_t)("s\_reinit\_time\_counter")$.
\end{lemma}

\begin{niproof}
  Given a $t\in{}T_i$ and an $id_t\in{}Comps(\Delta)$ s.t.
  $\gamma(t)=id_t$, let us show
  that\\
  \framebox{$s_0.reset_t(t)=\sigma_0(id_t)("s\_reinit\_time\_counter")$.}
  
  Rewriting $s_0.reset_t(t)$ as $false$, by definition of $s_0$,
  \framebox{$\sigma_0(id_t)("s\_reinit\_time\_counter")=false$.}\\
  
  \exT{}
  
  \noindent By property of the \hvhdl{} initialization relation,
  \InCsCompT, and through the examination of the \texttt{reinit\_time\_counter\_evaluation} process defined in the transition design architecture\\
  we can deduce
  $\sigma_0(id_t)("s\_reinit\_time\_counter")=\prod\limits_{i=0}^{\Delta(id_t)("ian")-1}\sigma_0(id_t)("rt")[i]$.

  Rewriting
  $\sigma_0(id_t)("s\_reinit\_time\_counter")$ as $\prod\limits_{i=0}^{\Delta(id_t)("ian")-1}\sigma_0(id_t)("rt")[i]$,\\
  \framebox{$\prod\limits_{i=0}^{\Delta(id_t)("ian")-1}\sigma_0(id_t)("rt")[i]=false$.}\\
  
  For all $t\in{}T$ (resp. $p\in{}P$), let $input(t)$
  (resp. $input(p)$) be the set of input places of $t$ (resp. input
  transitions of $p$), and let $output(t)$ (resp. $output(p)$) be the
  set of output places of $t$ (resp. output transitions of $p$).\\

  Let us perform case analysis on $input(t)$; there are 2 cases:

  \begin{itemize}
  \item \textbf{CASE} $input(t)=\emptyset$.

    By construction,
    ${<}\mathtt{input\_arcs\_number\Rightarrow}1{>}\in{}gm_t$, and by
    property of the elaboration relation, and \InCsCompT{}, we can
    deduce $\Delta(id_t)("ian")=1$.

    By construction, $<\mathtt{reinit\_time(0)\Rightarrow}false>\in{}ipm_t$, and
    by property of the initialization relation and \InCsCompT, we can
    deduce $\sigma_0(id_t)("rt")[0]=false$.

    Rewriting $\Delta(id_t)("ian")$ as $1$ and
    $\sigma_0(id_t)("rt")[0]$ as $false$, \qedbox{tautology.}
    
  \item \textbf{CASE} $input(t)\neq\emptyset$.

    To prove the current goal, we can equivalently prove that\\
    \fbox{$\exists{}i\in[0,\Delta(id_t)("ian")-1]~s.t.~\sigma_0(id_t)("rt")[i]=false$.}

    Since $input(t)\neq\emptyset,~\exists{}p~s.t.~p\in{}input(t)$. Let
    us take such a $p\in{}input(t)$.
    
    By construction, for all $p\in{}P$, there exist
    $id_p~s.t.~\gamma(p)=id_p$.

    \exP{}

    \noindent{}By construction, there exist
    $i\in[0,\vert{}input(t)\vert{}-1]$,
    $j\in[0,\vert{}output(p)\vert{}-1]$, $id_{ji}\in{}Sigs(\Delta)$
    s.t.
    ${<}\mathtt{reinit\_transitions\_time(j)\Rightarrow}id_{ji}{>}\in{}opm_p$
    and
    ${<}\mathtt{reinit\_time(i)\Rightarrow}id_{ji}{>}\in{}ipm_t$. Let us take such a $i$, $j$ and $id_{ji}$.

    By construction and $input(t)\neq\emptyset$,
    ${<}\mathtt{input\_arcs\_number\Rightarrow}\vert{}input(t)\vert{}{>}\in{}gm_t$.

    By property of the \hvhdl{} elaboration relation and
    ${<}\mathtt{input\_arcs\_number\Rightarrow}\vert{}input(t)\vert{}{>}\in{}gm_t$,
    we can deduce $\Delta(id_t)("ian")=\vert{}input(t)\vert$.

    Since $\Delta(id_t)("ian")=\vert{}input(t)\vert$ and we have an
    $i\in[0,\vert{}input(t)\vert-1]$, then, we have an
    $i\in[0, \Delta(id_t)("ian")-1]$. Let us take that i to
    prove the goal.

    Then, we must show \framebox{$\sigma_0(id_t)("rt")[i]=false$.}

    By property of the \hvhdl{} initialization relation and
    ${<}\mathtt{reinit\_time(i)\Rightarrow}id_{ji}{>}\in{}ipm_t$, we
    can deduce $\sigma_0(id_t)("rt")[i]=\sigma_0("id_{ji}")$.

    Rewriting $\sigma_0(id_t)("rt")[i]$ as $\sigma_0("id_{ji}")$,
    \framebox{$\sigma_0("id_{ji}")=false$.}

    By property of the \hvhdl{} initialization relation and
    $<\mathtt{reinit\_transitions\_time(j)\Rightarrow}id_{ji}>\in{}opm_p$,
    we can deduce $\sigma_0("id_{ji}")=\sigma_0(id_p)("rtt")[j]$.

    Rewriting $\sigma_0("id_{ji}")$ as $\sigma_0(id_p)("rtt")[j]$,
    \framebox{$\sigma_p^0("rtt")[j]=false$.}

    Since $t\in{}output(p)$, then we know that
    $output(p)\neq\emptyset$.

    Then, by construction,
    ${<}\mathtt{output\_arcs\_number\Rightarrow}\vert{}output(p)\vert{>}\in{}gm_p$.

    By property of the elaboration relation and
    ${<}\mathtt{output\_arcs\_number}\Rightarrow\vert{}output(p)\vert{>}\in{}gm_p$,
    we can deduce that $\Delta(id_p)("oan")=\vert{}output(p)\vert$.

    Since $\Delta(id_p)("oan")=\vert{}output(p)\vert$ and
    $j\in[0,\vert{}output(p)\vert-1]$, then
    $j\in[0,\Delta(id_p)("oan")-1]$.
    
    By property of the \hvhdl{} initialization relation, \InCsCompP,
    through the examination of the
    \texttt{reinit\_transitions\_time\_evaluation} process defined in
    the place design architecture, and since
    $j\in[0,\Delta(id_p)("oan")-1]$,
    \qedbox{$\sigma_0(id_p)("rtt")[j]=false$.}

  \end{itemize}
  
\end{niproof}

\subsection{Initial states and condition values}
\label{sec:init-states-cond-vals}

\begin{lemma}[Initial States Equal Condition Values]
  \label{lem:init-states-cond-vals}
  \inithyps{} 
  then
  $\forall{}c\in\mathcal{C},id_c\in{}Ins(\Delta)~s.t.~\gamma(c)=id_c,~s_0.cond(c)=\sigma_0(id_c)$.
\end{lemma}

\begin{proof}
  Given a $c\in\mathcal{C}$ and an
  $id_c\in{}Ins(\Delta)~s.t.~\gamma(c)=id_c$, let's show that
  \fbox{$s_0.cond(c)=\sigma_0(id_c)$.}\\

  \noindent{}Rewriting $s_0.cond(c)$ as $false$, by definition of
  $s_0$, \fbox{$\sigma_0(id_c)=false$.}

  \noindent{}By construction, $id_c$ is an input port identifier of
  boolean type in the \hvhdl{} design $d$.

  \noindent{}By property, of the \hvhdl{} elaboration relation,
  $\sigma_e(id_c)=false$, where $false$ is the default value
  associated to signals of the boolean type during the elaboration
  (see definition of default value in chapter \hvhdl{} semantics).

  \noindent{}By property of the \hvhdl{} initialization relation, we
  have $\sigma_e(id_c)=\sigma_0(id_c)$ (i.e, input ports are not
  assigned during the initialization phase).

  \noindent{}Rewriting $\sigma_e(id_c)$ as $false$,
  \colorbox{red!20}{$\sigma_0(id_c)=false$.}
  
\end{proof}

\subsection{Initial states and action executions}
\label{sec:init-states-act-exec}

\begin{todobox}
  Correction: $id_f$ is assigned by the reset block of the function process
\end{todobox}

\begin{lemma}[Initial States Equal Action Executions]
  \label{lem:init-states-act-exec}
  \inithyps{} 
  then
  $\forall{}a\in\mathcal{A},id_a\in{}Outs(\Delta)~s.t.~\gamma(a)=id_a,~s_0.ex(a)=\sigma_0(id_a)$.
\end{lemma}

\begin{proof}
  Given a $a\in\mathcal{A}$ and an
  $id_a\in{}Outs(\Delta)~s.t.~\gamma(a)=id_a$, let's show that
  \fbox{$s_0.ex(a)=\sigma_0(id_a)$.}\\

  \noindent{}Rewriting $s_0.ex(a)$ as $false$, by definition of
  $s_0$, \fbox{$\sigma_0(id_a)=false$.}

  \noindent{}By construction, $id_a$ is an output port identifier of
  boolean type in the \hvhdl{} design $d$.

  \noindent{}By property, of the \hvhdl{} elaboration relation,
  $\sigma_e(id_a)=false$, where $false$ is the default value
  associated to signals of the boolean type during the elaboration
  (see definition of default value in chapter \hvhdl{} semantics).

  \noindent{}By construction, we know that the output port identifier
  $id_a$ is assigned in the generated \texttt{action} process, only at
  the falling edge phase of the simulation cycle (i.e, the assignment
  takes place in a \texttt{falling} statement block).
  
  \noindent{}By property of the \hvhdl{} initialization relation, and
  we have $\sigma_e(id_a)=\sigma_0(id_a)$ (i.e, process
  \texttt{action} is idle during the initialization phase).

  \noindent{}Rewriting $\sigma_e(id_a)$ as $false$,
  \colorbox{red!20}{$\sigma_0(id_a)=false$.}
  
\end{proof}

\subsection{Initial states and function executions}
\label{sec:init-states-fun-exec}

\begin{todobox}
  Correction: $id_f$ is assigned by the reset block of the function process
\end{todobox}

\begin{lemma}[Initial States Equal Function Executions]
  \label{lem:init-states-fun-exec}
  \inithyps{} 
  then
  $\forall{}f\in\mathcal{F},id_f\in{}Outs(\Delta)~s.t.~\gamma(f)=id_f,~s_0.ex(f)=\sigma_0(id_f)$.
\end{lemma}

\begin{proof}
  Given a $f\in\mathcal{F}$ and an
  $id_f\in{}Outs(\Delta)~s.t.~\gamma(f)=id_f$, let's show that
  \fbox{$s_0.ex(f)=\sigma_0(id_f)$.}\\

  \noindent{}Rewriting $s_0.ex(f)$ as $false$, by definition of $s_0$,
  \fbox{$\sigma_0(id_f)=false$.}

  \noindent{}By construction, $id_f$ is an output port identifier of
  boolean type in the \hvhdl{} design $d$.

  \noindent{}By property, of the \hvhdl{} elaboration relation,
  $\sigma_e(id_f)=false$, where $false$ is the default value
  associated to signals of the boolean type during the elaboration
  (see definition of default value in chapter \hvhdl{} semantics).

  \noindent{}By construction, we know that the output port identifier
  $id_f$ is assigned in the generated \texttt{function} process (i.e,
  \texttt{function} is the process identifier), only at the rising
  edge phase of the simulation cycle (i.e, the assignment takes place
  in a \texttt{rising} statement block).
  
  \noindent{}By property of the \hvhdl{} initialization relation, and
  we have $\sigma_e(id_f)=\sigma_0(id_f)$ (i.e, process
  \texttt{function} is idle during the initialization phase).

  \noindent{}Rewriting $\sigma_e(id_f)$ as $false$,
  \colorbox{red!20}{$\sigma_0(id_f)=false$.}
  
\end{proof}

%%% Local Variables:
%%% mode: latex
%%% TeX-master: "../../main"
%%% End:
