%%%%%%%%%%%%%%%%%%%%%%%%%%%%%%%%%%%%%%%%%%%%%%%
%%%%%% SIMILAR INITIAL STATES HYPOTHESES %%%%%%
%%%%%%%%%%%%%%%%%%%%%%%%%%%%%%%%%%%%%%%%%%%%%%%

\begin{definition}[Initial state hypotheses]
  \label{def:init-states-hyps}
  Given an $sitpn\in{}SITPN$, $b\in{}P\rightarrow\mathbb{N}$,
  $d\in{}design$, $\gamma\in{}WM(sitpn,$ $d)$,
  $\Delta\in{}ElDesign,\sigma_{e},\sigma_0\in{}\Sigma$, assume that:
  \begin{itemize}
  \item SITPN $sitpn$ is transformed into the design $d$ and yields
    the binder $\gamma$: $\lfloor{}sitpn\rfloor_b=(d,\gamma)$
  \item $\Delta$ is the elaborated version of $d$, $\sigma_e$ is the
    default state of $\Delta$, i.e. the state of $\Delta$ where all signals are initialized to their default value:\\
    $\mathcal{D}_\mathcal{H},\emptyset\vdash{}d\srarrow{elab}{\fontsize{6}{8}\selectfont}(\Delta,\sigma_{e})$
    
  \item $\sigma_0$ is the initial state of $\Delta$: 
    $\Delta,\sigma_{e}\vdash{}d.cs\srarrow{init}{\fontsize{6}{8}\selectfont}\sigma_0$
  \end{itemize}
\end{definition}

\def\inithyps{For all $sitpn\in{}SITPN$,
  $b\in{}P\rightarrow\mathbb{N}$, $d\in{}design$,
  $\gamma\in{}WM(sitpn,d)$,
  $\Delta\in{}ElDesign,\sigma_{e},\sigma_0\in{}\Sigma$ that verify the
  hypotheses of Definition~\ref{def:init-states-hyps},}

%%%%%%%%%%%%%%%%%%%%%%%%%%%%%%%%%%%%%%%%%%
%%%%%% SIMILAR INITIAL STATES LEMMA %%%%%%
%%%%%%%%%%%%%%%%%%%%%%%%%%%%%%%%%%%%%%%%%%

\begin{lemma}[Similar initial states]
  \label{lem:sim-init-states}
  \inithyps{} then $\gamma\vdash{}s_0\sim\sigma_0$.
\end{lemma}

\begin{niproof}
  By definition of the \nameref{def:state-sim} relation, there are 6
  points to prove.
  \begin{frameb}
    \begin{enumerate}
    \item\label{item:init-sim-mark} $\forall{}p\in{}P,id_p\in{}Comps(\Delta)~s.t.~\gamma(p)=id_p,$
      $~s_0.M(p)=\sigma_0(id_p)(\texttt{s\_marking})$.
    \item\label{item:init-sim-tc}
      $\forall{}t\in{}T_i,id_t\in{}Comps(\Delta)~s.t.~\gamma(t)=id_t,$\\
      $\big(upper(I_s(t))=\infty\land{}s_0.I(t)\le{}lower(I_s(t))\Rightarrow{}s_0.I(t)=\sigma_0(id_t)(\texttt{s\_time\_counter})\big)$\\
      $\land\big(upper(I_s(t))=\infty\land{}s_0.I(t)>{}lower(I_s(t))\Rightarrow{}\sigma_0(id_t)(\texttt{s\_time\_counter})=lower(I_s(t))\big)$\\
      $\land\big(upper(I_s(t))\neq\infty\land{}s_0.I(t)>{}upper(I_s(t))\Rightarrow{}\sigma_0(id_t)(\texttt{s\_time\_counter})=upper(I_s(t))\big)$\\
      $\land\big(upper(I_s(t))\neq\infty\land{}s_0.I(t)\le{}upper(I_s(t))\Rightarrow{}s_0.I(t)=\sigma_0(id_t)(\texttt{s\_time\_counter})\big)$.
    \item\label{item:init-sim-reset}
      $\forall{}t\in{}T_i,id_t\in{}Comps(\Delta)~s.t.~\gamma(t)=id_t,$
      $s_0.reset_t(t)=\sigma_0(id_t)(\texttt{s\_reinit\_time\_counter})$.
    \item\label{item:init-sim-cond}
      $\forall{}c\in\mathcal{C},id_c\in{}Ins(\Delta)~s.t.~\gamma(c)=id_c,~s_0.cond(c)=\sigma_0(id_c)$.
    \item\label{item:init-sim-act}
      $\forall{}a\in\mathcal{A},id_a\in{}Outs(\Delta)~s.t.~\gamma(a)=id_a,~s_0.ex(a)=\sigma_0(id_a)$.
    \item\label{item:init-sim-fun}
      $\forall{}f\in\mathcal{F},id_f\in{}Outs(\Delta)~s.t.~\gamma(f)=id_f,~s_0.ex(f)=\sigma_0(id_f)$.
    \end{enumerate}
  \end{frameb}

  \begin{itemize}
  \item Apply the \nameref{lem:init-states-eq-marking} lemma to solve \ref{item:init-sim-mark}.
  \item Apply the \nameref{lem:init-states-eq-tc} lemma to solve \ref{item:init-sim-tc}.
  \item Apply the \nameref{lem:init-states-eq-rorders} lemma to solve \ref{item:init-sim-reset}.
  \item Apply the \nameref{lem:init-states-cond-vals} lemma to solve \ref{item:init-sim-cond}.
  \item Apply the \nameref{lem:init-states-act-exec} lemma to solve \ref{item:init-sim-act}.
  \item Apply the \nameref{lem:init-states-fun-exec} lemma to solve \ref{item:init-sim-fun}.
  \end{itemize}
\end{niproof}

\subsection{Initial states and marking}
\label{sec:init-states-marking}

\begin{lemma}[Initial states equal marking]
  \label{lem:init-states-eq-marking}
  \inithyps{} then
  $\forall{}p\in{}P,id_p\in{}Comps(\Delta)$ s.t. $\gamma(p)=id_p$,
  $~s_0.M(p)=\sigma_0(id_p)(\texttt{s\_marking})$.
\end{lemma}

\begin{niproof}
  Given a $p\in{}P$ and an $id_p\in{}Comps(\Delta)$
  s.t. $\gamma(p)=id_p$, let us show that\\
  \framebox{$s_0.M(p)=\sigma_0(id_p)(\texttt{s\_marking})$.}

  \exP{}
  
  By property of the \hvhdl{} initialization relation, \InCsCompP{},
  and through the examination of the \texttt{marking} process defined
  in the place design architecture, we can deduce
  $\sigma_0(id_p)(\texttt{s\_marking})=\sigma_0(id_p)(\texttt{initial\_marking})$.

  Rewriting $\sigma_0(id_p)(\texttt{sm})$ as
  $\sigma_0(id_p)(\texttt{initial\_marking})$,
  \framebox{$\sigma_0(id_p)(\texttt{initial\_marking})=s_0.M(p)$.}
  
  By construction,
  ${<}\mathtt{initial\_marking\Rightarrow}M_0(p){>}\in{}i_p$.

  By property of the \hvhdl{} initialization relation, and \InCsCompP,
  then $\sigma_0(id_p)(\texttt{initial\_marking})=M_0(p)$.  Rewriting
  $\sigma_0(id_p)(\texttt{initial\_marking})$ as $M_0(p)$ in the current
  goal: \framebox{$M_0(p)=s_0.M(p)$.}
  
  By definition of $s_0$, we can rewrite $s_0.M(p)$ as $M_0(p)$ in the
  current goal, \qedbox{tautology.}
  
\end{niproof}

\begin{lemma}[Null input token sum at initial state]
  \label{lem:init-states-sits-zero}
  \inithyps{} then
  $\forall{}p\in{}P,id_p\in{}Comps(\Delta)~s.t.~\gamma(p)=id_p$,
  $\sigma_0(id_p)(\texttt{s\_input\_token\_sum})=0$.
\end{lemma}

\begin{niproof}
  Given a $p$ and an $id_p$ s.t. $\gamma(p)=id_p$, let us show that
  \fbox{$\sigma_0(id_p)(\texttt{s\_input\_token\_sum})=0$.}

  \exP{}
  
  By property of the initialization relation, \InCsCompP{}, and
  through the examination of the \texttt{input\_tokens\_sum} process
  defined in the place design architecture, we can deduce:
  \begin{equation}
    \label{eq:sits-at-init-state}
    \sigma_0(id_p)(\texttt{sits})=\sum\limits_{i=0}^{\Delta(id_p)(\texttt{ian})-1}
    \begin{cases}
      \sigma_0(id_p)(\texttt{iaw})[i]~\mathtt{if}~\sigma_0(id_p)(\texttt{itf})[i]\\
      0~otherwise \\
    \end{cases}
  \end{equation}
  
  Rewriting the goal with Equation~\eqref{eq:sits-at-init-state}:\\
  \fbox{$\sum\limits_{i=0}^{\Delta(id_p)(\texttt{ian})-1}\begin{cases}
      \sigma_0(id_p)(\texttt{iaw})[i]~\mathtt{if}~\sigma_0(id_p)(\texttt{itf})[i]\\
      0~otherwise \\
    \end{cases}=0$.}

  \noindent{}Let us perform case analysis on $input(p)$; there are two cases:

  \begin{enumerate}
  \item $input(p)=\emptyset$:
    
    By construction, we have
    ${<}$\texttt{input\_arcs\_number}$\Rightarrow{}1{>}\in{}g_p$,\\
    ${<}$\texttt{input\_transitions\_fired(0)}$\Rightarrow{}\mathtt{true}{>}\in{}i_p$,\\
    and
    ${<}\texttt{input\_arcs\_weights(0)}\Rightarrow{}0{>}\in{}i_p$.

  By property of the elaboration relation, \InCsCompP{}, and
  ${<}$\texttt{input\_arcs\_number}$\Rightarrow{}1{>}\in{}g_p$, we
  can deduce $\Delta(id_p)(\texttt{ian})=1$.

  By property of the initialization relation, \InCsCompP,
  ${<}$\texttt{input\_transitions\_fired(0)}$\Rightarrow{}\mathtt{true}{>}\in{}i_p$
  and
  ${<}\mathtt{input\_arcs\_weights(0)\Rightarrow{}0}{>}\in{}i_p$, we
  can deduce $\sigma_0(id_p)(\texttt{itf})[0]=\mathtt{true}$ and
  $\sigma_0(id_p)(\texttt{iaw})[0]=0$.

  Rewriting the goal with $\Delta(id_p)(\texttt{ian})=1$,
  $\sigma_0(id_p)(\texttt{itf})[0]=\mathtt{true}$,
  $\sigma_0(id_p)(\texttt{iaw})[0]=0$ and simplifying the goal,
  \qedbox{tautology.}
  
\item $input(p)\neq\emptyset$:

  By construction,
  ${<}\texttt{input\_arcs\_number}\Rightarrow{}\vert{}input(p)\vert{>}\in{}g_p$,
  and by property of the elaboration relation, and \InCsCompP{}, we
  can deduce $\Delta(id_p)(\texttt{ian})=\vert{}input(p)\vert$.
  
  Let us reason by induction on the sum term of the goal.

  \begin{itemize}
  \item \textbf{BASE CASE}: The sum term equals 0, then \qedbox{tautology.}

  \item \textbf{INDUCTION CASE}:
    \begin{ih}
      $\sum\limits_{i=1}^{\Delta(id_p)(\texttt{ian})-1}\begin{cases}
        \sigma_0(id_p)(\texttt{iaw})[i]~\mathtt{if}~\sigma_0(id_p)(\texttt{itf})[i]\\
        0~otherwise \\
      \end{cases}=0$
    \end{ih}

    \begin{frameb}
      \begin{tabular}{c}
        $\begin{cases}
          \sigma_0(id_p)(\texttt{iaw})[0]~\mathtt{if}~\sigma_0(id_p)(\texttt{itf})[0]\\
          0~otherwise \\
        \end{cases}$ \\
        $+$ \\
        $\sum\limits_{i=1}^{\Delta(id_p)(\texttt{ian})-1}\begin{cases}
        \sigma_0(id_p)(\texttt{iaw})[i]~\mathtt{if}~\sigma_0(id_p)(\texttt{itf})[i]\\
        0~otherwise \\
      \end{cases}=0$
      \end{tabular}
    \end{frameb}

    Using the induction hypothesis to rewrite the goal:\\
    \fbox{$\begin{cases}
        \sigma_0(id_p)(\texttt{iaw})[0]~\mathtt{if}~\sigma_0(id_p)(\texttt{itf})[0]\\
        0~otherwise \\
      \end{cases}=0$}
    
    Since $input(p)\neq\emptyset$, by construction, there exist an
    $id_t\in{}Comps(\Delta),g_t,i_t,o_t$ s.t. \InCsCompT{},
    $id_{ft}\in{}Sigs(\Delta)$ s.t.
    ${<}\mathtt{fired\Rightarrow}id_{ft}{>}\in{}o_t$ and\\
    ${<}\mathtt{input\_transitions\_fired(0)\Rightarrow{}id_{ft}}{>}\in{}i_p$.

    By property of the initialization relation, \InCsCompP{},
    \InCsCompT{}, ${<}\mathtt{fired\Rightarrow}id_{ft}{>}\in{}o_t$
    and
    ${<}$\texttt{input\_transitions\_fired(0)}$\Rightarrow{}\mathtt{id_{ft}}{>}\in{}i_p$,
    we can deduce $\sigma_0(id_p)(\texttt{itf})[0]=\sigma_0(id_t)(\texttt{fired})$.

    Rewriting the goal with $\sigma_0(id_p)(\texttt{itf})[0]=\sigma_0(id_t)(\texttt{fired})$:\\
    \fbox{$\begin{cases}
        \sigma_0(id_p)(\texttt{iaw})[0]~\mathtt{if}~\sigma_0(id_t)(\texttt{fired})\\
        0~otherwise \\
      \end{cases}=0$}

    Appealing to Lemma~\ref{lem:init-states-fired-false}, we can
    deduce $\sigma_0(id_t)(\texttt{fired})=\mathtt{false}$.

    Rewriting the goal with $\sigma_0(id_t)(\texttt{fired})=\mathtt{false}$,
    and simplifying the goal, \qedbox{tautology.}
  \end{itemize}
\end{enumerate}
  
\end{niproof}

\begin{lemma}[Null output token sum at initial state]
  \label{lem:init-states-sots-zero}
  \inithyps{} then
  $\forall{}p\in{}P,id_p\in{}Comps(\Delta)~s.t.~\gamma(p)=id_p$,
  $\sigma_0(id_p)(\texttt{s\_output\_token\_sum})=0$.
\end{lemma}

\begin{niproof}
  The proof is similar to the proof of
  Lemma~\ref{lem:init-states-sits-zero}.
\end{niproof}

\subsection{Initial states and time counters}
\label{sec:init-states-tc}

\begin{lemma}[Initial states equal time counters]
  \label{lem:init-states-eq-tc}
  \inithyps{} then
  $\forall{}t\in{}T_i,id_t\in{}Comps(\Delta)$ s.t. $\gamma(t)=id_t$,\\
  $upper(I_s(t))=\infty\land{}s_0.I(t)\le{}lower(I_s(t))\Rightarrow{}s_0.I(t)=\sigma_0(id_t)(\texttt{s\_time\_counter})\land{}$\\
  $upper(I_s(t))=\infty\land{}s_0.I(t)>{}lower(I_s(t))\Rightarrow{}\sigma_0(id_t)(\texttt{s\_time\_counter})=lower(I_s(t))\land{}$\\
  $upper(I_s(t))\neq\infty\land{}s_0.I(t)>{}upper(I_s(t))\Rightarrow{}\sigma_0(id_t)(\texttt{s\_time\_counter})=upper(I_s(t))\land{}$\\
  $upper(I_s(t))\neq\infty\land{}s_0.I(t)\le{}upper(I_s(t))\Rightarrow{}s_0.I(t)=\sigma_0(id_t)(\texttt{s\_time\_counter})$.
\end{lemma}

\begin{niproof}
  Given a $t\in{}T_i$ and an $id_t\in{}Comps(\Delta)$
  s.t. $\gamma(t)=id_t$, let us show that:
  \begin{enumerate}
  \item \framebox{$upper(I_s(t))=\infty\land{}s_0.I(t)\le{}lower(I_s(t))\Rightarrow{}s_0.I(t)=\sigma_0(id_t)(\texttt{s\_time\_counter})$}
  \item \framebox{$upper(I_s(t))=\infty\land{}s_0.I(t)>{}lower(I_s(t))\Rightarrow{}\sigma_0(id_t)(\texttt{s\_time\_counter})=lower(I_s(t))$}
  \item \framebox{$upper(I_s(t))\neq\infty\land{}s_0.I(t)>{}upper(I_s(t))\Rightarrow{}\sigma_0(id_t)(\texttt{s\_time\_counter})=upper(I_s(t))$}
  \item \framebox{$upper(I_s(t))\neq\infty\land{}s_0.I(t)\le{}upper(I_s(t))\Rightarrow{}s_0.I(t)=\sigma_0(id_t)(\texttt{s\_time\_counter})$}
  \end{enumerate}

  \exP

  Then, let us show the 4 previous points.
  
  \begin{enumerate}
  \item Assuming that $upper(I_s(t))=\infty\land{}s_0.I(t)\le{}lower(I_s(t))$, then let us show\\
    \framebox{${}s_0.I(t)=\sigma_0(id_t)(\texttt{s\_time\_counter})$.}
    
    Rewriting $s_0.I(t)$ as $0$, by definition of $s_0$,
    \framebox{$\sigma_0(id_t)(\texttt{s\_time\_counter})=0$.}

    \noindent By property of the \hvhdl{} initialization relation,
    \InCsCompT, and through the examination of the
    \texttt{time\_counter} process defined in the transition design
    architecture, we can deduce
    \qedbox{$\sigma_0(id_t)(\texttt{s\_time\_counter})=0$.}
    
  \item Assuming that $upper(I_s(t))=\infty$ and
    $s_0.I(t)>{}lower(I_s(t))$, let us show\\
    \framebox{$\sigma_0(id_t)(\texttt{s\_time\_counter})=lower(I_s(t))$}.

    By definition, $lower(I_s(t))\in\mathbb{N}^{*}$ and
    $s_0.I(t)=0$. Then, \qedbox{$lower(I_s(t)){}<0$ is a
      contradiction.}
    
  \item Assuming that $upper(I_s(t))\neq\infty$ and
    $s_0.I(t)>{}upper(I_s(t))$, let us show\\
    \framebox{$\sigma_0(id_t)(\texttt{s\_time\_counter})=upper(I_s(t))$}.

    By definition, $upper(I_s(t))\in\mathbb{N}^{*}$ and
    $s_0.I(t)=0$. Then, \qedbox{$upper(I_s(t)){}<0$ is a
      contradiction.}
    
  \item Assuming that $upper(I_s(t))\neq\infty$ and
    $s_0.I(t)\le{}upper(I_s(t))$, let us
    show\\ \framebox{$s_0.I(t)=\sigma_0(id_t)(\texttt{s\_time\_counter})$}.
 
    Rewriting $s_0.I(t)$ as $0$, by definition of $s_0$,
    \framebox{$\sigma_0(id_t)(\texttt{s\_time\_counter})=0$.}

    By property of the \hvhdl{} initialization relation, \InCsCompT,
    and through the examination of the \texttt{time\_counter} process
    defined in the transition design architecture, we can deduce
    \qedbox{$\sigma_0(id_t)(\texttt{s\_time\_counter})=0$.}
  \end{enumerate}
\end{niproof}

\subsection{Initial states and reset orders}
\label{sec:init-states-rorders}

\begin{lemma}[Initial states equal reset orders]
  \label{lem:init-states-eq-rorders}
  \inithyps{} then $\forall{}t\in{}T_i,id_t\in{}Comps(\Delta)$
  s.t. $\gamma(t)=id_t$,
  $s_0.reset_t(t)=\sigma_0(id_t)(\texttt{s\_reinit\_time\_counter})$.
\end{lemma}

\begin{niproof}
  Given a $t\in{}T_i$ and an $id_t\in{}Comps(\Delta)$ s.t.
  $\gamma(t)=id_t$, let us show
  that\\
  \framebox{$s_0.reset_t(t)=\sigma_0(id_t)(\texttt{s\_reinit\_time\_counter})$.}
  
  Rewriting $s_0.reset_t(t)$ as $\mathtt{false}$, by definition of
  $s_0$,
  \fbox{$\sigma_0(id_t)(\texttt{s\_reinit\_time\_counter})=\mathtt{false}$.}
  
  \exT{}
  
  \noindent By property of the \hvhdl{} initialization relation,
  \InCsCompT, and through the examination of the \texttt{reinit\_time\_counter\_evaluation} process defined in the \texttt{transition} design architecture\\
  we can deduce
  $\sigma_0(id_t)(\texttt{s\_reinit\_time\_counter})=\prod\limits_{i=0}^{\Delta(id_t)(\texttt{ian})-1}\sigma_0(id_t)(\texttt{rt})[i]$.

  Rewriting
  $\sigma_0(id_t)(\texttt{s\_reinit\_time\_counter})$ as $\prod\limits_{i=0}^{\Delta(id_t)(\texttt{ian})-1}\sigma_0(id_t)(\texttt{rt})[i]$,\\
  \fbox{$\prod\limits_{i=0}^{\Delta(id_t)(\texttt{ian})-1}\sigma_0(id_t)(\texttt{rt})[i]=\mathtt{false}$.}
  
  For all $t\in{}T$ (resp. $p\in{}P$), let $input(t)$
  (resp. $input(p)$) be the set of input places of $t$ (resp. input
  transitions of $p$), and let $output(t)$ (resp. $output(p)$) be the
  set of output places of $t$ (resp. output transitions of $p$).

  Let us perform case analysis on $input(t)$; there are 2 cases:

  \begin{itemize}
  \item \textbf{CASE} $input(t)=\emptyset$.

    By construction,
    ${<}\mathtt{input\_arcs\_number\Rightarrow}1{>}\in{}g_t$, and by
    property of the elaboration relation, and \InCsCompT{}, we can
    deduce $\Delta(id_t)(\texttt{ian})=1$.

    By construction, $<\mathtt{reinit\_time(0)\Rightarrow{}false}>\in{}i_t$, and
    by property of the initialization relation and \InCsCompT, we can
    deduce $\sigma_0(id_t)(\texttt{rt})[0]=\mathtt{false}$.

    Rewriting $\Delta(id_t)(\texttt{ian})$ as $1$ and
    $\sigma_0(id_t)(\texttt{rt})[0]$ as $\mathtt{false}$, \qedbox{tautology.}
    
  \item \textbf{CASE} $input(t)\neq\emptyset$.

    To prove the current goal, we can equivalently prove that\\
    \fbox{$\exists{}i\in[0,\Delta(id_t)(\texttt{ian})-1]~s.t.~\sigma_0(id_t)(\texttt{rt})[i]=\mathtt{false}$.}

    Since $input(t)\neq\emptyset,~\exists{}p~s.t.~p\in{}input(t)$. Let
    us take such a $p\in{}input(t)$.
    
    By construction, for all $p\in{}P$, there exist
    $id_p~s.t.~\gamma(p)=id_p$.

    \exP{}

    \noindent{}By construction, there exist
    $i\in[0,\vert{}input(t)\vert{}-1]$,
    $j\in[0,\vert{}output(p)\vert{}-1]$, $id_{ji}\in{}Sigs(\Delta)$
    s.t.
    ${<}\mathtt{reinit\_transitions\_time(j)\Rightarrow}id_{ji}{>}\in{}o_p$
    and
    ${<}\mathtt{reinit\_time(i)\Rightarrow}id_{ji}{>}\in{}i_t$. Let us take such a $i$, $j$ and $id_{ji}$.

    By construction and $input(t)\neq\emptyset$,
    ${<}\mathtt{input\_arcs\_number\Rightarrow}\vert{}input(t)\vert{}{>}\in{}g_t$.

    By property of the \hvhdl{} elaboration relation and
    ${<}\mathtt{input\_arcs\_number\Rightarrow}\vert{}input(t)\vert{}{>}\in{}g_t$,
    we can deduce $\Delta(id_t)(\texttt{ian})=\vert{}input(t)\vert$.

    Since $\Delta(id_t)(\texttt{ian})=\vert{}input(t)\vert$ and we have an
    $i\in[0,\vert{}input(t)\vert-1]$, then, we have an
    $i\in[0, \Delta(id_t)(\texttt{ian})-1]$. Let us take that i to
    prove the goal.

    Then, we must show \framebox{$\sigma_0(id_t)(\texttt{rt})[i]=\mathtt{false}$.}

    By property of the \hvhdl{} initialization relation and
    ${<}\mathtt{reinit\_time(i)\Rightarrow}id_{ji}{>}\in{}i_t$, we can
    deduce $\sigma_0(id_t)(\texttt{rt})[i]=\sigma_0(id_{ji})$.

    Rewriting $\sigma_0(id_t)(\texttt{rt})[i]$ as $\sigma_0(id_{ji})$,
    \framebox{$\sigma_0(id_{ji})=\mathtt{false}$.}

    By property of the \hvhdl{} initialization relation and\\
    ${<}\mathtt{reinit\_transitions\_time(j)\Rightarrow}id_{ji}{>}\in{}o_p$,
    we can deduce
    $\sigma_0(id_{ji})=\sigma_0(id_p)(\texttt{rtt})[j]$.

    Rewriting $\sigma_0(id_{ji})$ as
    $\sigma_0(id_p)(\texttt{rtt})[j]$,
    \framebox{$\sigma_0(id_p)(\texttt{rtt})[j]=\mathtt{false}$.}

    Since $t\in{}output(p)$, then we know that
    $output(p)\neq\emptyset$.

    Then, by construction,
    ${<}\mathtt{output\_arcs\_number\Rightarrow}\vert{}output(p)\vert{>}\in{}g_p$.

    By property of the elaboration relation and
    ${<}\mathtt{output\_arcs\_number}\Rightarrow\vert{}output(p)\vert{>}\in{}g_p$,
    we can deduce that $\Delta(id_p)(\texttt{oan})=\vert{}output(p)\vert$.

    Since $\Delta(id_p)(\texttt{oan})=\vert{}output(p)\vert$ and
    $j\in[0,\vert{}output(p)\vert-1]$, then
    $j\in[0,\Delta(id_p)(\texttt{oan})-1]$.
    
    By property of the \hvhdl{} initialization relation, \InCsCompP,
    through the examination of the
    \texttt{reinit\_transitions\_time\_evaluation} process defined in
    the \texttt{place} design architecture, and since
    $j\in[0,\Delta(id_p)(\texttt{oan})-1]$,
    \qedbox{$\sigma_0(id_p)(\texttt{rtt})[j]=\mathtt{false}$.}

  \end{itemize}
  
\end{niproof}

\subsection{Initial states and condition values}
\label{sec:init-states-cond-vals}

\begin{lemma}[Initial states equal condition values]
  \label{lem:init-states-cond-vals}
  \inithyps{} 
  then
  $\forall{}c\in\mathcal{C},id_c\in{}Ins(\Delta)~s.t.~\gamma(c)=id_c,~s_0.cond(c)=\sigma_0(id_c)$.
\end{lemma}

\begin{niproof}
  Given a $c\in\mathcal{C}$ and an
  $id_c\in{}Ins(\Delta)~s.t.~\gamma(c)=id_c$, let us show that
  \fbox{$s_0.cond(c)=\sigma_0(id_c)$.}\\

  Rewriting $s_0.cond(c)$ as $\mathtt{false}$, by definition of $s_0$,
  \fbox{$\sigma_0(id_c)=\mathtt{false}$.}

  By construction, $id_c$ is an input port identifier of Boolean type
  in the \hvhdl{} design $d$, and thus, by property of the \hvhdl{}
  elaboration relation, we can deduce $\sigma_e(id_c)=\mathtt{false}$.

  By property of the \hvhdl{} initialization relation and
  $id_c\in{}Ins(\Delta)$, we can deduce
  $\sigma_e(id_c)=\sigma_0(id_c)$.

  Rewriting $\sigma_0(id_c)$ as $\sigma_e(id_c)$ and $\sigma_e(id_c)$
  as $\mathtt{false}$, \qedbox{tautology.}
  
\end{niproof}

\subsection{Initial states and action executions}
\label{sec:init-states-act-exec}

\begin{lemma}[Initial states equal action executions]
  \label{lem:init-states-act-exec}
  \inithyps{} 
  then
  $\forall{}a\in\mathcal{A},id_a\in{}Outs(\Delta)~s.t.~\gamma(a)=id_a,~s_0.ex(a)=\sigma_0(id_a)$.
\end{lemma}

\begin{niproof}
  Given a $a\in\mathcal{A}$ and an
  $id_a\in{}Outs(\Delta)~s.t.~\gamma(a)=id_a$, let us show that
  \fbox{$s_0.ex(a)=\sigma_0(id_a)$.}\\

  Rewriting $s_0.ex(a)$ as $\mathtt{false}$, by definition of $s_0$,
  \fbox{$\sigma_0(id_a)=\mathtt{false}$.}

  By construction, $id_a$ is an output port identifier of Boolean type
  in the \hvhdl{} design $d$. Moreover, we know that the output port
  identifier $id_a$ is assigned to $\mathtt{false}$ in the generated
  \texttt{action} process during the initialization phase (i.e. the
  assignment is a part of a \emph{reset} block). Thus, we can deduce
  that $\sigma_0(id_a)=\mathtt{false}$.
  
  Rewriting $\sigma_0(id_a)$ as $\mathtt{false}$, \qedbox{tautology.}
  
\end{niproof}

\subsection{Initial states and function executions}
\label{sec:init-states-fun-exec}

\begin{lemma}[Initial states equal function executions]
  \label{lem:init-states-fun-exec}
  \inithyps{} 
  then
  $\forall{}f\in\mathcal{F},id_f\in{}Outs(\Delta)~s.t.~\gamma(f)=id_f,~s_0.ex(f)=\sigma_0(id_f)$.
\end{lemma}

\begin{niproof}
  Given a $f\in\mathcal{F}$ and an
  $id_f\in{}Outs(\Delta)~s.t.~\gamma(f)=id_f$, let us show that
  \fbox{$s_0.ex(f)=\sigma_0(id_f)$.}\\

  Rewriting $s_0.ex(f)$ as $\mathtt{false}$, by definition of $s_0$,
  \fbox{$\sigma_0(id_f)=\mathtt{false}$.}

  By construction, $id_f$ is an output port identifier of Boolean type
  in the \hvhdl{} design $d$, and thus, by property of the \hvhdl{}
  elaboration relation, we can deduce $\sigma_e(id_f)=\mathtt{false}$.

  By construction, and by property of the initialization relation, we
  know that the output port identifier $id_f$ is assigned to
  \texttt{false} in the generated \texttt{function} process during the
  initialization phase (i.e. the assignment is a part of a
  \emph{reset} block). Thus, we can deduce
  $\sigma_0(id_f)=\mathtt{false}$.
  
  \noindent{}Rewriting $\sigma_0(id_f)$ as $\mathtt{false}$,
  \qedbox{tautology.}
  
\end{niproof}

\subsection{Initial states and fired transitions}
\label{sec:init-states-fired-false}

\begin{lemma}[No fired at initial state]
  \label{lem:init-states-fired-false}
  $\forall{}d\in{}design,\Delta\in{}ElDesign,\sigma_e,\sigma_0\in\Sigma,id_t\in{}Comps(\Delta),$
  $g_t,$ $i_t,$ $o_t$ s.t. :
  \begin{itemize}
  \item $\mathcal{D}_\mathcal{H},\emptyset\vdash{}d.cs\xrightarrow{elab}\sigma_0$
  \item $\Delta,\sigma_e\vdash{}d.cs\xrightarrow{init}\sigma_0$
  \item $\mathtt{comp}(id_t$, $\texttt{transition}$, $g_t$, $i_t$,
    $o_t)\in{}d.cs$
  \end{itemize}
  then $\sigma_0(id_t)(\texttt{fired})=\mathtt{false}$.
\end{lemma}

\begin{niproof}
  Assuming all the above hypotheses, let us show
  \fbox{$\sigma_0(id_t)(\texttt{fired})=\mathtt{false}$.}

  By property of the initialization relation, \InCsCompT, and through
  the examination of the \texttt{fired\_evaluation} process defined in
  the transition design architecture, we can deduce:
  \begin{equation}
    \label{eq:fired-at-init-state}
    \sigma_0(id_t)(\texttt{fired})=\sigma_0(id_t)(\texttt{s\_firable})~.~\sigma_0(id_t)(\texttt{s\_priority\_combination})
  \end{equation}

  Rewriting the goal with Equation~\eqref{eq:fired-at-init-state}:
  \fbox{$\sigma_0(id_t)(\texttt{sfa})~.~\sigma_0(id_t)(\texttt{spc})=\mathtt{false}$.}
  
  By property of the initialization relation, \InCsCompT, and through
  the examination of the \texttt{firable} process defined in the
  \texttt{transition} design architecture, we can deduce
  $\sigma_0(id_t)(\texttt{sfa})=\mathtt{false}$.

  Rewriting the goal with $\sigma_0(id_t)(\texttt{sfa})=\mathtt{false}$ and
  simplifying the goal, \qedbox{tautology.}
\end{niproof}


%%% Local Variables:
%%% mode: latex
%%% TeX-master: "../../main"
%%% End:
