\begin{definition}[First rising edge hypotheses]
  \label{def:fst-re-hyps}
  Given an $sitpn\in{}SITPN,$ $d\in{}design,$\\
  $\gamma\in{}WM(sitpn,d),$ $\Delta\in{}ElDesign,$
  $\sigma_{e},\sigma_0,\sigma_i,\sigma_{\uparrow},\sigma\in{}\Sigma$,
  $E_c\in{}\mathbb{N}\rightarrow{}\mathcal{C}\rightarrow{}\mathbb{B}$,\\
  $E_p\in\mathbb{N}\times{}\{\uparrow,\downarrow\})\rightarrow{}Ins(\Delta)\rightarrow{}value$,
  $\tau\in\mathbb{N}$, assume that:
  \begin{itemize}
  \item $\lfloor{}sitpn\rfloor_\mathcal{H}=(d,\gamma)$ and
    $\mathcal{D}_\mathcal{H},\emptyset\vdash{}d\srarrow{elab}{\fontsize{6}{8}\selectfont}(\Delta,\sigma_{e})$
    and $\gamma\vdash{}E_p\stackrel{env}{=}E_c$
  \item $\sigma_0$ is the initial state of $\Delta$: 
    $\Delta,\sigma_{e}\vdash{}d.cs\srarrow{init}{\fontsize{6}{8}\selectfont}\sigma_0$
  \item $E_c,\tau\vdash{}s_0\srarrow{\uparrow_0}{\fontsize{6}{8}\selectfont}s_0$
  \item $\mathtt{Inject}(\sigma_0, E_p, \tau, \sigma_i)$
    and
    $\Delta,\sigma_i\vdash\mathrm{d.cs}\xrightarrow{\uparrow}\sigma_{\uparrow}$
    and
    $\Delta,\sigma_{\uparrow}\vdash\mathrm{d.cs}\xrightarrow{\theta}\sigma$
  \end{itemize}
  
\end{definition}

\def\fstrehyps{For all $sitpn,d,\gamma,\Delta,$
  $\sigma_{e},\sigma_0,\sigma_i,\sigma_{\uparrow},\sigma$, $E_c$,
  $E_p$, $\tau$ that verify the hypotheses of
  Definition~\ref{def:fst-re-hyps},}

%%%%%%%%%%%%%%%%%%%%%%%%%%%%%%%%%%%%%%%%%%%%%
%%%%%%%%%% FIRST RISING EDGE LEMMA %%%%%%%%%%
%%%%%%%%%%%%%%%%%%%%%%%%%%%%%%%%%%%%%%%%%%%%%

\begin{lemma}[First rising edge]
  \label{lem:fst-re}
  \fstrehyps{} then
  $\gamma,E_c,\tau\vdash{}s_0\stackrel{\uparrow}{\approx}{}\sigma$.
\end{lemma}

\begin{proof}
  By definition of the \nameref{def:full-post-re-state-sim} relation,
  there are 8 points to prove.
  \begin{frameb}
    \begin{enumerate}
    \item
      $\forall{}p\in{}P,id_p\in{}Comps(\Delta)~s.t.~\gamma(p)=id_p,~s_0.M(p)=\sigma(id_p)(\texttt{s\_marking})$.
      \label{item:fst-re-marking-eq}
    \item
      $\forall{}t\in{}T_i,id_t\in{}Comps(\Delta)~s.t.~\gamma(t)=id_t,$\\
      $\big(upper(I_s(t))=\infty\land{}s_0.I(t)\le{}lower(I_s(t))\Rightarrow{}s_0.I(t)=\sigma(id_t)(\texttt{s\_time\_counter})\big)$\\
      $\land\big(upper(I_s(t))=\infty\land{}s_0.I(t)>{}lower(I_s(t))\Rightarrow{}\sigma(id_t)(\texttt{s\_time\_counter})=lower(I_s(t))\big)$\\
      $\land\big(upper(I_s(t))\neq\infty\land{}s_0.I(t)>{}upper(I_s(t))\Rightarrow{}\sigma(id_t)(\texttt{s\_time\_counter})=upper(I_s(t))\big)$\\
      $\land\big(upper(I_s(t))\neq\infty\land{}s_0.I(t)\le{}upper(I_s(t))\Rightarrow{}s_0.I(t)=\sigma(id_t)(\texttt{s\_time\_counter})\big)$.
      \label{item:fst-re-tc-eq}
    \item
      $\forall{}t\in{}T_i,id_t\in{}Comps(\Delta)~s.t.~\gamma(t)=id_t,$
      $s_0.reset_t(t)=\sigma(id_t)(\texttt{s\_reinit\_time\_counter})$.
      \label{item:fst-re-reset-eq}
    \item
      $\forall{}a\in\mathcal{A},id_a\in{}Outs(\Delta)~s.t.~\gamma(a)=id_a,~s_0.ex(a)=\sigma(id_a)$.
      \label{item:fst-re-action-eq}
    \item
      $\forall{}f\in\mathcal{F},id_f\in{}Outs(\Delta)~s.t.~\gamma(f)=id_f,~s_0.ex(f)=\sigma(id_f)$.
      \label{item:fst-re-fun-eq}
    \item $\forall{}t\in{}T,id_t\in{}Comps(\Delta)~s.t.~\gamma(t)=id_t,$
      $t\in{}Sens(s_0.M)\Leftrightarrow\sigma(id_t)(\texttt{s\_enabled})=\mathtt{true}$.
      \label{item:fst-re-sens-eq}
    \item $\forall{}t\in{}T,id_t\in{}Comps(\Delta)~s.t.~\gamma(t)=id_t,$
      $t\notin{}Sens(s_0.M)\Leftrightarrow\sigma(id_t)(\texttt{s\_enabled})=\mathtt{false}$.
      \label{item:fst-re-sens-neq}
    \item
      $\forall{}t\in{}T,id_t\in{}Comps(\Delta)~s.t.~\gamma(t)=id_t,$\\
      $\sigma(id_t)(\texttt{s\_condition\_combination})=
      \prod\limits_{c\in{}conds(t)}
      \begin{cases}
        E_c(\tau,c) & if~\mathbb{C}(t,c)=1 \\
        \mathtt{not}(E_c(\tau,c)) & if~\mathbb{C}(t,c)=-1 \\
      \end{cases}$\\
      where
      $conds(t)=\{c\in\mathcal{C}~\vert~\mathbb{C}(t,c)=1\lor\mathbb{C}(t,c)=-1\}$.
      \label{item:fst-re-cond-comb-eq}
    \item $\forall{}c\in\mathcal{C},id_c\in{}Ins(\Delta)$
      s.t. $\gamma(c)=id_c$, $\sigma(id_c)=E_c(\tau,c)$.\label{item:fst-re-cond-eq}
    \end{enumerate}
  \end{frameb}

  \begin{itemize}
  \item Apply the \nameref{lem:fst-re-equal-marking} lemma to solve \ref{item:fst-re-marking-eq}.
  \item Apply the \nameref{lem:fst-re-equal-tc} lemma to solve \ref{item:fst-re-tc-eq}.
  \item Apply the \nameref{lem:fst-re-equal-reset-orders} lemma to solve \ref{item:fst-re-reset-eq}.
  \item Apply the \nameref{lem:fst-re-equal-action-ex} lemma to solve
    \ref{item:fst-re-action-eq}.
  \item Apply the \nameref{lem:fst-re-equal-fun-ex} lemma to solve
    \ref{item:fst-re-fun-eq}.
  \item Apply the \nameref{lem:fst-re-equal-sens} lemma to solve
    \ref{item:fst-re-sens-eq}.
  \item Apply the \nameref{lem:fst-re-neq-sens} lemma to solve
    \ref{item:fst-re-sens-neq}.
  \item Apply the \nameref{lem:fst-re-equal-cond-comb} lemma to solve
    \ref{item:fst-re-cond-comb-eq}.
  \item Apply the \nameref{lem:fst-re-equal-cond} lemma to solve
    \ref{item:fst-re-cond-eq}.
  \end{itemize}
  
\end{proof}

\subsection{First rising edge and marking}
\label{sec:fst-re-marking}

\begin{lemma}[First rising edge equal marking]
  \label{lem:fst-re-equal-marking}
  \fstrehyps{} then
  $\forall{}p\in{}P,id_p\in{}Comps(\Delta)~s.t.~\gamma(p)=id_p$,
  $~s_0.M(p)=\sigma(id_p)(\texttt{s\_marking})$.
\end{lemma}

\begin{niproof}
  Given a $p$ and an $id_p$ s.t. $\gamma(p)=id_p$, let us show that
  \fbox{$~s_0.M(p)=\sigma(id_p)(\texttt{s\_marking})$.}
  
  \exP{}
    
  By property of the \texttt{Inject} relation, the \hvhdl{} rising
  edge relation, the stabilize relation, \InCsCompP, and through the
  examination of the \texttt{marking} process defined in the place
  design architecture, we can deduce:
  \begin{equation}
    \sigma(id_p)(\texttt{sm})=\sigma_0(id_p)(\texttt{sm})+\sigma_0(id_p)(\texttt{sits})-\sigma_0(id_p)(\texttt{sots})\label{eq:eq-sm-after-fst-re}
  \end{equation}

  Rewriting the goal with Equation~\eqref{eq:eq-sm-after-fst-re}:\\
  \fbox{$s_0.M(p)=\sigma_0(id_p)(\texttt{sm})+\sigma_0(id_p)(\texttt{sits})-\sigma_0(id_p)(\texttt{sots})$.}

  Appealing to Lemmas~\ref{lem:init-states-sits-zero} and
  \ref{lem:init-states-sots-zero}, we can deduce
  $\sigma_0(id_p)(\texttt{sits})=0$ and $\sigma_0(id_p)(\texttt{sots})=0$. Rewriting
  the goal with $\sigma_0(id_p)(\texttt{sits})=0$ and
  $\sigma_0(id_p)(\texttt{sots})=0$,
  \fbox{$s_0.M(p)=\sigma_0(id_p)(\texttt{sm})$.}\\

  Appealing to Lemma~\ref{lem:init-states-eq-marking},
  \qedbox{$s_0.M(p)=\sigma_0(id_p)(\texttt{sm})$.}
\end{niproof}

\subsection{First rising edge and time counters}
\label{sec:fst-re-tc}

\begin{lemma}[First rising edge equal time counters]
  \label{lem:fst-re-equal-tc}
  \fstrehyps{} then\\
  $\forall{}t\in{}T_i,id_t\in{}Comps(\Delta)~s.t.~\gamma(t)=id_t$,\\
  $upper(I_s(t))=\infty\land{}s_0.I(t)\le{}lower(I_s(t))\Rightarrow{}s_0.I(t)=\sigma(id_t)(\texttt{s\_time\_counter})\land{}$\\
  $upper(I_s(t))=\infty\land{}s_0.I(t)>{}lower(I_s(t))\Rightarrow{}\sigma(id_t)(\texttt{s\_time\_counter})=lower(I_s(t))\land{}$\\
  $upper(I_s(t))\neq\infty\land{}s_0.I(t)>{}upper(I_s(t))\Rightarrow{}\sigma(id_t)(\texttt{s\_time\_counter})=upper(I_s(t))\land{}$\\
  $upper(I_s(t))\neq\infty\land{}s_0.I(t)\le{}upper(I_s(t))\Rightarrow{}s_0.I(t)=\sigma(id_t)(\texttt{s\_time\_counter})$.
\end{lemma}

\begin{niproof}
  Given a $t\in{}T_i$ and an $id_t\in{}Comps(\Delta)$
  s.t. $\gamma(t)=id_t$, let us show that:
  \begin{enumerate}
  \item \framebox{$upper(I_s(t))=\infty\land{}s_0.I(t)\le{}lower(I_s(t))\Rightarrow{}s_0.I(t)=\sigma(id_t)(\texttt{s\_time\_counter})$}
  \item \framebox{$upper(I_s(t))=\infty\land{}s_0.I(t)>{}lower(I_s(t))\Rightarrow{}\sigma(id_t)(\texttt{s\_time\_counter})=lower(I_s(t))$}
  \item \framebox{$upper(I_s(t))\neq\infty\land{}s_0.I(t)>{}upper(I_s(t))\Rightarrow{}\sigma(id_t)(\texttt{s\_time\_counter})=upper(I_s(t))$}
  \item \framebox{$upper(I_s(t))\neq\infty\land{}s_0.I(t)\le{}upper(I_s(t))\Rightarrow{}s_0.I(t)=\sigma(id_t)(\texttt{s\_time\_counter})$}
  \end{enumerate}

  \exT{}

  Then, let us show the 4 previous points:
  
  \begin{enumerate}
  \item Assuming that $upper(I_s(t))=\infty$ and
    $s_0.I(t)\le{}lower(I_s(t))$, let us show\\
    \framebox{${}s_0.I(t)=\sigma(id_t)(\texttt{stc})$.}
    
    By property of the $\mathtt{Inject}$ relation, the
    \hvhdl{} rising edge and stabilize relations, and \InCsCompT{}, we
    can deduce $\sigma(id_t)(\texttt{stc})=\sigma_0(id_t)(\texttt{stc})$.
    
    Rewriting $\sigma(id_t)(\texttt{stc})$ as $\sigma_0(id_t)(\texttt{stc})$,
    \fbox{${}s_0.I(t)=\sigma_0(id_t)(\texttt{stc})$.}

    Appealing to Lemma~\ref{lem:init-states-eq-tc},
    \qedbox{${}s_0.I(t)=\sigma_0(id_t)(\texttt{stc})$.}
  
  \item Assuming that $upper(I_s(t))=\infty$ and
    ${}s_0.I(t)>{}lower(I_s(t))$, let us show\\
    \framebox{$\sigma(id_t)(\texttt{stc})=lower(I_s(t))$.}

    By definition, $lower(I_s(t))\in\mathbb{N}^{*}$ and
    $s_0.I(t)=0$. Then, \qedbox{$lower(I_s(t)){}<0$ is a
      contradiction.}
  \item Assuming that $upper(I_s(t))\neq\infty$ and
    $s_0.I(t)>{}upper(I_s(t))$, let us show\\
    \framebox{$\sigma(id_t)(\texttt{stc})=upper(I_s(t))$}.

    By definition, $upper(I_s(t))\in\mathbb{N}^{*}$ and
    $s_0.I(t)=0$. Then, \qedbox{$upper(I_s(t)){}<0$ is a
      contradiction.}
    
  \item Assuming that $upper(I_s(t))\neq\infty$ and
    $s_0.I(t)\le{}upper(I_s(t))$, let us show\\
    \framebox{$s_0.I(t)=\sigma(id_t)(\texttt{stc})$.}

    By property of the $\mathtt{Inject}$ relation, the
    \hvhdl{} rising edge and stabilize relations, and \InCsCompT{}, we
    can deduce $\sigma(id_t)(\texttt{stc})=\sigma_0(id_t)(\texttt{stc})$.

    Rewriting $\sigma(id_t)(\texttt{stc})$ as $\sigma_0(id_t)(\texttt{stc})$,
    \fbox{${}s_0.I(t)=\sigma_0(id_t)(\texttt{stc})$.}

    Appealing to Lemma~\ref{lem:init-states-eq-tc},
    \qedbox{${}s_0.I(t)=\sigma_0(id_t)(\texttt{stc})$.}
  \end{enumerate}
\end{niproof}

\subsection{First rising edge and reset orders}
\label{sec:fst-re-reset-orders}

\begin{lemma}[First rising edge equal reset orders]
  \label{lem:fst-re-equal-reset-orders}
  \fstrehyps{} then\\
  $\forall{}t\in{}T,id_t\in{}Comps(\Delta)~s.t.~\gamma(t)=id_t,$
  $s_0.reset_t(t)=\sigma(id_t)(\texttt{s\_reinit\_time\_counter})$.
\end{lemma}

\begin{niproof}
  Given a $t\in{}T$ and an $id_t\in{}Comps(\Delta)$
  s.t. $\gamma(t)=id_t$, let us show that\\
  \fbox{$s_0.reset_t(t)=\sigma(id_t)(\texttt{srtc})$.}

  \exT{}

  \noindent{}By property of the stabilize relation, \InCsCompT{}, and
  through the examination of the
  \texttt{reinit\_time\_counter\_evaluation} process defined in the
  transition design architecture, we can deduce:  
  \begin{equation}
    \sigma(id_t)(\texttt{srtc})=\sum\limits_{i=0}^{\Delta(id_t)(\texttt{input\_arcs\_number})-1}\sigma(id_t)(\texttt{reinit\_time})[i]\label{eq:srtc-at-init-states}
  \end{equation}

  Rewriting the goal with Equation~\eqref{eq:srtc-at-init-states}:
  \fbox{$s_0.reset_t(t)=\sum\limits_{i=0}^{\Delta(id_t)(\texttt{ian})-1}\sigma(id_t)(\texttt{rt})[i]$.}
  
  Let us perform case analysis on $input(t)$; there are two cases:

  \begin{itemize}
  \item \textbf{CASE} $input(t)=\emptyset$:

    By construction,
    ${<}\mathtt{input\_arcs\_number\Rightarrow{}1}{>}\in{}g_t$, and
    by property of the \hvhdl{} elaboration relation, we can deduce
    $\Delta(id_t)(\texttt{ian})=1$.

    By construction,
    $<\mathtt{reinit\_time(0)\Rightarrow{}false}>\in{}i_t$, and by
    property of the \hvhdl{} stabilize relation,
    $\sigma(id_t)(\texttt{rt})[0]=\mathtt{false}$.

    Rewriting the goal with $\Delta(id_t)(\texttt{ian})=1$ and
    $\sigma(id_t)(\texttt{rt})[0]=\mathtt{false}$,
    \fbox{$s_0.reset_t(t)=\mathtt{false}$.}

    By definition of $s_0$, \qedbox{$s_0.reset_t(t)=\mathtt{false}$.}
    
  \item \textbf{CASE} $input(t)\neq{}\emptyset$:

    By construction,
    ${<}\mathtt{input\_arcs\_number\Rightarrow{}}\vert{}input(t)\vert{>}\in{}g_t$,
    and by property of the \hvhdl{} elaboration relation, we can
    deduce $\Delta(id_t)(\texttt{ian})=\vert{}input(t)\vert$.

    Rewriting $\Delta(id_t)(\texttt{ian})$ as $\vert{}input(t)\vert$,
    \fbox{$s_0.reset_t(t)=\sum\limits_{i=0}^{\vert{}input(t)\vert-1}\sigma(id_t)(\texttt{rt})[i]$.}

    By definition of $s_0$, $s_0.reset_t(t)=\mathtt{false}$. Rewriting
    $s_0.reset_t(t)$ as $\mathtt{false}$,\\
    \fbox{$\sum\limits_{i=0}^{\vert{}input(t)\vert-1}\sigma(id_t)(\texttt{rt})[i]=\mathtt{false}$.}

    Given a $i\in[0,\vert{}input(t)\vert-1]$, let us show
    \fbox{$\sigma(id_t)(\texttt{rt})[i]=\mathtt{false}$.}
    
    By construction, and since $input(t)\neq{}\emptyset$, there exist
    a $p\in{}input(t)$, an $id_p\in{}Comps(\Delta)$
    s.t. $\gamma(p)=id_p$, a $g_p,$ an $i_p,$ an $o_p$
    s.t. \InCsCompP{}, and there exist a
    $j\in[0,\vert{}output(p)\vert-1]$ and an
    $id_{ji}\in{}Sigs(\Delta)$ s.t.
    ${<}\mathtt{reinit\_transition\_time(j)\Rightarrow{}id_{ji}}{>}\in{}o_p$
    and
    ${<}\mathtt{reinit\_time(i)\Rightarrow{}id_{ji}}{>}\in{}i_t$.

    By property of the stabilize relation,
    ${<}$\texttt{reinit\_transition\_time(j)$\Rightarrow{}id_{ji}$}${>}\in{}o_p$
    and\\
    ${<}$\texttt{reinit\_time(i)$\Rightarrow{}id_{ji}$}${>}\in{}i_t$,
    we can deduce
    $\sigma(id_t)(\texttt{rt})[i]=\sigma(id_{ji})=\sigma(id_p)(\texttt{rtt})[j]$.

    Rewriting $\sigma(id_t)(\texttt{rt})[i]$ as $\sigma(id_{ji})$ and
    $\sigma(id_{ji})$ as $\sigma(id_p)(\texttt{rtt})[j]$,
    \fbox{$\sigma(id_p)(\texttt{rtt})[j]=\mathtt{false}$.}

    By property of the \hvhdl{} rising edge and stabilize relations,
    \InCsCompP{}, and through the examination of the process defined
    in the place design architecture, we can deduce:
    \begin{equation}
      \label{eq:rtt-at-init}
      \begin{split}
        \sigma(id_p)(\texttt{rtt})[j]=& ((\sigma_0(id_p)(\texttt{oat})[j]=\mathtt{basic}+\sigma_0(id_p)(\texttt{oat})[j]=\mathtt{test}) \\
        & .(\sigma_0(id_p)(\texttt{sm})-\sigma_0(id_p)(\texttt{sots})<\sigma_0(id_p)(\texttt{oaw})[j])\\
        & .(\sigma_0(id_p)(\texttt{sots})>0))\\
        & +(\sigma_0(id_p)(\texttt{otf})[j]) \\
      \end{split}
    \end{equation}

    Rewriting the goal with Equation~\eqref{eq:rtt-at-init},
    \begin{equation*}
      \fbox{$\begin{split}
          \mathtt{false}=& ((\sigma_0(id_p)(\texttt{oat})[j]=\mathtt{basic}+\sigma_0(id_p)(\texttt{oat})[j]=\mathtt{test}) \\
          & .(\sigma_0(id_p)(\texttt{sm})-\sigma_0(id_p)(\texttt{sots})<\sigma_0(id_p)(\texttt{oaw})[j])\\
          & .(\sigma_0(id_p)(\texttt{sots})>0))\\
          & +(\sigma_0(id_p)(\texttt{otf})[j]) \\
        \end{split}$}
    \end{equation*}

    By construction, there exists an $id_{fj}\in{}Sigs(\Delta)$ s.t.
    ${<}$\texttt{fired}$\Rightarrow{}\mathtt{id_{fj}}{>}\in{}o_t$
    and
    ${<}$\texttt{output\_transitions\_fired(j)}$\Rightarrow{}\mathtt{id_{fj}}{>}\in{}i_p$.

    By property of the initialization relation,
    ${<}$\texttt{fired}$\Rightarrow{}\mathtt{id_{fj}}{>}\in{}o_t$
    and
    ${<}$\texttt{output\_transitions\_fired(j)}$\Rightarrow{}\mathtt{id_{fj}}{>}\in{}i_p$,
    we can deduce
    $\sigma_0(id_p)(\texttt{otf})[j]=\sigma_0(id_{fj})=\sigma_0(id_t)(\texttt{fired})$.
    
    Appealing to Lemma~\ref{lem:init-states-fired-false}, we can
    deduce $\sigma_0(id_t)(\texttt{fired})=\mathtt{false}$ and consequently\\
    $\sigma_0(id_p)(\texttt{otf})[j]=\mathtt{false}$.

    Rewriting $\sigma_0(id_p)(\texttt{otf})[j]$ as $\mathtt{false}$ and
    simplifying the goal,
    \begin{equation*}
      \fbox{$\begin{split}
          false=& ((\sigma_0(id_p)(\texttt{oat})[j]=\mathtt{BASIC}+\sigma_0(id_p)(\texttt{oat})[j]=\mathtt{TEST}) \\
          & .(\sigma_0(id_p)(\texttt{sm})-\sigma_0(id_p)(\texttt{sots})<\sigma_0(id_p)(\texttt{oaw})[j])\\
          & .(\sigma_0(id_p)(\texttt{sots})>0))\\
        \end{split}$}
    \end{equation*}

    Appealing to Lemma~\ref{lem:init-states-sots-zero}, we can deduce
    $\sigma_0(id_p)(\texttt{sots})=0$.

    Rewriting $\sigma_0(id_p)(\texttt{sots})$ as $0$ and simplifying the
    goal, \qedbox{tautology.}
    
  \end{itemize}
\end{niproof}

\subsection{First rising edge and action executions}
\label{sec:fst-re-actions-ex}

\begin{lemma}[First rising edge equal action executions]
  \label{lem:fst-re-equal-action-ex}
  \fstrehyps{} then\\
  $\forall{}a\in\mathcal{A},id_a\in{}Outs(\Delta)~s.t.~\gamma(a)=id_a,~s_0.ex(a)=\sigma(id_a)$.\\
\end{lemma}

\begin{niproof}
  Given an $a\in\mathcal{A}$ and an
  $id_a\in{}Outs(\Delta)~s.t.~\gamma(a)=id_a$, let us show that
  \fbox{$s_0.ex(a)=\sigma(id_a)$.}

  By construction, $id_a$ is an output port identifier of Boolean type
  in the \hvhdl{} design $d$. The generated \texttt{action} process
  assigns a value to the output port $id_a$ only during the
  initialization phase or a falling edge phase.

  \noindent{}By property of the $\mathtt{Inject}$, \hvhdl{}
  rising edge and stabilize relations, we can deduce
  $\sigma(id_a)=\sigma_0(id_a)$.
  
  Rewriting $\sigma(id_a)$ as $\sigma_0(id_a)$,
  \fbox{$s_0.ex(a)=\sigma_0(id_a)$.}  Appealing to
  Lemma~\ref{lem:init-states-act-exec},
  \qedbox{$s_0.ex(a)=\sigma_0(id_a)$.}
  
\end{niproof}

\subsection{First rising edge and function executions}
\label{sec:fst-re-fun-ex}

\begin{lemma}[First rising edge equal function executions]
  \label{lem:fst-re-equal-fun-ex}
  \fstrehyps{} then\\
  $\forall{}f\in\mathcal{F},id_f\in{}Outs(\Delta)~s.t.~\gamma(f)=id_f,~s_0.ex(f)=\sigma(id_f)$.
\end{lemma}

\begin{niproof}
  Given an $f\in\mathcal{F}$ and an $id_f\in{}Outs(\Delta)$
  s.t. $\gamma(f)=id_f$, let us show that
  \fbox{$s_0.ex(f)=\sigma(id_f)$.}

  Rewriting $s_0.ex(f)$ as $\mathtt{false}$, by definition of $s_0$,
  \fbox{$\sigma(id_f)=\mathtt{false}$.}

  By construction, $id_f$ is an output port identifier of Boolean type
  in the \hvhdl{} design $d$. The generated \texttt{function} process
  assigns a value to the output port $id_f$ only during the
  initialization phase or during a rising edge phase.
  
  By construction, the \texttt{function} process is defined in the
  behavior of design $d$, i.e.\\
  $\mathtt{ps}(\texttt{function}, \emptyset, sl, ss)\in{}d.cs$.
  
  Let $trs(f)$ be the set of transitions associated to function $f$,
  i.e $trs(f)=\{t\in{}T~\vert~\mathbb{F}(t,f)=true\}$.

  Let us perform case analysis on $trs(f)$; there are two cases:
  
  \begin{itemize}
  \item \textbf{CASE} $trs(f)=\emptyset$:
    
    By construction,
    $\mathtt{id_f\Leftarrow{}false}\in{}ss_{\uparrow}$ where
    $ss_\uparrow$ is the part of the \texttt{``function''} process
    body executed during a rising edge phase (i.e. a rising edge block
    statement).

    By property of the \hvhdl{} rising edge and the stabilize
    relation, \qedbox{$\sigma(id_f)=\mathtt{false}$.}
    
  \item \textbf{CASE} $trs(f)\neq\emptyset$:
    
    By construction,
    $\mathtt{id_f\Leftarrow{}id_{ft_0}+\dots+id_{ft_n}}\in{}ss_\uparrow$
    where $ss_\uparrow$ is the part of the \texttt{``function''}
    process body executed during the rising edge phase, and
    $n=\vert{}trs(f)\vert-1$, and for all $i\in[0,n-1]$, $id_{ft_i}$
    is a internal signal of design $d$.

    By property of the $\mathtt{Inject}$, the \hvhdl{} rising
    edge and stabilize relations, we can deduce
    $\sigma(id_f)=\sigma_0(id_{ft_0})+\dots+\sigma_0(id_{ft_n})$.

    Rewriting $\sigma(id_f)$ as
    $\sigma_0(id_{ft_0})+\dots+\sigma_0(id_{ft_n})$,
    \fbox{$\sigma_0(id_{ft_0})+\dots+\sigma_0(id_{ft_n})=\mathtt{false}$.}

    By construction, for all $id_{ft_i}$, there exist a
    $t_i\in{}trs(f)$ and an $id_{t_i}$ s.t. $\gamma(t_i)=id_{t_i}$.

    By construction and by definition of $id_{t_i}$, there exist
    $g_{t_i}$, $i_{t_i}$ and $o_{t_i}$ s.t.
    $\mathtt{comp}($$id_{t_i},$ $\texttt{transition},$
    $g_{t_i},$ $i_{t_i},$ $o_{t_i})\in{}d.cs$.

    By construction, we have
    ${<}\mathtt{fired\Rightarrow{id_{ft_i}}}{>}\in{}o_{t_i}$, and by
    property of the initialization relation, we have
    $\sigma_0(id_{ft_i})=\sigma_0(id_{t_i})(\texttt{fired})$.

    \noindent{}Rewriting $\sigma_0(id_{ft_i})$ as
    $\sigma_0(id_{t_i})(\texttt{fired})$,
    \fbox{$\sigma_0(id_{t_0})(\texttt{fired})+\dots+\sigma_0(id_{t_n})(\texttt{fired})=\mathtt{false}$.}

    Appealing to Lemma~\ref{lem:init-states-fired-false}, we can
    deduce $\sigma_0(id_{t_i})(\texttt{fired})=\mathtt{false}$.

    Rewriting all $\sigma_0(id_{t_i})(\texttt{fired})$ as $\mathtt{false}$
    and simplifying the goal, \qedbox{tautology.}
    
  \end{itemize}

\end{niproof}

\subsection{First rising edge and sensitization}
\label{sec:fst-re-sens}

\begin{lemma}[First rising edge equal sensitized]
  \label{lem:fst-re-equal-sens}
  \fstrehyps{} then\\
  $\forall{}t\in{}T,id_t\in{}Comps(\Delta)~s.t.~\gamma(t)=id_t,$
  $t\in{}Sens(s_0.M)\Leftrightarrow\sigma(id_t)(\texttt{s\_enabled})=\mathtt{true}$.
\end{lemma}

\begin{niproof}
  See the proof of Lemma~\ref{lem:re-equal-sens}.
\end{niproof}

\begin{lemma}[First rising edge not equal sensitized]
  \label{lem:fst-re-neq-sens}
  \fstrehyps{} then\\
  $\forall{}t\in{}T,id_t\in{}Comps(\Delta)~s.t.~\gamma(t)=id_t,$
  $t\notin{}Sens(s_0.M)\Leftrightarrow\sigma(id_t)(\texttt{s\_enabled})=\mathtt{false}$.
\end{lemma}

\begin{niproof}
  See the proof of Lemma~\ref{lem:re-equal-not-sens}.
\end{niproof}

\subsection{First rising edge and conditions}
\label{sec:fst-re-cond-comb}

\begin{lemma}[First rising edge equal condition combination]
  \label{lem:fst-re-equal-cond-comb}
  \fstrehyps{} then\\
  $\forall{}t\in{}T,id_t\in{}Comps(\Delta)~s.t.~\gamma(t)=id_t,$\\
  $\sigma(id_t)(\texttt{s\_condition\_combination})=
  \prod\limits_{c\in{}conds(t)}
  \begin{cases}
    E_c(\tau,c) & if~\mathbb{C}(t,c)=1 \\
    \mathtt{not}(E_c(\tau,c)) & if~\mathbb{C}(t,c)=-1 \\
  \end{cases}$\\
  where
  $conds(t)=\{c\in\mathcal{C}~\vert~\mathbb{C}(t,c)=1\lor\mathbb{C}(t,c)=-1\}$.
\end{lemma}

\begin{niproof}
  See the proof of Lemma~\ref{lem:re-equal-cond-comb}.
\end{niproof}

\begin{lemma}[First rising edge equal conditions]
  \label{lem:fst-re-equal-cond}
  \fstrehyps{} then\\
  $\forall{}c\in\mathcal{C},id_c\in{}Ins(\Delta)$
  s.t. $\gamma(c)=id_c$, $\sigma(id_c)=E_c(\tau,c)$.
\end{lemma}

\begin{niproof}
  See the proof of Lemma~\ref{lem:re-equal-cond}.
\end{niproof}


%%% Local Variables:
%%% mode: latex
%%% TeX-master: "../../main"
%%% End:
